\documentclass[11pt,a4paper]{article}
\usepackage[T1]{fontenc}
\usepackage{isabelle,isabellesym}

% further packages required for unusual symbols (see also
% isabellesym.sty), use only when needed

%\usepackage{amssymb}
  %for \<leadsto>, \<box>, \<diamond>, \<sqsupset>, \<mho>, \<Join>,
  %\<lhd>, \<lesssim>, \<greatersim>, \<lessapprox>, \<greaterapprox>,
  %\<triangleq>, \<yen>, \<lozenge>

%\usepackage{eurosym}
  %for \<euro>

%\usepackage[only,bigsqcap,bigparallel,fatsemi,interleave,sslash]{stmaryrd}
  %for \<Sqinter>, \<Parallel>, \<Zsemi>, \<Parallel>, \<sslash>

%\usepackage{eufrak}
  %for \<AA> ... \<ZZ>, \<aa> ... \<zz> (also included in amssymb)

%\usepackage{textcomp}
  %for \<onequarter>, \<onehalf>, \<threequarters>, \<degree>, \<cent>,
  %\<currency>

% this should be the last package used
\usepackage{pdfsetup}

% urls in roman style, theory text in math-similar italics
\urlstyle{rm}
\isabellestyle{it}

% for uniform font size
%\renewcommand{\isastyle}{\isastyleminor}


\begin{document}

\title{Quantum Fourier Transform}
\author{Pablo Manrique}
\maketitle

\begin{abstract}
	This work presents a formalization of the Quantum Fourier Transform, a fundamental component of Shor's factoring algorithm, with proofs of its correctness and unitarity. The proof is carried out by induction, relying on the algorithm's recursive definition. This formalization builds upon the \textit{Isabelle Marries Dirac} quantum computing library, developed by A. Bordg, H. Lachnitt, and Y. He.
\end{abstract}

\tableofcontents

% sane default for proof documents
\parindent 0pt\parskip 0.5ex

% generated text of all theories
%
\begin{isabellebody}%
\setisabellecontext{QFT}%
%
\isadelimtheory
\isanewline
\isanewline
%
\endisadelimtheory
%
\isatagtheory
\isacommand{theory}\isamarkupfalse%
\ QFT\isanewline
\isanewline
\isakeyword{imports}\isanewline
\ \ Isabelle{\isacharunderscore}{\kern0pt}Marries{\isacharunderscore}{\kern0pt}Dirac{\isachardot}{\kern0pt}Deutsch\isanewline
\isakeyword{begin}%
\endisatagtheory
{\isafoldtheory}%
%
\isadelimtheory
%
\endisadelimtheory
%
\isadelimdocument
%
\endisadelimdocument
%
\isatagdocument
%
\isamarkupsection{Some useful lemmas%
}
\isamarkuptrue%
%
\endisatagdocument
{\isafolddocument}%
%
\isadelimdocument
%
\endisadelimdocument
\isacommand{lemma}\isamarkupfalse%
\ gate{\isacharunderscore}{\kern0pt}carrier{\isacharunderscore}{\kern0pt}mat{\isacharbrackleft}{\kern0pt}simp{\isacharbrackright}{\kern0pt}{\isacharcolon}{\kern0pt}\isanewline
\ \ \isakeyword{assumes}\ {\isachardoublequoteopen}gate\ n\ U{\isachardoublequoteclose}\isanewline
\ \ \isakeyword{shows}\ {\isachardoublequoteopen}U\ {\isasymin}\ carrier{\isacharunderscore}{\kern0pt}mat\ {\isacharparenleft}{\kern0pt}{\isadigit{2}}{\isacharcircum}{\kern0pt}n{\isacharparenright}{\kern0pt}\ {\isacharparenleft}{\kern0pt}{\isadigit{2}}{\isacharcircum}{\kern0pt}n{\isacharparenright}{\kern0pt}{\isachardoublequoteclose}\isanewline
%
\isadelimproof
%
\endisadelimproof
%
\isatagproof
\isacommand{proof}\isamarkupfalse%
\isanewline
\ \ \isacommand{show}\isamarkupfalse%
\ {\isachardoublequoteopen}dim{\isacharunderscore}{\kern0pt}row\ U\ {\isacharequal}{\kern0pt}\ {\isadigit{2}}{\isacharcircum}{\kern0pt}n{\isachardoublequoteclose}\ \isacommand{using}\isamarkupfalse%
\ gate{\isacharunderscore}{\kern0pt}def\ assms\ \isacommand{by}\isamarkupfalse%
\ auto\isanewline
\isacommand{next}\isamarkupfalse%
\isanewline
\ \ \isacommand{show}\isamarkupfalse%
\ {\isachardoublequoteopen}dim{\isacharunderscore}{\kern0pt}col\ U\ {\isacharequal}{\kern0pt}\ {\isadigit{2}}{\isacharcircum}{\kern0pt}n{\isachardoublequoteclose}\ \isacommand{using}\isamarkupfalse%
\ gate{\isacharunderscore}{\kern0pt}def\ assms\ \isacommand{by}\isamarkupfalse%
\ auto\isanewline
\isacommand{qed}\isamarkupfalse%
%
\endisatagproof
{\isafoldproof}%
%
\isadelimproof
\isanewline
%
\endisadelimproof
\isanewline
\isacommand{lemma}\isamarkupfalse%
\ state{\isacharunderscore}{\kern0pt}carrier{\isacharunderscore}{\kern0pt}mat{\isacharbrackleft}{\kern0pt}simp{\isacharbrackright}{\kern0pt}{\isacharcolon}{\kern0pt}\isanewline
\ \ \isakeyword{assumes}\ {\isachardoublequoteopen}state\ n\ {\isasympsi}{\isachardoublequoteclose}\isanewline
\ \ \isakeyword{shows}\ {\isachardoublequoteopen}{\isasympsi}\ {\isasymin}\ carrier{\isacharunderscore}{\kern0pt}mat\ {\isacharparenleft}{\kern0pt}{\isadigit{2}}{\isacharcircum}{\kern0pt}n{\isacharparenright}{\kern0pt}\ {\isadigit{1}}{\isachardoublequoteclose}\isanewline
%
\isadelimproof
%
\endisadelimproof
%
\isatagproof
\isacommand{proof}\isamarkupfalse%
\isanewline
\ \ \isacommand{show}\isamarkupfalse%
\ {\isachardoublequoteopen}dim{\isacharunderscore}{\kern0pt}row\ {\isasympsi}\ {\isacharequal}{\kern0pt}\ {\isadigit{2}}{\isacharcircum}{\kern0pt}n{\isachardoublequoteclose}\ \isacommand{using}\isamarkupfalse%
\ state{\isacharunderscore}{\kern0pt}def\ assms\ \isacommand{by}\isamarkupfalse%
\ auto\isanewline
\isacommand{next}\isamarkupfalse%
\isanewline
\ \ \isacommand{show}\isamarkupfalse%
\ {\isachardoublequoteopen}dim{\isacharunderscore}{\kern0pt}col\ {\isasympsi}\ {\isacharequal}{\kern0pt}\ {\isadigit{1}}{\isachardoublequoteclose}\ \isacommand{using}\isamarkupfalse%
\ state{\isacharunderscore}{\kern0pt}def\ assms\ \isacommand{by}\isamarkupfalse%
\ auto\isanewline
\isacommand{qed}\isamarkupfalse%
%
\endisatagproof
{\isafoldproof}%
%
\isadelimproof
\isanewline
%
\endisadelimproof
\isanewline
\isacommand{lemma}\isamarkupfalse%
\ state{\isacharunderscore}{\kern0pt}basis{\isacharunderscore}{\kern0pt}carrier{\isacharunderscore}{\kern0pt}mat{\isacharbrackleft}{\kern0pt}simp{\isacharbrackright}{\kern0pt}{\isacharcolon}{\kern0pt}\isanewline
\ \ {\isachardoublequoteopen}{\isacharbar}{\kern0pt}state{\isacharunderscore}{\kern0pt}basis\ n\ j{\isasymrangle}\ {\isasymin}\ carrier{\isacharunderscore}{\kern0pt}mat\ {\isacharparenleft}{\kern0pt}{\isadigit{2}}{\isacharcircum}{\kern0pt}n{\isacharparenright}{\kern0pt}\ {\isadigit{1}}{\isachardoublequoteclose}\isanewline
%
\isadelimproof
\ \ %
\endisadelimproof
%
\isatagproof
\isacommand{by}\isamarkupfalse%
\ {\isacharparenleft}{\kern0pt}simp\ add{\isacharcolon}{\kern0pt}\ ket{\isacharunderscore}{\kern0pt}vec{\isacharunderscore}{\kern0pt}def\ state{\isacharunderscore}{\kern0pt}basis{\isacharunderscore}{\kern0pt}def{\isacharparenright}{\kern0pt}%
\endisatagproof
{\isafoldproof}%
%
\isadelimproof
\isanewline
%
\endisadelimproof
\isanewline
\isacommand{lemma}\isamarkupfalse%
\ left{\isacharunderscore}{\kern0pt}tensor{\isacharunderscore}{\kern0pt}id{\isacharbrackleft}{\kern0pt}simp{\isacharbrackright}{\kern0pt}{\isacharcolon}{\kern0pt}\isanewline
\ \ \isakeyword{assumes}\ {\isachardoublequoteopen}A\ {\isasymin}\ carrier{\isacharunderscore}{\kern0pt}mat\ nr\ nc{\isachardoublequoteclose}\isanewline
\ \ \isakeyword{shows}\ {\isachardoublequoteopen}{\isacharparenleft}{\kern0pt}{\isadigit{1}}\isactrlsub m\ {\isadigit{1}}{\isacharparenright}{\kern0pt}\ {\isasymOtimes}\ A\ {\isacharequal}{\kern0pt}\ A{\isachardoublequoteclose}\isanewline
%
\isadelimproof
\ \ %
\endisadelimproof
%
\isatagproof
\isacommand{by}\isamarkupfalse%
\ auto%
\endisatagproof
{\isafoldproof}%
%
\isadelimproof
\isanewline
%
\endisadelimproof
\isanewline
\isacommand{lemma}\isamarkupfalse%
\ right{\isacharunderscore}{\kern0pt}tensor{\isacharunderscore}{\kern0pt}id{\isacharbrackleft}{\kern0pt}simp{\isacharbrackright}{\kern0pt}{\isacharcolon}{\kern0pt}\isanewline
\ \ \isakeyword{assumes}\ {\isachardoublequoteopen}A\ {\isasymin}\ carrier{\isacharunderscore}{\kern0pt}mat\ nr\ nc{\isachardoublequoteclose}\isanewline
\ \ \isakeyword{shows}\ {\isachardoublequoteopen}A\ {\isasymOtimes}\ {\isacharparenleft}{\kern0pt}{\isadigit{1}}\isactrlsub m\ {\isadigit{1}}{\isacharparenright}{\kern0pt}\ {\isacharequal}{\kern0pt}\ A{\isachardoublequoteclose}\isanewline
%
\isadelimproof
\ \ %
\endisadelimproof
%
\isatagproof
\isacommand{by}\isamarkupfalse%
\ auto%
\endisatagproof
{\isafoldproof}%
%
\isadelimproof
\isanewline
%
\endisadelimproof
\isanewline
\isacommand{lemma}\isamarkupfalse%
\ tensor{\isacharunderscore}{\kern0pt}carrier{\isacharunderscore}{\kern0pt}mat{\isacharbrackleft}{\kern0pt}simp{\isacharbrackright}{\kern0pt}{\isacharcolon}{\kern0pt}\isanewline
\ \ \ \ \isakeyword{assumes}\ {\isachardoublequoteopen}A\ {\isasymin}\ carrier{\isacharunderscore}{\kern0pt}mat\ ra\ ca{\isachardoublequoteclose}\isanewline
\ \ \ \ \isakeyword{and}\ {\isachardoublequoteopen}B\ {\isasymin}\ carrier{\isacharunderscore}{\kern0pt}mat\ rb\ cb{\isachardoublequoteclose}\isanewline
\ \ \isakeyword{shows}\ {\isachardoublequoteopen}A\ {\isasymOtimes}\ B\ {\isasymin}\ carrier{\isacharunderscore}{\kern0pt}mat\ {\isacharparenleft}{\kern0pt}ra{\isacharasterisk}{\kern0pt}rb{\isacharparenright}{\kern0pt}\ {\isacharparenleft}{\kern0pt}ca{\isacharasterisk}{\kern0pt}cb{\isacharparenright}{\kern0pt}{\isachardoublequoteclose}\isanewline
%
\isadelimproof
%
\endisadelimproof
%
\isatagproof
\isacommand{proof}\isamarkupfalse%
\isanewline
\ \ \isacommand{show}\isamarkupfalse%
\ {\isachardoublequoteopen}dim{\isacharunderscore}{\kern0pt}row\ {\isacharparenleft}{\kern0pt}A\ {\isasymOtimes}\ B{\isacharparenright}{\kern0pt}\ {\isacharequal}{\kern0pt}\ ra\ {\isacharasterisk}{\kern0pt}\ rb{\isachardoublequoteclose}\ \isacommand{using}\isamarkupfalse%
\ dim{\isacharunderscore}{\kern0pt}row{\isacharunderscore}{\kern0pt}tensor{\isacharunderscore}{\kern0pt}mat\ assms\ \isacommand{by}\isamarkupfalse%
\ auto\isanewline
\ \ \isacommand{show}\isamarkupfalse%
\ {\isachardoublequoteopen}dim{\isacharunderscore}{\kern0pt}col\ {\isacharparenleft}{\kern0pt}A\ {\isasymOtimes}\ B{\isacharparenright}{\kern0pt}\ {\isacharequal}{\kern0pt}\ ca\ {\isacharasterisk}{\kern0pt}\ cb{\isachardoublequoteclose}\ \isacommand{using}\isamarkupfalse%
\ dim{\isacharunderscore}{\kern0pt}col{\isacharunderscore}{\kern0pt}tensor{\isacharunderscore}{\kern0pt}mat\ assms\ \isacommand{by}\isamarkupfalse%
\ auto\isanewline
\isacommand{qed}\isamarkupfalse%
%
\endisatagproof
{\isafoldproof}%
%
\isadelimproof
\isanewline
%
\endisadelimproof
\isanewline
\isacommand{lemma}\isamarkupfalse%
\ smult{\isacharunderscore}{\kern0pt}tensor{\isacharbrackleft}{\kern0pt}simp{\isacharbrackright}{\kern0pt}{\isacharcolon}{\kern0pt}\isanewline
\ \ \isakeyword{assumes}\ {\isachardoublequoteopen}dim{\isacharunderscore}{\kern0pt}col\ A\ {\isachargreater}{\kern0pt}\ {\isadigit{0}}{\isachardoublequoteclose}\ \isakeyword{and}\ {\isachardoublequoteopen}dim{\isacharunderscore}{\kern0pt}col\ B\ {\isachargreater}{\kern0pt}\ {\isadigit{0}}{\isachardoublequoteclose}\isanewline
\ \ \isakeyword{shows}\ {\isachardoublequoteopen}{\isacharparenleft}{\kern0pt}a\ {\isasymcdot}\isactrlsub m\ A{\isacharparenright}{\kern0pt}\ {\isasymOtimes}\ {\isacharparenleft}{\kern0pt}b\ {\isasymcdot}\isactrlsub m\ B{\isacharparenright}{\kern0pt}\ {\isacharequal}{\kern0pt}\ {\isacharparenleft}{\kern0pt}a{\isacharasterisk}{\kern0pt}b{\isacharparenright}{\kern0pt}\ {\isasymcdot}\isactrlsub m\ {\isacharparenleft}{\kern0pt}A\ {\isasymOtimes}\ B{\isacharparenright}{\kern0pt}{\isachardoublequoteclose}\isanewline
%
\isadelimproof
%
\endisadelimproof
%
\isatagproof
\isacommand{proof}\isamarkupfalse%
\isanewline
\ \ \isacommand{fix}\isamarkupfalse%
\ i\ j{\isacharcolon}{\kern0pt}{\isacharcolon}{\kern0pt}nat\isanewline
\ \ \isacommand{assume}\isamarkupfalse%
\ ai{\isacharcolon}{\kern0pt}{\isachardoublequoteopen}i\ {\isacharless}{\kern0pt}\ dim{\isacharunderscore}{\kern0pt}row\ {\isacharparenleft}{\kern0pt}a\ {\isacharasterisk}{\kern0pt}\ b\ {\isasymcdot}\isactrlsub m\ {\isacharparenleft}{\kern0pt}A\ {\isasymOtimes}\ B{\isacharparenright}{\kern0pt}{\isacharparenright}{\kern0pt}{\isachardoublequoteclose}\ \isakeyword{and}\ aj{\isacharcolon}{\kern0pt}{\isachardoublequoteopen}j\ {\isacharless}{\kern0pt}\ dim{\isacharunderscore}{\kern0pt}col\ {\isacharparenleft}{\kern0pt}a\ {\isacharasterisk}{\kern0pt}\ b\ {\isasymcdot}\isactrlsub m\ {\isacharparenleft}{\kern0pt}A\ {\isasymOtimes}\ B{\isacharparenright}{\kern0pt}{\isacharparenright}{\kern0pt}{\isachardoublequoteclose}\isanewline
\ \ \isacommand{show}\isamarkupfalse%
\ {\isachardoublequoteopen}{\isacharparenleft}{\kern0pt}a\ {\isasymcdot}\isactrlsub m\ A\ {\isasymOtimes}\ b\ {\isasymcdot}\isactrlsub m\ B{\isacharparenright}{\kern0pt}\ {\isachardollar}{\kern0pt}{\isachardollar}{\kern0pt}\ {\isacharparenleft}{\kern0pt}i{\isacharcomma}{\kern0pt}\ j{\isacharparenright}{\kern0pt}\ {\isacharequal}{\kern0pt}\ {\isacharparenleft}{\kern0pt}{\isacharparenleft}{\kern0pt}a\ {\isacharasterisk}{\kern0pt}\ b{\isacharparenright}{\kern0pt}\ {\isasymcdot}\isactrlsub m\ {\isacharparenleft}{\kern0pt}A\ {\isasymOtimes}\ B{\isacharparenright}{\kern0pt}{\isacharparenright}{\kern0pt}\ {\isachardollar}{\kern0pt}{\isachardollar}{\kern0pt}\ {\isacharparenleft}{\kern0pt}i{\isacharcomma}{\kern0pt}\ j{\isacharparenright}{\kern0pt}{\isachardoublequoteclose}\isanewline
\ \ \isacommand{proof}\isamarkupfalse%
\ {\isacharminus}{\kern0pt}\isanewline
\ \ \ \ \isacommand{define}\isamarkupfalse%
\ rA\ cA\ rB\ cB\ \isakeyword{where}\ {\isachardoublequoteopen}rA\ {\isacharequal}{\kern0pt}\ dim{\isacharunderscore}{\kern0pt}row\ A{\isachardoublequoteclose}\ \isakeyword{and}\ {\isachardoublequoteopen}cA\ {\isacharequal}{\kern0pt}\ dim{\isacharunderscore}{\kern0pt}col\ A{\isachardoublequoteclose}\ \isakeyword{and}\ {\isachardoublequoteopen}rB\ {\isacharequal}{\kern0pt}\ dim{\isacharunderscore}{\kern0pt}row\ B{\isachardoublequoteclose}\ \isanewline
\ \ \ \ \ \ \isakeyword{and}\ {\isachardoublequoteopen}cB\ {\isacharequal}{\kern0pt}\ dim{\isacharunderscore}{\kern0pt}col\ B{\isachardoublequoteclose}\isanewline
\ \ \ \ \isacommand{have}\isamarkupfalse%
\ {\isachardoublequoteopen}{\isacharparenleft}{\kern0pt}a\ {\isasymcdot}\isactrlsub m\ A\ {\isasymOtimes}\ b\ {\isasymcdot}\isactrlsub m\ B{\isacharparenright}{\kern0pt}{\isachardollar}{\kern0pt}{\isachardollar}{\kern0pt}{\isacharparenleft}{\kern0pt}i{\isacharcomma}{\kern0pt}\ j{\isacharparenright}{\kern0pt}\ {\isacharequal}{\kern0pt}\ {\isacharparenleft}{\kern0pt}a\ {\isasymcdot}\isactrlsub m\ A{\isacharparenright}{\kern0pt}{\isachardollar}{\kern0pt}{\isachardollar}{\kern0pt}{\isacharparenleft}{\kern0pt}i\ div\ rB{\isacharcomma}{\kern0pt}\ j\ div\ cB{\isacharparenright}{\kern0pt}{\isacharasterisk}{\kern0pt}{\isacharparenleft}{\kern0pt}b\ {\isasymcdot}\isactrlsub m\ B{\isacharparenright}{\kern0pt}{\isachardollar}{\kern0pt}{\isachardollar}{\kern0pt}{\isacharparenleft}{\kern0pt}i\ mod\ rB{\isacharcomma}{\kern0pt}\ j\ mod\ cB{\isacharparenright}{\kern0pt}{\isachardoublequoteclose}\isanewline
\ \ \ \ \isacommand{proof}\isamarkupfalse%
\ {\isacharparenleft}{\kern0pt}rule\ index{\isacharunderscore}{\kern0pt}tensor{\isacharunderscore}{\kern0pt}mat{\isacharparenright}{\kern0pt}\isanewline
\ \ \ \ \ \ \isacommand{show}\isamarkupfalse%
\ {\isachardoublequoteopen}dim{\isacharunderscore}{\kern0pt}row\ {\isacharparenleft}{\kern0pt}a\ {\isasymcdot}\isactrlsub m\ A{\isacharparenright}{\kern0pt}\ {\isacharequal}{\kern0pt}\ rA{\isachardoublequoteclose}\ \isacommand{using}\isamarkupfalse%
\ rA{\isacharunderscore}{\kern0pt}def\ \isacommand{by}\isamarkupfalse%
\ simp\isanewline
\ \ \ \ \ \ \isacommand{show}\isamarkupfalse%
\ {\isachardoublequoteopen}dim{\isacharunderscore}{\kern0pt}col\ {\isacharparenleft}{\kern0pt}a\ {\isasymcdot}\isactrlsub m\ A{\isacharparenright}{\kern0pt}\ {\isacharequal}{\kern0pt}\ cA{\isachardoublequoteclose}\ \isacommand{using}\isamarkupfalse%
\ cA{\isacharunderscore}{\kern0pt}def\ \isacommand{by}\isamarkupfalse%
\ simp\isanewline
\ \ \ \ \ \ \isacommand{show}\isamarkupfalse%
\ {\isachardoublequoteopen}dim{\isacharunderscore}{\kern0pt}row\ {\isacharparenleft}{\kern0pt}b\ {\isasymcdot}\isactrlsub m\ B{\isacharparenright}{\kern0pt}\ {\isacharequal}{\kern0pt}\ rB{\isachardoublequoteclose}\ \isacommand{using}\isamarkupfalse%
\ rB{\isacharunderscore}{\kern0pt}def\ \isacommand{by}\isamarkupfalse%
\ simp\isanewline
\ \ \ \ \ \ \isacommand{show}\isamarkupfalse%
\ {\isachardoublequoteopen}dim{\isacharunderscore}{\kern0pt}col\ {\isacharparenleft}{\kern0pt}b\ {\isasymcdot}\isactrlsub m\ B{\isacharparenright}{\kern0pt}\ {\isacharequal}{\kern0pt}\ cB{\isachardoublequoteclose}\ \isacommand{using}\isamarkupfalse%
\ cB{\isacharunderscore}{\kern0pt}def\ \isacommand{by}\isamarkupfalse%
\ simp\isanewline
\ \ \ \ \ \ \isacommand{show}\isamarkupfalse%
\ {\isachardoublequoteopen}i\ {\isacharless}{\kern0pt}\ rA\ {\isacharasterisk}{\kern0pt}\ rB{\isachardoublequoteclose}\ \isacommand{using}\isamarkupfalse%
\ ai\ rA{\isacharunderscore}{\kern0pt}def\ rB{\isacharunderscore}{\kern0pt}def\ smult{\isacharunderscore}{\kern0pt}carrier{\isacharunderscore}{\kern0pt}mat\ tensor{\isacharunderscore}{\kern0pt}carrier{\isacharunderscore}{\kern0pt}mat\ \isacommand{by}\isamarkupfalse%
\ auto\isanewline
\ \ \ \ \ \ \isacommand{show}\isamarkupfalse%
\ {\isachardoublequoteopen}j\ {\isacharless}{\kern0pt}\ cA\ {\isacharasterisk}{\kern0pt}\ cB{\isachardoublequoteclose}\ \isacommand{using}\isamarkupfalse%
\ aj\ cA{\isacharunderscore}{\kern0pt}def\ cB{\isacharunderscore}{\kern0pt}def\ smult{\isacharunderscore}{\kern0pt}carrier{\isacharunderscore}{\kern0pt}mat\ tensor{\isacharunderscore}{\kern0pt}carrier{\isacharunderscore}{\kern0pt}mat\ \isacommand{by}\isamarkupfalse%
\ auto\isanewline
\ \ \ \ \ \ \isacommand{show}\isamarkupfalse%
\ {\isachardoublequoteopen}{\isadigit{0}}\ {\isacharless}{\kern0pt}\ cA{\isachardoublequoteclose}\ \isacommand{using}\isamarkupfalse%
\ cA{\isacharunderscore}{\kern0pt}def\ assms{\isacharparenleft}{\kern0pt}{\isadigit{1}}{\isacharparenright}{\kern0pt}\ \isacommand{by}\isamarkupfalse%
\ simp\isanewline
\ \ \ \ \ \ \isacommand{show}\isamarkupfalse%
\ {\isachardoublequoteopen}{\isadigit{0}}\ {\isacharless}{\kern0pt}\ cB{\isachardoublequoteclose}\ \isacommand{using}\isamarkupfalse%
\ cB{\isacharunderscore}{\kern0pt}def\ assms{\isacharparenleft}{\kern0pt}{\isadigit{2}}{\isacharparenright}{\kern0pt}\ \isacommand{by}\isamarkupfalse%
\ simp\isanewline
\ \ \ \ \isacommand{qed}\isamarkupfalse%
\isanewline
\ \ \ \ \isacommand{also}\isamarkupfalse%
\ \isacommand{have}\isamarkupfalse%
\ {\isachardoublequoteopen}{\isasymdots}\ {\isacharequal}{\kern0pt}\ a{\isacharasterisk}{\kern0pt}A{\isachardollar}{\kern0pt}{\isachardollar}{\kern0pt}{\isacharparenleft}{\kern0pt}i\ div\ rB{\isacharcomma}{\kern0pt}\ j\ div\ cB{\isacharparenright}{\kern0pt}{\isacharasterisk}{\kern0pt}b{\isacharasterisk}{\kern0pt}B{\isachardollar}{\kern0pt}{\isachardollar}{\kern0pt}{\isacharparenleft}{\kern0pt}i\ mod\ rB{\isacharcomma}{\kern0pt}\ j\ mod\ cB{\isacharparenright}{\kern0pt}{\isachardoublequoteclose}\isanewline
\ \ \ \ \ \ \isacommand{using}\isamarkupfalse%
\ index{\isacharunderscore}{\kern0pt}smult{\isacharunderscore}{\kern0pt}mat\ \isacommand{by}\isamarkupfalse%
\ {\isacharparenleft}{\kern0pt}smt\ {\isacharparenleft}{\kern0pt}verit{\isacharparenright}{\kern0pt}\ Euclidean{\isacharunderscore}{\kern0pt}Rings{\isachardot}{\kern0pt}div{\isacharunderscore}{\kern0pt}eq{\isacharunderscore}{\kern0pt}{\isadigit{0}}{\isacharunderscore}{\kern0pt}iff\ \isanewline
\ \ \ \ \ \ \ \ \ \ ab{\isacharunderscore}{\kern0pt}semigroup{\isacharunderscore}{\kern0pt}mult{\isacharunderscore}{\kern0pt}class{\isachardot}{\kern0pt}mult{\isacharunderscore}{\kern0pt}ac{\isacharparenleft}{\kern0pt}{\isadigit{1}}{\isacharparenright}{\kern0pt}\ ai\ aj\ cB{\isacharunderscore}{\kern0pt}def\ dim{\isacharunderscore}{\kern0pt}col{\isacharunderscore}{\kern0pt}tensor{\isacharunderscore}{\kern0pt}mat\ dim{\isacharunderscore}{\kern0pt}row{\isacharunderscore}{\kern0pt}tensor{\isacharunderscore}{\kern0pt}mat\ \isanewline
\ \ \ \ \ \ \ \ \ \ less{\isacharunderscore}{\kern0pt}mult{\isacharunderscore}{\kern0pt}imp{\isacharunderscore}{\kern0pt}div{\isacharunderscore}{\kern0pt}less\ mod{\isacharunderscore}{\kern0pt}less{\isacharunderscore}{\kern0pt}divisor\ mult{\isacharunderscore}{\kern0pt}{\isadigit{0}}{\isacharunderscore}{\kern0pt}right\ not{\isacharunderscore}{\kern0pt}gr{\isadigit{0}}\ rB{\isacharunderscore}{\kern0pt}def{\isacharparenright}{\kern0pt}\isanewline
\ \ \ \ \isacommand{also}\isamarkupfalse%
\ \isacommand{have}\isamarkupfalse%
\ {\isachardoublequoteopen}{\isasymdots}\ {\isacharequal}{\kern0pt}\ {\isacharparenleft}{\kern0pt}a{\isacharasterisk}{\kern0pt}b{\isacharparenright}{\kern0pt}{\isacharasterisk}{\kern0pt}{\isacharparenleft}{\kern0pt}A{\isachardollar}{\kern0pt}{\isachardollar}{\kern0pt}{\isacharparenleft}{\kern0pt}i\ div\ rB{\isacharcomma}{\kern0pt}\ j\ div\ cB{\isacharparenright}{\kern0pt}{\isacharasterisk}{\kern0pt}B{\isachardollar}{\kern0pt}{\isachardollar}{\kern0pt}{\isacharparenleft}{\kern0pt}i\ mod\ rB{\isacharcomma}{\kern0pt}\ j\ mod\ cB{\isacharparenright}{\kern0pt}{\isacharparenright}{\kern0pt}{\isachardoublequoteclose}\ \isacommand{by}\isamarkupfalse%
\ auto\isanewline
\ \ \ \ \isacommand{also}\isamarkupfalse%
\ \isacommand{have}\isamarkupfalse%
\ {\isachardoublequoteopen}{\isasymdots}\ {\isacharequal}{\kern0pt}\ {\isacharparenleft}{\kern0pt}a{\isacharasterisk}{\kern0pt}b{\isacharparenright}{\kern0pt}{\isacharasterisk}{\kern0pt}{\isacharparenleft}{\kern0pt}{\isacharparenleft}{\kern0pt}A\ {\isasymOtimes}\ B{\isacharparenright}{\kern0pt}\ {\isachardollar}{\kern0pt}{\isachardollar}{\kern0pt}\ {\isacharparenleft}{\kern0pt}i{\isacharcomma}{\kern0pt}j{\isacharparenright}{\kern0pt}{\isacharparenright}{\kern0pt}{\isachardoublequoteclose}\isanewline
\ \ \ \ \isacommand{proof}\isamarkupfalse%
\ {\isacharminus}{\kern0pt}\isanewline
\ \ \ \ \ \ \isacommand{have}\isamarkupfalse%
\ {\isachardoublequoteopen}{\isacharparenleft}{\kern0pt}A\ {\isasymOtimes}\ B{\isacharparenright}{\kern0pt}\ {\isachardollar}{\kern0pt}{\isachardollar}{\kern0pt}\ {\isacharparenleft}{\kern0pt}i{\isacharcomma}{\kern0pt}j{\isacharparenright}{\kern0pt}\ {\isacharequal}{\kern0pt}\ A{\isachardollar}{\kern0pt}{\isachardollar}{\kern0pt}{\isacharparenleft}{\kern0pt}i\ div\ rB{\isacharcomma}{\kern0pt}\ j\ div\ cB{\isacharparenright}{\kern0pt}{\isacharasterisk}{\kern0pt}B{\isachardollar}{\kern0pt}{\isachardollar}{\kern0pt}{\isacharparenleft}{\kern0pt}i\ mod\ rB{\isacharcomma}{\kern0pt}\ j\ mod\ cB{\isacharparenright}{\kern0pt}{\isachardoublequoteclose}\isanewline
\ \ \ \ \ \ \ \ \isacommand{using}\isamarkupfalse%
\ index{\isacharunderscore}{\kern0pt}tensor{\isacharunderscore}{\kern0pt}mat\ rA{\isacharunderscore}{\kern0pt}def\ cA{\isacharunderscore}{\kern0pt}def\ rB{\isacharunderscore}{\kern0pt}def\ cB{\isacharunderscore}{\kern0pt}def\ ai\ aj\ smult{\isacharunderscore}{\kern0pt}carrier{\isacharunderscore}{\kern0pt}mat\ \isanewline
\ \ \ \ \ \ \ \ \ \ tensor{\isacharunderscore}{\kern0pt}carrier{\isacharunderscore}{\kern0pt}mat\ assms\ \isacommand{by}\isamarkupfalse%
\ auto\isanewline
\ \ \ \ \ \ \isacommand{thus}\isamarkupfalse%
\ {\isacharquery}{\kern0pt}thesis\ \isacommand{by}\isamarkupfalse%
\ simp\isanewline
\ \ \ \ \isacommand{qed}\isamarkupfalse%
\isanewline
\ \ \ \ \isacommand{also}\isamarkupfalse%
\ \isacommand{have}\isamarkupfalse%
\ {\isachardoublequoteopen}{\isasymdots}\ {\isacharequal}{\kern0pt}\ {\isacharparenleft}{\kern0pt}{\isacharparenleft}{\kern0pt}a{\isacharasterisk}{\kern0pt}b{\isacharparenright}{\kern0pt}\ {\isasymcdot}\isactrlsub m\ {\isacharparenleft}{\kern0pt}A\ {\isasymOtimes}\ B{\isacharparenright}{\kern0pt}{\isacharparenright}{\kern0pt}\ {\isachardollar}{\kern0pt}{\isachardollar}{\kern0pt}\ {\isacharparenleft}{\kern0pt}i{\isacharcomma}{\kern0pt}j{\isacharparenright}{\kern0pt}{\isachardoublequoteclose}\ \isacommand{using}\isamarkupfalse%
\ index{\isacharunderscore}{\kern0pt}smult{\isacharunderscore}{\kern0pt}mat{\isacharparenleft}{\kern0pt}{\isadigit{1}}{\isacharparenright}{\kern0pt}\isanewline
\ \ \ \ \ \ \isacommand{by}\isamarkupfalse%
\ {\isacharparenleft}{\kern0pt}metis\ ai\ aj\ index{\isacharunderscore}{\kern0pt}smult{\isacharunderscore}{\kern0pt}mat{\isacharparenleft}{\kern0pt}{\isadigit{2}}{\isacharparenright}{\kern0pt}\ index{\isacharunderscore}{\kern0pt}smult{\isacharunderscore}{\kern0pt}mat{\isacharparenleft}{\kern0pt}{\isadigit{3}}{\isacharparenright}{\kern0pt}{\isacharparenright}{\kern0pt}\isanewline
\ \ \ \ \isacommand{finally}\isamarkupfalse%
\ \isacommand{show}\isamarkupfalse%
\ {\isacharquery}{\kern0pt}thesis\ \isacommand{by}\isamarkupfalse%
\ this\isanewline
\ \ \isacommand{qed}\isamarkupfalse%
\isanewline
\isacommand{next}\isamarkupfalse%
\isanewline
\ \ \isacommand{show}\isamarkupfalse%
\ {\isachardoublequoteopen}dim{\isacharunderscore}{\kern0pt}row\ {\isacharparenleft}{\kern0pt}a\ {\isasymcdot}\isactrlsub m\ A\ {\isasymOtimes}\ b\ {\isasymcdot}\isactrlsub m\ B{\isacharparenright}{\kern0pt}\ {\isacharequal}{\kern0pt}\ dim{\isacharunderscore}{\kern0pt}row\ {\isacharparenleft}{\kern0pt}a\ {\isacharasterisk}{\kern0pt}\ b\ {\isasymcdot}\isactrlsub m\ {\isacharparenleft}{\kern0pt}A\ {\isasymOtimes}\ B{\isacharparenright}{\kern0pt}{\isacharparenright}{\kern0pt}{\isachardoublequoteclose}\ \isacommand{by}\isamarkupfalse%
\ simp\isanewline
\isacommand{next}\isamarkupfalse%
\isanewline
\ \ \isacommand{show}\isamarkupfalse%
\ {\isachardoublequoteopen}dim{\isacharunderscore}{\kern0pt}col\ {\isacharparenleft}{\kern0pt}a\ {\isasymcdot}\isactrlsub m\ A\ {\isasymOtimes}\ b\ {\isasymcdot}\isactrlsub m\ B{\isacharparenright}{\kern0pt}\ {\isacharequal}{\kern0pt}\ dim{\isacharunderscore}{\kern0pt}col\ {\isacharparenleft}{\kern0pt}a\ {\isacharasterisk}{\kern0pt}\ b\ {\isasymcdot}\isactrlsub m\ {\isacharparenleft}{\kern0pt}A\ {\isasymOtimes}\ B{\isacharparenright}{\kern0pt}{\isacharparenright}{\kern0pt}{\isachardoublequoteclose}\ \isacommand{by}\isamarkupfalse%
\ simp\isanewline
\isacommand{qed}\isamarkupfalse%
%
\endisatagproof
{\isafoldproof}%
%
\isadelimproof
\isanewline
%
\endisadelimproof
\isanewline
\isacommand{lemma}\isamarkupfalse%
\ smult{\isacharunderscore}{\kern0pt}tensor{\isadigit{1}}{\isacharbrackleft}{\kern0pt}simp{\isacharbrackright}{\kern0pt}{\isacharcolon}{\kern0pt}\isanewline
\ \ \isakeyword{assumes}\ {\isachardoublequoteopen}dim{\isacharunderscore}{\kern0pt}col\ A\ {\isachargreater}{\kern0pt}\ {\isadigit{0}}{\isachardoublequoteclose}\ \isakeyword{and}\ {\isachardoublequoteopen}dim{\isacharunderscore}{\kern0pt}col\ B\ {\isachargreater}{\kern0pt}\ {\isadigit{0}}{\isachardoublequoteclose}\isanewline
\ \ \isakeyword{shows}\ {\isachardoublequoteopen}a\ {\isasymcdot}\isactrlsub m\ {\isacharparenleft}{\kern0pt}A\ {\isasymOtimes}\ B{\isacharparenright}{\kern0pt}\ {\isacharequal}{\kern0pt}\ {\isacharparenleft}{\kern0pt}a\ {\isasymcdot}\isactrlsub m\ A{\isacharparenright}{\kern0pt}\ {\isasymOtimes}\ B{\isachardoublequoteclose}\isanewline
%
\isadelimproof
%
\endisadelimproof
%
\isatagproof
\isacommand{proof}\isamarkupfalse%
\ {\isacharminus}{\kern0pt}\isanewline
\ \ \isacommand{have}\isamarkupfalse%
\ {\isachardoublequoteopen}a\ {\isasymcdot}\isactrlsub m\ {\isacharparenleft}{\kern0pt}A\ {\isasymOtimes}\ B{\isacharparenright}{\kern0pt}\ {\isacharequal}{\kern0pt}\ {\isacharparenleft}{\kern0pt}a{\isacharasterisk}{\kern0pt}{\isadigit{1}}{\isacharparenright}{\kern0pt}\ {\isasymcdot}\isactrlsub m\ {\isacharparenleft}{\kern0pt}A\ {\isasymOtimes}\ B{\isacharparenright}{\kern0pt}{\isachardoublequoteclose}\ \isacommand{by}\isamarkupfalse%
\ auto\isanewline
\ \ \isacommand{also}\isamarkupfalse%
\ \isacommand{have}\isamarkupfalse%
\ {\isachardoublequoteopen}{\isasymdots}\ {\isacharequal}{\kern0pt}\ {\isacharparenleft}{\kern0pt}a\ {\isasymcdot}\isactrlsub m\ A{\isacharparenright}{\kern0pt}\ {\isasymOtimes}\ {\isacharparenleft}{\kern0pt}{\isadigit{1}}\ {\isasymcdot}\isactrlsub m\ B{\isacharparenright}{\kern0pt}{\isachardoublequoteclose}\ \isacommand{using}\isamarkupfalse%
\ assms\ smult{\isacharunderscore}{\kern0pt}tensor\ \isacommand{by}\isamarkupfalse%
\ simp\isanewline
\ \ \isacommand{also}\isamarkupfalse%
\ \isacommand{have}\isamarkupfalse%
\ {\isachardoublequoteopen}{\isasymdots}\ {\isacharequal}{\kern0pt}\ {\isacharparenleft}{\kern0pt}a\ {\isasymcdot}\isactrlsub m\ A{\isacharparenright}{\kern0pt}\ {\isasymOtimes}\ B{\isachardoublequoteclose}\ \isanewline
\ \ \ \ \isacommand{by}\isamarkupfalse%
\ {\isacharparenleft}{\kern0pt}metis\ eq{\isacharunderscore}{\kern0pt}matI\ index{\isacharunderscore}{\kern0pt}smult{\isacharunderscore}{\kern0pt}mat{\isacharparenleft}{\kern0pt}{\isadigit{1}}{\isacharparenright}{\kern0pt}\ index{\isacharunderscore}{\kern0pt}smult{\isacharunderscore}{\kern0pt}mat{\isacharparenleft}{\kern0pt}{\isadigit{2}}{\isacharparenright}{\kern0pt}\ index{\isacharunderscore}{\kern0pt}smult{\isacharunderscore}{\kern0pt}mat{\isacharparenleft}{\kern0pt}{\isadigit{3}}{\isacharparenright}{\kern0pt}\ mult{\isacharunderscore}{\kern0pt}cancel{\isacharunderscore}{\kern0pt}right{\isadigit{1}}{\isacharparenright}{\kern0pt}\isanewline
\ \ \isacommand{finally}\isamarkupfalse%
\ \isacommand{show}\isamarkupfalse%
\ {\isacharquery}{\kern0pt}thesis\ \isacommand{by}\isamarkupfalse%
\ this\isanewline
\isacommand{qed}\isamarkupfalse%
%
\endisatagproof
{\isafoldproof}%
%
\isadelimproof
\isanewline
%
\endisadelimproof
\isanewline
\isacommand{lemma}\isamarkupfalse%
\ set{\isacharunderscore}{\kern0pt}list{\isacharcolon}{\kern0pt}\isanewline
\ \ {\isachardoublequoteopen}set\ {\isacharbrackleft}{\kern0pt}m{\isachardot}{\kern0pt}{\isachardot}{\kern0pt}{\isacharless}{\kern0pt}n{\isacharbrackright}{\kern0pt}\ {\isacharequal}{\kern0pt}\ {\isacharbraceleft}{\kern0pt}m{\isachardot}{\kern0pt}{\isachardot}{\kern0pt}{\isacharless}{\kern0pt}n{\isacharbraceright}{\kern0pt}{\isachardoublequoteclose}\isanewline
%
\isadelimproof
\ \ %
\endisadelimproof
%
\isatagproof
\isacommand{by}\isamarkupfalse%
\ auto%
\endisatagproof
{\isafoldproof}%
%
\isadelimproof
\isanewline
%
\endisadelimproof
\isanewline
\isacommand{lemma}\isamarkupfalse%
\ sumof{\isadigit{2}}{\isacharcolon}{\kern0pt}\isanewline
\ \ {\isachardoublequoteopen}{\isacharparenleft}{\kern0pt}{\isasymSum}k{\isacharless}{\kern0pt}{\isacharparenleft}{\kern0pt}{\isadigit{2}}{\isacharcolon}{\kern0pt}{\isacharcolon}{\kern0pt}nat{\isacharparenright}{\kern0pt}{\isachardot}{\kern0pt}\ f\ k{\isacharparenright}{\kern0pt}\ {\isacharequal}{\kern0pt}\ f\ {\isadigit{0}}\ {\isacharplus}{\kern0pt}\ f\ {\isadigit{1}}{\isachardoublequoteclose}\isanewline
%
\isadelimproof
\ \ %
\endisadelimproof
%
\isatagproof
\isacommand{by}\isamarkupfalse%
\ {\isacharparenleft}{\kern0pt}metis\ One{\isacharunderscore}{\kern0pt}nat{\isacharunderscore}{\kern0pt}def\ Suc{\isacharunderscore}{\kern0pt}{\isadigit{1}}\ add{\isachardot}{\kern0pt}left{\isacharunderscore}{\kern0pt}neutral\ lessThan{\isacharunderscore}{\kern0pt}{\isadigit{0}}\ sum{\isachardot}{\kern0pt}empty\ sum{\isachardot}{\kern0pt}lessThan{\isacharunderscore}{\kern0pt}Suc{\isacharparenright}{\kern0pt}%
\endisatagproof
{\isafoldproof}%
%
\isadelimproof
\isanewline
%
\endisadelimproof
\isanewline
\isacommand{lemma}\isamarkupfalse%
\ sumof{\isadigit{4}}{\isacharcolon}{\kern0pt}\isanewline
\ \ {\isachardoublequoteopen}{\isacharparenleft}{\kern0pt}{\isasymSum}k{\isacharless}{\kern0pt}{\isacharparenleft}{\kern0pt}{\isadigit{4}}{\isacharcolon}{\kern0pt}{\isacharcolon}{\kern0pt}nat{\isacharparenright}{\kern0pt}{\isachardot}{\kern0pt}\ f\ k{\isacharparenright}{\kern0pt}\ {\isacharequal}{\kern0pt}\ f\ {\isadigit{0}}\ {\isacharplus}{\kern0pt}\ f\ {\isadigit{1}}\ {\isacharplus}{\kern0pt}\ f\ {\isadigit{2}}\ {\isacharplus}{\kern0pt}\ f\ {\isadigit{3}}{\isachardoublequoteclose}\isanewline
%
\isadelimproof
%
\endisadelimproof
%
\isatagproof
\isacommand{proof}\isamarkupfalse%
\ {\isacharminus}{\kern0pt}\isanewline
\ \ \isacommand{have}\isamarkupfalse%
\ {\isachardoublequoteopen}{\isacharparenleft}{\kern0pt}{\isasymSum}k{\isacharless}{\kern0pt}{\isacharparenleft}{\kern0pt}{\isadigit{4}}{\isacharcolon}{\kern0pt}{\isacharcolon}{\kern0pt}nat{\isacharparenright}{\kern0pt}{\isachardot}{\kern0pt}\ f\ k{\isacharparenright}{\kern0pt}\ {\isacharequal}{\kern0pt}\ sum\ f\ {\isacharparenleft}{\kern0pt}set\ {\isacharbrackleft}{\kern0pt}{\isadigit{0}}{\isachardot}{\kern0pt}{\isachardot}{\kern0pt}{\isacharless}{\kern0pt}{\isadigit{4}}{\isacharbrackright}{\kern0pt}{\isacharparenright}{\kern0pt}{\isachardoublequoteclose}\ \isacommand{using}\isamarkupfalse%
\ set{\isacharunderscore}{\kern0pt}list\ atLeast{\isacharunderscore}{\kern0pt}upt\ \isacommand{by}\isamarkupfalse%
\ presburger\isanewline
\ \ \isacommand{also}\isamarkupfalse%
\ \isacommand{have}\isamarkupfalse%
\ {\isachardoublequoteopen}{\isasymdots}\ {\isacharequal}{\kern0pt}\ f\ {\isadigit{0}}\ {\isacharplus}{\kern0pt}\ {\isacharparenleft}{\kern0pt}f\ {\isacharparenleft}{\kern0pt}Suc\ {\isadigit{0}}{\isacharparenright}{\kern0pt}\ {\isacharplus}{\kern0pt}\ {\isacharparenleft}{\kern0pt}f\ {\isadigit{2}}\ {\isacharplus}{\kern0pt}\ f\ {\isadigit{3}}{\isacharparenright}{\kern0pt}{\isacharparenright}{\kern0pt}{\isachardoublequoteclose}\ \isacommand{by}\isamarkupfalse%
\ simp\isanewline
\ \ \isacommand{also}\isamarkupfalse%
\ \isacommand{have}\isamarkupfalse%
\ {\isachardoublequoteopen}{\isasymdots}\ {\isacharequal}{\kern0pt}\ f\ {\isadigit{0}}\ {\isacharplus}{\kern0pt}\ f\ {\isadigit{1}}\ {\isacharplus}{\kern0pt}\ f\ {\isadigit{2}}\ {\isacharplus}{\kern0pt}\ f\ {\isadigit{3}}{\isachardoublequoteclose}\ \isacommand{by}\isamarkupfalse%
\ {\isacharparenleft}{\kern0pt}simp\ add{\isacharcolon}{\kern0pt}\ add{\isachardot}{\kern0pt}commute\ add{\isachardot}{\kern0pt}left{\isacharunderscore}{\kern0pt}commute{\isacharparenright}{\kern0pt}\isanewline
\ \ \isacommand{finally}\isamarkupfalse%
\ \isacommand{show}\isamarkupfalse%
\ {\isacharquery}{\kern0pt}thesis\ \isacommand{by}\isamarkupfalse%
\ this\isanewline
\isacommand{qed}\isamarkupfalse%
%
\endisatagproof
{\isafoldproof}%
%
\isadelimproof
%
\endisadelimproof
%
\isadelimdocument
%
\endisadelimdocument
%
\isatagdocument
%
\isamarkupsection{The operator $R_k$%
}
\isamarkuptrue%
%
\endisatagdocument
{\isafolddocument}%
%
\isadelimdocument
%
\endisadelimdocument
\isacommand{definition}\isamarkupfalse%
\ R{\isacharcolon}{\kern0pt}{\isacharcolon}{\kern0pt}\ {\isachardoublequoteopen}nat\ {\isasymRightarrow}\ complex\ Matrix{\isachardot}{\kern0pt}mat{\isachardoublequoteclose}\ \isakeyword{where}\isanewline
\ \ {\isachardoublequoteopen}R\ k\ {\isacharequal}{\kern0pt}\ mat{\isacharunderscore}{\kern0pt}of{\isacharunderscore}{\kern0pt}cols{\isacharunderscore}{\kern0pt}list\ {\isadigit{2}}\ {\isacharbrackleft}{\kern0pt}{\isacharbrackleft}{\kern0pt}{\isadigit{1}}{\isacharcomma}{\kern0pt}\ {\isadigit{0}}{\isacharbrackright}{\kern0pt}{\isacharcomma}{\kern0pt}\isanewline
\ \ \ \ \ \ \ \ \ \ \ \ \ \ \ \ \ \ \ \ \ \ \ \ \ \ \ \ {\isacharbrackleft}{\kern0pt}{\isadigit{0}}{\isacharcomma}{\kern0pt}\ exp{\isacharparenleft}{\kern0pt}{\isadigit{2}}{\isacharasterisk}{\kern0pt}pi{\isacharasterisk}{\kern0pt}{\isasymi}{\isacharslash}{\kern0pt}{\isadigit{2}}{\isacharcircum}{\kern0pt}k{\isacharparenright}{\kern0pt}{\isacharbrackright}{\kern0pt}{\isacharbrackright}{\kern0pt}{\isachardoublequoteclose}%
\isadelimdocument
%
\endisadelimdocument
%
\isatagdocument
%
\isamarkupsection{The SWAP gate:%
}
\isamarkuptrue%
%
\endisatagdocument
{\isafolddocument}%
%
\isadelimdocument
%
\endisadelimdocument
\isacommand{definition}\isamarkupfalse%
\ SWAP{\isacharcolon}{\kern0pt}{\isacharcolon}{\kern0pt}\ {\isachardoublequoteopen}complex\ Matrix{\isachardot}{\kern0pt}mat{\isachardoublequoteclose}\ \isakeyword{where}\isanewline
\ \ {\isachardoublequoteopen}SWAP\ {\isasymequiv}\ Matrix{\isachardot}{\kern0pt}mat\ {\isadigit{4}}\ {\isadigit{4}}\ {\isacharparenleft}{\kern0pt}{\isasymlambda}{\isacharparenleft}{\kern0pt}i{\isacharcomma}{\kern0pt}j{\isacharparenright}{\kern0pt}{\isachardot}{\kern0pt}\ if\ i{\isacharequal}{\kern0pt}{\isadigit{0}}\ {\isasymand}\ j{\isacharequal}{\kern0pt}{\isadigit{0}}\ then\ {\isadigit{1}}\ else\isanewline
\ \ \ \ \ \ \ \ \ \ \ \ \ \ \ \ \ \ \ \ \ \ \ \ \ \ \ \ \ \ \ \ \ \ if\ i{\isacharequal}{\kern0pt}{\isadigit{1}}\ {\isasymand}\ j{\isacharequal}{\kern0pt}{\isadigit{2}}\ then\ {\isadigit{1}}\ else\isanewline
\ \ \ \ \ \ \ \ \ \ \ \ \ \ \ \ \ \ \ \ \ \ \ \ \ \ \ \ \ \ \ \ \ \ if\ i{\isacharequal}{\kern0pt}{\isadigit{2}}\ {\isasymand}\ j{\isacharequal}{\kern0pt}{\isadigit{1}}\ then\ {\isadigit{1}}\ else\isanewline
\ \ \ \ \ \ \ \ \ \ \ \ \ \ \ \ \ \ \ \ \ \ \ \ \ \ \ \ \ \ \ \ \ \ if\ i{\isacharequal}{\kern0pt}{\isadigit{3}}\ {\isasymand}\ j{\isacharequal}{\kern0pt}{\isadigit{3}}\ then\ {\isadigit{1}}\ else\ {\isadigit{0}}{\isacharparenright}{\kern0pt}{\isachardoublequoteclose}\isanewline
\isanewline
\isacommand{lemma}\isamarkupfalse%
\ SWAP{\isacharunderscore}{\kern0pt}index{\isacharcolon}{\kern0pt}\isanewline
\ \ {\isachardoublequoteopen}SWAP\ {\isachardollar}{\kern0pt}{\isachardollar}{\kern0pt}\ {\isacharparenleft}{\kern0pt}{\isadigit{0}}{\isacharcomma}{\kern0pt}{\isadigit{0}}{\isacharparenright}{\kern0pt}\ {\isacharequal}{\kern0pt}\ {\isadigit{1}}\ {\isasymand}\isanewline
\ \ \ SWAP\ {\isachardollar}{\kern0pt}{\isachardollar}{\kern0pt}\ {\isacharparenleft}{\kern0pt}{\isadigit{0}}{\isacharcomma}{\kern0pt}{\isadigit{1}}{\isacharparenright}{\kern0pt}\ {\isacharequal}{\kern0pt}\ {\isadigit{0}}\ {\isasymand}\isanewline
\ \ \ SWAP\ {\isachardollar}{\kern0pt}{\isachardollar}{\kern0pt}\ {\isacharparenleft}{\kern0pt}{\isadigit{0}}{\isacharcomma}{\kern0pt}{\isadigit{2}}{\isacharparenright}{\kern0pt}\ {\isacharequal}{\kern0pt}\ {\isadigit{0}}\ {\isasymand}\isanewline
\ \ \ SWAP\ {\isachardollar}{\kern0pt}{\isachardollar}{\kern0pt}\ {\isacharparenleft}{\kern0pt}{\isadigit{0}}{\isacharcomma}{\kern0pt}{\isadigit{3}}{\isacharparenright}{\kern0pt}\ {\isacharequal}{\kern0pt}\ {\isadigit{0}}\ {\isasymand}\isanewline
\ \ \ SWAP\ {\isachardollar}{\kern0pt}{\isachardollar}{\kern0pt}\ {\isacharparenleft}{\kern0pt}{\isadigit{1}}{\isacharcomma}{\kern0pt}{\isadigit{0}}{\isacharparenright}{\kern0pt}\ {\isacharequal}{\kern0pt}\ {\isadigit{0}}\ {\isasymand}\isanewline
\ \ \ SWAP\ {\isachardollar}{\kern0pt}{\isachardollar}{\kern0pt}\ {\isacharparenleft}{\kern0pt}{\isadigit{1}}{\isacharcomma}{\kern0pt}{\isadigit{1}}{\isacharparenright}{\kern0pt}\ {\isacharequal}{\kern0pt}\ {\isadigit{0}}\ {\isasymand}\isanewline
\ \ \ SWAP\ {\isachardollar}{\kern0pt}{\isachardollar}{\kern0pt}\ {\isacharparenleft}{\kern0pt}{\isadigit{1}}{\isacharcomma}{\kern0pt}{\isadigit{2}}{\isacharparenright}{\kern0pt}\ {\isacharequal}{\kern0pt}\ {\isadigit{1}}\ {\isasymand}\isanewline
\ \ \ SWAP\ {\isachardollar}{\kern0pt}{\isachardollar}{\kern0pt}\ {\isacharparenleft}{\kern0pt}{\isadigit{1}}{\isacharcomma}{\kern0pt}{\isadigit{3}}{\isacharparenright}{\kern0pt}\ {\isacharequal}{\kern0pt}\ {\isadigit{0}}\ {\isasymand}\isanewline
\ \ \ SWAP\ {\isachardollar}{\kern0pt}{\isachardollar}{\kern0pt}\ {\isacharparenleft}{\kern0pt}{\isadigit{2}}{\isacharcomma}{\kern0pt}{\isadigit{0}}{\isacharparenright}{\kern0pt}\ {\isacharequal}{\kern0pt}\ {\isadigit{0}}\ {\isasymand}\isanewline
\ \ \ SWAP\ {\isachardollar}{\kern0pt}{\isachardollar}{\kern0pt}\ {\isacharparenleft}{\kern0pt}{\isadigit{2}}{\isacharcomma}{\kern0pt}{\isadigit{1}}{\isacharparenright}{\kern0pt}\ {\isacharequal}{\kern0pt}\ {\isadigit{1}}\ {\isasymand}\isanewline
\ \ \ SWAP\ {\isachardollar}{\kern0pt}{\isachardollar}{\kern0pt}\ {\isacharparenleft}{\kern0pt}{\isadigit{2}}{\isacharcomma}{\kern0pt}{\isadigit{2}}{\isacharparenright}{\kern0pt}\ {\isacharequal}{\kern0pt}\ {\isadigit{0}}\ {\isasymand}\isanewline
\ \ \ SWAP\ {\isachardollar}{\kern0pt}{\isachardollar}{\kern0pt}\ {\isacharparenleft}{\kern0pt}{\isadigit{2}}{\isacharcomma}{\kern0pt}{\isadigit{3}}{\isacharparenright}{\kern0pt}\ {\isacharequal}{\kern0pt}\ {\isadigit{0}}\ {\isasymand}\isanewline
\ \ \ SWAP\ {\isachardollar}{\kern0pt}{\isachardollar}{\kern0pt}\ {\isacharparenleft}{\kern0pt}{\isadigit{3}}{\isacharcomma}{\kern0pt}{\isadigit{0}}{\isacharparenright}{\kern0pt}\ {\isacharequal}{\kern0pt}\ {\isadigit{0}}\ {\isasymand}\isanewline
\ \ \ SWAP\ {\isachardollar}{\kern0pt}{\isachardollar}{\kern0pt}\ {\isacharparenleft}{\kern0pt}{\isadigit{3}}{\isacharcomma}{\kern0pt}{\isadigit{1}}{\isacharparenright}{\kern0pt}\ {\isacharequal}{\kern0pt}\ {\isadigit{0}}\ {\isasymand}\isanewline
\ \ \ SWAP\ {\isachardollar}{\kern0pt}{\isachardollar}{\kern0pt}\ {\isacharparenleft}{\kern0pt}{\isadigit{3}}{\isacharcomma}{\kern0pt}{\isadigit{2}}{\isacharparenright}{\kern0pt}\ {\isacharequal}{\kern0pt}\ {\isadigit{0}}\ {\isasymand}\isanewline
\ \ \ SWAP\ {\isachardollar}{\kern0pt}{\isachardollar}{\kern0pt}\ {\isacharparenleft}{\kern0pt}{\isadigit{3}}{\isacharcomma}{\kern0pt}{\isadigit{3}}{\isacharparenright}{\kern0pt}\ {\isacharequal}{\kern0pt}\ {\isadigit{1}}{\isachardoublequoteclose}\isanewline
%
\isadelimproof
\ \ %
\endisadelimproof
%
\isatagproof
\isacommand{by}\isamarkupfalse%
\ {\isacharparenleft}{\kern0pt}simp\ add{\isacharcolon}{\kern0pt}\ SWAP{\isacharunderscore}{\kern0pt}def{\isacharparenright}{\kern0pt}%
\endisatagproof
{\isafoldproof}%
%
\isadelimproof
\isanewline
%
\endisadelimproof
\isanewline
\isacommand{lemma}\isamarkupfalse%
\ SWAP{\isacharunderscore}{\kern0pt}nrows{\isacharcolon}{\kern0pt}\isanewline
\ \ {\isachardoublequoteopen}dim{\isacharunderscore}{\kern0pt}row\ SWAP\ {\isacharequal}{\kern0pt}\ {\isadigit{4}}{\isachardoublequoteclose}\isanewline
%
\isadelimproof
\ \ %
\endisadelimproof
%
\isatagproof
\isacommand{by}\isamarkupfalse%
\ {\isacharparenleft}{\kern0pt}simp\ add{\isacharcolon}{\kern0pt}\ SWAP{\isacharunderscore}{\kern0pt}def{\isacharparenright}{\kern0pt}%
\endisatagproof
{\isafoldproof}%
%
\isadelimproof
\isanewline
%
\endisadelimproof
\ \ \isanewline
\isacommand{lemma}\isamarkupfalse%
\ SWAP{\isacharunderscore}{\kern0pt}ncols{\isacharcolon}{\kern0pt}\isanewline
\ \ {\isachardoublequoteopen}dim{\isacharunderscore}{\kern0pt}col\ SWAP\ {\isacharequal}{\kern0pt}\ {\isadigit{4}}{\isachardoublequoteclose}\isanewline
%
\isadelimproof
\ \ %
\endisadelimproof
%
\isatagproof
\isacommand{by}\isamarkupfalse%
\ {\isacharparenleft}{\kern0pt}simp\ add{\isacharcolon}{\kern0pt}\ SWAP{\isacharunderscore}{\kern0pt}def{\isacharparenright}{\kern0pt}%
\endisatagproof
{\isafoldproof}%
%
\isadelimproof
\isanewline
%
\endisadelimproof
\isanewline
\isacommand{lemma}\isamarkupfalse%
\ SWAP{\isacharunderscore}{\kern0pt}carrier{\isacharunderscore}{\kern0pt}mat{\isacharbrackleft}{\kern0pt}simp{\isacharbrackright}{\kern0pt}{\isacharcolon}{\kern0pt}\isanewline
\ \ {\isachardoublequoteopen}SWAP\ {\isasymin}\ carrier{\isacharunderscore}{\kern0pt}mat\ {\isadigit{4}}\ {\isadigit{4}}{\isachardoublequoteclose}\isanewline
%
\isadelimproof
\ \ %
\endisadelimproof
%
\isatagproof
\isacommand{using}\isamarkupfalse%
\ SWAP{\isacharunderscore}{\kern0pt}nrows\ SWAP{\isacharunderscore}{\kern0pt}ncols\ \isacommand{by}\isamarkupfalse%
\ auto%
\endisatagproof
{\isafoldproof}%
%
\isadelimproof
%
\endisadelimproof
%
\begin{isamarkuptext}%
The SWAP gate indeed swaps the states of two qubits (it is not necessary to assume unitarity)%
\end{isamarkuptext}\isamarkuptrue%
\isacommand{lemma}\isamarkupfalse%
\ SWAP{\isacharunderscore}{\kern0pt}tensor{\isacharcolon}{\kern0pt}\isanewline
\ \ \ \ \isakeyword{assumes}\ {\isachardoublequoteopen}u\ {\isasymin}\ carrier{\isacharunderscore}{\kern0pt}mat\ {\isadigit{2}}\ {\isadigit{1}}{\isachardoublequoteclose}\isanewline
\ \ \ \ \isakeyword{and}\ {\isachardoublequoteopen}v\ {\isasymin}\ carrier{\isacharunderscore}{\kern0pt}mat\ {\isadigit{2}}\ {\isadigit{1}}{\isachardoublequoteclose}\isanewline
\ \ \isakeyword{shows}\ {\isachardoublequoteopen}SWAP\ {\isacharasterisk}{\kern0pt}\ {\isacharparenleft}{\kern0pt}u\ {\isasymOtimes}\ v{\isacharparenright}{\kern0pt}\ {\isacharequal}{\kern0pt}\ v\ {\isasymOtimes}\ u{\isachardoublequoteclose}\isanewline
%
\isadelimproof
%
\endisadelimproof
%
\isatagproof
\isacommand{proof}\isamarkupfalse%
\isanewline
\ \ \isacommand{show}\isamarkupfalse%
\ {\isachardoublequoteopen}dim{\isacharunderscore}{\kern0pt}row\ {\isacharparenleft}{\kern0pt}SWAP\ {\isacharasterisk}{\kern0pt}\ {\isacharparenleft}{\kern0pt}u\ {\isasymOtimes}\ v{\isacharparenright}{\kern0pt}{\isacharparenright}{\kern0pt}\ {\isacharequal}{\kern0pt}\ dim{\isacharunderscore}{\kern0pt}row\ {\isacharparenleft}{\kern0pt}v\ {\isasymOtimes}\ u{\isacharparenright}{\kern0pt}{\isachardoublequoteclose}\isanewline
\ \ \ \ \isacommand{using}\isamarkupfalse%
\ SWAP{\isacharunderscore}{\kern0pt}nrows\ assms{\isacharparenleft}{\kern0pt}{\isadigit{1}}{\isacharparenright}{\kern0pt}\ assms{\isacharparenleft}{\kern0pt}{\isadigit{2}}{\isacharparenright}{\kern0pt}\ \isacommand{by}\isamarkupfalse%
\ auto\isanewline
\isacommand{next}\isamarkupfalse%
\isanewline
\ \ \isacommand{show}\isamarkupfalse%
\ {\isachardoublequoteopen}dim{\isacharunderscore}{\kern0pt}col\ {\isacharparenleft}{\kern0pt}SWAP\ {\isacharasterisk}{\kern0pt}\ {\isacharparenleft}{\kern0pt}u\ {\isasymOtimes}\ v{\isacharparenright}{\kern0pt}{\isacharparenright}{\kern0pt}\ {\isacharequal}{\kern0pt}\ dim{\isacharunderscore}{\kern0pt}col\ {\isacharparenleft}{\kern0pt}v\ {\isasymOtimes}\ u{\isacharparenright}{\kern0pt}{\isachardoublequoteclose}\isanewline
\ \ \ \ \isacommand{using}\isamarkupfalse%
\ SWAP{\isacharunderscore}{\kern0pt}ncols\ assms\ \isacommand{by}\isamarkupfalse%
\ auto\isanewline
\isacommand{next}\isamarkupfalse%
\isanewline
\ \ \isacommand{fix}\isamarkupfalse%
\ i\ j{\isacharcolon}{\kern0pt}{\isacharcolon}{\kern0pt}nat\ \isacommand{assume}\isamarkupfalse%
\ {\isachardoublequoteopen}i\ {\isacharless}{\kern0pt}\ dim{\isacharunderscore}{\kern0pt}row\ {\isacharparenleft}{\kern0pt}v\ {\isasymOtimes}\ u{\isacharparenright}{\kern0pt}{\isachardoublequoteclose}\ \isakeyword{and}\ {\isachardoublequoteopen}j\ {\isacharless}{\kern0pt}\ dim{\isacharunderscore}{\kern0pt}col\ {\isacharparenleft}{\kern0pt}v\ {\isasymOtimes}\ u{\isacharparenright}{\kern0pt}{\isachardoublequoteclose}\isanewline
\ \ \isacommand{hence}\isamarkupfalse%
\ a{\isadigit{3}}{\isacharcolon}{\kern0pt}{\isachardoublequoteopen}i\ {\isacharless}{\kern0pt}\ {\isadigit{4}}{\isachardoublequoteclose}\ \isakeyword{and}\ a{\isadigit{4}}{\isacharcolon}{\kern0pt}{\isachardoublequoteopen}j\ {\isacharequal}{\kern0pt}\ {\isadigit{0}}{\isachardoublequoteclose}\ \isacommand{using}\isamarkupfalse%
\ assms\ \isacommand{by}\isamarkupfalse%
\ auto\isanewline
\ \ \isacommand{thus}\isamarkupfalse%
\ {\isachardoublequoteopen}{\isacharparenleft}{\kern0pt}SWAP\ {\isacharasterisk}{\kern0pt}\ {\isacharparenleft}{\kern0pt}u\ {\isasymOtimes}\ v{\isacharparenright}{\kern0pt}{\isacharparenright}{\kern0pt}\ {\isachardollar}{\kern0pt}{\isachardollar}{\kern0pt}\ {\isacharparenleft}{\kern0pt}i{\isacharcomma}{\kern0pt}\ j{\isacharparenright}{\kern0pt}\ {\isacharequal}{\kern0pt}\ {\isacharparenleft}{\kern0pt}v\ {\isasymOtimes}\ u{\isacharparenright}{\kern0pt}\ {\isachardollar}{\kern0pt}{\isachardollar}{\kern0pt}\ {\isacharparenleft}{\kern0pt}i{\isacharcomma}{\kern0pt}\ j{\isacharparenright}{\kern0pt}{\isachardoublequoteclose}\isanewline
\ \ \isacommand{proof}\isamarkupfalse%
\ {\isacharminus}{\kern0pt}\isanewline
\ \ \ \ \isacommand{define}\isamarkupfalse%
\ u{\isadigit{0}}\ \isakeyword{where}\ {\isachardoublequoteopen}u{\isadigit{0}}\ {\isacharequal}{\kern0pt}\ u\ {\isachardollar}{\kern0pt}{\isachardollar}{\kern0pt}\ {\isacharparenleft}{\kern0pt}{\isadigit{0}}{\isacharcomma}{\kern0pt}{\isadigit{0}}{\isacharparenright}{\kern0pt}{\isachardoublequoteclose}\isanewline
\ \ \ \ \isacommand{define}\isamarkupfalse%
\ u{\isadigit{1}}\ \isakeyword{where}\ {\isachardoublequoteopen}u{\isadigit{1}}\ {\isacharequal}{\kern0pt}\ u\ {\isachardollar}{\kern0pt}{\isachardollar}{\kern0pt}\ {\isacharparenleft}{\kern0pt}{\isadigit{1}}{\isacharcomma}{\kern0pt}{\isadigit{0}}{\isacharparenright}{\kern0pt}{\isachardoublequoteclose}\isanewline
\ \ \ \ \isacommand{define}\isamarkupfalse%
\ v{\isadigit{0}}\ \isakeyword{where}\ {\isachardoublequoteopen}v{\isadigit{0}}\ {\isacharequal}{\kern0pt}\ v\ {\isachardollar}{\kern0pt}{\isachardollar}{\kern0pt}\ {\isacharparenleft}{\kern0pt}{\isadigit{0}}{\isacharcomma}{\kern0pt}{\isadigit{0}}{\isacharparenright}{\kern0pt}{\isachardoublequoteclose}\isanewline
\ \ \ \ \isacommand{define}\isamarkupfalse%
\ v{\isadigit{1}}\ \isakeyword{where}\ {\isachardoublequoteopen}v{\isadigit{1}}\ {\isacharequal}{\kern0pt}\ v\ {\isachardollar}{\kern0pt}{\isachardollar}{\kern0pt}\ {\isacharparenleft}{\kern0pt}{\isadigit{1}}{\isacharcomma}{\kern0pt}{\isadigit{0}}{\isacharparenright}{\kern0pt}{\isachardoublequoteclose}\isanewline
\ \ \ \ \isacommand{have}\isamarkupfalse%
\ vu{\isadigit{0}}{\isacharcolon}{\kern0pt}{\isachardoublequoteopen}{\isacharparenleft}{\kern0pt}v\ {\isasymOtimes}\ u{\isacharparenright}{\kern0pt}\ {\isachardollar}{\kern0pt}{\isachardollar}{\kern0pt}\ {\isacharparenleft}{\kern0pt}{\isadigit{0}}{\isacharcomma}{\kern0pt}{\isadigit{0}}{\isacharparenright}{\kern0pt}\ {\isacharequal}{\kern0pt}\ v{\isadigit{0}}{\isacharasterisk}{\kern0pt}u{\isadigit{0}}{\isachardoublequoteclose}\ \isacommand{using}\isamarkupfalse%
\ index{\isacharunderscore}{\kern0pt}tensor{\isacharunderscore}{\kern0pt}mat\ assms\ u{\isadigit{0}}{\isacharunderscore}{\kern0pt}def\ v{\isadigit{0}}{\isacharunderscore}{\kern0pt}def\ \isacommand{by}\isamarkupfalse%
\ auto\isanewline
\ \ \ \ \isacommand{have}\isamarkupfalse%
\ vu{\isadigit{1}}{\isacharcolon}{\kern0pt}{\isachardoublequoteopen}{\isacharparenleft}{\kern0pt}v\ {\isasymOtimes}\ u{\isacharparenright}{\kern0pt}\ {\isachardollar}{\kern0pt}{\isachardollar}{\kern0pt}\ {\isacharparenleft}{\kern0pt}{\isadigit{1}}{\isacharcomma}{\kern0pt}{\isadigit{0}}{\isacharparenright}{\kern0pt}\ {\isacharequal}{\kern0pt}\ v{\isadigit{0}}{\isacharasterisk}{\kern0pt}u{\isadigit{1}}{\isachardoublequoteclose}\ \isacommand{using}\isamarkupfalse%
\ index{\isacharunderscore}{\kern0pt}tensor{\isacharunderscore}{\kern0pt}mat\ assms\ u{\isadigit{1}}{\isacharunderscore}{\kern0pt}def\ v{\isadigit{0}}{\isacharunderscore}{\kern0pt}def\ \isacommand{by}\isamarkupfalse%
\ auto\isanewline
\ \ \ \ \isacommand{have}\isamarkupfalse%
\ vu{\isadigit{2}}{\isacharcolon}{\kern0pt}{\isachardoublequoteopen}{\isacharparenleft}{\kern0pt}v\ {\isasymOtimes}\ u{\isacharparenright}{\kern0pt}\ {\isachardollar}{\kern0pt}{\isachardollar}{\kern0pt}\ {\isacharparenleft}{\kern0pt}{\isadigit{2}}{\isacharcomma}{\kern0pt}{\isadigit{0}}{\isacharparenright}{\kern0pt}\ {\isacharequal}{\kern0pt}\ v{\isadigit{1}}{\isacharasterisk}{\kern0pt}u{\isadigit{0}}{\isachardoublequoteclose}\ \isacommand{using}\isamarkupfalse%
\ index{\isacharunderscore}{\kern0pt}tensor{\isacharunderscore}{\kern0pt}mat\ assms\ u{\isadigit{0}}{\isacharunderscore}{\kern0pt}def\ v{\isadigit{1}}{\isacharunderscore}{\kern0pt}def\ \isacommand{by}\isamarkupfalse%
\ auto\isanewline
\ \ \ \ \isacommand{have}\isamarkupfalse%
\ vu{\isadigit{3}}{\isacharcolon}{\kern0pt}{\isachardoublequoteopen}{\isacharparenleft}{\kern0pt}v\ {\isasymOtimes}\ u{\isacharparenright}{\kern0pt}\ {\isachardollar}{\kern0pt}{\isachardollar}{\kern0pt}\ {\isacharparenleft}{\kern0pt}{\isadigit{3}}{\isacharcomma}{\kern0pt}{\isadigit{0}}{\isacharparenright}{\kern0pt}\ {\isacharequal}{\kern0pt}\ v{\isadigit{1}}{\isacharasterisk}{\kern0pt}u{\isadigit{1}}{\isachardoublequoteclose}\ \isacommand{using}\isamarkupfalse%
\ index{\isacharunderscore}{\kern0pt}tensor{\isacharunderscore}{\kern0pt}mat\ assms\ u{\isadigit{1}}{\isacharunderscore}{\kern0pt}def\ v{\isadigit{1}}{\isacharunderscore}{\kern0pt}def\ \isacommand{by}\isamarkupfalse%
\ auto\isanewline
\ \ \ \ \isacommand{have}\isamarkupfalse%
\ uv{\isadigit{0}}{\isacharcolon}{\kern0pt}{\isachardoublequoteopen}{\isacharparenleft}{\kern0pt}u\ {\isasymOtimes}\ v{\isacharparenright}{\kern0pt}\ {\isachardollar}{\kern0pt}{\isachardollar}{\kern0pt}\ {\isacharparenleft}{\kern0pt}{\isadigit{0}}{\isacharcomma}{\kern0pt}{\isadigit{0}}{\isacharparenright}{\kern0pt}\ {\isacharequal}{\kern0pt}\ u{\isadigit{0}}{\isacharasterisk}{\kern0pt}v{\isadigit{0}}{\isachardoublequoteclose}\ \isacommand{using}\isamarkupfalse%
\ index{\isacharunderscore}{\kern0pt}tensor{\isacharunderscore}{\kern0pt}mat\ assms\ u{\isadigit{0}}{\isacharunderscore}{\kern0pt}def\ v{\isadigit{0}}{\isacharunderscore}{\kern0pt}def\ \isacommand{by}\isamarkupfalse%
\ auto\isanewline
\ \ \ \ \isacommand{have}\isamarkupfalse%
\ uv{\isadigit{1}}{\isacharcolon}{\kern0pt}{\isachardoublequoteopen}{\isacharparenleft}{\kern0pt}u\ {\isasymOtimes}\ v{\isacharparenright}{\kern0pt}\ {\isachardollar}{\kern0pt}{\isachardollar}{\kern0pt}\ {\isacharparenleft}{\kern0pt}{\isadigit{1}}{\isacharcomma}{\kern0pt}{\isadigit{0}}{\isacharparenright}{\kern0pt}\ {\isacharequal}{\kern0pt}\ u{\isadigit{0}}{\isacharasterisk}{\kern0pt}v{\isadigit{1}}{\isachardoublequoteclose}\ \isacommand{using}\isamarkupfalse%
\ index{\isacharunderscore}{\kern0pt}tensor{\isacharunderscore}{\kern0pt}mat\ assms\ u{\isadigit{0}}{\isacharunderscore}{\kern0pt}def\ v{\isadigit{1}}{\isacharunderscore}{\kern0pt}def\ \isacommand{by}\isamarkupfalse%
\ auto\isanewline
\ \ \ \ \isacommand{have}\isamarkupfalse%
\ uv{\isadigit{2}}{\isacharcolon}{\kern0pt}{\isachardoublequoteopen}{\isacharparenleft}{\kern0pt}u\ {\isasymOtimes}\ v{\isacharparenright}{\kern0pt}\ {\isachardollar}{\kern0pt}{\isachardollar}{\kern0pt}\ {\isacharparenleft}{\kern0pt}{\isadigit{2}}{\isacharcomma}{\kern0pt}{\isadigit{0}}{\isacharparenright}{\kern0pt}\ {\isacharequal}{\kern0pt}\ u{\isadigit{1}}{\isacharasterisk}{\kern0pt}v{\isadigit{0}}{\isachardoublequoteclose}\ \isacommand{using}\isamarkupfalse%
\ index{\isacharunderscore}{\kern0pt}tensor{\isacharunderscore}{\kern0pt}mat\ assms\ u{\isadigit{1}}{\isacharunderscore}{\kern0pt}def\ v{\isadigit{0}}{\isacharunderscore}{\kern0pt}def\ \isacommand{by}\isamarkupfalse%
\ auto\isanewline
\ \ \ \ \isacommand{have}\isamarkupfalse%
\ uv{\isadigit{3}}{\isacharcolon}{\kern0pt}{\isachardoublequoteopen}{\isacharparenleft}{\kern0pt}u\ {\isasymOtimes}\ v{\isacharparenright}{\kern0pt}\ {\isachardollar}{\kern0pt}{\isachardollar}{\kern0pt}\ {\isacharparenleft}{\kern0pt}{\isadigit{3}}{\isacharcomma}{\kern0pt}{\isadigit{0}}{\isacharparenright}{\kern0pt}\ {\isacharequal}{\kern0pt}\ u{\isadigit{1}}{\isacharasterisk}{\kern0pt}v{\isadigit{1}}{\isachardoublequoteclose}\ \isacommand{using}\isamarkupfalse%
\ index{\isacharunderscore}{\kern0pt}tensor{\isacharunderscore}{\kern0pt}mat\ assms\ u{\isadigit{1}}{\isacharunderscore}{\kern0pt}def\ v{\isadigit{1}}{\isacharunderscore}{\kern0pt}def\ \isacommand{by}\isamarkupfalse%
\ auto\isanewline
\isanewline
\ \ \ \ \isacommand{have}\isamarkupfalse%
\ uvi{\isacharcolon}{\kern0pt}{\isachardoublequoteopen}Matrix{\isachardot}{\kern0pt}vec\ {\isadigit{4}}\ {\isacharparenleft}{\kern0pt}{\isasymlambda}\ i{\isachardot}{\kern0pt}\ {\isacharparenleft}{\kern0pt}u\ {\isasymOtimes}\ v{\isacharparenright}{\kern0pt}\ {\isachardollar}{\kern0pt}{\isachardollar}{\kern0pt}\ {\isacharparenleft}{\kern0pt}i{\isacharcomma}{\kern0pt}{\isadigit{0}}{\isacharparenright}{\kern0pt}{\isacharparenright}{\kern0pt}\ {\isachardollar}{\kern0pt}\ i\ {\isacharequal}{\kern0pt}\ {\isacharparenleft}{\kern0pt}u\ {\isasymOtimes}\ v{\isacharparenright}{\kern0pt}\ {\isachardollar}{\kern0pt}{\isachardollar}{\kern0pt}\ {\isacharparenleft}{\kern0pt}i{\isacharcomma}{\kern0pt}{\isadigit{0}}{\isacharparenright}{\kern0pt}{\isachardoublequoteclose}\isanewline
\ \ \ \ \ \ \isacommand{using}\isamarkupfalse%
\ a{\isadigit{3}}\ index{\isacharunderscore}{\kern0pt}vec\ \isacommand{by}\isamarkupfalse%
\ blast\isanewline
\ \ \ \ \isacommand{have}\isamarkupfalse%
\ sw{\isacharcolon}{\kern0pt}{\isachardoublequoteopen}{\isasymforall}k{\isacharless}{\kern0pt}{\isadigit{4}}{\isachardot}{\kern0pt}\ Matrix{\isachardot}{\kern0pt}vec\ {\isadigit{4}}\ {\isacharparenleft}{\kern0pt}{\isasymlambda}\ j{\isachardot}{\kern0pt}\ SWAP\ {\isachardollar}{\kern0pt}{\isachardollar}{\kern0pt}\ {\isacharparenleft}{\kern0pt}i{\isacharcomma}{\kern0pt}j{\isacharparenright}{\kern0pt}{\isacharparenright}{\kern0pt}\ {\isachardollar}{\kern0pt}\ k\ {\isacharequal}{\kern0pt}\ SWAP\ {\isachardollar}{\kern0pt}{\isachardollar}{\kern0pt}\ {\isacharparenleft}{\kern0pt}i{\isacharcomma}{\kern0pt}k{\isacharparenright}{\kern0pt}{\isachardoublequoteclose}\isanewline
\ \ \ \ \ \ \isacommand{using}\isamarkupfalse%
\ a{\isadigit{3}}\ index{\isacharunderscore}{\kern0pt}vec\ \isacommand{by}\isamarkupfalse%
\ auto\ \isanewline
\isanewline
\ \ \ \ \isacommand{have}\isamarkupfalse%
\ s{\isadigit{0}}{\isacharcolon}{\kern0pt}{\isachardoublequoteopen}{\isacharparenleft}{\kern0pt}SWAP\ {\isacharasterisk}{\kern0pt}\ {\isacharparenleft}{\kern0pt}u\ {\isasymOtimes}\ v{\isacharparenright}{\kern0pt}{\isacharparenright}{\kern0pt}\ {\isachardollar}{\kern0pt}{\isachardollar}{\kern0pt}\ {\isacharparenleft}{\kern0pt}i{\isacharcomma}{\kern0pt}{\isadigit{0}}{\isacharparenright}{\kern0pt}\ {\isacharequal}{\kern0pt}\ Matrix{\isachardot}{\kern0pt}vec\ {\isacharparenleft}{\kern0pt}dim{\isacharunderscore}{\kern0pt}col\ SWAP{\isacharparenright}{\kern0pt}\ {\isacharparenleft}{\kern0pt}{\isasymlambda}\ j{\isachardot}{\kern0pt}\ SWAP\ {\isachardollar}{\kern0pt}{\isachardollar}{\kern0pt}\ {\isacharparenleft}{\kern0pt}i{\isacharcomma}{\kern0pt}j{\isacharparenright}{\kern0pt}{\isacharparenright}{\kern0pt}\ {\isasymbullet}\ \isanewline
\ \ \ \ \ \ \ \ \ \ \ \ \ \ Matrix{\isachardot}{\kern0pt}vec\ {\isacharparenleft}{\kern0pt}dim{\isacharunderscore}{\kern0pt}row\ {\isacharparenleft}{\kern0pt}u\ {\isasymOtimes}\ v{\isacharparenright}{\kern0pt}{\isacharparenright}{\kern0pt}\ {\isacharparenleft}{\kern0pt}{\isasymlambda}\ i{\isachardot}{\kern0pt}\ {\isacharparenleft}{\kern0pt}u\ {\isasymOtimes}\ v{\isacharparenright}{\kern0pt}\ {\isachardollar}{\kern0pt}{\isachardollar}{\kern0pt}\ {\isacharparenleft}{\kern0pt}i{\isacharcomma}{\kern0pt}{\isadigit{0}}{\isacharparenright}{\kern0pt}{\isacharparenright}{\kern0pt}{\isachardoublequoteclose}\isanewline
\ \ \ \ \ \ \isacommand{by}\isamarkupfalse%
\ {\isacharparenleft}{\kern0pt}metis\ Matrix{\isachardot}{\kern0pt}col{\isacharunderscore}{\kern0pt}def\ Matrix{\isachardot}{\kern0pt}row{\isacharunderscore}{\kern0pt}def\ SWAP{\isacharunderscore}{\kern0pt}nrows\ {\isacartoucheopen}i\ {\isacharless}{\kern0pt}\ {\isadigit{4}}{\isacartoucheclose}\ {\isacartoucheopen}j\ {\isacharless}{\kern0pt}\ dim{\isacharunderscore}{\kern0pt}col\ {\isacharparenleft}{\kern0pt}v\ {\isasymOtimes}\ u{\isacharparenright}{\kern0pt}{\isacartoucheclose}\ {\isacartoucheopen}j\ {\isacharequal}{\kern0pt}\ {\isadigit{0}}{\isacartoucheclose}\ \isanewline
\ \ \ \ \ \ \ \ \ \ dim{\isacharunderscore}{\kern0pt}col{\isacharunderscore}{\kern0pt}tensor{\isacharunderscore}{\kern0pt}mat\ index{\isacharunderscore}{\kern0pt}mult{\isacharunderscore}{\kern0pt}mat{\isacharparenleft}{\kern0pt}{\isadigit{1}}{\isacharparenright}{\kern0pt}\ mult{\isachardot}{\kern0pt}commute{\isacharparenright}{\kern0pt}\isanewline
\ \ \ \ \isacommand{also}\isamarkupfalse%
\ \isacommand{have}\isamarkupfalse%
\ {\isachardoublequoteopen}{\isasymdots}\ {\isacharequal}{\kern0pt}\ Matrix{\isachardot}{\kern0pt}vec\ {\isadigit{4}}\ {\isacharparenleft}{\kern0pt}{\isasymlambda}\ j{\isachardot}{\kern0pt}\ SWAP\ {\isachardollar}{\kern0pt}{\isachardollar}{\kern0pt}\ {\isacharparenleft}{\kern0pt}i{\isacharcomma}{\kern0pt}j{\isacharparenright}{\kern0pt}{\isacharparenright}{\kern0pt}\ {\isasymbullet}\ Matrix{\isachardot}{\kern0pt}vec\ {\isadigit{4}}\ {\isacharparenleft}{\kern0pt}{\isasymlambda}\ i{\isachardot}{\kern0pt}\ {\isacharparenleft}{\kern0pt}u\ {\isasymOtimes}\ v{\isacharparenright}{\kern0pt}\ {\isachardollar}{\kern0pt}{\isachardollar}{\kern0pt}\ {\isacharparenleft}{\kern0pt}i{\isacharcomma}{\kern0pt}{\isadigit{0}}{\isacharparenright}{\kern0pt}{\isacharparenright}{\kern0pt}{\isachardoublequoteclose}\isanewline
\ \ \ \ \ \ \isacommand{using}\isamarkupfalse%
\ SWAP{\isacharunderscore}{\kern0pt}ncols\ assms{\isacharparenleft}{\kern0pt}{\isadigit{1}}{\isacharparenright}{\kern0pt}\ assms{\isacharparenleft}{\kern0pt}{\isadigit{2}}{\isacharparenright}{\kern0pt}\ \isacommand{by}\isamarkupfalse%
\ fastforce\isanewline
\ \ \ \ \isacommand{also}\isamarkupfalse%
\ \isacommand{have}\isamarkupfalse%
\ {\isachardoublequoteopen}{\isasymdots}\ {\isacharequal}{\kern0pt}\ \ {\isacharparenleft}{\kern0pt}{\isasymSum}k{\isacharless}{\kern0pt}{\isadigit{4}}{\isachardot}{\kern0pt}\ {\isacharparenleft}{\kern0pt}{\isacharparenleft}{\kern0pt}Matrix{\isachardot}{\kern0pt}vec\ {\isadigit{4}}\ {\isacharparenleft}{\kern0pt}{\isasymlambda}\ j{\isachardot}{\kern0pt}\ SWAP\ {\isachardollar}{\kern0pt}{\isachardollar}{\kern0pt}\ {\isacharparenleft}{\kern0pt}i{\isacharcomma}{\kern0pt}j{\isacharparenright}{\kern0pt}{\isacharparenright}{\kern0pt}{\isacharparenright}{\kern0pt}\ {\isachardollar}{\kern0pt}\ k{\isacharparenright}{\kern0pt}\ {\isacharasterisk}{\kern0pt}\ \isanewline
\ \ \ \ \ \ \ \ \ \ \ \ \ \ \ \ \ \ \ \ \ \ \ \ \ \ \ \ \ \ {\isacharparenleft}{\kern0pt}{\isacharparenleft}{\kern0pt}Matrix{\isachardot}{\kern0pt}vec\ {\isadigit{4}}\ {\isacharparenleft}{\kern0pt}{\isasymlambda}\ i{\isachardot}{\kern0pt}\ {\isacharparenleft}{\kern0pt}u\ {\isasymOtimes}\ v{\isacharparenright}{\kern0pt}\ {\isachardollar}{\kern0pt}{\isachardollar}{\kern0pt}\ {\isacharparenleft}{\kern0pt}i{\isacharcomma}{\kern0pt}{\isadigit{0}}{\isacharparenright}{\kern0pt}{\isacharparenright}{\kern0pt}{\isacharparenright}{\kern0pt}\ {\isachardollar}{\kern0pt}\ k{\isacharparenright}{\kern0pt}{\isacharparenright}{\kern0pt}{\isachardoublequoteclose}\isanewline
\ \ \ \ \ \ \isacommand{using}\isamarkupfalse%
\ scalar{\isacharunderscore}{\kern0pt}prod{\isacharunderscore}{\kern0pt}def\ \isacommand{by}\isamarkupfalse%
\ {\isacharparenleft}{\kern0pt}metis\ calculation\ dim{\isacharunderscore}{\kern0pt}vec\ lessThan{\isacharunderscore}{\kern0pt}atLeast{\isadigit{0}}{\isacharparenright}{\kern0pt}\isanewline
\ \ \ \ \isacommand{also}\isamarkupfalse%
\ \isacommand{have}\isamarkupfalse%
\ {\isachardoublequoteopen}{\isasymdots}\ {\isacharequal}{\kern0pt}\ SWAP\ {\isachardollar}{\kern0pt}{\isachardollar}{\kern0pt}\ {\isacharparenleft}{\kern0pt}i{\isacharcomma}{\kern0pt}{\isadigit{0}}{\isacharparenright}{\kern0pt}\ {\isacharasterisk}{\kern0pt}\ {\isacharparenleft}{\kern0pt}u\ {\isasymOtimes}\ v{\isacharparenright}{\kern0pt}\ {\isachardollar}{\kern0pt}{\isachardollar}{\kern0pt}\ {\isacharparenleft}{\kern0pt}{\isadigit{0}}{\isacharcomma}{\kern0pt}{\isadigit{0}}{\isacharparenright}{\kern0pt}\ {\isacharplus}{\kern0pt}\isanewline
\ \ \ \ \ \ \ \ \ \ \ \ \ \ \ \ \ \ \ \ SWAP\ {\isachardollar}{\kern0pt}{\isachardollar}{\kern0pt}\ {\isacharparenleft}{\kern0pt}i{\isacharcomma}{\kern0pt}{\isadigit{1}}{\isacharparenright}{\kern0pt}\ {\isacharasterisk}{\kern0pt}\ {\isacharparenleft}{\kern0pt}u\ {\isasymOtimes}\ v{\isacharparenright}{\kern0pt}\ {\isachardollar}{\kern0pt}{\isachardollar}{\kern0pt}\ {\isacharparenleft}{\kern0pt}{\isadigit{1}}{\isacharcomma}{\kern0pt}{\isadigit{0}}{\isacharparenright}{\kern0pt}\ {\isacharplus}{\kern0pt}\isanewline
\ \ \ \ \ \ \ \ \ \ \ \ \ \ \ \ \ \ \ \ SWAP\ {\isachardollar}{\kern0pt}{\isachardollar}{\kern0pt}\ {\isacharparenleft}{\kern0pt}i{\isacharcomma}{\kern0pt}{\isadigit{2}}{\isacharparenright}{\kern0pt}\ {\isacharasterisk}{\kern0pt}\ {\isacharparenleft}{\kern0pt}u\ {\isasymOtimes}\ v{\isacharparenright}{\kern0pt}\ {\isachardollar}{\kern0pt}{\isachardollar}{\kern0pt}\ {\isacharparenleft}{\kern0pt}{\isadigit{2}}{\isacharcomma}{\kern0pt}{\isadigit{0}}{\isacharparenright}{\kern0pt}\ {\isacharplus}{\kern0pt}\isanewline
\ \ \ \ \ \ \ \ \ \ \ \ \ \ \ \ \ \ \ \ SWAP\ {\isachardollar}{\kern0pt}{\isachardollar}{\kern0pt}\ {\isacharparenleft}{\kern0pt}i{\isacharcomma}{\kern0pt}{\isadigit{3}}{\isacharparenright}{\kern0pt}\ {\isacharasterisk}{\kern0pt}\ {\isacharparenleft}{\kern0pt}u\ {\isasymOtimes}\ v{\isacharparenright}{\kern0pt}\ {\isachardollar}{\kern0pt}{\isachardollar}{\kern0pt}\ {\isacharparenleft}{\kern0pt}{\isadigit{3}}{\isacharcomma}{\kern0pt}{\isadigit{0}}{\isacharparenright}{\kern0pt}{\isachardoublequoteclose}\isanewline
\ \ \ \ \ \ \isacommand{using}\isamarkupfalse%
\ sumof{\isadigit{4}}\ \isacommand{by}\isamarkupfalse%
\ auto\isanewline
\ \ \ \ \isacommand{also}\isamarkupfalse%
\ \isacommand{have}\isamarkupfalse%
\ {\isachardoublequoteopen}{\isasymdots}\ {\isacharequal}{\kern0pt}\ SWAP\ {\isachardollar}{\kern0pt}{\isachardollar}{\kern0pt}\ {\isacharparenleft}{\kern0pt}i{\isacharcomma}{\kern0pt}{\isadigit{0}}{\isacharparenright}{\kern0pt}\ {\isacharasterisk}{\kern0pt}\ u{\isadigit{0}}\ {\isacharasterisk}{\kern0pt}\ v{\isadigit{0}}\ {\isacharplus}{\kern0pt}\isanewline
\ \ \ \ \ \ \ \ \ \ \ \ \ \ \ \ \ \ \ \ SWAP\ {\isachardollar}{\kern0pt}{\isachardollar}{\kern0pt}\ {\isacharparenleft}{\kern0pt}i{\isacharcomma}{\kern0pt}{\isadigit{1}}{\isacharparenright}{\kern0pt}\ {\isacharasterisk}{\kern0pt}\ u{\isadigit{0}}\ {\isacharasterisk}{\kern0pt}\ v{\isadigit{1}}\ {\isacharplus}{\kern0pt}\isanewline
\ \ \ \ \ \ \ \ \ \ \ \ \ \ \ \ \ \ \ \ SWAP\ {\isachardollar}{\kern0pt}{\isachardollar}{\kern0pt}\ {\isacharparenleft}{\kern0pt}i{\isacharcomma}{\kern0pt}{\isadigit{2}}{\isacharparenright}{\kern0pt}\ {\isacharasterisk}{\kern0pt}\ u{\isadigit{1}}\ {\isacharasterisk}{\kern0pt}\ v{\isadigit{0}}\ {\isacharplus}{\kern0pt}\isanewline
\ \ \ \ \ \ \ \ \ \ \ \ \ \ \ \ \ \ \ \ SWAP\ {\isachardollar}{\kern0pt}{\isachardollar}{\kern0pt}\ {\isacharparenleft}{\kern0pt}i{\isacharcomma}{\kern0pt}{\isadigit{3}}{\isacharparenright}{\kern0pt}\ {\isacharasterisk}{\kern0pt}\ u{\isadigit{1}}\ {\isacharasterisk}{\kern0pt}\ v{\isadigit{1}}{\isachardoublequoteclose}\isanewline
\ \ \ \ \ \ \isacommand{using}\isamarkupfalse%
\ uv{\isadigit{0}}\ uv{\isadigit{1}}\ uv{\isadigit{2}}\ uv{\isadigit{3}}\ \isacommand{by}\isamarkupfalse%
\ simp\isanewline
\ \ \ \ \isacommand{also}\isamarkupfalse%
\ \isacommand{have}\isamarkupfalse%
\ {\isachardoublequoteopen}{\isasymdots}\ {\isacharequal}{\kern0pt}\ {\isacharparenleft}{\kern0pt}v\ {\isasymOtimes}\ u{\isacharparenright}{\kern0pt}\ {\isachardollar}{\kern0pt}{\isachardollar}{\kern0pt}\ {\isacharparenleft}{\kern0pt}i{\isacharcomma}{\kern0pt}j{\isacharparenright}{\kern0pt}{\isachardoublequoteclose}\isanewline
\ \ \ \ \isacommand{proof}\isamarkupfalse%
\ {\isacharparenleft}{\kern0pt}rule\ disjE{\isacharparenright}{\kern0pt}\isanewline
\ \ \ \ \ \ \isacommand{show}\isamarkupfalse%
\ {\isachardoublequoteopen}i{\isacharequal}{\kern0pt}{\isadigit{0}}\ {\isasymor}\ i{\isacharequal}{\kern0pt}{\isadigit{1}}\ {\isasymor}\ i{\isacharequal}{\kern0pt}{\isadigit{2}}\ {\isasymor}\ i{\isacharequal}{\kern0pt}{\isadigit{3}}{\isachardoublequoteclose}\ \isacommand{using}\isamarkupfalse%
\ a{\isadigit{3}}\ \isacommand{by}\isamarkupfalse%
\ auto\isanewline
\ \ \ \ \isacommand{next}\isamarkupfalse%
\isanewline
\ \ \ \ \ \ \isacommand{assume}\isamarkupfalse%
\ i{\isadigit{0}}{\isacharcolon}{\kern0pt}{\isachardoublequoteopen}i{\isacharequal}{\kern0pt}{\isadigit{0}}{\isachardoublequoteclose}\isanewline
\ \ \ \ \ \ \isacommand{hence}\isamarkupfalse%
\ {\isachardoublequoteopen}SWAP\ {\isachardollar}{\kern0pt}{\isachardollar}{\kern0pt}\ {\isacharparenleft}{\kern0pt}i{\isacharcomma}{\kern0pt}{\isadigit{0}}{\isacharparenright}{\kern0pt}\ {\isacharasterisk}{\kern0pt}\ u{\isadigit{0}}\ {\isacharasterisk}{\kern0pt}\ v{\isadigit{0}}\ {\isacharplus}{\kern0pt}\isanewline
\ \ \ \ \ \ \ \ \ \ \ \ \ SWAP\ {\isachardollar}{\kern0pt}{\isachardollar}{\kern0pt}\ {\isacharparenleft}{\kern0pt}i{\isacharcomma}{\kern0pt}{\isadigit{1}}{\isacharparenright}{\kern0pt}\ {\isacharasterisk}{\kern0pt}\ u{\isadigit{0}}\ {\isacharasterisk}{\kern0pt}\ v{\isadigit{1}}\ {\isacharplus}{\kern0pt}\isanewline
\ \ \ \ \ \ \ \ \ \ \ \ \ SWAP\ {\isachardollar}{\kern0pt}{\isachardollar}{\kern0pt}\ {\isacharparenleft}{\kern0pt}i{\isacharcomma}{\kern0pt}{\isadigit{2}}{\isacharparenright}{\kern0pt}\ {\isacharasterisk}{\kern0pt}\ u{\isadigit{1}}\ {\isacharasterisk}{\kern0pt}\ v{\isadigit{0}}\ {\isacharplus}{\kern0pt}\isanewline
\ \ \ \ \ \ \ \ \ \ \ \ \ SWAP\ {\isachardollar}{\kern0pt}{\isachardollar}{\kern0pt}\ {\isacharparenleft}{\kern0pt}i{\isacharcomma}{\kern0pt}{\isadigit{3}}{\isacharparenright}{\kern0pt}\ {\isacharasterisk}{\kern0pt}\ u{\isadigit{1}}\ {\isacharasterisk}{\kern0pt}\ v{\isadigit{1}}\ {\isacharequal}{\kern0pt}\isanewline
\ \ \ \ \ \ \ \ \ \ \ \ \ SWAP\ {\isachardollar}{\kern0pt}{\isachardollar}{\kern0pt}\ {\isacharparenleft}{\kern0pt}{\isadigit{0}}{\isacharcomma}{\kern0pt}{\isadigit{0}}{\isacharparenright}{\kern0pt}\ {\isacharasterisk}{\kern0pt}\ u{\isadigit{0}}\ {\isacharasterisk}{\kern0pt}\ v{\isadigit{0}}\ {\isacharplus}{\kern0pt}\isanewline
\ \ \ \ \ \ \ \ \ \ \ \ \ SWAP\ {\isachardollar}{\kern0pt}{\isachardollar}{\kern0pt}\ {\isacharparenleft}{\kern0pt}{\isadigit{0}}{\isacharcomma}{\kern0pt}{\isadigit{1}}{\isacharparenright}{\kern0pt}\ {\isacharasterisk}{\kern0pt}\ u{\isadigit{0}}\ {\isacharasterisk}{\kern0pt}\ v{\isadigit{1}}\ {\isacharplus}{\kern0pt}\isanewline
\ \ \ \ \ \ \ \ \ \ \ \ \ SWAP\ {\isachardollar}{\kern0pt}{\isachardollar}{\kern0pt}\ {\isacharparenleft}{\kern0pt}{\isadigit{0}}{\isacharcomma}{\kern0pt}{\isadigit{2}}{\isacharparenright}{\kern0pt}\ {\isacharasterisk}{\kern0pt}\ u{\isadigit{1}}\ {\isacharasterisk}{\kern0pt}\ v{\isadigit{0}}\ {\isacharplus}{\kern0pt}\isanewline
\ \ \ \ \ \ \ \ \ \ \ \ \ SWAP\ {\isachardollar}{\kern0pt}{\isachardollar}{\kern0pt}\ {\isacharparenleft}{\kern0pt}{\isadigit{0}}{\isacharcomma}{\kern0pt}{\isadigit{3}}{\isacharparenright}{\kern0pt}\ {\isacharasterisk}{\kern0pt}\ u{\isadigit{1}}\ {\isacharasterisk}{\kern0pt}\ v{\isadigit{1}}{\isachardoublequoteclose}\ \isacommand{by}\isamarkupfalse%
\ simp\isanewline
\ \ \ \ \ \ \isacommand{also}\isamarkupfalse%
\ \isacommand{have}\isamarkupfalse%
\ {\isachardoublequoteopen}{\isasymdots}\ {\isacharequal}{\kern0pt}\ {\isacharparenleft}{\kern0pt}v\ {\isasymOtimes}\ u{\isacharparenright}{\kern0pt}\ {\isachardollar}{\kern0pt}{\isachardollar}{\kern0pt}\ {\isacharparenleft}{\kern0pt}i{\isacharcomma}{\kern0pt}\ j{\isacharparenright}{\kern0pt}{\isachardoublequoteclose}\ \isacommand{using}\isamarkupfalse%
\ i{\isadigit{0}}\ vu{\isadigit{0}}\ SWAP{\isacharunderscore}{\kern0pt}index\ a{\isadigit{4}}\ \isacommand{by}\isamarkupfalse%
\ simp\isanewline
\ \ \ \ \ \ \isacommand{finally}\isamarkupfalse%
\ \isacommand{show}\isamarkupfalse%
\ {\isacharquery}{\kern0pt}thesis\ \isacommand{by}\isamarkupfalse%
\ this\isanewline
\ \ \ \ \isacommand{next}\isamarkupfalse%
\isanewline
\ \ \ \ \ \ \isacommand{assume}\isamarkupfalse%
\ disj{\isadigit{3}}{\isacharcolon}{\kern0pt}{\isachardoublequoteopen}i\ {\isacharequal}{\kern0pt}\ {\isadigit{1}}\ {\isasymor}\ i\ {\isacharequal}{\kern0pt}\ {\isadigit{2}}\ {\isasymor}\ i\ {\isacharequal}{\kern0pt}\ {\isadigit{3}}{\isachardoublequoteclose}\isanewline
\ \ \ \ \ \ \isacommand{show}\isamarkupfalse%
\ {\isacharquery}{\kern0pt}thesis\isanewline
\ \ \ \ \ \ \isacommand{proof}\isamarkupfalse%
\ {\isacharparenleft}{\kern0pt}rule\ disjE{\isacharparenright}{\kern0pt}\isanewline
\ \ \ \ \ \ \ \ \isacommand{show}\isamarkupfalse%
\ {\isachardoublequoteopen}i\ {\isacharequal}{\kern0pt}\ {\isadigit{1}}\ {\isasymor}\ i\ {\isacharequal}{\kern0pt}\ {\isadigit{2}}\ {\isasymor}\ i\ {\isacharequal}{\kern0pt}\ {\isadigit{3}}{\isachardoublequoteclose}\ \isacommand{using}\isamarkupfalse%
\ disj{\isadigit{3}}\ \isacommand{by}\isamarkupfalse%
\ this\isanewline
\ \ \ \ \ \ \isacommand{next}\isamarkupfalse%
\isanewline
\ \ \ \ \ \ \ \ \isacommand{assume}\isamarkupfalse%
\ i{\isadigit{1}}{\isacharcolon}{\kern0pt}{\isachardoublequoteopen}i{\isacharequal}{\kern0pt}{\isadigit{1}}{\isachardoublequoteclose}\isanewline
\ \ \ \ \ \ \ \ \isacommand{hence}\isamarkupfalse%
\ {\isachardoublequoteopen}SWAP\ {\isachardollar}{\kern0pt}{\isachardollar}{\kern0pt}\ {\isacharparenleft}{\kern0pt}i{\isacharcomma}{\kern0pt}{\isadigit{0}}{\isacharparenright}{\kern0pt}\ {\isacharasterisk}{\kern0pt}\ u{\isadigit{0}}\ {\isacharasterisk}{\kern0pt}\ v{\isadigit{0}}\ {\isacharplus}{\kern0pt}\isanewline
\ \ \ \ \ \ \ \ \ \ \ \ \ \ \ SWAP\ {\isachardollar}{\kern0pt}{\isachardollar}{\kern0pt}\ {\isacharparenleft}{\kern0pt}i{\isacharcomma}{\kern0pt}{\isadigit{1}}{\isacharparenright}{\kern0pt}\ {\isacharasterisk}{\kern0pt}\ u{\isadigit{0}}\ {\isacharasterisk}{\kern0pt}\ v{\isadigit{1}}\ {\isacharplus}{\kern0pt}\isanewline
\ \ \ \ \ \ \ \ \ \ \ \ \ \ \ SWAP\ {\isachardollar}{\kern0pt}{\isachardollar}{\kern0pt}\ {\isacharparenleft}{\kern0pt}i{\isacharcomma}{\kern0pt}{\isadigit{2}}{\isacharparenright}{\kern0pt}\ {\isacharasterisk}{\kern0pt}\ u{\isadigit{1}}\ {\isacharasterisk}{\kern0pt}\ v{\isadigit{0}}\ {\isacharplus}{\kern0pt}\isanewline
\ \ \ \ \ \ \ \ \ \ \ \ \ \ \ SWAP\ {\isachardollar}{\kern0pt}{\isachardollar}{\kern0pt}\ {\isacharparenleft}{\kern0pt}i{\isacharcomma}{\kern0pt}{\isadigit{3}}{\isacharparenright}{\kern0pt}\ {\isacharasterisk}{\kern0pt}\ u{\isadigit{1}}\ {\isacharasterisk}{\kern0pt}\ v{\isadigit{1}}\ {\isacharequal}{\kern0pt}\isanewline
\ \ \ \ \ \ \ \ \ \ \ \ \ \ \ SWAP\ {\isachardollar}{\kern0pt}{\isachardollar}{\kern0pt}\ {\isacharparenleft}{\kern0pt}{\isadigit{1}}{\isacharcomma}{\kern0pt}{\isadigit{0}}{\isacharparenright}{\kern0pt}\ {\isacharasterisk}{\kern0pt}\ u{\isadigit{0}}\ {\isacharasterisk}{\kern0pt}\ v{\isadigit{0}}\ {\isacharplus}{\kern0pt}\isanewline
\ \ \ \ \ \ \ \ \ \ \ \ \ \ \ SWAP\ {\isachardollar}{\kern0pt}{\isachardollar}{\kern0pt}\ {\isacharparenleft}{\kern0pt}{\isadigit{1}}{\isacharcomma}{\kern0pt}{\isadigit{1}}{\isacharparenright}{\kern0pt}\ {\isacharasterisk}{\kern0pt}\ u{\isadigit{0}}\ {\isacharasterisk}{\kern0pt}\ v{\isadigit{1}}\ {\isacharplus}{\kern0pt}\isanewline
\ \ \ \ \ \ \ \ \ \ \ \ \ \ \ SWAP\ {\isachardollar}{\kern0pt}{\isachardollar}{\kern0pt}\ {\isacharparenleft}{\kern0pt}{\isadigit{1}}{\isacharcomma}{\kern0pt}{\isadigit{2}}{\isacharparenright}{\kern0pt}\ {\isacharasterisk}{\kern0pt}\ u{\isadigit{1}}\ {\isacharasterisk}{\kern0pt}\ v{\isadigit{0}}\ {\isacharplus}{\kern0pt}\isanewline
\ \ \ \ \ \ \ \ \ \ \ \ \ \ \ SWAP\ {\isachardollar}{\kern0pt}{\isachardollar}{\kern0pt}\ {\isacharparenleft}{\kern0pt}{\isadigit{1}}{\isacharcomma}{\kern0pt}{\isadigit{3}}{\isacharparenright}{\kern0pt}\ {\isacharasterisk}{\kern0pt}\ u{\isadigit{1}}\ {\isacharasterisk}{\kern0pt}\ v{\isadigit{1}}{\isachardoublequoteclose}\ \isacommand{by}\isamarkupfalse%
\ simp\isanewline
\ \ \ \ \ \ \ \ \isacommand{also}\isamarkupfalse%
\ \isacommand{have}\isamarkupfalse%
\ {\isachardoublequoteopen}{\isasymdots}\ {\isacharequal}{\kern0pt}\ {\isacharparenleft}{\kern0pt}v\ {\isasymOtimes}\ u{\isacharparenright}{\kern0pt}\ {\isachardollar}{\kern0pt}{\isachardollar}{\kern0pt}\ {\isacharparenleft}{\kern0pt}i{\isacharcomma}{\kern0pt}\ j{\isacharparenright}{\kern0pt}{\isachardoublequoteclose}\ \isacommand{using}\isamarkupfalse%
\ i{\isadigit{1}}\ vu{\isadigit{1}}\ SWAP{\isacharunderscore}{\kern0pt}index\ a{\isadigit{4}}\ \isacommand{by}\isamarkupfalse%
\ simp\isanewline
\ \ \ \ \ \ \ \ \isacommand{finally}\isamarkupfalse%
\ \isacommand{show}\isamarkupfalse%
\ {\isacharquery}{\kern0pt}thesis\ \isacommand{by}\isamarkupfalse%
\ this\isanewline
\ \ \ \ \ \ \isacommand{next}\isamarkupfalse%
\isanewline
\ \ \ \ \ \ \ \ \isacommand{assume}\isamarkupfalse%
\ disj{\isadigit{2}}{\isacharcolon}{\kern0pt}{\isachardoublequoteopen}i\ {\isacharequal}{\kern0pt}\ {\isadigit{2}}\ {\isasymor}\ i\ {\isacharequal}{\kern0pt}\ {\isadigit{3}}{\isachardoublequoteclose}\isanewline
\ \ \ \ \ \ \ \ \isacommand{show}\isamarkupfalse%
\ {\isacharquery}{\kern0pt}thesis\isanewline
\ \ \ \ \ \ \ \ \isacommand{proof}\isamarkupfalse%
\ {\isacharparenleft}{\kern0pt}rule\ disjE{\isacharparenright}{\kern0pt}\isanewline
\ \ \ \ \ \ \ \ \ \ \isacommand{show}\isamarkupfalse%
\ {\isachardoublequoteopen}i\ {\isacharequal}{\kern0pt}\ {\isadigit{2}}\ {\isasymor}\ i\ {\isacharequal}{\kern0pt}\ {\isadigit{3}}{\isachardoublequoteclose}\ \isacommand{using}\isamarkupfalse%
\ disj{\isadigit{2}}\ \isacommand{by}\isamarkupfalse%
\ this\isanewline
\ \ \ \ \ \ \ \ \isacommand{next}\isamarkupfalse%
\isanewline
\ \ \ \ \ \ \ \ \ \ \isacommand{assume}\isamarkupfalse%
\ i{\isadigit{2}}{\isacharcolon}{\kern0pt}{\isachardoublequoteopen}i{\isacharequal}{\kern0pt}{\isadigit{2}}{\isachardoublequoteclose}\isanewline
\ \ \ \ \ \ \ \ \ \ \isacommand{hence}\isamarkupfalse%
\ {\isachardoublequoteopen}SWAP\ {\isachardollar}{\kern0pt}{\isachardollar}{\kern0pt}\ {\isacharparenleft}{\kern0pt}i{\isacharcomma}{\kern0pt}{\isadigit{0}}{\isacharparenright}{\kern0pt}\ {\isacharasterisk}{\kern0pt}\ u{\isadigit{0}}\ {\isacharasterisk}{\kern0pt}\ v{\isadigit{0}}\ {\isacharplus}{\kern0pt}\isanewline
\ \ \ \ \ \ \ \ \ \ \ \ \ \ \ \ \ SWAP\ {\isachardollar}{\kern0pt}{\isachardollar}{\kern0pt}\ {\isacharparenleft}{\kern0pt}i{\isacharcomma}{\kern0pt}{\isadigit{1}}{\isacharparenright}{\kern0pt}\ {\isacharasterisk}{\kern0pt}\ u{\isadigit{0}}\ {\isacharasterisk}{\kern0pt}\ v{\isadigit{1}}\ {\isacharplus}{\kern0pt}\isanewline
\ \ \ \ \ \ \ \ \ \ \ \ \ \ \ \ \ SWAP\ {\isachardollar}{\kern0pt}{\isachardollar}{\kern0pt}\ {\isacharparenleft}{\kern0pt}i{\isacharcomma}{\kern0pt}{\isadigit{2}}{\isacharparenright}{\kern0pt}\ {\isacharasterisk}{\kern0pt}\ u{\isadigit{1}}\ {\isacharasterisk}{\kern0pt}\ v{\isadigit{0}}\ {\isacharplus}{\kern0pt}\isanewline
\ \ \ \ \ \ \ \ \ \ \ \ \ \ \ \ \ SWAP\ {\isachardollar}{\kern0pt}{\isachardollar}{\kern0pt}\ {\isacharparenleft}{\kern0pt}i{\isacharcomma}{\kern0pt}{\isadigit{3}}{\isacharparenright}{\kern0pt}\ {\isacharasterisk}{\kern0pt}\ u{\isadigit{1}}\ {\isacharasterisk}{\kern0pt}\ v{\isadigit{1}}\ {\isacharequal}{\kern0pt}\isanewline
\ \ \ \ \ \ \ \ \ \ \ \ \ \ \ \ \ SWAP\ {\isachardollar}{\kern0pt}{\isachardollar}{\kern0pt}\ {\isacharparenleft}{\kern0pt}{\isadigit{2}}{\isacharcomma}{\kern0pt}{\isadigit{0}}{\isacharparenright}{\kern0pt}\ {\isacharasterisk}{\kern0pt}\ u{\isadigit{0}}\ {\isacharasterisk}{\kern0pt}\ v{\isadigit{0}}\ {\isacharplus}{\kern0pt}\isanewline
\ \ \ \ \ \ \ \ \ \ \ \ \ \ \ \ \ SWAP\ {\isachardollar}{\kern0pt}{\isachardollar}{\kern0pt}\ {\isacharparenleft}{\kern0pt}{\isadigit{2}}{\isacharcomma}{\kern0pt}{\isadigit{1}}{\isacharparenright}{\kern0pt}\ {\isacharasterisk}{\kern0pt}\ u{\isadigit{0}}\ {\isacharasterisk}{\kern0pt}\ v{\isadigit{1}}\ {\isacharplus}{\kern0pt}\isanewline
\ \ \ \ \ \ \ \ \ \ \ \ \ \ \ \ \ SWAP\ {\isachardollar}{\kern0pt}{\isachardollar}{\kern0pt}\ {\isacharparenleft}{\kern0pt}{\isadigit{2}}{\isacharcomma}{\kern0pt}{\isadigit{2}}{\isacharparenright}{\kern0pt}\ {\isacharasterisk}{\kern0pt}\ u{\isadigit{1}}\ {\isacharasterisk}{\kern0pt}\ v{\isadigit{0}}\ {\isacharplus}{\kern0pt}\isanewline
\ \ \ \ \ \ \ \ \ \ \ \ \ \ \ \ \ SWAP\ {\isachardollar}{\kern0pt}{\isachardollar}{\kern0pt}\ {\isacharparenleft}{\kern0pt}{\isadigit{2}}{\isacharcomma}{\kern0pt}{\isadigit{3}}{\isacharparenright}{\kern0pt}\ {\isacharasterisk}{\kern0pt}\ u{\isadigit{1}}\ {\isacharasterisk}{\kern0pt}\ v{\isadigit{1}}{\isachardoublequoteclose}\ \isacommand{by}\isamarkupfalse%
\ simp\isanewline
\ \ \ \ \ \ \ \ \isacommand{also}\isamarkupfalse%
\ \isacommand{have}\isamarkupfalse%
\ {\isachardoublequoteopen}{\isasymdots}\ {\isacharequal}{\kern0pt}\ {\isacharparenleft}{\kern0pt}v\ {\isasymOtimes}\ u{\isacharparenright}{\kern0pt}\ {\isachardollar}{\kern0pt}{\isachardollar}{\kern0pt}\ {\isacharparenleft}{\kern0pt}i{\isacharcomma}{\kern0pt}\ j{\isacharparenright}{\kern0pt}{\isachardoublequoteclose}\ \isacommand{using}\isamarkupfalse%
\ i{\isadigit{2}}\ vu{\isadigit{2}}\ SWAP{\isacharunderscore}{\kern0pt}index\ a{\isadigit{4}}\ \isacommand{by}\isamarkupfalse%
\ simp\isanewline
\ \ \ \ \ \ \ \ \isacommand{finally}\isamarkupfalse%
\ \isacommand{show}\isamarkupfalse%
\ {\isacharquery}{\kern0pt}thesis\ \isacommand{by}\isamarkupfalse%
\ this\isanewline
\ \ \ \ \ \ \isacommand{next}\isamarkupfalse%
\isanewline
\ \ \ \ \ \ \ \ \isacommand{assume}\isamarkupfalse%
\ i{\isadigit{3}}{\isacharcolon}{\kern0pt}{\isachardoublequoteopen}i{\isacharequal}{\kern0pt}{\isadigit{3}}{\isachardoublequoteclose}\isanewline
\ \ \ \ \ \ \ \ \isacommand{hence}\isamarkupfalse%
\ {\isachardoublequoteopen}SWAP\ {\isachardollar}{\kern0pt}{\isachardollar}{\kern0pt}\ {\isacharparenleft}{\kern0pt}i{\isacharcomma}{\kern0pt}{\isadigit{0}}{\isacharparenright}{\kern0pt}\ {\isacharasterisk}{\kern0pt}\ u{\isadigit{0}}\ {\isacharasterisk}{\kern0pt}\ v{\isadigit{0}}\ {\isacharplus}{\kern0pt}\isanewline
\ \ \ \ \ \ \ \ \ \ \ \ \ \ \ SWAP\ {\isachardollar}{\kern0pt}{\isachardollar}{\kern0pt}\ {\isacharparenleft}{\kern0pt}i{\isacharcomma}{\kern0pt}{\isadigit{1}}{\isacharparenright}{\kern0pt}\ {\isacharasterisk}{\kern0pt}\ u{\isadigit{0}}\ {\isacharasterisk}{\kern0pt}\ v{\isadigit{1}}\ {\isacharplus}{\kern0pt}\isanewline
\ \ \ \ \ \ \ \ \ \ \ \ \ \ \ SWAP\ {\isachardollar}{\kern0pt}{\isachardollar}{\kern0pt}\ {\isacharparenleft}{\kern0pt}i{\isacharcomma}{\kern0pt}{\isadigit{2}}{\isacharparenright}{\kern0pt}\ {\isacharasterisk}{\kern0pt}\ u{\isadigit{1}}\ {\isacharasterisk}{\kern0pt}\ v{\isadigit{0}}\ {\isacharplus}{\kern0pt}\isanewline
\ \ \ \ \ \ \ \ \ \ \ \ \ \ \ SWAP\ {\isachardollar}{\kern0pt}{\isachardollar}{\kern0pt}\ {\isacharparenleft}{\kern0pt}i{\isacharcomma}{\kern0pt}{\isadigit{3}}{\isacharparenright}{\kern0pt}\ {\isacharasterisk}{\kern0pt}\ u{\isadigit{1}}\ {\isacharasterisk}{\kern0pt}\ v{\isadigit{1}}\ {\isacharequal}{\kern0pt}\isanewline
\ \ \ \ \ \ \ \ \ \ \ \ \ \ \ SWAP\ {\isachardollar}{\kern0pt}{\isachardollar}{\kern0pt}\ {\isacharparenleft}{\kern0pt}{\isadigit{3}}{\isacharcomma}{\kern0pt}{\isadigit{0}}{\isacharparenright}{\kern0pt}\ {\isacharasterisk}{\kern0pt}\ u{\isadigit{0}}\ {\isacharasterisk}{\kern0pt}\ v{\isadigit{0}}\ {\isacharplus}{\kern0pt}\isanewline
\ \ \ \ \ \ \ \ \ \ \ \ \ \ \ SWAP\ {\isachardollar}{\kern0pt}{\isachardollar}{\kern0pt}\ {\isacharparenleft}{\kern0pt}{\isadigit{3}}{\isacharcomma}{\kern0pt}{\isadigit{1}}{\isacharparenright}{\kern0pt}\ {\isacharasterisk}{\kern0pt}\ u{\isadigit{0}}\ {\isacharasterisk}{\kern0pt}\ v{\isadigit{1}}\ {\isacharplus}{\kern0pt}\isanewline
\ \ \ \ \ \ \ \ \ \ \ \ \ \ \ SWAP\ {\isachardollar}{\kern0pt}{\isachardollar}{\kern0pt}\ {\isacharparenleft}{\kern0pt}{\isadigit{3}}{\isacharcomma}{\kern0pt}{\isadigit{2}}{\isacharparenright}{\kern0pt}\ {\isacharasterisk}{\kern0pt}\ u{\isadigit{1}}\ {\isacharasterisk}{\kern0pt}\ v{\isadigit{0}}\ {\isacharplus}{\kern0pt}\isanewline
\ \ \ \ \ \ \ \ \ \ \ \ \ \ \ SWAP\ {\isachardollar}{\kern0pt}{\isachardollar}{\kern0pt}\ {\isacharparenleft}{\kern0pt}{\isadigit{3}}{\isacharcomma}{\kern0pt}{\isadigit{3}}{\isacharparenright}{\kern0pt}\ {\isacharasterisk}{\kern0pt}\ u{\isadigit{1}}\ {\isacharasterisk}{\kern0pt}\ v{\isadigit{1}}{\isachardoublequoteclose}\ \isacommand{by}\isamarkupfalse%
\ simp\isanewline
\ \ \ \ \ \ \ \ \isacommand{also}\isamarkupfalse%
\ \isacommand{have}\isamarkupfalse%
\ {\isachardoublequoteopen}{\isasymdots}\ {\isacharequal}{\kern0pt}\ {\isacharparenleft}{\kern0pt}v\ {\isasymOtimes}\ u{\isacharparenright}{\kern0pt}\ {\isachardollar}{\kern0pt}{\isachardollar}{\kern0pt}\ {\isacharparenleft}{\kern0pt}i{\isacharcomma}{\kern0pt}\ j{\isacharparenright}{\kern0pt}{\isachardoublequoteclose}\ \isacommand{using}\isamarkupfalse%
\ i{\isadigit{3}}\ vu{\isadigit{3}}\ SWAP{\isacharunderscore}{\kern0pt}index\ a{\isadigit{4}}\ \isacommand{by}\isamarkupfalse%
\ simp\isanewline
\ \ \ \ \ \ \ \ \isacommand{finally}\isamarkupfalse%
\ \isacommand{show}\isamarkupfalse%
\ {\isacharquery}{\kern0pt}thesis\ \isacommand{by}\isamarkupfalse%
\ this\isanewline
\ \ \ \ \ \ \isacommand{qed}\isamarkupfalse%
\isanewline
\ \ \ \ \isacommand{qed}\isamarkupfalse%
\isanewline
\ \ \isacommand{qed}\isamarkupfalse%
\isanewline
\ \ \isacommand{finally}\isamarkupfalse%
\ \isacommand{show}\isamarkupfalse%
\ {\isacharquery}{\kern0pt}thesis\ \isacommand{using}\isamarkupfalse%
\ a{\isadigit{4}}\ \isacommand{by}\isamarkupfalse%
\ simp\isanewline
\isacommand{qed}\isamarkupfalse%
\isanewline
\isacommand{qed}\isamarkupfalse%
%
\endisatagproof
{\isafoldproof}%
%
\isadelimproof
%
\endisadelimproof
%
\isadelimdocument
%
\endisadelimdocument
%
\isatagdocument
%
\isamarkupsubsection{Downwards SWAP cascade%
}
\isamarkuptrue%
%
\endisatagdocument
{\isafolddocument}%
%
\isadelimdocument
%
\endisadelimdocument
\isacommand{fun}\isamarkupfalse%
\ SWAP{\isacharunderscore}{\kern0pt}down{\isacharcolon}{\kern0pt}{\isacharcolon}{\kern0pt}\ {\isachardoublequoteopen}nat\ {\isasymRightarrow}\ complex\ Matrix{\isachardot}{\kern0pt}mat{\isachardoublequoteclose}\ \isakeyword{where}\isanewline
\ \ {\isachardoublequoteopen}SWAP{\isacharunderscore}{\kern0pt}down\ {\isadigit{0}}\ {\isacharequal}{\kern0pt}\ {\isadigit{1}}\isactrlsub m\ {\isadigit{1}}{\isachardoublequoteclose}\isanewline
{\isacharbar}{\kern0pt}\ {\isachardoublequoteopen}SWAP{\isacharunderscore}{\kern0pt}down\ {\isacharparenleft}{\kern0pt}Suc\ {\isadigit{0}}{\isacharparenright}{\kern0pt}\ {\isacharequal}{\kern0pt}\ {\isadigit{1}}\isactrlsub m\ {\isadigit{2}}{\isachardoublequoteclose}\isanewline
{\isacharbar}{\kern0pt}\ {\isachardoublequoteopen}SWAP{\isacharunderscore}{\kern0pt}down\ {\isacharparenleft}{\kern0pt}Suc\ {\isacharparenleft}{\kern0pt}Suc\ {\isadigit{0}}{\isacharparenright}{\kern0pt}{\isacharparenright}{\kern0pt}\ {\isacharequal}{\kern0pt}\ SWAP{\isachardoublequoteclose}\isanewline
{\isacharbar}{\kern0pt}\ {\isachardoublequoteopen}SWAP{\isacharunderscore}{\kern0pt}down\ {\isacharparenleft}{\kern0pt}Suc\ {\isacharparenleft}{\kern0pt}Suc\ n{\isacharparenright}{\kern0pt}{\isacharparenright}{\kern0pt}\ {\isacharequal}{\kern0pt}\ {\isacharparenleft}{\kern0pt}{\isacharparenleft}{\kern0pt}{\isadigit{1}}\isactrlsub m\ {\isacharparenleft}{\kern0pt}{\isadigit{2}}{\isacharcircum}{\kern0pt}n{\isacharparenright}{\kern0pt}{\isacharparenright}{\kern0pt}\ {\isasymOtimes}\ SWAP{\isacharparenright}{\kern0pt}\ {\isacharasterisk}{\kern0pt}\ {\isacharparenleft}{\kern0pt}{\isacharparenleft}{\kern0pt}SWAP{\isacharunderscore}{\kern0pt}down\ {\isacharparenleft}{\kern0pt}Suc\ n{\isacharparenright}{\kern0pt}{\isacharparenright}{\kern0pt}\ {\isasymOtimes}\ {\isacharparenleft}{\kern0pt}{\isadigit{1}}\isactrlsub m\ {\isadigit{2}}{\isacharparenright}{\kern0pt}{\isacharparenright}{\kern0pt}{\isachardoublequoteclose}\isanewline
\isanewline
\isacommand{lemma}\isamarkupfalse%
\ SWAP{\isacharunderscore}{\kern0pt}down{\isacharunderscore}{\kern0pt}carrier{\isacharunderscore}{\kern0pt}mat{\isacharbrackleft}{\kern0pt}simp{\isacharbrackright}{\kern0pt}{\isacharcolon}{\kern0pt}\isanewline
\ \ \isakeyword{shows}\ {\isachardoublequoteopen}SWAP{\isacharunderscore}{\kern0pt}down\ n\ {\isasymin}\ carrier{\isacharunderscore}{\kern0pt}mat\ {\isacharparenleft}{\kern0pt}{\isadigit{2}}{\isacharcircum}{\kern0pt}n{\isacharparenright}{\kern0pt}\ {\isacharparenleft}{\kern0pt}{\isadigit{2}}{\isacharcircum}{\kern0pt}n{\isacharparenright}{\kern0pt}{\isachardoublequoteclose}\ {\isacharparenleft}{\kern0pt}\isakeyword{is}\ {\isachardoublequoteopen}{\isacharquery}{\kern0pt}P\ n{\isachardoublequoteclose}{\isacharparenright}{\kern0pt}\isanewline
%
\isadelimproof
%
\endisadelimproof
%
\isatagproof
\isacommand{proof}\isamarkupfalse%
\ {\isacharparenleft}{\kern0pt}induct\ n\ rule{\isacharcolon}{\kern0pt}\ SWAP{\isacharunderscore}{\kern0pt}down{\isachardot}{\kern0pt}induct{\isacharparenright}{\kern0pt}\isanewline
\ \ \isacommand{show}\isamarkupfalse%
\ {\isachardoublequoteopen}{\isacharquery}{\kern0pt}P\ {\isadigit{0}}{\isachardoublequoteclose}\ \isacommand{by}\isamarkupfalse%
\ auto\isanewline
\isacommand{next}\isamarkupfalse%
\isanewline
\ \ \isacommand{show}\isamarkupfalse%
\ {\isachardoublequoteopen}{\isacharquery}{\kern0pt}P\ {\isacharparenleft}{\kern0pt}Suc\ {\isadigit{0}}{\isacharparenright}{\kern0pt}{\isachardoublequoteclose}\ \isacommand{by}\isamarkupfalse%
\ auto\isanewline
\isacommand{next}\isamarkupfalse%
\isanewline
\ \ \isacommand{show}\isamarkupfalse%
\ {\isachardoublequoteopen}{\isacharquery}{\kern0pt}P\ {\isacharparenleft}{\kern0pt}Suc\ {\isacharparenleft}{\kern0pt}Suc\ {\isadigit{0}}{\isacharparenright}{\kern0pt}{\isacharparenright}{\kern0pt}{\isachardoublequoteclose}\ \isacommand{using}\isamarkupfalse%
\ SWAP{\isacharunderscore}{\kern0pt}carrier{\isacharunderscore}{\kern0pt}mat\ \isacommand{by}\isamarkupfalse%
\ auto\isanewline
\isacommand{next}\isamarkupfalse%
\isanewline
\ \ \isacommand{fix}\isamarkupfalse%
\ n{\isacharcolon}{\kern0pt}{\isacharcolon}{\kern0pt}nat\isanewline
\ \ \isacommand{define}\isamarkupfalse%
\ k{\isacharcolon}{\kern0pt}{\isacharcolon}{\kern0pt}nat\ \isakeyword{where}\ {\isachardoublequoteopen}k\ {\isacharequal}{\kern0pt}\ Suc\ n{\isachardoublequoteclose}\isanewline
\ \ \isacommand{assume}\isamarkupfalse%
\ HI{\isacharcolon}{\kern0pt}{\isachardoublequoteopen}SWAP{\isacharunderscore}{\kern0pt}down\ {\isacharparenleft}{\kern0pt}Suc\ k{\isacharparenright}{\kern0pt}\ {\isasymin}\ carrier{\isacharunderscore}{\kern0pt}mat\ {\isacharparenleft}{\kern0pt}{\isadigit{2}}{\isacharcircum}{\kern0pt}{\isacharparenleft}{\kern0pt}Suc\ k{\isacharparenright}{\kern0pt}{\isacharparenright}{\kern0pt}\ {\isacharparenleft}{\kern0pt}{\isadigit{2}}{\isacharcircum}{\kern0pt}{\isacharparenleft}{\kern0pt}Suc\ k{\isacharparenright}{\kern0pt}{\isacharparenright}{\kern0pt}{\isachardoublequoteclose}\isanewline
\ \ \isacommand{show}\isamarkupfalse%
\ {\isachardoublequoteopen}{\isacharquery}{\kern0pt}P\ {\isacharparenleft}{\kern0pt}Suc\ {\isacharparenleft}{\kern0pt}Suc\ k{\isacharparenright}{\kern0pt}{\isacharparenright}{\kern0pt}{\isachardoublequoteclose}\isanewline
\ \ \isacommand{proof}\isamarkupfalse%
\isanewline
\ \ \ \ \isacommand{have}\isamarkupfalse%
\ {\isachardoublequoteopen}dim{\isacharunderscore}{\kern0pt}row\ {\isacharparenleft}{\kern0pt}SWAP{\isacharunderscore}{\kern0pt}down\ {\isacharparenleft}{\kern0pt}Suc\ {\isacharparenleft}{\kern0pt}Suc\ k{\isacharparenright}{\kern0pt}{\isacharparenright}{\kern0pt}{\isacharparenright}{\kern0pt}\ {\isacharequal}{\kern0pt}\ \isanewline
\ \ \ \ \ \ \ \ \ \ dim{\isacharunderscore}{\kern0pt}row\ {\isacharparenleft}{\kern0pt}{\isacharparenleft}{\kern0pt}{\isacharparenleft}{\kern0pt}{\isadigit{1}}\isactrlsub m\ {\isacharparenleft}{\kern0pt}{\isadigit{2}}{\isacharcircum}{\kern0pt}k{\isacharparenright}{\kern0pt}{\isacharparenright}{\kern0pt}\ {\isasymOtimes}\ SWAP{\isacharparenright}{\kern0pt}\ {\isacharasterisk}{\kern0pt}\ {\isacharparenleft}{\kern0pt}{\isacharparenleft}{\kern0pt}SWAP{\isacharunderscore}{\kern0pt}down\ {\isacharparenleft}{\kern0pt}Suc\ k{\isacharparenright}{\kern0pt}{\isacharparenright}{\kern0pt}\ {\isasymOtimes}\ {\isacharparenleft}{\kern0pt}{\isadigit{1}}\isactrlsub m\ {\isadigit{2}}{\isacharparenright}{\kern0pt}{\isacharparenright}{\kern0pt}{\isacharparenright}{\kern0pt}{\isachardoublequoteclose}\isanewline
\ \ \ \ \ \ \isacommand{using}\isamarkupfalse%
\ SWAP{\isacharunderscore}{\kern0pt}down{\isachardot}{\kern0pt}simps{\isacharparenleft}{\kern0pt}{\isadigit{4}}{\isacharparenright}{\kern0pt}\ k{\isacharunderscore}{\kern0pt}def\ \isacommand{by}\isamarkupfalse%
\ simp\isanewline
\ \ \ \ \isacommand{also}\isamarkupfalse%
\ \isacommand{have}\isamarkupfalse%
\ {\isachardoublequoteopen}{\isasymdots}\ {\isacharequal}{\kern0pt}\ dim{\isacharunderscore}{\kern0pt}row\ {\isacharparenleft}{\kern0pt}{\isacharparenleft}{\kern0pt}{\isacharparenleft}{\kern0pt}{\isadigit{1}}\isactrlsub m\ {\isacharparenleft}{\kern0pt}{\isadigit{2}}{\isacharcircum}{\kern0pt}k{\isacharparenright}{\kern0pt}{\isacharparenright}{\kern0pt}\ {\isasymOtimes}\ SWAP{\isacharparenright}{\kern0pt}{\isacharparenright}{\kern0pt}{\isachardoublequoteclose}\ \isacommand{by}\isamarkupfalse%
\ simp\isanewline
\ \ \ \ \isacommand{also}\isamarkupfalse%
\ \isacommand{have}\isamarkupfalse%
\ {\isachardoublequoteopen}{\isasymdots}\ {\isacharequal}{\kern0pt}\ {\isacharparenleft}{\kern0pt}dim{\isacharunderscore}{\kern0pt}row\ {\isacharparenleft}{\kern0pt}{\isacharparenleft}{\kern0pt}{\isadigit{1}}\isactrlsub m\ {\isacharparenleft}{\kern0pt}{\isadigit{2}}{\isacharcircum}{\kern0pt}k{\isacharparenright}{\kern0pt}{\isacharparenright}{\kern0pt}{\isacharparenright}{\kern0pt}{\isacharparenright}{\kern0pt}\ {\isacharasterisk}{\kern0pt}\ {\isacharparenleft}{\kern0pt}dim{\isacharunderscore}{\kern0pt}row\ SWAP{\isacharparenright}{\kern0pt}{\isachardoublequoteclose}\ \isacommand{by}\isamarkupfalse%
\ simp\isanewline
\ \ \ \ \isacommand{thus}\isamarkupfalse%
\ {\isachardoublequoteopen}dim{\isacharunderscore}{\kern0pt}row\ {\isacharparenleft}{\kern0pt}SWAP{\isacharunderscore}{\kern0pt}down\ {\isacharparenleft}{\kern0pt}Suc\ {\isacharparenleft}{\kern0pt}Suc\ k{\isacharparenright}{\kern0pt}{\isacharparenright}{\kern0pt}{\isacharparenright}{\kern0pt}\ {\isacharequal}{\kern0pt}\ {\isadigit{2}}\ {\isacharcircum}{\kern0pt}\ Suc\ {\isacharparenleft}{\kern0pt}Suc\ k{\isacharparenright}{\kern0pt}{\isachardoublequoteclose}\ \isacommand{using}\isamarkupfalse%
\ SWAP{\isacharunderscore}{\kern0pt}nrows\ index{\isacharunderscore}{\kern0pt}one{\isacharunderscore}{\kern0pt}mat\isanewline
\ \ \ \ \ \ \isacommand{by}\isamarkupfalse%
\ {\isacharparenleft}{\kern0pt}simp\ add{\isacharcolon}{\kern0pt}\ calculation{\isacharparenright}{\kern0pt}\isanewline
\ \ \isacommand{next}\isamarkupfalse%
\isanewline
\ \ \ \ \isacommand{have}\isamarkupfalse%
\ {\isachardoublequoteopen}dim{\isacharunderscore}{\kern0pt}col\ {\isacharparenleft}{\kern0pt}SWAP{\isacharunderscore}{\kern0pt}down\ {\isacharparenleft}{\kern0pt}Suc\ {\isacharparenleft}{\kern0pt}Suc\ k{\isacharparenright}{\kern0pt}{\isacharparenright}{\kern0pt}{\isacharparenright}{\kern0pt}\ {\isacharequal}{\kern0pt}\isanewline
\ \ \ \ \ \ \ \ \ \ dim{\isacharunderscore}{\kern0pt}col\ {\isacharparenleft}{\kern0pt}{\isacharparenleft}{\kern0pt}{\isacharparenleft}{\kern0pt}{\isadigit{1}}\isactrlsub m\ {\isacharparenleft}{\kern0pt}{\isadigit{2}}{\isacharcircum}{\kern0pt}k{\isacharparenright}{\kern0pt}{\isacharparenright}{\kern0pt}\ {\isasymOtimes}\ SWAP{\isacharparenright}{\kern0pt}\ {\isacharasterisk}{\kern0pt}\ {\isacharparenleft}{\kern0pt}{\isacharparenleft}{\kern0pt}SWAP{\isacharunderscore}{\kern0pt}down\ {\isacharparenleft}{\kern0pt}Suc\ k{\isacharparenright}{\kern0pt}{\isacharparenright}{\kern0pt}\ {\isasymOtimes}\ {\isacharparenleft}{\kern0pt}{\isadigit{1}}\isactrlsub m\ {\isadigit{2}}{\isacharparenright}{\kern0pt}{\isacharparenright}{\kern0pt}{\isacharparenright}{\kern0pt}{\isachardoublequoteclose}\isanewline
\ \ \ \ \ \ \isacommand{using}\isamarkupfalse%
\ SWAP{\isacharunderscore}{\kern0pt}down{\isachardot}{\kern0pt}simps{\isacharparenleft}{\kern0pt}{\isadigit{4}}{\isacharparenright}{\kern0pt}\ k{\isacharunderscore}{\kern0pt}def\ \isacommand{by}\isamarkupfalse%
\ simp\isanewline
\ \ \ \ \isacommand{also}\isamarkupfalse%
\ \isacommand{have}\isamarkupfalse%
\ {\isachardoublequoteopen}{\isasymdots}\ {\isacharequal}{\kern0pt}\ dim{\isacharunderscore}{\kern0pt}col\ {\isacharparenleft}{\kern0pt}{\isacharparenleft}{\kern0pt}SWAP{\isacharunderscore}{\kern0pt}down\ {\isacharparenleft}{\kern0pt}Suc\ k{\isacharparenright}{\kern0pt}{\isacharparenright}{\kern0pt}\ {\isasymOtimes}\ {\isacharparenleft}{\kern0pt}{\isadigit{1}}\isactrlsub m\ {\isadigit{2}}{\isacharparenright}{\kern0pt}{\isacharparenright}{\kern0pt}{\isachardoublequoteclose}\ \isacommand{by}\isamarkupfalse%
\ simp\isanewline
\ \ \ \ \isacommand{also}\isamarkupfalse%
\ \isacommand{have}\isamarkupfalse%
\ {\isachardoublequoteopen}{\isasymdots}\ {\isacharequal}{\kern0pt}\ dim{\isacharunderscore}{\kern0pt}col\ {\isacharparenleft}{\kern0pt}SWAP{\isacharunderscore}{\kern0pt}down\ {\isacharparenleft}{\kern0pt}Suc\ k{\isacharparenright}{\kern0pt}{\isacharparenright}{\kern0pt}\ {\isacharasterisk}{\kern0pt}\ dim{\isacharunderscore}{\kern0pt}col\ {\isacharparenleft}{\kern0pt}{\isadigit{1}}\isactrlsub m\ {\isadigit{2}}{\isacharparenright}{\kern0pt}{\isachardoublequoteclose}\ \isacommand{by}\isamarkupfalse%
\ simp\isanewline
\ \ \ \ \isacommand{thus}\isamarkupfalse%
\ {\isachardoublequoteopen}dim{\isacharunderscore}{\kern0pt}col\ {\isacharparenleft}{\kern0pt}SWAP{\isacharunderscore}{\kern0pt}down\ {\isacharparenleft}{\kern0pt}Suc\ {\isacharparenleft}{\kern0pt}Suc\ k{\isacharparenright}{\kern0pt}{\isacharparenright}{\kern0pt}{\isacharparenright}{\kern0pt}\ {\isacharequal}{\kern0pt}\ {\isadigit{2}}\ {\isacharcircum}{\kern0pt}\ Suc\ {\isacharparenleft}{\kern0pt}Suc\ k{\isacharparenright}{\kern0pt}{\isachardoublequoteclose}\isanewline
\ \ \ \ \ \ \isacommand{using}\isamarkupfalse%
\ SWAP{\isacharunderscore}{\kern0pt}ncols\ index{\isacharunderscore}{\kern0pt}one{\isacharunderscore}{\kern0pt}mat\ calculation\ HI\ \isacommand{by}\isamarkupfalse%
\ simp\isanewline
\ \ \isacommand{qed}\isamarkupfalse%
\isanewline
\isacommand{qed}\isamarkupfalse%
%
\endisatagproof
{\isafoldproof}%
%
\isadelimproof
%
\endisadelimproof
%
\isadelimdocument
%
\endisadelimdocument
%
\isatagdocument
%
\isamarkupsubsection{Upwards SWAP cascade%
}
\isamarkuptrue%
%
\endisatagdocument
{\isafolddocument}%
%
\isadelimdocument
%
\endisadelimdocument
\isacommand{fun}\isamarkupfalse%
\ SWAP{\isacharunderscore}{\kern0pt}up{\isacharcolon}{\kern0pt}{\isacharcolon}{\kern0pt}\ {\isachardoublequoteopen}nat\ {\isasymRightarrow}\ complex\ Matrix{\isachardot}{\kern0pt}mat{\isachardoublequoteclose}\ \isakeyword{where}\isanewline
\ \ {\isachardoublequoteopen}SWAP{\isacharunderscore}{\kern0pt}up\ {\isadigit{0}}\ {\isacharequal}{\kern0pt}\ {\isadigit{1}}\isactrlsub m\ {\isadigit{1}}{\isachardoublequoteclose}\isanewline
{\isacharbar}{\kern0pt}\ {\isachardoublequoteopen}SWAP{\isacharunderscore}{\kern0pt}up\ {\isacharparenleft}{\kern0pt}Suc\ {\isadigit{0}}{\isacharparenright}{\kern0pt}\ {\isacharequal}{\kern0pt}\ {\isadigit{1}}\isactrlsub m\ {\isadigit{2}}{\isachardoublequoteclose}\isanewline
{\isacharbar}{\kern0pt}\ {\isachardoublequoteopen}SWAP{\isacharunderscore}{\kern0pt}up\ {\isacharparenleft}{\kern0pt}Suc\ {\isacharparenleft}{\kern0pt}Suc\ {\isadigit{0}}{\isacharparenright}{\kern0pt}{\isacharparenright}{\kern0pt}\ {\isacharequal}{\kern0pt}\ SWAP{\isachardoublequoteclose}\isanewline
{\isacharbar}{\kern0pt}\ {\isachardoublequoteopen}SWAP{\isacharunderscore}{\kern0pt}up\ {\isacharparenleft}{\kern0pt}Suc\ {\isacharparenleft}{\kern0pt}Suc\ n{\isacharparenright}{\kern0pt}{\isacharparenright}{\kern0pt}\ {\isacharequal}{\kern0pt}\ {\isacharparenleft}{\kern0pt}SWAP\ {\isasymOtimes}\ {\isacharparenleft}{\kern0pt}{\isadigit{1}}\isactrlsub m\ {\isacharparenleft}{\kern0pt}{\isadigit{2}}{\isacharcircum}{\kern0pt}n{\isacharparenright}{\kern0pt}{\isacharparenright}{\kern0pt}{\isacharparenright}{\kern0pt}\ {\isacharasterisk}{\kern0pt}\ {\isacharparenleft}{\kern0pt}{\isacharparenleft}{\kern0pt}{\isadigit{1}}\isactrlsub m\ {\isadigit{2}}{\isacharparenright}{\kern0pt}\ {\isasymOtimes}\ {\isacharparenleft}{\kern0pt}SWAP{\isacharunderscore}{\kern0pt}up\ {\isacharparenleft}{\kern0pt}Suc\ n{\isacharparenright}{\kern0pt}{\isacharparenright}{\kern0pt}{\isacharparenright}{\kern0pt}{\isachardoublequoteclose}\isanewline
\isanewline
\isacommand{lemma}\isamarkupfalse%
\ SWAP{\isacharunderscore}{\kern0pt}up{\isacharunderscore}{\kern0pt}carrier{\isacharunderscore}{\kern0pt}mat{\isacharbrackleft}{\kern0pt}simp{\isacharbrackright}{\kern0pt}{\isacharcolon}{\kern0pt}\isanewline
\ \ \isakeyword{shows}\ {\isachardoublequoteopen}SWAP{\isacharunderscore}{\kern0pt}up\ n\ {\isasymin}\ carrier{\isacharunderscore}{\kern0pt}mat\ {\isacharparenleft}{\kern0pt}{\isadigit{2}}{\isacharcircum}{\kern0pt}n{\isacharparenright}{\kern0pt}\ {\isacharparenleft}{\kern0pt}{\isadigit{2}}{\isacharcircum}{\kern0pt}n{\isacharparenright}{\kern0pt}{\isachardoublequoteclose}\ {\isacharparenleft}{\kern0pt}\isakeyword{is}\ {\isachardoublequoteopen}{\isacharquery}{\kern0pt}P\ n{\isachardoublequoteclose}{\isacharparenright}{\kern0pt}\isanewline
%
\isadelimproof
%
\endisadelimproof
%
\isatagproof
\isacommand{proof}\isamarkupfalse%
\ {\isacharparenleft}{\kern0pt}induct\ n\ rule{\isacharcolon}{\kern0pt}\ SWAP{\isacharunderscore}{\kern0pt}up{\isachardot}{\kern0pt}induct{\isacharparenright}{\kern0pt}\isanewline
\ \ \isacommand{case}\isamarkupfalse%
\ {\isadigit{1}}\isanewline
\ \ \isacommand{then}\isamarkupfalse%
\ \isacommand{show}\isamarkupfalse%
\ {\isacharquery}{\kern0pt}case\ \isacommand{by}\isamarkupfalse%
\ auto\isanewline
\isacommand{next}\isamarkupfalse%
\isanewline
\ \ \isacommand{case}\isamarkupfalse%
\ {\isadigit{2}}\isanewline
\ \ \isacommand{then}\isamarkupfalse%
\ \isacommand{show}\isamarkupfalse%
\ {\isacharquery}{\kern0pt}case\ \isacommand{by}\isamarkupfalse%
\ auto\isanewline
\isacommand{next}\isamarkupfalse%
\isanewline
\ \ \isacommand{case}\isamarkupfalse%
\ {\isadigit{3}}\isanewline
\ \ \isacommand{then}\isamarkupfalse%
\ \isacommand{show}\isamarkupfalse%
\ {\isacharquery}{\kern0pt}case\ \isacommand{by}\isamarkupfalse%
\ auto\isanewline
\isacommand{next}\isamarkupfalse%
\isanewline
\ \ \isacommand{case}\isamarkupfalse%
\ {\isacharparenleft}{\kern0pt}{\isadigit{4}}\ v{\isacharparenright}{\kern0pt}\isanewline
\ \ \isacommand{then}\isamarkupfalse%
\ \isacommand{show}\isamarkupfalse%
\ {\isacharquery}{\kern0pt}case\ \isacommand{using}\isamarkupfalse%
\ SWAP{\isacharunderscore}{\kern0pt}nrows\ \isacommand{by}\isamarkupfalse%
\ fastforce\isanewline
\isacommand{qed}\isamarkupfalse%
%
\endisatagproof
{\isafoldproof}%
%
\isadelimproof
%
\endisadelimproof
%
\isadelimdocument
%
\endisadelimdocument
%
\isatagdocument
%
\isamarkupsection{Reversing qubits%
}
\isamarkuptrue%
%
\endisatagdocument
{\isafolddocument}%
%
\isadelimdocument
%
\endisadelimdocument
%
\begin{isamarkuptext}%
In order to reverse the order of n qubits, we iteratively swap opposite qubits (swap 0th
and (n-1)th qubits, 1st and (n-2)th qubits, and so on).%
\end{isamarkuptext}\isamarkuptrue%
\isacommand{fun}\isamarkupfalse%
\ reverse{\isacharunderscore}{\kern0pt}qubits{\isacharcolon}{\kern0pt}{\isacharcolon}{\kern0pt}\ {\isachardoublequoteopen}nat\ {\isasymRightarrow}\ complex\ Matrix{\isachardot}{\kern0pt}mat{\isachardoublequoteclose}\ \isakeyword{where}\isanewline
\ \ {\isachardoublequoteopen}reverse{\isacharunderscore}{\kern0pt}qubits\ {\isadigit{0}}\ {\isacharequal}{\kern0pt}\ {\isadigit{1}}\isactrlsub m\ {\isadigit{1}}{\isachardoublequoteclose}\isanewline
{\isacharbar}{\kern0pt}\ {\isachardoublequoteopen}reverse{\isacharunderscore}{\kern0pt}qubits\ {\isacharparenleft}{\kern0pt}Suc\ {\isadigit{0}}{\isacharparenright}{\kern0pt}\ {\isacharequal}{\kern0pt}\ {\isacharparenleft}{\kern0pt}{\isadigit{1}}\isactrlsub m\ {\isadigit{2}}{\isacharparenright}{\kern0pt}{\isachardoublequoteclose}\isanewline
{\isacharbar}{\kern0pt}\ {\isachardoublequoteopen}reverse{\isacharunderscore}{\kern0pt}qubits\ {\isacharparenleft}{\kern0pt}Suc\ {\isacharparenleft}{\kern0pt}Suc\ {\isadigit{0}}{\isacharparenright}{\kern0pt}{\isacharparenright}{\kern0pt}\ {\isacharequal}{\kern0pt}\ SWAP{\isachardoublequoteclose}\isanewline
{\isacharbar}{\kern0pt}\ {\isachardoublequoteopen}reverse{\isacharunderscore}{\kern0pt}qubits\ {\isacharparenleft}{\kern0pt}Suc\ n{\isacharparenright}{\kern0pt}\ {\isacharequal}{\kern0pt}\ {\isacharparenleft}{\kern0pt}{\isacharparenleft}{\kern0pt}reverse{\isacharunderscore}{\kern0pt}qubits\ n{\isacharparenright}{\kern0pt}\ {\isasymOtimes}\ {\isacharparenleft}{\kern0pt}{\isadigit{1}}\isactrlsub m\ {\isadigit{2}}{\isacharparenright}{\kern0pt}{\isacharparenright}{\kern0pt}\ {\isacharasterisk}{\kern0pt}\ {\isacharparenleft}{\kern0pt}SWAP{\isacharunderscore}{\kern0pt}down\ {\isacharparenleft}{\kern0pt}Suc\ n{\isacharparenright}{\kern0pt}{\isacharparenright}{\kern0pt}{\isachardoublequoteclose}\isanewline
\isanewline
\isanewline
\isacommand{lemma}\isamarkupfalse%
\ reverse{\isacharunderscore}{\kern0pt}qubits{\isacharunderscore}{\kern0pt}carrier{\isacharunderscore}{\kern0pt}mat{\isacharbrackleft}{\kern0pt}simp{\isacharbrackright}{\kern0pt}{\isacharcolon}{\kern0pt}\isanewline
\ \ {\isachardoublequoteopen}{\isacharparenleft}{\kern0pt}reverse{\isacharunderscore}{\kern0pt}qubits\ n{\isacharparenright}{\kern0pt}\ {\isasymin}\ carrier{\isacharunderscore}{\kern0pt}mat\ {\isacharparenleft}{\kern0pt}{\isadigit{2}}{\isacharcircum}{\kern0pt}n{\isacharparenright}{\kern0pt}\ {\isacharparenleft}{\kern0pt}{\isadigit{2}}{\isacharcircum}{\kern0pt}n{\isacharparenright}{\kern0pt}{\isachardoublequoteclose}\isanewline
%
\isadelimproof
%
\endisadelimproof
%
\isatagproof
\isacommand{proof}\isamarkupfalse%
\ {\isacharparenleft}{\kern0pt}induct\ n\ rule{\isacharcolon}{\kern0pt}\ reverse{\isacharunderscore}{\kern0pt}qubits{\isachardot}{\kern0pt}induct{\isacharparenright}{\kern0pt}\isanewline
\ \ \isacommand{case}\isamarkupfalse%
\ {\isadigit{1}}\isanewline
\ \ \isacommand{then}\isamarkupfalse%
\ \isacommand{show}\isamarkupfalse%
\ {\isacharquery}{\kern0pt}case\ \isacommand{by}\isamarkupfalse%
\ auto\isanewline
\isacommand{next}\isamarkupfalse%
\isanewline
\ \ \isacommand{case}\isamarkupfalse%
\ {\isadigit{2}}\isanewline
\ \ \isacommand{then}\isamarkupfalse%
\ \isacommand{show}\isamarkupfalse%
\ {\isacharquery}{\kern0pt}case\ \isacommand{by}\isamarkupfalse%
\ auto\isanewline
\isacommand{next}\isamarkupfalse%
\isanewline
\ \ \isacommand{case}\isamarkupfalse%
\ {\isadigit{3}}\isanewline
\ \ \isacommand{then}\isamarkupfalse%
\ \isacommand{show}\isamarkupfalse%
\ {\isacharquery}{\kern0pt}case\ \isacommand{by}\isamarkupfalse%
\ auto\isanewline
\isacommand{next}\isamarkupfalse%
\isanewline
\ \ \isacommand{case}\isamarkupfalse%
\ {\isacharparenleft}{\kern0pt}{\isadigit{4}}\ va{\isacharparenright}{\kern0pt}\isanewline
\ \ \isacommand{then}\isamarkupfalse%
\ \isacommand{show}\isamarkupfalse%
\ {\isacharquery}{\kern0pt}case\isanewline
\ \ \isacommand{by}\isamarkupfalse%
\ {\isacharparenleft}{\kern0pt}metis\ SWAP{\isacharunderscore}{\kern0pt}down{\isacharunderscore}{\kern0pt}carrier{\isacharunderscore}{\kern0pt}mat\ carrier{\isacharunderscore}{\kern0pt}matD{\isacharparenleft}{\kern0pt}{\isadigit{1}}{\isacharparenright}{\kern0pt}\ carrier{\isacharunderscore}{\kern0pt}matD{\isacharparenleft}{\kern0pt}{\isadigit{2}}{\isacharparenright}{\kern0pt}\ carrier{\isacharunderscore}{\kern0pt}matI\ dim{\isacharunderscore}{\kern0pt}row{\isacharunderscore}{\kern0pt}tensor{\isacharunderscore}{\kern0pt}mat\isanewline
\ \ \ \ \ \ index{\isacharunderscore}{\kern0pt}mult{\isacharunderscore}{\kern0pt}mat{\isacharparenleft}{\kern0pt}{\isadigit{2}}{\isacharparenright}{\kern0pt}\ index{\isacharunderscore}{\kern0pt}mult{\isacharunderscore}{\kern0pt}mat{\isacharparenleft}{\kern0pt}{\isadigit{3}}{\isacharparenright}{\kern0pt}\ index{\isacharunderscore}{\kern0pt}one{\isacharunderscore}{\kern0pt}mat{\isacharparenleft}{\kern0pt}{\isadigit{2}}{\isacharparenright}{\kern0pt}\ power{\isacharunderscore}{\kern0pt}Suc{\isadigit{2}}\ reverse{\isacharunderscore}{\kern0pt}qubits{\isachardot}{\kern0pt}simps{\isacharparenleft}{\kern0pt}{\isadigit{4}}{\isacharparenright}{\kern0pt}{\isacharparenright}{\kern0pt}\isanewline
\isacommand{qed}\isamarkupfalse%
%
\endisatagproof
{\isafoldproof}%
%
\isadelimproof
%
\endisadelimproof
%
\isadelimdocument
%
\endisadelimdocument
%
\isatagdocument
%
\isamarkupsection{Controlled operations%
}
\isamarkuptrue%
%
\endisatagdocument
{\isafolddocument}%
%
\isadelimdocument
%
\endisadelimdocument
%
\begin{isamarkuptext}%
The two-qubit gate control2 performs a controlled U operation on the first qubit with the 
second qubit as control%
\end{isamarkuptext}\isamarkuptrue%
\isacommand{definition}\isamarkupfalse%
\ control{\isadigit{2}}{\isacharcolon}{\kern0pt}{\isacharcolon}{\kern0pt}\ {\isachardoublequoteopen}complex\ Matrix{\isachardot}{\kern0pt}mat\ {\isasymRightarrow}\ complex\ Matrix{\isachardot}{\kern0pt}mat{\isachardoublequoteclose}\ \isakeyword{where}\isanewline
\ \ {\isachardoublequoteopen}control{\isadigit{2}}\ U\ {\isasymequiv}\ mat{\isacharunderscore}{\kern0pt}of{\isacharunderscore}{\kern0pt}cols{\isacharunderscore}{\kern0pt}list\ {\isadigit{4}}\ {\isacharbrackleft}{\kern0pt}{\isacharbrackleft}{\kern0pt}{\isadigit{1}}{\isacharcomma}{\kern0pt}\ {\isadigit{0}}{\isacharcomma}{\kern0pt}\ {\isadigit{0}}{\isacharcomma}{\kern0pt}\ {\isadigit{0}}{\isacharbrackright}{\kern0pt}{\isacharcomma}{\kern0pt}\isanewline
\ \ \ \ \ \ \ \ \ \ \ \ \ \ \ \ \ \ \ \ \ \ \ \ \ \ \ \ \ \ \ \ \ \ \ \ {\isacharbrackleft}{\kern0pt}{\isadigit{0}}{\isacharcomma}{\kern0pt}\ U{\isachardollar}{\kern0pt}{\isachardollar}{\kern0pt}{\isacharparenleft}{\kern0pt}{\isadigit{0}}{\isacharcomma}{\kern0pt}{\isadigit{0}}{\isacharparenright}{\kern0pt}{\isacharcomma}{\kern0pt}\ {\isadigit{0}}{\isacharcomma}{\kern0pt}\ U{\isachardollar}{\kern0pt}{\isachardollar}{\kern0pt}{\isacharparenleft}{\kern0pt}{\isadigit{1}}{\isacharcomma}{\kern0pt}{\isadigit{0}}{\isacharparenright}{\kern0pt}{\isacharbrackright}{\kern0pt}{\isacharcomma}{\kern0pt}\isanewline
\ \ \ \ \ \ \ \ \ \ \ \ \ \ \ \ \ \ \ \ \ \ \ \ \ \ \ \ \ \ \ \ \ \ \ \ {\isacharbrackleft}{\kern0pt}{\isadigit{0}}{\isacharcomma}{\kern0pt}\ {\isadigit{0}}{\isacharcomma}{\kern0pt}\ {\isadigit{1}}{\isacharcomma}{\kern0pt}\ {\isadigit{0}}{\isacharbrackright}{\kern0pt}{\isacharcomma}{\kern0pt}\isanewline
\ \ \ \ \ \ \ \ \ \ \ \ \ \ \ \ \ \ \ \ \ \ \ \ \ \ \ \ \ \ \ \ \ \ \ \ {\isacharbrackleft}{\kern0pt}{\isadigit{0}}{\isacharcomma}{\kern0pt}\ U{\isachardollar}{\kern0pt}{\isachardollar}{\kern0pt}{\isacharparenleft}{\kern0pt}{\isadigit{0}}{\isacharcomma}{\kern0pt}{\isadigit{1}}{\isacharparenright}{\kern0pt}{\isacharcomma}{\kern0pt}\ {\isadigit{0}}{\isacharcomma}{\kern0pt}\ U{\isachardollar}{\kern0pt}{\isachardollar}{\kern0pt}{\isacharparenleft}{\kern0pt}{\isadigit{1}}{\isacharcomma}{\kern0pt}{\isadigit{1}}{\isacharparenright}{\kern0pt}{\isacharbrackright}{\kern0pt}{\isacharbrackright}{\kern0pt}{\isachardoublequoteclose}\isanewline
\isanewline
\isacommand{lemma}\isamarkupfalse%
\ control{\isadigit{2}}{\isacharunderscore}{\kern0pt}carrier{\isacharunderscore}{\kern0pt}mat{\isacharbrackleft}{\kern0pt}simp{\isacharbrackright}{\kern0pt}{\isacharcolon}{\kern0pt}\isanewline
\ \ \isakeyword{shows}\ {\isachardoublequoteopen}control{\isadigit{2}}\ U\ {\isasymin}\ carrier{\isacharunderscore}{\kern0pt}mat\ {\isadigit{4}}\ {\isadigit{4}}{\isachardoublequoteclose}\isanewline
%
\isadelimproof
\ \ %
\endisadelimproof
%
\isatagproof
\isacommand{by}\isamarkupfalse%
\ {\isacharparenleft}{\kern0pt}simp\ add{\isacharcolon}{\kern0pt}\ Tensor{\isachardot}{\kern0pt}mat{\isacharunderscore}{\kern0pt}of{\isacharunderscore}{\kern0pt}cols{\isacharunderscore}{\kern0pt}list{\isacharunderscore}{\kern0pt}def\ control{\isadigit{2}}{\isacharunderscore}{\kern0pt}def\ numeral{\isacharunderscore}{\kern0pt}Bit{\isadigit{0}}{\isacharparenright}{\kern0pt}%
\endisatagproof
{\isafoldproof}%
%
\isadelimproof
\isanewline
%
\endisadelimproof
\isanewline
\isanewline
\isacommand{lemma}\isamarkupfalse%
\ control{\isadigit{2}}{\isacharunderscore}{\kern0pt}zero{\isacharcolon}{\kern0pt}\isanewline
\ \ \isakeyword{assumes}\ {\isachardoublequoteopen}dim{\isacharunderscore}{\kern0pt}row\ v\ {\isacharequal}{\kern0pt}\ {\isadigit{2}}{\isachardoublequoteclose}\ \isakeyword{and}\ {\isachardoublequoteopen}dim{\isacharunderscore}{\kern0pt}col\ v\ {\isacharequal}{\kern0pt}\ {\isadigit{1}}{\isachardoublequoteclose}\isanewline
\ \ \isakeyword{shows}\ {\isachardoublequoteopen}control{\isadigit{2}}\ U\ {\isacharasterisk}{\kern0pt}\ {\isacharparenleft}{\kern0pt}v\ {\isasymOtimes}\ {\isacharbar}{\kern0pt}zero{\isasymrangle}{\isacharparenright}{\kern0pt}\ {\isacharequal}{\kern0pt}\ v\ {\isasymOtimes}\ {\isacharbar}{\kern0pt}zero{\isasymrangle}{\isachardoublequoteclose}\isanewline
%
\isadelimproof
%
\endisadelimproof
%
\isatagproof
\isacommand{proof}\isamarkupfalse%
\ \isanewline
\ \ \isacommand{fix}\isamarkupfalse%
\ i\ j{\isacharcolon}{\kern0pt}{\isacharcolon}{\kern0pt}nat\isanewline
\ \ \isacommand{assume}\isamarkupfalse%
\ {\isachardoublequoteopen}i\ {\isacharless}{\kern0pt}\ dim{\isacharunderscore}{\kern0pt}row\ {\isacharparenleft}{\kern0pt}v\ {\isasymOtimes}\ {\isacharbar}{\kern0pt}zero{\isasymrangle}{\isacharparenright}{\kern0pt}{\isachardoublequoteclose}\isanewline
\ \ \isacommand{hence}\isamarkupfalse%
\ i{\isadigit{4}}{\isacharcolon}{\kern0pt}{\isachardoublequoteopen}i\ {\isacharless}{\kern0pt}\ {\isadigit{4}}{\isachardoublequoteclose}\ \isacommand{using}\isamarkupfalse%
\ assms\ tensor{\isacharunderscore}{\kern0pt}carrier{\isacharunderscore}{\kern0pt}mat\ ket{\isacharunderscore}{\kern0pt}vec{\isacharunderscore}{\kern0pt}def\ \isacommand{by}\isamarkupfalse%
\ auto\isanewline
\ \ \isacommand{assume}\isamarkupfalse%
\ {\isachardoublequoteopen}j\ {\isacharless}{\kern0pt}\ dim{\isacharunderscore}{\kern0pt}col\ {\isacharparenleft}{\kern0pt}v\ {\isasymOtimes}\ {\isacharbar}{\kern0pt}zero{\isasymrangle}{\isacharparenright}{\kern0pt}{\isachardoublequoteclose}\isanewline
\ \ \isacommand{hence}\isamarkupfalse%
\ j{\isadigit{0}}{\isacharcolon}{\kern0pt}{\isachardoublequoteopen}j\ {\isacharequal}{\kern0pt}\ {\isadigit{0}}{\isachardoublequoteclose}\ \isacommand{using}\isamarkupfalse%
\ assms\ tensor{\isacharunderscore}{\kern0pt}carrier{\isacharunderscore}{\kern0pt}mat\ ket{\isacharunderscore}{\kern0pt}vec{\isacharunderscore}{\kern0pt}def\ \isacommand{by}\isamarkupfalse%
\ auto\isanewline
\ \ \isacommand{show}\isamarkupfalse%
\ {\isachardoublequoteopen}{\isacharparenleft}{\kern0pt}control{\isadigit{2}}\ U\ {\isacharasterisk}{\kern0pt}\ {\isacharparenleft}{\kern0pt}v\ {\isasymOtimes}\ {\isacharbar}{\kern0pt}zero{\isasymrangle}{\isacharparenright}{\kern0pt}{\isacharparenright}{\kern0pt}\ {\isachardollar}{\kern0pt}{\isachardollar}{\kern0pt}\ {\isacharparenleft}{\kern0pt}i{\isacharcomma}{\kern0pt}j{\isacharparenright}{\kern0pt}\ {\isacharequal}{\kern0pt}\ {\isacharparenleft}{\kern0pt}v\ {\isasymOtimes}\ {\isacharbar}{\kern0pt}zero{\isasymrangle}{\isacharparenright}{\kern0pt}\ {\isachardollar}{\kern0pt}{\isachardollar}{\kern0pt}\ {\isacharparenleft}{\kern0pt}i{\isacharcomma}{\kern0pt}j{\isacharparenright}{\kern0pt}{\isachardoublequoteclose}\isanewline
\ \ \isacommand{proof}\isamarkupfalse%
\ {\isacharminus}{\kern0pt}\isanewline
\ \ \ \ \isacommand{have}\isamarkupfalse%
\ {\isachardoublequoteopen}{\isacharparenleft}{\kern0pt}control{\isadigit{2}}\ U\ {\isacharasterisk}{\kern0pt}\ {\isacharparenleft}{\kern0pt}v\ {\isasymOtimes}\ {\isacharbar}{\kern0pt}zero{\isasymrangle}{\isacharparenright}{\kern0pt}{\isacharparenright}{\kern0pt}\ {\isachardollar}{\kern0pt}{\isachardollar}{\kern0pt}\ {\isacharparenleft}{\kern0pt}i{\isacharcomma}{\kern0pt}j{\isacharparenright}{\kern0pt}\ {\isacharequal}{\kern0pt}\ \isanewline
\ \ \ \ \ \ \ \ \ \ {\isacharparenleft}{\kern0pt}{\isasymSum}k{\isacharless}{\kern0pt}dim{\isacharunderscore}{\kern0pt}row\ {\isacharparenleft}{\kern0pt}v\ {\isasymOtimes}\ {\isacharbar}{\kern0pt}zero{\isasymrangle}{\isacharparenright}{\kern0pt}{\isachardot}{\kern0pt}\ control{\isadigit{2}}\ U\ {\isachardollar}{\kern0pt}{\isachardollar}{\kern0pt}\ {\isacharparenleft}{\kern0pt}i{\isacharcomma}{\kern0pt}\ k{\isacharparenright}{\kern0pt}\ {\isacharasterisk}{\kern0pt}\ {\isacharparenleft}{\kern0pt}v\ {\isasymOtimes}\ {\isacharbar}{\kern0pt}zero{\isasymrangle}{\isacharparenright}{\kern0pt}\ {\isachardollar}{\kern0pt}{\isachardollar}{\kern0pt}\ {\isacharparenleft}{\kern0pt}k{\isacharcomma}{\kern0pt}\ j{\isacharparenright}{\kern0pt}{\isacharparenright}{\kern0pt}{\isachardoublequoteclose}\isanewline
\ \ \ \ \ \ \isacommand{using}\isamarkupfalse%
\ assms\ index{\isacharunderscore}{\kern0pt}matrix{\isacharunderscore}{\kern0pt}prod\ \isanewline
\ \ \ \ \ \ \isacommand{by}\isamarkupfalse%
\ {\isacharparenleft}{\kern0pt}smt\ {\isacharparenleft}{\kern0pt}z{\isadigit{3}}{\isacharparenright}{\kern0pt}\ One{\isacharunderscore}{\kern0pt}nat{\isacharunderscore}{\kern0pt}def\ Suc{\isacharunderscore}{\kern0pt}{\isadigit{1}}\ Tensor{\isachardot}{\kern0pt}mat{\isacharunderscore}{\kern0pt}of{\isacharunderscore}{\kern0pt}cols{\isacharunderscore}{\kern0pt}list{\isacharunderscore}{\kern0pt}def\ {\isacartoucheopen}i\ {\isacharless}{\kern0pt}\ dim{\isacharunderscore}{\kern0pt}row\ {\isacharparenleft}{\kern0pt}v\ {\isasymOtimes}\ {\isacharbar}{\kern0pt}Deutsch{\isachardot}{\kern0pt}zero{\isasymrangle}{\isacharparenright}{\kern0pt}{\isacartoucheclose}\isanewline
\ \ \ \ \ \ \ \ \ \ {\isacartoucheopen}j\ {\isacharless}{\kern0pt}\ dim{\isacharunderscore}{\kern0pt}col\ {\isacharparenleft}{\kern0pt}v\ {\isasymOtimes}\ {\isacharbar}{\kern0pt}Deutsch{\isachardot}{\kern0pt}zero{\isasymrangle}{\isacharparenright}{\kern0pt}{\isacartoucheclose}\ add{\isachardot}{\kern0pt}commute\ add{\isacharunderscore}{\kern0pt}Suc{\isacharunderscore}{\kern0pt}right\ control{\isadigit{2}}{\isacharunderscore}{\kern0pt}def\ dim{\isacharunderscore}{\kern0pt}col{\isacharunderscore}{\kern0pt}mat{\isacharparenleft}{\kern0pt}{\isadigit{1}}{\isacharparenright}{\kern0pt}\isanewline
\ \ \ \ \ \ \ \ \ \ dim{\isacharunderscore}{\kern0pt}row{\isacharunderscore}{\kern0pt}mat{\isacharparenleft}{\kern0pt}{\isadigit{1}}{\isacharparenright}{\kern0pt}\ dim{\isacharunderscore}{\kern0pt}row{\isacharunderscore}{\kern0pt}tensor{\isacharunderscore}{\kern0pt}mat\ ket{\isacharunderscore}{\kern0pt}zero{\isacharunderscore}{\kern0pt}to{\isacharunderscore}{\kern0pt}mat{\isacharunderscore}{\kern0pt}of{\isacharunderscore}{\kern0pt}cols{\isacharunderscore}{\kern0pt}list\ list{\isachardot}{\kern0pt}size{\isacharparenleft}{\kern0pt}{\isadigit{3}}{\isacharparenright}{\kern0pt}\ list{\isachardot}{\kern0pt}size{\isacharparenleft}{\kern0pt}{\isadigit{4}}{\isacharparenright}{\kern0pt}\ \isanewline
\ \ \ \ \ \ \ \ \ \ mult{\isacharunderscore}{\kern0pt}{\isadigit{2}}\ numeral{\isacharunderscore}{\kern0pt}Bit{\isadigit{0}}\ plus{\isacharunderscore}{\kern0pt}{\isadigit{1}}{\isacharunderscore}{\kern0pt}eq{\isacharunderscore}{\kern0pt}Suc\ sum{\isachardot}{\kern0pt}cong{\isacharparenright}{\kern0pt}\isanewline
\ \ \ \ \isacommand{also}\isamarkupfalse%
\ \isacommand{have}\isamarkupfalse%
\ {\isachardoublequoteopen}{\isasymdots}\ {\isacharequal}{\kern0pt}\ {\isacharparenleft}{\kern0pt}{\isasymSum}k{\isacharless}{\kern0pt}{\isadigit{4}}{\isachardot}{\kern0pt}\ control{\isadigit{2}}\ U\ {\isachardollar}{\kern0pt}{\isachardollar}{\kern0pt}\ {\isacharparenleft}{\kern0pt}i{\isacharcomma}{\kern0pt}\ k{\isacharparenright}{\kern0pt}\ {\isacharasterisk}{\kern0pt}\ {\isacharparenleft}{\kern0pt}v\ {\isasymOtimes}\ {\isacharbar}{\kern0pt}zero{\isasymrangle}{\isacharparenright}{\kern0pt}\ {\isachardollar}{\kern0pt}{\isachardollar}{\kern0pt}\ {\isacharparenleft}{\kern0pt}k{\isacharcomma}{\kern0pt}\ j{\isacharparenright}{\kern0pt}{\isacharparenright}{\kern0pt}{\isachardoublequoteclose}\isanewline
\ \ \ \ \ \ \isacommand{using}\isamarkupfalse%
\ assms\ tensor{\isacharunderscore}{\kern0pt}carrier{\isacharunderscore}{\kern0pt}mat\ ket{\isacharunderscore}{\kern0pt}vec{\isacharunderscore}{\kern0pt}def\ \isacommand{by}\isamarkupfalse%
\ auto\isanewline
\ \ \ \ \isacommand{also}\isamarkupfalse%
\ \isacommand{have}\isamarkupfalse%
\ {\isachardoublequoteopen}{\isasymdots}\ {\isacharequal}{\kern0pt}\ control{\isadigit{2}}\ U\ {\isachardollar}{\kern0pt}{\isachardollar}{\kern0pt}\ {\isacharparenleft}{\kern0pt}i{\isacharcomma}{\kern0pt}\ {\isadigit{0}}{\isacharparenright}{\kern0pt}\ {\isacharasterisk}{\kern0pt}\ {\isacharparenleft}{\kern0pt}v\ {\isasymOtimes}\ {\isacharbar}{\kern0pt}zero{\isasymrangle}{\isacharparenright}{\kern0pt}\ {\isachardollar}{\kern0pt}{\isachardollar}{\kern0pt}\ {\isacharparenleft}{\kern0pt}{\isadigit{0}}{\isacharcomma}{\kern0pt}\ {\isadigit{0}}{\isacharparenright}{\kern0pt}\ {\isacharplus}{\kern0pt}\isanewline
\ \ \ \ \ \ \ \ \ \ \ \ \ \ \ \ \ \ \ \ control{\isadigit{2}}\ U\ {\isachardollar}{\kern0pt}{\isachardollar}{\kern0pt}\ {\isacharparenleft}{\kern0pt}i{\isacharcomma}{\kern0pt}\ {\isadigit{1}}{\isacharparenright}{\kern0pt}\ {\isacharasterisk}{\kern0pt}\ {\isacharparenleft}{\kern0pt}v\ {\isasymOtimes}\ {\isacharbar}{\kern0pt}zero{\isasymrangle}{\isacharparenright}{\kern0pt}\ {\isachardollar}{\kern0pt}{\isachardollar}{\kern0pt}\ {\isacharparenleft}{\kern0pt}{\isadigit{1}}{\isacharcomma}{\kern0pt}\ {\isadigit{0}}{\isacharparenright}{\kern0pt}\ {\isacharplus}{\kern0pt}\isanewline
\ \ \ \ \ \ \ \ \ \ \ \ \ \ \ \ \ \ \ \ control{\isadigit{2}}\ U\ {\isachardollar}{\kern0pt}{\isachardollar}{\kern0pt}\ {\isacharparenleft}{\kern0pt}i{\isacharcomma}{\kern0pt}\ {\isadigit{2}}{\isacharparenright}{\kern0pt}\ {\isacharasterisk}{\kern0pt}\ {\isacharparenleft}{\kern0pt}v\ {\isasymOtimes}\ {\isacharbar}{\kern0pt}zero{\isasymrangle}{\isacharparenright}{\kern0pt}\ {\isachardollar}{\kern0pt}{\isachardollar}{\kern0pt}\ {\isacharparenleft}{\kern0pt}{\isadigit{2}}{\isacharcomma}{\kern0pt}\ {\isadigit{0}}{\isacharparenright}{\kern0pt}\ {\isacharplus}{\kern0pt}\isanewline
\ \ \ \ \ \ \ \ \ \ \ \ \ \ \ \ \ \ \ \ control{\isadigit{2}}\ U\ {\isachardollar}{\kern0pt}{\isachardollar}{\kern0pt}\ {\isacharparenleft}{\kern0pt}i{\isacharcomma}{\kern0pt}\ {\isadigit{3}}{\isacharparenright}{\kern0pt}\ {\isacharasterisk}{\kern0pt}\ {\isacharparenleft}{\kern0pt}v\ {\isasymOtimes}\ {\isacharbar}{\kern0pt}zero{\isasymrangle}{\isacharparenright}{\kern0pt}\ {\isachardollar}{\kern0pt}{\isachardollar}{\kern0pt}\ {\isacharparenleft}{\kern0pt}{\isadigit{3}}{\isacharcomma}{\kern0pt}\ {\isadigit{0}}{\isacharparenright}{\kern0pt}{\isachardoublequoteclose}\isanewline
\ \ \ \ \ \ \isacommand{using}\isamarkupfalse%
\ sumof{\isadigit{4}}\ j{\isadigit{0}}\ \isacommand{by}\isamarkupfalse%
\ blast\isanewline
\ \ \ \ \isacommand{also}\isamarkupfalse%
\ \isacommand{have}\isamarkupfalse%
\ {\isachardoublequoteopen}{\isasymdots}\ {\isacharequal}{\kern0pt}\ {\isacharparenleft}{\kern0pt}v\ {\isasymOtimes}\ {\isacharbar}{\kern0pt}zero{\isasymrangle}{\isacharparenright}{\kern0pt}\ {\isachardollar}{\kern0pt}{\isachardollar}{\kern0pt}\ {\isacharparenleft}{\kern0pt}i{\isacharcomma}{\kern0pt}{\isadigit{0}}{\isacharparenright}{\kern0pt}{\isachardoublequoteclose}\isanewline
\ \ \ \ \isacommand{proof}\isamarkupfalse%
\ {\isacharparenleft}{\kern0pt}rule\ disjE{\isacharparenright}{\kern0pt}\isanewline
\ \ \ \ \ \ \isacommand{show}\isamarkupfalse%
\ {\isachardoublequoteopen}i\ {\isacharequal}{\kern0pt}\ {\isadigit{0}}\ {\isasymor}\ i\ {\isacharequal}{\kern0pt}\ {\isadigit{1}}\ {\isasymor}\ i\ {\isacharequal}{\kern0pt}\ {\isadigit{2}}\ {\isasymor}\ i\ {\isacharequal}{\kern0pt}\ {\isadigit{3}}{\isachardoublequoteclose}\ \isacommand{using}\isamarkupfalse%
\ i{\isadigit{4}}\ \isacommand{by}\isamarkupfalse%
\ auto\isanewline
\ \ \ \ \isacommand{next}\isamarkupfalse%
\isanewline
\ \ \ \ \ \ \isacommand{assume}\isamarkupfalse%
\ i{\isadigit{0}}{\isacharcolon}{\kern0pt}{\isachardoublequoteopen}i\ {\isacharequal}{\kern0pt}\ {\isadigit{0}}{\isachardoublequoteclose}\isanewline
\ \ \ \ \ \ \isacommand{have}\isamarkupfalse%
\ c{\isadigit{0}}{\isadigit{0}}{\isacharcolon}{\kern0pt}{\isachardoublequoteopen}control{\isadigit{2}}\ U\ {\isachardollar}{\kern0pt}{\isachardollar}{\kern0pt}\ {\isacharparenleft}{\kern0pt}{\isadigit{0}}{\isacharcomma}{\kern0pt}{\isadigit{0}}{\isacharparenright}{\kern0pt}\ {\isacharequal}{\kern0pt}\ {\isadigit{1}}{\isachardoublequoteclose}\isanewline
\ \ \ \ \ \ \ \ \isacommand{by}\isamarkupfalse%
\ {\isacharparenleft}{\kern0pt}simp\ add{\isacharcolon}{\kern0pt}\ control{\isadigit{2}}{\isacharunderscore}{\kern0pt}def\ one{\isacharunderscore}{\kern0pt}complex{\isachardot}{\kern0pt}code{\isacharparenright}{\kern0pt}\isanewline
\ \ \ \ \ \ \isacommand{have}\isamarkupfalse%
\ c{\isadigit{0}}{\isadigit{1}}{\isacharcolon}{\kern0pt}{\isachardoublequoteopen}control{\isadigit{2}}\ U\ {\isachardollar}{\kern0pt}{\isachardollar}{\kern0pt}\ {\isacharparenleft}{\kern0pt}{\isadigit{0}}{\isacharcomma}{\kern0pt}{\isadigit{1}}{\isacharparenright}{\kern0pt}\ {\isacharequal}{\kern0pt}\ {\isadigit{0}}{\isachardoublequoteclose}\isanewline
\ \ \ \ \ \ \ \ \isacommand{by}\isamarkupfalse%
\ {\isacharparenleft}{\kern0pt}simp\ add{\isacharcolon}{\kern0pt}\ control{\isadigit{2}}{\isacharunderscore}{\kern0pt}def\ zero{\isacharunderscore}{\kern0pt}complex{\isachardot}{\kern0pt}code{\isacharparenright}{\kern0pt}\isanewline
\ \ \ \ \ \ \isacommand{have}\isamarkupfalse%
\ c{\isadigit{0}}{\isadigit{2}}{\isacharcolon}{\kern0pt}{\isachardoublequoteopen}control{\isadigit{2}}\ U\ {\isachardollar}{\kern0pt}{\isachardollar}{\kern0pt}\ {\isacharparenleft}{\kern0pt}{\isadigit{0}}{\isacharcomma}{\kern0pt}{\isadigit{2}}{\isacharparenright}{\kern0pt}\ {\isacharequal}{\kern0pt}\ {\isadigit{0}}{\isachardoublequoteclose}\isanewline
\ \ \ \ \ \ \ \ \isacommand{by}\isamarkupfalse%
\ {\isacharparenleft}{\kern0pt}simp\ add{\isacharcolon}{\kern0pt}\ control{\isadigit{2}}{\isacharunderscore}{\kern0pt}def\ zero{\isacharunderscore}{\kern0pt}complex{\isachardot}{\kern0pt}code{\isacharparenright}{\kern0pt}\isanewline
\ \ \ \ \ \ \isacommand{have}\isamarkupfalse%
\ c{\isadigit{0}}{\isadigit{3}}{\isacharcolon}{\kern0pt}{\isachardoublequoteopen}control{\isadigit{2}}\ U\ {\isachardollar}{\kern0pt}{\isachardollar}{\kern0pt}\ {\isacharparenleft}{\kern0pt}{\isadigit{0}}{\isacharcomma}{\kern0pt}{\isadigit{3}}{\isacharparenright}{\kern0pt}\ {\isacharequal}{\kern0pt}\ {\isadigit{0}}{\isachardoublequoteclose}\isanewline
\ \ \ \ \ \ \ \ \isacommand{by}\isamarkupfalse%
\ {\isacharparenleft}{\kern0pt}simp\ add{\isacharcolon}{\kern0pt}\ control{\isadigit{2}}{\isacharunderscore}{\kern0pt}def\ zero{\isacharunderscore}{\kern0pt}complex{\isachardot}{\kern0pt}code{\isacharparenright}{\kern0pt}\isanewline
\ \ \ \ \ \ \isacommand{have}\isamarkupfalse%
\ {\isachardoublequoteopen}control{\isadigit{2}}\ U\ {\isachardollar}{\kern0pt}{\isachardollar}{\kern0pt}\ {\isacharparenleft}{\kern0pt}{\isadigit{0}}{\isacharcomma}{\kern0pt}\ {\isadigit{0}}{\isacharparenright}{\kern0pt}\ {\isacharasterisk}{\kern0pt}\ {\isacharparenleft}{\kern0pt}v\ {\isasymOtimes}\ {\isacharbar}{\kern0pt}zero{\isasymrangle}{\isacharparenright}{\kern0pt}\ {\isachardollar}{\kern0pt}{\isachardollar}{\kern0pt}\ {\isacharparenleft}{\kern0pt}{\isadigit{0}}{\isacharcomma}{\kern0pt}\ {\isadigit{0}}{\isacharparenright}{\kern0pt}\ {\isacharplus}{\kern0pt}\isanewline
\ \ \ \ \ \ \ \ \ \ \ \ \ control{\isadigit{2}}\ U\ {\isachardollar}{\kern0pt}{\isachardollar}{\kern0pt}\ {\isacharparenleft}{\kern0pt}{\isadigit{0}}{\isacharcomma}{\kern0pt}\ {\isadigit{1}}{\isacharparenright}{\kern0pt}\ {\isacharasterisk}{\kern0pt}\ {\isacharparenleft}{\kern0pt}v\ {\isasymOtimes}\ {\isacharbar}{\kern0pt}zero{\isasymrangle}{\isacharparenright}{\kern0pt}\ {\isachardollar}{\kern0pt}{\isachardollar}{\kern0pt}\ {\isacharparenleft}{\kern0pt}{\isadigit{1}}{\isacharcomma}{\kern0pt}\ {\isadigit{0}}{\isacharparenright}{\kern0pt}\ {\isacharplus}{\kern0pt}\isanewline
\ \ \ \ \ \ \ \ \ \ \ \ \ control{\isadigit{2}}\ U\ {\isachardollar}{\kern0pt}{\isachardollar}{\kern0pt}\ {\isacharparenleft}{\kern0pt}{\isadigit{0}}{\isacharcomma}{\kern0pt}\ {\isadigit{2}}{\isacharparenright}{\kern0pt}\ {\isacharasterisk}{\kern0pt}\ {\isacharparenleft}{\kern0pt}v\ {\isasymOtimes}\ {\isacharbar}{\kern0pt}zero{\isasymrangle}{\isacharparenright}{\kern0pt}\ {\isachardollar}{\kern0pt}{\isachardollar}{\kern0pt}\ {\isacharparenleft}{\kern0pt}{\isadigit{2}}{\isacharcomma}{\kern0pt}\ {\isadigit{0}}{\isacharparenright}{\kern0pt}\ {\isacharplus}{\kern0pt}\isanewline
\ \ \ \ \ \ \ \ \ \ \ \ \ control{\isadigit{2}}\ U\ {\isachardollar}{\kern0pt}{\isachardollar}{\kern0pt}\ {\isacharparenleft}{\kern0pt}{\isadigit{0}}{\isacharcomma}{\kern0pt}\ {\isadigit{3}}{\isacharparenright}{\kern0pt}\ {\isacharasterisk}{\kern0pt}\ {\isacharparenleft}{\kern0pt}v\ {\isasymOtimes}\ {\isacharbar}{\kern0pt}zero{\isasymrangle}{\isacharparenright}{\kern0pt}\ {\isachardollar}{\kern0pt}{\isachardollar}{\kern0pt}\ {\isacharparenleft}{\kern0pt}{\isadigit{3}}{\isacharcomma}{\kern0pt}\ {\isadigit{0}}{\isacharparenright}{\kern0pt}\ {\isacharequal}{\kern0pt}\isanewline
\ \ \ \ \ \ \ \ \ \ \ \ \ {\isadigit{1}}\ {\isacharasterisk}{\kern0pt}\ {\isacharparenleft}{\kern0pt}v\ {\isasymOtimes}\ {\isacharbar}{\kern0pt}zero{\isasymrangle}{\isacharparenright}{\kern0pt}\ {\isachardollar}{\kern0pt}{\isachardollar}{\kern0pt}\ {\isacharparenleft}{\kern0pt}{\isadigit{0}}{\isacharcomma}{\kern0pt}\ {\isadigit{0}}{\isacharparenright}{\kern0pt}\ {\isacharplus}{\kern0pt}\isanewline
\ \ \ \ \ \ \ \ \ \ \ \ \ {\isadigit{0}}\ {\isacharasterisk}{\kern0pt}\ {\isacharparenleft}{\kern0pt}v\ {\isasymOtimes}\ {\isacharbar}{\kern0pt}zero{\isasymrangle}{\isacharparenright}{\kern0pt}\ {\isachardollar}{\kern0pt}{\isachardollar}{\kern0pt}\ {\isacharparenleft}{\kern0pt}{\isadigit{1}}{\isacharcomma}{\kern0pt}\ {\isadigit{0}}{\isacharparenright}{\kern0pt}\ {\isacharplus}{\kern0pt}\isanewline
\ \ \ \ \ \ \ \ \ \ \ \ \ {\isadigit{0}}\ {\isacharasterisk}{\kern0pt}\ {\isacharparenleft}{\kern0pt}v\ {\isasymOtimes}\ {\isacharbar}{\kern0pt}zero{\isasymrangle}{\isacharparenright}{\kern0pt}\ {\isachardollar}{\kern0pt}{\isachardollar}{\kern0pt}\ {\isacharparenleft}{\kern0pt}{\isadigit{2}}{\isacharcomma}{\kern0pt}\ {\isadigit{0}}{\isacharparenright}{\kern0pt}\ {\isacharplus}{\kern0pt}\isanewline
\ \ \ \ \ \ \ \ \ \ \ \ \ {\isadigit{0}}\ {\isacharasterisk}{\kern0pt}\ {\isacharparenleft}{\kern0pt}v\ {\isasymOtimes}\ {\isacharbar}{\kern0pt}zero{\isasymrangle}{\isacharparenright}{\kern0pt}\ {\isachardollar}{\kern0pt}{\isachardollar}{\kern0pt}\ {\isacharparenleft}{\kern0pt}{\isadigit{3}}{\isacharcomma}{\kern0pt}\ {\isadigit{0}}{\isacharparenright}{\kern0pt}{\isachardoublequoteclose}\isanewline
\ \ \ \ \ \ \ \ \isacommand{using}\isamarkupfalse%
\ c{\isadigit{0}}{\isadigit{0}}\ c{\isadigit{0}}{\isadigit{1}}\ c{\isadigit{0}}{\isadigit{2}}\ c{\isadigit{0}}{\isadigit{3}}\ \isacommand{by}\isamarkupfalse%
\ simp\isanewline
\ \ \ \ \ \ \isacommand{also}\isamarkupfalse%
\ \isacommand{have}\isamarkupfalse%
\ {\isachardoublequoteopen}{\isasymdots}\ {\isacharequal}{\kern0pt}\ {\isacharparenleft}{\kern0pt}v\ {\isasymOtimes}\ {\isacharbar}{\kern0pt}zero{\isasymrangle}{\isacharparenright}{\kern0pt}\ {\isachardollar}{\kern0pt}{\isachardollar}{\kern0pt}\ {\isacharparenleft}{\kern0pt}{\isadigit{0}}{\isacharcomma}{\kern0pt}\ {\isadigit{0}}{\isacharparenright}{\kern0pt}{\isachardoublequoteclose}\ \isacommand{by}\isamarkupfalse%
\ auto\isanewline
\ \ \ \ \ \ \isacommand{finally}\isamarkupfalse%
\ \isacommand{show}\isamarkupfalse%
\ {\isachardoublequoteopen}control{\isadigit{2}}\ U\ {\isachardollar}{\kern0pt}{\isachardollar}{\kern0pt}\ {\isacharparenleft}{\kern0pt}i{\isacharcomma}{\kern0pt}\ {\isadigit{0}}{\isacharparenright}{\kern0pt}\ {\isacharasterisk}{\kern0pt}\ {\isacharparenleft}{\kern0pt}v\ {\isasymOtimes}\ {\isacharbar}{\kern0pt}Deutsch{\isachardot}{\kern0pt}zero{\isasymrangle}{\isacharparenright}{\kern0pt}\ {\isachardollar}{\kern0pt}{\isachardollar}{\kern0pt}\ {\isacharparenleft}{\kern0pt}{\isadigit{0}}{\isacharcomma}{\kern0pt}\ {\isadigit{0}}{\isacharparenright}{\kern0pt}\ {\isacharplus}{\kern0pt}\isanewline
\ \ \ \ \ \ \ \ \ \ \ \ \ \ \ \ \ \ \ \ control{\isadigit{2}}\ U\ {\isachardollar}{\kern0pt}{\isachardollar}{\kern0pt}\ {\isacharparenleft}{\kern0pt}i{\isacharcomma}{\kern0pt}\ {\isadigit{1}}{\isacharparenright}{\kern0pt}\ {\isacharasterisk}{\kern0pt}\ {\isacharparenleft}{\kern0pt}v\ {\isasymOtimes}\ {\isacharbar}{\kern0pt}Deutsch{\isachardot}{\kern0pt}zero{\isasymrangle}{\isacharparenright}{\kern0pt}\ {\isachardollar}{\kern0pt}{\isachardollar}{\kern0pt}\ {\isacharparenleft}{\kern0pt}{\isadigit{1}}{\isacharcomma}{\kern0pt}\ {\isadigit{0}}{\isacharparenright}{\kern0pt}\ {\isacharplus}{\kern0pt}\isanewline
\ \ \ \ \ \ \ \ \ \ \ \ \ \ \ \ \ \ \ \ control{\isadigit{2}}\ U\ {\isachardollar}{\kern0pt}{\isachardollar}{\kern0pt}\ {\isacharparenleft}{\kern0pt}i{\isacharcomma}{\kern0pt}\ {\isadigit{2}}{\isacharparenright}{\kern0pt}\ {\isacharasterisk}{\kern0pt}\ {\isacharparenleft}{\kern0pt}v\ {\isasymOtimes}\ {\isacharbar}{\kern0pt}Deutsch{\isachardot}{\kern0pt}zero{\isasymrangle}{\isacharparenright}{\kern0pt}\ {\isachardollar}{\kern0pt}{\isachardollar}{\kern0pt}\ {\isacharparenleft}{\kern0pt}{\isadigit{2}}{\isacharcomma}{\kern0pt}\ {\isadigit{0}}{\isacharparenright}{\kern0pt}\ {\isacharplus}{\kern0pt}\isanewline
\ \ \ \ \ \ \ \ \ \ \ \ \ \ \ \ \ \ \ \ control{\isadigit{2}}\ U\ {\isachardollar}{\kern0pt}{\isachardollar}{\kern0pt}\ {\isacharparenleft}{\kern0pt}i{\isacharcomma}{\kern0pt}\ {\isadigit{3}}{\isacharparenright}{\kern0pt}\ {\isacharasterisk}{\kern0pt}\ {\isacharparenleft}{\kern0pt}v\ {\isasymOtimes}\ {\isacharbar}{\kern0pt}Deutsch{\isachardot}{\kern0pt}zero{\isasymrangle}{\isacharparenright}{\kern0pt}\ {\isachardollar}{\kern0pt}{\isachardollar}{\kern0pt}\ {\isacharparenleft}{\kern0pt}{\isadigit{3}}{\isacharcomma}{\kern0pt}\ {\isadigit{0}}{\isacharparenright}{\kern0pt}\ {\isacharequal}{\kern0pt}\isanewline
\ \ \ \ \ \ \ \ \ \ \ \ \ \ \ \ \ \ \ \ {\isacharparenleft}{\kern0pt}v\ {\isasymOtimes}\ {\isacharbar}{\kern0pt}Deutsch{\isachardot}{\kern0pt}zero{\isasymrangle}{\isacharparenright}{\kern0pt}\ {\isachardollar}{\kern0pt}{\isachardollar}{\kern0pt}\ {\isacharparenleft}{\kern0pt}i{\isacharcomma}{\kern0pt}\ {\isadigit{0}}{\isacharparenright}{\kern0pt}{\isachardoublequoteclose}\isanewline
\ \ \ \ \ \ \ \ \isacommand{using}\isamarkupfalse%
\ i{\isadigit{0}}\ \isacommand{by}\isamarkupfalse%
\ simp\isanewline
\ \ \ \ \isacommand{next}\isamarkupfalse%
\isanewline
\ \ \ \ \ \ \isacommand{assume}\isamarkupfalse%
\ id{\isacharcolon}{\kern0pt}{\isachardoublequoteopen}i\ {\isacharequal}{\kern0pt}\ {\isadigit{1}}\ {\isasymor}\ i\ {\isacharequal}{\kern0pt}\ {\isadigit{2}}\ {\isasymor}\ i\ {\isacharequal}{\kern0pt}\ {\isadigit{3}}{\isachardoublequoteclose}\isanewline
\ \ \ \ \ \ \isacommand{show}\isamarkupfalse%
\ {\isachardoublequoteopen}control{\isadigit{2}}\ U\ {\isachardollar}{\kern0pt}{\isachardollar}{\kern0pt}\ {\isacharparenleft}{\kern0pt}i{\isacharcomma}{\kern0pt}\ {\isadigit{0}}{\isacharparenright}{\kern0pt}\ {\isacharasterisk}{\kern0pt}\ {\isacharparenleft}{\kern0pt}v\ {\isasymOtimes}\ {\isacharbar}{\kern0pt}Deutsch{\isachardot}{\kern0pt}zero{\isasymrangle}{\isacharparenright}{\kern0pt}\ {\isachardollar}{\kern0pt}{\isachardollar}{\kern0pt}\ {\isacharparenleft}{\kern0pt}{\isadigit{0}}{\isacharcomma}{\kern0pt}\ {\isadigit{0}}{\isacharparenright}{\kern0pt}\ {\isacharplus}{\kern0pt}\isanewline
\ \ \ \ \ \ \ \ \ \ \ \ control{\isadigit{2}}\ U\ {\isachardollar}{\kern0pt}{\isachardollar}{\kern0pt}\ {\isacharparenleft}{\kern0pt}i{\isacharcomma}{\kern0pt}\ {\isadigit{1}}{\isacharparenright}{\kern0pt}\ {\isacharasterisk}{\kern0pt}\ {\isacharparenleft}{\kern0pt}v\ {\isasymOtimes}\ {\isacharbar}{\kern0pt}Deutsch{\isachardot}{\kern0pt}zero{\isasymrangle}{\isacharparenright}{\kern0pt}\ {\isachardollar}{\kern0pt}{\isachardollar}{\kern0pt}\ {\isacharparenleft}{\kern0pt}{\isadigit{1}}{\isacharcomma}{\kern0pt}\ {\isadigit{0}}{\isacharparenright}{\kern0pt}\ {\isacharplus}{\kern0pt}\isanewline
\ \ \ \ \ \ \ \ \ \ \ \ control{\isadigit{2}}\ U\ {\isachardollar}{\kern0pt}{\isachardollar}{\kern0pt}\ {\isacharparenleft}{\kern0pt}i{\isacharcomma}{\kern0pt}\ {\isadigit{2}}{\isacharparenright}{\kern0pt}\ {\isacharasterisk}{\kern0pt}\ {\isacharparenleft}{\kern0pt}v\ {\isasymOtimes}\ {\isacharbar}{\kern0pt}Deutsch{\isachardot}{\kern0pt}zero{\isasymrangle}{\isacharparenright}{\kern0pt}\ {\isachardollar}{\kern0pt}{\isachardollar}{\kern0pt}\ {\isacharparenleft}{\kern0pt}{\isadigit{2}}{\isacharcomma}{\kern0pt}\ {\isadigit{0}}{\isacharparenright}{\kern0pt}\ {\isacharplus}{\kern0pt}\isanewline
\ \ \ \ \ \ \ \ \ \ \ \ control{\isadigit{2}}\ U\ {\isachardollar}{\kern0pt}{\isachardollar}{\kern0pt}\ {\isacharparenleft}{\kern0pt}i{\isacharcomma}{\kern0pt}\ {\isadigit{3}}{\isacharparenright}{\kern0pt}\ {\isacharasterisk}{\kern0pt}\ {\isacharparenleft}{\kern0pt}v\ {\isasymOtimes}\ {\isacharbar}{\kern0pt}Deutsch{\isachardot}{\kern0pt}zero{\isasymrangle}{\isacharparenright}{\kern0pt}\ {\isachardollar}{\kern0pt}{\isachardollar}{\kern0pt}\ {\isacharparenleft}{\kern0pt}{\isadigit{3}}{\isacharcomma}{\kern0pt}\ {\isadigit{0}}{\isacharparenright}{\kern0pt}\ {\isacharequal}{\kern0pt}\isanewline
\ \ \ \ \ \ \ \ \ \ \ \ {\isacharparenleft}{\kern0pt}v\ {\isasymOtimes}\ {\isacharbar}{\kern0pt}Deutsch{\isachardot}{\kern0pt}zero{\isasymrangle}{\isacharparenright}{\kern0pt}\ {\isachardollar}{\kern0pt}{\isachardollar}{\kern0pt}\ {\isacharparenleft}{\kern0pt}i{\isacharcomma}{\kern0pt}\ {\isadigit{0}}{\isacharparenright}{\kern0pt}{\isachardoublequoteclose}\isanewline
\ \ \ \ \ \ \isacommand{proof}\isamarkupfalse%
\ {\isacharparenleft}{\kern0pt}rule\ disjE{\isacharparenright}{\kern0pt}\isanewline
\ \ \ \ \ \ \ \ \isacommand{show}\isamarkupfalse%
\ {\isachardoublequoteopen}i\ {\isacharequal}{\kern0pt}\ {\isadigit{1}}\ {\isasymor}\ i\ {\isacharequal}{\kern0pt}\ {\isadigit{2}}\ {\isasymor}\ i\ {\isacharequal}{\kern0pt}\ {\isadigit{3}}{\isachardoublequoteclose}\ \isacommand{using}\isamarkupfalse%
\ id\ \isacommand{by}\isamarkupfalse%
\ this\isanewline
\ \ \ \ \ \ \isacommand{next}\isamarkupfalse%
\isanewline
\ \ \ \ \ \ \ \ \isacommand{assume}\isamarkupfalse%
\ i{\isadigit{1}}{\isacharcolon}{\kern0pt}{\isachardoublequoteopen}i\ {\isacharequal}{\kern0pt}\ {\isadigit{1}}{\isachardoublequoteclose}\isanewline
\ \ \ \ \ \ \ \ \isacommand{have}\isamarkupfalse%
\ c{\isadigit{1}}{\isadigit{0}}{\isacharcolon}{\kern0pt}{\isachardoublequoteopen}control{\isadigit{2}}\ U\ {\isachardollar}{\kern0pt}{\isachardollar}{\kern0pt}\ {\isacharparenleft}{\kern0pt}{\isadigit{1}}{\isacharcomma}{\kern0pt}{\isadigit{0}}{\isacharparenright}{\kern0pt}\ {\isacharequal}{\kern0pt}\ {\isadigit{0}}{\isachardoublequoteclose}\isanewline
\ \ \ \ \ \ \ \ \ \ \isacommand{by}\isamarkupfalse%
\ {\isacharparenleft}{\kern0pt}simp\ add{\isacharcolon}{\kern0pt}\ control{\isadigit{2}}{\isacharunderscore}{\kern0pt}def\ zero{\isacharunderscore}{\kern0pt}complex{\isachardot}{\kern0pt}code{\isacharparenright}{\kern0pt}\isanewline
\ \ \ \ \ \ \ \ \isacommand{have}\isamarkupfalse%
\ t{\isadigit{1}}{\isadigit{0}}{\isacharcolon}{\kern0pt}{\isachardoublequoteopen}{\isacharparenleft}{\kern0pt}v\ {\isasymOtimes}\ {\isacharbar}{\kern0pt}zero{\isasymrangle}{\isacharparenright}{\kern0pt}\ {\isachardollar}{\kern0pt}{\isachardollar}{\kern0pt}\ {\isacharparenleft}{\kern0pt}{\isadigit{1}}{\isacharcomma}{\kern0pt}{\isadigit{0}}{\isacharparenright}{\kern0pt}\ {\isacharequal}{\kern0pt}\ {\isadigit{0}}{\isachardoublequoteclose}\isanewline
\ \ \ \ \ \ \ \ \ \ \isacommand{using}\isamarkupfalse%
\ index{\isacharunderscore}{\kern0pt}tensor{\isacharunderscore}{\kern0pt}mat\ ket{\isacharunderscore}{\kern0pt}vec{\isacharunderscore}{\kern0pt}def\ Tensor{\isachardot}{\kern0pt}mat{\isacharunderscore}{\kern0pt}of{\isacharunderscore}{\kern0pt}cols{\isacharunderscore}{\kern0pt}list{\isacharunderscore}{\kern0pt}def\ \isanewline
\ \ \ \ \ \ \ \ \ \ \ \ {\isacartoucheopen}i\ {\isacharless}{\kern0pt}\ dim{\isacharunderscore}{\kern0pt}row\ {\isacharparenleft}{\kern0pt}v\ {\isasymOtimes}\ {\isacharbar}{\kern0pt}Deutsch{\isachardot}{\kern0pt}zero{\isasymrangle}{\isacharparenright}{\kern0pt}{\isacartoucheclose}\ {\isacartoucheopen}j\ {\isacharless}{\kern0pt}\ dim{\isacharunderscore}{\kern0pt}col\ {\isacharparenleft}{\kern0pt}v\ {\isasymOtimes}\ {\isacharbar}{\kern0pt}Deutsch{\isachardot}{\kern0pt}zero{\isasymrangle}{\isacharparenright}{\kern0pt}{\isacartoucheclose}\ i{\isadigit{1}}\ \isanewline
\ \ \ \ \ \ \ \ \ \ \isacommand{by}\isamarkupfalse%
\ fastforce\isanewline
\ \ \ \ \ \ \ \ \isacommand{have}\isamarkupfalse%
\ c{\isadigit{1}}{\isadigit{2}}{\isacharcolon}{\kern0pt}{\isachardoublequoteopen}control{\isadigit{2}}\ U\ {\isachardollar}{\kern0pt}{\isachardollar}{\kern0pt}\ {\isacharparenleft}{\kern0pt}{\isadigit{1}}{\isacharcomma}{\kern0pt}{\isadigit{2}}{\isacharparenright}{\kern0pt}\ {\isacharequal}{\kern0pt}\ {\isadigit{0}}{\isachardoublequoteclose}\isanewline
\ \ \ \ \ \ \ \ \ \ \isacommand{by}\isamarkupfalse%
\ {\isacharparenleft}{\kern0pt}simp\ add{\isacharcolon}{\kern0pt}\ control{\isadigit{2}}{\isacharunderscore}{\kern0pt}def\ zero{\isacharunderscore}{\kern0pt}complex{\isachardot}{\kern0pt}code{\isacharparenright}{\kern0pt}\isanewline
\ \ \ \ \ \ \ \ \isacommand{have}\isamarkupfalse%
\ t{\isadigit{3}}{\isadigit{0}}{\isacharcolon}{\kern0pt}{\isachardoublequoteopen}{\isacharparenleft}{\kern0pt}v\ {\isasymOtimes}\ {\isacharbar}{\kern0pt}zero{\isasymrangle}{\isacharparenright}{\kern0pt}\ {\isachardollar}{\kern0pt}{\isachardollar}{\kern0pt}\ {\isacharparenleft}{\kern0pt}{\isadigit{3}}{\isacharcomma}{\kern0pt}{\isadigit{0}}{\isacharparenright}{\kern0pt}\ {\isacharequal}{\kern0pt}\ {\isadigit{0}}{\isachardoublequoteclose}\isanewline
\ \ \ \ \ \ \ \ \isacommand{proof}\isamarkupfalse%
\ {\isacharminus}{\kern0pt}\isanewline
\ \ \ \ \ \ \ \ \ \ \isacommand{have}\isamarkupfalse%
\ {\isachardoublequoteopen}{\isacharparenleft}{\kern0pt}v\ {\isasymOtimes}\ {\isacharbar}{\kern0pt}zero{\isasymrangle}{\isacharparenright}{\kern0pt}\ {\isachardollar}{\kern0pt}{\isachardollar}{\kern0pt}\ {\isacharparenleft}{\kern0pt}{\isadigit{3}}{\isacharcomma}{\kern0pt}{\isadigit{0}}{\isacharparenright}{\kern0pt}\ {\isacharequal}{\kern0pt}\ v\ {\isachardollar}{\kern0pt}{\isachardollar}{\kern0pt}\ {\isacharparenleft}{\kern0pt}{\isadigit{1}}{\isacharcomma}{\kern0pt}{\isadigit{0}}{\isacharparenright}{\kern0pt}\ {\isacharasterisk}{\kern0pt}\ {\isacharbar}{\kern0pt}zero{\isasymrangle}\ {\isachardollar}{\kern0pt}{\isachardollar}{\kern0pt}\ {\isacharparenleft}{\kern0pt}{\isadigit{1}}{\isacharcomma}{\kern0pt}{\isadigit{0}}{\isacharparenright}{\kern0pt}{\isachardoublequoteclose}\isanewline
\ \ \ \ \ \ \ \ \ \ \ \ \isacommand{using}\isamarkupfalse%
\ index{\isacharunderscore}{\kern0pt}tensor{\isacharunderscore}{\kern0pt}mat\isanewline
\ \ \ \ \ \ \ \ \ \ \ \ \isacommand{by}\isamarkupfalse%
\ {\isacharparenleft}{\kern0pt}smt\ {\isacharparenleft}{\kern0pt}verit{\isacharparenright}{\kern0pt}\ Euclidean{\isacharunderscore}{\kern0pt}Rings{\isachardot}{\kern0pt}div{\isacharunderscore}{\kern0pt}eq{\isacharunderscore}{\kern0pt}{\isadigit{0}}{\isacharunderscore}{\kern0pt}iff\ H{\isacharunderscore}{\kern0pt}on{\isacharunderscore}{\kern0pt}ket{\isacharunderscore}{\kern0pt}zero{\isacharunderscore}{\kern0pt}is{\isacharunderscore}{\kern0pt}state\ \isanewline
\ \ \ \ \ \ \ \ \ \ \ \ \ \ \ \ H{\isacharunderscore}{\kern0pt}without{\isacharunderscore}{\kern0pt}scalar{\isacharunderscore}{\kern0pt}prod\ One{\isacharunderscore}{\kern0pt}nat{\isacharunderscore}{\kern0pt}def\ Suc{\isacharunderscore}{\kern0pt}{\isadigit{1}}\ {\isacartoucheopen}j\ {\isacharless}{\kern0pt}\ dim{\isacharunderscore}{\kern0pt}col\ {\isacharparenleft}{\kern0pt}v\ {\isasymOtimes}\ {\isacharbar}{\kern0pt}Deutsch{\isachardot}{\kern0pt}zero{\isasymrangle}{\isacharparenright}{\kern0pt}{\isacartoucheclose}\ \isanewline
\ \ \ \ \ \ \ \ \ \ \ \ \ \ \ \ add{\isachardot}{\kern0pt}commute\ assms{\isacharparenleft}{\kern0pt}{\isadigit{1}}{\isacharparenright}{\kern0pt}\ dim{\isacharunderscore}{\kern0pt}col{\isacharunderscore}{\kern0pt}tensor{\isacharunderscore}{\kern0pt}mat\ dim{\isacharunderscore}{\kern0pt}row{\isacharunderscore}{\kern0pt}mat{\isacharparenleft}{\kern0pt}{\isadigit{1}}{\isacharparenright}{\kern0pt}\ index{\isacharunderscore}{\kern0pt}mult{\isacharunderscore}{\kern0pt}mat{\isacharparenleft}{\kern0pt}{\isadigit{2}}{\isacharparenright}{\kern0pt}\ j{\isadigit{0}}\ \isanewline
\ \ \ \ \ \ \ \ \ \ \ \ \ \ \ \ ket{\isacharunderscore}{\kern0pt}zero{\isacharunderscore}{\kern0pt}is{\isacharunderscore}{\kern0pt}state\ mod{\isacharunderscore}{\kern0pt}less\ mod{\isacharunderscore}{\kern0pt}less{\isacharunderscore}{\kern0pt}divisor\ mod{\isacharunderscore}{\kern0pt}mult{\isadigit{2}}{\isacharunderscore}{\kern0pt}eq\ mult{\isacharunderscore}{\kern0pt}{\isadigit{2}}\ nat{\isacharunderscore}{\kern0pt}{\isadigit{0}}{\isacharunderscore}{\kern0pt}less{\isacharunderscore}{\kern0pt}mult{\isacharunderscore}{\kern0pt}iff\ \isanewline
\ \ \ \ \ \ \ \ \ \ \ \ \ \ \ \ numeral{\isacharunderscore}{\kern0pt}{\isadigit{3}}{\isacharunderscore}{\kern0pt}eq{\isacharunderscore}{\kern0pt}{\isadigit{3}}\ plus{\isacharunderscore}{\kern0pt}{\isadigit{1}}{\isacharunderscore}{\kern0pt}eq{\isacharunderscore}{\kern0pt}Suc\ pos{\isadigit{2}}\ state{\isachardot}{\kern0pt}dim{\isacharunderscore}{\kern0pt}row\ three{\isacharunderscore}{\kern0pt}div{\isacharunderscore}{\kern0pt}two\ three{\isacharunderscore}{\kern0pt}mod{\isacharunderscore}{\kern0pt}two{\isacharparenright}{\kern0pt}\isanewline
\ \ \ \ \ \ \ \ \ \ \isacommand{also}\isamarkupfalse%
\ \isacommand{have}\isamarkupfalse%
\ {\isachardoublequoteopen}{\isasymdots}\ {\isacharequal}{\kern0pt}\ {\isadigit{0}}{\isachardoublequoteclose}\ \isacommand{by}\isamarkupfalse%
\ auto\isanewline
\ \ \ \ \ \ \ \ \ \ \isacommand{finally}\isamarkupfalse%
\ \isacommand{show}\isamarkupfalse%
\ {\isacharquery}{\kern0pt}thesis\ \isacommand{by}\isamarkupfalse%
\ this\isanewline
\ \ \ \ \ \ \ \ \isacommand{qed}\isamarkupfalse%
\isanewline
\ \ \ \ \ \ \ \ \isacommand{show}\isamarkupfalse%
\ {\isachardoublequoteopen}control{\isadigit{2}}\ U\ {\isachardollar}{\kern0pt}{\isachardollar}{\kern0pt}\ {\isacharparenleft}{\kern0pt}i{\isacharcomma}{\kern0pt}\ {\isadigit{0}}{\isacharparenright}{\kern0pt}\ {\isacharasterisk}{\kern0pt}\ {\isacharparenleft}{\kern0pt}v\ {\isasymOtimes}\ {\isacharbar}{\kern0pt}Deutsch{\isachardot}{\kern0pt}zero{\isasymrangle}{\isacharparenright}{\kern0pt}\ {\isachardollar}{\kern0pt}{\isachardollar}{\kern0pt}\ {\isacharparenleft}{\kern0pt}{\isadigit{0}}{\isacharcomma}{\kern0pt}\ {\isadigit{0}}{\isacharparenright}{\kern0pt}\ {\isacharplus}{\kern0pt}\isanewline
\ \ \ \ \ \ \ \ \ \ \ \ \ \ control{\isadigit{2}}\ U\ {\isachardollar}{\kern0pt}{\isachardollar}{\kern0pt}\ {\isacharparenleft}{\kern0pt}i{\isacharcomma}{\kern0pt}\ {\isadigit{1}}{\isacharparenright}{\kern0pt}\ {\isacharasterisk}{\kern0pt}\ {\isacharparenleft}{\kern0pt}v\ {\isasymOtimes}\ {\isacharbar}{\kern0pt}Deutsch{\isachardot}{\kern0pt}zero{\isasymrangle}{\isacharparenright}{\kern0pt}\ {\isachardollar}{\kern0pt}{\isachardollar}{\kern0pt}\ {\isacharparenleft}{\kern0pt}{\isadigit{1}}{\isacharcomma}{\kern0pt}\ {\isadigit{0}}{\isacharparenright}{\kern0pt}\ {\isacharplus}{\kern0pt}\isanewline
\ \ \ \ \ \ \ \ \ \ \ \ \ \ control{\isadigit{2}}\ U\ {\isachardollar}{\kern0pt}{\isachardollar}{\kern0pt}\ {\isacharparenleft}{\kern0pt}i{\isacharcomma}{\kern0pt}\ {\isadigit{2}}{\isacharparenright}{\kern0pt}\ {\isacharasterisk}{\kern0pt}\ {\isacharparenleft}{\kern0pt}v\ {\isasymOtimes}\ {\isacharbar}{\kern0pt}Deutsch{\isachardot}{\kern0pt}zero{\isasymrangle}{\isacharparenright}{\kern0pt}\ {\isachardollar}{\kern0pt}{\isachardollar}{\kern0pt}\ {\isacharparenleft}{\kern0pt}{\isadigit{2}}{\isacharcomma}{\kern0pt}\ {\isadigit{0}}{\isacharparenright}{\kern0pt}\ {\isacharplus}{\kern0pt}\isanewline
\ \ \ \ \ \ \ \ \ \ \ \ \ \ control{\isadigit{2}}\ U\ {\isachardollar}{\kern0pt}{\isachardollar}{\kern0pt}\ {\isacharparenleft}{\kern0pt}i{\isacharcomma}{\kern0pt}\ {\isadigit{3}}{\isacharparenright}{\kern0pt}\ {\isacharasterisk}{\kern0pt}\ {\isacharparenleft}{\kern0pt}v\ {\isasymOtimes}\ {\isacharbar}{\kern0pt}Deutsch{\isachardot}{\kern0pt}zero{\isasymrangle}{\isacharparenright}{\kern0pt}\ {\isachardollar}{\kern0pt}{\isachardollar}{\kern0pt}\ {\isacharparenleft}{\kern0pt}{\isadigit{3}}{\isacharcomma}{\kern0pt}\ {\isadigit{0}}{\isacharparenright}{\kern0pt}\ {\isacharequal}{\kern0pt}\isanewline
\ \ \ \ \ \ \ \ \ \ \ \ \ \ {\isacharparenleft}{\kern0pt}v\ {\isasymOtimes}\ {\isacharbar}{\kern0pt}Deutsch{\isachardot}{\kern0pt}zero{\isasymrangle}{\isacharparenright}{\kern0pt}\ {\isachardollar}{\kern0pt}{\isachardollar}{\kern0pt}\ {\isacharparenleft}{\kern0pt}i{\isacharcomma}{\kern0pt}\ {\isadigit{0}}{\isacharparenright}{\kern0pt}{\isachardoublequoteclose}\isanewline
\ \ \ \ \ \ \ \ \ \ \isacommand{using}\isamarkupfalse%
\ i{\isadigit{1}}\ c{\isadigit{1}}{\isadigit{0}}\ t{\isadigit{1}}{\isadigit{0}}\ c{\isadigit{1}}{\isadigit{2}}\ t{\isadigit{3}}{\isadigit{0}}\ \isacommand{by}\isamarkupfalse%
\ auto\isanewline
\ \ \ \ \ \ \isacommand{next}\isamarkupfalse%
\isanewline
\ \ \ \ \ \ \ \ \isacommand{assume}\isamarkupfalse%
\ id{\isadigit{2}}{\isacharcolon}{\kern0pt}{\isachardoublequoteopen}i\ {\isacharequal}{\kern0pt}\ {\isadigit{2}}\ {\isasymor}\ i\ {\isacharequal}{\kern0pt}\ {\isadigit{3}}{\isachardoublequoteclose}\isanewline
\ \ \ \ \ \ \ \ \isacommand{show}\isamarkupfalse%
\ {\isachardoublequoteopen}control{\isadigit{2}}\ U\ {\isachardollar}{\kern0pt}{\isachardollar}{\kern0pt}\ {\isacharparenleft}{\kern0pt}i{\isacharcomma}{\kern0pt}\ {\isadigit{0}}{\isacharparenright}{\kern0pt}\ {\isacharasterisk}{\kern0pt}\ {\isacharparenleft}{\kern0pt}v\ {\isasymOtimes}\ {\isacharbar}{\kern0pt}Deutsch{\isachardot}{\kern0pt}zero{\isasymrangle}{\isacharparenright}{\kern0pt}\ {\isachardollar}{\kern0pt}{\isachardollar}{\kern0pt}\ {\isacharparenleft}{\kern0pt}{\isadigit{0}}{\isacharcomma}{\kern0pt}\ {\isadigit{0}}{\isacharparenright}{\kern0pt}\ {\isacharplus}{\kern0pt}\isanewline
\ \ \ \ \ \ \ \ \ \ \ \ \ \ control{\isadigit{2}}\ U\ {\isachardollar}{\kern0pt}{\isachardollar}{\kern0pt}\ {\isacharparenleft}{\kern0pt}i{\isacharcomma}{\kern0pt}\ {\isadigit{1}}{\isacharparenright}{\kern0pt}\ {\isacharasterisk}{\kern0pt}\ {\isacharparenleft}{\kern0pt}v\ {\isasymOtimes}\ {\isacharbar}{\kern0pt}Deutsch{\isachardot}{\kern0pt}zero{\isasymrangle}{\isacharparenright}{\kern0pt}\ {\isachardollar}{\kern0pt}{\isachardollar}{\kern0pt}\ {\isacharparenleft}{\kern0pt}{\isadigit{1}}{\isacharcomma}{\kern0pt}\ {\isadigit{0}}{\isacharparenright}{\kern0pt}\ {\isacharplus}{\kern0pt}\isanewline
\ \ \ \ \ \ \ \ \ \ \ \ \ \ control{\isadigit{2}}\ U\ {\isachardollar}{\kern0pt}{\isachardollar}{\kern0pt}\ {\isacharparenleft}{\kern0pt}i{\isacharcomma}{\kern0pt}\ {\isadigit{2}}{\isacharparenright}{\kern0pt}\ {\isacharasterisk}{\kern0pt}\ {\isacharparenleft}{\kern0pt}v\ {\isasymOtimes}\ {\isacharbar}{\kern0pt}Deutsch{\isachardot}{\kern0pt}zero{\isasymrangle}{\isacharparenright}{\kern0pt}\ {\isachardollar}{\kern0pt}{\isachardollar}{\kern0pt}\ {\isacharparenleft}{\kern0pt}{\isadigit{2}}{\isacharcomma}{\kern0pt}\ {\isadigit{0}}{\isacharparenright}{\kern0pt}\ {\isacharplus}{\kern0pt}\isanewline
\ \ \ \ \ \ \ \ \ \ \ \ \ \ control{\isadigit{2}}\ U\ {\isachardollar}{\kern0pt}{\isachardollar}{\kern0pt}\ {\isacharparenleft}{\kern0pt}i{\isacharcomma}{\kern0pt}\ {\isadigit{3}}{\isacharparenright}{\kern0pt}\ {\isacharasterisk}{\kern0pt}\ {\isacharparenleft}{\kern0pt}v\ {\isasymOtimes}\ {\isacharbar}{\kern0pt}Deutsch{\isachardot}{\kern0pt}zero{\isasymrangle}{\isacharparenright}{\kern0pt}\ {\isachardollar}{\kern0pt}{\isachardollar}{\kern0pt}\ {\isacharparenleft}{\kern0pt}{\isadigit{3}}{\isacharcomma}{\kern0pt}\ {\isadigit{0}}{\isacharparenright}{\kern0pt}\ {\isacharequal}{\kern0pt}\isanewline
\ \ \ \ \ \ \ \ \ \ \ \ \ \ {\isacharparenleft}{\kern0pt}v\ {\isasymOtimes}\ {\isacharbar}{\kern0pt}Deutsch{\isachardot}{\kern0pt}zero{\isasymrangle}{\isacharparenright}{\kern0pt}\ {\isachardollar}{\kern0pt}{\isachardollar}{\kern0pt}\ {\isacharparenleft}{\kern0pt}i{\isacharcomma}{\kern0pt}\ {\isadigit{0}}{\isacharparenright}{\kern0pt}{\isachardoublequoteclose}\isanewline
\ \ \ \ \ \ \ \ \isacommand{proof}\isamarkupfalse%
\ {\isacharparenleft}{\kern0pt}rule\ disjE{\isacharparenright}{\kern0pt}\isanewline
\ \ \ \ \ \ \ \ \ \ \isacommand{show}\isamarkupfalse%
\ {\isachardoublequoteopen}i\ {\isacharequal}{\kern0pt}\ {\isadigit{2}}\ {\isasymor}\ i\ {\isacharequal}{\kern0pt}\ {\isadigit{3}}{\isachardoublequoteclose}\isanewline
\ \ \ \ \ \ \ \ \ \ \ \ \isacommand{using}\isamarkupfalse%
\ id{\isadigit{2}}\ \isacommand{by}\isamarkupfalse%
\ this\isanewline
\ \ \ \ \ \ \ \ \isacommand{next}\isamarkupfalse%
\isanewline
\ \ \ \ \ \ \ \ \ \ \isacommand{assume}\isamarkupfalse%
\ i{\isadigit{2}}{\isacharcolon}{\kern0pt}{\isachardoublequoteopen}i\ {\isacharequal}{\kern0pt}\ {\isadigit{2}}{\isachardoublequoteclose}\isanewline
\ \ \ \ \ \ \ \ \ \ \isacommand{have}\isamarkupfalse%
\ c{\isadigit{2}}{\isadigit{0}}{\isacharcolon}{\kern0pt}{\isachardoublequoteopen}control{\isadigit{2}}\ U\ {\isachardollar}{\kern0pt}{\isachardollar}{\kern0pt}\ {\isacharparenleft}{\kern0pt}{\isadigit{2}}{\isacharcomma}{\kern0pt}{\isadigit{0}}{\isacharparenright}{\kern0pt}\ {\isacharequal}{\kern0pt}\ {\isadigit{0}}{\isachardoublequoteclose}\isanewline
\ \ \ \ \ \ \ \ \ \ \ \ \isacommand{by}\isamarkupfalse%
\ {\isacharparenleft}{\kern0pt}simp\ add{\isacharcolon}{\kern0pt}\ control{\isadigit{2}}{\isacharunderscore}{\kern0pt}def\ zero{\isacharunderscore}{\kern0pt}complex{\isachardot}{\kern0pt}code{\isacharparenright}{\kern0pt}\isanewline
\ \ \ \ \ \ \ \ \ \ \isacommand{have}\isamarkupfalse%
\ c{\isadigit{2}}{\isadigit{1}}{\isacharcolon}{\kern0pt}{\isachardoublequoteopen}control{\isadigit{2}}\ U\ {\isachardollar}{\kern0pt}{\isachardollar}{\kern0pt}\ {\isacharparenleft}{\kern0pt}{\isadigit{2}}{\isacharcomma}{\kern0pt}{\isadigit{1}}{\isacharparenright}{\kern0pt}\ {\isacharequal}{\kern0pt}\ {\isadigit{0}}{\isachardoublequoteclose}\isanewline
\ \ \ \ \ \ \ \ \ \ \ \ \isacommand{by}\isamarkupfalse%
\ {\isacharparenleft}{\kern0pt}simp\ add{\isacharcolon}{\kern0pt}\ control{\isadigit{2}}{\isacharunderscore}{\kern0pt}def\ zero{\isacharunderscore}{\kern0pt}complex{\isachardot}{\kern0pt}code{\isacharparenright}{\kern0pt}\isanewline
\ \ \ \ \ \ \ \ \ \ \isacommand{have}\isamarkupfalse%
\ c{\isadigit{2}}{\isadigit{2}}{\isacharcolon}{\kern0pt}{\isachardoublequoteopen}control{\isadigit{2}}\ U\ {\isachardollar}{\kern0pt}{\isachardollar}{\kern0pt}\ {\isacharparenleft}{\kern0pt}{\isadigit{2}}{\isacharcomma}{\kern0pt}{\isadigit{2}}{\isacharparenright}{\kern0pt}\ {\isacharequal}{\kern0pt}\ {\isadigit{1}}{\isachardoublequoteclose}\isanewline
\ \ \ \ \ \ \ \ \ \ \ \ \isacommand{by}\isamarkupfalse%
\ {\isacharparenleft}{\kern0pt}simp\ add{\isacharcolon}{\kern0pt}\ control{\isadigit{2}}{\isacharunderscore}{\kern0pt}def\ one{\isacharunderscore}{\kern0pt}complex{\isachardot}{\kern0pt}code{\isacharparenright}{\kern0pt}\isanewline
\ \ \ \ \ \ \ \ \ \ \isacommand{have}\isamarkupfalse%
\ c{\isadigit{2}}{\isadigit{3}}{\isacharcolon}{\kern0pt}{\isachardoublequoteopen}control{\isadigit{2}}\ U\ {\isachardollar}{\kern0pt}{\isachardollar}{\kern0pt}\ {\isacharparenleft}{\kern0pt}{\isadigit{2}}{\isacharcomma}{\kern0pt}{\isadigit{3}}{\isacharparenright}{\kern0pt}\ {\isacharequal}{\kern0pt}\ {\isadigit{0}}{\isachardoublequoteclose}\isanewline
\ \ \ \ \ \ \ \ \ \ \ \ \isacommand{by}\isamarkupfalse%
\ {\isacharparenleft}{\kern0pt}simp\ add{\isacharcolon}{\kern0pt}\ control{\isadigit{2}}{\isacharunderscore}{\kern0pt}def\ zero{\isacharunderscore}{\kern0pt}complex{\isachardot}{\kern0pt}code{\isacharparenright}{\kern0pt}\isanewline
\ \ \ \ \ \ \ \ \ \ \isacommand{show}\isamarkupfalse%
\ {\isachardoublequoteopen}control{\isadigit{2}}\ U\ {\isachardollar}{\kern0pt}{\isachardollar}{\kern0pt}\ {\isacharparenleft}{\kern0pt}i{\isacharcomma}{\kern0pt}\ {\isadigit{0}}{\isacharparenright}{\kern0pt}\ {\isacharasterisk}{\kern0pt}\ {\isacharparenleft}{\kern0pt}v\ {\isasymOtimes}\ {\isacharbar}{\kern0pt}Deutsch{\isachardot}{\kern0pt}zero{\isasymrangle}{\isacharparenright}{\kern0pt}\ {\isachardollar}{\kern0pt}{\isachardollar}{\kern0pt}\ {\isacharparenleft}{\kern0pt}{\isadigit{0}}{\isacharcomma}{\kern0pt}\ {\isadigit{0}}{\isacharparenright}{\kern0pt}\ {\isacharplus}{\kern0pt}\isanewline
\ \ \ \ \ \ \ \ \ \ \ \ \ \ \ \ control{\isadigit{2}}\ U\ {\isachardollar}{\kern0pt}{\isachardollar}{\kern0pt}\ {\isacharparenleft}{\kern0pt}i{\isacharcomma}{\kern0pt}\ {\isadigit{1}}{\isacharparenright}{\kern0pt}\ {\isacharasterisk}{\kern0pt}\ {\isacharparenleft}{\kern0pt}v\ {\isasymOtimes}\ {\isacharbar}{\kern0pt}Deutsch{\isachardot}{\kern0pt}zero{\isasymrangle}{\isacharparenright}{\kern0pt}\ {\isachardollar}{\kern0pt}{\isachardollar}{\kern0pt}\ {\isacharparenleft}{\kern0pt}{\isadigit{1}}{\isacharcomma}{\kern0pt}\ {\isadigit{0}}{\isacharparenright}{\kern0pt}\ {\isacharplus}{\kern0pt}\isanewline
\ \ \ \ \ \ \ \ \ \ \ \ \ \ \ \ control{\isadigit{2}}\ U\ {\isachardollar}{\kern0pt}{\isachardollar}{\kern0pt}\ {\isacharparenleft}{\kern0pt}i{\isacharcomma}{\kern0pt}\ {\isadigit{2}}{\isacharparenright}{\kern0pt}\ {\isacharasterisk}{\kern0pt}\ {\isacharparenleft}{\kern0pt}v\ {\isasymOtimes}\ {\isacharbar}{\kern0pt}Deutsch{\isachardot}{\kern0pt}zero{\isasymrangle}{\isacharparenright}{\kern0pt}\ {\isachardollar}{\kern0pt}{\isachardollar}{\kern0pt}\ {\isacharparenleft}{\kern0pt}{\isadigit{2}}{\isacharcomma}{\kern0pt}\ {\isadigit{0}}{\isacharparenright}{\kern0pt}\ {\isacharplus}{\kern0pt}\isanewline
\ \ \ \ \ \ \ \ \ \ \ \ \ \ \ \ control{\isadigit{2}}\ U\ {\isachardollar}{\kern0pt}{\isachardollar}{\kern0pt}\ {\isacharparenleft}{\kern0pt}i{\isacharcomma}{\kern0pt}\ {\isadigit{3}}{\isacharparenright}{\kern0pt}\ {\isacharasterisk}{\kern0pt}\ {\isacharparenleft}{\kern0pt}v\ {\isasymOtimes}\ {\isacharbar}{\kern0pt}Deutsch{\isachardot}{\kern0pt}zero{\isasymrangle}{\isacharparenright}{\kern0pt}\ {\isachardollar}{\kern0pt}{\isachardollar}{\kern0pt}\ {\isacharparenleft}{\kern0pt}{\isadigit{3}}{\isacharcomma}{\kern0pt}\ {\isadigit{0}}{\isacharparenright}{\kern0pt}\ {\isacharequal}{\kern0pt}\isanewline
\ \ \ \ \ \ \ \ \ \ \ \ \ \ \ \ {\isacharparenleft}{\kern0pt}v\ {\isasymOtimes}\ {\isacharbar}{\kern0pt}Deutsch{\isachardot}{\kern0pt}zero{\isasymrangle}{\isacharparenright}{\kern0pt}\ {\isachardollar}{\kern0pt}{\isachardollar}{\kern0pt}\ {\isacharparenleft}{\kern0pt}i{\isacharcomma}{\kern0pt}\ {\isadigit{0}}{\isacharparenright}{\kern0pt}{\isachardoublequoteclose}\isanewline
\ \ \ \ \ \ \ \ \ \ \ \ \isacommand{using}\isamarkupfalse%
\ i{\isadigit{2}}\ c{\isadigit{2}}{\isadigit{0}}\ c{\isadigit{2}}{\isadigit{1}}\ c{\isadigit{2}}{\isadigit{2}}\ c{\isadigit{2}}{\isadigit{3}}\ \isacommand{by}\isamarkupfalse%
\ auto\isanewline
\ \ \ \ \ \ \ \ \isacommand{next}\isamarkupfalse%
\isanewline
\ \ \ \ \ \ \ \ \ \ \isacommand{assume}\isamarkupfalse%
\ i{\isadigit{3}}{\isacharcolon}{\kern0pt}{\isachardoublequoteopen}i\ {\isacharequal}{\kern0pt}\ {\isadigit{3}}{\isachardoublequoteclose}\isanewline
\ \ \ \ \ \ \ \ \ \ \isacommand{have}\isamarkupfalse%
\ c{\isadigit{3}}{\isadigit{0}}{\isacharcolon}{\kern0pt}{\isachardoublequoteopen}control{\isadigit{2}}\ U\ {\isachardollar}{\kern0pt}{\isachardollar}{\kern0pt}\ {\isacharparenleft}{\kern0pt}{\isadigit{3}}{\isacharcomma}{\kern0pt}{\isadigit{0}}{\isacharparenright}{\kern0pt}\ {\isacharequal}{\kern0pt}\ {\isadigit{0}}{\isachardoublequoteclose}\isanewline
\ \ \ \ \ \ \ \ \ \ \ \ \isacommand{by}\isamarkupfalse%
\ {\isacharparenleft}{\kern0pt}simp\ add{\isacharcolon}{\kern0pt}\ control{\isadigit{2}}{\isacharunderscore}{\kern0pt}def\ zero{\isacharunderscore}{\kern0pt}complex{\isachardot}{\kern0pt}code{\isacharparenright}{\kern0pt}\isanewline
\ \ \ \ \ \ \ \ \ \ \isacommand{have}\isamarkupfalse%
\ t{\isadigit{1}}{\isadigit{0}}{\isacharcolon}{\kern0pt}{\isachardoublequoteopen}{\isacharparenleft}{\kern0pt}v\ {\isasymOtimes}\ {\isacharbar}{\kern0pt}zero{\isasymrangle}{\isacharparenright}{\kern0pt}\ {\isachardollar}{\kern0pt}{\isachardollar}{\kern0pt}\ {\isacharparenleft}{\kern0pt}{\isadigit{1}}{\isacharcomma}{\kern0pt}{\isadigit{0}}{\isacharparenright}{\kern0pt}\ {\isacharequal}{\kern0pt}\ {\isadigit{0}}{\isachardoublequoteclose}\isanewline
\ \ \ \ \ \ \ \ \ \ \ \ \isacommand{using}\isamarkupfalse%
\ index{\isacharunderscore}{\kern0pt}tensor{\isacharunderscore}{\kern0pt}mat\ ket{\isacharunderscore}{\kern0pt}vec{\isacharunderscore}{\kern0pt}def\ Tensor{\isachardot}{\kern0pt}mat{\isacharunderscore}{\kern0pt}of{\isacharunderscore}{\kern0pt}cols{\isacharunderscore}{\kern0pt}list{\isacharunderscore}{\kern0pt}def\ \isanewline
\ \ \ \ \ \ \ \ \ \ \ \ {\isacartoucheopen}i\ {\isacharless}{\kern0pt}\ dim{\isacharunderscore}{\kern0pt}row\ {\isacharparenleft}{\kern0pt}v\ {\isasymOtimes}\ {\isacharbar}{\kern0pt}Deutsch{\isachardot}{\kern0pt}zero{\isasymrangle}{\isacharparenright}{\kern0pt}{\isacartoucheclose}\ {\isacartoucheopen}j\ {\isacharless}{\kern0pt}\ dim{\isacharunderscore}{\kern0pt}col\ {\isacharparenleft}{\kern0pt}v\ {\isasymOtimes}\ {\isacharbar}{\kern0pt}Deutsch{\isachardot}{\kern0pt}zero{\isasymrangle}{\isacharparenright}{\kern0pt}{\isacartoucheclose}\ i{\isadigit{3}}\isanewline
\ \ \ \ \ \ \ \ \ \ \ \ \isacommand{by}\isamarkupfalse%
\ fastforce\isanewline
\ \ \ \ \ \ \ \ \ \ \isacommand{have}\isamarkupfalse%
\ c{\isadigit{3}}{\isadigit{2}}{\isacharcolon}{\kern0pt}{\isachardoublequoteopen}control{\isadigit{2}}\ U\ {\isachardollar}{\kern0pt}{\isachardollar}{\kern0pt}\ {\isacharparenleft}{\kern0pt}{\isadigit{3}}{\isacharcomma}{\kern0pt}{\isadigit{2}}{\isacharparenright}{\kern0pt}\ {\isacharequal}{\kern0pt}\ {\isadigit{0}}{\isachardoublequoteclose}\isanewline
\ \ \ \ \ \ \ \ \ \ \ \ \isacommand{by}\isamarkupfalse%
\ {\isacharparenleft}{\kern0pt}simp\ add{\isacharcolon}{\kern0pt}\ control{\isadigit{2}}{\isacharunderscore}{\kern0pt}def\ zero{\isacharunderscore}{\kern0pt}complex{\isachardot}{\kern0pt}code{\isacharparenright}{\kern0pt}\isanewline
\ \ \ \ \ \ \ \ \ \ \isacommand{have}\isamarkupfalse%
\ t{\isadigit{3}}{\isadigit{0}}{\isacharcolon}{\kern0pt}{\isachardoublequoteopen}{\isacharparenleft}{\kern0pt}v\ {\isasymOtimes}\ {\isacharbar}{\kern0pt}zero{\isasymrangle}{\isacharparenright}{\kern0pt}\ {\isachardollar}{\kern0pt}{\isachardollar}{\kern0pt}\ {\isacharparenleft}{\kern0pt}{\isadigit{3}}{\isacharcomma}{\kern0pt}{\isadigit{0}}{\isacharparenright}{\kern0pt}\ {\isacharequal}{\kern0pt}\ {\isadigit{0}}{\isachardoublequoteclose}\isanewline
\ \ \ \ \ \ \ \ \ \ \isacommand{proof}\isamarkupfalse%
\ {\isacharminus}{\kern0pt}\isanewline
\ \ \ \ \ \ \ \ \ \ \ \ \isacommand{have}\isamarkupfalse%
\ {\isachardoublequoteopen}{\isacharparenleft}{\kern0pt}v\ {\isasymOtimes}\ {\isacharbar}{\kern0pt}zero{\isasymrangle}{\isacharparenright}{\kern0pt}\ {\isachardollar}{\kern0pt}{\isachardollar}{\kern0pt}\ {\isacharparenleft}{\kern0pt}{\isadigit{3}}{\isacharcomma}{\kern0pt}{\isadigit{0}}{\isacharparenright}{\kern0pt}\ {\isacharequal}{\kern0pt}\ v\ {\isachardollar}{\kern0pt}{\isachardollar}{\kern0pt}\ {\isacharparenleft}{\kern0pt}{\isadigit{1}}{\isacharcomma}{\kern0pt}{\isadigit{0}}{\isacharparenright}{\kern0pt}\ {\isacharasterisk}{\kern0pt}\ {\isacharbar}{\kern0pt}zero{\isasymrangle}\ {\isachardollar}{\kern0pt}{\isachardollar}{\kern0pt}\ {\isacharparenleft}{\kern0pt}{\isadigit{1}}{\isacharcomma}{\kern0pt}{\isadigit{0}}{\isacharparenright}{\kern0pt}{\isachardoublequoteclose}\isanewline
\ \ \ \ \ \ \ \ \ \ \ \ \ \ \isacommand{using}\isamarkupfalse%
\ index{\isacharunderscore}{\kern0pt}tensor{\isacharunderscore}{\kern0pt}mat\isanewline
\ \ \ \ \ \ \ \ \ \ \ \ \ \ \isacommand{by}\isamarkupfalse%
\ {\isacharparenleft}{\kern0pt}smt\ {\isacharparenleft}{\kern0pt}verit{\isacharparenright}{\kern0pt}\ Euclidean{\isacharunderscore}{\kern0pt}Rings{\isachardot}{\kern0pt}div{\isacharunderscore}{\kern0pt}eq{\isacharunderscore}{\kern0pt}{\isadigit{0}}{\isacharunderscore}{\kern0pt}iff\ H{\isacharunderscore}{\kern0pt}on{\isacharunderscore}{\kern0pt}ket{\isacharunderscore}{\kern0pt}zero{\isacharunderscore}{\kern0pt}is{\isacharunderscore}{\kern0pt}state\ \isanewline
\ \ \ \ \ \ \ \ \ \ \ \ \ \ \ \ H{\isacharunderscore}{\kern0pt}without{\isacharunderscore}{\kern0pt}scalar{\isacharunderscore}{\kern0pt}prod\ One{\isacharunderscore}{\kern0pt}nat{\isacharunderscore}{\kern0pt}def\ Suc{\isacharunderscore}{\kern0pt}{\isadigit{1}}\ {\isacartoucheopen}j\ {\isacharless}{\kern0pt}\ dim{\isacharunderscore}{\kern0pt}col\ {\isacharparenleft}{\kern0pt}v\ {\isasymOtimes}\ {\isacharbar}{\kern0pt}Deutsch{\isachardot}{\kern0pt}zero{\isasymrangle}{\isacharparenright}{\kern0pt}{\isacartoucheclose}\ \isanewline
\ \ \ \ \ \ \ \ \ \ \ \ \ \ \ \ add{\isachardot}{\kern0pt}commute\ assms{\isacharparenleft}{\kern0pt}{\isadigit{1}}{\isacharparenright}{\kern0pt}\ dim{\isacharunderscore}{\kern0pt}col{\isacharunderscore}{\kern0pt}tensor{\isacharunderscore}{\kern0pt}mat\ dim{\isacharunderscore}{\kern0pt}row{\isacharunderscore}{\kern0pt}mat{\isacharparenleft}{\kern0pt}{\isadigit{1}}{\isacharparenright}{\kern0pt}\ index{\isacharunderscore}{\kern0pt}mult{\isacharunderscore}{\kern0pt}mat{\isacharparenleft}{\kern0pt}{\isadigit{2}}{\isacharparenright}{\kern0pt}\ j{\isadigit{0}}\ \isanewline
\ \ \ \ \ \ \ \ \ \ \ \ \ \ \ \ ket{\isacharunderscore}{\kern0pt}zero{\isacharunderscore}{\kern0pt}is{\isacharunderscore}{\kern0pt}state\ mod{\isacharunderscore}{\kern0pt}less\ mod{\isacharunderscore}{\kern0pt}less{\isacharunderscore}{\kern0pt}divisor\ mod{\isacharunderscore}{\kern0pt}mult{\isadigit{2}}{\isacharunderscore}{\kern0pt}eq\ mult{\isacharunderscore}{\kern0pt}{\isadigit{2}}\ nat{\isacharunderscore}{\kern0pt}{\isadigit{0}}{\isacharunderscore}{\kern0pt}less{\isacharunderscore}{\kern0pt}mult{\isacharunderscore}{\kern0pt}iff\ \isanewline
\ \ \ \ \ \ \ \ \ \ \ \ \ \ \ \ numeral{\isacharunderscore}{\kern0pt}{\isadigit{3}}{\isacharunderscore}{\kern0pt}eq{\isacharunderscore}{\kern0pt}{\isadigit{3}}\ plus{\isacharunderscore}{\kern0pt}{\isadigit{1}}{\isacharunderscore}{\kern0pt}eq{\isacharunderscore}{\kern0pt}Suc\ pos{\isadigit{2}}\ state{\isachardot}{\kern0pt}dim{\isacharunderscore}{\kern0pt}row\ three{\isacharunderscore}{\kern0pt}div{\isacharunderscore}{\kern0pt}two\ three{\isacharunderscore}{\kern0pt}mod{\isacharunderscore}{\kern0pt}two{\isacharparenright}{\kern0pt}\isanewline
\ \ \ \ \ \ \ \ \ \ \ \ \isacommand{also}\isamarkupfalse%
\ \isacommand{have}\isamarkupfalse%
\ {\isachardoublequoteopen}{\isasymdots}\ {\isacharequal}{\kern0pt}\ {\isadigit{0}}{\isachardoublequoteclose}\ \isacommand{by}\isamarkupfalse%
\ auto\isanewline
\ \ \ \ \ \ \ \ \ \ \ \ \isacommand{finally}\isamarkupfalse%
\ \isacommand{show}\isamarkupfalse%
\ {\isacharquery}{\kern0pt}thesis\ \isacommand{by}\isamarkupfalse%
\ this\isanewline
\ \ \ \ \ \ \ \ \ \ \isacommand{qed}\isamarkupfalse%
\isanewline
\ \ \ \ \ \ \ \ \ \ \isacommand{show}\isamarkupfalse%
\ {\isachardoublequoteopen}control{\isadigit{2}}\ U\ {\isachardollar}{\kern0pt}{\isachardollar}{\kern0pt}\ {\isacharparenleft}{\kern0pt}i{\isacharcomma}{\kern0pt}\ {\isadigit{0}}{\isacharparenright}{\kern0pt}\ {\isacharasterisk}{\kern0pt}\ {\isacharparenleft}{\kern0pt}v\ {\isasymOtimes}\ {\isacharbar}{\kern0pt}Deutsch{\isachardot}{\kern0pt}zero{\isasymrangle}{\isacharparenright}{\kern0pt}\ {\isachardollar}{\kern0pt}{\isachardollar}{\kern0pt}\ {\isacharparenleft}{\kern0pt}{\isadigit{0}}{\isacharcomma}{\kern0pt}\ {\isadigit{0}}{\isacharparenright}{\kern0pt}\ {\isacharplus}{\kern0pt}\isanewline
\ \ \ \ \ \ \ \ \ \ \ \ \ \ \ \ control{\isadigit{2}}\ U\ {\isachardollar}{\kern0pt}{\isachardollar}{\kern0pt}\ {\isacharparenleft}{\kern0pt}i{\isacharcomma}{\kern0pt}\ {\isadigit{1}}{\isacharparenright}{\kern0pt}\ {\isacharasterisk}{\kern0pt}\ {\isacharparenleft}{\kern0pt}v\ {\isasymOtimes}\ {\isacharbar}{\kern0pt}Deutsch{\isachardot}{\kern0pt}zero{\isasymrangle}{\isacharparenright}{\kern0pt}\ {\isachardollar}{\kern0pt}{\isachardollar}{\kern0pt}\ {\isacharparenleft}{\kern0pt}{\isadigit{1}}{\isacharcomma}{\kern0pt}\ {\isadigit{0}}{\isacharparenright}{\kern0pt}\ {\isacharplus}{\kern0pt}\isanewline
\ \ \ \ \ \ \ \ \ \ \ \ \ \ \ \ control{\isadigit{2}}\ U\ {\isachardollar}{\kern0pt}{\isachardollar}{\kern0pt}\ {\isacharparenleft}{\kern0pt}i{\isacharcomma}{\kern0pt}\ {\isadigit{2}}{\isacharparenright}{\kern0pt}\ {\isacharasterisk}{\kern0pt}\ {\isacharparenleft}{\kern0pt}v\ {\isasymOtimes}\ {\isacharbar}{\kern0pt}Deutsch{\isachardot}{\kern0pt}zero{\isasymrangle}{\isacharparenright}{\kern0pt}\ {\isachardollar}{\kern0pt}{\isachardollar}{\kern0pt}\ {\isacharparenleft}{\kern0pt}{\isadigit{2}}{\isacharcomma}{\kern0pt}\ {\isadigit{0}}{\isacharparenright}{\kern0pt}\ {\isacharplus}{\kern0pt}\isanewline
\ \ \ \ \ \ \ \ \ \ \ \ \ \ \ \ control{\isadigit{2}}\ U\ {\isachardollar}{\kern0pt}{\isachardollar}{\kern0pt}\ {\isacharparenleft}{\kern0pt}i{\isacharcomma}{\kern0pt}\ {\isadigit{3}}{\isacharparenright}{\kern0pt}\ {\isacharasterisk}{\kern0pt}\ {\isacharparenleft}{\kern0pt}v\ {\isasymOtimes}\ {\isacharbar}{\kern0pt}Deutsch{\isachardot}{\kern0pt}zero{\isasymrangle}{\isacharparenright}{\kern0pt}\ {\isachardollar}{\kern0pt}{\isachardollar}{\kern0pt}\ {\isacharparenleft}{\kern0pt}{\isadigit{3}}{\isacharcomma}{\kern0pt}\ {\isadigit{0}}{\isacharparenright}{\kern0pt}\ {\isacharequal}{\kern0pt}\isanewline
\ \ \ \ \ \ \ \ \ \ \ \ \ \ \ \ {\isacharparenleft}{\kern0pt}v\ {\isasymOtimes}\ {\isacharbar}{\kern0pt}Deutsch{\isachardot}{\kern0pt}zero{\isasymrangle}{\isacharparenright}{\kern0pt}\ {\isachardollar}{\kern0pt}{\isachardollar}{\kern0pt}\ {\isacharparenleft}{\kern0pt}i{\isacharcomma}{\kern0pt}\ {\isadigit{0}}{\isacharparenright}{\kern0pt}{\isachardoublequoteclose}\isanewline
\ \ \ \ \ \ \ \ \ \ \ \ \isacommand{using}\isamarkupfalse%
\ i{\isadigit{3}}\ c{\isadigit{3}}{\isadigit{0}}\ t{\isadigit{1}}{\isadigit{0}}\ c{\isadigit{3}}{\isadigit{2}}\ t{\isadigit{3}}{\isadigit{0}}\ \isacommand{by}\isamarkupfalse%
\ auto\isanewline
\ \ \ \ \ \ \ \ \isacommand{qed}\isamarkupfalse%
\isanewline
\ \ \ \ \ \ \isacommand{qed}\isamarkupfalse%
\isanewline
\ \ \ \ \isacommand{qed}\isamarkupfalse%
\isanewline
\ \ \ \ \isacommand{finally}\isamarkupfalse%
\ \isacommand{show}\isamarkupfalse%
\ {\isacharquery}{\kern0pt}thesis\ \isacommand{using}\isamarkupfalse%
\ j{\isadigit{0}}\ \isacommand{by}\isamarkupfalse%
\ simp\isanewline
\ \ \isacommand{qed}\isamarkupfalse%
\isanewline
\isacommand{next}\isamarkupfalse%
\isanewline
\ \ \isacommand{show}\isamarkupfalse%
\ {\isachardoublequoteopen}dim{\isacharunderscore}{\kern0pt}row\ {\isacharparenleft}{\kern0pt}control{\isadigit{2}}\ U\ {\isacharasterisk}{\kern0pt}\ {\isacharparenleft}{\kern0pt}v\ {\isasymOtimes}\ {\isacharbar}{\kern0pt}Deutsch{\isachardot}{\kern0pt}zero{\isasymrangle}{\isacharparenright}{\kern0pt}{\isacharparenright}{\kern0pt}\ {\isacharequal}{\kern0pt}\ dim{\isacharunderscore}{\kern0pt}row\ {\isacharparenleft}{\kern0pt}v\ {\isasymOtimes}\ {\isacharbar}{\kern0pt}Deutsch{\isachardot}{\kern0pt}zero{\isasymrangle}{\isacharparenright}{\kern0pt}{\isachardoublequoteclose}\isanewline
\ \ \ \ \isacommand{by}\isamarkupfalse%
\ {\isacharparenleft}{\kern0pt}metis\ assms{\isacharparenleft}{\kern0pt}{\isadigit{1}}{\isacharparenright}{\kern0pt}\ carrier{\isacharunderscore}{\kern0pt}matD{\isacharparenleft}{\kern0pt}{\isadigit{1}}{\isacharparenright}{\kern0pt}\ control{\isadigit{2}}{\isacharunderscore}{\kern0pt}carrier{\isacharunderscore}{\kern0pt}mat\ dim{\isacharunderscore}{\kern0pt}row{\isacharunderscore}{\kern0pt}mat{\isacharparenleft}{\kern0pt}{\isadigit{1}}{\isacharparenright}{\kern0pt}\ dim{\isacharunderscore}{\kern0pt}row{\isacharunderscore}{\kern0pt}tensor{\isacharunderscore}{\kern0pt}mat\ \isanewline
\ \ \ \ \ \ \ \ index{\isacharunderscore}{\kern0pt}mult{\isacharunderscore}{\kern0pt}mat{\isacharparenleft}{\kern0pt}{\isadigit{2}}{\isacharparenright}{\kern0pt}\ index{\isacharunderscore}{\kern0pt}unit{\isacharunderscore}{\kern0pt}vec{\isacharparenleft}{\kern0pt}{\isadigit{3}}{\isacharparenright}{\kern0pt}\ ket{\isacharunderscore}{\kern0pt}vec{\isacharunderscore}{\kern0pt}def\ num{\isacharunderscore}{\kern0pt}double\ numeral{\isacharunderscore}{\kern0pt}times{\isacharunderscore}{\kern0pt}numeral{\isacharparenright}{\kern0pt}\isanewline
\isacommand{next}\isamarkupfalse%
\isanewline
\ \ \isacommand{show}\isamarkupfalse%
\ {\isachardoublequoteopen}dim{\isacharunderscore}{\kern0pt}col\ {\isacharparenleft}{\kern0pt}control{\isadigit{2}}\ U\ {\isacharasterisk}{\kern0pt}\ {\isacharparenleft}{\kern0pt}v\ {\isasymOtimes}\ {\isacharbar}{\kern0pt}Deutsch{\isachardot}{\kern0pt}zero{\isasymrangle}{\isacharparenright}{\kern0pt}{\isacharparenright}{\kern0pt}\ {\isacharequal}{\kern0pt}\ dim{\isacharunderscore}{\kern0pt}col\ {\isacharparenleft}{\kern0pt}v\ {\isasymOtimes}\ {\isacharbar}{\kern0pt}Deutsch{\isachardot}{\kern0pt}zero{\isasymrangle}{\isacharparenright}{\kern0pt}{\isachardoublequoteclose}\isanewline
\ \ \ \ \isacommand{using}\isamarkupfalse%
\ index{\isacharunderscore}{\kern0pt}mult{\isacharunderscore}{\kern0pt}mat{\isacharparenleft}{\kern0pt}{\isadigit{3}}{\isacharparenright}{\kern0pt}\ \isacommand{by}\isamarkupfalse%
\ blast\isanewline
\isacommand{qed}\isamarkupfalse%
%
\endisatagproof
{\isafoldproof}%
%
\isadelimproof
\isanewline
%
\endisadelimproof
\ \ \ \ \ \ \ \ \isanewline
\isanewline
\isacommand{lemma}\isamarkupfalse%
\ vtensorone{\isacharunderscore}{\kern0pt}index{\isacharbrackleft}{\kern0pt}simp{\isacharbrackright}{\kern0pt}{\isacharcolon}{\kern0pt}\isanewline
\ \ \isakeyword{assumes}\ {\isachardoublequoteopen}dim{\isacharunderscore}{\kern0pt}row\ v\ {\isacharequal}{\kern0pt}\ {\isadigit{2}}{\isachardoublequoteclose}\ \isakeyword{and}\ {\isachardoublequoteopen}dim{\isacharunderscore}{\kern0pt}col\ v\ {\isacharequal}{\kern0pt}\ {\isadigit{1}}{\isachardoublequoteclose}\isanewline
\ \ \isakeyword{shows}\ {\isachardoublequoteopen}{\isacharparenleft}{\kern0pt}v\ {\isasymOtimes}\ {\isacharbar}{\kern0pt}one{\isasymrangle}{\isacharparenright}{\kern0pt}\ {\isachardollar}{\kern0pt}{\isachardollar}{\kern0pt}\ {\isacharparenleft}{\kern0pt}{\isadigit{0}}{\isacharcomma}{\kern0pt}{\isadigit{0}}{\isacharparenright}{\kern0pt}\ {\isacharequal}{\kern0pt}\ {\isadigit{0}}\ {\isasymand}\isanewline
\ \ \ \ \ \ \ \ \ {\isacharparenleft}{\kern0pt}v\ {\isasymOtimes}\ {\isacharbar}{\kern0pt}one{\isasymrangle}{\isacharparenright}{\kern0pt}\ {\isachardollar}{\kern0pt}{\isachardollar}{\kern0pt}\ {\isacharparenleft}{\kern0pt}{\isadigit{1}}{\isacharcomma}{\kern0pt}{\isadigit{0}}{\isacharparenright}{\kern0pt}\ {\isacharequal}{\kern0pt}\ v\ {\isachardollar}{\kern0pt}{\isachardollar}{\kern0pt}\ {\isacharparenleft}{\kern0pt}{\isadigit{0}}{\isacharcomma}{\kern0pt}{\isadigit{0}}{\isacharparenright}{\kern0pt}\ {\isasymand}\isanewline
\ \ \ \ \ \ \ \ \ {\isacharparenleft}{\kern0pt}v\ {\isasymOtimes}\ {\isacharbar}{\kern0pt}one{\isasymrangle}{\isacharparenright}{\kern0pt}\ {\isachardollar}{\kern0pt}{\isachardollar}{\kern0pt}\ {\isacharparenleft}{\kern0pt}{\isadigit{2}}{\isacharcomma}{\kern0pt}{\isadigit{0}}{\isacharparenright}{\kern0pt}\ {\isacharequal}{\kern0pt}\ {\isadigit{0}}\ {\isasymand}\isanewline
\ \ \ \ \ \ \ \ \ {\isacharparenleft}{\kern0pt}v\ {\isasymOtimes}\ {\isacharbar}{\kern0pt}one{\isasymrangle}{\isacharparenright}{\kern0pt}\ {\isachardollar}{\kern0pt}{\isachardollar}{\kern0pt}\ {\isacharparenleft}{\kern0pt}{\isadigit{3}}{\isacharcomma}{\kern0pt}{\isadigit{0}}{\isacharparenright}{\kern0pt}\ {\isacharequal}{\kern0pt}\ v\ {\isachardollar}{\kern0pt}{\isachardollar}{\kern0pt}\ {\isacharparenleft}{\kern0pt}{\isadigit{1}}{\isacharcomma}{\kern0pt}{\isadigit{0}}{\isacharparenright}{\kern0pt}{\isachardoublequoteclose}\isanewline
%
\isadelimproof
\ \ %
\endisadelimproof
%
\isatagproof
\isacommand{by}\isamarkupfalse%
\ {\isacharparenleft}{\kern0pt}simp\ add{\isacharcolon}{\kern0pt}\ assms{\isacharparenleft}{\kern0pt}{\isadigit{1}}{\isacharparenright}{\kern0pt}\ assms{\isacharparenleft}{\kern0pt}{\isadigit{2}}{\isacharparenright}{\kern0pt}\ ket{\isacharunderscore}{\kern0pt}vec{\isacharunderscore}{\kern0pt}def{\isacharparenright}{\kern0pt}%
\endisatagproof
{\isafoldproof}%
%
\isadelimproof
\isanewline
%
\endisadelimproof
\isanewline
\isacommand{lemma}\isamarkupfalse%
\ control{\isadigit{2}}{\isacharunderscore}{\kern0pt}one{\isacharcolon}{\kern0pt}\isanewline
\ \ \isakeyword{assumes}\ {\isachardoublequoteopen}dim{\isacharunderscore}{\kern0pt}row\ v\ {\isacharequal}{\kern0pt}\ {\isadigit{2}}{\isachardoublequoteclose}\ \isakeyword{and}\ {\isachardoublequoteopen}dim{\isacharunderscore}{\kern0pt}col\ v\ {\isacharequal}{\kern0pt}\ {\isadigit{1}}{\isachardoublequoteclose}\ \isakeyword{and}\ {\isachardoublequoteopen}dim{\isacharunderscore}{\kern0pt}row\ U\ {\isacharequal}{\kern0pt}\ {\isadigit{2}}{\isachardoublequoteclose}\ \isakeyword{and}\ {\isachardoublequoteopen}dim{\isacharunderscore}{\kern0pt}col\ U\ {\isacharequal}{\kern0pt}\ {\isadigit{2}}{\isachardoublequoteclose}\isanewline
\ \ \isakeyword{shows}\ {\isachardoublequoteopen}control{\isadigit{2}}\ U\ {\isacharasterisk}{\kern0pt}\ {\isacharparenleft}{\kern0pt}v\ {\isasymOtimes}\ {\isacharbar}{\kern0pt}one{\isasymrangle}{\isacharparenright}{\kern0pt}\ {\isacharequal}{\kern0pt}\ {\isacharparenleft}{\kern0pt}U{\isacharasterisk}{\kern0pt}v{\isacharparenright}{\kern0pt}\ {\isasymOtimes}\ {\isacharbar}{\kern0pt}one{\isasymrangle}{\isachardoublequoteclose}\isanewline
%
\isadelimproof
%
\endisadelimproof
%
\isatagproof
\isacommand{proof}\isamarkupfalse%
\isanewline
\ \ \isacommand{fix}\isamarkupfalse%
\ i\ j{\isacharcolon}{\kern0pt}{\isacharcolon}{\kern0pt}nat\isanewline
\ \ \isacommand{assume}\isamarkupfalse%
\ {\isachardoublequoteopen}i\ {\isacharless}{\kern0pt}\ dim{\isacharunderscore}{\kern0pt}row\ {\isacharparenleft}{\kern0pt}{\isacharparenleft}{\kern0pt}U{\isacharasterisk}{\kern0pt}v{\isacharparenright}{\kern0pt}\ {\isasymOtimes}\ {\isacharbar}{\kern0pt}one{\isasymrangle}{\isacharparenright}{\kern0pt}{\isachardoublequoteclose}\isanewline
\ \ \isacommand{hence}\isamarkupfalse%
\ il{\isadigit{4}}{\isacharcolon}{\kern0pt}{\isachardoublequoteopen}i\ {\isacharless}{\kern0pt}\ {\isadigit{4}}{\isachardoublequoteclose}\ \isacommand{by}\isamarkupfalse%
\ {\isacharparenleft}{\kern0pt}simp\ add{\isacharcolon}{\kern0pt}\ assms{\isacharparenleft}{\kern0pt}{\isadigit{3}}{\isacharparenright}{\kern0pt}\ ket{\isacharunderscore}{\kern0pt}vec{\isacharunderscore}{\kern0pt}def{\isacharparenright}{\kern0pt}\isanewline
\ \ \isacommand{assume}\isamarkupfalse%
\ {\isachardoublequoteopen}j\ {\isacharless}{\kern0pt}\ dim{\isacharunderscore}{\kern0pt}col\ {\isacharparenleft}{\kern0pt}{\isacharparenleft}{\kern0pt}U{\isacharasterisk}{\kern0pt}v{\isacharparenright}{\kern0pt}\ {\isasymOtimes}\ {\isacharbar}{\kern0pt}one{\isasymrangle}{\isacharparenright}{\kern0pt}{\isachardoublequoteclose}\isanewline
\ \ \isacommand{hence}\isamarkupfalse%
\ j{\isadigit{0}}{\isacharcolon}{\kern0pt}{\isachardoublequoteopen}j\ {\isacharequal}{\kern0pt}\ {\isadigit{0}}{\isachardoublequoteclose}\ \isacommand{using}\isamarkupfalse%
\ assms\ ket{\isacharunderscore}{\kern0pt}vec{\isacharunderscore}{\kern0pt}def\ \isacommand{by}\isamarkupfalse%
\ simp\isanewline
\ \ \isacommand{show}\isamarkupfalse%
\ {\isachardoublequoteopen}{\isacharparenleft}{\kern0pt}control{\isadigit{2}}\ U\ {\isacharasterisk}{\kern0pt}\ {\isacharparenleft}{\kern0pt}v\ {\isasymOtimes}\ {\isacharbar}{\kern0pt}Deutsch{\isachardot}{\kern0pt}one{\isasymrangle}{\isacharparenright}{\kern0pt}{\isacharparenright}{\kern0pt}\ {\isachardollar}{\kern0pt}{\isachardollar}{\kern0pt}\ {\isacharparenleft}{\kern0pt}i{\isacharcomma}{\kern0pt}\ j{\isacharparenright}{\kern0pt}\ {\isacharequal}{\kern0pt}\ {\isacharparenleft}{\kern0pt}U\ {\isacharasterisk}{\kern0pt}\ v\ {\isasymOtimes}\ {\isacharbar}{\kern0pt}Deutsch{\isachardot}{\kern0pt}one{\isasymrangle}{\isacharparenright}{\kern0pt}\ {\isachardollar}{\kern0pt}{\isachardollar}{\kern0pt}\ {\isacharparenleft}{\kern0pt}i{\isacharcomma}{\kern0pt}\ j{\isacharparenright}{\kern0pt}{\isachardoublequoteclose}\isanewline
\ \ \isacommand{proof}\isamarkupfalse%
\ {\isacharminus}{\kern0pt}\isanewline
\ \ \ \ \isacommand{have}\isamarkupfalse%
\ {\isachardoublequoteopen}{\isacharparenleft}{\kern0pt}control{\isadigit{2}}\ U\ {\isacharasterisk}{\kern0pt}\ {\isacharparenleft}{\kern0pt}v\ {\isasymOtimes}\ {\isacharbar}{\kern0pt}one{\isasymrangle}{\isacharparenright}{\kern0pt}{\isacharparenright}{\kern0pt}\ {\isachardollar}{\kern0pt}{\isachardollar}{\kern0pt}\ {\isacharparenleft}{\kern0pt}i{\isacharcomma}{\kern0pt}j{\isacharparenright}{\kern0pt}\ {\isacharequal}{\kern0pt}\ \isanewline
\ \ \ \ \ \ \ \ \ \ {\isacharparenleft}{\kern0pt}{\isasymSum}k{\isacharless}{\kern0pt}dim{\isacharunderscore}{\kern0pt}row\ {\isacharparenleft}{\kern0pt}v\ {\isasymOtimes}\ {\isacharbar}{\kern0pt}one{\isasymrangle}{\isacharparenright}{\kern0pt}{\isachardot}{\kern0pt}\ {\isacharparenleft}{\kern0pt}control{\isadigit{2}}\ U{\isacharparenright}{\kern0pt}\ {\isachardollar}{\kern0pt}{\isachardollar}{\kern0pt}\ {\isacharparenleft}{\kern0pt}i{\isacharcomma}{\kern0pt}\ k{\isacharparenright}{\kern0pt}\ {\isacharasterisk}{\kern0pt}\ {\isacharparenleft}{\kern0pt}v\ {\isasymOtimes}\ {\isacharbar}{\kern0pt}one{\isasymrangle}{\isacharparenright}{\kern0pt}\ {\isachardollar}{\kern0pt}{\isachardollar}{\kern0pt}\ {\isacharparenleft}{\kern0pt}k{\isacharcomma}{\kern0pt}\ j{\isacharparenright}{\kern0pt}{\isacharparenright}{\kern0pt}{\isachardoublequoteclose}\isanewline
\ \ \ \ \ \ \isacommand{using}\isamarkupfalse%
\ assms\ index{\isacharunderscore}{\kern0pt}matrix{\isacharunderscore}{\kern0pt}prod\ tensor{\isacharunderscore}{\kern0pt}carrier{\isacharunderscore}{\kern0pt}mat\isanewline
\ \ \ \ \isacommand{proof}\isamarkupfalse%
\ {\isacharminus}{\kern0pt}\isanewline
\ \ \ \ \ \ \isacommand{have}\isamarkupfalse%
\ {\isachardoublequoteopen}{\isasymAnd}m{\isachardot}{\kern0pt}\ dim{\isacharunderscore}{\kern0pt}col\ {\isacharparenleft}{\kern0pt}v\ {\isasymOtimes}\ m{\isacharparenright}{\kern0pt}\ {\isacharequal}{\kern0pt}\ dim{\isacharunderscore}{\kern0pt}col\ m{\isachardoublequoteclose}\isanewline
\ \ \ \ \ \ \ \ \isacommand{by}\isamarkupfalse%
\ {\isacharparenleft}{\kern0pt}simp\ add{\isacharcolon}{\kern0pt}\ assms{\isacharparenleft}{\kern0pt}{\isadigit{2}}{\isacharparenright}{\kern0pt}{\isacharparenright}{\kern0pt}\isanewline
\ \ \ \ \ \ \isacommand{then}\isamarkupfalse%
\ \isacommand{have}\isamarkupfalse%
\ {\isachardoublequoteopen}i\ {\isacharless}{\kern0pt}\ dim{\isacharunderscore}{\kern0pt}row\ {\isacharparenleft}{\kern0pt}control{\isadigit{2}}\ U{\isacharparenright}{\kern0pt}\ {\isasymand}\ {\isadigit{0}}\ {\isacharless}{\kern0pt}\ dim{\isacharunderscore}{\kern0pt}col\ {\isacharparenleft}{\kern0pt}v\ {\isasymOtimes}\ Matrix{\isachardot}{\kern0pt}mat\ {\isadigit{2}}\ {\isadigit{1}}\ {\isacharparenleft}{\kern0pt}{\isasymlambda}{\isacharparenleft}{\kern0pt}n{\isacharcomma}{\kern0pt}\ n{\isacharparenright}{\kern0pt}{\isachardot}{\kern0pt}\ Deutsch{\isachardot}{\kern0pt}one\ {\isachardollar}{\kern0pt}\ n{\isacharparenright}{\kern0pt}{\isacharparenright}{\kern0pt}\ {\isasymand}\ dim{\isacharunderscore}{\kern0pt}row\ {\isacharparenleft}{\kern0pt}v\ {\isasymOtimes}\ Matrix{\isachardot}{\kern0pt}mat\ {\isadigit{2}}\ {\isadigit{1}}\ {\isacharparenleft}{\kern0pt}{\isasymlambda}{\isacharparenleft}{\kern0pt}n{\isacharcomma}{\kern0pt}\ n{\isacharparenright}{\kern0pt}{\isachardot}{\kern0pt}\ Deutsch{\isachardot}{\kern0pt}one\ {\isachardollar}{\kern0pt}\ n{\isacharparenright}{\kern0pt}{\isacharparenright}{\kern0pt}\ {\isacharequal}{\kern0pt}\ dim{\isacharunderscore}{\kern0pt}col\ {\isacharparenleft}{\kern0pt}control{\isadigit{2}}\ U{\isacharparenright}{\kern0pt}{\isachardoublequoteclose}\isanewline
\ \ \ \ \ \ \ \ \isacommand{by}\isamarkupfalse%
\ {\isacharparenleft}{\kern0pt}smt\ {\isacharparenleft}{\kern0pt}z{\isadigit{3}}{\isacharparenright}{\kern0pt}\ assms{\isacharparenleft}{\kern0pt}{\isadigit{1}}{\isacharparenright}{\kern0pt}\ carrier{\isacharunderscore}{\kern0pt}matD{\isacharparenleft}{\kern0pt}{\isadigit{1}}{\isacharparenright}{\kern0pt}\ carrier{\isacharunderscore}{\kern0pt}matD{\isacharparenleft}{\kern0pt}{\isadigit{2}}{\isacharparenright}{\kern0pt}\ control{\isadigit{2}}{\isacharunderscore}{\kern0pt}carrier{\isacharunderscore}{\kern0pt}mat\ dim{\isacharunderscore}{\kern0pt}col{\isacharunderscore}{\kern0pt}mat{\isacharparenleft}{\kern0pt}{\isadigit{1}}{\isacharparenright}{\kern0pt}\ dim{\isacharunderscore}{\kern0pt}row{\isacharunderscore}{\kern0pt}mat{\isacharparenleft}{\kern0pt}{\isadigit{1}}{\isacharparenright}{\kern0pt}\ dim{\isacharunderscore}{\kern0pt}row{\isacharunderscore}{\kern0pt}tensor{\isacharunderscore}{\kern0pt}mat\ il{\isadigit{4}}\ mult{\isacharunderscore}{\kern0pt}{\isadigit{2}}\ numeral{\isacharunderscore}{\kern0pt}Bit{\isadigit{0}}\ zero{\isacharunderscore}{\kern0pt}less{\isacharunderscore}{\kern0pt}one{\isacharunderscore}{\kern0pt}class{\isachardot}{\kern0pt}zero{\isacharunderscore}{\kern0pt}less{\isacharunderscore}{\kern0pt}one{\isacharparenright}{\kern0pt}\isanewline
\ \ \ \ \ \ \isacommand{then}\isamarkupfalse%
\ \isacommand{show}\isamarkupfalse%
\ {\isacharquery}{\kern0pt}thesis\isanewline
\ \ \ \ \ \ \ \ \isacommand{by}\isamarkupfalse%
\ {\isacharparenleft}{\kern0pt}simp\ add{\isacharcolon}{\kern0pt}\ j{\isadigit{0}}\ ket{\isacharunderscore}{\kern0pt}vec{\isacharunderscore}{\kern0pt}def{\isacharparenright}{\kern0pt}\isanewline
\ \ \ \ \isacommand{qed}\isamarkupfalse%
\isanewline
\ \ \ \ \isacommand{also}\isamarkupfalse%
\ \isacommand{have}\isamarkupfalse%
\ {\isachardoublequoteopen}{\isasymdots}\ {\isacharequal}{\kern0pt}\ {\isacharparenleft}{\kern0pt}{\isasymSum}k{\isacharless}{\kern0pt}{\isadigit{4}}{\isachardot}{\kern0pt}\ control{\isadigit{2}}\ U\ {\isachardollar}{\kern0pt}{\isachardollar}{\kern0pt}\ {\isacharparenleft}{\kern0pt}i{\isacharcomma}{\kern0pt}\ k{\isacharparenright}{\kern0pt}\ {\isacharasterisk}{\kern0pt}\ {\isacharparenleft}{\kern0pt}v\ {\isasymOtimes}\ {\isacharbar}{\kern0pt}one{\isasymrangle}{\isacharparenright}{\kern0pt}\ {\isachardollar}{\kern0pt}{\isachardollar}{\kern0pt}\ {\isacharparenleft}{\kern0pt}k{\isacharcomma}{\kern0pt}\ j{\isacharparenright}{\kern0pt}{\isacharparenright}{\kern0pt}{\isachardoublequoteclose}\isanewline
\ \ \ \ \ \ \isacommand{using}\isamarkupfalse%
\ assms\ tensor{\isacharunderscore}{\kern0pt}carrier{\isacharunderscore}{\kern0pt}mat\ ket{\isacharunderscore}{\kern0pt}vec{\isacharunderscore}{\kern0pt}def\ \isacommand{by}\isamarkupfalse%
\ auto\isanewline
\ \ \ \ \isacommand{also}\isamarkupfalse%
\ \isacommand{have}\isamarkupfalse%
\ {\isachardoublequoteopen}{\isasymdots}\ {\isacharequal}{\kern0pt}\ control{\isadigit{2}}\ U\ {\isachardollar}{\kern0pt}{\isachardollar}{\kern0pt}\ {\isacharparenleft}{\kern0pt}i{\isacharcomma}{\kern0pt}\ {\isadigit{0}}{\isacharparenright}{\kern0pt}\ {\isacharasterisk}{\kern0pt}\ {\isacharparenleft}{\kern0pt}v\ {\isasymOtimes}\ {\isacharbar}{\kern0pt}one{\isasymrangle}{\isacharparenright}{\kern0pt}\ {\isachardollar}{\kern0pt}{\isachardollar}{\kern0pt}\ {\isacharparenleft}{\kern0pt}{\isadigit{0}}{\isacharcomma}{\kern0pt}\ {\isadigit{0}}{\isacharparenright}{\kern0pt}\ {\isacharplus}{\kern0pt}\isanewline
\ \ \ \ \ \ \ \ \ \ \ \ \ \ \ \ \ \ \ \ control{\isadigit{2}}\ U\ {\isachardollar}{\kern0pt}{\isachardollar}{\kern0pt}\ {\isacharparenleft}{\kern0pt}i{\isacharcomma}{\kern0pt}\ {\isadigit{1}}{\isacharparenright}{\kern0pt}\ {\isacharasterisk}{\kern0pt}\ {\isacharparenleft}{\kern0pt}v\ {\isasymOtimes}\ {\isacharbar}{\kern0pt}one{\isasymrangle}{\isacharparenright}{\kern0pt}\ {\isachardollar}{\kern0pt}{\isachardollar}{\kern0pt}\ {\isacharparenleft}{\kern0pt}{\isadigit{1}}{\isacharcomma}{\kern0pt}\ {\isadigit{0}}{\isacharparenright}{\kern0pt}\ {\isacharplus}{\kern0pt}\isanewline
\ \ \ \ \ \ \ \ \ \ \ \ \ \ \ \ \ \ \ \ control{\isadigit{2}}\ U\ {\isachardollar}{\kern0pt}{\isachardollar}{\kern0pt}\ {\isacharparenleft}{\kern0pt}i{\isacharcomma}{\kern0pt}\ {\isadigit{2}}{\isacharparenright}{\kern0pt}\ {\isacharasterisk}{\kern0pt}\ {\isacharparenleft}{\kern0pt}v\ {\isasymOtimes}\ {\isacharbar}{\kern0pt}one{\isasymrangle}{\isacharparenright}{\kern0pt}\ {\isachardollar}{\kern0pt}{\isachardollar}{\kern0pt}\ {\isacharparenleft}{\kern0pt}{\isadigit{2}}{\isacharcomma}{\kern0pt}\ {\isadigit{0}}{\isacharparenright}{\kern0pt}\ {\isacharplus}{\kern0pt}\isanewline
\ \ \ \ \ \ \ \ \ \ \ \ \ \ \ \ \ \ \ \ control{\isadigit{2}}\ U\ {\isachardollar}{\kern0pt}{\isachardollar}{\kern0pt}\ {\isacharparenleft}{\kern0pt}i{\isacharcomma}{\kern0pt}\ {\isadigit{3}}{\isacharparenright}{\kern0pt}\ {\isacharasterisk}{\kern0pt}\ {\isacharparenleft}{\kern0pt}v\ {\isasymOtimes}\ {\isacharbar}{\kern0pt}one{\isasymrangle}{\isacharparenright}{\kern0pt}\ {\isachardollar}{\kern0pt}{\isachardollar}{\kern0pt}\ {\isacharparenleft}{\kern0pt}{\isadigit{3}}{\isacharcomma}{\kern0pt}\ {\isadigit{0}}{\isacharparenright}{\kern0pt}{\isachardoublequoteclose}\isanewline
\ \ \ \ \ \ \isacommand{using}\isamarkupfalse%
\ sumof{\isadigit{4}}\ j{\isadigit{0}}\ \isacommand{by}\isamarkupfalse%
\ blast\isanewline
\ \ \ \ \isacommand{also}\isamarkupfalse%
\ \isacommand{have}\isamarkupfalse%
\ {\isachardoublequoteopen}{\isasymdots}\ {\isacharequal}{\kern0pt}\ {\isacharparenleft}{\kern0pt}{\isacharparenleft}{\kern0pt}U{\isacharasterisk}{\kern0pt}v{\isacharparenright}{\kern0pt}\ {\isasymOtimes}\ {\isacharbar}{\kern0pt}one{\isasymrangle}{\isacharparenright}{\kern0pt}\ {\isachardollar}{\kern0pt}{\isachardollar}{\kern0pt}\ {\isacharparenleft}{\kern0pt}i{\isacharcomma}{\kern0pt}{\isadigit{0}}{\isacharparenright}{\kern0pt}{\isachardoublequoteclose}\isanewline
\ \ \ \ \isacommand{proof}\isamarkupfalse%
\ {\isacharparenleft}{\kern0pt}rule\ disjE{\isacharparenright}{\kern0pt}\isanewline
\ \ \ \ \ \ \isacommand{show}\isamarkupfalse%
\ {\isachardoublequoteopen}i\ {\isacharequal}{\kern0pt}\ {\isadigit{0}}\ {\isasymor}\ i\ {\isacharequal}{\kern0pt}\ {\isadigit{1}}\ {\isasymor}\ i\ {\isacharequal}{\kern0pt}\ {\isadigit{2}}\ {\isasymor}\ i\ {\isacharequal}{\kern0pt}\ {\isadigit{3}}{\isachardoublequoteclose}\ \isacommand{using}\isamarkupfalse%
\ il{\isadigit{4}}\ \isacommand{by}\isamarkupfalse%
\ auto\isanewline
\ \ \ \ \isacommand{next}\isamarkupfalse%
\isanewline
\ \ \ \ \ \ \isacommand{assume}\isamarkupfalse%
\ i{\isadigit{0}}{\isacharcolon}{\kern0pt}{\isachardoublequoteopen}i\ {\isacharequal}{\kern0pt}\ {\isadigit{0}}{\isachardoublequoteclose}\isanewline
\ \ \ \ \ \ \isacommand{thus}\isamarkupfalse%
\ {\isachardoublequoteopen}control{\isadigit{2}}\ U\ {\isachardollar}{\kern0pt}{\isachardollar}{\kern0pt}\ {\isacharparenleft}{\kern0pt}i{\isacharcomma}{\kern0pt}\ {\isadigit{0}}{\isacharparenright}{\kern0pt}\ {\isacharasterisk}{\kern0pt}\ {\isacharparenleft}{\kern0pt}v\ {\isasymOtimes}\ {\isacharbar}{\kern0pt}Deutsch{\isachardot}{\kern0pt}one{\isasymrangle}{\isacharparenright}{\kern0pt}\ {\isachardollar}{\kern0pt}{\isachardollar}{\kern0pt}\ {\isacharparenleft}{\kern0pt}{\isadigit{0}}{\isacharcomma}{\kern0pt}\ {\isadigit{0}}{\isacharparenright}{\kern0pt}\ {\isacharplus}{\kern0pt}\isanewline
\ \ \ \ \ \ \ \ \ \ \ \ control{\isadigit{2}}\ U\ {\isachardollar}{\kern0pt}{\isachardollar}{\kern0pt}\ {\isacharparenleft}{\kern0pt}i{\isacharcomma}{\kern0pt}\ {\isadigit{1}}{\isacharparenright}{\kern0pt}\ {\isacharasterisk}{\kern0pt}\ {\isacharparenleft}{\kern0pt}v\ {\isasymOtimes}\ {\isacharbar}{\kern0pt}Deutsch{\isachardot}{\kern0pt}one{\isasymrangle}{\isacharparenright}{\kern0pt}\ {\isachardollar}{\kern0pt}{\isachardollar}{\kern0pt}\ {\isacharparenleft}{\kern0pt}{\isadigit{1}}{\isacharcomma}{\kern0pt}\ {\isadigit{0}}{\isacharparenright}{\kern0pt}\ {\isacharplus}{\kern0pt}\isanewline
\ \ \ \ \ \ \ \ \ \ \ \ control{\isadigit{2}}\ U\ {\isachardollar}{\kern0pt}{\isachardollar}{\kern0pt}\ {\isacharparenleft}{\kern0pt}i{\isacharcomma}{\kern0pt}\ {\isadigit{2}}{\isacharparenright}{\kern0pt}\ {\isacharasterisk}{\kern0pt}\ {\isacharparenleft}{\kern0pt}v\ {\isasymOtimes}\ {\isacharbar}{\kern0pt}Deutsch{\isachardot}{\kern0pt}one{\isasymrangle}{\isacharparenright}{\kern0pt}\ {\isachardollar}{\kern0pt}{\isachardollar}{\kern0pt}\ {\isacharparenleft}{\kern0pt}{\isadigit{2}}{\isacharcomma}{\kern0pt}\ {\isadigit{0}}{\isacharparenright}{\kern0pt}\ {\isacharplus}{\kern0pt}\isanewline
\ \ \ \ \ \ \ \ \ \ \ \ control{\isadigit{2}}\ U\ {\isachardollar}{\kern0pt}{\isachardollar}{\kern0pt}\ {\isacharparenleft}{\kern0pt}i{\isacharcomma}{\kern0pt}\ {\isadigit{3}}{\isacharparenright}{\kern0pt}\ {\isacharasterisk}{\kern0pt}\ {\isacharparenleft}{\kern0pt}v\ {\isasymOtimes}\ {\isacharbar}{\kern0pt}Deutsch{\isachardot}{\kern0pt}one{\isasymrangle}{\isacharparenright}{\kern0pt}\ {\isachardollar}{\kern0pt}{\isachardollar}{\kern0pt}\ {\isacharparenleft}{\kern0pt}{\isadigit{3}}{\isacharcomma}{\kern0pt}\ {\isadigit{0}}{\isacharparenright}{\kern0pt}\ {\isacharequal}{\kern0pt}\isanewline
\ \ \ \ \ \ \ \ \ \ \ \ {\isacharparenleft}{\kern0pt}U\ {\isacharasterisk}{\kern0pt}\ v\ {\isasymOtimes}\ {\isacharbar}{\kern0pt}Deutsch{\isachardot}{\kern0pt}one{\isasymrangle}{\isacharparenright}{\kern0pt}\ {\isachardollar}{\kern0pt}{\isachardollar}{\kern0pt}\ {\isacharparenleft}{\kern0pt}i{\isacharcomma}{\kern0pt}\ {\isadigit{0}}{\isacharparenright}{\kern0pt}{\isachardoublequoteclose}\isanewline
\ \ \ \ \ \ \ \ \isacommand{using}\isamarkupfalse%
\ j{\isadigit{0}}\ control{\isadigit{2}}{\isacharunderscore}{\kern0pt}def\ zero{\isacharunderscore}{\kern0pt}complex{\isachardot}{\kern0pt}code\ one{\isacharunderscore}{\kern0pt}complex{\isachardot}{\kern0pt}code\ vtensorone{\isacharunderscore}{\kern0pt}index\ assms\ \isacommand{by}\isamarkupfalse%
\ auto\isanewline
\ \ \ \ \isacommand{next}\isamarkupfalse%
\isanewline
\ \ \ \ \ \ \isacommand{assume}\isamarkupfalse%
\ id{\isadigit{3}}{\isacharcolon}{\kern0pt}{\isachardoublequoteopen}i\ {\isacharequal}{\kern0pt}\ {\isadigit{1}}\ {\isasymor}\ i\ {\isacharequal}{\kern0pt}\ {\isadigit{2}}\ {\isasymor}\ i\ {\isacharequal}{\kern0pt}\ {\isadigit{3}}{\isachardoublequoteclose}\isanewline
\ \ \ \ \ \ \isacommand{show}\isamarkupfalse%
\ {\isachardoublequoteopen}control{\isadigit{2}}\ U\ {\isachardollar}{\kern0pt}{\isachardollar}{\kern0pt}\ {\isacharparenleft}{\kern0pt}i{\isacharcomma}{\kern0pt}\ {\isadigit{0}}{\isacharparenright}{\kern0pt}\ {\isacharasterisk}{\kern0pt}\ {\isacharparenleft}{\kern0pt}v\ {\isasymOtimes}\ {\isacharbar}{\kern0pt}Deutsch{\isachardot}{\kern0pt}one{\isasymrangle}{\isacharparenright}{\kern0pt}\ {\isachardollar}{\kern0pt}{\isachardollar}{\kern0pt}\ {\isacharparenleft}{\kern0pt}{\isadigit{0}}{\isacharcomma}{\kern0pt}\ {\isadigit{0}}{\isacharparenright}{\kern0pt}\ {\isacharplus}{\kern0pt}\isanewline
\ \ \ \ \ \ \ \ \ \ \ \ control{\isadigit{2}}\ U\ {\isachardollar}{\kern0pt}{\isachardollar}{\kern0pt}\ {\isacharparenleft}{\kern0pt}i{\isacharcomma}{\kern0pt}\ {\isadigit{1}}{\isacharparenright}{\kern0pt}\ {\isacharasterisk}{\kern0pt}\ {\isacharparenleft}{\kern0pt}v\ {\isasymOtimes}\ {\isacharbar}{\kern0pt}Deutsch{\isachardot}{\kern0pt}one{\isasymrangle}{\isacharparenright}{\kern0pt}\ {\isachardollar}{\kern0pt}{\isachardollar}{\kern0pt}\ {\isacharparenleft}{\kern0pt}{\isadigit{1}}{\isacharcomma}{\kern0pt}\ {\isadigit{0}}{\isacharparenright}{\kern0pt}\ {\isacharplus}{\kern0pt}\isanewline
\ \ \ \ \ \ \ \ \ \ \ \ control{\isadigit{2}}\ U\ {\isachardollar}{\kern0pt}{\isachardollar}{\kern0pt}\ {\isacharparenleft}{\kern0pt}i{\isacharcomma}{\kern0pt}\ {\isadigit{2}}{\isacharparenright}{\kern0pt}\ {\isacharasterisk}{\kern0pt}\ {\isacharparenleft}{\kern0pt}v\ {\isasymOtimes}\ {\isacharbar}{\kern0pt}Deutsch{\isachardot}{\kern0pt}one{\isasymrangle}{\isacharparenright}{\kern0pt}\ {\isachardollar}{\kern0pt}{\isachardollar}{\kern0pt}\ {\isacharparenleft}{\kern0pt}{\isadigit{2}}{\isacharcomma}{\kern0pt}\ {\isadigit{0}}{\isacharparenright}{\kern0pt}\ {\isacharplus}{\kern0pt}\isanewline
\ \ \ \ \ \ \ \ \ \ \ \ control{\isadigit{2}}\ U\ {\isachardollar}{\kern0pt}{\isachardollar}{\kern0pt}\ {\isacharparenleft}{\kern0pt}i{\isacharcomma}{\kern0pt}\ {\isadigit{3}}{\isacharparenright}{\kern0pt}\ {\isacharasterisk}{\kern0pt}\ {\isacharparenleft}{\kern0pt}v\ {\isasymOtimes}\ {\isacharbar}{\kern0pt}Deutsch{\isachardot}{\kern0pt}one{\isasymrangle}{\isacharparenright}{\kern0pt}\ {\isachardollar}{\kern0pt}{\isachardollar}{\kern0pt}\ {\isacharparenleft}{\kern0pt}{\isadigit{3}}{\isacharcomma}{\kern0pt}\ {\isadigit{0}}{\isacharparenright}{\kern0pt}\ {\isacharequal}{\kern0pt}\isanewline
\ \ \ \ \ \ \ \ \ \ \ \ {\isacharparenleft}{\kern0pt}U\ {\isacharasterisk}{\kern0pt}\ v\ {\isasymOtimes}\ {\isacharbar}{\kern0pt}Deutsch{\isachardot}{\kern0pt}one{\isasymrangle}{\isacharparenright}{\kern0pt}\ {\isachardollar}{\kern0pt}{\isachardollar}{\kern0pt}\ {\isacharparenleft}{\kern0pt}i{\isacharcomma}{\kern0pt}\ {\isadigit{0}}{\isacharparenright}{\kern0pt}{\isachardoublequoteclose}\isanewline
\ \ \ \ \ \ \isacommand{proof}\isamarkupfalse%
\ {\isacharparenleft}{\kern0pt}rule\ disjE{\isacharparenright}{\kern0pt}\isanewline
\ \ \ \ \ \ \ \ \isacommand{show}\isamarkupfalse%
\ {\isachardoublequoteopen}i\ {\isacharequal}{\kern0pt}\ {\isadigit{1}}\ {\isasymor}\ i\ {\isacharequal}{\kern0pt}\ {\isadigit{2}}\ {\isasymor}\ i\ {\isacharequal}{\kern0pt}\ {\isadigit{3}}{\isachardoublequoteclose}\ \isacommand{using}\isamarkupfalse%
\ id{\isadigit{3}}\ \isacommand{by}\isamarkupfalse%
\ this\isanewline
\ \ \ \ \ \ \isacommand{next}\isamarkupfalse%
\isanewline
\ \ \ \ \ \ \ \ \isacommand{assume}\isamarkupfalse%
\ i{\isadigit{1}}{\isacharcolon}{\kern0pt}{\isachardoublequoteopen}i\ {\isacharequal}{\kern0pt}\ {\isadigit{1}}{\isachardoublequoteclose}\isanewline
\ \ \ \ \ \ \ \ \isacommand{thus}\isamarkupfalse%
\ {\isachardoublequoteopen}control{\isadigit{2}}\ U\ {\isachardollar}{\kern0pt}{\isachardollar}{\kern0pt}\ {\isacharparenleft}{\kern0pt}i{\isacharcomma}{\kern0pt}\ {\isadigit{0}}{\isacharparenright}{\kern0pt}\ {\isacharasterisk}{\kern0pt}\ {\isacharparenleft}{\kern0pt}v\ {\isasymOtimes}\ {\isacharbar}{\kern0pt}Deutsch{\isachardot}{\kern0pt}one{\isasymrangle}{\isacharparenright}{\kern0pt}\ {\isachardollar}{\kern0pt}{\isachardollar}{\kern0pt}\ {\isacharparenleft}{\kern0pt}{\isadigit{0}}{\isacharcomma}{\kern0pt}\ {\isadigit{0}}{\isacharparenright}{\kern0pt}\ {\isacharplus}{\kern0pt}\isanewline
\ \ \ \ \ \ \ \ \ \ \ \ control{\isadigit{2}}\ U\ {\isachardollar}{\kern0pt}{\isachardollar}{\kern0pt}\ {\isacharparenleft}{\kern0pt}i{\isacharcomma}{\kern0pt}\ {\isadigit{1}}{\isacharparenright}{\kern0pt}\ {\isacharasterisk}{\kern0pt}\ {\isacharparenleft}{\kern0pt}v\ {\isasymOtimes}\ {\isacharbar}{\kern0pt}Deutsch{\isachardot}{\kern0pt}one{\isasymrangle}{\isacharparenright}{\kern0pt}\ {\isachardollar}{\kern0pt}{\isachardollar}{\kern0pt}\ {\isacharparenleft}{\kern0pt}{\isadigit{1}}{\isacharcomma}{\kern0pt}\ {\isadigit{0}}{\isacharparenright}{\kern0pt}\ {\isacharplus}{\kern0pt}\isanewline
\ \ \ \ \ \ \ \ \ \ \ \ control{\isadigit{2}}\ U\ {\isachardollar}{\kern0pt}{\isachardollar}{\kern0pt}\ {\isacharparenleft}{\kern0pt}i{\isacharcomma}{\kern0pt}\ {\isadigit{2}}{\isacharparenright}{\kern0pt}\ {\isacharasterisk}{\kern0pt}\ {\isacharparenleft}{\kern0pt}v\ {\isasymOtimes}\ {\isacharbar}{\kern0pt}Deutsch{\isachardot}{\kern0pt}one{\isasymrangle}{\isacharparenright}{\kern0pt}\ {\isachardollar}{\kern0pt}{\isachardollar}{\kern0pt}\ {\isacharparenleft}{\kern0pt}{\isadigit{2}}{\isacharcomma}{\kern0pt}\ {\isadigit{0}}{\isacharparenright}{\kern0pt}\ {\isacharplus}{\kern0pt}\isanewline
\ \ \ \ \ \ \ \ \ \ \ \ control{\isadigit{2}}\ U\ {\isachardollar}{\kern0pt}{\isachardollar}{\kern0pt}\ {\isacharparenleft}{\kern0pt}i{\isacharcomma}{\kern0pt}\ {\isadigit{3}}{\isacharparenright}{\kern0pt}\ {\isacharasterisk}{\kern0pt}\ {\isacharparenleft}{\kern0pt}v\ {\isasymOtimes}\ {\isacharbar}{\kern0pt}Deutsch{\isachardot}{\kern0pt}one{\isasymrangle}{\isacharparenright}{\kern0pt}\ {\isachardollar}{\kern0pt}{\isachardollar}{\kern0pt}\ {\isacharparenleft}{\kern0pt}{\isadigit{3}}{\isacharcomma}{\kern0pt}\ {\isadigit{0}}{\isacharparenright}{\kern0pt}\ {\isacharequal}{\kern0pt}\isanewline
\ \ \ \ \ \ \ \ \ \ \ \ {\isacharparenleft}{\kern0pt}U\ {\isacharasterisk}{\kern0pt}\ v\ {\isasymOtimes}\ {\isacharbar}{\kern0pt}Deutsch{\isachardot}{\kern0pt}one{\isasymrangle}{\isacharparenright}{\kern0pt}\ {\isachardollar}{\kern0pt}{\isachardollar}{\kern0pt}\ {\isacharparenleft}{\kern0pt}i{\isacharcomma}{\kern0pt}\ {\isadigit{0}}{\isacharparenright}{\kern0pt}{\isachardoublequoteclose}\isanewline
\ \ \ \ \ \ \ \ \ \ \isacommand{using}\isamarkupfalse%
\ j{\isadigit{0}}\ control{\isadigit{2}}{\isacharunderscore}{\kern0pt}def\ zero{\isacharunderscore}{\kern0pt}complex{\isachardot}{\kern0pt}code\ one{\isacharunderscore}{\kern0pt}complex{\isachardot}{\kern0pt}code\ vtensorone{\isacharunderscore}{\kern0pt}index\ assms\isanewline
\ \ \ \ \ \ \ \ \ \ \isacommand{by}\isamarkupfalse%
\ {\isacharparenleft}{\kern0pt}simp\ add{\isacharcolon}{\kern0pt}\ sumof{\isadigit{2}}{\isacharparenright}{\kern0pt}\isanewline
\ \ \ \ \ \ \isacommand{next}\isamarkupfalse%
\isanewline
\ \ \ \ \ \ \ \ \isacommand{assume}\isamarkupfalse%
\ il{\isadigit{2}}{\isacharcolon}{\kern0pt}{\isachardoublequoteopen}i\ {\isacharequal}{\kern0pt}\ {\isadigit{2}}\ {\isasymor}\ i\ {\isacharequal}{\kern0pt}\ {\isadigit{3}}{\isachardoublequoteclose}\isanewline
\ \ \ \ \ \ \ \ \isacommand{show}\isamarkupfalse%
\ {\isachardoublequoteopen}control{\isadigit{2}}\ U\ {\isachardollar}{\kern0pt}{\isachardollar}{\kern0pt}\ {\isacharparenleft}{\kern0pt}i{\isacharcomma}{\kern0pt}\ {\isadigit{0}}{\isacharparenright}{\kern0pt}\ {\isacharasterisk}{\kern0pt}\ {\isacharparenleft}{\kern0pt}v\ {\isasymOtimes}\ {\isacharbar}{\kern0pt}Deutsch{\isachardot}{\kern0pt}one{\isasymrangle}{\isacharparenright}{\kern0pt}\ {\isachardollar}{\kern0pt}{\isachardollar}{\kern0pt}\ {\isacharparenleft}{\kern0pt}{\isadigit{0}}{\isacharcomma}{\kern0pt}\ {\isadigit{0}}{\isacharparenright}{\kern0pt}\ {\isacharplus}{\kern0pt}\isanewline
\ \ \ \ \ \ \ \ \ \ \ \ control{\isadigit{2}}\ U\ {\isachardollar}{\kern0pt}{\isachardollar}{\kern0pt}\ {\isacharparenleft}{\kern0pt}i{\isacharcomma}{\kern0pt}\ {\isadigit{1}}{\isacharparenright}{\kern0pt}\ {\isacharasterisk}{\kern0pt}\ {\isacharparenleft}{\kern0pt}v\ {\isasymOtimes}\ {\isacharbar}{\kern0pt}Deutsch{\isachardot}{\kern0pt}one{\isasymrangle}{\isacharparenright}{\kern0pt}\ {\isachardollar}{\kern0pt}{\isachardollar}{\kern0pt}\ {\isacharparenleft}{\kern0pt}{\isadigit{1}}{\isacharcomma}{\kern0pt}\ {\isadigit{0}}{\isacharparenright}{\kern0pt}\ {\isacharplus}{\kern0pt}\isanewline
\ \ \ \ \ \ \ \ \ \ \ \ control{\isadigit{2}}\ U\ {\isachardollar}{\kern0pt}{\isachardollar}{\kern0pt}\ {\isacharparenleft}{\kern0pt}i{\isacharcomma}{\kern0pt}\ {\isadigit{2}}{\isacharparenright}{\kern0pt}\ {\isacharasterisk}{\kern0pt}\ {\isacharparenleft}{\kern0pt}v\ {\isasymOtimes}\ {\isacharbar}{\kern0pt}Deutsch{\isachardot}{\kern0pt}one{\isasymrangle}{\isacharparenright}{\kern0pt}\ {\isachardollar}{\kern0pt}{\isachardollar}{\kern0pt}\ {\isacharparenleft}{\kern0pt}{\isadigit{2}}{\isacharcomma}{\kern0pt}\ {\isadigit{0}}{\isacharparenright}{\kern0pt}\ {\isacharplus}{\kern0pt}\isanewline
\ \ \ \ \ \ \ \ \ \ \ \ control{\isadigit{2}}\ U\ {\isachardollar}{\kern0pt}{\isachardollar}{\kern0pt}\ {\isacharparenleft}{\kern0pt}i{\isacharcomma}{\kern0pt}\ {\isadigit{3}}{\isacharparenright}{\kern0pt}\ {\isacharasterisk}{\kern0pt}\ {\isacharparenleft}{\kern0pt}v\ {\isasymOtimes}\ {\isacharbar}{\kern0pt}Deutsch{\isachardot}{\kern0pt}one{\isasymrangle}{\isacharparenright}{\kern0pt}\ {\isachardollar}{\kern0pt}{\isachardollar}{\kern0pt}\ {\isacharparenleft}{\kern0pt}{\isadigit{3}}{\isacharcomma}{\kern0pt}\ {\isadigit{0}}{\isacharparenright}{\kern0pt}\ {\isacharequal}{\kern0pt}\isanewline
\ \ \ \ \ \ \ \ \ \ \ \ {\isacharparenleft}{\kern0pt}U\ {\isacharasterisk}{\kern0pt}\ v\ {\isasymOtimes}\ {\isacharbar}{\kern0pt}Deutsch{\isachardot}{\kern0pt}one{\isasymrangle}{\isacharparenright}{\kern0pt}\ {\isachardollar}{\kern0pt}{\isachardollar}{\kern0pt}\ {\isacharparenleft}{\kern0pt}i{\isacharcomma}{\kern0pt}\ {\isadigit{0}}{\isacharparenright}{\kern0pt}{\isachardoublequoteclose}\isanewline
\ \ \ \ \ \ \ \ \isacommand{proof}\isamarkupfalse%
\ {\isacharparenleft}{\kern0pt}rule\ disjE{\isacharparenright}{\kern0pt}\isanewline
\ \ \ \ \ \ \ \ \ \ \isacommand{show}\isamarkupfalse%
\ {\isachardoublequoteopen}i\ {\isacharequal}{\kern0pt}\ {\isadigit{2}}\ {\isasymor}\ i\ {\isacharequal}{\kern0pt}\ {\isadigit{3}}{\isachardoublequoteclose}\ \isacommand{using}\isamarkupfalse%
\ il{\isadigit{2}}\ \isacommand{by}\isamarkupfalse%
\ this\isanewline
\ \ \ \ \ \ \ \ \isacommand{next}\isamarkupfalse%
\isanewline
\ \ \ \ \ \ \ \ \ \ \isacommand{assume}\isamarkupfalse%
\ i{\isadigit{2}}{\isacharcolon}{\kern0pt}{\isachardoublequoteopen}i\ {\isacharequal}{\kern0pt}\ {\isadigit{2}}{\isachardoublequoteclose}\isanewline
\ \ \ \ \ \ \ \ \ \ \isacommand{thus}\isamarkupfalse%
\ {\isachardoublequoteopen}control{\isadigit{2}}\ U\ {\isachardollar}{\kern0pt}{\isachardollar}{\kern0pt}\ {\isacharparenleft}{\kern0pt}i{\isacharcomma}{\kern0pt}\ {\isadigit{0}}{\isacharparenright}{\kern0pt}\ {\isacharasterisk}{\kern0pt}\ {\isacharparenleft}{\kern0pt}v\ {\isasymOtimes}\ {\isacharbar}{\kern0pt}Deutsch{\isachardot}{\kern0pt}one{\isasymrangle}{\isacharparenright}{\kern0pt}\ {\isachardollar}{\kern0pt}{\isachardollar}{\kern0pt}\ {\isacharparenleft}{\kern0pt}{\isadigit{0}}{\isacharcomma}{\kern0pt}\ {\isadigit{0}}{\isacharparenright}{\kern0pt}\ {\isacharplus}{\kern0pt}\isanewline
\ \ \ \ \ \ \ \ \ \ \ \ \ \ \ \ control{\isadigit{2}}\ U\ {\isachardollar}{\kern0pt}{\isachardollar}{\kern0pt}\ {\isacharparenleft}{\kern0pt}i{\isacharcomma}{\kern0pt}\ {\isadigit{1}}{\isacharparenright}{\kern0pt}\ {\isacharasterisk}{\kern0pt}\ {\isacharparenleft}{\kern0pt}v\ {\isasymOtimes}\ {\isacharbar}{\kern0pt}Deutsch{\isachardot}{\kern0pt}one{\isasymrangle}{\isacharparenright}{\kern0pt}\ {\isachardollar}{\kern0pt}{\isachardollar}{\kern0pt}\ {\isacharparenleft}{\kern0pt}{\isadigit{1}}{\isacharcomma}{\kern0pt}\ {\isadigit{0}}{\isacharparenright}{\kern0pt}\ {\isacharplus}{\kern0pt}\isanewline
\ \ \ \ \ \ \ \ \ \ \ \ \ \ \ \ control{\isadigit{2}}\ U\ {\isachardollar}{\kern0pt}{\isachardollar}{\kern0pt}\ {\isacharparenleft}{\kern0pt}i{\isacharcomma}{\kern0pt}\ {\isadigit{2}}{\isacharparenright}{\kern0pt}\ {\isacharasterisk}{\kern0pt}\ {\isacharparenleft}{\kern0pt}v\ {\isasymOtimes}\ {\isacharbar}{\kern0pt}Deutsch{\isachardot}{\kern0pt}one{\isasymrangle}{\isacharparenright}{\kern0pt}\ {\isachardollar}{\kern0pt}{\isachardollar}{\kern0pt}\ {\isacharparenleft}{\kern0pt}{\isadigit{2}}{\isacharcomma}{\kern0pt}\ {\isadigit{0}}{\isacharparenright}{\kern0pt}\ {\isacharplus}{\kern0pt}\isanewline
\ \ \ \ \ \ \ \ \ \ \ \ \ \ \ \ control{\isadigit{2}}\ U\ {\isachardollar}{\kern0pt}{\isachardollar}{\kern0pt}\ {\isacharparenleft}{\kern0pt}i{\isacharcomma}{\kern0pt}\ {\isadigit{3}}{\isacharparenright}{\kern0pt}\ {\isacharasterisk}{\kern0pt}\ {\isacharparenleft}{\kern0pt}v\ {\isasymOtimes}\ {\isacharbar}{\kern0pt}Deutsch{\isachardot}{\kern0pt}one{\isasymrangle}{\isacharparenright}{\kern0pt}\ {\isachardollar}{\kern0pt}{\isachardollar}{\kern0pt}\ {\isacharparenleft}{\kern0pt}{\isadigit{3}}{\isacharcomma}{\kern0pt}\ {\isadigit{0}}{\isacharparenright}{\kern0pt}\ {\isacharequal}{\kern0pt}\isanewline
\ \ \ \ \ \ \ \ \ \ \ \ \ \ \ \ {\isacharparenleft}{\kern0pt}U\ {\isacharasterisk}{\kern0pt}\ v\ {\isasymOtimes}\ {\isacharbar}{\kern0pt}Deutsch{\isachardot}{\kern0pt}one{\isasymrangle}{\isacharparenright}{\kern0pt}\ {\isachardollar}{\kern0pt}{\isachardollar}{\kern0pt}\ {\isacharparenleft}{\kern0pt}i{\isacharcomma}{\kern0pt}\ {\isadigit{0}}{\isacharparenright}{\kern0pt}{\isachardoublequoteclose}\isanewline
\ \ \ \ \ \ \ \ \ \ \ \ \isacommand{using}\isamarkupfalse%
\ j{\isadigit{0}}\ control{\isadigit{2}}{\isacharunderscore}{\kern0pt}def\ zero{\isacharunderscore}{\kern0pt}complex{\isachardot}{\kern0pt}code\ one{\isacharunderscore}{\kern0pt}complex{\isachardot}{\kern0pt}code\ vtensorone{\isacharunderscore}{\kern0pt}index\ assms\ \isacommand{by}\isamarkupfalse%
\ auto\isanewline
\ \ \ \ \ \ \ \ \isacommand{next}\isamarkupfalse%
\isanewline
\ \ \ \ \ \ \ \ \ \ \isacommand{assume}\isamarkupfalse%
\ i{\isadigit{3}}{\isacharcolon}{\kern0pt}{\isachardoublequoteopen}i\ {\isacharequal}{\kern0pt}\ {\isadigit{3}}{\isachardoublequoteclose}\isanewline
\ \ \ \ \ \ \ \ \ \ \isacommand{thus}\isamarkupfalse%
\ {\isachardoublequoteopen}control{\isadigit{2}}\ U\ {\isachardollar}{\kern0pt}{\isachardollar}{\kern0pt}\ {\isacharparenleft}{\kern0pt}i{\isacharcomma}{\kern0pt}\ {\isadigit{0}}{\isacharparenright}{\kern0pt}\ {\isacharasterisk}{\kern0pt}\ {\isacharparenleft}{\kern0pt}v\ {\isasymOtimes}\ {\isacharbar}{\kern0pt}Deutsch{\isachardot}{\kern0pt}one{\isasymrangle}{\isacharparenright}{\kern0pt}\ {\isachardollar}{\kern0pt}{\isachardollar}{\kern0pt}\ {\isacharparenleft}{\kern0pt}{\isadigit{0}}{\isacharcomma}{\kern0pt}\ {\isadigit{0}}{\isacharparenright}{\kern0pt}\ {\isacharplus}{\kern0pt}\isanewline
\ \ \ \ \ \ \ \ \ \ \ \ \ \ \ \ control{\isadigit{2}}\ U\ {\isachardollar}{\kern0pt}{\isachardollar}{\kern0pt}\ {\isacharparenleft}{\kern0pt}i{\isacharcomma}{\kern0pt}\ {\isadigit{1}}{\isacharparenright}{\kern0pt}\ {\isacharasterisk}{\kern0pt}\ {\isacharparenleft}{\kern0pt}v\ {\isasymOtimes}\ {\isacharbar}{\kern0pt}Deutsch{\isachardot}{\kern0pt}one{\isasymrangle}{\isacharparenright}{\kern0pt}\ {\isachardollar}{\kern0pt}{\isachardollar}{\kern0pt}\ {\isacharparenleft}{\kern0pt}{\isadigit{1}}{\isacharcomma}{\kern0pt}\ {\isadigit{0}}{\isacharparenright}{\kern0pt}\ {\isacharplus}{\kern0pt}\isanewline
\ \ \ \ \ \ \ \ \ \ \ \ \ \ \ \ control{\isadigit{2}}\ U\ {\isachardollar}{\kern0pt}{\isachardollar}{\kern0pt}\ {\isacharparenleft}{\kern0pt}i{\isacharcomma}{\kern0pt}\ {\isadigit{2}}{\isacharparenright}{\kern0pt}\ {\isacharasterisk}{\kern0pt}\ {\isacharparenleft}{\kern0pt}v\ {\isasymOtimes}\ {\isacharbar}{\kern0pt}Deutsch{\isachardot}{\kern0pt}one{\isasymrangle}{\isacharparenright}{\kern0pt}\ {\isachardollar}{\kern0pt}{\isachardollar}{\kern0pt}\ {\isacharparenleft}{\kern0pt}{\isadigit{2}}{\isacharcomma}{\kern0pt}\ {\isadigit{0}}{\isacharparenright}{\kern0pt}\ {\isacharplus}{\kern0pt}\isanewline
\ \ \ \ \ \ \ \ \ \ \ \ \ \ \ \ control{\isadigit{2}}\ U\ {\isachardollar}{\kern0pt}{\isachardollar}{\kern0pt}\ {\isacharparenleft}{\kern0pt}i{\isacharcomma}{\kern0pt}\ {\isadigit{3}}{\isacharparenright}{\kern0pt}\ {\isacharasterisk}{\kern0pt}\ {\isacharparenleft}{\kern0pt}v\ {\isasymOtimes}\ {\isacharbar}{\kern0pt}Deutsch{\isachardot}{\kern0pt}one{\isasymrangle}{\isacharparenright}{\kern0pt}\ {\isachardollar}{\kern0pt}{\isachardollar}{\kern0pt}\ {\isacharparenleft}{\kern0pt}{\isadigit{3}}{\isacharcomma}{\kern0pt}\ {\isadigit{0}}{\isacharparenright}{\kern0pt}\ {\isacharequal}{\kern0pt}\isanewline
\ \ \ \ \ \ \ \ \ \ \ \ \ \ \ \ {\isacharparenleft}{\kern0pt}U\ {\isacharasterisk}{\kern0pt}\ v\ {\isasymOtimes}\ {\isacharbar}{\kern0pt}Deutsch{\isachardot}{\kern0pt}one{\isasymrangle}{\isacharparenright}{\kern0pt}\ {\isachardollar}{\kern0pt}{\isachardollar}{\kern0pt}\ {\isacharparenleft}{\kern0pt}i{\isacharcomma}{\kern0pt}\ {\isadigit{0}}{\isacharparenright}{\kern0pt}{\isachardoublequoteclose}\isanewline
\ \ \ \ \ \ \ \ \ \ \ \ \isacommand{using}\isamarkupfalse%
\ j{\isadigit{0}}\ control{\isadigit{2}}{\isacharunderscore}{\kern0pt}def\ zero{\isacharunderscore}{\kern0pt}complex{\isachardot}{\kern0pt}code\ one{\isacharunderscore}{\kern0pt}complex{\isachardot}{\kern0pt}code\ vtensorone{\isacharunderscore}{\kern0pt}index\ assms\isanewline
\ \ \ \ \ \ \ \ \ \ \ \ \isacommand{by}\isamarkupfalse%
\ {\isacharparenleft}{\kern0pt}simp\ add{\isacharcolon}{\kern0pt}\ sumof{\isadigit{2}}{\isacharparenright}{\kern0pt}\isanewline
\ \ \ \ \ \ \ \ \isacommand{qed}\isamarkupfalse%
\isanewline
\ \ \ \ \ \ \isacommand{qed}\isamarkupfalse%
\isanewline
\ \ \ \ \isacommand{qed}\isamarkupfalse%
\isanewline
\ \ \ \ \isacommand{finally}\isamarkupfalse%
\ \isacommand{show}\isamarkupfalse%
\ {\isacharquery}{\kern0pt}thesis\ \isacommand{using}\isamarkupfalse%
\ j{\isadigit{0}}\ \isacommand{by}\isamarkupfalse%
\ simp\isanewline
\ \ \isacommand{qed}\isamarkupfalse%
\isanewline
\isacommand{next}\isamarkupfalse%
\isanewline
\ \ \isacommand{show}\isamarkupfalse%
\ {\isachardoublequoteopen}dim{\isacharunderscore}{\kern0pt}row\ {\isacharparenleft}{\kern0pt}control{\isadigit{2}}\ U\ {\isacharasterisk}{\kern0pt}\ {\isacharparenleft}{\kern0pt}v\ {\isasymOtimes}\ {\isacharbar}{\kern0pt}Deutsch{\isachardot}{\kern0pt}one{\isasymrangle}{\isacharparenright}{\kern0pt}{\isacharparenright}{\kern0pt}\ {\isacharequal}{\kern0pt}\ dim{\isacharunderscore}{\kern0pt}row\ {\isacharparenleft}{\kern0pt}U\ {\isacharasterisk}{\kern0pt}\ v\ {\isasymOtimes}\ {\isacharbar}{\kern0pt}Deutsch{\isachardot}{\kern0pt}one{\isasymrangle}{\isacharparenright}{\kern0pt}{\isachardoublequoteclose}\isanewline
\ \ \ \ \isacommand{by}\isamarkupfalse%
\ {\isacharparenleft}{\kern0pt}metis\ assms{\isacharparenleft}{\kern0pt}{\isadigit{3}}{\isacharparenright}{\kern0pt}\ carrier{\isacharunderscore}{\kern0pt}matD{\isacharparenleft}{\kern0pt}{\isadigit{1}}{\isacharparenright}{\kern0pt}\ control{\isadigit{2}}{\isacharunderscore}{\kern0pt}carrier{\isacharunderscore}{\kern0pt}mat\ dim{\isacharunderscore}{\kern0pt}row{\isacharunderscore}{\kern0pt}mat{\isacharparenleft}{\kern0pt}{\isadigit{1}}{\isacharparenright}{\kern0pt}\ dim{\isacharunderscore}{\kern0pt}row{\isacharunderscore}{\kern0pt}tensor{\isacharunderscore}{\kern0pt}mat\ \isanewline
\ \ \ \ \ \ \ \ index{\isacharunderscore}{\kern0pt}mult{\isacharunderscore}{\kern0pt}mat{\isacharparenleft}{\kern0pt}{\isadigit{2}}{\isacharparenright}{\kern0pt}\ index{\isacharunderscore}{\kern0pt}unit{\isacharunderscore}{\kern0pt}vec{\isacharparenleft}{\kern0pt}{\isadigit{3}}{\isacharparenright}{\kern0pt}\ ket{\isacharunderscore}{\kern0pt}vec{\isacharunderscore}{\kern0pt}def\ mult{\isacharunderscore}{\kern0pt}{\isadigit{2}}{\isacharunderscore}{\kern0pt}right\ numeral{\isacharunderscore}{\kern0pt}Bit{\isadigit{0}}{\isacharparenright}{\kern0pt}\isanewline
\isacommand{next}\isamarkupfalse%
\isanewline
\ \ \isacommand{show}\isamarkupfalse%
\ {\isachardoublequoteopen}dim{\isacharunderscore}{\kern0pt}col\ {\isacharparenleft}{\kern0pt}control{\isadigit{2}}\ U\ {\isacharasterisk}{\kern0pt}\ {\isacharparenleft}{\kern0pt}v\ {\isasymOtimes}\ {\isacharbar}{\kern0pt}Deutsch{\isachardot}{\kern0pt}one{\isasymrangle}{\isacharparenright}{\kern0pt}{\isacharparenright}{\kern0pt}\ {\isacharequal}{\kern0pt}\ dim{\isacharunderscore}{\kern0pt}col\ {\isacharparenleft}{\kern0pt}U\ {\isacharasterisk}{\kern0pt}\ v\ {\isasymOtimes}\ {\isacharbar}{\kern0pt}Deutsch{\isachardot}{\kern0pt}one{\isasymrangle}{\isacharparenright}{\kern0pt}{\isachardoublequoteclose}\isanewline
\ \ \ \ \isacommand{by}\isamarkupfalse%
\ simp\isanewline
\isacommand{qed}\isamarkupfalse%
%
\endisatagproof
{\isafoldproof}%
%
\isadelimproof
%
\endisadelimproof
%
\begin{isamarkuptext}%
Given a single qubit gate U, control n U creates a quantum n-qubit gate that performs
a controlled-U operation on the first qubit using the last qubit as control.%
\end{isamarkuptext}\isamarkuptrue%
\isacommand{fun}\isamarkupfalse%
\ control{\isacharcolon}{\kern0pt}{\isacharcolon}{\kern0pt}\ {\isachardoublequoteopen}nat\ {\isasymRightarrow}\ complex\ Matrix{\isachardot}{\kern0pt}mat\ {\isasymRightarrow}\ complex\ Matrix{\isachardot}{\kern0pt}mat{\isachardoublequoteclose}\ \isakeyword{where}\isanewline
\ \ {\isachardoublequoteopen}control\ {\isadigit{0}}\ U\ {\isacharequal}{\kern0pt}\ {\isadigit{1}}\isactrlsub m\ {\isadigit{1}}{\isachardoublequoteclose}\isanewline
{\isacharbar}{\kern0pt}\ {\isachardoublequoteopen}control\ {\isacharparenleft}{\kern0pt}Suc\ {\isadigit{0}}{\isacharparenright}{\kern0pt}\ U\ {\isacharequal}{\kern0pt}\ {\isadigit{1}}\isactrlsub m\ {\isadigit{2}}{\isachardoublequoteclose}\isanewline
{\isacharbar}{\kern0pt}\ {\isachardoublequoteopen}control\ {\isacharparenleft}{\kern0pt}Suc\ {\isacharparenleft}{\kern0pt}Suc\ {\isadigit{0}}{\isacharparenright}{\kern0pt}{\isacharparenright}{\kern0pt}\ U\ {\isacharequal}{\kern0pt}\ control{\isadigit{2}}\ U{\isachardoublequoteclose}\isanewline
{\isacharbar}{\kern0pt}\ {\isachardoublequoteopen}control\ {\isacharparenleft}{\kern0pt}Suc\ {\isacharparenleft}{\kern0pt}Suc\ n{\isacharparenright}{\kern0pt}{\isacharparenright}{\kern0pt}\ U\ {\isacharequal}{\kern0pt}\ \isanewline
\ \ \ {\isacharparenleft}{\kern0pt}{\isacharparenleft}{\kern0pt}{\isadigit{1}}\isactrlsub m\ {\isadigit{2}}{\isacharparenright}{\kern0pt}\ {\isasymOtimes}\ SWAP{\isacharunderscore}{\kern0pt}down\ {\isacharparenleft}{\kern0pt}Suc\ n{\isacharparenright}{\kern0pt}{\isacharparenright}{\kern0pt}\ {\isacharasterisk}{\kern0pt}\ {\isacharparenleft}{\kern0pt}control{\isadigit{2}}\ U\ {\isasymOtimes}\ {\isacharparenleft}{\kern0pt}{\isadigit{1}}\isactrlsub m\ {\isacharparenleft}{\kern0pt}{\isadigit{2}}{\isacharcircum}{\kern0pt}n{\isacharparenright}{\kern0pt}{\isacharparenright}{\kern0pt}{\isacharparenright}{\kern0pt}\ {\isacharasterisk}{\kern0pt}\ {\isacharparenleft}{\kern0pt}{\isacharparenleft}{\kern0pt}{\isadigit{1}}\isactrlsub m\ {\isadigit{2}}{\isacharparenright}{\kern0pt}\ {\isasymOtimes}\ SWAP{\isacharunderscore}{\kern0pt}up\ {\isacharparenleft}{\kern0pt}Suc\ n{\isacharparenright}{\kern0pt}{\isacharparenright}{\kern0pt}{\isachardoublequoteclose}\isanewline
\isanewline
\isacommand{lemma}\isamarkupfalse%
\ control{\isacharunderscore}{\kern0pt}carrier{\isacharunderscore}{\kern0pt}mat{\isacharbrackleft}{\kern0pt}simp{\isacharbrackright}{\kern0pt}{\isacharcolon}{\kern0pt}\isanewline
\ \ \isakeyword{shows}\ {\isachardoublequoteopen}control\ n\ U\ {\isasymin}\ carrier{\isacharunderscore}{\kern0pt}mat\ {\isacharparenleft}{\kern0pt}{\isadigit{2}}{\isacharcircum}{\kern0pt}n{\isacharparenright}{\kern0pt}\ {\isacharparenleft}{\kern0pt}{\isadigit{2}}{\isacharcircum}{\kern0pt}n{\isacharparenright}{\kern0pt}{\isachardoublequoteclose}\isanewline
%
\isadelimproof
%
\endisadelimproof
%
\isatagproof
\isacommand{proof}\isamarkupfalse%
\ {\isacharparenleft}{\kern0pt}cases\ n{\isacharparenright}{\kern0pt}\isanewline
\ \ \isacommand{case}\isamarkupfalse%
\ {\isadigit{0}}\isanewline
\ \ \isacommand{then}\isamarkupfalse%
\ \isacommand{show}\isamarkupfalse%
\ {\isacharquery}{\kern0pt}thesis\ \isacommand{by}\isamarkupfalse%
\ auto\isanewline
\isacommand{next}\isamarkupfalse%
\isanewline
\ \ \isacommand{case}\isamarkupfalse%
\ {\isacharparenleft}{\kern0pt}Suc\ nat{\isacharparenright}{\kern0pt}\isanewline
\ \ \isacommand{then}\isamarkupfalse%
\ \isacommand{show}\isamarkupfalse%
\ {\isacharquery}{\kern0pt}thesis\isanewline
\ \ \ \ \isacommand{by}\isamarkupfalse%
\ {\isacharparenleft}{\kern0pt}smt\ {\isacharparenleft}{\kern0pt}verit{\isacharcomma}{\kern0pt}\ best{\isacharparenright}{\kern0pt}\ One{\isacharunderscore}{\kern0pt}nat{\isacharunderscore}{\kern0pt}def\ SWAP{\isacharunderscore}{\kern0pt}down{\isacharunderscore}{\kern0pt}carrier{\isacharunderscore}{\kern0pt}mat\ SWAP{\isacharunderscore}{\kern0pt}up{\isachardot}{\kern0pt}simps{\isacharparenleft}{\kern0pt}{\isadigit{2}}{\isacharparenright}{\kern0pt}\ SWAP{\isacharunderscore}{\kern0pt}up{\isachardot}{\kern0pt}simps{\isacharparenleft}{\kern0pt}{\isadigit{4}}{\isacharparenright}{\kern0pt}\ \isanewline
\ \ \ \ \ \ \ \ SWAP{\isacharunderscore}{\kern0pt}up{\isacharunderscore}{\kern0pt}carrier{\isacharunderscore}{\kern0pt}mat\ Suc{\isacharunderscore}{\kern0pt}{\isadigit{1}}\ Zero{\isacharunderscore}{\kern0pt}not{\isacharunderscore}{\kern0pt}Suc\ carrier{\isacharunderscore}{\kern0pt}matD{\isacharparenleft}{\kern0pt}{\isadigit{1}}{\isacharparenright}{\kern0pt}\ carrier{\isacharunderscore}{\kern0pt}matD{\isacharparenleft}{\kern0pt}{\isadigit{2}}{\isacharparenright}{\kern0pt}\ carrier{\isacharunderscore}{\kern0pt}matI\ \isanewline
\ \ \ \ \ \ \ \ control{\isachardot}{\kern0pt}elims\ control{\isadigit{2}}{\isacharunderscore}{\kern0pt}carrier{\isacharunderscore}{\kern0pt}mat\ dim{\isacharunderscore}{\kern0pt}col{\isacharunderscore}{\kern0pt}tensor{\isacharunderscore}{\kern0pt}mat\ dim{\isacharunderscore}{\kern0pt}row{\isacharunderscore}{\kern0pt}tensor{\isacharunderscore}{\kern0pt}mat\ index{\isacharunderscore}{\kern0pt}mult{\isacharunderscore}{\kern0pt}mat{\isacharparenleft}{\kern0pt}{\isadigit{2}}{\isacharparenright}{\kern0pt}\isanewline
\ \ \ \ \ \ \ \ index{\isacharunderscore}{\kern0pt}mult{\isacharunderscore}{\kern0pt}mat{\isacharparenleft}{\kern0pt}{\isadigit{3}}{\isacharparenright}{\kern0pt}\ mult{\isacharunderscore}{\kern0pt}{\isadigit{2}}\ numeral{\isacharunderscore}{\kern0pt}Bit{\isadigit{0}}\ power{\isadigit{2}}{\isacharunderscore}{\kern0pt}eq{\isacharunderscore}{\kern0pt}square{\isacharparenright}{\kern0pt}\isanewline
\isacommand{qed}\isamarkupfalse%
%
\endisatagproof
{\isafoldproof}%
%
\isadelimproof
%
\endisadelimproof
%
\isadelimdocument
%
\endisadelimdocument
%
\isatagdocument
%
\isamarkupsection{Quantum Fourier Transform Circuit%
}
\isamarkuptrue%
%
\isamarkupsubsection{QFT definition%
}
\isamarkuptrue%
%
\endisatagdocument
{\isafolddocument}%
%
\isadelimdocument
%
\endisadelimdocument
%
\begin{isamarkuptext}%
The function kron is the generalization of the Kronecker product to a finite number of qubits%
\end{isamarkuptext}\isamarkuptrue%
\isacommand{fun}\isamarkupfalse%
\ kron{\isacharcolon}{\kern0pt}{\isacharcolon}{\kern0pt}\ {\isachardoublequoteopen}{\isacharparenleft}{\kern0pt}nat\ {\isasymRightarrow}\ complex\ Matrix{\isachardot}{\kern0pt}mat{\isacharparenright}{\kern0pt}\ {\isasymRightarrow}\ nat\ list\ {\isasymRightarrow}\ complex\ Matrix{\isachardot}{\kern0pt}mat{\isachardoublequoteclose}\ \isakeyword{where}\isanewline
\ \ {\isachardoublequoteopen}kron\ f\ {\isacharbrackleft}{\kern0pt}{\isacharbrackright}{\kern0pt}\ {\isacharequal}{\kern0pt}\ {\isadigit{1}}\isactrlsub m\ {\isadigit{1}}{\isachardoublequoteclose}\isanewline
{\isacharbar}{\kern0pt}\ {\isachardoublequoteopen}kron\ f\ {\isacharparenleft}{\kern0pt}x{\isacharhash}{\kern0pt}xs{\isacharparenright}{\kern0pt}\ {\isacharequal}{\kern0pt}\ {\isacharparenleft}{\kern0pt}f\ x{\isacharparenright}{\kern0pt}\ {\isasymOtimes}\ {\isacharparenleft}{\kern0pt}kron\ f\ xs{\isacharparenright}{\kern0pt}{\isachardoublequoteclose}\isanewline
\isanewline
\isanewline
\isacommand{lemma}\isamarkupfalse%
\ kron{\isacharunderscore}{\kern0pt}carrier{\isacharunderscore}{\kern0pt}mat{\isacharbrackleft}{\kern0pt}simp{\isacharbrackright}{\kern0pt}{\isacharcolon}{\kern0pt}\isanewline
\ \ \isakeyword{assumes}\ {\isachardoublequoteopen}{\isasymforall}m{\isachardot}{\kern0pt}\ dim{\isacharunderscore}{\kern0pt}row\ {\isacharparenleft}{\kern0pt}f\ m{\isacharparenright}{\kern0pt}\ {\isacharequal}{\kern0pt}\ {\isadigit{2}}\ {\isasymand}\ dim{\isacharunderscore}{\kern0pt}col\ {\isacharparenleft}{\kern0pt}f\ m{\isacharparenright}{\kern0pt}\ {\isacharequal}{\kern0pt}\ {\isadigit{1}}{\isachardoublequoteclose}\ \isanewline
\ \ \isakeyword{shows}\ {\isachardoublequoteopen}kron\ f\ xs\ {\isasymin}\ carrier{\isacharunderscore}{\kern0pt}mat\ {\isacharparenleft}{\kern0pt}{\isadigit{2}}{\isacharcircum}{\kern0pt}{\isacharparenleft}{\kern0pt}length\ xs{\isacharparenright}{\kern0pt}{\isacharparenright}{\kern0pt}\ {\isadigit{1}}{\isachardoublequoteclose}\isanewline
%
\isadelimproof
%
\endisadelimproof
%
\isatagproof
\isacommand{proof}\isamarkupfalse%
\ {\isacharparenleft}{\kern0pt}induct\ xs{\isacharparenright}{\kern0pt}\isanewline
\ \ \isacommand{case}\isamarkupfalse%
\ Nil\isanewline
\ \ \isacommand{show}\isamarkupfalse%
\ {\isacharquery}{\kern0pt}case\isanewline
\ \ \isacommand{proof}\isamarkupfalse%
\isanewline
\ \ \ \ \isacommand{have}\isamarkupfalse%
\ {\isachardoublequoteopen}dim{\isacharunderscore}{\kern0pt}row\ {\isacharparenleft}{\kern0pt}kron\ f\ {\isacharbrackleft}{\kern0pt}{\isacharbrackright}{\kern0pt}{\isacharparenright}{\kern0pt}\ {\isacharequal}{\kern0pt}\ dim{\isacharunderscore}{\kern0pt}row\ {\isacharparenleft}{\kern0pt}{\isadigit{1}}\isactrlsub m\ {\isadigit{1}}{\isacharparenright}{\kern0pt}{\isachardoublequoteclose}\ \isacommand{using}\isamarkupfalse%
\ kron{\isachardot}{\kern0pt}simps{\isacharparenleft}{\kern0pt}{\isadigit{1}}{\isacharparenright}{\kern0pt}\ \isacommand{by}\isamarkupfalse%
\ simp\isanewline
\ \ \ \ \isacommand{then}\isamarkupfalse%
\ \isacommand{show}\isamarkupfalse%
\ {\isachardoublequoteopen}dim{\isacharunderscore}{\kern0pt}row\ {\isacharparenleft}{\kern0pt}kron\ f\ {\isacharbrackleft}{\kern0pt}{\isacharbrackright}{\kern0pt}{\isacharparenright}{\kern0pt}\ {\isacharequal}{\kern0pt}\ {\isadigit{2}}\ {\isacharcircum}{\kern0pt}\ length\ {\isacharbrackleft}{\kern0pt}{\isacharbrackright}{\kern0pt}{\isachardoublequoteclose}\ \isacommand{by}\isamarkupfalse%
\ simp\isanewline
\ \ \isacommand{next}\isamarkupfalse%
\isanewline
\ \ \ \ \isacommand{have}\isamarkupfalse%
\ {\isachardoublequoteopen}dim{\isacharunderscore}{\kern0pt}col\ {\isacharparenleft}{\kern0pt}kron\ f\ {\isacharbrackleft}{\kern0pt}{\isacharbrackright}{\kern0pt}{\isacharparenright}{\kern0pt}\ {\isacharequal}{\kern0pt}\ dim{\isacharunderscore}{\kern0pt}col\ {\isacharparenleft}{\kern0pt}{\isadigit{1}}\isactrlsub m\ {\isadigit{1}}{\isacharparenright}{\kern0pt}{\isachardoublequoteclose}\ \isacommand{using}\isamarkupfalse%
\ kron{\isachardot}{\kern0pt}simps{\isacharparenleft}{\kern0pt}{\isadigit{1}}{\isacharparenright}{\kern0pt}\ \isacommand{by}\isamarkupfalse%
\ simp\isanewline
\ \ \ \ \isacommand{then}\isamarkupfalse%
\ \isacommand{show}\isamarkupfalse%
\ {\isachardoublequoteopen}dim{\isacharunderscore}{\kern0pt}col\ {\isacharparenleft}{\kern0pt}kron\ f\ {\isacharbrackleft}{\kern0pt}{\isacharbrackright}{\kern0pt}{\isacharparenright}{\kern0pt}\ {\isacharequal}{\kern0pt}\ {\isadigit{1}}{\isachardoublequoteclose}\ \isacommand{by}\isamarkupfalse%
\ simp\isanewline
\ \ \isacommand{qed}\isamarkupfalse%
\isanewline
\isacommand{next}\isamarkupfalse%
\isanewline
\ \ \isacommand{case}\isamarkupfalse%
\ {\isacharparenleft}{\kern0pt}Cons\ x\ xs{\isacharparenright}{\kern0pt}\isanewline
\ \ \isacommand{assume}\isamarkupfalse%
\ HI{\isacharcolon}{\kern0pt}{\isachardoublequoteopen}kron\ f\ xs\ {\isasymin}\ carrier{\isacharunderscore}{\kern0pt}mat\ {\isacharparenleft}{\kern0pt}{\isadigit{2}}\ {\isacharcircum}{\kern0pt}\ length\ xs{\isacharparenright}{\kern0pt}\ {\isadigit{1}}{\isachardoublequoteclose}\isanewline
\ \ \isacommand{have}\isamarkupfalse%
\ {\isachardoublequoteopen}f\ x\ {\isasymin}\ carrier{\isacharunderscore}{\kern0pt}mat\ {\isadigit{2}}\ {\isadigit{1}}{\isachardoublequoteclose}\ \isacommand{using}\isamarkupfalse%
\ assms\ \isacommand{by}\isamarkupfalse%
\ auto\isanewline
\ \ \isacommand{moreover}\isamarkupfalse%
\ \isacommand{have}\isamarkupfalse%
\ {\isachardoublequoteopen}{\isacharparenleft}{\kern0pt}f\ x{\isacharparenright}{\kern0pt}\ {\isasymOtimes}\ {\isacharparenleft}{\kern0pt}kron\ f\ xs{\isacharparenright}{\kern0pt}\ {\isasymin}\ carrier{\isacharunderscore}{\kern0pt}mat\ {\isacharparenleft}{\kern0pt}{\isacharparenleft}{\kern0pt}{\isadigit{2}}\ {\isacharcircum}{\kern0pt}\ length\ xs{\isacharparenright}{\kern0pt}\ {\isacharasterisk}{\kern0pt}\ {\isadigit{2}}{\isacharparenright}{\kern0pt}\ {\isadigit{1}}{\isachardoublequoteclose}\isanewline
\ \ \ \ \isacommand{using}\isamarkupfalse%
\ tensor{\isacharunderscore}{\kern0pt}carrier{\isacharunderscore}{\kern0pt}mat\ HI\ calculation\ \isacommand{by}\isamarkupfalse%
\ auto\isanewline
\ \ \isacommand{moreover}\isamarkupfalse%
\ \isacommand{have}\isamarkupfalse%
\ {\isachardoublequoteopen}kron\ f\ {\isacharparenleft}{\kern0pt}x{\isacharhash}{\kern0pt}xs{\isacharparenright}{\kern0pt}\ {\isasymin}\ carrier{\isacharunderscore}{\kern0pt}mat\ {\isacharparenleft}{\kern0pt}{\isadigit{2}}\ {\isacharcircum}{\kern0pt}\ {\isacharparenleft}{\kern0pt}length\ {\isacharparenleft}{\kern0pt}x{\isacharhash}{\kern0pt}xs{\isacharparenright}{\kern0pt}{\isacharparenright}{\kern0pt}{\isacharparenright}{\kern0pt}\ {\isadigit{1}}{\isachardoublequoteclose}\isanewline
\ \ \ \ \isacommand{using}\isamarkupfalse%
\ kron{\isachardot}{\kern0pt}simps{\isacharparenleft}{\kern0pt}{\isadigit{2}}{\isacharparenright}{\kern0pt}\ length{\isacharunderscore}{\kern0pt}Cons\ \isacommand{by}\isamarkupfalse%
\ {\isacharparenleft}{\kern0pt}metis\ calculation{\isacharparenleft}{\kern0pt}{\isadigit{2}}{\isacharparenright}{\kern0pt}\ power{\isacharunderscore}{\kern0pt}Suc{\isadigit{2}}{\isacharparenright}{\kern0pt}\isanewline
\ \ \isacommand{thus}\isamarkupfalse%
\ {\isacharquery}{\kern0pt}case\ \isacommand{by}\isamarkupfalse%
\ this\isanewline
\isacommand{qed}\isamarkupfalse%
%
\endisatagproof
{\isafoldproof}%
%
\isadelimproof
\isanewline
%
\endisadelimproof
\isanewline
\isacommand{lemma}\isamarkupfalse%
\ kron{\isacharunderscore}{\kern0pt}cons{\isacharunderscore}{\kern0pt}right{\isacharcolon}{\kern0pt}\isanewline
\ \ \isakeyword{shows}\ {\isachardoublequoteopen}kron\ f\ {\isacharparenleft}{\kern0pt}xs{\isacharat}{\kern0pt}{\isacharbrackleft}{\kern0pt}x{\isacharbrackright}{\kern0pt}{\isacharparenright}{\kern0pt}\ {\isacharequal}{\kern0pt}\ kron\ f\ xs\ {\isasymOtimes}\ f\ x{\isachardoublequoteclose}\isanewline
%
\isadelimproof
%
\endisadelimproof
%
\isatagproof
\isacommand{proof}\isamarkupfalse%
\ {\isacharparenleft}{\kern0pt}induct\ xs{\isacharparenright}{\kern0pt}\isanewline
\ \ \isacommand{case}\isamarkupfalse%
\ Nil\isanewline
\ \ \isacommand{have}\isamarkupfalse%
\ {\isachardoublequoteopen}kron\ f\ {\isacharparenleft}{\kern0pt}{\isacharbrackleft}{\kern0pt}{\isacharbrackright}{\kern0pt}{\isacharat}{\kern0pt}{\isacharbrackleft}{\kern0pt}x{\isacharbrackright}{\kern0pt}{\isacharparenright}{\kern0pt}\ {\isacharequal}{\kern0pt}\ kron\ f\ {\isacharbrackleft}{\kern0pt}x{\isacharbrackright}{\kern0pt}{\isachardoublequoteclose}\ \isacommand{by}\isamarkupfalse%
\ simp\isanewline
\ \ \isacommand{also}\isamarkupfalse%
\ \isacommand{have}\isamarkupfalse%
\ {\isachardoublequoteopen}{\isasymdots}\ {\isacharequal}{\kern0pt}\ f\ x{\isachardoublequoteclose}\ \isacommand{using}\isamarkupfalse%
\ kron{\isachardot}{\kern0pt}simps\ \isacommand{by}\isamarkupfalse%
\ auto\isanewline
\ \ \isacommand{also}\isamarkupfalse%
\ \isacommand{have}\isamarkupfalse%
\ {\isachardoublequoteopen}{\isasymdots}\ {\isacharequal}{\kern0pt}\ kron\ f\ {\isacharbrackleft}{\kern0pt}{\isacharbrackright}{\kern0pt}\ {\isasymOtimes}\ f\ x{\isachardoublequoteclose}\ \isacommand{by}\isamarkupfalse%
\ auto\isanewline
\ \ \isacommand{finally}\isamarkupfalse%
\ \isacommand{show}\isamarkupfalse%
\ {\isacharquery}{\kern0pt}case\ \isacommand{by}\isamarkupfalse%
\ this\isanewline
\isacommand{next}\isamarkupfalse%
\isanewline
\ \ \isacommand{case}\isamarkupfalse%
\ {\isacharparenleft}{\kern0pt}Cons\ a\ xs{\isacharparenright}{\kern0pt}\isanewline
\ \ \isacommand{assume}\isamarkupfalse%
\ IH{\isacharcolon}{\kern0pt}{\isachardoublequoteopen}kron\ f\ {\isacharparenleft}{\kern0pt}xs{\isacharat}{\kern0pt}{\isacharbrackleft}{\kern0pt}x{\isacharbrackright}{\kern0pt}{\isacharparenright}{\kern0pt}\ {\isacharequal}{\kern0pt}\ kron\ f\ xs\ {\isasymOtimes}\ f\ x{\isachardoublequoteclose}\isanewline
\ \ \isacommand{have}\isamarkupfalse%
\ {\isachardoublequoteopen}kron\ f\ {\isacharparenleft}{\kern0pt}{\isacharparenleft}{\kern0pt}a{\isacharhash}{\kern0pt}xs{\isacharparenright}{\kern0pt}{\isacharat}{\kern0pt}{\isacharbrackleft}{\kern0pt}x{\isacharbrackright}{\kern0pt}{\isacharparenright}{\kern0pt}\ {\isacharequal}{\kern0pt}\ f\ a\ {\isasymOtimes}\ {\isacharparenleft}{\kern0pt}kron\ f\ {\isacharparenleft}{\kern0pt}xs{\isacharat}{\kern0pt}{\isacharbrackleft}{\kern0pt}x{\isacharbrackright}{\kern0pt}{\isacharparenright}{\kern0pt}{\isacharparenright}{\kern0pt}{\isachardoublequoteclose}\ \isacommand{using}\isamarkupfalse%
\ kron{\isachardot}{\kern0pt}simps\ \isacommand{by}\isamarkupfalse%
\ auto\isanewline
\ \ \isacommand{also}\isamarkupfalse%
\ \isacommand{have}\isamarkupfalse%
\ {\isachardoublequoteopen}{\isasymdots}\ {\isacharequal}{\kern0pt}\ f\ a\ {\isasymOtimes}\ {\isacharparenleft}{\kern0pt}kron\ f\ xs\ {\isasymOtimes}\ f\ x{\isacharparenright}{\kern0pt}{\isachardoublequoteclose}\ \isacommand{using}\isamarkupfalse%
\ IH\ \isacommand{by}\isamarkupfalse%
\ simp\isanewline
\ \ \isacommand{also}\isamarkupfalse%
\ \isacommand{have}\isamarkupfalse%
\ {\isachardoublequoteopen}{\isasymdots}\ {\isacharequal}{\kern0pt}\ kron\ f\ {\isacharparenleft}{\kern0pt}a{\isacharhash}{\kern0pt}xs{\isacharparenright}{\kern0pt}\ {\isasymOtimes}\ f\ x{\isachardoublequoteclose}\ \isacommand{using}\isamarkupfalse%
\ kron{\isachardot}{\kern0pt}simps\ tensor{\isacharunderscore}{\kern0pt}mat{\isacharunderscore}{\kern0pt}is{\isacharunderscore}{\kern0pt}assoc\ \isacommand{by}\isamarkupfalse%
\ auto\isanewline
\ \ \isacommand{finally}\isamarkupfalse%
\ \isacommand{show}\isamarkupfalse%
\ {\isacharquery}{\kern0pt}case\ \isacommand{by}\isamarkupfalse%
\ this\isanewline
\isacommand{qed}\isamarkupfalse%
%
\endisatagproof
{\isafoldproof}%
%
\isadelimproof
%
\endisadelimproof
%
\begin{isamarkuptext}%
We define the QFT product representation%
\end{isamarkuptext}\isamarkuptrue%
\isacommand{definition}\isamarkupfalse%
\ QFT{\isacharunderscore}{\kern0pt}product{\isacharunderscore}{\kern0pt}representation{\isacharcolon}{\kern0pt}{\isacharcolon}{\kern0pt}\ {\isachardoublequoteopen}nat\ {\isasymRightarrow}\ nat\ {\isasymRightarrow}\ complex\ Matrix{\isachardot}{\kern0pt}mat{\isachardoublequoteclose}\ \isakeyword{where}\isanewline
\ \ {\isachardoublequoteopen}QFT{\isacharunderscore}{\kern0pt}product{\isacharunderscore}{\kern0pt}representation\ j\ n\ {\isasymequiv}\ {\isadigit{1}}{\isacharslash}{\kern0pt}{\isacharparenleft}{\kern0pt}sqrt\ {\isacharparenleft}{\kern0pt}{\isadigit{2}}{\isacharcircum}{\kern0pt}n{\isacharparenright}{\kern0pt}{\isacharparenright}{\kern0pt}\ {\isasymcdot}\isactrlsub m\ \isanewline
\ \ \ \ \ \ \ \ \ \ \ \ \ \ \ \ \ \ \ \ \ \ \ \ \ \ \ \ \ \ \ \ \ \ \ \ {\isacharparenleft}{\kern0pt}kron\ {\isacharparenleft}{\kern0pt}{\isasymlambda}{\isacharparenleft}{\kern0pt}l{\isacharcolon}{\kern0pt}{\isacharcolon}{\kern0pt}nat{\isacharparenright}{\kern0pt}{\isachardot}{\kern0pt}\ {\isacharbar}{\kern0pt}zero{\isasymrangle}\ {\isacharplus}{\kern0pt}\ exp\ {\isacharparenleft}{\kern0pt}{\isadigit{2}}{\isacharasterisk}{\kern0pt}{\isasymi}{\isacharasterisk}{\kern0pt}pi{\isacharasterisk}{\kern0pt}j{\isacharslash}{\kern0pt}{\isacharparenleft}{\kern0pt}{\isadigit{2}}{\isacharcircum}{\kern0pt}l{\isacharparenright}{\kern0pt}{\isacharparenright}{\kern0pt}\ {\isasymcdot}\isactrlsub m\ {\isacharbar}{\kern0pt}one{\isasymrangle}{\isacharparenright}{\kern0pt}\ \isanewline
\ \ \ \ \ \ \ \ \ \ \ \ \ \ \ \ \ \ \ \ \ \ \ \ \ \ \ \ \ \ \ \ \ \ \ \ {\isacharparenleft}{\kern0pt}map\ nat\ {\isacharbrackleft}{\kern0pt}{\isadigit{1}}{\isachardot}{\kern0pt}{\isachardot}{\kern0pt}n{\isacharbrackright}{\kern0pt}{\isacharparenright}{\kern0pt}{\isacharparenright}{\kern0pt}{\isachardoublequoteclose}%
\begin{isamarkuptext}%
We also define the reverse version of the QFT product representation, which is the output
state of the QFT circuit alone%
\end{isamarkuptext}\isamarkuptrue%
\isacommand{definition}\isamarkupfalse%
\ reverse{\isacharunderscore}{\kern0pt}QFT{\isacharunderscore}{\kern0pt}product{\isacharunderscore}{\kern0pt}representation{\isacharcolon}{\kern0pt}{\isacharcolon}{\kern0pt}\ {\isachardoublequoteopen}nat\ {\isasymRightarrow}\ nat\ {\isasymRightarrow}\ complex\ Matrix{\isachardot}{\kern0pt}mat{\isachardoublequoteclose}\ \isakeyword{where}\isanewline
\ \ {\isachardoublequoteopen}reverse{\isacharunderscore}{\kern0pt}QFT{\isacharunderscore}{\kern0pt}product{\isacharunderscore}{\kern0pt}representation\ j\ n\ {\isasymequiv}\ {\isadigit{1}}{\isacharslash}{\kern0pt}{\isacharparenleft}{\kern0pt}sqrt\ {\isacharparenleft}{\kern0pt}{\isadigit{2}}{\isacharcircum}{\kern0pt}n{\isacharparenright}{\kern0pt}{\isacharparenright}{\kern0pt}\ {\isasymcdot}\isactrlsub m\ \isanewline
\ \ \ \ \ \ \ \ \ \ \ \ \ \ \ \ \ \ \ \ \ \ \ \ \ \ \ \ \ \ \ \ \ \ \ \ \ \ \ \ \ \ \ \ {\isacharparenleft}{\kern0pt}kron\ {\isacharparenleft}{\kern0pt}{\isasymlambda}{\isacharparenleft}{\kern0pt}l{\isacharcolon}{\kern0pt}{\isacharcolon}{\kern0pt}nat{\isacharparenright}{\kern0pt}{\isachardot}{\kern0pt}\ {\isacharbar}{\kern0pt}zero{\isasymrangle}\ {\isacharplus}{\kern0pt}\ exp\ {\isacharparenleft}{\kern0pt}{\isadigit{2}}{\isacharasterisk}{\kern0pt}{\isasymi}{\isacharasterisk}{\kern0pt}pi{\isacharasterisk}{\kern0pt}j{\isacharslash}{\kern0pt}{\isacharparenleft}{\kern0pt}{\isadigit{2}}{\isacharcircum}{\kern0pt}l{\isacharparenright}{\kern0pt}{\isacharparenright}{\kern0pt}\ {\isasymcdot}\isactrlsub m\ {\isacharbar}{\kern0pt}one{\isasymrangle}{\isacharparenright}{\kern0pt}\ \isanewline
\ \ \ \ \ \ \ \ \ \ \ \ \ \ \ \ \ \ \ \ \ \ \ \ \ \ \ \ \ \ \ \ \ \ \ \ \ \ \ \ \ \ \ \ {\isacharparenleft}{\kern0pt}map\ nat\ {\isacharparenleft}{\kern0pt}rev\ {\isacharbrackleft}{\kern0pt}{\isadigit{1}}{\isachardot}{\kern0pt}{\isachardot}{\kern0pt}n{\isacharbrackright}{\kern0pt}{\isacharparenright}{\kern0pt}{\isacharparenright}{\kern0pt}{\isacharparenright}{\kern0pt}{\isachardoublequoteclose}%
\isadelimdocument
%
\endisadelimdocument
%
\isatagdocument
%
\isamarkupsubsection{QFT circuit%
}
\isamarkuptrue%
%
\endisatagdocument
{\isafolddocument}%
%
\isadelimdocument
%
\endisadelimdocument
%
\begin{isamarkuptext}%
The recursive function controlled$\_$rotations computes the controlled-$R_k$ gates subcircuit
 of the QFT circuit at each stage (i.e. for each qubit).%
\end{isamarkuptext}\isamarkuptrue%
\isacommand{fun}\isamarkupfalse%
\ controlled{\isacharunderscore}{\kern0pt}rotations{\isacharcolon}{\kern0pt}{\isacharcolon}{\kern0pt}\ {\isachardoublequoteopen}nat\ {\isasymRightarrow}\ complex\ Matrix{\isachardot}{\kern0pt}mat{\isachardoublequoteclose}\ \isakeyword{where}\isanewline
\ \ {\isachardoublequoteopen}controlled{\isacharunderscore}{\kern0pt}rotations\ {\isadigit{0}}\ {\isacharequal}{\kern0pt}\ {\isadigit{1}}\isactrlsub m\ {\isadigit{1}}{\isachardoublequoteclose}\isanewline
{\isacharbar}{\kern0pt}\ {\isachardoublequoteopen}controlled{\isacharunderscore}{\kern0pt}rotations\ {\isacharparenleft}{\kern0pt}Suc\ {\isadigit{0}}{\isacharparenright}{\kern0pt}\ {\isacharequal}{\kern0pt}\ {\isadigit{1}}\isactrlsub m\ {\isadigit{2}}{\isachardoublequoteclose}\isanewline
{\isacharbar}{\kern0pt}\ {\isachardoublequoteopen}controlled{\isacharunderscore}{\kern0pt}rotations\ {\isacharparenleft}{\kern0pt}Suc\ n{\isacharparenright}{\kern0pt}\ {\isacharequal}{\kern0pt}\ {\isacharparenleft}{\kern0pt}control\ {\isacharparenleft}{\kern0pt}Suc\ n{\isacharparenright}{\kern0pt}\ {\isacharparenleft}{\kern0pt}R\ {\isacharparenleft}{\kern0pt}Suc\ n{\isacharparenright}{\kern0pt}{\isacharparenright}{\kern0pt}{\isacharparenright}{\kern0pt}\ {\isacharasterisk}{\kern0pt}\isanewline
\ \ \ \ \ \ \ \ \ \ \ \ \ \ \ \ \ \ \ \ \ \ \ \ \ \ \ \ \ \ \ \ \ \ {\isacharparenleft}{\kern0pt}{\isacharparenleft}{\kern0pt}controlled{\isacharunderscore}{\kern0pt}rotations\ n{\isacharparenright}{\kern0pt}\ {\isasymOtimes}\ {\isacharparenleft}{\kern0pt}{\isadigit{1}}\isactrlsub m\ {\isadigit{2}}{\isacharparenright}{\kern0pt}{\isacharparenright}{\kern0pt}{\isachardoublequoteclose}\isanewline
\isanewline
\isanewline
\isacommand{lemma}\isamarkupfalse%
\ controlled{\isacharunderscore}{\kern0pt}rotations{\isacharunderscore}{\kern0pt}carrier{\isacharunderscore}{\kern0pt}mat{\isacharbrackleft}{\kern0pt}simp{\isacharbrackright}{\kern0pt}{\isacharcolon}{\kern0pt}\isanewline
\ \ {\isachardoublequoteopen}controlled{\isacharunderscore}{\kern0pt}rotations\ n\ {\isasymin}\ carrier{\isacharunderscore}{\kern0pt}mat\ {\isacharparenleft}{\kern0pt}{\isadigit{2}}{\isacharcircum}{\kern0pt}n{\isacharparenright}{\kern0pt}\ {\isacharparenleft}{\kern0pt}{\isadigit{2}}{\isacharcircum}{\kern0pt}n{\isacharparenright}{\kern0pt}{\isachardoublequoteclose}\isanewline
%
\isadelimproof
%
\endisadelimproof
%
\isatagproof
\isacommand{proof}\isamarkupfalse%
\ {\isacharparenleft}{\kern0pt}induct\ n\ rule{\isacharcolon}{\kern0pt}\ controlled{\isacharunderscore}{\kern0pt}rotations{\isachardot}{\kern0pt}induct{\isacharparenright}{\kern0pt}\isanewline
\ \ \isacommand{case}\isamarkupfalse%
\ {\isadigit{1}}\isanewline
\ \ \isacommand{then}\isamarkupfalse%
\ \isacommand{show}\isamarkupfalse%
\ {\isacharquery}{\kern0pt}case\ \isacommand{by}\isamarkupfalse%
\ auto\isanewline
\isacommand{next}\isamarkupfalse%
\isanewline
\ \ \isacommand{case}\isamarkupfalse%
\ {\isadigit{2}}\ \isanewline
\ \ \isacommand{then}\isamarkupfalse%
\ \isacommand{show}\isamarkupfalse%
\ {\isacharquery}{\kern0pt}case\ \isacommand{by}\isamarkupfalse%
\ auto\isanewline
\isacommand{next}\isamarkupfalse%
\isanewline
\ \ \isacommand{case}\isamarkupfalse%
\ {\isadigit{3}}\isanewline
\ \ \isacommand{then}\isamarkupfalse%
\ \isacommand{show}\isamarkupfalse%
\ {\isacharquery}{\kern0pt}case\ \isanewline
\ \ \ \ \isacommand{by}\isamarkupfalse%
\ {\isacharparenleft}{\kern0pt}smt\ {\isacharparenleft}{\kern0pt}verit{\isacharcomma}{\kern0pt}\ del{\isacharunderscore}{\kern0pt}insts{\isacharparenright}{\kern0pt}\ carrier{\isacharunderscore}{\kern0pt}matD{\isacharparenleft}{\kern0pt}{\isadigit{1}}{\isacharparenright}{\kern0pt}\ carrier{\isacharunderscore}{\kern0pt}matD{\isacharparenleft}{\kern0pt}{\isadigit{2}}{\isacharparenright}{\kern0pt}\ carrier{\isacharunderscore}{\kern0pt}mat{\isacharunderscore}{\kern0pt}triv\ control{\isacharunderscore}{\kern0pt}carrier{\isacharunderscore}{\kern0pt}mat\isanewline
\ \ \ \ \ \ \ \ controlled{\isacharunderscore}{\kern0pt}rotations{\isachardot}{\kern0pt}simps{\isacharparenleft}{\kern0pt}{\isadigit{3}}{\isacharparenright}{\kern0pt}\ dim{\isacharunderscore}{\kern0pt}col{\isacharunderscore}{\kern0pt}tensor{\isacharunderscore}{\kern0pt}mat\ index{\isacharunderscore}{\kern0pt}mult{\isacharunderscore}{\kern0pt}mat{\isacharparenleft}{\kern0pt}{\isadigit{2}}{\isacharparenright}{\kern0pt}\ index{\isacharunderscore}{\kern0pt}mult{\isacharunderscore}{\kern0pt}mat{\isacharparenleft}{\kern0pt}{\isadigit{3}}{\isacharparenright}{\kern0pt}\isanewline
\ \ \ \ \ \ \ \ index{\isacharunderscore}{\kern0pt}one{\isacharunderscore}{\kern0pt}mat{\isacharparenleft}{\kern0pt}{\isadigit{3}}{\isacharparenright}{\kern0pt}\ mult{\isachardot}{\kern0pt}commute\ power{\isacharunderscore}{\kern0pt}Suc{\isacharparenright}{\kern0pt}\isanewline
\isacommand{qed}\isamarkupfalse%
%
\endisatagproof
{\isafoldproof}%
%
\isadelimproof
%
\endisadelimproof
%
\begin{isamarkuptext}%
The recursive function QFT computes the Quantum Fourier Transform circuit.%
\end{isamarkuptext}\isamarkuptrue%
\isacommand{fun}\isamarkupfalse%
\ QFT{\isacharcolon}{\kern0pt}{\isacharcolon}{\kern0pt}\ {\isachardoublequoteopen}nat\ {\isasymRightarrow}\ complex\ Matrix{\isachardot}{\kern0pt}mat{\isachardoublequoteclose}\ \isakeyword{where}\isanewline
\ \ {\isachardoublequoteopen}QFT\ {\isadigit{0}}\ {\isacharequal}{\kern0pt}\ {\isadigit{1}}\isactrlsub m\ {\isadigit{1}}{\isachardoublequoteclose}\isanewline
{\isacharbar}{\kern0pt}\ {\isachardoublequoteopen}QFT\ {\isacharparenleft}{\kern0pt}Suc\ {\isadigit{0}}{\isacharparenright}{\kern0pt}\ {\isacharequal}{\kern0pt}\ H{\isachardoublequoteclose}\isanewline
{\isacharbar}{\kern0pt}\ {\isachardoublequoteopen}QFT\ {\isacharparenleft}{\kern0pt}Suc\ n{\isacharparenright}{\kern0pt}\ {\isacharequal}{\kern0pt}\ \ {\isacharparenleft}{\kern0pt}{\isacharparenleft}{\kern0pt}{\isadigit{1}}\isactrlsub m\ {\isadigit{2}}{\isacharparenright}{\kern0pt}\ {\isasymOtimes}\ {\isacharparenleft}{\kern0pt}QFT\ n{\isacharparenright}{\kern0pt}{\isacharparenright}{\kern0pt}\ {\isacharasterisk}{\kern0pt}\ {\isacharparenleft}{\kern0pt}controlled{\isacharunderscore}{\kern0pt}rotations\ {\isacharparenleft}{\kern0pt}Suc\ n{\isacharparenright}{\kern0pt}{\isacharparenright}{\kern0pt}\ {\isacharasterisk}{\kern0pt}\ {\isacharparenleft}{\kern0pt}H\ {\isasymOtimes}\ {\isacharparenleft}{\kern0pt}{\isacharparenleft}{\kern0pt}{\isadigit{1}}\isactrlsub m\ {\isacharparenleft}{\kern0pt}{\isadigit{2}}{\isacharcircum}{\kern0pt}n{\isacharparenright}{\kern0pt}{\isacharparenright}{\kern0pt}{\isacharparenright}{\kern0pt}{\isacharparenright}{\kern0pt}{\isachardoublequoteclose}\isanewline
\isanewline
\isanewline
\isacommand{lemma}\isamarkupfalse%
\ QFT{\isacharunderscore}{\kern0pt}carrier{\isacharunderscore}{\kern0pt}mat{\isacharbrackleft}{\kern0pt}simp{\isacharbrackright}{\kern0pt}{\isacharcolon}{\kern0pt}\isanewline
\ \ {\isachardoublequoteopen}QFT\ n\ {\isasymin}\ carrier{\isacharunderscore}{\kern0pt}mat\ {\isacharparenleft}{\kern0pt}{\isadigit{2}}{\isacharcircum}{\kern0pt}n{\isacharparenright}{\kern0pt}\ {\isacharparenleft}{\kern0pt}{\isadigit{2}}{\isacharcircum}{\kern0pt}n{\isacharparenright}{\kern0pt}{\isachardoublequoteclose}\isanewline
%
\isadelimproof
%
\endisadelimproof
%
\isatagproof
\isacommand{proof}\isamarkupfalse%
\ {\isacharparenleft}{\kern0pt}induct\ n\ rule{\isacharcolon}{\kern0pt}\ QFT{\isachardot}{\kern0pt}induct{\isacharparenright}{\kern0pt}\isanewline
\ \ \isacommand{case}\isamarkupfalse%
\ {\isadigit{1}}\isanewline
\ \ \isacommand{then}\isamarkupfalse%
\ \isacommand{show}\isamarkupfalse%
\ {\isacharquery}{\kern0pt}case\ \isacommand{by}\isamarkupfalse%
\ auto\isanewline
\isacommand{next}\isamarkupfalse%
\isanewline
\ \ \isacommand{case}\isamarkupfalse%
\ {\isadigit{2}}\isanewline
\ \ \isacommand{then}\isamarkupfalse%
\ \isacommand{show}\isamarkupfalse%
\ {\isacharquery}{\kern0pt}case\isanewline
\ \ \ \ \isacommand{using}\isamarkupfalse%
\ H{\isacharunderscore}{\kern0pt}is{\isacharunderscore}{\kern0pt}gate\ One{\isacharunderscore}{\kern0pt}nat{\isacharunderscore}{\kern0pt}def\ QFT{\isachardot}{\kern0pt}simps{\isacharparenleft}{\kern0pt}{\isadigit{2}}{\isacharparenright}{\kern0pt}\ gate{\isacharunderscore}{\kern0pt}carrier{\isacharunderscore}{\kern0pt}mat\ \isacommand{by}\isamarkupfalse%
\ presburger\isanewline
\isacommand{next}\isamarkupfalse%
\isanewline
\ \ \isacommand{case}\isamarkupfalse%
\ {\isadigit{3}}\isanewline
\ \ \isacommand{then}\isamarkupfalse%
\ \isacommand{show}\isamarkupfalse%
\ {\isacharquery}{\kern0pt}case\isanewline
\ \ \ \ \isacommand{by}\isamarkupfalse%
\ {\isacharparenleft}{\kern0pt}metis\ H{\isacharunderscore}{\kern0pt}inv\ QFT{\isachardot}{\kern0pt}simps{\isacharparenleft}{\kern0pt}{\isadigit{3}}{\isacharparenright}{\kern0pt}\ carrier{\isacharunderscore}{\kern0pt}matD{\isacharparenleft}{\kern0pt}{\isadigit{1}}{\isacharparenright}{\kern0pt}\ carrier{\isacharunderscore}{\kern0pt}mat{\isacharunderscore}{\kern0pt}triv\ dim{\isacharunderscore}{\kern0pt}col{\isacharunderscore}{\kern0pt}tensor{\isacharunderscore}{\kern0pt}mat\isanewline
\ \ \ \ \ \ \ \ dim{\isacharunderscore}{\kern0pt}row{\isacharunderscore}{\kern0pt}tensor{\isacharunderscore}{\kern0pt}mat\ index{\isacharunderscore}{\kern0pt}mult{\isacharunderscore}{\kern0pt}mat{\isacharparenleft}{\kern0pt}{\isadigit{2}}{\isacharparenright}{\kern0pt}\ index{\isacharunderscore}{\kern0pt}mult{\isacharunderscore}{\kern0pt}mat{\isacharparenleft}{\kern0pt}{\isadigit{3}}{\isacharparenright}{\kern0pt}\ index{\isacharunderscore}{\kern0pt}one{\isacharunderscore}{\kern0pt}mat{\isacharparenleft}{\kern0pt}{\isadigit{2}}{\isacharparenright}{\kern0pt}\ index{\isacharunderscore}{\kern0pt}one{\isacharunderscore}{\kern0pt}mat{\isacharparenleft}{\kern0pt}{\isadigit{3}}{\isacharparenright}{\kern0pt}\ \isanewline
\ \ \ \ \ \ \ \ power{\isachardot}{\kern0pt}simps{\isacharparenleft}{\kern0pt}{\isadigit{2}}{\isacharparenright}{\kern0pt}{\isacharparenright}{\kern0pt}\isanewline
\isacommand{qed}\isamarkupfalse%
%
\endisatagproof
{\isafoldproof}%
%
\isadelimproof
%
\endisadelimproof
%
\begin{isamarkuptext}%
ordered$\_$QFT reverses the order of the qubits at the end of the QFT circuit%
\end{isamarkuptext}\isamarkuptrue%
\isacommand{definition}\isamarkupfalse%
\ ordered{\isacharunderscore}{\kern0pt}QFT{\isacharcolon}{\kern0pt}{\isacharcolon}{\kern0pt}\ {\isachardoublequoteopen}nat\ {\isasymRightarrow}\ complex\ Matrix{\isachardot}{\kern0pt}mat{\isachardoublequoteclose}\ \isakeyword{where}\isanewline
\ \ {\isachardoublequoteopen}ordered{\isacharunderscore}{\kern0pt}QFT\ n\ {\isasymequiv}\ {\isacharparenleft}{\kern0pt}reverse{\isacharunderscore}{\kern0pt}qubits\ n{\isacharparenright}{\kern0pt}\ {\isacharasterisk}{\kern0pt}\ {\isacharparenleft}{\kern0pt}QFT\ n{\isacharparenright}{\kern0pt}{\isachardoublequoteclose}%
\isadelimdocument
%
\endisadelimdocument
%
\isatagdocument
%
\isamarkupsection{QFT circuit correctness%
}
\isamarkuptrue%
%
\endisatagdocument
{\isafolddocument}%
%
\isadelimdocument
%
\endisadelimdocument
%
\begin{isamarkuptext}%
Some useful lemmas:%
\end{isamarkuptext}\isamarkuptrue%
\isacommand{lemma}\isamarkupfalse%
\ state{\isacharunderscore}{\kern0pt}basis{\isacharunderscore}{\kern0pt}dec{\isacharcolon}{\kern0pt}\isanewline
\ \ \isakeyword{assumes}\ {\isachardoublequoteopen}j\ {\isacharless}{\kern0pt}\ {\isadigit{2}}\ {\isacharcircum}{\kern0pt}\ Suc\ n{\isachardoublequoteclose}\isanewline
\ \ \isakeyword{shows}\ {\isachardoublequoteopen}{\isacharbar}{\kern0pt}state{\isacharunderscore}{\kern0pt}basis\ {\isadigit{1}}\ {\isacharparenleft}{\kern0pt}j\ div\ {\isadigit{2}}{\isacharcircum}{\kern0pt}n{\isacharparenright}{\kern0pt}{\isasymrangle}\ {\isasymOtimes}\ {\isacharbar}{\kern0pt}state{\isacharunderscore}{\kern0pt}basis\ n\ {\isacharparenleft}{\kern0pt}j\ mod\ {\isadigit{2}}{\isacharcircum}{\kern0pt}n{\isacharparenright}{\kern0pt}{\isasymrangle}\ {\isacharequal}{\kern0pt}\ {\isacharbar}{\kern0pt}state{\isacharunderscore}{\kern0pt}basis\ {\isacharparenleft}{\kern0pt}Suc\ n{\isacharparenright}{\kern0pt}\ j{\isasymrangle}{\isachardoublequoteclose}\isanewline
%
\isadelimproof
%
\endisadelimproof
%
\isatagproof
\isacommand{proof}\isamarkupfalse%
\ {\isacharminus}{\kern0pt}\isanewline
\ \ \isacommand{define}\isamarkupfalse%
\ jd\ jm\ \isakeyword{where}\ {\isachardoublequoteopen}jd\ {\isacharequal}{\kern0pt}\ j\ div\ {\isadigit{2}}{\isacharcircum}{\kern0pt}n{\isachardoublequoteclose}\ \isakeyword{and}\ {\isachardoublequoteopen}jm\ {\isacharequal}{\kern0pt}\ j\ mod\ {\isadigit{2}}{\isacharcircum}{\kern0pt}n{\isachardoublequoteclose}\isanewline
\ \ \isacommand{hence}\isamarkupfalse%
\ jml{\isacharcolon}{\kern0pt}{\isachardoublequoteopen}jm\ {\isacharless}{\kern0pt}\ {\isadigit{2}}{\isacharcircum}{\kern0pt}n{\isachardoublequoteclose}\ \isacommand{by}\isamarkupfalse%
\ auto\isanewline
\ \ \isacommand{have}\isamarkupfalse%
\ j{\isacharunderscore}{\kern0pt}dec{\isacharcolon}{\kern0pt}{\isachardoublequoteopen}j\ {\isacharequal}{\kern0pt}\ jd{\isacharasterisk}{\kern0pt}{\isacharparenleft}{\kern0pt}{\isadigit{2}}{\isacharcircum}{\kern0pt}n{\isacharparenright}{\kern0pt}\ {\isacharplus}{\kern0pt}\ jm{\isachardoublequoteclose}\ \isacommand{using}\isamarkupfalse%
\ jd{\isacharunderscore}{\kern0pt}def\ jm{\isacharunderscore}{\kern0pt}def\ \isacommand{by}\isamarkupfalse%
\ presburger\isanewline
\ \ \isacommand{show}\isamarkupfalse%
\ {\isacharquery}{\kern0pt}thesis\isanewline
\ \ \isacommand{proof}\isamarkupfalse%
\ {\isacharparenleft}{\kern0pt}rule\ disjE{\isacharparenright}{\kern0pt}\isanewline
\ \ \ \ \isacommand{show}\isamarkupfalse%
\ {\isachardoublequoteopen}jd\ {\isacharequal}{\kern0pt}\ {\isadigit{0}}\ {\isasymor}\ jd\ {\isacharequal}{\kern0pt}\ {\isadigit{1}}{\isachardoublequoteclose}\ \isacommand{using}\isamarkupfalse%
\ jd{\isacharunderscore}{\kern0pt}def\ assms\isanewline
\ \ \ \ \ \ \isacommand{by}\isamarkupfalse%
\ {\isacharparenleft}{\kern0pt}metis\ One{\isacharunderscore}{\kern0pt}nat{\isacharunderscore}{\kern0pt}def\ less{\isacharunderscore}{\kern0pt}{\isadigit{2}}{\isacharunderscore}{\kern0pt}cases\ less{\isacharunderscore}{\kern0pt}power{\isacharunderscore}{\kern0pt}add{\isacharunderscore}{\kern0pt}imp{\isacharunderscore}{\kern0pt}div{\isacharunderscore}{\kern0pt}less\ plus{\isacharunderscore}{\kern0pt}{\isadigit{1}}{\isacharunderscore}{\kern0pt}eq{\isacharunderscore}{\kern0pt}Suc\ power{\isacharunderscore}{\kern0pt}one{\isacharunderscore}{\kern0pt}right{\isacharparenright}{\kern0pt}\isanewline
\ \ \isacommand{next}\isamarkupfalse%
\isanewline
\ \ \ \ \isacommand{assume}\isamarkupfalse%
\ jd{\isadigit{0}}{\isacharcolon}{\kern0pt}{\isachardoublequoteopen}jd\ {\isacharequal}{\kern0pt}\ {\isadigit{0}}{\isachardoublequoteclose}\isanewline
\ \ \ \ \isacommand{hence}\isamarkupfalse%
\ jjm{\isacharcolon}{\kern0pt}{\isachardoublequoteopen}j\ {\isacharequal}{\kern0pt}\ jm{\isachardoublequoteclose}\ \isacommand{using}\isamarkupfalse%
\ j{\isacharunderscore}{\kern0pt}dec\ \isacommand{by}\isamarkupfalse%
\ auto\isanewline
\ \ \ \ \isacommand{show}\isamarkupfalse%
\ {\isachardoublequoteopen}{\isacharbar}{\kern0pt}state{\isacharunderscore}{\kern0pt}basis\ {\isadigit{1}}\ {\isacharparenleft}{\kern0pt}j\ div\ {\isadigit{2}}{\isacharcircum}{\kern0pt}n{\isacharparenright}{\kern0pt}{\isasymrangle}\ {\isasymOtimes}\ {\isacharbar}{\kern0pt}state{\isacharunderscore}{\kern0pt}basis\ n\ {\isacharparenleft}{\kern0pt}j\ mod\ {\isadigit{2}}{\isacharcircum}{\kern0pt}n{\isacharparenright}{\kern0pt}{\isasymrangle}\ {\isacharequal}{\kern0pt}\ {\isacharbar}{\kern0pt}state{\isacharunderscore}{\kern0pt}basis\ {\isacharparenleft}{\kern0pt}Suc\ n{\isacharparenright}{\kern0pt}\ j{\isasymrangle}{\isachardoublequoteclose}\isanewline
\ \ \ \ \isacommand{proof}\isamarkupfalse%
\isanewline
\ \ \ \ \ \ \isacommand{fix}\isamarkupfalse%
\ i\ ja\isanewline
\ \ \ \ \ \ \isacommand{assume}\isamarkupfalse%
\ {\isachardoublequoteopen}i\ {\isacharless}{\kern0pt}\ dim{\isacharunderscore}{\kern0pt}row\ {\isacharparenleft}{\kern0pt}\ {\isacharbar}{\kern0pt}state{\isacharunderscore}{\kern0pt}basis\ {\isacharparenleft}{\kern0pt}Suc\ n{\isacharparenright}{\kern0pt}\ j{\isasymrangle}{\isacharparenright}{\kern0pt}{\isachardoublequoteclose}\isanewline
\ \ \ \ \ \ \ \ \ \isakeyword{and}\ ja{\isacharunderscore}{\kern0pt}dim{\isacharcolon}{\kern0pt}{\isachardoublequoteopen}ja\ {\isacharless}{\kern0pt}\ dim{\isacharunderscore}{\kern0pt}col\ {\isacharparenleft}{\kern0pt}\ {\isacharbar}{\kern0pt}state{\isacharunderscore}{\kern0pt}basis\ {\isacharparenleft}{\kern0pt}Suc\ n{\isacharparenright}{\kern0pt}\ j{\isasymrangle}{\isacharparenright}{\kern0pt}{\isachardoublequoteclose}\isanewline
\ \ \ \ \ \ \isacommand{hence}\isamarkupfalse%
\ il{\isacharcolon}{\kern0pt}{\isachardoublequoteopen}i\ {\isacharless}{\kern0pt}\ {\isadigit{2}}{\isacharcircum}{\kern0pt}Suc\ n{\isachardoublequoteclose}\ \isacommand{using}\isamarkupfalse%
\ state{\isacharunderscore}{\kern0pt}basis{\isacharunderscore}{\kern0pt}carrier{\isacharunderscore}{\kern0pt}mat\ ket{\isacharunderscore}{\kern0pt}vec{\isacharunderscore}{\kern0pt}def\ state{\isacharunderscore}{\kern0pt}basis{\isacharunderscore}{\kern0pt}def\ \isacommand{by}\isamarkupfalse%
\ simp\isanewline
\ \ \ \ \ \ \isacommand{have}\isamarkupfalse%
\ jal{\isacharcolon}{\kern0pt}{\isachardoublequoteopen}ja\ {\isacharless}{\kern0pt}\ {\isadigit{1}}{\isachardoublequoteclose}\ \isacommand{using}\isamarkupfalse%
\ ja{\isacharunderscore}{\kern0pt}dim\ state{\isacharunderscore}{\kern0pt}basis{\isacharunderscore}{\kern0pt}carrier{\isacharunderscore}{\kern0pt}mat\ state{\isacharunderscore}{\kern0pt}basis{\isacharunderscore}{\kern0pt}def\ ket{\isacharunderscore}{\kern0pt}vec{\isacharunderscore}{\kern0pt}def\ \isacommand{by}\isamarkupfalse%
\ simp\isanewline
\ \ \ \ \ \ \isacommand{hence}\isamarkupfalse%
\ ja{\isadigit{0}}{\isacharcolon}{\kern0pt}{\isachardoublequoteopen}ja\ {\isacharequal}{\kern0pt}\ {\isadigit{0}}{\isachardoublequoteclose}\ \isacommand{by}\isamarkupfalse%
\ auto\isanewline
\ \ \ \ \ \ \isacommand{show}\isamarkupfalse%
\ {\isachardoublequoteopen}{\isacharparenleft}{\kern0pt}\ {\isacharbar}{\kern0pt}state{\isacharunderscore}{\kern0pt}basis\ {\isadigit{1}}\ {\isacharparenleft}{\kern0pt}j\ div\ {\isadigit{2}}\ {\isacharcircum}{\kern0pt}\ n{\isacharparenright}{\kern0pt}{\isasymrangle}\ {\isasymOtimes}\ {\isacharbar}{\kern0pt}state{\isacharunderscore}{\kern0pt}basis\ n\ {\isacharparenleft}{\kern0pt}j\ mod\ {\isadigit{2}}\ {\isacharcircum}{\kern0pt}\ n{\isacharparenright}{\kern0pt}{\isasymrangle}{\isacharparenright}{\kern0pt}\ {\isachardollar}{\kern0pt}{\isachardollar}{\kern0pt}\ {\isacharparenleft}{\kern0pt}i{\isacharcomma}{\kern0pt}\ ja{\isacharparenright}{\kern0pt}\ {\isacharequal}{\kern0pt}\isanewline
\ \ \ \ \ \ \ \ \ \ \ \ \ \ {\isacharbar}{\kern0pt}state{\isacharunderscore}{\kern0pt}basis\ {\isacharparenleft}{\kern0pt}Suc\ n{\isacharparenright}{\kern0pt}\ j{\isasymrangle}\ {\isachardollar}{\kern0pt}{\isachardollar}{\kern0pt}\ {\isacharparenleft}{\kern0pt}i{\isacharcomma}{\kern0pt}\ ja{\isacharparenright}{\kern0pt}{\isachardoublequoteclose}\isanewline
\ \ \ \ \ \ \isacommand{proof}\isamarkupfalse%
\ {\isacharminus}{\kern0pt}\isanewline
\ \ \ \ \ \ \ \ \isacommand{have}\isamarkupfalse%
\ {\isachardoublequoteopen}{\isacharparenleft}{\kern0pt}\ {\isacharbar}{\kern0pt}state{\isacharunderscore}{\kern0pt}basis\ {\isadigit{1}}\ {\isacharparenleft}{\kern0pt}j\ div\ {\isadigit{2}}\ {\isacharcircum}{\kern0pt}\ n{\isacharparenright}{\kern0pt}{\isasymrangle}\ {\isasymOtimes}\ {\isacharbar}{\kern0pt}state{\isacharunderscore}{\kern0pt}basis\ n\ {\isacharparenleft}{\kern0pt}j\ mod\ {\isadigit{2}}\ {\isacharcircum}{\kern0pt}\ n{\isacharparenright}{\kern0pt}{\isasymrangle}{\isacharparenright}{\kern0pt}\ {\isachardollar}{\kern0pt}{\isachardollar}{\kern0pt}\ {\isacharparenleft}{\kern0pt}i{\isacharcomma}{\kern0pt}\ ja{\isacharparenright}{\kern0pt}\ {\isacharequal}{\kern0pt}\isanewline
\ \ \ \ \ \ \ \ \ \ \ \ \ \ {\isacharparenleft}{\kern0pt}\ {\isacharbar}{\kern0pt}state{\isacharunderscore}{\kern0pt}basis\ {\isadigit{1}}\ {\isadigit{0}}{\isasymrangle}\ {\isasymOtimes}\ {\isacharbar}{\kern0pt}state{\isacharunderscore}{\kern0pt}basis\ n\ jm{\isasymrangle}{\isacharparenright}{\kern0pt}\ {\isachardollar}{\kern0pt}{\isachardollar}{\kern0pt}\ {\isacharparenleft}{\kern0pt}i{\isacharcomma}{\kern0pt}{\isadigit{0}}{\isacharparenright}{\kern0pt}{\isachardoublequoteclose}\isanewline
\ \ \ \ \ \ \ \ \ \ \isacommand{using}\isamarkupfalse%
\ jm{\isacharunderscore}{\kern0pt}def\ jd{\isadigit{0}}\ ja{\isadigit{0}}\ jd{\isacharunderscore}{\kern0pt}def\ \isacommand{by}\isamarkupfalse%
\ auto\isanewline
\ \ \ \ \ \ \ \ \isacommand{also}\isamarkupfalse%
\ \isacommand{have}\isamarkupfalse%
\ {\isachardoublequoteopen}{\isasymdots}\ {\isacharequal}{\kern0pt}\ {\isacharbar}{\kern0pt}state{\isacharunderscore}{\kern0pt}basis\ {\isadigit{1}}\ {\isadigit{0}}{\isasymrangle}\ {\isachardollar}{\kern0pt}{\isachardollar}{\kern0pt}\ \isanewline
\ \ \ \ \ \ \ \ \ \ \ \ \ \ \ \ \ \ \ \ \ \ \ \ {\isacharparenleft}{\kern0pt}i\ div\ {\isacharparenleft}{\kern0pt}dim{\isacharunderscore}{\kern0pt}row\ {\isacharbar}{\kern0pt}state{\isacharunderscore}{\kern0pt}basis\ n\ jm{\isasymrangle}{\isacharparenright}{\kern0pt}{\isacharcomma}{\kern0pt}\ {\isadigit{0}}\ div\ {\isacharparenleft}{\kern0pt}dim{\isacharunderscore}{\kern0pt}col\ {\isacharbar}{\kern0pt}state{\isacharunderscore}{\kern0pt}basis\ n\ jm{\isasymrangle}{\isacharparenright}{\kern0pt}{\isacharparenright}{\kern0pt}\ {\isacharasterisk}{\kern0pt}\isanewline
\ \ \ \ \ \ \ \ \ \ \ \ \ \ \ \ \ \ \ \ \ \ \ \ {\isacharbar}{\kern0pt}state{\isacharunderscore}{\kern0pt}basis\ n\ jm{\isasymrangle}\ {\isachardollar}{\kern0pt}{\isachardollar}{\kern0pt}\ \isanewline
\ \ \ \ \ \ \ \ \ \ \ \ \ \ \ \ \ \ \ \ \ \ \ \ {\isacharparenleft}{\kern0pt}i\ mod\ {\isacharparenleft}{\kern0pt}dim{\isacharunderscore}{\kern0pt}row\ {\isacharbar}{\kern0pt}state{\isacharunderscore}{\kern0pt}basis\ n\ jm{\isasymrangle}{\isacharparenright}{\kern0pt}{\isacharcomma}{\kern0pt}\ {\isadigit{0}}\ mod\ {\isacharparenleft}{\kern0pt}dim{\isacharunderscore}{\kern0pt}col\ {\isacharbar}{\kern0pt}state{\isacharunderscore}{\kern0pt}basis\ n\ jm{\isasymrangle}{\isacharparenright}{\kern0pt}{\isacharparenright}{\kern0pt}{\isachardoublequoteclose}\isanewline
\ \ \ \ \ \ \ \ \isacommand{proof}\isamarkupfalse%
\ {\isacharparenleft}{\kern0pt}rule\ index{\isacharunderscore}{\kern0pt}tensor{\isacharunderscore}{\kern0pt}mat{\isacharparenright}{\kern0pt}\isanewline
\ \ \ \ \ \ \ \ \ \ \isacommand{show}\isamarkupfalse%
\ {\isachardoublequoteopen}dim{\isacharunderscore}{\kern0pt}row\ {\isacharbar}{\kern0pt}state{\isacharunderscore}{\kern0pt}basis\ {\isadigit{1}}\ {\isadigit{0}}{\isasymrangle}\ {\isacharequal}{\kern0pt}\ {\isadigit{2}}{\isachardoublequoteclose}\ \isanewline
\ \ \ \ \ \ \ \ \ \ \ \ \isacommand{using}\isamarkupfalse%
\ state{\isacharunderscore}{\kern0pt}basis{\isacharunderscore}{\kern0pt}carrier{\isacharunderscore}{\kern0pt}mat\ state{\isacharunderscore}{\kern0pt}basis{\isacharunderscore}{\kern0pt}def\ ket{\isacharunderscore}{\kern0pt}vec{\isacharunderscore}{\kern0pt}def\ \isacommand{by}\isamarkupfalse%
\ simp\isanewline
\ \ \ \ \ \ \ \ \ \ \isacommand{show}\isamarkupfalse%
\ {\isachardoublequoteopen}dim{\isacharunderscore}{\kern0pt}col\ {\isacharbar}{\kern0pt}state{\isacharunderscore}{\kern0pt}basis\ {\isadigit{1}}\ {\isadigit{0}}{\isasymrangle}\ {\isacharequal}{\kern0pt}\ {\isadigit{1}}{\isachardoublequoteclose}\isanewline
\ \ \ \ \ \ \ \ \ \ \ \ \isacommand{using}\isamarkupfalse%
\ state{\isacharunderscore}{\kern0pt}basis{\isacharunderscore}{\kern0pt}carrier{\isacharunderscore}{\kern0pt}mat\ state{\isacharunderscore}{\kern0pt}basis{\isacharunderscore}{\kern0pt}def\ ket{\isacharunderscore}{\kern0pt}vec{\isacharunderscore}{\kern0pt}def\ \isacommand{by}\isamarkupfalse%
\ simp\isanewline
\ \ \ \ \ \ \ \ \ \ \isacommand{show}\isamarkupfalse%
\ {\isachardoublequoteopen}dim{\isacharunderscore}{\kern0pt}row\ {\isacharbar}{\kern0pt}state{\isacharunderscore}{\kern0pt}basis\ n\ jm{\isasymrangle}\ {\isacharequal}{\kern0pt}\ dim{\isacharunderscore}{\kern0pt}row\ {\isacharbar}{\kern0pt}state{\isacharunderscore}{\kern0pt}basis\ n\ jm{\isasymrangle}{\isachardoublequoteclose}\ \isacommand{by}\isamarkupfalse%
\ auto\isanewline
\ \ \ \ \ \ \ \ \ \ \isacommand{show}\isamarkupfalse%
\ {\isachardoublequoteopen}dim{\isacharunderscore}{\kern0pt}col\ {\isacharbar}{\kern0pt}state{\isacharunderscore}{\kern0pt}basis\ n\ jm{\isasymrangle}\ {\isacharequal}{\kern0pt}\ dim{\isacharunderscore}{\kern0pt}col\ {\isacharbar}{\kern0pt}state{\isacharunderscore}{\kern0pt}basis\ n\ jm{\isasymrangle}{\isachardoublequoteclose}\ \isacommand{by}\isamarkupfalse%
\ auto\isanewline
\ \ \ \ \ \ \ \ \ \ \isacommand{show}\isamarkupfalse%
\ {\isachardoublequoteopen}i\ {\isacharless}{\kern0pt}\ {\isadigit{2}}\ {\isacharasterisk}{\kern0pt}\ dim{\isacharunderscore}{\kern0pt}row\ {\isacharbar}{\kern0pt}state{\isacharunderscore}{\kern0pt}basis\ n\ jm{\isasymrangle}{\isachardoublequoteclose}\ \isanewline
\ \ \ \ \ \ \ \ \ \ \ \ \isacommand{using}\isamarkupfalse%
\ il\ state{\isacharunderscore}{\kern0pt}basis{\isacharunderscore}{\kern0pt}def\ state{\isacharunderscore}{\kern0pt}basis{\isacharunderscore}{\kern0pt}carrier{\isacharunderscore}{\kern0pt}mat\ ket{\isacharunderscore}{\kern0pt}vec{\isacharunderscore}{\kern0pt}def\ \isacommand{by}\isamarkupfalse%
\ simp\isanewline
\ \ \ \ \ \ \ \ \ \ \isacommand{show}\isamarkupfalse%
\ {\isachardoublequoteopen}{\isadigit{0}}\ {\isacharless}{\kern0pt}\ {\isadigit{1}}\ {\isacharasterisk}{\kern0pt}\ dim{\isacharunderscore}{\kern0pt}col\ {\isacharbar}{\kern0pt}state{\isacharunderscore}{\kern0pt}basis\ n\ jm{\isasymrangle}{\isachardoublequoteclose}\isanewline
\ \ \ \ \ \ \ \ \ \ \ \ \isacommand{using}\isamarkupfalse%
\ state{\isacharunderscore}{\kern0pt}basis{\isacharunderscore}{\kern0pt}def\ state{\isacharunderscore}{\kern0pt}basis{\isacharunderscore}{\kern0pt}carrier{\isacharunderscore}{\kern0pt}mat\ ket{\isacharunderscore}{\kern0pt}vec{\isacharunderscore}{\kern0pt}def\ \isacommand{by}\isamarkupfalse%
\ simp\isanewline
\ \ \ \ \ \ \ \ \ \ \isacommand{show}\isamarkupfalse%
\ {\isachardoublequoteopen}{\isadigit{0}}\ {\isacharless}{\kern0pt}\ {\isacharparenleft}{\kern0pt}{\isadigit{1}}{\isacharcolon}{\kern0pt}{\isacharcolon}{\kern0pt}nat{\isacharparenright}{\kern0pt}{\isachardoublequoteclose}\ \isacommand{using}\isamarkupfalse%
\ zero{\isacharunderscore}{\kern0pt}less{\isacharunderscore}{\kern0pt}Suc\ One{\isacharunderscore}{\kern0pt}nat{\isacharunderscore}{\kern0pt}def\ \isacommand{by}\isamarkupfalse%
\ blast\isanewline
\ \ \ \ \ \ \ \ \ \ \isacommand{show}\isamarkupfalse%
\ {\isachardoublequoteopen}{\isadigit{0}}\ {\isacharless}{\kern0pt}\ dim{\isacharunderscore}{\kern0pt}col\ {\isacharbar}{\kern0pt}state{\isacharunderscore}{\kern0pt}basis\ n\ jm{\isasymrangle}{\isachardoublequoteclose}\isanewline
\ \ \ \ \ \ \ \ \ \ \ \ \isacommand{using}\isamarkupfalse%
\ state{\isacharunderscore}{\kern0pt}basis{\isacharunderscore}{\kern0pt}def\ state{\isacharunderscore}{\kern0pt}basis{\isacharunderscore}{\kern0pt}carrier{\isacharunderscore}{\kern0pt}mat\ ket{\isacharunderscore}{\kern0pt}vec{\isacharunderscore}{\kern0pt}def\ \isacommand{by}\isamarkupfalse%
\ simp\isanewline
\ \ \ \ \ \ \ \ \isacommand{qed}\isamarkupfalse%
\isanewline
\ \ \ \ \ \ \ \ \isacommand{also}\isamarkupfalse%
\ \isacommand{have}\isamarkupfalse%
\ {\isachardoublequoteopen}{\isasymdots}\ {\isacharequal}{\kern0pt}\ {\isacharbar}{\kern0pt}state{\isacharunderscore}{\kern0pt}basis\ {\isadigit{1}}\ {\isadigit{0}}{\isasymrangle}\ {\isachardollar}{\kern0pt}{\isachardollar}{\kern0pt}\ {\isacharparenleft}{\kern0pt}i\ div\ {\isadigit{2}}{\isacharcircum}{\kern0pt}n{\isacharcomma}{\kern0pt}\ {\isadigit{0}}{\isacharparenright}{\kern0pt}\ {\isacharasterisk}{\kern0pt}\ {\isacharbar}{\kern0pt}state{\isacharunderscore}{\kern0pt}basis\ n\ jm{\isasymrangle}\ {\isachardollar}{\kern0pt}{\isachardollar}{\kern0pt}\ {\isacharparenleft}{\kern0pt}i\ mod\ {\isadigit{2}}{\isacharcircum}{\kern0pt}n{\isacharcomma}{\kern0pt}\ {\isadigit{0}}{\isacharparenright}{\kern0pt}{\isachardoublequoteclose}\isanewline
\ \ \ \ \ \ \ \ \ \ \isacommand{using}\isamarkupfalse%
\ state{\isacharunderscore}{\kern0pt}basis{\isacharunderscore}{\kern0pt}def\ state{\isacharunderscore}{\kern0pt}basis{\isacharunderscore}{\kern0pt}carrier{\isacharunderscore}{\kern0pt}mat\ ket{\isacharunderscore}{\kern0pt}vec{\isacharunderscore}{\kern0pt}def\ \isacommand{by}\isamarkupfalse%
\ auto\isanewline
\ \ \ \ \ \ \ \ \isacommand{also}\isamarkupfalse%
\ \isacommand{have}\isamarkupfalse%
\ {\isachardoublequoteopen}{\isasymdots}\ {\isacharequal}{\kern0pt}\ {\isacharparenleft}{\kern0pt}mat{\isacharunderscore}{\kern0pt}of{\isacharunderscore}{\kern0pt}cols{\isacharunderscore}{\kern0pt}list\ {\isadigit{2}}\ {\isacharbrackleft}{\kern0pt}{\isacharbrackleft}{\kern0pt}{\isadigit{1}}{\isacharcomma}{\kern0pt}{\isadigit{0}}{\isacharbrackright}{\kern0pt}{\isacharbrackright}{\kern0pt}{\isacharparenright}{\kern0pt}\ {\isachardollar}{\kern0pt}{\isachardollar}{\kern0pt}\ {\isacharparenleft}{\kern0pt}i\ div\ {\isadigit{2}}{\isacharcircum}{\kern0pt}n{\isacharcomma}{\kern0pt}\ {\isadigit{0}}{\isacharparenright}{\kern0pt}\ {\isacharasterisk}{\kern0pt}\ \isanewline
\ \ \ \ \ \ \ \ \ \ \ \ \ \ \ \ \ \ \ \ \ \ \ \ {\isacharbar}{\kern0pt}state{\isacharunderscore}{\kern0pt}basis\ n\ jm{\isasymrangle}\ {\isachardollar}{\kern0pt}{\isachardollar}{\kern0pt}\ {\isacharparenleft}{\kern0pt}i\ mod\ {\isadigit{2}}{\isacharcircum}{\kern0pt}n{\isacharcomma}{\kern0pt}\ {\isadigit{0}}{\isacharparenright}{\kern0pt}{\isachardoublequoteclose}\isanewline
\ \ \ \ \ \ \ \ \ \ \isacommand{using}\isamarkupfalse%
\ state{\isacharunderscore}{\kern0pt}basis{\isacharunderscore}{\kern0pt}def\ unit{\isacharunderscore}{\kern0pt}vec{\isacharunderscore}{\kern0pt}def\ \isacommand{by}\isamarkupfalse%
\ auto\isanewline
\ \ \ \ \ \ \ \ \isacommand{also}\isamarkupfalse%
\ \isacommand{have}\isamarkupfalse%
\ {\isachardoublequoteopen}{\isasymdots}\ {\isacharequal}{\kern0pt}\ {\isacharbar}{\kern0pt}state{\isacharunderscore}{\kern0pt}basis\ {\isacharparenleft}{\kern0pt}Suc\ n{\isacharparenright}{\kern0pt}\ j{\isasymrangle}\ {\isachardollar}{\kern0pt}{\isachardollar}{\kern0pt}\ {\isacharparenleft}{\kern0pt}i{\isacharcomma}{\kern0pt}{\isadigit{0}}{\isacharparenright}{\kern0pt}{\isachardoublequoteclose}\isanewline
\ \ \ \ \ \ \ \ \isacommand{proof}\isamarkupfalse%
\ {\isacharminus}{\kern0pt}\isanewline
\ \ \ \ \ \ \ \ \ \ \isacommand{define}\isamarkupfalse%
\ id\ im\ \isakeyword{where}\ {\isachardoublequoteopen}id\ {\isacharequal}{\kern0pt}\ i\ div\ {\isadigit{2}}{\isacharcircum}{\kern0pt}n{\isachardoublequoteclose}\ \isakeyword{and}\ {\isachardoublequoteopen}im\ {\isacharequal}{\kern0pt}\ i\ mod\ {\isadigit{2}}{\isacharcircum}{\kern0pt}n{\isachardoublequoteclose}\isanewline
\ \ \ \ \ \ \ \ \ \ \isacommand{have}\isamarkupfalse%
\ i{\isacharunderscore}{\kern0pt}dec{\isacharcolon}{\kern0pt}{\isachardoublequoteopen}i\ {\isacharequal}{\kern0pt}\ id{\isacharasterisk}{\kern0pt}{\isacharparenleft}{\kern0pt}{\isadigit{2}}{\isacharcircum}{\kern0pt}n{\isacharparenright}{\kern0pt}\ {\isacharplus}{\kern0pt}\ im{\isachardoublequoteclose}\ \isacommand{using}\isamarkupfalse%
\ id{\isacharunderscore}{\kern0pt}def\ im{\isacharunderscore}{\kern0pt}def\ \isacommand{by}\isamarkupfalse%
\ presburger\isanewline
\ \ \ \ \ \ \ \ \ \ \isacommand{show}\isamarkupfalse%
\ {\isacharquery}{\kern0pt}thesis\isanewline
\ \ \ \ \ \ \ \ \ \ \isacommand{proof}\isamarkupfalse%
\ {\isacharparenleft}{\kern0pt}rule\ disjE{\isacharparenright}{\kern0pt}\isanewline
\ \ \ \ \ \ \ \ \ \ \ \ \isacommand{show}\isamarkupfalse%
\ {\isachardoublequoteopen}id\ {\isacharequal}{\kern0pt}\ {\isadigit{0}}\ {\isasymor}\ id\ {\isacharequal}{\kern0pt}\ {\isadigit{1}}{\isachardoublequoteclose}\ \isacommand{using}\isamarkupfalse%
\ id{\isacharunderscore}{\kern0pt}def\ \isacommand{by}\isamarkupfalse%
\ {\isacharparenleft}{\kern0pt}metis\ One{\isacharunderscore}{\kern0pt}nat{\isacharunderscore}{\kern0pt}def\ il\ less{\isacharunderscore}{\kern0pt}{\isadigit{2}}{\isacharunderscore}{\kern0pt}cases\ \isanewline
\ \ \ \ \ \ \ \ \ \ \ \ \ \ \ \ \ \ less{\isacharunderscore}{\kern0pt}power{\isacharunderscore}{\kern0pt}add{\isacharunderscore}{\kern0pt}imp{\isacharunderscore}{\kern0pt}div{\isacharunderscore}{\kern0pt}less\ plus{\isacharunderscore}{\kern0pt}{\isadigit{1}}{\isacharunderscore}{\kern0pt}eq{\isacharunderscore}{\kern0pt}Suc\ power{\isacharunderscore}{\kern0pt}one{\isacharunderscore}{\kern0pt}right{\isacharparenright}{\kern0pt}\isanewline
\ \ \ \ \ \ \ \ \ \ \isacommand{next}\isamarkupfalse%
\isanewline
\ \ \ \ \ \ \ \ \ \ \ \ \isacommand{assume}\isamarkupfalse%
\ id{\isadigit{0}}{\isacharcolon}{\kern0pt}{\isachardoublequoteopen}id\ {\isacharequal}{\kern0pt}\ {\isadigit{0}}{\isachardoublequoteclose}\isanewline
\ \ \ \ \ \ \ \ \ \ \ \ \isacommand{hence}\isamarkupfalse%
\ iim{\isacharcolon}{\kern0pt}{\isachardoublequoteopen}i\ {\isacharequal}{\kern0pt}\ im{\isachardoublequoteclose}\ \isacommand{using}\isamarkupfalse%
\ i{\isacharunderscore}{\kern0pt}dec\ \isacommand{by}\isamarkupfalse%
\ presburger\isanewline
\ \ \ \ \ \ \ \ \ \ \ \ \isacommand{have}\isamarkupfalse%
\ {\isachardoublequoteopen}mat{\isacharunderscore}{\kern0pt}of{\isacharunderscore}{\kern0pt}cols{\isacharunderscore}{\kern0pt}list\ {\isadigit{2}}\ {\isacharbrackleft}{\kern0pt}{\isacharbrackleft}{\kern0pt}{\isadigit{1}}{\isacharcomma}{\kern0pt}{\isadigit{0}}{\isacharbrackright}{\kern0pt}{\isacharbrackright}{\kern0pt}\ {\isachardollar}{\kern0pt}{\isachardollar}{\kern0pt}\ {\isacharparenleft}{\kern0pt}i\ div\ {\isadigit{2}}{\isacharcircum}{\kern0pt}n{\isacharcomma}{\kern0pt}{\isadigit{0}}{\isacharparenright}{\kern0pt}\ {\isacharasterisk}{\kern0pt}\ {\isacharbar}{\kern0pt}state{\isacharunderscore}{\kern0pt}basis\ n\ jm{\isasymrangle}\ {\isachardollar}{\kern0pt}{\isachardollar}{\kern0pt}\ {\isacharparenleft}{\kern0pt}i\ mod\ {\isadigit{2}}{\isacharcircum}{\kern0pt}n{\isacharcomma}{\kern0pt}\ {\isadigit{0}}{\isacharparenright}{\kern0pt}\isanewline
\ \ \ \ \ \ \ \ \ \ \ \ \ \ \ \ {\isacharequal}{\kern0pt}\ mat{\isacharunderscore}{\kern0pt}of{\isacharunderscore}{\kern0pt}cols{\isacharunderscore}{\kern0pt}list\ {\isadigit{2}}\ {\isacharbrackleft}{\kern0pt}{\isacharbrackleft}{\kern0pt}{\isadigit{1}}{\isacharcomma}{\kern0pt}{\isadigit{0}}{\isacharbrackright}{\kern0pt}{\isacharbrackright}{\kern0pt}\ {\isachardollar}{\kern0pt}{\isachardollar}{\kern0pt}\ {\isacharparenleft}{\kern0pt}{\isadigit{0}}{\isacharcomma}{\kern0pt}{\isadigit{0}}{\isacharparenright}{\kern0pt}\ {\isacharasterisk}{\kern0pt}\ {\isacharbar}{\kern0pt}state{\isacharunderscore}{\kern0pt}basis\ n\ jm{\isasymrangle}\ {\isachardollar}{\kern0pt}{\isachardollar}{\kern0pt}\ {\isacharparenleft}{\kern0pt}im{\isacharcomma}{\kern0pt}{\isadigit{0}}{\isacharparenright}{\kern0pt}{\isachardoublequoteclose}\isanewline
\ \ \ \ \ \ \ \ \ \ \ \ \ \ \isacommand{using}\isamarkupfalse%
\ id{\isacharunderscore}{\kern0pt}def\ id{\isadigit{0}}\ im{\isacharunderscore}{\kern0pt}def\ \isacommand{by}\isamarkupfalse%
\ simp\isanewline
\ \ \ \ \ \ \ \ \ \ \ \ \isacommand{also}\isamarkupfalse%
\ \isacommand{have}\isamarkupfalse%
\ {\isachardoublequoteopen}{\isasymdots}\ {\isacharequal}{\kern0pt}\ {\isadigit{1}}\ {\isacharasterisk}{\kern0pt}\ {\isacharbar}{\kern0pt}state{\isacharunderscore}{\kern0pt}basis\ n\ jm{\isasymrangle}\ {\isachardollar}{\kern0pt}{\isachardollar}{\kern0pt}\ {\isacharparenleft}{\kern0pt}im{\isacharcomma}{\kern0pt}{\isadigit{0}}{\isacharparenright}{\kern0pt}{\isachardoublequoteclose}\ \isacommand{using}\isamarkupfalse%
\ mat{\isacharunderscore}{\kern0pt}of{\isacharunderscore}{\kern0pt}cols{\isacharunderscore}{\kern0pt}list{\isacharunderscore}{\kern0pt}def\ \isacommand{by}\isamarkupfalse%
\ auto\isanewline
\ \ \ \ \ \ \ \ \ \ \ \ \isacommand{also}\isamarkupfalse%
\ \isacommand{have}\isamarkupfalse%
\ {\isachardoublequoteopen}{\isasymdots}\ {\isacharequal}{\kern0pt}\ {\isacharbar}{\kern0pt}state{\isacharunderscore}{\kern0pt}basis\ {\isacharparenleft}{\kern0pt}Suc\ n{\isacharparenright}{\kern0pt}\ jm{\isasymrangle}\ {\isachardollar}{\kern0pt}{\isachardollar}{\kern0pt}\ {\isacharparenleft}{\kern0pt}im{\isacharcomma}{\kern0pt}{\isadigit{0}}{\isacharparenright}{\kern0pt}{\isachardoublequoteclose}\ \isacommand{using}\isamarkupfalse%
\ iim\ jjm\ state{\isacharunderscore}{\kern0pt}basis{\isacharunderscore}{\kern0pt}def\isanewline
\ \ \ \ \ \ \ \ \ \ \ \ \ \ \isacommand{by}\isamarkupfalse%
\ {\isacharparenleft}{\kern0pt}smt\ {\isacharparenleft}{\kern0pt}verit{\isacharcomma}{\kern0pt}\ best{\isacharparenright}{\kern0pt}\ il\ im{\isacharunderscore}{\kern0pt}def\ index{\isacharunderscore}{\kern0pt}unit{\isacharunderscore}{\kern0pt}vec{\isacharparenleft}{\kern0pt}{\isadigit{3}}{\isacharparenright}{\kern0pt}\ index{\isacharunderscore}{\kern0pt}vec\ ket{\isacharunderscore}{\kern0pt}vec{\isacharunderscore}{\kern0pt}index\ lambda{\isacharunderscore}{\kern0pt}one\ \isanewline
\ \ \ \ \ \ \ \ \ \ \ \ \ \ \ \ \ \ mod{\isacharunderscore}{\kern0pt}less{\isacharunderscore}{\kern0pt}divisor\ pos{\isadigit{2}}\ unit{\isacharunderscore}{\kern0pt}vec{\isacharunderscore}{\kern0pt}def\ zero{\isacharunderscore}{\kern0pt}less{\isacharunderscore}{\kern0pt}power{\isacharparenright}{\kern0pt}\isanewline
\ \ \ \ \ \ \ \ \ \ \ \ \isacommand{also}\isamarkupfalse%
\ \isacommand{have}\isamarkupfalse%
\ {\isachardoublequoteopen}{\isasymdots}\ {\isacharequal}{\kern0pt}\ {\isacharbar}{\kern0pt}state{\isacharunderscore}{\kern0pt}basis\ {\isacharparenleft}{\kern0pt}Suc\ n{\isacharparenright}{\kern0pt}\ j{\isasymrangle}\ {\isachardollar}{\kern0pt}{\isachardollar}{\kern0pt}\ {\isacharparenleft}{\kern0pt}i{\isacharcomma}{\kern0pt}{\isadigit{0}}{\isacharparenright}{\kern0pt}{\isachardoublequoteclose}\ \isacommand{using}\isamarkupfalse%
\ iim\ jjm\ \isacommand{by}\isamarkupfalse%
\ simp\isanewline
\ \ \ \ \ \ \ \ \ \ \ \ \isacommand{finally}\isamarkupfalse%
\ \isacommand{show}\isamarkupfalse%
\ {\isacharquery}{\kern0pt}thesis\ \isacommand{by}\isamarkupfalse%
\ this\isanewline
\ \ \ \ \ \ \ \ \ \ \isacommand{next}\isamarkupfalse%
\isanewline
\ \ \ \ \ \ \ \ \ \ \ \ \isacommand{assume}\isamarkupfalse%
\ id{\isadigit{1}}{\isacharcolon}{\kern0pt}{\isachardoublequoteopen}id\ {\isacharequal}{\kern0pt}\ {\isadigit{1}}{\isachardoublequoteclose}\isanewline
\ \ \ \ \ \ \ \ \ \ \ \ \isacommand{hence}\isamarkupfalse%
\ iid{\isacharcolon}{\kern0pt}{\isachardoublequoteopen}i\ {\isacharequal}{\kern0pt}\ {\isadigit{2}}{\isacharcircum}{\kern0pt}n\ {\isacharplus}{\kern0pt}\ im{\isachardoublequoteclose}\ \isacommand{using}\isamarkupfalse%
\ i{\isacharunderscore}{\kern0pt}dec\ \isacommand{by}\isamarkupfalse%
\ simp\isanewline
\ \ \ \ \ \ \ \ \ \ \ \ \isacommand{have}\isamarkupfalse%
\ jma{\isacharcolon}{\kern0pt}{\isachardoublequoteopen}jm\ {\isasymnoteq}\ {\isadigit{2}}{\isacharcircum}{\kern0pt}n\ {\isacharplus}{\kern0pt}\ im{\isachardoublequoteclose}\ \isacommand{using}\isamarkupfalse%
\ jml\ iid\ \isacommand{by}\isamarkupfalse%
\ auto\isanewline
\ \ \ \ \ \ \ \ \ \ \ \ \isacommand{have}\isamarkupfalse%
\ {\isachardoublequoteopen}mat{\isacharunderscore}{\kern0pt}of{\isacharunderscore}{\kern0pt}cols{\isacharunderscore}{\kern0pt}list\ {\isadigit{2}}\ {\isacharbrackleft}{\kern0pt}{\isacharbrackleft}{\kern0pt}{\isadigit{1}}{\isacharcomma}{\kern0pt}{\isadigit{0}}{\isacharbrackright}{\kern0pt}{\isacharbrackright}{\kern0pt}\ {\isachardollar}{\kern0pt}{\isachardollar}{\kern0pt}\ {\isacharparenleft}{\kern0pt}i\ div\ {\isadigit{2}}{\isacharcircum}{\kern0pt}n{\isacharcomma}{\kern0pt}{\isadigit{0}}{\isacharparenright}{\kern0pt}\ {\isacharasterisk}{\kern0pt}\ {\isacharbar}{\kern0pt}state{\isacharunderscore}{\kern0pt}basis\ n\ jm{\isasymrangle}\ {\isachardollar}{\kern0pt}{\isachardollar}{\kern0pt}\ {\isacharparenleft}{\kern0pt}i\ mod\ {\isadigit{2}}{\isacharcircum}{\kern0pt}n{\isacharcomma}{\kern0pt}{\isadigit{0}}{\isacharparenright}{\kern0pt}\isanewline
\ \ \ \ \ \ \ \ \ \ \ \ \ \ \ \ {\isacharequal}{\kern0pt}\ mat{\isacharunderscore}{\kern0pt}of{\isacharunderscore}{\kern0pt}cols{\isacharunderscore}{\kern0pt}list\ {\isadigit{2}}\ {\isacharbrackleft}{\kern0pt}{\isacharbrackleft}{\kern0pt}{\isadigit{1}}{\isacharcomma}{\kern0pt}{\isadigit{0}}{\isacharbrackright}{\kern0pt}{\isacharbrackright}{\kern0pt}\ {\isachardollar}{\kern0pt}{\isachardollar}{\kern0pt}\ {\isacharparenleft}{\kern0pt}{\isadigit{1}}{\isacharcomma}{\kern0pt}{\isadigit{0}}{\isacharparenright}{\kern0pt}\ {\isacharasterisk}{\kern0pt}\ {\isacharbar}{\kern0pt}state{\isacharunderscore}{\kern0pt}basis\ n\ jm{\isasymrangle}\ {\isachardollar}{\kern0pt}{\isachardollar}{\kern0pt}\ {\isacharparenleft}{\kern0pt}im{\isacharcomma}{\kern0pt}{\isadigit{0}}{\isacharparenright}{\kern0pt}{\isachardoublequoteclose}\isanewline
\ \ \ \ \ \ \ \ \ \ \ \ \ \ \isacommand{using}\isamarkupfalse%
\ id{\isadigit{1}}\ id{\isacharunderscore}{\kern0pt}def\ im{\isacharunderscore}{\kern0pt}def\ \isacommand{by}\isamarkupfalse%
\ simp\isanewline
\ \ \ \ \ \ \ \ \ \ \ \ \isacommand{also}\isamarkupfalse%
\ \isacommand{have}\isamarkupfalse%
\ {\isachardoublequoteopen}{\isasymdots}\ {\isacharequal}{\kern0pt}\ {\isadigit{0}}{\isachardoublequoteclose}\ \isacommand{using}\isamarkupfalse%
\ mat{\isacharunderscore}{\kern0pt}of{\isacharunderscore}{\kern0pt}cols{\isacharunderscore}{\kern0pt}list{\isacharunderscore}{\kern0pt}def\ \isacommand{by}\isamarkupfalse%
\ auto\isanewline
\ \ \ \ \ \ \ \ \ \ \ \ \isacommand{also}\isamarkupfalse%
\ \isacommand{have}\isamarkupfalse%
\ {\isachardoublequoteopen}{\isasymdots}\ {\isacharequal}{\kern0pt}\ {\isacharbar}{\kern0pt}state{\isacharunderscore}{\kern0pt}basis\ {\isacharparenleft}{\kern0pt}Suc\ n{\isacharparenright}{\kern0pt}\ jm{\isasymrangle}\ {\isachardollar}{\kern0pt}{\isachardollar}{\kern0pt}\ {\isacharparenleft}{\kern0pt}{\isadigit{2}}{\isacharcircum}{\kern0pt}n\ {\isacharplus}{\kern0pt}\ im{\isacharcomma}{\kern0pt}{\isadigit{0}}{\isacharparenright}{\kern0pt}{\isachardoublequoteclose}\ \isanewline
\ \ \ \ \ \ \ \ \ \ \ \ \isacommand{proof}\isamarkupfalse%
\ {\isacharminus}{\kern0pt}\isanewline
\ \ \ \ \ \ \ \ \ \ \ \ \ \ \isacommand{have}\isamarkupfalse%
\ {\isachardoublequoteopen}{\isacharbar}{\kern0pt}state{\isacharunderscore}{\kern0pt}basis\ {\isacharparenleft}{\kern0pt}Suc\ n{\isacharparenright}{\kern0pt}\ jm{\isasymrangle}\ {\isachardollar}{\kern0pt}{\isachardollar}{\kern0pt}\ {\isacharparenleft}{\kern0pt}{\isadigit{2}}{\isacharcircum}{\kern0pt}n\ {\isacharplus}{\kern0pt}\ im{\isacharcomma}{\kern0pt}{\isadigit{0}}{\isacharparenright}{\kern0pt}\ {\isacharequal}{\kern0pt}\ \isanewline
\ \ \ \ \ \ \ \ \ \ \ \ \ \ \ \ \ \ \ \ {\isacharbar}{\kern0pt}unit{\isacharunderscore}{\kern0pt}vec\ {\isacharparenleft}{\kern0pt}{\isadigit{2}}{\isacharcircum}{\kern0pt}{\isacharparenleft}{\kern0pt}Suc\ n{\isacharparenright}{\kern0pt}{\isacharparenright}{\kern0pt}\ jm{\isasymrangle}\ {\isachardollar}{\kern0pt}{\isachardollar}{\kern0pt}\ {\isacharparenleft}{\kern0pt}{\isadigit{2}}{\isacharcircum}{\kern0pt}n{\isacharplus}{\kern0pt}im{\isacharcomma}{\kern0pt}{\isadigit{0}}{\isacharparenright}{\kern0pt}{\isachardoublequoteclose}\isanewline
\ \ \ \ \ \ \ \ \ \ \ \ \ \ \ \ \isacommand{using}\isamarkupfalse%
\ state{\isacharunderscore}{\kern0pt}basis{\isacharunderscore}{\kern0pt}def\ \isacommand{by}\isamarkupfalse%
\ simp\isanewline
\ \ \ \ \ \ \ \ \ \ \ \ \ \ \isacommand{also}\isamarkupfalse%
\ \isacommand{have}\isamarkupfalse%
\ {\isachardoublequoteopen}{\isasymdots}\ {\isacharequal}{\kern0pt}\ Matrix{\isachardot}{\kern0pt}mat\ {\isacharparenleft}{\kern0pt}{\isadigit{2}}{\isacharcircum}{\kern0pt}{\isacharparenleft}{\kern0pt}Suc\ n{\isacharparenright}{\kern0pt}{\isacharparenright}{\kern0pt}\ {\isadigit{1}}\ {\isacharparenleft}{\kern0pt}{\isasymlambda}{\isacharparenleft}{\kern0pt}i{\isacharcomma}{\kern0pt}\ j{\isacharparenright}{\kern0pt}{\isachardot}{\kern0pt}\ {\isacharparenleft}{\kern0pt}unit{\isacharunderscore}{\kern0pt}vec\ {\isacharparenleft}{\kern0pt}{\isadigit{2}}{\isacharcircum}{\kern0pt}{\isacharparenleft}{\kern0pt}Suc\ n{\isacharparenright}{\kern0pt}{\isacharparenright}{\kern0pt}\ jm{\isacharparenright}{\kern0pt}\ {\isachardollar}{\kern0pt}\ i{\isacharparenright}{\kern0pt}\isanewline
\ \ \ \ \ \ \ \ \ \ \ \ \ \ \ \ \ \ \ \ \ \ \ \ \ \ \ \ \ \ {\isachardollar}{\kern0pt}{\isachardollar}{\kern0pt}\ {\isacharparenleft}{\kern0pt}{\isadigit{2}}{\isacharcircum}{\kern0pt}n{\isacharplus}{\kern0pt}im{\isacharcomma}{\kern0pt}{\isadigit{0}}{\isacharparenright}{\kern0pt}{\isachardoublequoteclose}\isanewline
\ \ \ \ \ \ \ \ \ \ \ \ \ \ \ \ \isacommand{using}\isamarkupfalse%
\ ket{\isacharunderscore}{\kern0pt}vec{\isacharunderscore}{\kern0pt}def\ \isacommand{by}\isamarkupfalse%
\ simp\isanewline
\ \ \ \ \ \ \ \ \ \ \ \ \ \ \isacommand{also}\isamarkupfalse%
\ \isacommand{have}\isamarkupfalse%
\ {\isachardoublequoteopen}{\isasymdots}\ {\isacharequal}{\kern0pt}\ Matrix{\isachardot}{\kern0pt}mat\ {\isacharparenleft}{\kern0pt}{\isadigit{2}}{\isacharcircum}{\kern0pt}{\isacharparenleft}{\kern0pt}Suc\ n{\isacharparenright}{\kern0pt}{\isacharparenright}{\kern0pt}\ {\isadigit{1}}\ {\isacharparenleft}{\kern0pt}{\isasymlambda}{\isacharparenleft}{\kern0pt}i{\isacharcomma}{\kern0pt}j{\isacharparenright}{\kern0pt}{\isachardot}{\kern0pt}\ Matrix{\isachardot}{\kern0pt}vec\ {\isacharparenleft}{\kern0pt}{\isadigit{2}}{\isacharcircum}{\kern0pt}{\isacharparenleft}{\kern0pt}Suc\ n{\isacharparenright}{\kern0pt}{\isacharparenright}{\kern0pt}\ \isanewline
\ \ \ \ \ \ \ \ \ \ \ \ \ \ \ \ \ \ \ \ \ \ \ \ \ \ \ \ \ \ {\isacharparenleft}{\kern0pt}{\isasymlambda}j{\isacharprime}{\kern0pt}{\isachardot}{\kern0pt}\ if\ j{\isacharprime}{\kern0pt}{\isacharequal}{\kern0pt}jm\ then\ {\isadigit{1}}\ else\ {\isadigit{0}}{\isacharparenright}{\kern0pt}\ {\isachardollar}{\kern0pt}\ i{\isacharparenright}{\kern0pt}\ {\isachardollar}{\kern0pt}{\isachardollar}{\kern0pt}\ {\isacharparenleft}{\kern0pt}{\isadigit{2}}{\isacharcircum}{\kern0pt}n{\isacharplus}{\kern0pt}im{\isacharcomma}{\kern0pt}{\isadigit{0}}{\isacharparenright}{\kern0pt}{\isachardoublequoteclose}\isanewline
\ \ \ \ \ \ \ \ \ \ \ \ \ \ \ \ \isacommand{using}\isamarkupfalse%
\ unit{\isacharunderscore}{\kern0pt}vec{\isacharunderscore}{\kern0pt}def\ \isacommand{by}\isamarkupfalse%
\ metis\isanewline
\ \ \ \ \ \ \ \ \ \ \ \ \ \ \isacommand{also}\isamarkupfalse%
\ \isacommand{have}\isamarkupfalse%
\ {\isachardoublequoteopen}{\isasymdots}\ {\isacharequal}{\kern0pt}\ {\isadigit{0}}{\isachardoublequoteclose}\ \isacommand{using}\isamarkupfalse%
\ iid\ il\ jma\ \isacommand{by}\isamarkupfalse%
\ fastforce\isanewline
\ \ \ \ \ \ \ \ \ \ \ \ \ \ \isacommand{finally}\isamarkupfalse%
\ \isacommand{show}\isamarkupfalse%
\ {\isacharquery}{\kern0pt}thesis\ \isacommand{by}\isamarkupfalse%
\ auto\isanewline
\ \ \ \ \ \ \ \ \ \ \ \ \isacommand{qed}\isamarkupfalse%
\isanewline
\ \ \ \ \ \ \ \ \ \ \ \ \isacommand{also}\isamarkupfalse%
\ \isacommand{have}\isamarkupfalse%
\ {\isachardoublequoteopen}{\isasymdots}\ {\isacharequal}{\kern0pt}\ {\isacharbar}{\kern0pt}state{\isacharunderscore}{\kern0pt}basis\ {\isacharparenleft}{\kern0pt}Suc\ n{\isacharparenright}{\kern0pt}\ j{\isasymrangle}\ {\isachardollar}{\kern0pt}{\isachardollar}{\kern0pt}\ {\isacharparenleft}{\kern0pt}i{\isacharcomma}{\kern0pt}{\isadigit{0}}{\isacharparenright}{\kern0pt}{\isachardoublequoteclose}\ \isacommand{using}\isamarkupfalse%
\ jjm\ iid\ \isacommand{by}\isamarkupfalse%
\ simp\isanewline
\ \ \ \ \ \ \ \ \ \ \ \ \isacommand{finally}\isamarkupfalse%
\ \isacommand{show}\isamarkupfalse%
\ {\isacharquery}{\kern0pt}thesis\ \isacommand{by}\isamarkupfalse%
\ this\isanewline
\ \ \ \ \ \ \ \ \ \ \isacommand{qed}\isamarkupfalse%
\isanewline
\ \ \ \ \ \ \ \ \isacommand{qed}\isamarkupfalse%
\isanewline
\ \ \ \ \ \ \ \ \isacommand{finally}\isamarkupfalse%
\ \isacommand{show}\isamarkupfalse%
\ {\isacharquery}{\kern0pt}thesis\ \isacommand{using}\isamarkupfalse%
\ ja{\isadigit{0}}\ \isacommand{by}\isamarkupfalse%
\ auto\isanewline
\ \ \ \ \ \ \isacommand{qed}\isamarkupfalse%
\isanewline
\ \ \ \ \isacommand{next}\isamarkupfalse%
\isanewline
\ \ \ \ \ \ \isacommand{show}\isamarkupfalse%
\ {\isachardoublequoteopen}dim{\isacharunderscore}{\kern0pt}row\ {\isacharparenleft}{\kern0pt}\ {\isacharbar}{\kern0pt}state{\isacharunderscore}{\kern0pt}basis\ {\isadigit{1}}\ {\isacharparenleft}{\kern0pt}j\ div\ {\isadigit{2}}\ {\isacharcircum}{\kern0pt}\ n{\isacharparenright}{\kern0pt}{\isasymrangle}\ {\isasymOtimes}\ {\isacharbar}{\kern0pt}state{\isacharunderscore}{\kern0pt}basis\ n\ {\isacharparenleft}{\kern0pt}j\ mod\ {\isadigit{2}}\ {\isacharcircum}{\kern0pt}\ n{\isacharparenright}{\kern0pt}{\isasymrangle}{\isacharparenright}{\kern0pt}\ {\isacharequal}{\kern0pt}\isanewline
\ \ \ \ \ \ \ \ \ \ \ \ dim{\isacharunderscore}{\kern0pt}row\ {\isacharbar}{\kern0pt}state{\isacharunderscore}{\kern0pt}basis\ {\isacharparenleft}{\kern0pt}Suc\ n{\isacharparenright}{\kern0pt}\ j{\isasymrangle}{\isachardoublequoteclose}\ \isanewline
\ \ \ \ \ \ \ \ \isacommand{using}\isamarkupfalse%
\ state{\isacharunderscore}{\kern0pt}basis{\isacharunderscore}{\kern0pt}def\ state{\isacharunderscore}{\kern0pt}basis{\isacharunderscore}{\kern0pt}carrier{\isacharunderscore}{\kern0pt}mat\ ket{\isacharunderscore}{\kern0pt}vec{\isacharunderscore}{\kern0pt}def\ \isacommand{by}\isamarkupfalse%
\ auto\isanewline
\ \ \ \ \isacommand{next}\isamarkupfalse%
\isanewline
\ \ \ \ \ \ \isacommand{show}\isamarkupfalse%
\ {\isachardoublequoteopen}dim{\isacharunderscore}{\kern0pt}col\ {\isacharparenleft}{\kern0pt}\ {\isacharbar}{\kern0pt}state{\isacharunderscore}{\kern0pt}basis\ {\isadigit{1}}\ {\isacharparenleft}{\kern0pt}j\ div\ {\isadigit{2}}\ {\isacharcircum}{\kern0pt}\ n{\isacharparenright}{\kern0pt}{\isasymrangle}\ {\isasymOtimes}\ {\isacharbar}{\kern0pt}state{\isacharunderscore}{\kern0pt}basis\ n\ {\isacharparenleft}{\kern0pt}j\ mod\ {\isadigit{2}}\ {\isacharcircum}{\kern0pt}\ n{\isacharparenright}{\kern0pt}{\isasymrangle}{\isacharparenright}{\kern0pt}\ {\isacharequal}{\kern0pt}\isanewline
\ \ \ \ \ \ \ \ \ \ \ \ dim{\isacharunderscore}{\kern0pt}col\ {\isacharbar}{\kern0pt}state{\isacharunderscore}{\kern0pt}basis\ {\isacharparenleft}{\kern0pt}Suc\ n{\isacharparenright}{\kern0pt}\ j{\isasymrangle}{\isachardoublequoteclose}\isanewline
\ \ \ \ \ \ \ \ \isacommand{using}\isamarkupfalse%
\ state{\isacharunderscore}{\kern0pt}basis{\isacharunderscore}{\kern0pt}def\ state{\isacharunderscore}{\kern0pt}basis{\isacharunderscore}{\kern0pt}carrier{\isacharunderscore}{\kern0pt}mat\ ket{\isacharunderscore}{\kern0pt}vec{\isacharunderscore}{\kern0pt}def\ \isacommand{by}\isamarkupfalse%
\ auto\isanewline
\ \ \ \ \isacommand{qed}\isamarkupfalse%
\isanewline
\ \ \isacommand{next}\isamarkupfalse%
\isanewline
\ \ \ \ \isacommand{assume}\isamarkupfalse%
\ jd{\isadigit{1}}{\isacharcolon}{\kern0pt}{\isachardoublequoteopen}jd\ {\isacharequal}{\kern0pt}\ {\isadigit{1}}{\isachardoublequoteclose}\isanewline
\ \ \ \ \isacommand{hence}\isamarkupfalse%
\ j{\isacharunderscore}{\kern0pt}dec{\isadigit{2}}{\isacharcolon}{\kern0pt}{\isachardoublequoteopen}j\ {\isacharequal}{\kern0pt}\ {\isadigit{2}}{\isacharcircum}{\kern0pt}n\ {\isacharplus}{\kern0pt}\ jm{\isachardoublequoteclose}\ \isacommand{using}\isamarkupfalse%
\ j{\isacharunderscore}{\kern0pt}dec\ \isacommand{by}\isamarkupfalse%
\ auto\isanewline
\ \ \ \ \isacommand{show}\isamarkupfalse%
\ {\isachardoublequoteopen}{\isacharbar}{\kern0pt}state{\isacharunderscore}{\kern0pt}basis\ {\isadigit{1}}\ {\isacharparenleft}{\kern0pt}j\ div\ {\isadigit{2}}\ {\isacharcircum}{\kern0pt}\ n{\isacharparenright}{\kern0pt}{\isasymrangle}\ {\isasymOtimes}\ {\isacharbar}{\kern0pt}state{\isacharunderscore}{\kern0pt}basis\ n\ {\isacharparenleft}{\kern0pt}j\ mod\ {\isadigit{2}}\ {\isacharcircum}{\kern0pt}\ n{\isacharparenright}{\kern0pt}{\isasymrangle}\ {\isacharequal}{\kern0pt}\ {\isacharbar}{\kern0pt}state{\isacharunderscore}{\kern0pt}basis\ {\isacharparenleft}{\kern0pt}Suc\ n{\isacharparenright}{\kern0pt}\ j{\isasymrangle}{\isachardoublequoteclose}\isanewline
\ \ \ \ \isacommand{proof}\isamarkupfalse%
\isanewline
\ \ \ \ \ \ \isacommand{fix}\isamarkupfalse%
\ i\ ja\isanewline
\ \ \ \ \ \ \isacommand{assume}\isamarkupfalse%
\ {\isachardoublequoteopen}i\ {\isacharless}{\kern0pt}\ dim{\isacharunderscore}{\kern0pt}row\ {\isacharbar}{\kern0pt}state{\isacharunderscore}{\kern0pt}basis\ {\isacharparenleft}{\kern0pt}Suc\ n{\isacharparenright}{\kern0pt}\ j{\isasymrangle}{\isachardoublequoteclose}\isanewline
\ \ \ \ \ \ \isacommand{hence}\isamarkupfalse%
\ il{\isacharcolon}{\kern0pt}{\isachardoublequoteopen}i\ {\isacharless}{\kern0pt}\ {\isadigit{2}}{\isacharcircum}{\kern0pt}{\isacharparenleft}{\kern0pt}Suc\ n{\isacharparenright}{\kern0pt}{\isachardoublequoteclose}\ \isacommand{using}\isamarkupfalse%
\ state{\isacharunderscore}{\kern0pt}basis{\isacharunderscore}{\kern0pt}def\ state{\isacharunderscore}{\kern0pt}basis{\isacharunderscore}{\kern0pt}carrier{\isacharunderscore}{\kern0pt}mat\ ket{\isacharunderscore}{\kern0pt}vec{\isacharunderscore}{\kern0pt}def\ \isacommand{by}\isamarkupfalse%
\ simp\isanewline
\ \ \ \ \ \ \isacommand{assume}\isamarkupfalse%
\ {\isachardoublequoteopen}ja\ {\isacharless}{\kern0pt}\ dim{\isacharunderscore}{\kern0pt}col\ {\isacharbar}{\kern0pt}state{\isacharunderscore}{\kern0pt}basis\ {\isacharparenleft}{\kern0pt}Suc\ n{\isacharparenright}{\kern0pt}\ j{\isasymrangle}{\isachardoublequoteclose}\isanewline
\ \ \ \ \ \ \isacommand{hence}\isamarkupfalse%
\ jal{\isacharcolon}{\kern0pt}{\isachardoublequoteopen}ja\ {\isacharless}{\kern0pt}\ {\isadigit{1}}{\isachardoublequoteclose}\ \isacommand{using}\isamarkupfalse%
\ state{\isacharunderscore}{\kern0pt}basis{\isacharunderscore}{\kern0pt}def\ state{\isacharunderscore}{\kern0pt}basis{\isacharunderscore}{\kern0pt}carrier{\isacharunderscore}{\kern0pt}mat\ ket{\isacharunderscore}{\kern0pt}vec{\isacharunderscore}{\kern0pt}def\ \isacommand{by}\isamarkupfalse%
\ simp\isanewline
\ \ \ \ \ \ \isacommand{hence}\isamarkupfalse%
\ ja{\isadigit{0}}{\isacharcolon}{\kern0pt}{\isachardoublequoteopen}ja\ {\isacharequal}{\kern0pt}\ {\isadigit{0}}{\isachardoublequoteclose}\ \isacommand{by}\isamarkupfalse%
\ auto\isanewline
\ \ \ \ \ \ \isacommand{show}\isamarkupfalse%
\ {\isachardoublequoteopen}{\isacharparenleft}{\kern0pt}\ {\isacharbar}{\kern0pt}state{\isacharunderscore}{\kern0pt}basis\ {\isadigit{1}}\ {\isacharparenleft}{\kern0pt}j\ div\ {\isadigit{2}}\ {\isacharcircum}{\kern0pt}\ n{\isacharparenright}{\kern0pt}{\isasymrangle}\ {\isasymOtimes}\ {\isacharbar}{\kern0pt}state{\isacharunderscore}{\kern0pt}basis\ n\ {\isacharparenleft}{\kern0pt}j\ mod\ {\isadigit{2}}\ {\isacharcircum}{\kern0pt}\ n{\isacharparenright}{\kern0pt}{\isasymrangle}{\isacharparenright}{\kern0pt}\ {\isachardollar}{\kern0pt}{\isachardollar}{\kern0pt}\ {\isacharparenleft}{\kern0pt}i{\isacharcomma}{\kern0pt}\ ja{\isacharparenright}{\kern0pt}\ {\isacharequal}{\kern0pt}\isanewline
\ \ \ \ \ \ \ \ \ \ \ \ \ \ {\isacharbar}{\kern0pt}state{\isacharunderscore}{\kern0pt}basis\ {\isacharparenleft}{\kern0pt}Suc\ n{\isacharparenright}{\kern0pt}\ j{\isasymrangle}\ {\isachardollar}{\kern0pt}{\isachardollar}{\kern0pt}\ {\isacharparenleft}{\kern0pt}i{\isacharcomma}{\kern0pt}\ ja{\isacharparenright}{\kern0pt}{\isachardoublequoteclose}\isanewline
\ \ \ \ \ \ \isacommand{proof}\isamarkupfalse%
\ {\isacharminus}{\kern0pt}\isanewline
\ \ \ \ \ \ \ \ \isacommand{have}\isamarkupfalse%
\ {\isachardoublequoteopen}{\isacharparenleft}{\kern0pt}\ {\isacharbar}{\kern0pt}state{\isacharunderscore}{\kern0pt}basis\ {\isadigit{1}}\ jd{\isasymrangle}\ {\isasymOtimes}\ {\isacharbar}{\kern0pt}state{\isacharunderscore}{\kern0pt}basis\ n\ jm{\isasymrangle}{\isacharparenright}{\kern0pt}\ {\isachardollar}{\kern0pt}{\isachardollar}{\kern0pt}\ {\isacharparenleft}{\kern0pt}i{\isacharcomma}{\kern0pt}\ {\isadigit{0}}{\isacharparenright}{\kern0pt}\ {\isacharequal}{\kern0pt}\isanewline
\ \ \ \ \ \ \ \ \ \ \ \ \ \ {\isacharparenleft}{\kern0pt}\ {\isacharbar}{\kern0pt}state{\isacharunderscore}{\kern0pt}basis\ {\isadigit{1}}\ {\isadigit{1}}{\isasymrangle}\ {\isasymOtimes}\ {\isacharbar}{\kern0pt}state{\isacharunderscore}{\kern0pt}basis\ n\ jm{\isasymrangle}{\isacharparenright}{\kern0pt}\ {\isachardollar}{\kern0pt}{\isachardollar}{\kern0pt}\ {\isacharparenleft}{\kern0pt}i{\isacharcomma}{\kern0pt}\ {\isadigit{0}}{\isacharparenright}{\kern0pt}{\isachardoublequoteclose}\isanewline
\ \ \ \ \ \ \ \ \ \ \isacommand{using}\isamarkupfalse%
\ jd{\isadigit{1}}\ \isacommand{by}\isamarkupfalse%
\ simp\isanewline
\ \ \ \ \ \ \ \ \isacommand{also}\isamarkupfalse%
\ \isacommand{have}\isamarkupfalse%
\ {\isachardoublequoteopen}{\isasymdots}\ {\isacharequal}{\kern0pt}\ {\isacharbar}{\kern0pt}state{\isacharunderscore}{\kern0pt}basis\ {\isadigit{1}}\ {\isadigit{1}}{\isasymrangle}\ {\isachardollar}{\kern0pt}{\isachardollar}{\kern0pt}\ \isanewline
\ \ \ \ \ \ \ \ \ \ \ \ \ \ \ \ \ \ \ \ \ \ \ \ {\isacharparenleft}{\kern0pt}i\ div\ {\isacharparenleft}{\kern0pt}dim{\isacharunderscore}{\kern0pt}row\ {\isacharbar}{\kern0pt}state{\isacharunderscore}{\kern0pt}basis\ n\ jm{\isasymrangle}{\isacharparenright}{\kern0pt}{\isacharcomma}{\kern0pt}\ {\isadigit{0}}\ div\ {\isacharparenleft}{\kern0pt}dim{\isacharunderscore}{\kern0pt}col\ {\isacharbar}{\kern0pt}state{\isacharunderscore}{\kern0pt}basis\ n\ jm{\isasymrangle}{\isacharparenright}{\kern0pt}{\isacharparenright}{\kern0pt}\ {\isacharasterisk}{\kern0pt}\isanewline
\ \ \ \ \ \ \ \ \ \ \ \ \ \ \ \ \ \ \ \ \ \ \ \ {\isacharbar}{\kern0pt}state{\isacharunderscore}{\kern0pt}basis\ n\ jm{\isasymrangle}\ {\isachardollar}{\kern0pt}{\isachardollar}{\kern0pt}\ \isanewline
\ \ \ \ \ \ \ \ \ \ \ \ \ \ \ \ \ \ \ \ \ \ \ \ {\isacharparenleft}{\kern0pt}i\ mod\ {\isacharparenleft}{\kern0pt}dim{\isacharunderscore}{\kern0pt}row\ {\isacharbar}{\kern0pt}state{\isacharunderscore}{\kern0pt}basis\ n\ jm{\isasymrangle}{\isacharparenright}{\kern0pt}{\isacharcomma}{\kern0pt}\ {\isadigit{0}}\ mod\ {\isacharparenleft}{\kern0pt}dim{\isacharunderscore}{\kern0pt}col\ {\isacharbar}{\kern0pt}state{\isacharunderscore}{\kern0pt}basis\ n\ jm{\isasymrangle}{\isacharparenright}{\kern0pt}{\isacharparenright}{\kern0pt}{\isachardoublequoteclose}\isanewline
\ \ \ \ \ \ \ \ \isacommand{proof}\isamarkupfalse%
\ {\isacharparenleft}{\kern0pt}rule\ index{\isacharunderscore}{\kern0pt}tensor{\isacharunderscore}{\kern0pt}mat{\isacharparenright}{\kern0pt}\isanewline
\ \ \ \ \ \ \ \ \ \ \isacommand{show}\isamarkupfalse%
\ {\isachardoublequoteopen}dim{\isacharunderscore}{\kern0pt}row\ {\isacharbar}{\kern0pt}state{\isacharunderscore}{\kern0pt}basis\ {\isadigit{1}}\ {\isadigit{1}}{\isasymrangle}\ {\isacharequal}{\kern0pt}\ {\isadigit{2}}{\isachardoublequoteclose}\ \isanewline
\ \ \ \ \ \ \ \ \ \ \ \ \isacommand{using}\isamarkupfalse%
\ state{\isacharunderscore}{\kern0pt}basis{\isacharunderscore}{\kern0pt}carrier{\isacharunderscore}{\kern0pt}mat\ state{\isacharunderscore}{\kern0pt}basis{\isacharunderscore}{\kern0pt}def\ ket{\isacharunderscore}{\kern0pt}vec{\isacharunderscore}{\kern0pt}def\ \isacommand{by}\isamarkupfalse%
\ simp\isanewline
\ \ \ \ \ \ \ \ \ \ \isacommand{show}\isamarkupfalse%
\ {\isachardoublequoteopen}dim{\isacharunderscore}{\kern0pt}col\ {\isacharbar}{\kern0pt}state{\isacharunderscore}{\kern0pt}basis\ {\isadigit{1}}\ {\isadigit{1}}{\isasymrangle}\ {\isacharequal}{\kern0pt}\ {\isadigit{1}}{\isachardoublequoteclose}\isanewline
\ \ \ \ \ \ \ \ \ \ \ \ \isacommand{using}\isamarkupfalse%
\ state{\isacharunderscore}{\kern0pt}basis{\isacharunderscore}{\kern0pt}carrier{\isacharunderscore}{\kern0pt}mat\ state{\isacharunderscore}{\kern0pt}basis{\isacharunderscore}{\kern0pt}def\ ket{\isacharunderscore}{\kern0pt}vec{\isacharunderscore}{\kern0pt}def\ \isacommand{by}\isamarkupfalse%
\ simp\isanewline
\ \ \ \ \ \ \ \ \ \ \isacommand{show}\isamarkupfalse%
\ {\isachardoublequoteopen}dim{\isacharunderscore}{\kern0pt}row\ {\isacharbar}{\kern0pt}state{\isacharunderscore}{\kern0pt}basis\ n\ jm{\isasymrangle}\ {\isacharequal}{\kern0pt}\ dim{\isacharunderscore}{\kern0pt}row\ {\isacharbar}{\kern0pt}state{\isacharunderscore}{\kern0pt}basis\ n\ jm{\isasymrangle}{\isachardoublequoteclose}\ \isacommand{by}\isamarkupfalse%
\ auto\isanewline
\ \ \ \ \ \ \ \ \ \ \isacommand{show}\isamarkupfalse%
\ {\isachardoublequoteopen}dim{\isacharunderscore}{\kern0pt}col\ {\isacharbar}{\kern0pt}state{\isacharunderscore}{\kern0pt}basis\ n\ jm{\isasymrangle}\ {\isacharequal}{\kern0pt}\ dim{\isacharunderscore}{\kern0pt}col\ {\isacharbar}{\kern0pt}state{\isacharunderscore}{\kern0pt}basis\ n\ jm{\isasymrangle}{\isachardoublequoteclose}\ \isacommand{by}\isamarkupfalse%
\ auto\isanewline
\ \ \ \ \ \ \ \ \ \ \isacommand{show}\isamarkupfalse%
\ {\isachardoublequoteopen}i\ {\isacharless}{\kern0pt}\ {\isadigit{2}}\ {\isacharasterisk}{\kern0pt}\ dim{\isacharunderscore}{\kern0pt}row\ {\isacharbar}{\kern0pt}state{\isacharunderscore}{\kern0pt}basis\ n\ jm{\isasymrangle}{\isachardoublequoteclose}\isanewline
\ \ \ \ \ \ \ \ \ \ \ \ \isacommand{using}\isamarkupfalse%
\ state{\isacharunderscore}{\kern0pt}basis{\isacharunderscore}{\kern0pt}carrier{\isacharunderscore}{\kern0pt}mat\ state{\isacharunderscore}{\kern0pt}basis{\isacharunderscore}{\kern0pt}def\ ket{\isacharunderscore}{\kern0pt}vec{\isacharunderscore}{\kern0pt}def\ il\ \isacommand{by}\isamarkupfalse%
\ auto\isanewline
\ \ \ \ \ \ \ \ \ \ \isacommand{show}\isamarkupfalse%
\ {\isachardoublequoteopen}{\isadigit{0}}\ {\isacharless}{\kern0pt}\ {\isadigit{1}}\ {\isacharasterisk}{\kern0pt}\ dim{\isacharunderscore}{\kern0pt}col\ {\isacharbar}{\kern0pt}state{\isacharunderscore}{\kern0pt}basis\ n\ jm{\isasymrangle}{\isachardoublequoteclose}\isanewline
\ \ \ \ \ \ \ \ \ \ \ \ \isacommand{using}\isamarkupfalse%
\ state{\isacharunderscore}{\kern0pt}basis{\isacharunderscore}{\kern0pt}carrier{\isacharunderscore}{\kern0pt}mat\ state{\isacharunderscore}{\kern0pt}basis{\isacharunderscore}{\kern0pt}def\ ket{\isacharunderscore}{\kern0pt}vec{\isacharunderscore}{\kern0pt}def\ \isacommand{by}\isamarkupfalse%
\ auto\isanewline
\ \ \ \ \ \ \ \ \ \ \isacommand{show}\isamarkupfalse%
\ {\isachardoublequoteopen}{\isadigit{0}}\ {\isacharless}{\kern0pt}\ {\isacharparenleft}{\kern0pt}{\isadigit{1}}{\isacharcolon}{\kern0pt}{\isacharcolon}{\kern0pt}nat{\isacharparenright}{\kern0pt}{\isachardoublequoteclose}\ \isacommand{by}\isamarkupfalse%
\ simp\isanewline
\ \ \ \ \ \ \ \ \ \ \isacommand{show}\isamarkupfalse%
\ {\isachardoublequoteopen}{\isadigit{0}}\ {\isacharless}{\kern0pt}\ dim{\isacharunderscore}{\kern0pt}col\ {\isacharbar}{\kern0pt}state{\isacharunderscore}{\kern0pt}basis\ n\ jm{\isasymrangle}{\isachardoublequoteclose}\isanewline
\ \ \ \ \ \ \ \ \ \ \ \ \isacommand{using}\isamarkupfalse%
\ state{\isacharunderscore}{\kern0pt}basis{\isacharunderscore}{\kern0pt}carrier{\isacharunderscore}{\kern0pt}mat\ state{\isacharunderscore}{\kern0pt}basis{\isacharunderscore}{\kern0pt}def\ ket{\isacharunderscore}{\kern0pt}vec{\isacharunderscore}{\kern0pt}def\ \isacommand{by}\isamarkupfalse%
\ auto\isanewline
\ \ \ \ \ \ \ \ \isacommand{qed}\isamarkupfalse%
\isanewline
\ \ \ \ \ \ \ \ \isacommand{also}\isamarkupfalse%
\ \isacommand{have}\isamarkupfalse%
\ {\isachardoublequoteopen}{\isasymdots}\ {\isacharequal}{\kern0pt}\ {\isacharparenleft}{\kern0pt}mat{\isacharunderscore}{\kern0pt}of{\isacharunderscore}{\kern0pt}cols{\isacharunderscore}{\kern0pt}list\ {\isadigit{2}}\ {\isacharbrackleft}{\kern0pt}{\isacharbrackleft}{\kern0pt}{\isadigit{0}}{\isacharcomma}{\kern0pt}{\isadigit{1}}{\isacharbrackright}{\kern0pt}{\isacharbrackright}{\kern0pt}{\isacharparenright}{\kern0pt}\ {\isachardollar}{\kern0pt}{\isachardollar}{\kern0pt}\ {\isacharparenleft}{\kern0pt}i\ div\ {\isadigit{2}}{\isacharcircum}{\kern0pt}n{\isacharcomma}{\kern0pt}{\isadigit{0}}{\isacharparenright}{\kern0pt}\ {\isacharasterisk}{\kern0pt}\isanewline
\ \ \ \ \ \ \ \ \ \ \ \ \ \ \ \ \ \ \ \ \ \ \ \ {\isacharbar}{\kern0pt}state{\isacharunderscore}{\kern0pt}basis\ n\ jm{\isasymrangle}\ {\isachardollar}{\kern0pt}{\isachardollar}{\kern0pt}\ {\isacharparenleft}{\kern0pt}i\ mod\ {\isadigit{2}}{\isacharcircum}{\kern0pt}n{\isacharcomma}{\kern0pt}{\isadigit{0}}{\isacharparenright}{\kern0pt}{\isachardoublequoteclose}\isanewline
\ \ \ \ \ \ \ \ \ \ \isacommand{using}\isamarkupfalse%
\ state{\isacharunderscore}{\kern0pt}basis{\isacharunderscore}{\kern0pt}carrier{\isacharunderscore}{\kern0pt}mat\ state{\isacharunderscore}{\kern0pt}basis{\isacharunderscore}{\kern0pt}def\ ket{\isacharunderscore}{\kern0pt}vec{\isacharunderscore}{\kern0pt}def\ mat{\isacharunderscore}{\kern0pt}of{\isacharunderscore}{\kern0pt}cols{\isacharunderscore}{\kern0pt}list{\isacharunderscore}{\kern0pt}def\ \isanewline
\ \ \ \ \ \ \ \ \ \ \ \ ket{\isacharunderscore}{\kern0pt}one{\isacharunderscore}{\kern0pt}to{\isacharunderscore}{\kern0pt}mat{\isacharunderscore}{\kern0pt}of{\isacharunderscore}{\kern0pt}cols{\isacharunderscore}{\kern0pt}list\isanewline
\ \ \ \ \ \ \ \ \ \ \isacommand{by}\isamarkupfalse%
\ auto\isanewline
\ \ \ \ \ \ \ \ \isacommand{also}\isamarkupfalse%
\ \isacommand{have}\isamarkupfalse%
\ {\isachardoublequoteopen}{\isasymdots}\ {\isacharequal}{\kern0pt}\ {\isacharbar}{\kern0pt}state{\isacharunderscore}{\kern0pt}basis\ {\isacharparenleft}{\kern0pt}Suc\ n{\isacharparenright}{\kern0pt}\ j{\isasymrangle}\ {\isachardollar}{\kern0pt}{\isachardollar}{\kern0pt}\ {\isacharparenleft}{\kern0pt}i{\isacharcomma}{\kern0pt}{\isadigit{0}}{\isacharparenright}{\kern0pt}{\isachardoublequoteclose}\isanewline
\ \ \ \ \ \ \ \ \isacommand{proof}\isamarkupfalse%
\ {\isacharminus}{\kern0pt}\isanewline
\ \ \ \ \ \ \ \ \ \ \isacommand{define}\isamarkupfalse%
\ id\ im\ \isakeyword{where}\ {\isachardoublequoteopen}id\ {\isacharequal}{\kern0pt}\ i\ div\ {\isadigit{2}}{\isacharcircum}{\kern0pt}n{\isachardoublequoteclose}\ \isakeyword{and}\ {\isachardoublequoteopen}im\ {\isacharequal}{\kern0pt}\ i\ mod\ {\isadigit{2}}{\isacharcircum}{\kern0pt}n{\isachardoublequoteclose}\isanewline
\ \ \ \ \ \ \ \ \ \ \isacommand{have}\isamarkupfalse%
\ i{\isacharunderscore}{\kern0pt}dec{\isacharcolon}{\kern0pt}{\isachardoublequoteopen}i\ {\isacharequal}{\kern0pt}\ id{\isacharasterisk}{\kern0pt}{\isacharparenleft}{\kern0pt}{\isadigit{2}}{\isacharcircum}{\kern0pt}n{\isacharparenright}{\kern0pt}\ {\isacharplus}{\kern0pt}\ im{\isachardoublequoteclose}\ \isacommand{using}\isamarkupfalse%
\ id{\isacharunderscore}{\kern0pt}def\ im{\isacharunderscore}{\kern0pt}def\ \isacommand{by}\isamarkupfalse%
\ presburger\isanewline
\ \ \ \ \ \ \ \ \ \ \isacommand{show}\isamarkupfalse%
\ {\isacharquery}{\kern0pt}thesis\isanewline
\ \ \ \ \ \ \ \ \ \ \isacommand{proof}\isamarkupfalse%
\ {\isacharparenleft}{\kern0pt}rule\ disjE{\isacharparenright}{\kern0pt}\isanewline
\ \ \ \ \ \ \ \ \ \ \ \ \isacommand{show}\isamarkupfalse%
\ {\isachardoublequoteopen}id\ {\isacharequal}{\kern0pt}\ {\isadigit{0}}\ {\isasymor}\ id\ {\isacharequal}{\kern0pt}\ {\isadigit{1}}{\isachardoublequoteclose}\ \isacommand{using}\isamarkupfalse%
\ id{\isacharunderscore}{\kern0pt}def\ il\isanewline
\ \ \ \ \ \ \ \ \ \ \ \ \ \ \isacommand{by}\isamarkupfalse%
\ {\isacharparenleft}{\kern0pt}metis\ One{\isacharunderscore}{\kern0pt}nat{\isacharunderscore}{\kern0pt}def\ less{\isacharunderscore}{\kern0pt}{\isadigit{2}}{\isacharunderscore}{\kern0pt}cases\ less{\isacharunderscore}{\kern0pt}power{\isacharunderscore}{\kern0pt}add{\isacharunderscore}{\kern0pt}imp{\isacharunderscore}{\kern0pt}div{\isacharunderscore}{\kern0pt}less\ plus{\isacharunderscore}{\kern0pt}{\isadigit{1}}{\isacharunderscore}{\kern0pt}eq{\isacharunderscore}{\kern0pt}Suc\ \isanewline
\ \ \ \ \ \ \ \ \ \ \ \ \ \ \ \ \ \ power{\isacharunderscore}{\kern0pt}one{\isacharunderscore}{\kern0pt}right{\isacharparenright}{\kern0pt}\isanewline
\ \ \ \ \ \ \ \ \ \ \isacommand{next}\isamarkupfalse%
\isanewline
\ \ \ \ \ \ \ \ \ \ \ \ \isacommand{assume}\isamarkupfalse%
\ id{\isadigit{0}}{\isacharcolon}{\kern0pt}{\isachardoublequoteopen}id\ {\isacharequal}{\kern0pt}\ {\isadigit{0}}{\isachardoublequoteclose}\isanewline
\ \ \ \ \ \ \ \ \ \ \ \ \isacommand{hence}\isamarkupfalse%
\ iim{\isacharcolon}{\kern0pt}{\isachardoublequoteopen}i\ {\isacharequal}{\kern0pt}\ im{\isachardoublequoteclose}\ \isacommand{using}\isamarkupfalse%
\ i{\isacharunderscore}{\kern0pt}dec\ \isacommand{by}\isamarkupfalse%
\ presburger\isanewline
\ \ \ \ \ \ \ \ \ \ \ \ \isacommand{have}\isamarkupfalse%
\ {\isachardoublequoteopen}mat{\isacharunderscore}{\kern0pt}of{\isacharunderscore}{\kern0pt}cols{\isacharunderscore}{\kern0pt}list\ {\isadigit{2}}\ {\isacharbrackleft}{\kern0pt}{\isacharbrackleft}{\kern0pt}{\isadigit{0}}{\isacharcomma}{\kern0pt}{\isadigit{1}}{\isacharbrackright}{\kern0pt}{\isacharbrackright}{\kern0pt}\ {\isachardollar}{\kern0pt}{\isachardollar}{\kern0pt}\ {\isacharparenleft}{\kern0pt}i\ div\ {\isadigit{2}}{\isacharcircum}{\kern0pt}n{\isacharcomma}{\kern0pt}{\isadigit{0}}{\isacharparenright}{\kern0pt}\ {\isacharasterisk}{\kern0pt}\ {\isacharbar}{\kern0pt}state{\isacharunderscore}{\kern0pt}basis\ n\ jm{\isasymrangle}\ {\isachardollar}{\kern0pt}{\isachardollar}{\kern0pt}\ {\isacharparenleft}{\kern0pt}i\ mod\ {\isadigit{2}}{\isacharcircum}{\kern0pt}n{\isacharcomma}{\kern0pt}{\isadigit{0}}{\isacharparenright}{\kern0pt}\isanewline
\ \ \ \ \ \ \ \ \ \ \ \ \ \ \ \ {\isacharequal}{\kern0pt}\ mat{\isacharunderscore}{\kern0pt}of{\isacharunderscore}{\kern0pt}cols{\isacharunderscore}{\kern0pt}list\ {\isadigit{2}}\ {\isacharbrackleft}{\kern0pt}{\isacharbrackleft}{\kern0pt}{\isadigit{0}}{\isacharcomma}{\kern0pt}{\isadigit{1}}{\isacharbrackright}{\kern0pt}{\isacharbrackright}{\kern0pt}\ {\isachardollar}{\kern0pt}{\isachardollar}{\kern0pt}\ {\isacharparenleft}{\kern0pt}{\isadigit{0}}{\isacharcomma}{\kern0pt}{\isadigit{0}}{\isacharparenright}{\kern0pt}\ {\isacharasterisk}{\kern0pt}\ {\isacharbar}{\kern0pt}state{\isacharunderscore}{\kern0pt}basis\ n\ jm{\isasymrangle}\ {\isachardollar}{\kern0pt}{\isachardollar}{\kern0pt}\ {\isacharparenleft}{\kern0pt}im{\isacharcomma}{\kern0pt}{\isadigit{0}}{\isacharparenright}{\kern0pt}{\isachardoublequoteclose}\isanewline
\ \ \ \ \ \ \ \ \ \ \ \ \ \ \isacommand{using}\isamarkupfalse%
\ id{\isadigit{0}}\ id{\isacharunderscore}{\kern0pt}def\ im{\isacharunderscore}{\kern0pt}def\ \isacommand{by}\isamarkupfalse%
\ simp\isanewline
\ \ \ \ \ \ \ \ \ \ \ \ \isacommand{also}\isamarkupfalse%
\ \isacommand{have}\isamarkupfalse%
\ {\isachardoublequoteopen}{\isasymdots}\ {\isacharequal}{\kern0pt}\ {\isadigit{0}}{\isachardoublequoteclose}\ \isacommand{using}\isamarkupfalse%
\ mat{\isacharunderscore}{\kern0pt}of{\isacharunderscore}{\kern0pt}cols{\isacharunderscore}{\kern0pt}list{\isacharunderscore}{\kern0pt}def\ \isacommand{by}\isamarkupfalse%
\ auto\isanewline
\ \ \ \ \ \ \ \ \ \ \ \ \isacommand{also}\isamarkupfalse%
\ \isacommand{have}\isamarkupfalse%
\ {\isachardoublequoteopen}{\isasymdots}\ {\isacharequal}{\kern0pt}\ {\isacharbar}{\kern0pt}state{\isacharunderscore}{\kern0pt}basis\ {\isacharparenleft}{\kern0pt}Suc\ n{\isacharparenright}{\kern0pt}\ j{\isasymrangle}\ {\isachardollar}{\kern0pt}{\isachardollar}{\kern0pt}\ {\isacharparenleft}{\kern0pt}im{\isacharcomma}{\kern0pt}{\isadigit{0}}{\isacharparenright}{\kern0pt}{\isachardoublequoteclose}\isanewline
\ \ \ \ \ \ \ \ \ \ \ \ \ \ \isacommand{using}\isamarkupfalse%
\ state{\isacharunderscore}{\kern0pt}basis{\isacharunderscore}{\kern0pt}def\ ket{\isacharunderscore}{\kern0pt}vec{\isacharunderscore}{\kern0pt}def\ j{\isacharunderscore}{\kern0pt}dec{\isadigit{2}}\ assms\ id{\isadigit{0}}\ iim\ il\ local{\isachardot}{\kern0pt}id{\isacharunderscore}{\kern0pt}def\ \isacommand{by}\isamarkupfalse%
\ force\isanewline
\ \ \ \ \ \ \ \ \ \ \ \ \isacommand{also}\isamarkupfalse%
\ \isacommand{have}\isamarkupfalse%
\ {\isachardoublequoteopen}{\isasymdots}\ {\isacharequal}{\kern0pt}\ {\isacharbar}{\kern0pt}state{\isacharunderscore}{\kern0pt}basis\ {\isacharparenleft}{\kern0pt}Suc\ n{\isacharparenright}{\kern0pt}\ j{\isasymrangle}\ {\isachardollar}{\kern0pt}{\isachardollar}{\kern0pt}\ {\isacharparenleft}{\kern0pt}i{\isacharcomma}{\kern0pt}{\isadigit{0}}{\isacharparenright}{\kern0pt}{\isachardoublequoteclose}\ \isacommand{using}\isamarkupfalse%
\ iim\ \isacommand{by}\isamarkupfalse%
\ simp\isanewline
\ \ \ \ \ \ \ \ \ \ \ \ \isacommand{finally}\isamarkupfalse%
\ \isacommand{show}\isamarkupfalse%
\ {\isacharquery}{\kern0pt}thesis\ \isacommand{by}\isamarkupfalse%
\ this\isanewline
\ \ \ \ \ \ \ \ \ \ \isacommand{next}\isamarkupfalse%
\isanewline
\ \ \ \ \ \ \ \ \ \ \ \ \isacommand{assume}\isamarkupfalse%
\ id{\isadigit{1}}{\isacharcolon}{\kern0pt}{\isachardoublequoteopen}id\ {\isacharequal}{\kern0pt}\ {\isadigit{1}}{\isachardoublequoteclose}\isanewline
\ \ \ \ \ \ \ \ \ \ \ \ \isacommand{hence}\isamarkupfalse%
\ i{\isadigit{2}}m{\isacharcolon}{\kern0pt}{\isachardoublequoteopen}i\ {\isacharequal}{\kern0pt}\ {\isadigit{2}}{\isacharcircum}{\kern0pt}n\ {\isacharplus}{\kern0pt}\ im{\isachardoublequoteclose}\ \isacommand{using}\isamarkupfalse%
\ i{\isacharunderscore}{\kern0pt}dec\ \isacommand{by}\isamarkupfalse%
\ presburger\isanewline
\ \ \ \ \ \ \ \ \ \ \ \ \isacommand{have}\isamarkupfalse%
\ {\isachardoublequoteopen}mat{\isacharunderscore}{\kern0pt}of{\isacharunderscore}{\kern0pt}cols{\isacharunderscore}{\kern0pt}list\ {\isadigit{2}}\ {\isacharbrackleft}{\kern0pt}{\isacharbrackleft}{\kern0pt}{\isadigit{0}}{\isacharcomma}{\kern0pt}{\isadigit{1}}{\isacharbrackright}{\kern0pt}{\isacharbrackright}{\kern0pt}\ {\isachardollar}{\kern0pt}{\isachardollar}{\kern0pt}\ {\isacharparenleft}{\kern0pt}i\ div\ {\isadigit{2}}{\isacharcircum}{\kern0pt}n{\isacharcomma}{\kern0pt}{\isadigit{0}}{\isacharparenright}{\kern0pt}\ {\isacharasterisk}{\kern0pt}\ {\isacharbar}{\kern0pt}state{\isacharunderscore}{\kern0pt}basis\ n\ jm{\isasymrangle}\ {\isachardollar}{\kern0pt}{\isachardollar}{\kern0pt}\ {\isacharparenleft}{\kern0pt}i\ mod\ {\isadigit{2}}{\isacharcircum}{\kern0pt}n{\isacharcomma}{\kern0pt}{\isadigit{0}}{\isacharparenright}{\kern0pt}\isanewline
\ \ \ \ \ \ \ \ \ \ \ \ \ \ \ \ {\isacharequal}{\kern0pt}\ mat{\isacharunderscore}{\kern0pt}of{\isacharunderscore}{\kern0pt}cols{\isacharunderscore}{\kern0pt}list\ {\isadigit{2}}\ {\isacharbrackleft}{\kern0pt}{\isacharbrackleft}{\kern0pt}{\isadigit{0}}{\isacharcomma}{\kern0pt}{\isadigit{1}}{\isacharbrackright}{\kern0pt}{\isacharbrackright}{\kern0pt}\ {\isachardollar}{\kern0pt}{\isachardollar}{\kern0pt}\ {\isacharparenleft}{\kern0pt}{\isadigit{1}}{\isacharcomma}{\kern0pt}{\isadigit{0}}{\isacharparenright}{\kern0pt}\ {\isacharasterisk}{\kern0pt}\ {\isacharbar}{\kern0pt}state{\isacharunderscore}{\kern0pt}basis\ n\ jm{\isasymrangle}\ {\isachardollar}{\kern0pt}{\isachardollar}{\kern0pt}\ {\isacharparenleft}{\kern0pt}im{\isacharcomma}{\kern0pt}{\isadigit{0}}{\isacharparenright}{\kern0pt}{\isachardoublequoteclose}\isanewline
\ \ \ \ \ \ \ \ \ \ \ \ \ \ \isacommand{using}\isamarkupfalse%
\ id{\isadigit{1}}\ id{\isacharunderscore}{\kern0pt}def\ im{\isacharunderscore}{\kern0pt}def\ \isacommand{by}\isamarkupfalse%
\ simp\isanewline
\ \ \ \ \ \ \ \ \ \ \ \ \isacommand{also}\isamarkupfalse%
\ \isacommand{have}\isamarkupfalse%
\ {\isachardoublequoteopen}{\isasymdots}\ {\isacharequal}{\kern0pt}\ {\isacharbar}{\kern0pt}state{\isacharunderscore}{\kern0pt}basis\ n\ jm{\isasymrangle}\ {\isachardollar}{\kern0pt}{\isachardollar}{\kern0pt}\ {\isacharparenleft}{\kern0pt}im{\isacharcomma}{\kern0pt}{\isadigit{0}}{\isacharparenright}{\kern0pt}{\isachardoublequoteclose}\ \isacommand{using}\isamarkupfalse%
\ mat{\isacharunderscore}{\kern0pt}of{\isacharunderscore}{\kern0pt}cols{\isacharunderscore}{\kern0pt}list{\isacharunderscore}{\kern0pt}def\ \isacommand{by}\isamarkupfalse%
\ auto\isanewline
\ \ \ \ \ \ \ \ \ \ \ \ \isacommand{also}\isamarkupfalse%
\ \isacommand{have}\isamarkupfalse%
\ {\isachardoublequoteopen}{\isasymdots}\ {\isacharequal}{\kern0pt}\ {\isacharbar}{\kern0pt}state{\isacharunderscore}{\kern0pt}basis\ {\isacharparenleft}{\kern0pt}Suc\ n{\isacharparenright}{\kern0pt}\ j{\isasymrangle}\ {\isachardollar}{\kern0pt}{\isachardollar}{\kern0pt}\ {\isacharparenleft}{\kern0pt}i{\isacharcomma}{\kern0pt}{\isadigit{0}}{\isacharparenright}{\kern0pt}{\isachardoublequoteclose}\isanewline
\ \ \ \ \ \ \ \ \ \ \ \ \ \ \isacommand{using}\isamarkupfalse%
\ i{\isadigit{2}}m\ j{\isacharunderscore}{\kern0pt}dec{\isadigit{2}}\ il\ assms\ state{\isacharunderscore}{\kern0pt}basis{\isacharunderscore}{\kern0pt}def\ \isacommand{by}\isamarkupfalse%
\ auto\isanewline
\ \ \ \ \ \ \ \ \ \ \ \ \isacommand{finally}\isamarkupfalse%
\ \isacommand{show}\isamarkupfalse%
\ {\isacharquery}{\kern0pt}thesis\ \isacommand{by}\isamarkupfalse%
\ this\isanewline
\ \ \ \ \ \ \ \ \ \ \isacommand{qed}\isamarkupfalse%
\isanewline
\ \ \ \ \ \ \ \ \isacommand{qed}\isamarkupfalse%
\isanewline
\ \ \ \ \ \ \ \ \isacommand{finally}\isamarkupfalse%
\ \isacommand{show}\isamarkupfalse%
\ {\isachardoublequoteopen}{\isacharparenleft}{\kern0pt}\ {\isacharbar}{\kern0pt}state{\isacharunderscore}{\kern0pt}basis\ {\isadigit{1}}\ {\isacharparenleft}{\kern0pt}j\ div\ {\isadigit{2}}\ {\isacharcircum}{\kern0pt}\ n{\isacharparenright}{\kern0pt}{\isasymrangle}\ {\isasymOtimes}\ {\isacharbar}{\kern0pt}state{\isacharunderscore}{\kern0pt}basis\ n\ {\isacharparenleft}{\kern0pt}j\ mod\ {\isadigit{2}}\ {\isacharcircum}{\kern0pt}\ n{\isacharparenright}{\kern0pt}{\isasymrangle}{\isacharparenright}{\kern0pt}\ {\isachardollar}{\kern0pt}{\isachardollar}{\kern0pt}\ {\isacharparenleft}{\kern0pt}i{\isacharcomma}{\kern0pt}\ ja{\isacharparenright}{\kern0pt}\ {\isacharequal}{\kern0pt}\isanewline
\ \ \ \ \ \ \ \ \ \ \ \ \ \ \ \ \ \ \ \ \ \ {\isacharbar}{\kern0pt}state{\isacharunderscore}{\kern0pt}basis\ {\isacharparenleft}{\kern0pt}Suc\ n{\isacharparenright}{\kern0pt}\ j{\isasymrangle}\ {\isachardollar}{\kern0pt}{\isachardollar}{\kern0pt}\ {\isacharparenleft}{\kern0pt}i{\isacharcomma}{\kern0pt}\ ja{\isacharparenright}{\kern0pt}{\isachardoublequoteclose}\ \isanewline
\ \ \ \ \ \ \ \ \ \ \isacommand{using}\isamarkupfalse%
\ ja{\isadigit{0}}\ jd{\isacharunderscore}{\kern0pt}def\ jm{\isacharunderscore}{\kern0pt}def\ \isacommand{by}\isamarkupfalse%
\ auto\isanewline
\ \ \ \ \ \ \isacommand{qed}\isamarkupfalse%
\isanewline
\ \ \ \ \isacommand{next}\isamarkupfalse%
\isanewline
\ \ \ \ \ \ \isacommand{show}\isamarkupfalse%
\ {\isachardoublequoteopen}dim{\isacharunderscore}{\kern0pt}row\ {\isacharparenleft}{\kern0pt}\ {\isacharbar}{\kern0pt}state{\isacharunderscore}{\kern0pt}basis\ {\isadigit{1}}\ {\isacharparenleft}{\kern0pt}j\ div\ {\isadigit{2}}\ {\isacharcircum}{\kern0pt}\ n{\isacharparenright}{\kern0pt}{\isasymrangle}\ {\isasymOtimes}\ {\isacharbar}{\kern0pt}state{\isacharunderscore}{\kern0pt}basis\ n\ {\isacharparenleft}{\kern0pt}j\ mod\ {\isadigit{2}}\ {\isacharcircum}{\kern0pt}\ n{\isacharparenright}{\kern0pt}{\isasymrangle}{\isacharparenright}{\kern0pt}\ {\isacharequal}{\kern0pt}\isanewline
\ \ \ \ \ \ \ \ \ \ \ \ dim{\isacharunderscore}{\kern0pt}row\ {\isacharbar}{\kern0pt}state{\isacharunderscore}{\kern0pt}basis\ {\isacharparenleft}{\kern0pt}Suc\ n{\isacharparenright}{\kern0pt}\ j{\isasymrangle}{\isachardoublequoteclose}\isanewline
\ \ \ \ \ \ \ \ \isacommand{using}\isamarkupfalse%
\ state{\isacharunderscore}{\kern0pt}basis{\isacharunderscore}{\kern0pt}def\ state{\isacharunderscore}{\kern0pt}basis{\isacharunderscore}{\kern0pt}carrier{\isacharunderscore}{\kern0pt}mat\ ket{\isacharunderscore}{\kern0pt}vec{\isacharunderscore}{\kern0pt}def\ \isacommand{by}\isamarkupfalse%
\ simp\isanewline
\ \ \ \ \isacommand{next}\isamarkupfalse%
\isanewline
\ \ \ \ \ \ \isacommand{show}\isamarkupfalse%
\ {\isachardoublequoteopen}dim{\isacharunderscore}{\kern0pt}col\ {\isacharparenleft}{\kern0pt}\ {\isacharbar}{\kern0pt}state{\isacharunderscore}{\kern0pt}basis\ {\isadigit{1}}\ {\isacharparenleft}{\kern0pt}j\ div\ {\isadigit{2}}\ {\isacharcircum}{\kern0pt}\ n{\isacharparenright}{\kern0pt}{\isasymrangle}\ {\isasymOtimes}\ {\isacharbar}{\kern0pt}state{\isacharunderscore}{\kern0pt}basis\ n\ {\isacharparenleft}{\kern0pt}j\ mod\ {\isadigit{2}}\ {\isacharcircum}{\kern0pt}\ n{\isacharparenright}{\kern0pt}{\isasymrangle}{\isacharparenright}{\kern0pt}\ {\isacharequal}{\kern0pt}\isanewline
\ \ \ \ \ \ \ \ \ \ \ \ dim{\isacharunderscore}{\kern0pt}col\ {\isacharbar}{\kern0pt}state{\isacharunderscore}{\kern0pt}basis\ {\isacharparenleft}{\kern0pt}Suc\ n{\isacharparenright}{\kern0pt}\ j{\isasymrangle}{\isachardoublequoteclose}\isanewline
\ \ \ \ \ \ \ \ \isacommand{using}\isamarkupfalse%
\ state{\isacharunderscore}{\kern0pt}basis{\isacharunderscore}{\kern0pt}def\ state{\isacharunderscore}{\kern0pt}basis{\isacharunderscore}{\kern0pt}carrier{\isacharunderscore}{\kern0pt}mat\ ket{\isacharunderscore}{\kern0pt}vec{\isacharunderscore}{\kern0pt}def\ \isacommand{by}\isamarkupfalse%
\ simp\isanewline
\ \ \ \ \isacommand{qed}\isamarkupfalse%
\isanewline
\ \ \isacommand{qed}\isamarkupfalse%
\isanewline
\isacommand{qed}\isamarkupfalse%
%
\endisatagproof
{\isafoldproof}%
%
\isadelimproof
\isanewline
%
\endisadelimproof
\isanewline
\isacommand{lemma}\isamarkupfalse%
\ state{\isacharunderscore}{\kern0pt}basis{\isacharunderscore}{\kern0pt}dec{\isacharprime}{\kern0pt}{\isacharcolon}{\kern0pt}\isanewline
\ \ {\isachardoublequoteopen}{\isasymforall}j{\isachardot}{\kern0pt}\ j\ {\isacharless}{\kern0pt}\ {\isadigit{2}}\ {\isacharcircum}{\kern0pt}\ Suc\ n\ {\isasymlongrightarrow}\ \isanewline
\ \ \ \ {\isacharbar}{\kern0pt}state{\isacharunderscore}{\kern0pt}basis\ n\ {\isacharparenleft}{\kern0pt}j\ div\ {\isadigit{2}}{\isacharparenright}{\kern0pt}{\isasymrangle}\ {\isasymOtimes}\ {\isacharbar}{\kern0pt}state{\isacharunderscore}{\kern0pt}basis\ {\isadigit{1}}\ {\isacharparenleft}{\kern0pt}j\ mod\ {\isadigit{2}}{\isacharparenright}{\kern0pt}{\isasymrangle}\ {\isacharequal}{\kern0pt}\ {\isacharbar}{\kern0pt}state{\isacharunderscore}{\kern0pt}basis\ {\isacharparenleft}{\kern0pt}Suc\ n{\isacharparenright}{\kern0pt}\ j{\isasymrangle}{\isachardoublequoteclose}\isanewline
%
\isadelimproof
%
\endisadelimproof
%
\isatagproof
\isacommand{proof}\isamarkupfalse%
\ {\isacharparenleft}{\kern0pt}induct\ n{\isacharparenright}{\kern0pt}\isanewline
\ \ \isacommand{case}\isamarkupfalse%
\ {\isadigit{0}}\isanewline
\ \ \isacommand{show}\isamarkupfalse%
\ {\isacharquery}{\kern0pt}case\isanewline
\ \ \isacommand{proof}\isamarkupfalse%
\ \isanewline
\ \ \ \ \isacommand{fix}\isamarkupfalse%
\ j{\isacharcolon}{\kern0pt}{\isacharcolon}{\kern0pt}nat\isanewline
\ \ \ \ \isacommand{show}\isamarkupfalse%
\ {\isachardoublequoteopen}j\ {\isacharless}{\kern0pt}\ {\isadigit{2}}\ {\isacharcircum}{\kern0pt}\ Suc\ {\isadigit{0}}\ {\isasymlongrightarrow}\isanewline
\ \ \ \ \ \ \ \ \ {\isacharbar}{\kern0pt}state{\isacharunderscore}{\kern0pt}basis\ {\isadigit{0}}\ {\isacharparenleft}{\kern0pt}j\ div\ {\isadigit{2}}{\isacharparenright}{\kern0pt}{\isasymrangle}\ {\isasymOtimes}\ {\isacharbar}{\kern0pt}state{\isacharunderscore}{\kern0pt}basis\ {\isadigit{1}}\ {\isacharparenleft}{\kern0pt}j\ mod\ {\isadigit{2}}{\isacharparenright}{\kern0pt}{\isasymrangle}\ {\isacharequal}{\kern0pt}\ {\isacharbar}{\kern0pt}state{\isacharunderscore}{\kern0pt}basis\ {\isacharparenleft}{\kern0pt}Suc\ {\isadigit{0}}{\isacharparenright}{\kern0pt}\ j{\isasymrangle}{\isachardoublequoteclose}\isanewline
\ \ \ \ \isacommand{proof}\isamarkupfalse%
\isanewline
\ \ \ \ \ \ \isacommand{assume}\isamarkupfalse%
\ {\isachardoublequoteopen}j\ {\isacharless}{\kern0pt}\ {\isadigit{2}}\ {\isacharcircum}{\kern0pt}\ Suc\ {\isadigit{0}}{\isachardoublequoteclose}\isanewline
\ \ \ \ \ \ \isacommand{hence}\isamarkupfalse%
\ j{\isadigit{2}}{\isacharcolon}{\kern0pt}{\isachardoublequoteopen}j\ {\isacharless}{\kern0pt}\ {\isadigit{2}}{\isachardoublequoteclose}\ \isacommand{by}\isamarkupfalse%
\ auto\isanewline
\ \ \ \ \ \ \isacommand{hence}\isamarkupfalse%
\ jd{\isadigit{0}}{\isacharcolon}{\kern0pt}{\isachardoublequoteopen}j\ div\ {\isadigit{2}}\ {\isacharequal}{\kern0pt}\ {\isadigit{0}}{\isachardoublequoteclose}\ \isacommand{by}\isamarkupfalse%
\ auto\isanewline
\ \ \ \ \ \ \isacommand{have}\isamarkupfalse%
\ jmj{\isacharcolon}{\kern0pt}{\isachardoublequoteopen}j\ mod\ {\isadigit{2}}\ {\isacharequal}{\kern0pt}\ j{\isachardoublequoteclose}\ \isacommand{using}\isamarkupfalse%
\ j{\isadigit{2}}\ \isacommand{by}\isamarkupfalse%
\ auto\isanewline
\ \ \ \ \ \ \isacommand{have}\isamarkupfalse%
\ {\isachardoublequoteopen}{\isacharbar}{\kern0pt}state{\isacharunderscore}{\kern0pt}basis\ {\isadigit{0}}\ {\isacharparenleft}{\kern0pt}j\ div\ {\isadigit{2}}{\isacharparenright}{\kern0pt}{\isasymrangle}\ {\isasymOtimes}\ {\isacharbar}{\kern0pt}state{\isacharunderscore}{\kern0pt}basis\ {\isadigit{1}}\ {\isacharparenleft}{\kern0pt}j\ mod\ {\isadigit{2}}{\isacharparenright}{\kern0pt}{\isasymrangle}\ {\isacharequal}{\kern0pt}\isanewline
\ \ \ \ \ \ \ \ \ \ \ \ {\isacharbar}{\kern0pt}state{\isacharunderscore}{\kern0pt}basis\ {\isadigit{0}}\ {\isadigit{0}}{\isasymrangle}\ {\isasymOtimes}\ {\isacharbar}{\kern0pt}state{\isacharunderscore}{\kern0pt}basis\ {\isadigit{1}}\ j{\isasymrangle}{\isachardoublequoteclose}\isanewline
\ \ \ \ \ \ \ \ \isacommand{using}\isamarkupfalse%
\ jmj\ jd{\isadigit{0}}\ \isacommand{by}\isamarkupfalse%
\ simp\isanewline
\ \ \ \ \ \ \isacommand{also}\isamarkupfalse%
\ \isacommand{have}\isamarkupfalse%
\ {\isachardoublequoteopen}{\isasymdots}\ {\isacharequal}{\kern0pt}\ {\isacharparenleft}{\kern0pt}{\isadigit{1}}\isactrlsub m\ {\isadigit{1}}{\isacharparenright}{\kern0pt}\ {\isasymOtimes}\ {\isacharbar}{\kern0pt}state{\isacharunderscore}{\kern0pt}basis\ {\isadigit{1}}\ j{\isasymrangle}{\isachardoublequoteclose}\isanewline
\ \ \ \ \ \ \ \ \isacommand{using}\isamarkupfalse%
\ state{\isacharunderscore}{\kern0pt}basis{\isacharunderscore}{\kern0pt}def\ unit{\isacharunderscore}{\kern0pt}vec{\isacharunderscore}{\kern0pt}def\ ket{\isacharunderscore}{\kern0pt}vec{\isacharunderscore}{\kern0pt}def\ \isacommand{by}\isamarkupfalse%
\ auto\isanewline
\ \ \ \ \ \ \isacommand{also}\isamarkupfalse%
\ \isacommand{have}\isamarkupfalse%
\ {\isachardoublequoteopen}{\isasymdots}\ {\isacharequal}{\kern0pt}\ {\isacharbar}{\kern0pt}state{\isacharunderscore}{\kern0pt}basis\ {\isadigit{1}}\ j{\isasymrangle}{\isachardoublequoteclose}\ \isacommand{using}\isamarkupfalse%
\ left{\isacharunderscore}{\kern0pt}tensor{\isacharunderscore}{\kern0pt}id\ \isacommand{by}\isamarkupfalse%
\ blast\isanewline
\ \ \ \ \ \ \isacommand{finally}\isamarkupfalse%
\ \isacommand{show}\isamarkupfalse%
\ {\isachardoublequoteopen}{\isacharbar}{\kern0pt}state{\isacharunderscore}{\kern0pt}basis\ {\isadigit{0}}\ {\isacharparenleft}{\kern0pt}j\ div\ {\isadigit{2}}{\isacharparenright}{\kern0pt}{\isasymrangle}\ {\isasymOtimes}\ {\isacharbar}{\kern0pt}state{\isacharunderscore}{\kern0pt}basis\ {\isadigit{1}}\ {\isacharparenleft}{\kern0pt}j\ mod\ {\isadigit{2}}{\isacharparenright}{\kern0pt}{\isasymrangle}\ {\isacharequal}{\kern0pt}\ {\isacharbar}{\kern0pt}state{\isacharunderscore}{\kern0pt}basis\ {\isacharparenleft}{\kern0pt}Suc\ {\isadigit{0}}{\isacharparenright}{\kern0pt}\ j{\isasymrangle}{\isachardoublequoteclose}\isanewline
\ \ \ \ \ \ \ \ \isacommand{by}\isamarkupfalse%
\ auto\isanewline
\ \ \ \ \isacommand{qed}\isamarkupfalse%
\isanewline
\ \ \isacommand{qed}\isamarkupfalse%
\isanewline
\isacommand{next}\isamarkupfalse%
\isanewline
\ \ \isacommand{case}\isamarkupfalse%
\ {\isacharparenleft}{\kern0pt}Suc\ n{\isacharparenright}{\kern0pt}\isanewline
\ \ \isacommand{assume}\isamarkupfalse%
\ HI{\isacharcolon}{\kern0pt}{\isachardoublequoteopen}{\isasymforall}j{\isacharless}{\kern0pt}{\isadigit{2}}\ {\isacharcircum}{\kern0pt}\ Suc\ n{\isachardot}{\kern0pt}\ {\isacharbar}{\kern0pt}state{\isacharunderscore}{\kern0pt}basis\ n\ {\isacharparenleft}{\kern0pt}j\ div\ {\isadigit{2}}{\isacharparenright}{\kern0pt}{\isasymrangle}\ {\isasymOtimes}\ {\isacharbar}{\kern0pt}state{\isacharunderscore}{\kern0pt}basis\ {\isadigit{1}}\ {\isacharparenleft}{\kern0pt}j\ mod\ {\isadigit{2}}{\isacharparenright}{\kern0pt}{\isasymrangle}\ {\isacharequal}{\kern0pt}\isanewline
\ \ \ \ \ \ \ \ \ \ \ \ \ \ \ \ \ \ \ \ \ \ \ \ \ \ \ {\isacharbar}{\kern0pt}state{\isacharunderscore}{\kern0pt}basis\ {\isacharparenleft}{\kern0pt}Suc\ n{\isacharparenright}{\kern0pt}\ j{\isasymrangle}{\isachardoublequoteclose}\isanewline
\ \ \isacommand{define}\isamarkupfalse%
\ m\ \isakeyword{where}\ {\isachardoublequoteopen}m\ {\isacharequal}{\kern0pt}\ Suc\ n{\isachardoublequoteclose}\isanewline
\ \ \isacommand{show}\isamarkupfalse%
\ {\isacharquery}{\kern0pt}case\isanewline
\ \ \isacommand{proof}\isamarkupfalse%
\ \isanewline
\ \ \ \ \isacommand{fix}\isamarkupfalse%
\ j{\isacharcolon}{\kern0pt}{\isacharcolon}{\kern0pt}nat\isanewline
\ \ \ \ \isacommand{show}\isamarkupfalse%
\ {\isachardoublequoteopen}j\ {\isacharless}{\kern0pt}\ {\isadigit{2}}\ {\isacharcircum}{\kern0pt}\ Suc\ {\isacharparenleft}{\kern0pt}Suc\ n{\isacharparenright}{\kern0pt}\ {\isasymlongrightarrow}\isanewline
\ \ \ \ \ \ \ {\isacharbar}{\kern0pt}state{\isacharunderscore}{\kern0pt}basis\ {\isacharparenleft}{\kern0pt}Suc\ n{\isacharparenright}{\kern0pt}\ {\isacharparenleft}{\kern0pt}j\ div\ {\isadigit{2}}{\isacharparenright}{\kern0pt}{\isasymrangle}\ {\isasymOtimes}\ {\isacharbar}{\kern0pt}state{\isacharunderscore}{\kern0pt}basis\ {\isadigit{1}}\ {\isacharparenleft}{\kern0pt}j\ mod\ {\isadigit{2}}{\isacharparenright}{\kern0pt}{\isasymrangle}\ {\isacharequal}{\kern0pt}\ {\isacharbar}{\kern0pt}state{\isacharunderscore}{\kern0pt}basis\ {\isacharparenleft}{\kern0pt}Suc\ {\isacharparenleft}{\kern0pt}Suc\ n{\isacharparenright}{\kern0pt}{\isacharparenright}{\kern0pt}\ j{\isasymrangle}{\isachardoublequoteclose}\isanewline
\ \ \ \ \isacommand{proof}\isamarkupfalse%
\ \isanewline
\ \ \ \ \ \ \isacommand{assume}\isamarkupfalse%
\ jleq{\isacharcolon}{\kern0pt}{\isachardoublequoteopen}j\ {\isacharless}{\kern0pt}\ {\isadigit{2}}\ {\isacharcircum}{\kern0pt}\ Suc\ {\isacharparenleft}{\kern0pt}Suc\ n{\isacharparenright}{\kern0pt}{\isachardoublequoteclose}\isanewline
\ \ \ \ \ \ \isacommand{define}\isamarkupfalse%
\ jd{\isadigit{2}}\ \isakeyword{where}\ {\isachardoublequoteopen}jd{\isadigit{2}}\ {\isacharequal}{\kern0pt}\ j\ div\ {\isadigit{2}}{\isachardoublequoteclose}\isanewline
\ \ \ \ \ \ \isacommand{define}\isamarkupfalse%
\ jm{\isadigit{2}}\ \isakeyword{where}\ {\isachardoublequoteopen}jm{\isadigit{2}}\ {\isacharequal}{\kern0pt}\ j\ mod\ {\isadigit{2}}{\isachardoublequoteclose}\isanewline
\ \ \ \ \ \ \isacommand{define}\isamarkupfalse%
\ jd{\isadigit{2}}m\ \isakeyword{where}\ {\isachardoublequoteopen}jd{\isadigit{2}}m\ {\isacharequal}{\kern0pt}\ j\ div\ {\isadigit{2}}{\isacharcircum}{\kern0pt}m{\isachardoublequoteclose}\isanewline
\ \ \ \ \ \ \isacommand{define}\isamarkupfalse%
\ jm{\isadigit{2}}m\ \isakeyword{where}\ {\isachardoublequoteopen}jm{\isadigit{2}}m\ {\isacharequal}{\kern0pt}\ j\ mod\ {\isadigit{2}}{\isacharcircum}{\kern0pt}m{\isachardoublequoteclose}\isanewline
\ \ \ \ \ \ \isacommand{define}\isamarkupfalse%
\ jmm\ \isakeyword{where}\ {\isachardoublequoteopen}jmm\ {\isacharequal}{\kern0pt}\ jd{\isadigit{2}}\ mod\ {\isadigit{2}}{\isacharcircum}{\kern0pt}n{\isachardoublequoteclose}\isanewline
\ \ \ \ \ \ \isacommand{have}\isamarkupfalse%
\ {\isachardoublequoteopen}{\isacharbar}{\kern0pt}state{\isacharunderscore}{\kern0pt}basis\ m\ jd{\isadigit{2}}{\isasymrangle}\ {\isasymOtimes}\ {\isacharbar}{\kern0pt}state{\isacharunderscore}{\kern0pt}basis\ {\isadigit{1}}\ jm{\isadigit{2}}{\isasymrangle}\ {\isacharequal}{\kern0pt}\isanewline
\ \ \ \ \ \ \ \ \ \ \ \ {\isacharparenleft}{\kern0pt}\ {\isacharbar}{\kern0pt}state{\isacharunderscore}{\kern0pt}basis\ {\isadigit{1}}\ jd{\isadigit{2}}m{\isasymrangle}\ {\isasymOtimes}\ {\isacharbar}{\kern0pt}state{\isacharunderscore}{\kern0pt}basis\ n\ jmm{\isasymrangle}{\isacharparenright}{\kern0pt}\ {\isasymOtimes}\ {\isacharbar}{\kern0pt}state{\isacharunderscore}{\kern0pt}basis\ {\isadigit{1}}\ jm{\isadigit{2}}{\isasymrangle}{\isachardoublequoteclose}\isanewline
\ \ \ \ \ \ \ \ \isacommand{using}\isamarkupfalse%
\ jleq\ state{\isacharunderscore}{\kern0pt}basis{\isacharunderscore}{\kern0pt}dec\ m{\isacharunderscore}{\kern0pt}def\ jd{\isadigit{2}}{\isacharunderscore}{\kern0pt}def\ jm{\isadigit{2}}{\isacharunderscore}{\kern0pt}def\ jd{\isadigit{2}}m{\isacharunderscore}{\kern0pt}def\ jmm{\isacharunderscore}{\kern0pt}def\ jm{\isadigit{2}}{\isacharunderscore}{\kern0pt}def\isanewline
\ \ \ \ \ \ \ \ \isacommand{by}\isamarkupfalse%
\ {\isacharparenleft}{\kern0pt}metis\ Suc{\isacharunderscore}{\kern0pt}eq{\isacharunderscore}{\kern0pt}plus{\isadigit{1}}\ div{\isacharunderscore}{\kern0pt}exp{\isacharunderscore}{\kern0pt}eq\ less{\isacharunderscore}{\kern0pt}power{\isacharunderscore}{\kern0pt}add{\isacharunderscore}{\kern0pt}imp{\isacharunderscore}{\kern0pt}div{\isacharunderscore}{\kern0pt}less\ plus{\isacharunderscore}{\kern0pt}{\isadigit{1}}{\isacharunderscore}{\kern0pt}eq{\isacharunderscore}{\kern0pt}Suc\ power{\isacharunderscore}{\kern0pt}one{\isacharunderscore}{\kern0pt}right{\isacharparenright}{\kern0pt}\isanewline
\ \ \ \ \ \ \isacommand{also}\isamarkupfalse%
\ \isacommand{have}\isamarkupfalse%
\ {\isachardoublequoteopen}{\isasymdots}\ {\isacharequal}{\kern0pt}\ {\isacharbar}{\kern0pt}state{\isacharunderscore}{\kern0pt}basis\ {\isadigit{1}}\ jd{\isadigit{2}}m{\isasymrangle}\ {\isasymOtimes}\ {\isacharparenleft}{\kern0pt}\ {\isacharbar}{\kern0pt}state{\isacharunderscore}{\kern0pt}basis\ n\ jmm{\isasymrangle}\ {\isasymOtimes}\ {\isacharbar}{\kern0pt}state{\isacharunderscore}{\kern0pt}basis\ {\isadigit{1}}\ jm{\isadigit{2}}{\isasymrangle}{\isacharparenright}{\kern0pt}{\isachardoublequoteclose}\isanewline
\ \ \ \ \ \ \ \ \isacommand{using}\isamarkupfalse%
\ tensor{\isacharunderscore}{\kern0pt}mat{\isacharunderscore}{\kern0pt}is{\isacharunderscore}{\kern0pt}assoc\ \isacommand{by}\isamarkupfalse%
\ presburger\isanewline
\ \ \ \ \ \ \isacommand{also}\isamarkupfalse%
\ \isacommand{have}\isamarkupfalse%
\ {\isachardoublequoteopen}{\isasymdots}\ {\isacharequal}{\kern0pt}\ {\isacharbar}{\kern0pt}state{\isacharunderscore}{\kern0pt}basis\ {\isadigit{1}}\ jd{\isadigit{2}}m{\isasymrangle}\ {\isasymOtimes}\ {\isacharbar}{\kern0pt}state{\isacharunderscore}{\kern0pt}basis\ m\ jm{\isadigit{2}}m{\isasymrangle}{\isachardoublequoteclose}\isanewline
\ \ \ \ \ \ \ \ \isacommand{using}\isamarkupfalse%
\ HI\ jm{\isadigit{2}}m{\isacharunderscore}{\kern0pt}def\ jmm{\isacharunderscore}{\kern0pt}def\ jm{\isadigit{2}}{\isacharunderscore}{\kern0pt}def\ \isanewline
\ \ \ \ \ \ \ \ \isacommand{by}\isamarkupfalse%
\ {\isacharparenleft}{\kern0pt}metis\ Suc{\isacharunderscore}{\kern0pt}eq{\isacharunderscore}{\kern0pt}plus{\isadigit{1}}\ div{\isacharunderscore}{\kern0pt}exp{\isacharunderscore}{\kern0pt}mod{\isacharunderscore}{\kern0pt}exp{\isacharunderscore}{\kern0pt}eq\ jd{\isadigit{2}}{\isacharunderscore}{\kern0pt}def\ le{\isacharunderscore}{\kern0pt}simps{\isacharparenleft}{\kern0pt}{\isadigit{2}}{\isacharparenright}{\kern0pt}\ less{\isacharunderscore}{\kern0pt}add{\isacharunderscore}{\kern0pt}same{\isacharunderscore}{\kern0pt}cancel{\isadigit{2}}\ m{\isacharunderscore}{\kern0pt}def\ \isanewline
\ \ \ \ \ \ \ \ \ \ \ \ mod{\isacharunderscore}{\kern0pt}less{\isacharunderscore}{\kern0pt}divisor\ mod{\isacharunderscore}{\kern0pt}mod{\isacharunderscore}{\kern0pt}power{\isacharunderscore}{\kern0pt}cancel\ plus{\isacharunderscore}{\kern0pt}{\isadigit{1}}{\isacharunderscore}{\kern0pt}eq{\isacharunderscore}{\kern0pt}Suc\ pos{\isadigit{2}}\ power{\isacharunderscore}{\kern0pt}one{\isacharunderscore}{\kern0pt}right\ zero{\isacharunderscore}{\kern0pt}less{\isacharunderscore}{\kern0pt}Suc\ \isanewline
\ \ \ \ \ \ \ \ \ \ \ \ zero{\isacharunderscore}{\kern0pt}less{\isacharunderscore}{\kern0pt}power{\isacharparenright}{\kern0pt}\isanewline
\ \ \ \ \ \ \isacommand{also}\isamarkupfalse%
\ \isacommand{have}\isamarkupfalse%
\ {\isachardoublequoteopen}{\isasymdots}\ {\isacharequal}{\kern0pt}\ {\isacharbar}{\kern0pt}state{\isacharunderscore}{\kern0pt}basis\ {\isacharparenleft}{\kern0pt}Suc\ m{\isacharparenright}{\kern0pt}\ j{\isasymrangle}{\isachardoublequoteclose}\isanewline
\ \ \ \ \ \ \ \ \isacommand{using}\isamarkupfalse%
\ state{\isacharunderscore}{\kern0pt}basis{\isacharunderscore}{\kern0pt}dec\ m{\isacharunderscore}{\kern0pt}def\ jleq\ jd{\isadigit{2}}m{\isacharunderscore}{\kern0pt}def\ jm{\isadigit{2}}m{\isacharunderscore}{\kern0pt}def\ \isacommand{by}\isamarkupfalse%
\ presburger\isanewline
\ \ \ \ \ \ \isacommand{finally}\isamarkupfalse%
\ \isacommand{show}\isamarkupfalse%
\ {\isachardoublequoteopen}{\isacharbar}{\kern0pt}state{\isacharunderscore}{\kern0pt}basis\ {\isacharparenleft}{\kern0pt}Suc\ n{\isacharparenright}{\kern0pt}\ {\isacharparenleft}{\kern0pt}j\ div\ {\isadigit{2}}{\isacharparenright}{\kern0pt}{\isasymrangle}\ {\isasymOtimes}\ {\isacharbar}{\kern0pt}state{\isacharunderscore}{\kern0pt}basis\ {\isadigit{1}}\ {\isacharparenleft}{\kern0pt}j\ mod\ {\isadigit{2}}{\isacharparenright}{\kern0pt}{\isasymrangle}\ {\isacharequal}{\kern0pt}\isanewline
\ \ \ \ \ \ \ \ \ \ \ \ \ \ \ \ \ \ \ \ {\isacharbar}{\kern0pt}state{\isacharunderscore}{\kern0pt}basis\ {\isacharparenleft}{\kern0pt}Suc\ {\isacharparenleft}{\kern0pt}Suc\ n{\isacharparenright}{\kern0pt}{\isacharparenright}{\kern0pt}\ j{\isasymrangle}{\isachardoublequoteclose}\isanewline
\ \ \ \ \ \ \ \ \isacommand{using}\isamarkupfalse%
\ jd{\isadigit{2}}{\isacharunderscore}{\kern0pt}def\ jm{\isadigit{2}}{\isacharunderscore}{\kern0pt}def\ m{\isacharunderscore}{\kern0pt}def\ \isacommand{by}\isamarkupfalse%
\ simp\isanewline
\ \ \ \ \isacommand{qed}\isamarkupfalse%
\isanewline
\ \ \isacommand{qed}\isamarkupfalse%
\isanewline
\isacommand{qed}\isamarkupfalse%
%
\endisatagproof
{\isafoldproof}%
%
\isadelimproof
%
\endisadelimproof
%
\begin{isamarkuptext}%
Action of the H gate in the circuit%
\end{isamarkuptext}\isamarkuptrue%
\isacommand{lemma}\isamarkupfalse%
\ H{\isacharunderscore}{\kern0pt}on{\isacharunderscore}{\kern0pt}first{\isacharunderscore}{\kern0pt}qubit{\isacharcolon}{\kern0pt}\isanewline
\ \ \isakeyword{assumes}\ {\isachardoublequoteopen}j\ {\isacharless}{\kern0pt}\ {\isadigit{2}}\ {\isacharcircum}{\kern0pt}\ Suc\ n{\isachardoublequoteclose}\isanewline
\ \ \isakeyword{shows}\ {\isachardoublequoteopen}{\isacharparenleft}{\kern0pt}{\isacharparenleft}{\kern0pt}H\ {\isasymOtimes}\ {\isacharparenleft}{\kern0pt}{\isacharparenleft}{\kern0pt}{\isadigit{1}}\isactrlsub m\ {\isacharparenleft}{\kern0pt}{\isadigit{2}}{\isacharcircum}{\kern0pt}n{\isacharparenright}{\kern0pt}{\isacharparenright}{\kern0pt}{\isacharparenright}{\kern0pt}{\isacharparenright}{\kern0pt}{\isacharparenright}{\kern0pt}\ {\isacharasterisk}{\kern0pt}\ {\isacharbar}{\kern0pt}state{\isacharunderscore}{\kern0pt}basis\ {\isacharparenleft}{\kern0pt}Suc\ n{\isacharparenright}{\kern0pt}\ j{\isasymrangle}\ {\isacharequal}{\kern0pt}\ \isanewline
\ \ \ \ \ \ \ \ \ {\isadigit{1}}{\isacharslash}{\kern0pt}sqrt\ {\isadigit{2}}\ {\isasymcdot}\isactrlsub m\ {\isacharparenleft}{\kern0pt}\ {\isacharbar}{\kern0pt}zero{\isasymrangle}\ {\isacharplus}{\kern0pt}\ exp{\isacharparenleft}{\kern0pt}{\isadigit{2}}{\isacharasterisk}{\kern0pt}{\isasymi}{\isacharasterisk}{\kern0pt}pi{\isacharasterisk}{\kern0pt}{\isacharparenleft}{\kern0pt}complex{\isacharunderscore}{\kern0pt}of{\isacharunderscore}{\kern0pt}nat\ {\isacharparenleft}{\kern0pt}j\ div\ {\isadigit{2}}{\isacharcircum}{\kern0pt}n{\isacharparenright}{\kern0pt}{\isacharparenright}{\kern0pt}{\isacharslash}{\kern0pt}{\isadigit{2}}{\isacharparenright}{\kern0pt}\ {\isasymcdot}\isactrlsub m\ {\isacharbar}{\kern0pt}one{\isasymrangle}{\isacharparenright}{\kern0pt}\ {\isasymOtimes}\ \isanewline
\ \ \ \ \ \ \ \ \ {\isacharbar}{\kern0pt}state{\isacharunderscore}{\kern0pt}basis\ n\ {\isacharparenleft}{\kern0pt}j\ mod\ {\isadigit{2}}{\isacharcircum}{\kern0pt}n{\isacharparenright}{\kern0pt}{\isasymrangle}{\isachardoublequoteclose}\isanewline
%
\isadelimproof
%
\endisadelimproof
%
\isatagproof
\isacommand{proof}\isamarkupfalse%
\ {\isacharminus}{\kern0pt}\isanewline
\ \ \isacommand{define}\isamarkupfalse%
\ jd\ jm\ \isakeyword{where}\ {\isachardoublequoteopen}jd\ {\isacharequal}{\kern0pt}\ j\ div\ {\isadigit{2}}{\isacharcircum}{\kern0pt}n{\isachardoublequoteclose}\ \isakeyword{and}\ {\isachardoublequoteopen}jm\ {\isacharequal}{\kern0pt}\ j\ mod\ {\isadigit{2}}{\isacharcircum}{\kern0pt}n{\isachardoublequoteclose}\isanewline
\ \ \isacommand{have}\isamarkupfalse%
\ {\isachardoublequoteopen}{\isacharparenleft}{\kern0pt}{\isacharparenleft}{\kern0pt}H\ {\isasymOtimes}\ {\isacharparenleft}{\kern0pt}{\isacharparenleft}{\kern0pt}{\isadigit{1}}\isactrlsub m\ {\isacharparenleft}{\kern0pt}{\isadigit{2}}{\isacharcircum}{\kern0pt}n{\isacharparenright}{\kern0pt}{\isacharparenright}{\kern0pt}{\isacharparenright}{\kern0pt}{\isacharparenright}{\kern0pt}{\isacharparenright}{\kern0pt}\ {\isacharasterisk}{\kern0pt}\ {\isacharbar}{\kern0pt}state{\isacharunderscore}{\kern0pt}basis\ {\isacharparenleft}{\kern0pt}Suc\ n{\isacharparenright}{\kern0pt}\ j{\isasymrangle}\ {\isacharequal}{\kern0pt}\ \isanewline
\ \ \ \ \ \ \ \ {\isacharparenleft}{\kern0pt}{\isacharparenleft}{\kern0pt}H\ {\isasymOtimes}\ {\isacharparenleft}{\kern0pt}{\isacharparenleft}{\kern0pt}{\isadigit{1}}\isactrlsub m\ {\isacharparenleft}{\kern0pt}{\isadigit{2}}{\isacharcircum}{\kern0pt}n{\isacharparenright}{\kern0pt}{\isacharparenright}{\kern0pt}{\isacharparenright}{\kern0pt}{\isacharparenright}{\kern0pt}{\isacharparenright}{\kern0pt}\ {\isacharasterisk}{\kern0pt}\ {\isacharparenleft}{\kern0pt}\ {\isacharbar}{\kern0pt}state{\isacharunderscore}{\kern0pt}basis\ {\isadigit{1}}\ jd{\isasymrangle}\ {\isasymOtimes}\ {\isacharbar}{\kern0pt}state{\isacharunderscore}{\kern0pt}basis\ n\ jm{\isasymrangle}{\isacharparenright}{\kern0pt}{\isachardoublequoteclose}\isanewline
\ \ \ \ \isacommand{using}\isamarkupfalse%
\ jd{\isacharunderscore}{\kern0pt}def\ jm{\isacharunderscore}{\kern0pt}def\ state{\isacharunderscore}{\kern0pt}basis{\isacharunderscore}{\kern0pt}dec\ assms\ \isacommand{by}\isamarkupfalse%
\ simp\isanewline
\ \ \isacommand{also}\isamarkupfalse%
\ \isacommand{have}\isamarkupfalse%
\ {\isachardoublequoteopen}{\isasymdots}\ {\isacharequal}{\kern0pt}\ {\isacharparenleft}{\kern0pt}H\ {\isacharasterisk}{\kern0pt}\ {\isacharbar}{\kern0pt}state{\isacharunderscore}{\kern0pt}basis\ {\isadigit{1}}\ jd{\isasymrangle}{\isacharparenright}{\kern0pt}\ {\isasymOtimes}\ {\isacharparenleft}{\kern0pt}{\isacharparenleft}{\kern0pt}{\isadigit{1}}\isactrlsub m\ {\isacharparenleft}{\kern0pt}{\isadigit{2}}{\isacharcircum}{\kern0pt}n{\isacharparenright}{\kern0pt}{\isacharparenright}{\kern0pt}\ {\isacharasterisk}{\kern0pt}\ {\isacharbar}{\kern0pt}state{\isacharunderscore}{\kern0pt}basis\ n\ jm{\isasymrangle}{\isacharparenright}{\kern0pt}{\isachardoublequoteclose}\isanewline
\ \ \ \ \isacommand{using}\isamarkupfalse%
\ H{\isacharunderscore}{\kern0pt}def\ state{\isacharunderscore}{\kern0pt}basis{\isacharunderscore}{\kern0pt}carrier{\isacharunderscore}{\kern0pt}mat\ state{\isacharunderscore}{\kern0pt}basis{\isacharunderscore}{\kern0pt}def\ ket{\isacharunderscore}{\kern0pt}vec{\isacharunderscore}{\kern0pt}def\ mult{\isacharunderscore}{\kern0pt}distr{\isacharunderscore}{\kern0pt}tensor\ \isanewline
\ \ \ \ \isacommand{by}\isamarkupfalse%
\ {\isacharparenleft}{\kern0pt}metis\ {\isacharparenleft}{\kern0pt}no{\isacharunderscore}{\kern0pt}types{\isacharcomma}{\kern0pt}\ lifting{\isacharparenright}{\kern0pt}\ H{\isacharunderscore}{\kern0pt}without{\isacharunderscore}{\kern0pt}scalar{\isacharunderscore}{\kern0pt}prod\ carrier{\isacharunderscore}{\kern0pt}matD{\isacharparenleft}{\kern0pt}{\isadigit{1}}{\isacharparenright}{\kern0pt}\ dim{\isacharunderscore}{\kern0pt}col{\isacharunderscore}{\kern0pt}mat{\isacharparenleft}{\kern0pt}{\isadigit{1}}{\isacharparenright}{\kern0pt}\ \isanewline
\ \ \ \ \ \ \ \ index{\isacharunderscore}{\kern0pt}one{\isacharunderscore}{\kern0pt}mat{\isacharparenleft}{\kern0pt}{\isadigit{3}}{\isacharparenright}{\kern0pt}\ pos{\isadigit{2}}\ power{\isacharunderscore}{\kern0pt}one{\isacharunderscore}{\kern0pt}right\ zero{\isacharunderscore}{\kern0pt}less{\isacharunderscore}{\kern0pt}one{\isacharunderscore}{\kern0pt}class{\isachardot}{\kern0pt}zero{\isacharunderscore}{\kern0pt}less{\isacharunderscore}{\kern0pt}one\ zero{\isacharunderscore}{\kern0pt}less{\isacharunderscore}{\kern0pt}power{\isacharparenright}{\kern0pt}\isanewline
\ \ \isacommand{also}\isamarkupfalse%
\ \isacommand{have}\isamarkupfalse%
\ {\isachardoublequoteopen}{\isasymdots}\ {\isacharequal}{\kern0pt}\ {\isadigit{1}}{\isacharslash}{\kern0pt}sqrt\ {\isadigit{2}}\ {\isasymcdot}\isactrlsub m\ {\isacharparenleft}{\kern0pt}\ {\isacharbar}{\kern0pt}zero{\isasymrangle}\ {\isacharplus}{\kern0pt}\ exp{\isacharparenleft}{\kern0pt}{\isadigit{2}}{\isacharasterisk}{\kern0pt}{\isasymi}{\isacharasterisk}{\kern0pt}pi{\isacharasterisk}{\kern0pt}{\isacharparenleft}{\kern0pt}complex{\isacharunderscore}{\kern0pt}of{\isacharunderscore}{\kern0pt}nat\ jd{\isacharparenright}{\kern0pt}{\isacharslash}{\kern0pt}{\isadigit{2}}{\isacharparenright}{\kern0pt}\ {\isasymcdot}\isactrlsub m\ {\isacharbar}{\kern0pt}one{\isasymrangle}{\isacharparenright}{\kern0pt}\ {\isasymOtimes}\ \isanewline
\ \ \ \ \ \ \ \ \ \ \ \ \ \ \ \ \ \ {\isacharbar}{\kern0pt}state{\isacharunderscore}{\kern0pt}basis\ n\ jm{\isasymrangle}{\isachardoublequoteclose}\isanewline
\ \ \isacommand{proof}\isamarkupfalse%
\ {\isacharminus}{\kern0pt}\isanewline
\ \ \ \ \isacommand{have}\isamarkupfalse%
\ {\isadigit{0}}{\isacharcolon}{\kern0pt}{\isachardoublequoteopen}{\isadigit{1}}\isactrlsub m\ {\isacharparenleft}{\kern0pt}{\isadigit{2}}\ {\isacharcircum}{\kern0pt}\ n{\isacharparenright}{\kern0pt}\ {\isacharasterisk}{\kern0pt}\ {\isacharbar}{\kern0pt}state{\isacharunderscore}{\kern0pt}basis\ n\ jm{\isasymrangle}\ {\isacharequal}{\kern0pt}\ {\isacharbar}{\kern0pt}state{\isacharunderscore}{\kern0pt}basis\ n\ jm{\isasymrangle}{\isachardoublequoteclose}\ \isanewline
\ \ \ \ \ \ \isacommand{using}\isamarkupfalse%
\ left{\isacharunderscore}{\kern0pt}mult{\isacharunderscore}{\kern0pt}one{\isacharunderscore}{\kern0pt}mat\ state{\isacharunderscore}{\kern0pt}basis{\isacharunderscore}{\kern0pt}carrier{\isacharunderscore}{\kern0pt}mat\ \isacommand{by}\isamarkupfalse%
\ metis\isanewline
\ \ \ \ \isacommand{have}\isamarkupfalse%
\ {\isachardoublequoteopen}H\ {\isacharasterisk}{\kern0pt}\ {\isacharbar}{\kern0pt}state{\isacharunderscore}{\kern0pt}basis\ {\isadigit{1}}\ jd{\isasymrangle}\ {\isacharequal}{\kern0pt}\isanewline
\ \ \ \ \ \ \ \ \ \ {\isadigit{1}}{\isacharslash}{\kern0pt}sqrt\ {\isadigit{2}}\ {\isasymcdot}\isactrlsub m\ {\isacharparenleft}{\kern0pt}\ {\isacharbar}{\kern0pt}zero{\isasymrangle}\ {\isacharplus}{\kern0pt}\ exp{\isacharparenleft}{\kern0pt}{\isadigit{2}}{\isacharasterisk}{\kern0pt}{\isasymi}{\isacharasterisk}{\kern0pt}pi{\isacharasterisk}{\kern0pt}{\isacharparenleft}{\kern0pt}complex{\isacharunderscore}{\kern0pt}of{\isacharunderscore}{\kern0pt}nat\ jd{\isacharparenright}{\kern0pt}{\isacharslash}{\kern0pt}{\isadigit{2}}{\isacharparenright}{\kern0pt}\ {\isasymcdot}\isactrlsub m\ {\isacharbar}{\kern0pt}one{\isasymrangle}{\isacharparenright}{\kern0pt}{\isachardoublequoteclose}\isanewline
\ \ \ \ \isacommand{proof}\isamarkupfalse%
\ {\isacharparenleft}{\kern0pt}rule\ disjE{\isacharparenright}{\kern0pt}\isanewline
\ \ \ \ \ \ \isacommand{show}\isamarkupfalse%
\ {\isachardoublequoteopen}jd\ {\isacharequal}{\kern0pt}\ {\isadigit{0}}\ {\isasymor}\ jd\ {\isacharequal}{\kern0pt}\ {\isadigit{1}}{\isachardoublequoteclose}\ \isacommand{using}\isamarkupfalse%
\ jd{\isacharunderscore}{\kern0pt}def\ assms\ \isacommand{by}\isamarkupfalse%
\ {\isacharparenleft}{\kern0pt}metis\ One{\isacharunderscore}{\kern0pt}nat{\isacharunderscore}{\kern0pt}def\ less{\isacharunderscore}{\kern0pt}{\isadigit{2}}{\isacharunderscore}{\kern0pt}cases\ \isanewline
\ \ \ \ \ \ \ \ \ \ \ \ less{\isacharunderscore}{\kern0pt}power{\isacharunderscore}{\kern0pt}add{\isacharunderscore}{\kern0pt}imp{\isacharunderscore}{\kern0pt}div{\isacharunderscore}{\kern0pt}less\ plus{\isacharunderscore}{\kern0pt}{\isadigit{1}}{\isacharunderscore}{\kern0pt}eq{\isacharunderscore}{\kern0pt}Suc\ power{\isacharunderscore}{\kern0pt}one{\isacharunderscore}{\kern0pt}right{\isacharparenright}{\kern0pt}\isanewline
\ \ \ \ \isacommand{next}\isamarkupfalse%
\isanewline
\ \ \ \ \ \ \isacommand{assume}\isamarkupfalse%
\ jd{\isadigit{0}}{\isacharcolon}{\kern0pt}{\isachardoublequoteopen}jd\ {\isacharequal}{\kern0pt}\ {\isadigit{0}}{\isachardoublequoteclose}\isanewline
\ \ \ \ \ \ \isacommand{have}\isamarkupfalse%
\ {\isachardoublequoteopen}H\ {\isacharasterisk}{\kern0pt}\ {\isacharbar}{\kern0pt}state{\isacharunderscore}{\kern0pt}basis\ {\isadigit{1}}\ {\isadigit{0}}{\isasymrangle}\ {\isacharequal}{\kern0pt}\ \isanewline
\ \ \ \ \ \ \ \ \ \ \ \ mat{\isacharunderscore}{\kern0pt}of{\isacharunderscore}{\kern0pt}cols{\isacharunderscore}{\kern0pt}list\ {\isadigit{2}}\ {\isacharparenleft}{\kern0pt}map\ {\isacharparenleft}{\kern0pt}map\ complex{\isacharunderscore}{\kern0pt}of{\isacharunderscore}{\kern0pt}real{\isacharparenright}{\kern0pt}\ {\isacharbrackleft}{\kern0pt}{\isacharbrackleft}{\kern0pt}{\isadigit{1}}\ {\isacharslash}{\kern0pt}\ sqrt\ {\isadigit{2}}{\isacharcomma}{\kern0pt}\ {\isadigit{1}}\ {\isacharslash}{\kern0pt}\ sqrt\ {\isadigit{2}}{\isacharbrackright}{\kern0pt}{\isacharbrackright}{\kern0pt}{\isacharparenright}{\kern0pt}{\isachardoublequoteclose}\ \isanewline
\ \ \ \ \ \ \ \ \isacommand{using}\isamarkupfalse%
\ H{\isacharunderscore}{\kern0pt}on{\isacharunderscore}{\kern0pt}ket{\isacharunderscore}{\kern0pt}zero\ state{\isacharunderscore}{\kern0pt}basis{\isacharunderscore}{\kern0pt}def\ \isacommand{by}\isamarkupfalse%
\ auto\isanewline
\ \ \ \ \ \ \isacommand{also}\isamarkupfalse%
\ \isacommand{have}\isamarkupfalse%
\ {\isachardoublequoteopen}{\isasymdots}\ {\isacharequal}{\kern0pt}\ {\isadigit{1}}{\isacharslash}{\kern0pt}sqrt\ {\isadigit{2}}\ {\isasymcdot}\isactrlsub m\ {\isacharparenleft}{\kern0pt}\ {\isacharbar}{\kern0pt}zero{\isasymrangle}\ {\isacharplus}{\kern0pt}\ exp{\isacharparenleft}{\kern0pt}{\isadigit{2}}{\isacharasterisk}{\kern0pt}{\isasymi}{\isacharasterisk}{\kern0pt}pi{\isacharasterisk}{\kern0pt}{\isacharparenleft}{\kern0pt}complex{\isacharunderscore}{\kern0pt}of{\isacharunderscore}{\kern0pt}nat\ {\isadigit{0}}{\isacharparenright}{\kern0pt}{\isacharslash}{\kern0pt}{\isadigit{2}}{\isacharparenright}{\kern0pt}\ {\isasymcdot}\isactrlsub m\ {\isacharbar}{\kern0pt}one{\isasymrangle}{\isacharparenright}{\kern0pt}{\isachardoublequoteclose}\isanewline
\ \ \ \ \ \ \isacommand{proof}\isamarkupfalse%
\ \isanewline
\ \ \ \ \ \ \ \ \isacommand{fix}\isamarkupfalse%
\ i\ j\isanewline
\ \ \ \ \ \ \ \ \isacommand{assume}\isamarkupfalse%
\ ai{\isacharcolon}{\kern0pt}{\isachardoublequoteopen}i\ {\isacharless}{\kern0pt}\ dim{\isacharunderscore}{\kern0pt}row\ {\isacharparenleft}{\kern0pt}{\isacharparenleft}{\kern0pt}{\isadigit{1}}{\isacharslash}{\kern0pt}sqrt\ {\isadigit{2}}{\isacharparenright}{\kern0pt}\ {\isasymcdot}\isactrlsub m\ {\isacharparenleft}{\kern0pt}\ {\isacharbar}{\kern0pt}zero{\isasymrangle}\ {\isacharplus}{\kern0pt}\ exp\ {\isacharparenleft}{\kern0pt}{\isadigit{2}}{\isacharasterisk}{\kern0pt}{\isasymi}{\isacharasterisk}{\kern0pt}pi{\isacharasterisk}{\kern0pt}complex{\isacharunderscore}{\kern0pt}of{\isacharunderscore}{\kern0pt}nat\ {\isadigit{0}}{\isacharslash}{\kern0pt}{\isadigit{2}}{\isacharparenright}{\kern0pt}\ {\isasymcdot}\isactrlsub m\ {\isacharbar}{\kern0pt}one{\isasymrangle}{\isacharparenright}{\kern0pt}{\isacharparenright}{\kern0pt}{\isachardoublequoteclose}\isanewline
\ \ \ \ \ \ \ \ \isacommand{hence}\isamarkupfalse%
\ {\isachardoublequoteopen}i\ {\isacharless}{\kern0pt}\ {\isadigit{2}}{\isachardoublequoteclose}\ \isacommand{using}\isamarkupfalse%
\ mat{\isacharunderscore}{\kern0pt}of{\isacharunderscore}{\kern0pt}cols{\isacharunderscore}{\kern0pt}list{\isacharunderscore}{\kern0pt}def\ smult{\isacharunderscore}{\kern0pt}carrier{\isacharunderscore}{\kern0pt}mat\ ket{\isacharunderscore}{\kern0pt}vec{\isacharunderscore}{\kern0pt}def\ \isacommand{by}\isamarkupfalse%
\ simp\isanewline
\ \ \ \ \ \ \ \ \isacommand{hence}\isamarkupfalse%
\ i{\isadigit{2}}{\isacharcolon}{\kern0pt}{\isachardoublequoteopen}i\ {\isasymin}\ {\isacharbraceleft}{\kern0pt}{\isadigit{0}}{\isacharcomma}{\kern0pt}{\isadigit{1}}{\isacharbraceright}{\kern0pt}{\isachardoublequoteclose}\ \isacommand{by}\isamarkupfalse%
\ auto\isanewline
\ \ \ \ \ \ \ \ \isacommand{assume}\isamarkupfalse%
\ aj{\isacharcolon}{\kern0pt}{\isachardoublequoteopen}j\ {\isacharless}{\kern0pt}\ dim{\isacharunderscore}{\kern0pt}col\ {\isacharparenleft}{\kern0pt}{\isacharparenleft}{\kern0pt}{\isadigit{1}}{\isacharslash}{\kern0pt}sqrt\ {\isadigit{2}}{\isacharparenright}{\kern0pt}\ {\isasymcdot}\isactrlsub m\ {\isacharparenleft}{\kern0pt}\ {\isacharbar}{\kern0pt}zero{\isasymrangle}\ {\isacharplus}{\kern0pt}\ exp\ {\isacharparenleft}{\kern0pt}{\isadigit{2}}{\isacharasterisk}{\kern0pt}{\isasymi}{\isacharasterisk}{\kern0pt}pi{\isacharasterisk}{\kern0pt}complex{\isacharunderscore}{\kern0pt}of{\isacharunderscore}{\kern0pt}nat\ {\isadigit{0}}{\isacharslash}{\kern0pt}{\isadigit{2}}{\isacharparenright}{\kern0pt}\ {\isasymcdot}\isactrlsub m\ {\isacharbar}{\kern0pt}one{\isasymrangle}{\isacharparenright}{\kern0pt}{\isacharparenright}{\kern0pt}{\isachardoublequoteclose}\isanewline
\ \ \ \ \ \ \ \ \isacommand{hence}\isamarkupfalse%
\ j{\isadigit{0}}{\isacharcolon}{\kern0pt}{\isachardoublequoteopen}j\ {\isacharequal}{\kern0pt}\ {\isadigit{0}}{\isachardoublequoteclose}\ \isacommand{using}\isamarkupfalse%
\ mat{\isacharunderscore}{\kern0pt}of{\isacharunderscore}{\kern0pt}cols{\isacharunderscore}{\kern0pt}list{\isacharunderscore}{\kern0pt}def\ smult{\isacharunderscore}{\kern0pt}carrier{\isacharunderscore}{\kern0pt}mat\ ket{\isacharunderscore}{\kern0pt}vec{\isacharunderscore}{\kern0pt}def\ \isacommand{by}\isamarkupfalse%
\ simp\isanewline
\ \ \ \ \ \ \ \ \isacommand{have}\isamarkupfalse%
\ {\isachardoublequoteopen}{\isacharparenleft}{\kern0pt}mat{\isacharunderscore}{\kern0pt}of{\isacharunderscore}{\kern0pt}cols{\isacharunderscore}{\kern0pt}list\ {\isadigit{2}}\ {\isacharparenleft}{\kern0pt}map\ {\isacharparenleft}{\kern0pt}map\ complex{\isacharunderscore}{\kern0pt}of{\isacharunderscore}{\kern0pt}real{\isacharparenright}{\kern0pt}\ {\isacharbrackleft}{\kern0pt}{\isacharbrackleft}{\kern0pt}{\isadigit{1}}\ {\isacharslash}{\kern0pt}\ sqrt\ {\isadigit{2}}{\isacharcomma}{\kern0pt}\ {\isadigit{1}}\ {\isacharslash}{\kern0pt}\ sqrt\ {\isadigit{2}}{\isacharbrackright}{\kern0pt}{\isacharbrackright}{\kern0pt}{\isacharparenright}{\kern0pt}{\isacharparenright}{\kern0pt}\ {\isachardollar}{\kern0pt}{\isachardollar}{\kern0pt}\ {\isacharparenleft}{\kern0pt}i{\isacharcomma}{\kern0pt}{\isadigit{0}}{\isacharparenright}{\kern0pt}\ {\isacharequal}{\kern0pt}\isanewline
\ \ \ \ \ \ \ \ \ \ \ \ \ \ {\isacharparenleft}{\kern0pt}mat{\isacharunderscore}{\kern0pt}of{\isacharunderscore}{\kern0pt}cols{\isacharunderscore}{\kern0pt}list\ {\isadigit{2}}\ {\isacharbrackleft}{\kern0pt}{\isacharbrackleft}{\kern0pt}{\isadigit{1}}{\isacharslash}{\kern0pt}sqrt\ {\isadigit{2}}{\isacharcomma}{\kern0pt}\ {\isadigit{1}}{\isacharslash}{\kern0pt}sqrt\ {\isadigit{2}}{\isacharbrackright}{\kern0pt}{\isacharbrackright}{\kern0pt}{\isacharparenright}{\kern0pt}\ {\isachardollar}{\kern0pt}{\isachardollar}{\kern0pt}\ {\isacharparenleft}{\kern0pt}i{\isacharcomma}{\kern0pt}{\isadigit{0}}{\isacharparenright}{\kern0pt}{\isachardoublequoteclose}\isanewline
\ \ \ \ \ \ \ \ \ \ \isacommand{using}\isamarkupfalse%
\ map{\isacharunderscore}{\kern0pt}def\ \isacommand{by}\isamarkupfalse%
\ simp\isanewline
\ \ \ \ \ \ \ \ \isacommand{also}\isamarkupfalse%
\ \isacommand{have}\isamarkupfalse%
\ {\isachardoublequoteopen}{\isasymdots}\ {\isacharequal}{\kern0pt}\ {\isadigit{1}}{\isacharslash}{\kern0pt}sqrt\ {\isadigit{2}}{\isachardoublequoteclose}\ \isacommand{using}\isamarkupfalse%
\ i{\isadigit{2}}\ index{\isacharunderscore}{\kern0pt}mat{\isacharunderscore}{\kern0pt}of{\isacharunderscore}{\kern0pt}cols{\isacharunderscore}{\kern0pt}list\ \isacommand{by}\isamarkupfalse%
\ auto\isanewline
\ \ \ \ \ \ \ \ \isacommand{also}\isamarkupfalse%
\ \isacommand{have}\isamarkupfalse%
\ {\isachardoublequoteopen}{\isasymdots}\ {\isacharequal}{\kern0pt}\ {\isacharparenleft}{\kern0pt}{\isadigit{1}}{\isacharslash}{\kern0pt}sqrt\ {\isadigit{2}}\ {\isasymcdot}\isactrlsub m\ {\isacharparenleft}{\kern0pt}mat{\isacharunderscore}{\kern0pt}of{\isacharunderscore}{\kern0pt}cols{\isacharunderscore}{\kern0pt}list\ {\isadigit{2}}\ {\isacharbrackleft}{\kern0pt}{\isacharbrackleft}{\kern0pt}{\isadigit{1}}{\isacharcomma}{\kern0pt}{\isadigit{1}}{\isacharbrackright}{\kern0pt}{\isacharbrackright}{\kern0pt}{\isacharparenright}{\kern0pt}{\isacharparenright}{\kern0pt}\ {\isachardollar}{\kern0pt}{\isachardollar}{\kern0pt}\ {\isacharparenleft}{\kern0pt}i{\isacharcomma}{\kern0pt}{\isadigit{0}}{\isacharparenright}{\kern0pt}{\isachardoublequoteclose}\isanewline
\ \ \ \ \ \ \ \ \ \ \isacommand{using}\isamarkupfalse%
\ smult{\isacharunderscore}{\kern0pt}mat{\isacharunderscore}{\kern0pt}def\ mat{\isacharunderscore}{\kern0pt}of{\isacharunderscore}{\kern0pt}cols{\isacharunderscore}{\kern0pt}list{\isacharunderscore}{\kern0pt}def\ index{\isacharunderscore}{\kern0pt}mat{\isacharunderscore}{\kern0pt}of{\isacharunderscore}{\kern0pt}cols{\isacharunderscore}{\kern0pt}list\ \isanewline
\ \ \ \ \ \ \ \ \ \ \isacommand{by}\isamarkupfalse%
\ {\isacharparenleft}{\kern0pt}smt\ {\isacharparenleft}{\kern0pt}verit{\isacharcomma}{\kern0pt}\ best{\isacharparenright}{\kern0pt}\ Suc{\isacharunderscore}{\kern0pt}{\isadigit{1}}\ {\isacartoucheopen}i\ {\isacharless}{\kern0pt}\ {\isadigit{2}}{\isacartoucheclose}\ dim{\isacharunderscore}{\kern0pt}col{\isacharunderscore}{\kern0pt}mat{\isacharparenleft}{\kern0pt}{\isadigit{1}}{\isacharparenright}{\kern0pt}\ dim{\isacharunderscore}{\kern0pt}row{\isacharunderscore}{\kern0pt}mat{\isacharparenleft}{\kern0pt}{\isadigit{1}}{\isacharparenright}{\kern0pt}\ index{\isacharunderscore}{\kern0pt}smult{\isacharunderscore}{\kern0pt}mat{\isacharparenleft}{\kern0pt}{\isadigit{1}}{\isacharparenright}{\kern0pt}\ \isanewline
\ \ \ \ \ \ \ \ \ \ \ \ \ \ ket{\isacharunderscore}{\kern0pt}one{\isacharunderscore}{\kern0pt}is{\isacharunderscore}{\kern0pt}state\ ket{\isacharunderscore}{\kern0pt}one{\isacharunderscore}{\kern0pt}to{\isacharunderscore}{\kern0pt}mat{\isacharunderscore}{\kern0pt}of{\isacharunderscore}{\kern0pt}cols{\isacharunderscore}{\kern0pt}list\ less{\isacharunderscore}{\kern0pt}Suc{\isacharunderscore}{\kern0pt}eq{\isacharunderscore}{\kern0pt}{\isadigit{0}}{\isacharunderscore}{\kern0pt}disj\ less{\isacharunderscore}{\kern0pt}one\ list{\isachardot}{\kern0pt}size{\isacharparenleft}{\kern0pt}{\isadigit{4}}{\isacharparenright}{\kern0pt}\ \isanewline
\ \ \ \ \ \ \ \ \ \ \ \ \ \ mult{\isachardot}{\kern0pt}right{\isacharunderscore}{\kern0pt}neutral\ nth{\isacharunderscore}{\kern0pt}Cons{\isacharunderscore}{\kern0pt}{\isadigit{0}}\ nth{\isacharunderscore}{\kern0pt}Cons{\isacharunderscore}{\kern0pt}Suc\ state{\isacharunderscore}{\kern0pt}def{\isacharparenright}{\kern0pt}\isanewline
\ \ \ \ \ \ \ \ \isacommand{also}\isamarkupfalse%
\ \isacommand{have}\isamarkupfalse%
\ {\isachardoublequoteopen}{\isasymdots}\ {\isacharequal}{\kern0pt}\ {\isacharparenleft}{\kern0pt}{\isadigit{1}}{\isacharslash}{\kern0pt}sqrt\ {\isadigit{2}}\ {\isasymcdot}\isactrlsub m\ {\isacharparenleft}{\kern0pt}\ {\isacharbar}{\kern0pt}zero{\isasymrangle}\ {\isacharplus}{\kern0pt}\ {\isacharbar}{\kern0pt}one{\isasymrangle}{\isacharparenright}{\kern0pt}{\isacharparenright}{\kern0pt}\ {\isachardollar}{\kern0pt}{\isachardollar}{\kern0pt}\ {\isacharparenleft}{\kern0pt}i{\isacharcomma}{\kern0pt}{\isadigit{0}}{\isacharparenright}{\kern0pt}{\isachardoublequoteclose}\isanewline
\ \ \ \ \ \ \ \ \isacommand{proof}\isamarkupfalse%
\ {\isacharminus}{\kern0pt}\isanewline
\ \ \ \ \ \ \ \ \ \ \isacommand{have}\isamarkupfalse%
\ {\isachardoublequoteopen}mat{\isacharunderscore}{\kern0pt}of{\isacharunderscore}{\kern0pt}cols{\isacharunderscore}{\kern0pt}list\ {\isadigit{2}}\ {\isacharbrackleft}{\kern0pt}{\isacharbrackleft}{\kern0pt}{\isadigit{1}}{\isacharcomma}{\kern0pt}{\isadigit{1}}{\isacharbrackright}{\kern0pt}{\isacharbrackright}{\kern0pt}\ {\isacharequal}{\kern0pt}\ {\isacharbar}{\kern0pt}zero{\isasymrangle}\ {\isacharplus}{\kern0pt}\ {\isacharbar}{\kern0pt}one{\isasymrangle}{\isachardoublequoteclose}\isanewline
\ \ \ \ \ \ \ \ \ \ \isacommand{proof}\isamarkupfalse%
\ \isanewline
\ \ \ \ \ \ \ \ \ \ \ \ \isacommand{fix}\isamarkupfalse%
\ i\ j{\isacharcolon}{\kern0pt}{\isacharcolon}{\kern0pt}nat\ \isanewline
\ \ \ \ \ \ \ \ \ \ \ \ \isacommand{define}\isamarkupfalse%
\ s{\isadigit{1}}\ s{\isadigit{2}}\ \isakeyword{where}\ {\isachardoublequoteopen}s{\isadigit{1}}\ {\isacharequal}{\kern0pt}\ mat{\isacharunderscore}{\kern0pt}of{\isacharunderscore}{\kern0pt}cols{\isacharunderscore}{\kern0pt}list\ {\isadigit{2}}\ {\isacharbrackleft}{\kern0pt}{\isacharbrackleft}{\kern0pt}{\isadigit{1}}{\isacharcomma}{\kern0pt}{\isadigit{1}}{\isacharbrackright}{\kern0pt}{\isacharbrackright}{\kern0pt}{\isachardoublequoteclose}\ \isakeyword{and}\ {\isachardoublequoteopen}s{\isadigit{2}}\ {\isacharequal}{\kern0pt}\ {\isacharbar}{\kern0pt}zero{\isasymrangle}\ {\isacharplus}{\kern0pt}\ {\isacharbar}{\kern0pt}one{\isasymrangle}{\isachardoublequoteclose}\isanewline
\ \ \ \ \ \ \ \ \ \ \ \ \isacommand{assume}\isamarkupfalse%
\ {\isachardoublequoteopen}i\ {\isacharless}{\kern0pt}\ dim{\isacharunderscore}{\kern0pt}row\ s{\isadigit{2}}{\isachardoublequoteclose}\ \isakeyword{and}\ {\isachardoublequoteopen}j\ {\isacharless}{\kern0pt}\ dim{\isacharunderscore}{\kern0pt}col\ s{\isadigit{2}}{\isachardoublequoteclose}\isanewline
\ \ \ \ \ \ \ \ \ \ \ \ \isacommand{hence}\isamarkupfalse%
\ {\isachardoublequoteopen}i\ {\isasymin}\ {\isacharbraceleft}{\kern0pt}{\isadigit{0}}{\isacharcomma}{\kern0pt}{\isadigit{1}}{\isacharbraceright}{\kern0pt}\ {\isasymand}\ j\ {\isacharequal}{\kern0pt}\ {\isadigit{0}}{\isachardoublequoteclose}\ \isacommand{using}\isamarkupfalse%
\ index{\isacharunderscore}{\kern0pt}add{\isacharunderscore}{\kern0pt}mat\ \isanewline
\ \ \ \ \ \ \ \ \ \ \ \ \ \ \isacommand{by}\isamarkupfalse%
\ {\isacharparenleft}{\kern0pt}simp\ add{\isacharcolon}{\kern0pt}\ ket{\isacharunderscore}{\kern0pt}vec{\isacharunderscore}{\kern0pt}def\ less{\isacharunderscore}{\kern0pt}Suc{\isacharunderscore}{\kern0pt}eq\ numerals{\isacharparenleft}{\kern0pt}{\isadigit{2}}{\isacharparenright}{\kern0pt}\ s{\isadigit{2}}{\isacharunderscore}{\kern0pt}def{\isacharparenright}{\kern0pt}\isanewline
\ \ \ \ \ \ \ \ \ \ \ \ \isacommand{thus}\isamarkupfalse%
\ {\isachardoublequoteopen}s{\isadigit{1}}\ {\isachardollar}{\kern0pt}{\isachardollar}{\kern0pt}\ {\isacharparenleft}{\kern0pt}i{\isacharcomma}{\kern0pt}j{\isacharparenright}{\kern0pt}\ {\isacharequal}{\kern0pt}\ s{\isadigit{2}}\ {\isachardollar}{\kern0pt}{\isachardollar}{\kern0pt}\ {\isacharparenleft}{\kern0pt}i{\isacharcomma}{\kern0pt}j{\isacharparenright}{\kern0pt}{\isachardoublequoteclose}\ \isacommand{using}\isamarkupfalse%
\ s{\isadigit{1}}{\isacharunderscore}{\kern0pt}def\ s{\isadigit{2}}{\isacharunderscore}{\kern0pt}def\ mat{\isacharunderscore}{\kern0pt}of{\isacharunderscore}{\kern0pt}cols{\isacharunderscore}{\kern0pt}list{\isacharunderscore}{\kern0pt}def\ \isanewline
\ \ \ \ \ \ \ \ \ \ \ \ \ \ \ \ \ \ {\isacartoucheopen}i\ {\isacharless}{\kern0pt}\ dim{\isacharunderscore}{\kern0pt}row\ s{\isadigit{2}}{\isacartoucheclose}\ ket{\isacharunderscore}{\kern0pt}one{\isacharunderscore}{\kern0pt}to{\isacharunderscore}{\kern0pt}mat{\isacharunderscore}{\kern0pt}of{\isacharunderscore}{\kern0pt}cols{\isacharunderscore}{\kern0pt}list\ \isacommand{by}\isamarkupfalse%
\ force\isanewline
\ \ \ \ \ \ \ \ \ \ \isacommand{next}\isamarkupfalse%
\isanewline
\ \ \ \ \ \ \ \ \ \ \ \ \isacommand{define}\isamarkupfalse%
\ s{\isadigit{1}}\ s{\isadigit{2}}\ \isakeyword{where}\ {\isachardoublequoteopen}s{\isadigit{1}}\ {\isacharequal}{\kern0pt}\ mat{\isacharunderscore}{\kern0pt}of{\isacharunderscore}{\kern0pt}cols{\isacharunderscore}{\kern0pt}list\ {\isadigit{2}}\ {\isacharbrackleft}{\kern0pt}{\isacharbrackleft}{\kern0pt}{\isadigit{1}}{\isacharcomma}{\kern0pt}{\isadigit{1}}{\isacharbrackright}{\kern0pt}{\isacharbrackright}{\kern0pt}{\isachardoublequoteclose}\ \isakeyword{and}\ {\isachardoublequoteopen}s{\isadigit{2}}\ {\isacharequal}{\kern0pt}\ {\isacharbar}{\kern0pt}zero{\isasymrangle}\ {\isacharplus}{\kern0pt}\ {\isacharbar}{\kern0pt}one{\isasymrangle}{\isachardoublequoteclose}\isanewline
\ \ \ \ \ \ \ \ \ \ \ \ \isacommand{thus}\isamarkupfalse%
\ {\isachardoublequoteopen}dim{\isacharunderscore}{\kern0pt}row\ s{\isadigit{1}}\ {\isacharequal}{\kern0pt}\ dim{\isacharunderscore}{\kern0pt}row\ s{\isadigit{2}}{\isachardoublequoteclose}\ \isacommand{using}\isamarkupfalse%
\ mat{\isacharunderscore}{\kern0pt}of{\isacharunderscore}{\kern0pt}cols{\isacharunderscore}{\kern0pt}list{\isacharunderscore}{\kern0pt}def\ \isacommand{by}\isamarkupfalse%
\ {\isacharparenleft}{\kern0pt}simp\ add{\isacharcolon}{\kern0pt}\ ket{\isacharunderscore}{\kern0pt}vec{\isacharunderscore}{\kern0pt}def{\isacharparenright}{\kern0pt}\isanewline
\ \ \ \ \ \ \ \ \ \ \isacommand{next}\isamarkupfalse%
\isanewline
\ \ \ \ \ \ \ \ \ \ \ \ \isacommand{define}\isamarkupfalse%
\ s{\isadigit{1}}\ s{\isadigit{2}}\ \isakeyword{where}\ {\isachardoublequoteopen}s{\isadigit{1}}\ {\isacharequal}{\kern0pt}\ mat{\isacharunderscore}{\kern0pt}of{\isacharunderscore}{\kern0pt}cols{\isacharunderscore}{\kern0pt}list\ {\isadigit{2}}\ {\isacharbrackleft}{\kern0pt}{\isacharbrackleft}{\kern0pt}{\isadigit{1}}{\isacharcomma}{\kern0pt}{\isadigit{1}}{\isacharbrackright}{\kern0pt}{\isacharbrackright}{\kern0pt}{\isachardoublequoteclose}\ \isakeyword{and}\ {\isachardoublequoteopen}s{\isadigit{2}}\ {\isacharequal}{\kern0pt}\ {\isacharbar}{\kern0pt}zero{\isasymrangle}\ {\isacharplus}{\kern0pt}\ {\isacharbar}{\kern0pt}one{\isasymrangle}{\isachardoublequoteclose}\isanewline
\ \ \ \ \ \ \ \ \ \ \ \ \isacommand{thus}\isamarkupfalse%
\ {\isachardoublequoteopen}dim{\isacharunderscore}{\kern0pt}col\ s{\isadigit{1}}\ {\isacharequal}{\kern0pt}\ dim{\isacharunderscore}{\kern0pt}col\ s{\isadigit{2}}{\isachardoublequoteclose}\ \isacommand{using}\isamarkupfalse%
\ mat{\isacharunderscore}{\kern0pt}of{\isacharunderscore}{\kern0pt}cols{\isacharunderscore}{\kern0pt}list{\isacharunderscore}{\kern0pt}def\ \isacommand{by}\isamarkupfalse%
\ {\isacharparenleft}{\kern0pt}simp\ add{\isacharcolon}{\kern0pt}\ ket{\isacharunderscore}{\kern0pt}vec{\isacharunderscore}{\kern0pt}def{\isacharparenright}{\kern0pt}\isanewline
\ \ \ \ \ \ \ \ \ \ \isacommand{qed}\isamarkupfalse%
\isanewline
\ \ \ \ \ \ \ \ \ \ \isacommand{thus}\isamarkupfalse%
\ {\isacharquery}{\kern0pt}thesis\ \isacommand{by}\isamarkupfalse%
\ simp\isanewline
\ \ \ \ \ \ \ \ \isacommand{qed}\isamarkupfalse%
\isanewline
\ \ \ \ \ \ \ \ \isacommand{also}\isamarkupfalse%
\ \isacommand{have}\isamarkupfalse%
\ {\isachardoublequoteopen}{\isasymdots}\ {\isacharequal}{\kern0pt}\ {\isacharparenleft}{\kern0pt}{\isadigit{1}}{\isacharslash}{\kern0pt}sqrt\ {\isadigit{2}}\ {\isasymcdot}\isactrlsub m\ {\isacharparenleft}{\kern0pt}\ {\isacharbar}{\kern0pt}zero{\isasymrangle}\ {\isacharplus}{\kern0pt}\ {\isadigit{1}}\ {\isasymcdot}\isactrlsub m\ {\isacharbar}{\kern0pt}one{\isasymrangle}{\isacharparenright}{\kern0pt}{\isacharparenright}{\kern0pt}\ {\isachardollar}{\kern0pt}{\isachardollar}{\kern0pt}\ {\isacharparenleft}{\kern0pt}i{\isacharcomma}{\kern0pt}{\isadigit{0}}{\isacharparenright}{\kern0pt}{\isachardoublequoteclose}\isanewline
\ \ \ \ \ \ \ \ \ \ \isacommand{using}\isamarkupfalse%
\ smult{\isacharunderscore}{\kern0pt}mat{\isacharunderscore}{\kern0pt}def\ {\isacartoucheopen}i\ {\isacharless}{\kern0pt}\ {\isadigit{2}}{\isacartoucheclose}\ ket{\isacharunderscore}{\kern0pt}one{\isacharunderscore}{\kern0pt}is{\isacharunderscore}{\kern0pt}state\ state{\isacharunderscore}{\kern0pt}def\ \isacommand{by}\isamarkupfalse%
\ force\isanewline
\ \ \ \ \ \ \ \ \isacommand{also}\isamarkupfalse%
\ \isacommand{have}\isamarkupfalse%
\ {\isachardoublequoteopen}{\isasymdots}\ {\isacharequal}{\kern0pt}\ {\isacharparenleft}{\kern0pt}{\isadigit{1}}{\isacharslash}{\kern0pt}sqrt\ {\isadigit{2}}\ {\isasymcdot}\isactrlsub m\ {\isacharparenleft}{\kern0pt}\ {\isacharbar}{\kern0pt}zero{\isasymrangle}\ {\isacharplus}{\kern0pt}\ exp\ {\isacharparenleft}{\kern0pt}{\isadigit{2}}{\isacharasterisk}{\kern0pt}{\isasymi}{\isacharasterisk}{\kern0pt}pi{\isacharasterisk}{\kern0pt}{\isacharparenleft}{\kern0pt}complex{\isacharunderscore}{\kern0pt}of{\isacharunderscore}{\kern0pt}nat\ {\isadigit{0}}{\isacharparenright}{\kern0pt}{\isacharslash}{\kern0pt}{\isadigit{2}}{\isacharparenright}{\kern0pt}\ {\isasymcdot}\isactrlsub m\ {\isacharbar}{\kern0pt}one{\isasymrangle}{\isacharparenright}{\kern0pt}{\isacharparenright}{\kern0pt}\ {\isachardollar}{\kern0pt}{\isachardollar}{\kern0pt}\ {\isacharparenleft}{\kern0pt}i{\isacharcomma}{\kern0pt}{\isadigit{0}}{\isacharparenright}{\kern0pt}{\isachardoublequoteclose}\isanewline
\ \ \ \ \ \ \ \ \ \ \isacommand{by}\isamarkupfalse%
\ auto\isanewline
\ \ \ \ \ \ \ \ \isacommand{finally}\isamarkupfalse%
\ \isacommand{show}\isamarkupfalse%
\ {\isachardoublequoteopen}Tensor{\isachardot}{\kern0pt}mat{\isacharunderscore}{\kern0pt}of{\isacharunderscore}{\kern0pt}cols{\isacharunderscore}{\kern0pt}list\ {\isadigit{2}}\ {\isacharparenleft}{\kern0pt}map\ {\isacharparenleft}{\kern0pt}map\ complex{\isacharunderscore}{\kern0pt}of{\isacharunderscore}{\kern0pt}real{\isacharparenright}{\kern0pt}\ \isanewline
\ \ \ \ \ \ \ \ \ \ \ \ \ \ \ \ \ \ \ \ \ \ {\isacharbrackleft}{\kern0pt}{\isacharbrackleft}{\kern0pt}{\isadigit{1}}\ {\isacharslash}{\kern0pt}\ sqrt\ {\isadigit{2}}{\isacharcomma}{\kern0pt}\ {\isadigit{1}}\ {\isacharslash}{\kern0pt}\ sqrt\ {\isadigit{2}}{\isacharbrackright}{\kern0pt}{\isacharbrackright}{\kern0pt}{\isacharparenright}{\kern0pt}\ {\isachardollar}{\kern0pt}{\isachardollar}{\kern0pt}\ {\isacharparenleft}{\kern0pt}i{\isacharcomma}{\kern0pt}\ j{\isacharparenright}{\kern0pt}\ {\isacharequal}{\kern0pt}\isanewline
\ \ \ \ \ \ \ \ \ \ \ \ \ \ \ \ \ \ \ \ \ \ {\isacharparenleft}{\kern0pt}complex{\isacharunderscore}{\kern0pt}of{\isacharunderscore}{\kern0pt}real\ {\isacharparenleft}{\kern0pt}{\isadigit{1}}\ {\isacharslash}{\kern0pt}\ sqrt\ {\isadigit{2}}{\isacharparenright}{\kern0pt}\ {\isasymcdot}\isactrlsub m\ {\isacharparenleft}{\kern0pt}\ {\isacharbar}{\kern0pt}Deutsch{\isachardot}{\kern0pt}zero{\isasymrangle}\ {\isacharplus}{\kern0pt}\ \isanewline
\ \ \ \ \ \ \ \ \ \ \ \ \ \ \ \ \ \ \ \ \ \ \ exp\ {\isacharparenleft}{\kern0pt}{\isadigit{2}}\ {\isacharasterisk}{\kern0pt}\ {\isasymi}\ {\isacharasterisk}{\kern0pt}\ complex{\isacharunderscore}{\kern0pt}of{\isacharunderscore}{\kern0pt}real\ pi\ {\isacharasterisk}{\kern0pt}\ complex{\isacharunderscore}{\kern0pt}of{\isacharunderscore}{\kern0pt}nat\ {\isadigit{0}}\ {\isacharslash}{\kern0pt}\ {\isadigit{2}}{\isacharparenright}{\kern0pt}\ {\isasymcdot}\isactrlsub m\ {\isacharbar}{\kern0pt}Deutsch{\isachardot}{\kern0pt}one{\isasymrangle}{\isacharparenright}{\kern0pt}{\isacharparenright}{\kern0pt}\ {\isachardollar}{\kern0pt}{\isachardollar}{\kern0pt}\isanewline
\ \ \ \ \ \ \ \ \ \ \ \ \ \ \ \ \ \ \ \ \ \ \ {\isacharparenleft}{\kern0pt}i{\isacharcomma}{\kern0pt}\ j{\isacharparenright}{\kern0pt}{\isachardoublequoteclose}\ \isanewline
\ \ \ \ \ \ \ \ \ \ \isacommand{using}\isamarkupfalse%
\ j{\isadigit{0}}\ i{\isadigit{2}}\ ai\ aj\ \isacommand{by}\isamarkupfalse%
\ auto\isanewline
\ \ \ \ \ \ \isacommand{next}\isamarkupfalse%
\isanewline
\ \ \ \ \ \ \ \ \isacommand{show}\isamarkupfalse%
\ {\isachardoublequoteopen}dim{\isacharunderscore}{\kern0pt}row\ {\isacharparenleft}{\kern0pt}Tensor{\isachardot}{\kern0pt}mat{\isacharunderscore}{\kern0pt}of{\isacharunderscore}{\kern0pt}cols{\isacharunderscore}{\kern0pt}list\ {\isadigit{2}}\ {\isacharparenleft}{\kern0pt}map\ {\isacharparenleft}{\kern0pt}map\ complex{\isacharunderscore}{\kern0pt}of{\isacharunderscore}{\kern0pt}real{\isacharparenright}{\kern0pt}\isanewline
\ \ \ \ \ \ \ \ \ \ \ \ \ \ {\isacharbrackleft}{\kern0pt}{\isacharbrackleft}{\kern0pt}{\isadigit{1}}\ {\isacharslash}{\kern0pt}\ sqrt\ {\isadigit{2}}{\isacharcomma}{\kern0pt}\ {\isadigit{1}}\ {\isacharslash}{\kern0pt}\ sqrt\ {\isadigit{2}}{\isacharbrackright}{\kern0pt}{\isacharbrackright}{\kern0pt}{\isacharparenright}{\kern0pt}{\isacharparenright}{\kern0pt}\ {\isacharequal}{\kern0pt}\ dim{\isacharunderscore}{\kern0pt}row\ {\isacharparenleft}{\kern0pt}complex{\isacharunderscore}{\kern0pt}of{\isacharunderscore}{\kern0pt}real\ {\isacharparenleft}{\kern0pt}{\isadigit{1}}\ {\isacharslash}{\kern0pt}\ sqrt\ {\isadigit{2}}{\isacharparenright}{\kern0pt}\ {\isasymcdot}\isactrlsub m\isanewline
\ \ \ \ \ \ \ \ \ \ \ \ \ \ {\isacharparenleft}{\kern0pt}\ {\isacharbar}{\kern0pt}Deutsch{\isachardot}{\kern0pt}zero{\isasymrangle}\ {\isacharplus}{\kern0pt}\ exp\ {\isacharparenleft}{\kern0pt}{\isadigit{2}}\ {\isacharasterisk}{\kern0pt}\ {\isasymi}\ {\isacharasterisk}{\kern0pt}\ complex{\isacharunderscore}{\kern0pt}of{\isacharunderscore}{\kern0pt}real\ pi\ {\isacharasterisk}{\kern0pt}\ complex{\isacharunderscore}{\kern0pt}of{\isacharunderscore}{\kern0pt}nat\ {\isadigit{0}}\ {\isacharslash}{\kern0pt}{\isadigit{2}}{\isacharparenright}{\kern0pt}\ {\isasymcdot}\isactrlsub m\isanewline
\ \ \ \ \ \ \ \ \ \ \ \ \ \ \ \ {\isacharbar}{\kern0pt}Deutsch{\isachardot}{\kern0pt}one{\isasymrangle}{\isacharparenright}{\kern0pt}{\isacharparenright}{\kern0pt}{\isachardoublequoteclose}\ \isanewline
\ \ \ \ \ \ \ \ \ \ \isacommand{using}\isamarkupfalse%
\ mat{\isacharunderscore}{\kern0pt}of{\isacharunderscore}{\kern0pt}cols{\isacharunderscore}{\kern0pt}list{\isacharunderscore}{\kern0pt}def\ index{\isacharunderscore}{\kern0pt}mat{\isacharunderscore}{\kern0pt}of{\isacharunderscore}{\kern0pt}cols{\isacharunderscore}{\kern0pt}list\ smult{\isacharunderscore}{\kern0pt}carrier{\isacharunderscore}{\kern0pt}mat\ ket{\isacharunderscore}{\kern0pt}vec{\isacharunderscore}{\kern0pt}def\ \isacommand{by}\isamarkupfalse%
\ auto\isanewline
\ \ \ \ \ \ \isacommand{next}\isamarkupfalse%
\isanewline
\ \ \ \ \ \ \ \ \isacommand{show}\isamarkupfalse%
\ {\isachardoublequoteopen}dim{\isacharunderscore}{\kern0pt}col\ {\isacharparenleft}{\kern0pt}Tensor{\isachardot}{\kern0pt}mat{\isacharunderscore}{\kern0pt}of{\isacharunderscore}{\kern0pt}cols{\isacharunderscore}{\kern0pt}list\ {\isadigit{2}}\ {\isacharparenleft}{\kern0pt}map\ {\isacharparenleft}{\kern0pt}map\ complex{\isacharunderscore}{\kern0pt}of{\isacharunderscore}{\kern0pt}real{\isacharparenright}{\kern0pt}\isanewline
\ \ \ \ \ \ \ \ \ \ \ \ \ \ {\isacharbrackleft}{\kern0pt}{\isacharbrackleft}{\kern0pt}{\isadigit{1}}\ {\isacharslash}{\kern0pt}\ sqrt\ {\isadigit{2}}{\isacharcomma}{\kern0pt}\ {\isadigit{1}}\ {\isacharslash}{\kern0pt}\ sqrt\ {\isadigit{2}}{\isacharbrackright}{\kern0pt}{\isacharbrackright}{\kern0pt}{\isacharparenright}{\kern0pt}{\isacharparenright}{\kern0pt}\ {\isacharequal}{\kern0pt}\ dim{\isacharunderscore}{\kern0pt}col\ {\isacharparenleft}{\kern0pt}complex{\isacharunderscore}{\kern0pt}of{\isacharunderscore}{\kern0pt}real\ {\isacharparenleft}{\kern0pt}{\isadigit{1}}\ {\isacharslash}{\kern0pt}\ sqrt\ {\isadigit{2}}{\isacharparenright}{\kern0pt}\ {\isasymcdot}\isactrlsub m\isanewline
\ \ \ \ \ \ \ \ \ \ \ \ \ \ {\isacharparenleft}{\kern0pt}\ {\isacharbar}{\kern0pt}Deutsch{\isachardot}{\kern0pt}zero{\isasymrangle}\ {\isacharplus}{\kern0pt}\ exp\ {\isacharparenleft}{\kern0pt}{\isadigit{2}}\ {\isacharasterisk}{\kern0pt}\ {\isasymi}\ {\isacharasterisk}{\kern0pt}\ complex{\isacharunderscore}{\kern0pt}of{\isacharunderscore}{\kern0pt}real\ pi\ {\isacharasterisk}{\kern0pt}\ complex{\isacharunderscore}{\kern0pt}of{\isacharunderscore}{\kern0pt}nat\ {\isadigit{0}}\ {\isacharslash}{\kern0pt}{\isadigit{2}}{\isacharparenright}{\kern0pt}\ {\isasymcdot}\isactrlsub m\isanewline
\ \ \ \ \ \ \ \ \ \ \ \ \ \ \ \ {\isacharbar}{\kern0pt}Deutsch{\isachardot}{\kern0pt}one{\isasymrangle}{\isacharparenright}{\kern0pt}{\isacharparenright}{\kern0pt}{\isachardoublequoteclose}\isanewline
\ \ \ \ \ \ \ \ \ \ \isacommand{using}\isamarkupfalse%
\ mat{\isacharunderscore}{\kern0pt}of{\isacharunderscore}{\kern0pt}cols{\isacharunderscore}{\kern0pt}list{\isacharunderscore}{\kern0pt}def\ index{\isacharunderscore}{\kern0pt}mat{\isacharunderscore}{\kern0pt}of{\isacharunderscore}{\kern0pt}cols{\isacharunderscore}{\kern0pt}list\ smult{\isacharunderscore}{\kern0pt}carrier{\isacharunderscore}{\kern0pt}mat\ ket{\isacharunderscore}{\kern0pt}vec{\isacharunderscore}{\kern0pt}def\ \isacommand{by}\isamarkupfalse%
\ auto\isanewline
\ \ \ \ \ \ \isacommand{qed}\isamarkupfalse%
\isanewline
\ \ \ \ \ \ \isacommand{finally}\isamarkupfalse%
\ \isacommand{show}\isamarkupfalse%
\ {\isacharquery}{\kern0pt}thesis\ \isacommand{using}\isamarkupfalse%
\ jd{\isadigit{0}}\ \isacommand{by}\isamarkupfalse%
\ simp\isanewline
\ \ \ \ \isacommand{next}\isamarkupfalse%
\isanewline
\ \ \ \ \ \ \isacommand{assume}\isamarkupfalse%
\ jd{\isadigit{1}}{\isacharcolon}{\kern0pt}{\isachardoublequoteopen}jd\ {\isacharequal}{\kern0pt}\ {\isadigit{1}}{\isachardoublequoteclose}\isanewline
\ \ \ \ \ \ \isacommand{have}\isamarkupfalse%
\ {\isachardoublequoteopen}H\ {\isacharasterisk}{\kern0pt}\ {\isacharbar}{\kern0pt}state{\isacharunderscore}{\kern0pt}basis\ {\isadigit{1}}\ {\isadigit{1}}{\isasymrangle}\ {\isacharequal}{\kern0pt}\ \isanewline
\ \ \ \ \ \ \ \ \ \ \ \ mat{\isacharunderscore}{\kern0pt}of{\isacharunderscore}{\kern0pt}cols{\isacharunderscore}{\kern0pt}list\ {\isadigit{2}}\ {\isacharparenleft}{\kern0pt}map\ {\isacharparenleft}{\kern0pt}map\ complex{\isacharunderscore}{\kern0pt}of{\isacharunderscore}{\kern0pt}real{\isacharparenright}{\kern0pt}\ {\isacharbrackleft}{\kern0pt}{\isacharbrackleft}{\kern0pt}{\isadigit{1}}\ {\isacharslash}{\kern0pt}\ sqrt\ {\isadigit{2}}{\isacharcomma}{\kern0pt}\ {\isacharminus}{\kern0pt}\ {\isadigit{1}}\ {\isacharslash}{\kern0pt}\ sqrt\ {\isadigit{2}}{\isacharbrackright}{\kern0pt}{\isacharbrackright}{\kern0pt}{\isacharparenright}{\kern0pt}{\isachardoublequoteclose}\isanewline
\ \ \ \ \ \ \ \ \isacommand{using}\isamarkupfalse%
\ H{\isacharunderscore}{\kern0pt}on{\isacharunderscore}{\kern0pt}ket{\isacharunderscore}{\kern0pt}one\ map{\isacharunderscore}{\kern0pt}def\ \isacommand{by}\isamarkupfalse%
\ {\isacharparenleft}{\kern0pt}simp\ add{\isacharcolon}{\kern0pt}\ state{\isacharunderscore}{\kern0pt}basis{\isacharunderscore}{\kern0pt}def{\isacharparenright}{\kern0pt}\isanewline
\ \ \ \ \ \ \isacommand{also}\isamarkupfalse%
\ \isacommand{have}\isamarkupfalse%
\ {\isachardoublequoteopen}{\isasymdots}\ {\isacharequal}{\kern0pt}\ {\isacharparenleft}{\kern0pt}{\isadigit{1}}\ {\isacharslash}{\kern0pt}\ sqrt\ {\isadigit{2}}{\isacharparenright}{\kern0pt}\ {\isasymcdot}\isactrlsub m\ {\isacharparenleft}{\kern0pt}\ {\isacharbar}{\kern0pt}zero{\isasymrangle}\ {\isacharplus}{\kern0pt}\ exp\ {\isacharparenleft}{\kern0pt}{\isadigit{2}}{\isacharasterisk}{\kern0pt}{\isasymi}{\isacharasterisk}{\kern0pt}pi{\isacharasterisk}{\kern0pt}complex{\isacharunderscore}{\kern0pt}of{\isacharunderscore}{\kern0pt}nat\ {\isadigit{1}}\ {\isacharslash}{\kern0pt}\ {\isadigit{2}}{\isacharparenright}{\kern0pt}\ {\isasymcdot}\isactrlsub m\ {\isacharbar}{\kern0pt}one{\isasymrangle}{\isacharparenright}{\kern0pt}{\isachardoublequoteclose}\isanewline
\ \ \ \ \ \ \isacommand{proof}\isamarkupfalse%
\ \isanewline
\ \ \ \ \ \ \ \ \isacommand{fix}\isamarkupfalse%
\ i\ j\isanewline
\ \ \ \ \ \ \ \ \isacommand{assume}\isamarkupfalse%
\ ai{\isacharcolon}{\kern0pt}{\isachardoublequoteopen}i\ {\isacharless}{\kern0pt}\ dim{\isacharunderscore}{\kern0pt}row\ {\isacharparenleft}{\kern0pt}complex{\isacharunderscore}{\kern0pt}of{\isacharunderscore}{\kern0pt}real\ {\isacharparenleft}{\kern0pt}{\isadigit{1}}\ {\isacharslash}{\kern0pt}\ sqrt\ {\isadigit{2}}{\isacharparenright}{\kern0pt}\ {\isasymcdot}\isactrlsub m\ {\isacharparenleft}{\kern0pt}\ {\isacharbar}{\kern0pt}zero{\isasymrangle}\ {\isacharplus}{\kern0pt}\ \isanewline
\ \ \ \ \ \ \ \ \ \ \ \ \ \ \ \ \ \ \ \ \ \ \ exp\ {\isacharparenleft}{\kern0pt}{\isadigit{2}}{\isacharasterisk}{\kern0pt}{\isasymi}{\isacharasterisk}{\kern0pt}complex{\isacharunderscore}{\kern0pt}of{\isacharunderscore}{\kern0pt}real\ pi\ {\isacharasterisk}{\kern0pt}complex{\isacharunderscore}{\kern0pt}of{\isacharunderscore}{\kern0pt}nat\ {\isadigit{1}}\ {\isacharslash}{\kern0pt}{\isadigit{2}}{\isacharparenright}{\kern0pt}\ {\isasymcdot}\isactrlsub m\ {\isacharbar}{\kern0pt}one{\isasymrangle}{\isacharparenright}{\kern0pt}{\isacharparenright}{\kern0pt}{\isachardoublequoteclose}\isanewline
\ \ \ \ \ \ \ \ \isacommand{hence}\isamarkupfalse%
\ {\isachardoublequoteopen}i\ {\isacharless}{\kern0pt}\ {\isadigit{2}}{\isachardoublequoteclose}\ \isacommand{using}\isamarkupfalse%
\ mat{\isacharunderscore}{\kern0pt}of{\isacharunderscore}{\kern0pt}cols{\isacharunderscore}{\kern0pt}list{\isacharunderscore}{\kern0pt}def\ smult{\isacharunderscore}{\kern0pt}carrier{\isacharunderscore}{\kern0pt}mat\ ket{\isacharunderscore}{\kern0pt}vec{\isacharunderscore}{\kern0pt}def\ \isacommand{by}\isamarkupfalse%
\ simp\isanewline
\ \ \ \ \ \ \ \ \isacommand{hence}\isamarkupfalse%
\ i{\isadigit{2}}{\isacharcolon}{\kern0pt}{\isachardoublequoteopen}i\ {\isasymin}\ {\isacharbraceleft}{\kern0pt}{\isadigit{0}}{\isacharcomma}{\kern0pt}{\isadigit{1}}{\isacharbraceright}{\kern0pt}{\isachardoublequoteclose}\ \isacommand{by}\isamarkupfalse%
\ auto\isanewline
\ \ \ \ \ \ \ \ \isacommand{assume}\isamarkupfalse%
\ aj{\isacharcolon}{\kern0pt}{\isachardoublequoteopen}j\ {\isacharless}{\kern0pt}\ dim{\isacharunderscore}{\kern0pt}col\ {\isacharparenleft}{\kern0pt}complex{\isacharunderscore}{\kern0pt}of{\isacharunderscore}{\kern0pt}real\ {\isacharparenleft}{\kern0pt}{\isadigit{1}}\ {\isacharslash}{\kern0pt}\ sqrt\ {\isadigit{2}}{\isacharparenright}{\kern0pt}\ {\isasymcdot}\isactrlsub m\ {\isacharparenleft}{\kern0pt}\ {\isacharbar}{\kern0pt}zero{\isasymrangle}\ {\isacharplus}{\kern0pt}\ \isanewline
\ \ \ \ \ \ \ \ \ \ \ \ \ \ \ \ \ \ \ \ \ \ \ exp\ {\isacharparenleft}{\kern0pt}{\isadigit{2}}{\isacharasterisk}{\kern0pt}{\isasymi}{\isacharasterisk}{\kern0pt}complex{\isacharunderscore}{\kern0pt}of{\isacharunderscore}{\kern0pt}real\ pi\ {\isacharasterisk}{\kern0pt}complex{\isacharunderscore}{\kern0pt}of{\isacharunderscore}{\kern0pt}nat\ {\isadigit{1}}\ {\isacharslash}{\kern0pt}{\isadigit{2}}{\isacharparenright}{\kern0pt}\ {\isasymcdot}\isactrlsub m\ {\isacharbar}{\kern0pt}one{\isasymrangle}{\isacharparenright}{\kern0pt}{\isacharparenright}{\kern0pt}{\isachardoublequoteclose}\isanewline
\ \ \ \ \ \ \ \ \isacommand{hence}\isamarkupfalse%
\ j{\isadigit{0}}{\isacharcolon}{\kern0pt}{\isachardoublequoteopen}j\ {\isacharequal}{\kern0pt}\ {\isadigit{0}}{\isachardoublequoteclose}\ \isacommand{using}\isamarkupfalse%
\ mat{\isacharunderscore}{\kern0pt}of{\isacharunderscore}{\kern0pt}cols{\isacharunderscore}{\kern0pt}list{\isacharunderscore}{\kern0pt}def\ smult{\isacharunderscore}{\kern0pt}carrier{\isacharunderscore}{\kern0pt}mat\ ket{\isacharunderscore}{\kern0pt}vec{\isacharunderscore}{\kern0pt}def\ \isacommand{by}\isamarkupfalse%
\ simp\isanewline
\ \ \ \ \ \ \ \ \isacommand{have}\isamarkupfalse%
\ {\isachardoublequoteopen}{\isacharparenleft}{\kern0pt}mat{\isacharunderscore}{\kern0pt}of{\isacharunderscore}{\kern0pt}cols{\isacharunderscore}{\kern0pt}list\ {\isadigit{2}}\ {\isacharparenleft}{\kern0pt}map\ {\isacharparenleft}{\kern0pt}map\ complex{\isacharunderscore}{\kern0pt}of{\isacharunderscore}{\kern0pt}real{\isacharparenright}{\kern0pt}\ {\isacharbrackleft}{\kern0pt}{\isacharbrackleft}{\kern0pt}{\isadigit{1}}\ {\isacharslash}{\kern0pt}\ sqrt\ {\isadigit{2}}{\isacharcomma}{\kern0pt}{\isacharminus}{\kern0pt}{\isadigit{1}}\ {\isacharslash}{\kern0pt}\ sqrt\ {\isadigit{2}}{\isacharbrackright}{\kern0pt}{\isacharbrackright}{\kern0pt}{\isacharparenright}{\kern0pt}{\isacharparenright}{\kern0pt}\ {\isachardollar}{\kern0pt}{\isachardollar}{\kern0pt}\ {\isacharparenleft}{\kern0pt}i{\isacharcomma}{\kern0pt}{\isadigit{0}}{\isacharparenright}{\kern0pt}\ {\isacharequal}{\kern0pt}\isanewline
\ \ \ \ \ \ \ \ \ \ \ \ \ \ {\isacharparenleft}{\kern0pt}mat{\isacharunderscore}{\kern0pt}of{\isacharunderscore}{\kern0pt}cols{\isacharunderscore}{\kern0pt}list\ {\isadigit{2}}\ {\isacharbrackleft}{\kern0pt}{\isacharbrackleft}{\kern0pt}{\isadigit{1}}{\isacharslash}{\kern0pt}sqrt\ {\isadigit{2}}{\isacharcomma}{\kern0pt}{\isacharminus}{\kern0pt}\ {\isadigit{1}}{\isacharslash}{\kern0pt}sqrt\ {\isadigit{2}}{\isacharbrackright}{\kern0pt}{\isacharbrackright}{\kern0pt}{\isacharparenright}{\kern0pt}\ {\isachardollar}{\kern0pt}{\isachardollar}{\kern0pt}\ {\isacharparenleft}{\kern0pt}i{\isacharcomma}{\kern0pt}{\isadigit{0}}{\isacharparenright}{\kern0pt}{\isachardoublequoteclose}\isanewline
\ \ \ \ \ \ \ \ \ \ \isacommand{using}\isamarkupfalse%
\ map{\isacharunderscore}{\kern0pt}def\ \isacommand{by}\isamarkupfalse%
\ simp\isanewline
\ \ \ \ \ \ \ \ \isacommand{also}\isamarkupfalse%
\ \isacommand{have}\isamarkupfalse%
\ {\isachardoublequoteopen}{\isasymdots}\ {\isacharequal}{\kern0pt}\ {\isacharparenleft}{\kern0pt}{\isacharparenleft}{\kern0pt}{\isadigit{1}}{\isacharslash}{\kern0pt}sqrt\ {\isadigit{2}}{\isacharparenright}{\kern0pt}\ {\isasymcdot}\isactrlsub m\ {\isacharparenleft}{\kern0pt}mat{\isacharunderscore}{\kern0pt}of{\isacharunderscore}{\kern0pt}cols{\isacharunderscore}{\kern0pt}list\ {\isadigit{2}}\ {\isacharbrackleft}{\kern0pt}{\isacharbrackleft}{\kern0pt}{\isadigit{1}}{\isacharcomma}{\kern0pt}{\isacharminus}{\kern0pt}{\isadigit{1}}{\isacharbrackright}{\kern0pt}{\isacharbrackright}{\kern0pt}{\isacharparenright}{\kern0pt}{\isacharparenright}{\kern0pt}\ {\isachardollar}{\kern0pt}{\isachardollar}{\kern0pt}\ {\isacharparenleft}{\kern0pt}i{\isacharcomma}{\kern0pt}{\isadigit{0}}{\isacharparenright}{\kern0pt}{\isachardoublequoteclose}\isanewline
\ \ \ \ \ \ \ \ \ \ \isacommand{using}\isamarkupfalse%
\ i{\isadigit{2}}\ smult{\isacharunderscore}{\kern0pt}mat{\isacharunderscore}{\kern0pt}def\ index{\isacharunderscore}{\kern0pt}mat{\isacharunderscore}{\kern0pt}of{\isacharunderscore}{\kern0pt}cols{\isacharunderscore}{\kern0pt}list\ mat{\isacharunderscore}{\kern0pt}of{\isacharunderscore}{\kern0pt}cols{\isacharunderscore}{\kern0pt}list{\isacharunderscore}{\kern0pt}def\ Suc{\isacharunderscore}{\kern0pt}{\isadigit{1}}\ {\isacartoucheopen}i\ {\isacharless}{\kern0pt}\ {\isadigit{2}}{\isacartoucheclose}\ \isanewline
\ \ \ \ \ \ \ \ \ \ \ \ dim{\isacharunderscore}{\kern0pt}col{\isacharunderscore}{\kern0pt}mat{\isacharparenleft}{\kern0pt}{\isadigit{1}}{\isacharparenright}{\kern0pt}\ dim{\isacharunderscore}{\kern0pt}row{\isacharunderscore}{\kern0pt}mat{\isacharparenleft}{\kern0pt}{\isadigit{1}}{\isacharparenright}{\kern0pt}\ index{\isacharunderscore}{\kern0pt}smult{\isacharunderscore}{\kern0pt}mat{\isacharparenleft}{\kern0pt}{\isadigit{1}}{\isacharparenright}{\kern0pt}\ nth{\isacharunderscore}{\kern0pt}Cons{\isacharunderscore}{\kern0pt}{\isadigit{0}}\ nth{\isacharunderscore}{\kern0pt}Cons{\isacharunderscore}{\kern0pt}Suc\isanewline
\ \ \ \ \ \ \ \ \ \ \ \ ket{\isacharunderscore}{\kern0pt}one{\isacharunderscore}{\kern0pt}is{\isacharunderscore}{\kern0pt}state\ ket{\isacharunderscore}{\kern0pt}one{\isacharunderscore}{\kern0pt}to{\isacharunderscore}{\kern0pt}mat{\isacharunderscore}{\kern0pt}of{\isacharunderscore}{\kern0pt}cols{\isacharunderscore}{\kern0pt}list\isanewline
\ \ \ \ \ \ \ \ \ \ \isacommand{by}\isamarkupfalse%
\ {\isacharparenleft}{\kern0pt}smt\ {\isacharparenleft}{\kern0pt}z{\isadigit{3}}{\isacharparenright}{\kern0pt}\ One{\isacharunderscore}{\kern0pt}nat{\isacharunderscore}{\kern0pt}def\ {\isasympsi}\isactrlsub {\isadigit{0}}{\isacharunderscore}{\kern0pt}to{\isacharunderscore}{\kern0pt}{\isasympsi}\isactrlsub {\isadigit{1}}\ bot{\isacharunderscore}{\kern0pt}nat{\isacharunderscore}{\kern0pt}{\isadigit{0}}{\isachardot}{\kern0pt}not{\isacharunderscore}{\kern0pt}eq{\isacharunderscore}{\kern0pt}extremum\ dim{\isacharunderscore}{\kern0pt}col{\isacharunderscore}{\kern0pt}tensor{\isacharunderscore}{\kern0pt}mat\ \isanewline
\ \ \ \ \ \ \ \ \ \ \ \ \ \ less{\isacharunderscore}{\kern0pt}{\isadigit{2}}{\isacharunderscore}{\kern0pt}cases{\isacharunderscore}{\kern0pt}iff\ list{\isachardot}{\kern0pt}map{\isacharparenleft}{\kern0pt}{\isadigit{2}}{\isacharparenright}{\kern0pt}\ list{\isachardot}{\kern0pt}size{\isacharparenleft}{\kern0pt}{\isadigit{4}}{\isacharparenright}{\kern0pt}\ mult{\isacharunderscore}{\kern0pt}{\isadigit{0}}{\isacharunderscore}{\kern0pt}right\ mult{\isacharunderscore}{\kern0pt}{\isadigit{1}}\ of{\isacharunderscore}{\kern0pt}real{\isacharunderscore}{\kern0pt}{\isadigit{1}}\ \isanewline
\ \ \ \ \ \ \ \ \ \ \ \ \ \ of{\isacharunderscore}{\kern0pt}real{\isacharunderscore}{\kern0pt}divide\ of{\isacharunderscore}{\kern0pt}real{\isacharunderscore}{\kern0pt}minus\ state{\isacharunderscore}{\kern0pt}def\ times{\isacharunderscore}{\kern0pt}divide{\isacharunderscore}{\kern0pt}eq{\isacharunderscore}{\kern0pt}left{\isacharparenright}{\kern0pt}\isanewline
\ \ \ \ \ \ \ \ \isacommand{also}\isamarkupfalse%
\ \isacommand{have}\isamarkupfalse%
\ {\isachardoublequoteopen}{\isasymdots}\ {\isacharequal}{\kern0pt}\ {\isacharparenleft}{\kern0pt}{\isadigit{1}}{\isacharslash}{\kern0pt}sqrt\ {\isadigit{2}}\ {\isasymcdot}\isactrlsub m\ {\isacharparenleft}{\kern0pt}\ {\isacharbar}{\kern0pt}zero{\isasymrangle}\ {\isacharminus}{\kern0pt}\ {\isacharbar}{\kern0pt}one{\isasymrangle}{\isacharparenright}{\kern0pt}{\isacharparenright}{\kern0pt}\ {\isachardollar}{\kern0pt}{\isachardollar}{\kern0pt}\ {\isacharparenleft}{\kern0pt}i{\isacharcomma}{\kern0pt}{\isadigit{0}}{\isacharparenright}{\kern0pt}{\isachardoublequoteclose}\isanewline
\ \ \ \ \ \ \ \ \isacommand{proof}\isamarkupfalse%
\ {\isacharminus}{\kern0pt}\isanewline
\ \ \ \ \ \ \ \ \ \ \isacommand{define}\isamarkupfalse%
\ r{\isadigit{1}}\ r{\isadigit{2}}\ \isakeyword{where}\ {\isachardoublequoteopen}r{\isadigit{1}}\ {\isacharequal}{\kern0pt}\ mat{\isacharunderscore}{\kern0pt}of{\isacharunderscore}{\kern0pt}cols{\isacharunderscore}{\kern0pt}list\ {\isadigit{2}}\ {\isacharbrackleft}{\kern0pt}{\isacharbrackleft}{\kern0pt}{\isadigit{1}}{\isacharcomma}{\kern0pt}{\isacharminus}{\kern0pt}{\isadigit{1}}{\isacharbrackright}{\kern0pt}{\isacharbrackright}{\kern0pt}{\isachardoublequoteclose}\ \isakeyword{and}\ {\isachardoublequoteopen}r{\isadigit{2}}\ {\isacharequal}{\kern0pt}\ {\isacharbar}{\kern0pt}zero{\isasymrangle}\ {\isacharminus}{\kern0pt}\ {\isacharbar}{\kern0pt}one{\isasymrangle}{\isachardoublequoteclose}\isanewline
\ \ \ \ \ \ \ \ \ \ \isacommand{have}\isamarkupfalse%
\ {\isachardoublequoteopen}r{\isadigit{1}}\ {\isachardollar}{\kern0pt}{\isachardollar}{\kern0pt}\ {\isacharparenleft}{\kern0pt}{\isadigit{0}}{\isacharcomma}{\kern0pt}{\isadigit{0}}{\isacharparenright}{\kern0pt}\ {\isacharequal}{\kern0pt}\ r{\isadigit{2}}\ {\isachardollar}{\kern0pt}{\isachardollar}{\kern0pt}\ {\isacharparenleft}{\kern0pt}{\isadigit{0}}{\isacharcomma}{\kern0pt}{\isadigit{0}}{\isacharparenright}{\kern0pt}{\isachardoublequoteclose}\ \isacommand{using}\isamarkupfalse%
\ r{\isadigit{1}}{\isacharunderscore}{\kern0pt}def\ r{\isadigit{2}}{\isacharunderscore}{\kern0pt}def\ mat{\isacharunderscore}{\kern0pt}of{\isacharunderscore}{\kern0pt}cols{\isacharunderscore}{\kern0pt}list{\isacharunderscore}{\kern0pt}def\isanewline
\ \ \ \ \ \ \ \ \ \ \ \ \isacommand{by}\isamarkupfalse%
\ {\isacharparenleft}{\kern0pt}smt\ {\isacharparenleft}{\kern0pt}verit{\isacharcomma}{\kern0pt}\ ccfv{\isacharunderscore}{\kern0pt}threshold{\isacharparenright}{\kern0pt}\ One{\isacharunderscore}{\kern0pt}nat{\isacharunderscore}{\kern0pt}def\ add{\isachardot}{\kern0pt}commute\ diff{\isacharunderscore}{\kern0pt}zero\ dim{\isacharunderscore}{\kern0pt}row{\isacharunderscore}{\kern0pt}mat{\isacharparenleft}{\kern0pt}{\isadigit{1}}{\isacharparenright}{\kern0pt}\ \isanewline
\ \ \ \ \ \ \ \ \ \ \ \ \ \ \ \ index{\isacharunderscore}{\kern0pt}mat{\isacharparenleft}{\kern0pt}{\isadigit{1}}{\isacharparenright}{\kern0pt}\ index{\isacharunderscore}{\kern0pt}mat{\isacharunderscore}{\kern0pt}of{\isacharunderscore}{\kern0pt}cols{\isacharunderscore}{\kern0pt}list\ ket{\isacharunderscore}{\kern0pt}one{\isacharunderscore}{\kern0pt}is{\isacharunderscore}{\kern0pt}state\ ket{\isacharunderscore}{\kern0pt}one{\isacharunderscore}{\kern0pt}to{\isacharunderscore}{\kern0pt}mat{\isacharunderscore}{\kern0pt}of{\isacharunderscore}{\kern0pt}cols{\isacharunderscore}{\kern0pt}list\ \isanewline
\ \ \ \ \ \ \ \ \ \ \ \ \ \ \ \ ket{\isacharunderscore}{\kern0pt}zero{\isacharunderscore}{\kern0pt}to{\isacharunderscore}{\kern0pt}mat{\isacharunderscore}{\kern0pt}of{\isacharunderscore}{\kern0pt}cols{\isacharunderscore}{\kern0pt}list\ list{\isachardot}{\kern0pt}size{\isacharparenleft}{\kern0pt}{\isadigit{3}}{\isacharparenright}{\kern0pt}\ list{\isachardot}{\kern0pt}size{\isacharparenleft}{\kern0pt}{\isadigit{4}}{\isacharparenright}{\kern0pt}\ minus{\isacharunderscore}{\kern0pt}mat{\isacharunderscore}{\kern0pt}def\ nth{\isacharunderscore}{\kern0pt}Cons{\isacharunderscore}{\kern0pt}{\isadigit{0}}\ \isanewline
\ \ \ \ \ \ \ \ \ \ \ \ \ \ \ \ plus{\isacharunderscore}{\kern0pt}{\isadigit{1}}{\isacharunderscore}{\kern0pt}eq{\isacharunderscore}{\kern0pt}Suc\ pos{\isadigit{2}}\ state{\isacharunderscore}{\kern0pt}def\ zero{\isacharunderscore}{\kern0pt}less{\isacharunderscore}{\kern0pt}one{\isacharunderscore}{\kern0pt}class{\isachardot}{\kern0pt}zero{\isacharunderscore}{\kern0pt}less{\isacharunderscore}{\kern0pt}one{\isacharparenright}{\kern0pt}\isanewline
\ \ \ \ \ \ \ \ \ \ \isacommand{moreover}\isamarkupfalse%
\ \isacommand{have}\isamarkupfalse%
\ {\isachardoublequoteopen}r{\isadigit{1}}\ {\isachardollar}{\kern0pt}{\isachardollar}{\kern0pt}\ {\isacharparenleft}{\kern0pt}{\isadigit{1}}{\isacharcomma}{\kern0pt}{\isadigit{0}}{\isacharparenright}{\kern0pt}\ {\isacharequal}{\kern0pt}\ r{\isadigit{2}}\ {\isachardollar}{\kern0pt}{\isachardollar}{\kern0pt}\ {\isacharparenleft}{\kern0pt}{\isadigit{1}}{\isacharcomma}{\kern0pt}{\isadigit{0}}{\isacharparenright}{\kern0pt}{\isachardoublequoteclose}\ \isanewline
\ \ \ \ \ \ \ \ \ \ \ \ \isacommand{using}\isamarkupfalse%
\ r{\isadigit{1}}{\isacharunderscore}{\kern0pt}def\ r{\isadigit{2}}{\isacharunderscore}{\kern0pt}def\ mat{\isacharunderscore}{\kern0pt}of{\isacharunderscore}{\kern0pt}cols{\isacharunderscore}{\kern0pt}list{\isacharunderscore}{\kern0pt}def\ ket{\isacharunderscore}{\kern0pt}vec{\isacharunderscore}{\kern0pt}def\ \isacommand{by}\isamarkupfalse%
\ simp\isanewline
\ \ \ \ \ \ \ \ \ \ \isacommand{ultimately}\isamarkupfalse%
\ \isacommand{show}\isamarkupfalse%
\ {\isacharquery}{\kern0pt}thesis\ \isacommand{using}\isamarkupfalse%
\ r{\isadigit{1}}{\isacharunderscore}{\kern0pt}def\ r{\isadigit{2}}{\isacharunderscore}{\kern0pt}def\ i{\isadigit{2}}\ \isanewline
\ \ \ \ \ \ \ \ \ \ \ \ \isacommand{by}\isamarkupfalse%
\ {\isacharparenleft}{\kern0pt}smt\ {\isacharparenleft}{\kern0pt}verit{\isacharparenright}{\kern0pt}\ One{\isacharunderscore}{\kern0pt}nat{\isacharunderscore}{\kern0pt}def\ Tensor{\isachardot}{\kern0pt}mat{\isacharunderscore}{\kern0pt}of{\isacharunderscore}{\kern0pt}cols{\isacharunderscore}{\kern0pt}list{\isacharunderscore}{\kern0pt}def\ {\isacartoucheopen}i\ {\isacharless}{\kern0pt}\ {\isadigit{2}}{\isacartoucheclose}\ add{\isachardot}{\kern0pt}commute\ \isanewline
\ \ \ \ \ \ \ \ \ \ \ \ \ \ \ \ dim{\isacharunderscore}{\kern0pt}col{\isacharunderscore}{\kern0pt}mat{\isacharparenleft}{\kern0pt}{\isadigit{1}}{\isacharparenright}{\kern0pt}\ dim{\isacharunderscore}{\kern0pt}row{\isacharunderscore}{\kern0pt}mat{\isacharparenleft}{\kern0pt}{\isadigit{1}}{\isacharparenright}{\kern0pt}\ empty{\isacharunderscore}{\kern0pt}iff\ index{\isacharunderscore}{\kern0pt}smult{\isacharunderscore}{\kern0pt}mat{\isacharparenleft}{\kern0pt}{\isadigit{1}}{\isacharparenright}{\kern0pt}\ index{\isacharunderscore}{\kern0pt}unit{\isacharunderscore}{\kern0pt}vec{\isacharparenleft}{\kern0pt}{\isadigit{3}}{\isacharparenright}{\kern0pt}\ \isanewline
\ \ \ \ \ \ \ \ \ \ \ \ \ \ \ \ insert{\isacharunderscore}{\kern0pt}iff\ ket{\isacharunderscore}{\kern0pt}vec{\isacharunderscore}{\kern0pt}def\ list{\isachardot}{\kern0pt}size{\isacharparenleft}{\kern0pt}{\isadigit{3}}{\isacharparenright}{\kern0pt}\ list{\isachardot}{\kern0pt}size{\isacharparenleft}{\kern0pt}{\isadigit{4}}{\isacharparenright}{\kern0pt}\ minus{\isacharunderscore}{\kern0pt}mat{\isacharunderscore}{\kern0pt}def\ plus{\isacharunderscore}{\kern0pt}{\isadigit{1}}{\isacharunderscore}{\kern0pt}eq{\isacharunderscore}{\kern0pt}Suc\ \isanewline
\ \ \ \ \ \ \ \ \ \ \ \ \ \ \ \ zero{\isacharunderscore}{\kern0pt}less{\isacharunderscore}{\kern0pt}one{\isacharunderscore}{\kern0pt}class{\isachardot}{\kern0pt}zero{\isacharunderscore}{\kern0pt}less{\isacharunderscore}{\kern0pt}one{\isacharparenright}{\kern0pt}\isanewline
\ \ \ \ \ \ \ \ \isacommand{qed}\isamarkupfalse%
\isanewline
\ \ \ \ \ \ \ \ \isacommand{also}\isamarkupfalse%
\ \isacommand{have}\isamarkupfalse%
\ {\isachardoublequoteopen}{\isasymdots}\ {\isacharequal}{\kern0pt}\ {\isacharparenleft}{\kern0pt}{\isadigit{1}}{\isacharslash}{\kern0pt}sqrt\ {\isadigit{2}}\ {\isasymcdot}\isactrlsub m\ {\isacharparenleft}{\kern0pt}\ {\isacharbar}{\kern0pt}zero{\isasymrangle}\ {\isacharplus}{\kern0pt}\ {\isacharparenleft}{\kern0pt}{\isacharminus}{\kern0pt}{\isadigit{1}}{\isacharparenright}{\kern0pt}\ {\isasymcdot}\isactrlsub m\ {\isacharbar}{\kern0pt}one{\isasymrangle}{\isacharparenright}{\kern0pt}{\isacharparenright}{\kern0pt}\ {\isachardollar}{\kern0pt}{\isachardollar}{\kern0pt}\ {\isacharparenleft}{\kern0pt}i{\isacharcomma}{\kern0pt}{\isadigit{0}}{\isacharparenright}{\kern0pt}{\isachardoublequoteclose}\isanewline
\ \ \ \ \ \ \ \ \ \ \isacommand{using}\isamarkupfalse%
\ smult{\isacharunderscore}{\kern0pt}mat{\isacharunderscore}{\kern0pt}def\ {\isacartoucheopen}i\ {\isacharless}{\kern0pt}\ {\isadigit{2}}{\isacartoucheclose}\ ket{\isacharunderscore}{\kern0pt}one{\isacharunderscore}{\kern0pt}is{\isacharunderscore}{\kern0pt}state\ state{\isacharunderscore}{\kern0pt}def\ \isacommand{by}\isamarkupfalse%
\ force\isanewline
\ \ \ \ \ \ \ \ \isacommand{also}\isamarkupfalse%
\ \isacommand{have}\isamarkupfalse%
\ {\isachardoublequoteopen}{\isasymdots}\ {\isacharequal}{\kern0pt}\ {\isacharparenleft}{\kern0pt}{\isadigit{1}}{\isacharslash}{\kern0pt}sqrt\ {\isadigit{2}}\ {\isasymcdot}\isactrlsub m\ {\isacharparenleft}{\kern0pt}\ {\isacharbar}{\kern0pt}zero{\isasymrangle}\ {\isacharplus}{\kern0pt}\ exp\ {\isacharparenleft}{\kern0pt}{\isadigit{2}}{\isacharasterisk}{\kern0pt}{\isasymi}{\isacharasterisk}{\kern0pt}pi{\isacharasterisk}{\kern0pt}complex{\isacharunderscore}{\kern0pt}of{\isacharunderscore}{\kern0pt}nat\ {\isadigit{1}}\ {\isacharslash}{\kern0pt}\ {\isadigit{2}}{\isacharparenright}{\kern0pt}\ {\isasymcdot}\isactrlsub m\ {\isacharbar}{\kern0pt}one{\isasymrangle}{\isacharparenright}{\kern0pt}{\isacharparenright}{\kern0pt}\ {\isachardollar}{\kern0pt}{\isachardollar}{\kern0pt}\ {\isacharparenleft}{\kern0pt}i{\isacharcomma}{\kern0pt}{\isadigit{0}}{\isacharparenright}{\kern0pt}{\isachardoublequoteclose}\isanewline
\ \ \ \ \ \ \ \ \ \ \isacommand{using}\isamarkupfalse%
\ exp{\isacharunderscore}{\kern0pt}pi{\isacharunderscore}{\kern0pt}i{\isacharprime}{\kern0pt}\ \isacommand{by}\isamarkupfalse%
\ auto\isanewline
\ \ \ \ \ \ \ \ \isacommand{finally}\isamarkupfalse%
\ \isacommand{show}\isamarkupfalse%
\ {\isachardoublequoteopen}mat{\isacharunderscore}{\kern0pt}of{\isacharunderscore}{\kern0pt}cols{\isacharunderscore}{\kern0pt}list\ {\isadigit{2}}\ {\isacharparenleft}{\kern0pt}map\ {\isacharparenleft}{\kern0pt}map\ complex{\isacharunderscore}{\kern0pt}of{\isacharunderscore}{\kern0pt}real{\isacharparenright}{\kern0pt}\ {\isacharbrackleft}{\kern0pt}{\isacharbrackleft}{\kern0pt}{\isadigit{1}}{\isacharslash}{\kern0pt}sqrt\ {\isadigit{2}}{\isacharcomma}{\kern0pt}{\isacharminus}{\kern0pt}{\isadigit{1}}{\isacharslash}{\kern0pt}sqrt\ {\isadigit{2}}{\isacharbrackright}{\kern0pt}{\isacharbrackright}{\kern0pt}{\isacharparenright}{\kern0pt}\ {\isachardollar}{\kern0pt}{\isachardollar}{\kern0pt}\ {\isacharparenleft}{\kern0pt}i{\isacharcomma}{\kern0pt}j{\isacharparenright}{\kern0pt}\isanewline
\ \ \ \ \ \ \ \ \ \ \ \ \ \ \ \ \ \ \ {\isacharequal}{\kern0pt}\ {\isacharparenleft}{\kern0pt}complex{\isacharunderscore}{\kern0pt}of{\isacharunderscore}{\kern0pt}real\ {\isacharparenleft}{\kern0pt}{\isadigit{1}}\ {\isacharslash}{\kern0pt}\ sqrt\ {\isadigit{2}}{\isacharparenright}{\kern0pt}\ {\isasymcdot}\isactrlsub m\ {\isacharparenleft}{\kern0pt}\ {\isacharbar}{\kern0pt}zero{\isasymrangle}\ {\isacharplus}{\kern0pt}\ exp\ {\isacharparenleft}{\kern0pt}{\isadigit{2}}{\isacharasterisk}{\kern0pt}{\isasymi}{\isacharasterisk}{\kern0pt}pi{\isacharasterisk}{\kern0pt}complex{\isacharunderscore}{\kern0pt}of{\isacharunderscore}{\kern0pt}nat\ {\isadigit{1}}\ {\isacharslash}{\kern0pt}{\isadigit{2}}{\isacharparenright}{\kern0pt}\ {\isasymcdot}\isactrlsub m\isanewline
\ \ \ \ \ \ \ \ \ \ \ \ \ \ \ \ \ \ \ \ \ {\isacharbar}{\kern0pt}one{\isasymrangle}{\isacharparenright}{\kern0pt}{\isacharparenright}{\kern0pt}\ {\isachardollar}{\kern0pt}{\isachardollar}{\kern0pt}\ {\isacharparenleft}{\kern0pt}i{\isacharcomma}{\kern0pt}\ j{\isacharparenright}{\kern0pt}{\isachardoublequoteclose}\ \isacommand{using}\isamarkupfalse%
\ i{\isadigit{2}}\ ai\ aj\ j{\isadigit{0}}\ \isacommand{by}\isamarkupfalse%
\ auto\isanewline
\ \ \ \ \ \ \isacommand{next}\isamarkupfalse%
\isanewline
\ \ \ \ \ \ \ \ \isacommand{show}\isamarkupfalse%
\ {\isachardoublequoteopen}dim{\isacharunderscore}{\kern0pt}row\ {\isacharparenleft}{\kern0pt}Tensor{\isachardot}{\kern0pt}mat{\isacharunderscore}{\kern0pt}of{\isacharunderscore}{\kern0pt}cols{\isacharunderscore}{\kern0pt}list\ {\isadigit{2}}\ {\isacharparenleft}{\kern0pt}map\ {\isacharparenleft}{\kern0pt}map\ complex{\isacharunderscore}{\kern0pt}of{\isacharunderscore}{\kern0pt}real{\isacharparenright}{\kern0pt}\isanewline
\ \ \ \ \ \ \ \ \ \ \ \ \ \ {\isacharbrackleft}{\kern0pt}{\isacharbrackleft}{\kern0pt}{\isadigit{1}}\ {\isacharslash}{\kern0pt}\ sqrt\ {\isadigit{2}}{\isacharcomma}{\kern0pt}{\isacharminus}{\kern0pt}\ {\isadigit{1}}\ {\isacharslash}{\kern0pt}\ sqrt\ {\isadigit{2}}{\isacharbrackright}{\kern0pt}{\isacharbrackright}{\kern0pt}{\isacharparenright}{\kern0pt}{\isacharparenright}{\kern0pt}\ {\isacharequal}{\kern0pt}\ dim{\isacharunderscore}{\kern0pt}row\ {\isacharparenleft}{\kern0pt}complex{\isacharunderscore}{\kern0pt}of{\isacharunderscore}{\kern0pt}real\ {\isacharparenleft}{\kern0pt}{\isadigit{1}}\ {\isacharslash}{\kern0pt}\ sqrt\ {\isadigit{2}}{\isacharparenright}{\kern0pt}\ {\isasymcdot}\isactrlsub m\isanewline
\ \ \ \ \ \ \ \ \ \ \ \ \ \ {\isacharparenleft}{\kern0pt}\ {\isacharbar}{\kern0pt}Deutsch{\isachardot}{\kern0pt}zero{\isasymrangle}\ {\isacharplus}{\kern0pt}\ exp\ {\isacharparenleft}{\kern0pt}{\isadigit{2}}\ {\isacharasterisk}{\kern0pt}\ {\isasymi}\ {\isacharasterisk}{\kern0pt}\ complex{\isacharunderscore}{\kern0pt}of{\isacharunderscore}{\kern0pt}real\ pi\ {\isacharasterisk}{\kern0pt}\ complex{\isacharunderscore}{\kern0pt}of{\isacharunderscore}{\kern0pt}nat\ {\isadigit{1}}\ {\isacharslash}{\kern0pt}{\isadigit{2}}{\isacharparenright}{\kern0pt}\ {\isasymcdot}\isactrlsub m\isanewline
\ \ \ \ \ \ \ \ \ \ \ \ \ \ \ \ {\isacharbar}{\kern0pt}Deutsch{\isachardot}{\kern0pt}one{\isasymrangle}{\isacharparenright}{\kern0pt}{\isacharparenright}{\kern0pt}{\isachardoublequoteclose}\ \isanewline
\ \ \ \ \ \ \ \ \ \ \isacommand{using}\isamarkupfalse%
\ mat{\isacharunderscore}{\kern0pt}of{\isacharunderscore}{\kern0pt}cols{\isacharunderscore}{\kern0pt}list{\isacharunderscore}{\kern0pt}def\ index{\isacharunderscore}{\kern0pt}mat{\isacharunderscore}{\kern0pt}of{\isacharunderscore}{\kern0pt}cols{\isacharunderscore}{\kern0pt}list\ smult{\isacharunderscore}{\kern0pt}carrier{\isacharunderscore}{\kern0pt}mat\ ket{\isacharunderscore}{\kern0pt}vec{\isacharunderscore}{\kern0pt}def\ \isacommand{by}\isamarkupfalse%
\ auto\isanewline
\ \ \ \ \ \ \isacommand{next}\isamarkupfalse%
\isanewline
\ \ \ \ \ \ \ \ \isacommand{show}\isamarkupfalse%
\ {\isachardoublequoteopen}dim{\isacharunderscore}{\kern0pt}col\ {\isacharparenleft}{\kern0pt}Tensor{\isachardot}{\kern0pt}mat{\isacharunderscore}{\kern0pt}of{\isacharunderscore}{\kern0pt}cols{\isacharunderscore}{\kern0pt}list\ {\isadigit{2}}\ {\isacharparenleft}{\kern0pt}map\ {\isacharparenleft}{\kern0pt}map\ complex{\isacharunderscore}{\kern0pt}of{\isacharunderscore}{\kern0pt}real{\isacharparenright}{\kern0pt}\isanewline
\ \ \ \ \ \ \ \ \ \ \ \ \ \ {\isacharbrackleft}{\kern0pt}{\isacharbrackleft}{\kern0pt}{\isadigit{1}}\ {\isacharslash}{\kern0pt}\ sqrt\ {\isadigit{2}}{\isacharcomma}{\kern0pt}{\isacharminus}{\kern0pt}\ {\isadigit{1}}\ {\isacharslash}{\kern0pt}\ sqrt\ {\isadigit{2}}{\isacharbrackright}{\kern0pt}{\isacharbrackright}{\kern0pt}{\isacharparenright}{\kern0pt}{\isacharparenright}{\kern0pt}\ {\isacharequal}{\kern0pt}\ dim{\isacharunderscore}{\kern0pt}col\ {\isacharparenleft}{\kern0pt}complex{\isacharunderscore}{\kern0pt}of{\isacharunderscore}{\kern0pt}real\ {\isacharparenleft}{\kern0pt}{\isadigit{1}}\ {\isacharslash}{\kern0pt}\ sqrt\ {\isadigit{2}}{\isacharparenright}{\kern0pt}\ {\isasymcdot}\isactrlsub m\isanewline
\ \ \ \ \ \ \ \ \ \ \ \ \ \ {\isacharparenleft}{\kern0pt}\ {\isacharbar}{\kern0pt}Deutsch{\isachardot}{\kern0pt}zero{\isasymrangle}\ {\isacharplus}{\kern0pt}\ exp\ {\isacharparenleft}{\kern0pt}{\isadigit{2}}\ {\isacharasterisk}{\kern0pt}\ {\isasymi}\ {\isacharasterisk}{\kern0pt}\ complex{\isacharunderscore}{\kern0pt}of{\isacharunderscore}{\kern0pt}real\ pi\ {\isacharasterisk}{\kern0pt}\ complex{\isacharunderscore}{\kern0pt}of{\isacharunderscore}{\kern0pt}nat\ {\isadigit{1}}\ {\isacharslash}{\kern0pt}{\isadigit{2}}{\isacharparenright}{\kern0pt}\ {\isasymcdot}\isactrlsub m\isanewline
\ \ \ \ \ \ \ \ \ \ \ \ \ \ \ \ {\isacharbar}{\kern0pt}Deutsch{\isachardot}{\kern0pt}one{\isasymrangle}{\isacharparenright}{\kern0pt}{\isacharparenright}{\kern0pt}{\isachardoublequoteclose}\isanewline
\ \ \ \ \ \ \ \ \ \ \isacommand{using}\isamarkupfalse%
\ mat{\isacharunderscore}{\kern0pt}of{\isacharunderscore}{\kern0pt}cols{\isacharunderscore}{\kern0pt}list{\isacharunderscore}{\kern0pt}def\ index{\isacharunderscore}{\kern0pt}mat{\isacharunderscore}{\kern0pt}of{\isacharunderscore}{\kern0pt}cols{\isacharunderscore}{\kern0pt}list\ smult{\isacharunderscore}{\kern0pt}carrier{\isacharunderscore}{\kern0pt}mat\ ket{\isacharunderscore}{\kern0pt}vec{\isacharunderscore}{\kern0pt}def\ \isacommand{by}\isamarkupfalse%
\ auto\isanewline
\ \ \ \ \ \ \isacommand{qed}\isamarkupfalse%
\isanewline
\ \ \ \ \ \ \isacommand{finally}\isamarkupfalse%
\ \isacommand{show}\isamarkupfalse%
\ {\isacharquery}{\kern0pt}thesis\ \isacommand{using}\isamarkupfalse%
\ jd{\isadigit{1}}\ \isacommand{by}\isamarkupfalse%
\ simp\isanewline
\ \ \ \ \isacommand{qed}\isamarkupfalse%
\isanewline
\ \ \ \ \isacommand{hence}\isamarkupfalse%
\ {\isachardoublequoteopen}{\isacharparenleft}{\kern0pt}H\ {\isacharasterisk}{\kern0pt}\ {\isacharbar}{\kern0pt}state{\isacharunderscore}{\kern0pt}basis\ {\isadigit{1}}\ jd{\isasymrangle}{\isacharparenright}{\kern0pt}\ {\isasymOtimes}\ {\isacharbar}{\kern0pt}state{\isacharunderscore}{\kern0pt}basis\ n\ jm{\isasymrangle}\ {\isacharequal}{\kern0pt}\ \isanewline
\ \ \ \ \ \ \ \ \ \ {\isacharparenleft}{\kern0pt}{\isadigit{1}}{\isacharslash}{\kern0pt}sqrt\ {\isadigit{2}}\ {\isasymcdot}\isactrlsub m\ {\isacharparenleft}{\kern0pt}{\isacharparenleft}{\kern0pt}\ {\isacharbar}{\kern0pt}zero{\isasymrangle}\ {\isacharplus}{\kern0pt}\ exp{\isacharparenleft}{\kern0pt}{\isadigit{2}}{\isacharasterisk}{\kern0pt}{\isasymi}{\isacharasterisk}{\kern0pt}pi{\isacharasterisk}{\kern0pt}{\isacharparenleft}{\kern0pt}complex{\isacharunderscore}{\kern0pt}of{\isacharunderscore}{\kern0pt}nat\ jd{\isacharparenright}{\kern0pt}{\isacharslash}{\kern0pt}{\isadigit{2}}{\isacharparenright}{\kern0pt}\ {\isasymcdot}\isactrlsub m\ {\isacharbar}{\kern0pt}one{\isasymrangle}{\isacharparenright}{\kern0pt}{\isacharparenright}{\kern0pt}{\isacharparenright}{\kern0pt}\ {\isasymOtimes}\ {\isacharbar}{\kern0pt}state{\isacharunderscore}{\kern0pt}basis\ n\ jm{\isasymrangle}{\isachardoublequoteclose}\isanewline
\ \ \ \ \ \ \isacommand{by}\isamarkupfalse%
\ simp\isanewline
\ \ \ \ \isacommand{thus}\isamarkupfalse%
\ {\isacharquery}{\kern0pt}thesis\ \isacommand{using}\isamarkupfalse%
\ {\isadigit{0}}\ \isacommand{by}\isamarkupfalse%
\ presburger\isanewline
\ \ \isacommand{qed}\isamarkupfalse%
\isanewline
\ \ \isacommand{finally}\isamarkupfalse%
\ \isacommand{show}\isamarkupfalse%
\ {\isacharquery}{\kern0pt}thesis\ \isacommand{using}\isamarkupfalse%
\ jm{\isacharunderscore}{\kern0pt}def\ jd{\isacharunderscore}{\kern0pt}def\ \isacommand{by}\isamarkupfalse%
\ auto\isanewline
\isacommand{qed}\isamarkupfalse%
%
\endisatagproof
{\isafoldproof}%
%
\isadelimproof
%
\endisadelimproof
%
\begin{isamarkuptext}%
Action of the R gate in the circuit%
\end{isamarkuptext}\isamarkuptrue%
\isacommand{lemma}\isamarkupfalse%
\ R{\isacharunderscore}{\kern0pt}action{\isacharcolon}{\kern0pt}\isanewline
\ \ \isakeyword{assumes}\ {\isachardoublequoteopen}j\ {\isacharless}{\kern0pt}\ {\isadigit{2}}\ {\isacharcircum}{\kern0pt}\ Suc\ n{\isachardoublequoteclose}\ \isakeyword{and}\ {\isachardoublequoteopen}j\ mod\ {\isadigit{2}}\ {\isacharequal}{\kern0pt}\ {\isadigit{1}}{\isachardoublequoteclose}\isanewline
\ \ \isakeyword{shows}\ {\isachardoublequoteopen}{\isacharparenleft}{\kern0pt}R\ {\isacharparenleft}{\kern0pt}Suc\ n{\isacharparenright}{\kern0pt}{\isacharparenright}{\kern0pt}\ {\isacharasterisk}{\kern0pt}\ {\isacharparenleft}{\kern0pt}\ {\isacharbar}{\kern0pt}zero{\isasymrangle}\ {\isacharplus}{\kern0pt}\ exp\ {\isacharparenleft}{\kern0pt}{\isadigit{2}}{\isacharasterisk}{\kern0pt}{\isasymi}{\isacharasterisk}{\kern0pt}pi{\isacharasterisk}{\kern0pt}complex{\isacharunderscore}{\kern0pt}of{\isacharunderscore}{\kern0pt}nat\ {\isacharparenleft}{\kern0pt}j\ div\ {\isadigit{2}}{\isacharparenright}{\kern0pt}\ {\isacharslash}{\kern0pt}\ {\isadigit{2}}{\isacharcircum}{\kern0pt}n{\isacharparenright}{\kern0pt}\ {\isasymcdot}\isactrlsub m\ {\isacharbar}{\kern0pt}one{\isasymrangle}{\isacharparenright}{\kern0pt}\ {\isacharequal}{\kern0pt}\isanewline
\ \ \ \ \ \ \ \ \ {\isacharbar}{\kern0pt}zero{\isasymrangle}\ {\isacharplus}{\kern0pt}\ exp\ {\isacharparenleft}{\kern0pt}{\isadigit{2}}{\isacharasterisk}{\kern0pt}{\isasymi}{\isacharasterisk}{\kern0pt}pi{\isacharasterisk}{\kern0pt}complex{\isacharunderscore}{\kern0pt}of{\isacharunderscore}{\kern0pt}nat\ j\ {\isacharslash}{\kern0pt}\ {\isadigit{2}}{\isacharcircum}{\kern0pt}{\isacharparenleft}{\kern0pt}Suc\ n{\isacharparenright}{\kern0pt}{\isacharparenright}{\kern0pt}\ {\isasymcdot}\isactrlsub m\ {\isacharbar}{\kern0pt}one{\isasymrangle}{\isachardoublequoteclose}\isanewline
%
\isadelimproof
%
\endisadelimproof
%
\isatagproof
\isacommand{proof}\isamarkupfalse%
\ \isanewline
\ \ \isacommand{fix}\isamarkupfalse%
\ i\ ja{\isacharcolon}{\kern0pt}{\isacharcolon}{\kern0pt}nat\isanewline
\ \ \isacommand{assume}\isamarkupfalse%
\ {\isachardoublequoteopen}i\ {\isacharless}{\kern0pt}\ dim{\isacharunderscore}{\kern0pt}row\ {\isacharparenleft}{\kern0pt}\ {\isacharbar}{\kern0pt}zero{\isasymrangle}\ {\isacharplus}{\kern0pt}\ exp\ {\isacharparenleft}{\kern0pt}{\isadigit{2}}{\isacharasterisk}{\kern0pt}{\isasymi}{\isacharasterisk}{\kern0pt}pi{\isacharasterisk}{\kern0pt}complex{\isacharunderscore}{\kern0pt}of{\isacharunderscore}{\kern0pt}nat\ j\ {\isacharslash}{\kern0pt}\ {\isadigit{2}}{\isacharcircum}{\kern0pt}{\isacharparenleft}{\kern0pt}Suc\ n{\isacharparenright}{\kern0pt}{\isacharparenright}{\kern0pt}\ {\isasymcdot}\isactrlsub m\ {\isacharbar}{\kern0pt}one{\isasymrangle}{\isacharparenright}{\kern0pt}{\isachardoublequoteclose}\isanewline
\ \ \isacommand{hence}\isamarkupfalse%
\ il{\isadigit{2}}{\isacharcolon}{\kern0pt}{\isachardoublequoteopen}i\ {\isacharless}{\kern0pt}\ {\isadigit{2}}{\isachardoublequoteclose}\ \isacommand{by}\isamarkupfalse%
\ {\isacharparenleft}{\kern0pt}simp\ add{\isacharcolon}{\kern0pt}\ ket{\isacharunderscore}{\kern0pt}vec{\isacharunderscore}{\kern0pt}def{\isacharparenright}{\kern0pt}\isanewline
\ \ \isacommand{assume}\isamarkupfalse%
\ {\isachardoublequoteopen}ja\ {\isacharless}{\kern0pt}\ dim{\isacharunderscore}{\kern0pt}col\ {\isacharparenleft}{\kern0pt}\ {\isacharbar}{\kern0pt}zero{\isasymrangle}\ {\isacharplus}{\kern0pt}\ exp\ {\isacharparenleft}{\kern0pt}{\isadigit{2}}{\isacharasterisk}{\kern0pt}{\isasymi}{\isacharasterisk}{\kern0pt}pi{\isacharasterisk}{\kern0pt}complex{\isacharunderscore}{\kern0pt}of{\isacharunderscore}{\kern0pt}nat\ j\ {\isacharslash}{\kern0pt}\ {\isadigit{2}}{\isacharcircum}{\kern0pt}{\isacharparenleft}{\kern0pt}Suc\ n{\isacharparenright}{\kern0pt}{\isacharparenright}{\kern0pt}\ {\isasymcdot}\isactrlsub m\ {\isacharbar}{\kern0pt}one{\isasymrangle}{\isacharparenright}{\kern0pt}{\isachardoublequoteclose}\isanewline
\ \ \isacommand{hence}\isamarkupfalse%
\ ja{\isadigit{0}}{\isacharcolon}{\kern0pt}{\isachardoublequoteopen}ja\ {\isacharequal}{\kern0pt}\ {\isadigit{0}}{\isachardoublequoteclose}\ \isacommand{by}\isamarkupfalse%
\ {\isacharparenleft}{\kern0pt}simp\ add{\isacharcolon}{\kern0pt}\ ket{\isacharunderscore}{\kern0pt}vec{\isacharunderscore}{\kern0pt}def{\isacharparenright}{\kern0pt}\isanewline
\ \ \isacommand{have}\isamarkupfalse%
\ {\isachardoublequoteopen}{\isacharparenleft}{\kern0pt}R\ {\isacharparenleft}{\kern0pt}Suc\ n{\isacharparenright}{\kern0pt}{\isacharparenright}{\kern0pt}\ {\isacharasterisk}{\kern0pt}\ {\isacharparenleft}{\kern0pt}\ {\isacharbar}{\kern0pt}zero{\isasymrangle}\ {\isacharplus}{\kern0pt}\ exp\ {\isacharparenleft}{\kern0pt}{\isadigit{2}}{\isacharasterisk}{\kern0pt}{\isasymi}{\isacharasterisk}{\kern0pt}pi{\isacharasterisk}{\kern0pt}complex{\isacharunderscore}{\kern0pt}of{\isacharunderscore}{\kern0pt}nat\ {\isacharparenleft}{\kern0pt}j\ div\ {\isadigit{2}}{\isacharparenright}{\kern0pt}\ {\isacharslash}{\kern0pt}\ {\isadigit{2}}{\isacharcircum}{\kern0pt}n{\isacharparenright}{\kern0pt}\ {\isasymcdot}\isactrlsub m\ {\isacharbar}{\kern0pt}one{\isasymrangle}{\isacharparenright}{\kern0pt}\ {\isacharequal}{\kern0pt}\isanewline
\ \ \ \ \ \ \ \ {\isacharparenleft}{\kern0pt}mat{\isacharunderscore}{\kern0pt}of{\isacharunderscore}{\kern0pt}cols{\isacharunderscore}{\kern0pt}list\ {\isadigit{2}}\ {\isacharbrackleft}{\kern0pt}{\isacharbrackleft}{\kern0pt}{\isadigit{1}}{\isacharcomma}{\kern0pt}\ {\isadigit{0}}{\isacharbrackright}{\kern0pt}{\isacharcomma}{\kern0pt}{\isacharbrackleft}{\kern0pt}{\isadigit{0}}{\isacharcomma}{\kern0pt}\ exp{\isacharparenleft}{\kern0pt}{\isadigit{2}}{\isacharasterisk}{\kern0pt}pi{\isacharasterisk}{\kern0pt}{\isasymi}{\isacharslash}{\kern0pt}{\isadigit{2}}{\isacharcircum}{\kern0pt}{\isacharparenleft}{\kern0pt}Suc\ n{\isacharparenright}{\kern0pt}{\isacharparenright}{\kern0pt}{\isacharbrackright}{\kern0pt}{\isacharbrackright}{\kern0pt}{\isacharparenright}{\kern0pt}\ {\isacharasterisk}{\kern0pt}\isanewline
\ \ \ \ \ \ \ \ {\isacharparenleft}{\kern0pt}\ {\isacharbar}{\kern0pt}zero{\isasymrangle}\ {\isacharplus}{\kern0pt}\ exp\ {\isacharparenleft}{\kern0pt}{\isadigit{2}}{\isacharasterisk}{\kern0pt}{\isasymi}{\isacharasterisk}{\kern0pt}pi{\isacharasterisk}{\kern0pt}complex{\isacharunderscore}{\kern0pt}of{\isacharunderscore}{\kern0pt}nat\ {\isacharparenleft}{\kern0pt}j\ div\ {\isadigit{2}}{\isacharparenright}{\kern0pt}\ {\isacharslash}{\kern0pt}\ {\isadigit{2}}{\isacharcircum}{\kern0pt}n{\isacharparenright}{\kern0pt}\ {\isasymcdot}\isactrlsub m\ {\isacharbar}{\kern0pt}one{\isasymrangle}{\isacharparenright}{\kern0pt}{\isachardoublequoteclose}\isanewline
\ \ \ \ \isacommand{using}\isamarkupfalse%
\ R{\isacharunderscore}{\kern0pt}def\ \isacommand{by}\isamarkupfalse%
\ simp\isanewline
\ \ \isacommand{also}\isamarkupfalse%
\ \isacommand{have}\isamarkupfalse%
\ {\isachardoublequoteopen}{\isasymdots}\ {\isacharequal}{\kern0pt}\ {\isacharparenleft}{\kern0pt}mat{\isacharunderscore}{\kern0pt}of{\isacharunderscore}{\kern0pt}cols{\isacharunderscore}{\kern0pt}list\ {\isadigit{2}}\ {\isacharbrackleft}{\kern0pt}{\isacharbrackleft}{\kern0pt}{\isadigit{1}}{\isacharcomma}{\kern0pt}\ {\isadigit{0}}{\isacharbrackright}{\kern0pt}{\isacharcomma}{\kern0pt}{\isacharbrackleft}{\kern0pt}{\isadigit{0}}{\isacharcomma}{\kern0pt}\ exp{\isacharparenleft}{\kern0pt}{\isadigit{2}}{\isacharasterisk}{\kern0pt}pi{\isacharasterisk}{\kern0pt}{\isasymi}{\isacharslash}{\kern0pt}{\isadigit{2}}{\isacharcircum}{\kern0pt}{\isacharparenleft}{\kern0pt}Suc\ n{\isacharparenright}{\kern0pt}{\isacharparenright}{\kern0pt}{\isacharbrackright}{\kern0pt}{\isacharbrackright}{\kern0pt}{\isacharparenright}{\kern0pt}\ {\isacharasterisk}{\kern0pt}\isanewline
\ \ \ \ \ \ \ \ \ \ \ \ \ \ \ \ \ \ {\isacharparenleft}{\kern0pt}mat{\isacharunderscore}{\kern0pt}of{\isacharunderscore}{\kern0pt}cols{\isacharunderscore}{\kern0pt}list\ {\isadigit{2}}\ {\isacharbrackleft}{\kern0pt}{\isacharbrackleft}{\kern0pt}{\isadigit{1}}{\isacharcomma}{\kern0pt}{\isadigit{0}}{\isacharbrackright}{\kern0pt}{\isacharbrackright}{\kern0pt}\ {\isacharplus}{\kern0pt}\ \isanewline
\ \ \ \ \ \ \ \ \ \ \ \ \ \ \ \ \ \ \ exp\ {\isacharparenleft}{\kern0pt}{\isadigit{2}}{\isacharasterisk}{\kern0pt}{\isasymi}{\isacharasterisk}{\kern0pt}pi{\isacharasterisk}{\kern0pt}complex{\isacharunderscore}{\kern0pt}of{\isacharunderscore}{\kern0pt}nat\ {\isacharparenleft}{\kern0pt}j\ div\ {\isadigit{2}}{\isacharparenright}{\kern0pt}\ {\isacharslash}{\kern0pt}\ {\isadigit{2}}{\isacharcircum}{\kern0pt}n{\isacharparenright}{\kern0pt}\ {\isasymcdot}\isactrlsub m\ mat{\isacharunderscore}{\kern0pt}of{\isacharunderscore}{\kern0pt}cols{\isacharunderscore}{\kern0pt}list\ {\isadigit{2}}\ {\isacharbrackleft}{\kern0pt}{\isacharbrackleft}{\kern0pt}{\isadigit{0}}{\isacharcomma}{\kern0pt}{\isadigit{1}}{\isacharbrackright}{\kern0pt}{\isacharbrackright}{\kern0pt}{\isacharparenright}{\kern0pt}{\isachardoublequoteclose}\isanewline
\ \ \ \ \isacommand{using}\isamarkupfalse%
\ ket{\isacharunderscore}{\kern0pt}one{\isacharunderscore}{\kern0pt}to{\isacharunderscore}{\kern0pt}mat{\isacharunderscore}{\kern0pt}of{\isacharunderscore}{\kern0pt}cols{\isacharunderscore}{\kern0pt}list\ ket{\isacharunderscore}{\kern0pt}zero{\isacharunderscore}{\kern0pt}to{\isacharunderscore}{\kern0pt}mat{\isacharunderscore}{\kern0pt}of{\isacharunderscore}{\kern0pt}cols{\isacharunderscore}{\kern0pt}list\ \isacommand{by}\isamarkupfalse%
\ presburger\isanewline
\ \ \isacommand{also}\isamarkupfalse%
\ \isacommand{have}\isamarkupfalse%
\ {\isachardoublequoteopen}{\isasymdots}\ {\isacharequal}{\kern0pt}\ {\isacharparenleft}{\kern0pt}mat{\isacharunderscore}{\kern0pt}of{\isacharunderscore}{\kern0pt}cols{\isacharunderscore}{\kern0pt}list\ {\isadigit{2}}\ {\isacharbrackleft}{\kern0pt}{\isacharbrackleft}{\kern0pt}{\isadigit{1}}{\isacharcomma}{\kern0pt}\ {\isadigit{0}}{\isacharbrackright}{\kern0pt}{\isacharcomma}{\kern0pt}{\isacharbrackleft}{\kern0pt}{\isadigit{0}}{\isacharcomma}{\kern0pt}\ exp{\isacharparenleft}{\kern0pt}{\isadigit{2}}{\isacharasterisk}{\kern0pt}pi{\isacharasterisk}{\kern0pt}{\isasymi}{\isacharslash}{\kern0pt}{\isadigit{2}}{\isacharcircum}{\kern0pt}{\isacharparenleft}{\kern0pt}Suc\ n{\isacharparenright}{\kern0pt}{\isacharparenright}{\kern0pt}{\isacharbrackright}{\kern0pt}{\isacharbrackright}{\kern0pt}{\isacharparenright}{\kern0pt}\ {\isacharasterisk}{\kern0pt}\isanewline
\ \ \ \ \ \ \ \ \ \ \ \ \ \ \ \ \ \ {\isacharparenleft}{\kern0pt}mat{\isacharunderscore}{\kern0pt}of{\isacharunderscore}{\kern0pt}cols{\isacharunderscore}{\kern0pt}list\ {\isadigit{2}}\ {\isacharbrackleft}{\kern0pt}{\isacharbrackleft}{\kern0pt}{\isadigit{1}}{\isacharcomma}{\kern0pt}{\isadigit{0}}{\isacharbrackright}{\kern0pt}{\isacharbrackright}{\kern0pt}\ {\isacharplus}{\kern0pt}\ \isanewline
\ \ \ \ \ \ \ \ \ \ \ \ \ \ \ \ \ \ \ mat{\isacharunderscore}{\kern0pt}of{\isacharunderscore}{\kern0pt}cols{\isacharunderscore}{\kern0pt}list\ {\isadigit{2}}\ {\isacharbrackleft}{\kern0pt}{\isacharbrackleft}{\kern0pt}{\isadigit{0}}{\isacharcomma}{\kern0pt}exp\ {\isacharparenleft}{\kern0pt}{\isadigit{2}}{\isacharasterisk}{\kern0pt}{\isasymi}{\isacharasterisk}{\kern0pt}pi{\isacharasterisk}{\kern0pt}complex{\isacharunderscore}{\kern0pt}of{\isacharunderscore}{\kern0pt}nat\ {\isacharparenleft}{\kern0pt}j\ div\ {\isadigit{2}}{\isacharparenright}{\kern0pt}\ {\isacharslash}{\kern0pt}\ {\isadigit{2}}{\isacharcircum}{\kern0pt}n{\isacharparenright}{\kern0pt}{\isacharbrackright}{\kern0pt}{\isacharbrackright}{\kern0pt}{\isacharparenright}{\kern0pt}{\isachardoublequoteclose}\isanewline
\ \ \isacommand{proof}\isamarkupfalse%
\ {\isacharminus}{\kern0pt}\isanewline
\ \ \ \ \isacommand{have}\isamarkupfalse%
\ {\isachardoublequoteopen}exp\ {\isacharparenleft}{\kern0pt}{\isadigit{2}}{\isacharasterisk}{\kern0pt}{\isasymi}{\isacharasterisk}{\kern0pt}pi{\isacharasterisk}{\kern0pt}complex{\isacharunderscore}{\kern0pt}of{\isacharunderscore}{\kern0pt}nat\ {\isacharparenleft}{\kern0pt}j\ div\ {\isadigit{2}}{\isacharparenright}{\kern0pt}\ {\isacharslash}{\kern0pt}\ {\isadigit{2}}{\isacharcircum}{\kern0pt}n{\isacharparenright}{\kern0pt}\ {\isasymcdot}\isactrlsub m\ mat{\isacharunderscore}{\kern0pt}of{\isacharunderscore}{\kern0pt}cols{\isacharunderscore}{\kern0pt}list\ {\isadigit{2}}\ {\isacharbrackleft}{\kern0pt}{\isacharbrackleft}{\kern0pt}{\isadigit{0}}{\isacharcomma}{\kern0pt}{\isadigit{1}}{\isacharbrackright}{\kern0pt}{\isacharbrackright}{\kern0pt}\ {\isacharequal}{\kern0pt}\isanewline
\ \ \ \ \ \ \ \ \ \ mat{\isacharunderscore}{\kern0pt}of{\isacharunderscore}{\kern0pt}cols{\isacharunderscore}{\kern0pt}list\ {\isadigit{2}}\ {\isacharbrackleft}{\kern0pt}{\isacharbrackleft}{\kern0pt}{\isadigit{0}}{\isacharcomma}{\kern0pt}exp\ {\isacharparenleft}{\kern0pt}{\isadigit{2}}{\isacharasterisk}{\kern0pt}{\isasymi}{\isacharasterisk}{\kern0pt}pi{\isacharasterisk}{\kern0pt}complex{\isacharunderscore}{\kern0pt}of{\isacharunderscore}{\kern0pt}nat\ {\isacharparenleft}{\kern0pt}j\ div\ {\isadigit{2}}{\isacharparenright}{\kern0pt}\ {\isacharslash}{\kern0pt}\ {\isadigit{2}}{\isacharcircum}{\kern0pt}n{\isacharparenright}{\kern0pt}{\isacharbrackright}{\kern0pt}{\isacharbrackright}{\kern0pt}{\isachardoublequoteclose}\isanewline
\ \ \ \ \isacommand{proof}\isamarkupfalse%
\ \isanewline
\ \ \ \ \ \ \isacommand{fix}\isamarkupfalse%
\ a\ b{\isacharcolon}{\kern0pt}{\isacharcolon}{\kern0pt}nat\isanewline
\ \ \ \ \ \ \isacommand{assume}\isamarkupfalse%
\ {\isachardoublequoteopen}a\ {\isacharless}{\kern0pt}\ dim{\isacharunderscore}{\kern0pt}row\ {\isacharparenleft}{\kern0pt}mat{\isacharunderscore}{\kern0pt}of{\isacharunderscore}{\kern0pt}cols{\isacharunderscore}{\kern0pt}list\ {\isadigit{2}}\ {\isacharbrackleft}{\kern0pt}{\isacharbrackleft}{\kern0pt}{\isadigit{0}}{\isacharcomma}{\kern0pt}exp\ {\isacharparenleft}{\kern0pt}{\isadigit{2}}{\isacharasterisk}{\kern0pt}{\isasymi}{\isacharasterisk}{\kern0pt}pi{\isacharasterisk}{\kern0pt}complex{\isacharunderscore}{\kern0pt}of{\isacharunderscore}{\kern0pt}nat\ {\isacharparenleft}{\kern0pt}j\ div\ {\isadigit{2}}{\isacharparenright}{\kern0pt}\ {\isacharslash}{\kern0pt}\ {\isadigit{2}}{\isacharcircum}{\kern0pt}n{\isacharparenright}{\kern0pt}{\isacharbrackright}{\kern0pt}{\isacharbrackright}{\kern0pt}{\isacharparenright}{\kern0pt}{\isachardoublequoteclose}\isanewline
\ \ \ \ \ \ \isacommand{hence}\isamarkupfalse%
\ a{\isadigit{2}}{\isacharcolon}{\kern0pt}{\isachardoublequoteopen}a\ {\isacharless}{\kern0pt}\ {\isadigit{2}}{\isachardoublequoteclose}\ \isacommand{by}\isamarkupfalse%
\ {\isacharparenleft}{\kern0pt}simp\ add{\isacharcolon}{\kern0pt}\ Tensor{\isachardot}{\kern0pt}mat{\isacharunderscore}{\kern0pt}of{\isacharunderscore}{\kern0pt}cols{\isacharunderscore}{\kern0pt}list{\isacharunderscore}{\kern0pt}def{\isacharparenright}{\kern0pt}\isanewline
\ \ \ \ \ \ \isacommand{assume}\isamarkupfalse%
\ {\isachardoublequoteopen}b\ {\isacharless}{\kern0pt}\ dim{\isacharunderscore}{\kern0pt}col\ {\isacharparenleft}{\kern0pt}mat{\isacharunderscore}{\kern0pt}of{\isacharunderscore}{\kern0pt}cols{\isacharunderscore}{\kern0pt}list\ {\isadigit{2}}\ {\isacharbrackleft}{\kern0pt}{\isacharbrackleft}{\kern0pt}{\isadigit{0}}{\isacharcomma}{\kern0pt}exp\ {\isacharparenleft}{\kern0pt}{\isadigit{2}}{\isacharasterisk}{\kern0pt}{\isasymi}{\isacharasterisk}{\kern0pt}pi{\isacharasterisk}{\kern0pt}complex{\isacharunderscore}{\kern0pt}of{\isacharunderscore}{\kern0pt}nat\ {\isacharparenleft}{\kern0pt}j\ div\ {\isadigit{2}}{\isacharparenright}{\kern0pt}\ {\isacharslash}{\kern0pt}\ {\isadigit{2}}{\isacharcircum}{\kern0pt}n{\isacharparenright}{\kern0pt}{\isacharbrackright}{\kern0pt}{\isacharbrackright}{\kern0pt}{\isacharparenright}{\kern0pt}{\isachardoublequoteclose}\isanewline
\ \ \ \ \ \ \isacommand{hence}\isamarkupfalse%
\ b{\isadigit{0}}{\isacharcolon}{\kern0pt}{\isachardoublequoteopen}b\ {\isacharequal}{\kern0pt}\ {\isadigit{0}}{\isachardoublequoteclose}\ \isanewline
\ \ \ \ \ \ \ \ \isacommand{by}\isamarkupfalse%
\ {\isacharparenleft}{\kern0pt}metis\ One{\isacharunderscore}{\kern0pt}nat{\isacharunderscore}{\kern0pt}def\ Suc{\isacharunderscore}{\kern0pt}eq{\isacharunderscore}{\kern0pt}plus{\isadigit{1}}\ Tensor{\isachardot}{\kern0pt}mat{\isacharunderscore}{\kern0pt}of{\isacharunderscore}{\kern0pt}cols{\isacharunderscore}{\kern0pt}list{\isacharunderscore}{\kern0pt}def\ dim{\isacharunderscore}{\kern0pt}col{\isacharunderscore}{\kern0pt}mat{\isacharparenleft}{\kern0pt}{\isadigit{1}}{\isacharparenright}{\kern0pt}\ less{\isacharunderscore}{\kern0pt}Suc{\isadigit{0}}\ \isanewline
\ \ \ \ \ \ \ \ \ \ \ \ list{\isachardot}{\kern0pt}size{\isacharparenleft}{\kern0pt}{\isadigit{3}}{\isacharparenright}{\kern0pt}\ list{\isachardot}{\kern0pt}size{\isacharparenleft}{\kern0pt}{\isadigit{4}}{\isacharparenright}{\kern0pt}{\isacharparenright}{\kern0pt}\isanewline
\ \ \ \ \ \ \isacommand{have}\isamarkupfalse%
\ {\isachardoublequoteopen}{\isacharparenleft}{\kern0pt}exp\ {\isacharparenleft}{\kern0pt}{\isadigit{2}}{\isacharasterisk}{\kern0pt}{\isasymi}{\isacharasterisk}{\kern0pt}pi{\isacharasterisk}{\kern0pt}complex{\isacharunderscore}{\kern0pt}of{\isacharunderscore}{\kern0pt}nat\ {\isacharparenleft}{\kern0pt}j\ div\ {\isadigit{2}}{\isacharparenright}{\kern0pt}\ {\isacharslash}{\kern0pt}\ {\isadigit{2}}{\isacharcircum}{\kern0pt}n{\isacharparenright}{\kern0pt}\ {\isasymcdot}\isactrlsub m\ mat{\isacharunderscore}{\kern0pt}of{\isacharunderscore}{\kern0pt}cols{\isacharunderscore}{\kern0pt}list\ {\isadigit{2}}\ {\isacharbrackleft}{\kern0pt}{\isacharbrackleft}{\kern0pt}{\isadigit{0}}{\isacharcomma}{\kern0pt}{\isadigit{1}}{\isacharbrackright}{\kern0pt}{\isacharbrackright}{\kern0pt}{\isacharparenright}{\kern0pt}\ {\isachardollar}{\kern0pt}{\isachardollar}{\kern0pt}\ {\isacharparenleft}{\kern0pt}a{\isacharcomma}{\kern0pt}{\isadigit{0}}{\isacharparenright}{\kern0pt}\ {\isacharequal}{\kern0pt}\isanewline
\ \ \ \ \ \ \ \ \ \ \ \ exp\ {\isacharparenleft}{\kern0pt}{\isadigit{2}}{\isacharasterisk}{\kern0pt}{\isasymi}{\isacharasterisk}{\kern0pt}pi{\isacharasterisk}{\kern0pt}complex{\isacharunderscore}{\kern0pt}of{\isacharunderscore}{\kern0pt}nat\ {\isacharparenleft}{\kern0pt}j\ div\ {\isadigit{2}}{\isacharparenright}{\kern0pt}\ {\isacharslash}{\kern0pt}\ {\isadigit{2}}{\isacharcircum}{\kern0pt}n{\isacharparenright}{\kern0pt}\ {\isacharasterisk}{\kern0pt}\ {\isacharparenleft}{\kern0pt}mat{\isacharunderscore}{\kern0pt}of{\isacharunderscore}{\kern0pt}cols{\isacharunderscore}{\kern0pt}list\ {\isadigit{2}}\ {\isacharbrackleft}{\kern0pt}{\isacharbrackleft}{\kern0pt}{\isadigit{0}}{\isacharcomma}{\kern0pt}{\isadigit{1}}{\isacharbrackright}{\kern0pt}{\isacharbrackright}{\kern0pt}\ {\isachardollar}{\kern0pt}{\isachardollar}{\kern0pt}\ {\isacharparenleft}{\kern0pt}a{\isacharcomma}{\kern0pt}{\isadigit{0}}{\isacharparenright}{\kern0pt}{\isacharparenright}{\kern0pt}{\isachardoublequoteclose}\isanewline
\ \ \ \ \ \ \ \ \isacommand{using}\isamarkupfalse%
\ index{\isacharunderscore}{\kern0pt}smult{\isacharunderscore}{\kern0pt}mat\ a{\isadigit{2}}\ ket{\isacharunderscore}{\kern0pt}one{\isacharunderscore}{\kern0pt}is{\isacharunderscore}{\kern0pt}state\ ket{\isacharunderscore}{\kern0pt}one{\isacharunderscore}{\kern0pt}to{\isacharunderscore}{\kern0pt}mat{\isacharunderscore}{\kern0pt}of{\isacharunderscore}{\kern0pt}cols{\isacharunderscore}{\kern0pt}list\ state{\isacharunderscore}{\kern0pt}def\ \isacommand{by}\isamarkupfalse%
\ force\isanewline
\ \ \ \ \ \ \isacommand{also}\isamarkupfalse%
\ \isacommand{have}\isamarkupfalse%
\ {\isachardoublequoteopen}{\isasymdots}\ {\isacharequal}{\kern0pt}\ {\isacharparenleft}{\kern0pt}mat{\isacharunderscore}{\kern0pt}of{\isacharunderscore}{\kern0pt}cols{\isacharunderscore}{\kern0pt}list\ {\isadigit{2}}\ {\isacharbrackleft}{\kern0pt}{\isacharbrackleft}{\kern0pt}{\isadigit{0}}{\isacharcomma}{\kern0pt}exp\ {\isacharparenleft}{\kern0pt}{\isadigit{2}}{\isacharasterisk}{\kern0pt}{\isasymi}{\isacharasterisk}{\kern0pt}pi{\isacharasterisk}{\kern0pt}complex{\isacharunderscore}{\kern0pt}of{\isacharunderscore}{\kern0pt}nat\ {\isacharparenleft}{\kern0pt}j\ div\ {\isadigit{2}}{\isacharparenright}{\kern0pt}\ {\isacharslash}{\kern0pt}\ {\isadigit{2}}{\isacharcircum}{\kern0pt}n{\isacharparenright}{\kern0pt}{\isacharbrackright}{\kern0pt}{\isacharbrackright}{\kern0pt}{\isacharparenright}{\kern0pt}\ {\isachardollar}{\kern0pt}{\isachardollar}{\kern0pt}\ {\isacharparenleft}{\kern0pt}a{\isacharcomma}{\kern0pt}{\isadigit{0}}{\isacharparenright}{\kern0pt}{\isachardoublequoteclose}\isanewline
\ \ \ \ \ \ \isacommand{proof}\isamarkupfalse%
\ {\isacharparenleft}{\kern0pt}rule\ disjE{\isacharparenright}{\kern0pt}\isanewline
\ \ \ \ \ \ \ \ \isacommand{show}\isamarkupfalse%
\ {\isachardoublequoteopen}a\ {\isacharequal}{\kern0pt}\ {\isadigit{0}}\ {\isasymor}\ a\ {\isacharequal}{\kern0pt}\ {\isadigit{1}}{\isachardoublequoteclose}\ \isacommand{using}\isamarkupfalse%
\ a{\isadigit{2}}\ \isacommand{by}\isamarkupfalse%
\ auto\isanewline
\ \ \ \ \ \ \isacommand{next}\isamarkupfalse%
\isanewline
\ \ \ \ \ \ \ \ \isacommand{assume}\isamarkupfalse%
\ a{\isadigit{0}}{\isacharcolon}{\kern0pt}{\isachardoublequoteopen}a\ {\isacharequal}{\kern0pt}\ {\isadigit{0}}{\isachardoublequoteclose}\isanewline
\ \ \ \ \ \ \ \ \isacommand{have}\isamarkupfalse%
\ {\isachardoublequoteopen}exp\ {\isacharparenleft}{\kern0pt}{\isadigit{2}}{\isacharasterisk}{\kern0pt}{\isasymi}{\isacharasterisk}{\kern0pt}pi{\isacharasterisk}{\kern0pt}complex{\isacharunderscore}{\kern0pt}of{\isacharunderscore}{\kern0pt}nat\ {\isacharparenleft}{\kern0pt}j\ div\ {\isadigit{2}}{\isacharparenright}{\kern0pt}\ {\isacharslash}{\kern0pt}\ {\isadigit{2}}{\isacharcircum}{\kern0pt}n{\isacharparenright}{\kern0pt}\ {\isacharasterisk}{\kern0pt}\ {\isacharparenleft}{\kern0pt}mat{\isacharunderscore}{\kern0pt}of{\isacharunderscore}{\kern0pt}cols{\isacharunderscore}{\kern0pt}list\ {\isadigit{2}}\ {\isacharbrackleft}{\kern0pt}{\isacharbrackleft}{\kern0pt}{\isadigit{0}}{\isacharcomma}{\kern0pt}{\isadigit{1}}{\isacharbrackright}{\kern0pt}{\isacharbrackright}{\kern0pt}\ {\isachardollar}{\kern0pt}{\isachardollar}{\kern0pt}\ {\isacharparenleft}{\kern0pt}{\isadigit{0}}{\isacharcomma}{\kern0pt}{\isadigit{0}}{\isacharparenright}{\kern0pt}{\isacharparenright}{\kern0pt}\ {\isacharequal}{\kern0pt}\isanewline
\ \ \ \ \ \ \ \ \ \ \ \ \ \ exp\ {\isacharparenleft}{\kern0pt}{\isadigit{2}}{\isacharasterisk}{\kern0pt}{\isasymi}{\isacharasterisk}{\kern0pt}pi{\isacharasterisk}{\kern0pt}complex{\isacharunderscore}{\kern0pt}of{\isacharunderscore}{\kern0pt}nat\ {\isacharparenleft}{\kern0pt}j\ div\ {\isadigit{2}}{\isacharparenright}{\kern0pt}\ {\isacharslash}{\kern0pt}\ {\isadigit{2}}{\isacharcircum}{\kern0pt}n{\isacharparenright}{\kern0pt}\ {\isacharasterisk}{\kern0pt}\ {\isadigit{0}}{\isachardoublequoteclose}\isanewline
\ \ \ \ \ \ \ \ \ \ \isacommand{using}\isamarkupfalse%
\ index{\isacharunderscore}{\kern0pt}mat{\isacharunderscore}{\kern0pt}of{\isacharunderscore}{\kern0pt}cols{\isacharunderscore}{\kern0pt}list\ \isacommand{by}\isamarkupfalse%
\ auto\isanewline
\ \ \ \ \ \ \ \ \isacommand{thus}\isamarkupfalse%
\ {\isachardoublequoteopen}exp\ {\isacharparenleft}{\kern0pt}{\isadigit{2}}{\isacharasterisk}{\kern0pt}{\isasymi}{\isacharasterisk}{\kern0pt}pi{\isacharasterisk}{\kern0pt}complex{\isacharunderscore}{\kern0pt}of{\isacharunderscore}{\kern0pt}nat\ {\isacharparenleft}{\kern0pt}j\ div\ {\isadigit{2}}{\isacharparenright}{\kern0pt}\ {\isacharslash}{\kern0pt}\ {\isadigit{2}}{\isacharcircum}{\kern0pt}n{\isacharparenright}{\kern0pt}\ {\isacharasterisk}{\kern0pt}\ {\isacharparenleft}{\kern0pt}mat{\isacharunderscore}{\kern0pt}of{\isacharunderscore}{\kern0pt}cols{\isacharunderscore}{\kern0pt}list\ {\isadigit{2}}\ {\isacharbrackleft}{\kern0pt}{\isacharbrackleft}{\kern0pt}{\isadigit{0}}{\isacharcomma}{\kern0pt}{\isadigit{1}}{\isacharbrackright}{\kern0pt}{\isacharbrackright}{\kern0pt}\ {\isachardollar}{\kern0pt}{\isachardollar}{\kern0pt}\ {\isacharparenleft}{\kern0pt}a{\isacharcomma}{\kern0pt}{\isadigit{0}}{\isacharparenright}{\kern0pt}{\isacharparenright}{\kern0pt}\ {\isacharequal}{\kern0pt}\isanewline
\ \ \ \ \ \ \ \ \ \ \ \ \ \ {\isacharparenleft}{\kern0pt}mat{\isacharunderscore}{\kern0pt}of{\isacharunderscore}{\kern0pt}cols{\isacharunderscore}{\kern0pt}list\ {\isadigit{2}}\ {\isacharbrackleft}{\kern0pt}{\isacharbrackleft}{\kern0pt}{\isadigit{0}}{\isacharcomma}{\kern0pt}exp\ {\isacharparenleft}{\kern0pt}{\isadigit{2}}{\isacharasterisk}{\kern0pt}{\isasymi}{\isacharasterisk}{\kern0pt}pi{\isacharasterisk}{\kern0pt}complex{\isacharunderscore}{\kern0pt}of{\isacharunderscore}{\kern0pt}nat\ {\isacharparenleft}{\kern0pt}j\ div\ {\isadigit{2}}{\isacharparenright}{\kern0pt}\ {\isacharslash}{\kern0pt}\ {\isadigit{2}}{\isacharcircum}{\kern0pt}n{\isacharparenright}{\kern0pt}{\isacharbrackright}{\kern0pt}{\isacharbrackright}{\kern0pt}{\isacharparenright}{\kern0pt}\ {\isachardollar}{\kern0pt}{\isachardollar}{\kern0pt}\ {\isacharparenleft}{\kern0pt}a{\isacharcomma}{\kern0pt}{\isadigit{0}}{\isacharparenright}{\kern0pt}{\isachardoublequoteclose}\isanewline
\ \ \ \ \ \ \ \ \ \ \isacommand{using}\isamarkupfalse%
\ a{\isadigit{0}}\ \isacommand{by}\isamarkupfalse%
\ auto\isanewline
\ \ \ \ \ \ \isacommand{next}\isamarkupfalse%
\isanewline
\ \ \ \ \ \ \ \ \isacommand{assume}\isamarkupfalse%
\ a{\isadigit{1}}{\isacharcolon}{\kern0pt}{\isachardoublequoteopen}a\ {\isacharequal}{\kern0pt}\ {\isadigit{1}}{\isachardoublequoteclose}\isanewline
\ \ \ \ \ \ \ \ \isacommand{have}\isamarkupfalse%
\ {\isachardoublequoteopen}exp\ {\isacharparenleft}{\kern0pt}{\isadigit{2}}{\isacharasterisk}{\kern0pt}{\isasymi}{\isacharasterisk}{\kern0pt}pi{\isacharasterisk}{\kern0pt}complex{\isacharunderscore}{\kern0pt}of{\isacharunderscore}{\kern0pt}nat\ {\isacharparenleft}{\kern0pt}j\ div\ {\isadigit{2}}{\isacharparenright}{\kern0pt}\ {\isacharslash}{\kern0pt}\ {\isadigit{2}}{\isacharcircum}{\kern0pt}n{\isacharparenright}{\kern0pt}\ {\isacharasterisk}{\kern0pt}\ {\isacharparenleft}{\kern0pt}mat{\isacharunderscore}{\kern0pt}of{\isacharunderscore}{\kern0pt}cols{\isacharunderscore}{\kern0pt}list\ {\isadigit{2}}\ {\isacharbrackleft}{\kern0pt}{\isacharbrackleft}{\kern0pt}{\isadigit{0}}{\isacharcomma}{\kern0pt}{\isadigit{1}}{\isacharbrackright}{\kern0pt}{\isacharbrackright}{\kern0pt}\ {\isachardollar}{\kern0pt}{\isachardollar}{\kern0pt}\ {\isacharparenleft}{\kern0pt}{\isadigit{1}}{\isacharcomma}{\kern0pt}{\isadigit{0}}{\isacharparenright}{\kern0pt}{\isacharparenright}{\kern0pt}\ {\isacharequal}{\kern0pt}\isanewline
\ \ \ \ \ \ \ \ \ \ \ \ \ \ exp\ {\isacharparenleft}{\kern0pt}{\isadigit{2}}{\isacharasterisk}{\kern0pt}{\isasymi}{\isacharasterisk}{\kern0pt}pi{\isacharasterisk}{\kern0pt}complex{\isacharunderscore}{\kern0pt}of{\isacharunderscore}{\kern0pt}nat\ {\isacharparenleft}{\kern0pt}j\ div\ {\isadigit{2}}{\isacharparenright}{\kern0pt}\ {\isacharslash}{\kern0pt}\ {\isadigit{2}}{\isacharcircum}{\kern0pt}n{\isacharparenright}{\kern0pt}\ {\isacharasterisk}{\kern0pt}\ {\isadigit{1}}{\isachardoublequoteclose}\isanewline
\ \ \ \ \ \ \ \ \ \ \isacommand{using}\isamarkupfalse%
\ index{\isacharunderscore}{\kern0pt}mat{\isacharunderscore}{\kern0pt}of{\isacharunderscore}{\kern0pt}cols{\isacharunderscore}{\kern0pt}list\ \isacommand{by}\isamarkupfalse%
\ auto\isanewline
\ \ \ \ \ \ \ \ \isacommand{thus}\isamarkupfalse%
\ {\isachardoublequoteopen}exp\ {\isacharparenleft}{\kern0pt}{\isadigit{2}}{\isacharasterisk}{\kern0pt}{\isasymi}{\isacharasterisk}{\kern0pt}pi{\isacharasterisk}{\kern0pt}complex{\isacharunderscore}{\kern0pt}of{\isacharunderscore}{\kern0pt}nat\ {\isacharparenleft}{\kern0pt}j\ div\ {\isadigit{2}}{\isacharparenright}{\kern0pt}\ {\isacharslash}{\kern0pt}\ {\isadigit{2}}{\isacharcircum}{\kern0pt}n{\isacharparenright}{\kern0pt}\ {\isacharasterisk}{\kern0pt}\ {\isacharparenleft}{\kern0pt}mat{\isacharunderscore}{\kern0pt}of{\isacharunderscore}{\kern0pt}cols{\isacharunderscore}{\kern0pt}list\ {\isadigit{2}}\ {\isacharbrackleft}{\kern0pt}{\isacharbrackleft}{\kern0pt}{\isadigit{0}}{\isacharcomma}{\kern0pt}{\isadigit{1}}{\isacharbrackright}{\kern0pt}{\isacharbrackright}{\kern0pt}\ {\isachardollar}{\kern0pt}{\isachardollar}{\kern0pt}\ {\isacharparenleft}{\kern0pt}a{\isacharcomma}{\kern0pt}{\isadigit{0}}{\isacharparenright}{\kern0pt}{\isacharparenright}{\kern0pt}\ {\isacharequal}{\kern0pt}\isanewline
\ \ \ \ \ \ \ \ \ \ \ \ \ \ {\isacharparenleft}{\kern0pt}mat{\isacharunderscore}{\kern0pt}of{\isacharunderscore}{\kern0pt}cols{\isacharunderscore}{\kern0pt}list\ {\isadigit{2}}\ {\isacharbrackleft}{\kern0pt}{\isacharbrackleft}{\kern0pt}{\isadigit{0}}{\isacharcomma}{\kern0pt}exp\ {\isacharparenleft}{\kern0pt}{\isadigit{2}}{\isacharasterisk}{\kern0pt}{\isasymi}{\isacharasterisk}{\kern0pt}pi{\isacharasterisk}{\kern0pt}complex{\isacharunderscore}{\kern0pt}of{\isacharunderscore}{\kern0pt}nat\ {\isacharparenleft}{\kern0pt}j\ div\ {\isadigit{2}}{\isacharparenright}{\kern0pt}\ {\isacharslash}{\kern0pt}\ {\isadigit{2}}{\isacharcircum}{\kern0pt}n{\isacharparenright}{\kern0pt}{\isacharbrackright}{\kern0pt}{\isacharbrackright}{\kern0pt}{\isacharparenright}{\kern0pt}\ {\isachardollar}{\kern0pt}{\isachardollar}{\kern0pt}\ {\isacharparenleft}{\kern0pt}a{\isacharcomma}{\kern0pt}{\isadigit{0}}{\isacharparenright}{\kern0pt}{\isachardoublequoteclose}\isanewline
\ \ \ \ \ \ \ \ \ \ \isacommand{using}\isamarkupfalse%
\ a{\isadigit{1}}\ \isacommand{by}\isamarkupfalse%
\ auto\isanewline
\ \ \ \ \ \ \isacommand{qed}\isamarkupfalse%
\isanewline
\ \ \ \ \ \ \isacommand{finally}\isamarkupfalse%
\ \isacommand{show}\isamarkupfalse%
\ {\isachardoublequoteopen}{\isacharparenleft}{\kern0pt}exp\ {\isacharparenleft}{\kern0pt}{\isadigit{2}}{\isacharasterisk}{\kern0pt}{\isasymi}{\isacharasterisk}{\kern0pt}pi{\isacharasterisk}{\kern0pt}complex{\isacharunderscore}{\kern0pt}of{\isacharunderscore}{\kern0pt}nat\ {\isacharparenleft}{\kern0pt}j\ div\ {\isadigit{2}}{\isacharparenright}{\kern0pt}\ {\isacharslash}{\kern0pt}\ {\isadigit{2}}{\isacharcircum}{\kern0pt}n{\isacharparenright}{\kern0pt}\ {\isasymcdot}\isactrlsub m\ mat{\isacharunderscore}{\kern0pt}of{\isacharunderscore}{\kern0pt}cols{\isacharunderscore}{\kern0pt}list\ {\isadigit{2}}\ {\isacharbrackleft}{\kern0pt}{\isacharbrackleft}{\kern0pt}{\isadigit{0}}{\isacharcomma}{\kern0pt}{\isadigit{1}}{\isacharbrackright}{\kern0pt}{\isacharbrackright}{\kern0pt}{\isacharparenright}{\kern0pt}\ \isanewline
\ \ \ \ \ \ \ \ {\isachardollar}{\kern0pt}{\isachardollar}{\kern0pt}\ {\isacharparenleft}{\kern0pt}a{\isacharcomma}{\kern0pt}b{\isacharparenright}{\kern0pt}\ {\isacharequal}{\kern0pt}\ {\isacharparenleft}{\kern0pt}mat{\isacharunderscore}{\kern0pt}of{\isacharunderscore}{\kern0pt}cols{\isacharunderscore}{\kern0pt}list\ {\isadigit{2}}\ {\isacharbrackleft}{\kern0pt}{\isacharbrackleft}{\kern0pt}{\isadigit{0}}{\isacharcomma}{\kern0pt}exp\ {\isacharparenleft}{\kern0pt}{\isadigit{2}}{\isacharasterisk}{\kern0pt}{\isasymi}{\isacharasterisk}{\kern0pt}pi{\isacharasterisk}{\kern0pt}complex{\isacharunderscore}{\kern0pt}of{\isacharunderscore}{\kern0pt}nat\ {\isacharparenleft}{\kern0pt}j\ div\ {\isadigit{2}}{\isacharparenright}{\kern0pt}\ {\isacharslash}{\kern0pt}\ {\isadigit{2}}{\isacharcircum}{\kern0pt}n{\isacharparenright}{\kern0pt}{\isacharbrackright}{\kern0pt}{\isacharbrackright}{\kern0pt}{\isacharparenright}{\kern0pt}\ {\isachardollar}{\kern0pt}{\isachardollar}{\kern0pt}\ {\isacharparenleft}{\kern0pt}a{\isacharcomma}{\kern0pt}b{\isacharparenright}{\kern0pt}{\isachardoublequoteclose}\isanewline
\ \ \ \ \ \ \ \ \isacommand{using}\isamarkupfalse%
\ b{\isadigit{0}}\ \isacommand{by}\isamarkupfalse%
\ simp\isanewline
\ \ \ \ \isacommand{next}\isamarkupfalse%
\isanewline
\ \ \ \ \ \ \isacommand{show}\isamarkupfalse%
\ {\isachardoublequoteopen}dim{\isacharunderscore}{\kern0pt}row\ {\isacharparenleft}{\kern0pt}exp\ {\isacharparenleft}{\kern0pt}{\isadigit{2}}\ {\isacharasterisk}{\kern0pt}\ {\isasymi}\ {\isacharasterisk}{\kern0pt}\ complex{\isacharunderscore}{\kern0pt}of{\isacharunderscore}{\kern0pt}real\ pi\ {\isacharasterisk}{\kern0pt}\ complex{\isacharunderscore}{\kern0pt}of{\isacharunderscore}{\kern0pt}nat\ {\isacharparenleft}{\kern0pt}j\ div\ {\isadigit{2}}{\isacharparenright}{\kern0pt}\ {\isacharslash}{\kern0pt}\ {\isadigit{2}}\ {\isacharcircum}{\kern0pt}\ n{\isacharparenright}{\kern0pt}\ {\isasymcdot}\isactrlsub m\isanewline
\ \ \ \ \ \ \ \ \ \ \ \ Tensor{\isachardot}{\kern0pt}mat{\isacharunderscore}{\kern0pt}of{\isacharunderscore}{\kern0pt}cols{\isacharunderscore}{\kern0pt}list\ {\isadigit{2}}\ {\isacharbrackleft}{\kern0pt}{\isacharbrackleft}{\kern0pt}{\isadigit{0}}{\isacharcomma}{\kern0pt}\ {\isadigit{1}}{\isacharbrackright}{\kern0pt}{\isacharbrackright}{\kern0pt}{\isacharparenright}{\kern0pt}\ {\isacharequal}{\kern0pt}\isanewline
\ \ \ \ \ \ \ \ \ \ \ \ dim{\isacharunderscore}{\kern0pt}row\ {\isacharparenleft}{\kern0pt}Tensor{\isachardot}{\kern0pt}mat{\isacharunderscore}{\kern0pt}of{\isacharunderscore}{\kern0pt}cols{\isacharunderscore}{\kern0pt}list\ {\isadigit{2}}\ {\isacharbrackleft}{\kern0pt}{\isacharbrackleft}{\kern0pt}{\isadigit{0}}{\isacharcomma}{\kern0pt}\ exp\ {\isacharparenleft}{\kern0pt}{\isadigit{2}}\ {\isacharasterisk}{\kern0pt}\ {\isasymi}\ {\isacharasterisk}{\kern0pt}\ complex{\isacharunderscore}{\kern0pt}of{\isacharunderscore}{\kern0pt}real\ pi\ {\isacharasterisk}{\kern0pt}\isanewline
\ \ \ \ \ \ \ \ \ \ \ \ \ \ \ \ \ \ \ \ \ \ complex{\isacharunderscore}{\kern0pt}of{\isacharunderscore}{\kern0pt}nat\ {\isacharparenleft}{\kern0pt}j\ div\ {\isadigit{2}}{\isacharparenright}{\kern0pt}\ {\isacharslash}{\kern0pt}\ {\isadigit{2}}\ {\isacharcircum}{\kern0pt}\ n{\isacharparenright}{\kern0pt}{\isacharbrackright}{\kern0pt}{\isacharbrackright}{\kern0pt}{\isacharparenright}{\kern0pt}{\isachardoublequoteclose}\ \isanewline
\ \ \ \ \ \ \ \ \isacommand{by}\isamarkupfalse%
\ {\isacharparenleft}{\kern0pt}simp\ add{\isacharcolon}{\kern0pt}\ Tensor{\isachardot}{\kern0pt}mat{\isacharunderscore}{\kern0pt}of{\isacharunderscore}{\kern0pt}cols{\isacharunderscore}{\kern0pt}list{\isacharunderscore}{\kern0pt}def{\isacharparenright}{\kern0pt}\isanewline
\ \ \ \ \isacommand{next}\isamarkupfalse%
\isanewline
\ \ \ \ \ \ \isacommand{show}\isamarkupfalse%
\ {\isachardoublequoteopen}dim{\isacharunderscore}{\kern0pt}col\ {\isacharparenleft}{\kern0pt}exp\ {\isacharparenleft}{\kern0pt}{\isadigit{2}}\ {\isacharasterisk}{\kern0pt}\ {\isasymi}\ {\isacharasterisk}{\kern0pt}\ complex{\isacharunderscore}{\kern0pt}of{\isacharunderscore}{\kern0pt}real\ pi\ {\isacharasterisk}{\kern0pt}\ complex{\isacharunderscore}{\kern0pt}of{\isacharunderscore}{\kern0pt}nat\ {\isacharparenleft}{\kern0pt}j\ div\ {\isadigit{2}}{\isacharparenright}{\kern0pt}\ {\isacharslash}{\kern0pt}\ {\isadigit{2}}\ {\isacharcircum}{\kern0pt}\ n{\isacharparenright}{\kern0pt}\ {\isasymcdot}\isactrlsub m\isanewline
\ \ \ \ \ \ \ \ \ \ \ \ Tensor{\isachardot}{\kern0pt}mat{\isacharunderscore}{\kern0pt}of{\isacharunderscore}{\kern0pt}cols{\isacharunderscore}{\kern0pt}list\ {\isadigit{2}}\ {\isacharbrackleft}{\kern0pt}{\isacharbrackleft}{\kern0pt}{\isadigit{0}}{\isacharcomma}{\kern0pt}\ {\isadigit{1}}{\isacharbrackright}{\kern0pt}{\isacharbrackright}{\kern0pt}{\isacharparenright}{\kern0pt}\ {\isacharequal}{\kern0pt}\isanewline
\ \ \ \ \ \ \ \ \ \ \ \ dim{\isacharunderscore}{\kern0pt}col\ {\isacharparenleft}{\kern0pt}Tensor{\isachardot}{\kern0pt}mat{\isacharunderscore}{\kern0pt}of{\isacharunderscore}{\kern0pt}cols{\isacharunderscore}{\kern0pt}list\ {\isadigit{2}}\ {\isacharbrackleft}{\kern0pt}{\isacharbrackleft}{\kern0pt}{\isadigit{0}}{\isacharcomma}{\kern0pt}\ exp\ {\isacharparenleft}{\kern0pt}{\isadigit{2}}\ {\isacharasterisk}{\kern0pt}\ {\isasymi}\ {\isacharasterisk}{\kern0pt}\ complex{\isacharunderscore}{\kern0pt}of{\isacharunderscore}{\kern0pt}real\ pi\ {\isacharasterisk}{\kern0pt}\isanewline
\ \ \ \ \ \ \ \ \ \ \ \ \ \ \ \ \ \ \ \ \ \ complex{\isacharunderscore}{\kern0pt}of{\isacharunderscore}{\kern0pt}nat\ {\isacharparenleft}{\kern0pt}j\ div\ {\isadigit{2}}{\isacharparenright}{\kern0pt}\ {\isacharslash}{\kern0pt}\ {\isadigit{2}}\ {\isacharcircum}{\kern0pt}\ n{\isacharparenright}{\kern0pt}{\isacharbrackright}{\kern0pt}{\isacharbrackright}{\kern0pt}{\isacharparenright}{\kern0pt}{\isachardoublequoteclose}\isanewline
\ \ \ \ \ \ \ \ \isacommand{by}\isamarkupfalse%
\ {\isacharparenleft}{\kern0pt}simp\ add{\isacharcolon}{\kern0pt}\ mat{\isacharunderscore}{\kern0pt}of{\isacharunderscore}{\kern0pt}cols{\isacharunderscore}{\kern0pt}list{\isacharunderscore}{\kern0pt}def{\isacharparenright}{\kern0pt}\isanewline
\ \ \ \ \isacommand{qed}\isamarkupfalse%
\isanewline
\ \ \ \ \isacommand{thus}\isamarkupfalse%
\ {\isacharquery}{\kern0pt}thesis\ \isacommand{by}\isamarkupfalse%
\ auto\isanewline
\ \ \isacommand{qed}\isamarkupfalse%
\isanewline
\ \ \isacommand{also}\isamarkupfalse%
\ \isacommand{have}\isamarkupfalse%
\ {\isachardoublequoteopen}{\isasymdots}\ {\isacharequal}{\kern0pt}\ {\isacharparenleft}{\kern0pt}mat{\isacharunderscore}{\kern0pt}of{\isacharunderscore}{\kern0pt}cols{\isacharunderscore}{\kern0pt}list\ {\isadigit{2}}\ {\isacharbrackleft}{\kern0pt}{\isacharbrackleft}{\kern0pt}{\isadigit{1}}{\isacharcomma}{\kern0pt}\ {\isadigit{0}}{\isacharbrackright}{\kern0pt}{\isacharcomma}{\kern0pt}{\isacharbrackleft}{\kern0pt}{\isadigit{0}}{\isacharcomma}{\kern0pt}\ exp{\isacharparenleft}{\kern0pt}{\isadigit{2}}{\isacharasterisk}{\kern0pt}pi{\isacharasterisk}{\kern0pt}{\isasymi}{\isacharslash}{\kern0pt}{\isadigit{2}}{\isacharcircum}{\kern0pt}{\isacharparenleft}{\kern0pt}Suc\ n{\isacharparenright}{\kern0pt}{\isacharparenright}{\kern0pt}{\isacharbrackright}{\kern0pt}{\isacharbrackright}{\kern0pt}{\isacharparenright}{\kern0pt}\ {\isacharasterisk}{\kern0pt}\isanewline
\ \ \ \ \ \ \ \ \ \ \ \ \ \ \ \ \ \ {\isacharparenleft}{\kern0pt}mat{\isacharunderscore}{\kern0pt}of{\isacharunderscore}{\kern0pt}cols{\isacharunderscore}{\kern0pt}list\ {\isadigit{2}}\ {\isacharbrackleft}{\kern0pt}{\isacharbrackleft}{\kern0pt}{\isadigit{1}}{\isacharcomma}{\kern0pt}exp\ {\isacharparenleft}{\kern0pt}{\isadigit{2}}{\isacharasterisk}{\kern0pt}{\isasymi}{\isacharasterisk}{\kern0pt}pi{\isacharasterisk}{\kern0pt}complex{\isacharunderscore}{\kern0pt}of{\isacharunderscore}{\kern0pt}nat\ {\isacharparenleft}{\kern0pt}j\ div\ {\isadigit{2}}{\isacharparenright}{\kern0pt}\ {\isacharslash}{\kern0pt}\ {\isadigit{2}}{\isacharcircum}{\kern0pt}n{\isacharparenright}{\kern0pt}{\isacharbrackright}{\kern0pt}{\isacharbrackright}{\kern0pt}{\isacharparenright}{\kern0pt}{\isachardoublequoteclose}\isanewline
\ \ \isacommand{proof}\isamarkupfalse%
\ {\isacharminus}{\kern0pt}\isanewline
\ \ \ \ \isacommand{have}\isamarkupfalse%
\ {\isachardoublequoteopen}mat{\isacharunderscore}{\kern0pt}of{\isacharunderscore}{\kern0pt}cols{\isacharunderscore}{\kern0pt}list\ {\isadigit{2}}\ {\isacharbrackleft}{\kern0pt}{\isacharbrackleft}{\kern0pt}{\isadigit{1}}{\isacharcomma}{\kern0pt}{\isadigit{0}}{\isacharbrackright}{\kern0pt}{\isacharbrackright}{\kern0pt}\ {\isacharplus}{\kern0pt}\ \isanewline
\ \ \ \ \ \ \ \ \ \ mat{\isacharunderscore}{\kern0pt}of{\isacharunderscore}{\kern0pt}cols{\isacharunderscore}{\kern0pt}list\ {\isadigit{2}}\ {\isacharbrackleft}{\kern0pt}{\isacharbrackleft}{\kern0pt}{\isadigit{0}}{\isacharcomma}{\kern0pt}exp\ {\isacharparenleft}{\kern0pt}{\isadigit{2}}{\isacharasterisk}{\kern0pt}{\isasymi}{\isacharasterisk}{\kern0pt}pi{\isacharasterisk}{\kern0pt}complex{\isacharunderscore}{\kern0pt}of{\isacharunderscore}{\kern0pt}nat\ {\isacharparenleft}{\kern0pt}j\ div\ {\isadigit{2}}{\isacharparenright}{\kern0pt}\ {\isacharslash}{\kern0pt}\ {\isadigit{2}}{\isacharcircum}{\kern0pt}n{\isacharparenright}{\kern0pt}{\isacharbrackright}{\kern0pt}{\isacharbrackright}{\kern0pt}\ {\isacharequal}{\kern0pt}\ \isanewline
\ \ \ \ \ \ \ \ \ \ mat{\isacharunderscore}{\kern0pt}of{\isacharunderscore}{\kern0pt}cols{\isacharunderscore}{\kern0pt}list\ {\isadigit{2}}\ {\isacharbrackleft}{\kern0pt}{\isacharbrackleft}{\kern0pt}{\isadigit{1}}{\isacharcomma}{\kern0pt}exp\ {\isacharparenleft}{\kern0pt}{\isadigit{2}}{\isacharasterisk}{\kern0pt}{\isasymi}{\isacharasterisk}{\kern0pt}pi{\isacharasterisk}{\kern0pt}complex{\isacharunderscore}{\kern0pt}of{\isacharunderscore}{\kern0pt}nat\ {\isacharparenleft}{\kern0pt}j\ div\ {\isadigit{2}}{\isacharparenright}{\kern0pt}\ {\isacharslash}{\kern0pt}\ {\isadigit{2}}{\isacharcircum}{\kern0pt}n{\isacharparenright}{\kern0pt}{\isacharbrackright}{\kern0pt}{\isacharbrackright}{\kern0pt}{\isachardoublequoteclose}\isanewline
\ \ \ \ \isacommand{proof}\isamarkupfalse%
\ \isanewline
\ \ \ \ \ \ \isacommand{fix}\isamarkupfalse%
\ a\ b{\isacharcolon}{\kern0pt}{\isacharcolon}{\kern0pt}nat\isanewline
\ \ \ \ \ \ \isacommand{assume}\isamarkupfalse%
\ {\isachardoublequoteopen}a\ {\isacharless}{\kern0pt}\ dim{\isacharunderscore}{\kern0pt}row\ {\isacharparenleft}{\kern0pt}mat{\isacharunderscore}{\kern0pt}of{\isacharunderscore}{\kern0pt}cols{\isacharunderscore}{\kern0pt}list\ {\isadigit{2}}\ {\isacharbrackleft}{\kern0pt}{\isacharbrackleft}{\kern0pt}{\isadigit{1}}{\isacharcomma}{\kern0pt}exp\ {\isacharparenleft}{\kern0pt}{\isadigit{2}}{\isacharasterisk}{\kern0pt}{\isasymi}{\isacharasterisk}{\kern0pt}pi{\isacharasterisk}{\kern0pt}complex{\isacharunderscore}{\kern0pt}of{\isacharunderscore}{\kern0pt}nat\ {\isacharparenleft}{\kern0pt}j\ div\ {\isadigit{2}}{\isacharparenright}{\kern0pt}\ {\isacharslash}{\kern0pt}\ {\isadigit{2}}{\isacharcircum}{\kern0pt}n{\isacharparenright}{\kern0pt}{\isacharbrackright}{\kern0pt}{\isacharbrackright}{\kern0pt}{\isacharparenright}{\kern0pt}{\isachardoublequoteclose}\isanewline
\ \ \ \ \ \ \isacommand{hence}\isamarkupfalse%
\ a{\isadigit{2}}{\isacharcolon}{\kern0pt}{\isachardoublequoteopen}a\ {\isacharless}{\kern0pt}\ {\isadigit{2}}{\isachardoublequoteclose}\ \isacommand{using}\isamarkupfalse%
\ mat{\isacharunderscore}{\kern0pt}of{\isacharunderscore}{\kern0pt}cols{\isacharunderscore}{\kern0pt}list{\isacharunderscore}{\kern0pt}def\ \isacommand{by}\isamarkupfalse%
\ simp\ \isanewline
\ \ \ \ \ \ \isacommand{assume}\isamarkupfalse%
\ {\isachardoublequoteopen}b\ {\isacharless}{\kern0pt}\ dim{\isacharunderscore}{\kern0pt}col\ {\isacharparenleft}{\kern0pt}mat{\isacharunderscore}{\kern0pt}of{\isacharunderscore}{\kern0pt}cols{\isacharunderscore}{\kern0pt}list\ {\isadigit{2}}\ {\isacharbrackleft}{\kern0pt}{\isacharbrackleft}{\kern0pt}{\isadigit{1}}{\isacharcomma}{\kern0pt}exp\ {\isacharparenleft}{\kern0pt}{\isadigit{2}}{\isacharasterisk}{\kern0pt}{\isasymi}{\isacharasterisk}{\kern0pt}pi{\isacharasterisk}{\kern0pt}complex{\isacharunderscore}{\kern0pt}of{\isacharunderscore}{\kern0pt}nat\ {\isacharparenleft}{\kern0pt}j\ div\ {\isadigit{2}}{\isacharparenright}{\kern0pt}\ {\isacharslash}{\kern0pt}\ {\isadigit{2}}{\isacharcircum}{\kern0pt}n{\isacharparenright}{\kern0pt}{\isacharbrackright}{\kern0pt}{\isacharbrackright}{\kern0pt}{\isacharparenright}{\kern0pt}{\isachardoublequoteclose}\isanewline
\ \ \ \ \ \ \isacommand{hence}\isamarkupfalse%
\ b{\isadigit{0}}{\isacharcolon}{\kern0pt}{\isachardoublequoteopen}b\ {\isacharequal}{\kern0pt}\ {\isadigit{0}}{\isachardoublequoteclose}\ \isacommand{using}\isamarkupfalse%
\ mat{\isacharunderscore}{\kern0pt}of{\isacharunderscore}{\kern0pt}cols{\isacharunderscore}{\kern0pt}list{\isacharunderscore}{\kern0pt}def\ \isacommand{by}\isamarkupfalse%
\ auto\isanewline
\ \ \ \ \ \ \isacommand{show}\isamarkupfalse%
\ {\isachardoublequoteopen}{\isacharparenleft}{\kern0pt}mat{\isacharunderscore}{\kern0pt}of{\isacharunderscore}{\kern0pt}cols{\isacharunderscore}{\kern0pt}list\ {\isadigit{2}}\ {\isacharbrackleft}{\kern0pt}{\isacharbrackleft}{\kern0pt}{\isadigit{1}}{\isacharcomma}{\kern0pt}{\isadigit{0}}{\isacharbrackright}{\kern0pt}{\isacharbrackright}{\kern0pt}\ {\isacharplus}{\kern0pt}\ \isanewline
\ \ \ \ \ \ \ \ \ \ \ \ \ mat{\isacharunderscore}{\kern0pt}of{\isacharunderscore}{\kern0pt}cols{\isacharunderscore}{\kern0pt}list\ {\isadigit{2}}\ {\isacharbrackleft}{\kern0pt}{\isacharbrackleft}{\kern0pt}{\isadigit{0}}{\isacharcomma}{\kern0pt}exp\ {\isacharparenleft}{\kern0pt}{\isadigit{2}}{\isacharasterisk}{\kern0pt}{\isasymi}{\isacharasterisk}{\kern0pt}pi{\isacharasterisk}{\kern0pt}complex{\isacharunderscore}{\kern0pt}of{\isacharunderscore}{\kern0pt}nat\ {\isacharparenleft}{\kern0pt}j\ div\ {\isadigit{2}}{\isacharparenright}{\kern0pt}\ {\isacharslash}{\kern0pt}\ {\isadigit{2}}{\isacharcircum}{\kern0pt}n{\isacharparenright}{\kern0pt}{\isacharbrackright}{\kern0pt}{\isacharbrackright}{\kern0pt}{\isacharparenright}{\kern0pt}\ {\isachardollar}{\kern0pt}{\isachardollar}{\kern0pt}\ {\isacharparenleft}{\kern0pt}a{\isacharcomma}{\kern0pt}b{\isacharparenright}{\kern0pt}\ {\isacharequal}{\kern0pt}\ \isanewline
\ \ \ \ \ \ \ \ \ \ \ \ {\isacharparenleft}{\kern0pt}mat{\isacharunderscore}{\kern0pt}of{\isacharunderscore}{\kern0pt}cols{\isacharunderscore}{\kern0pt}list\ {\isadigit{2}}\ {\isacharbrackleft}{\kern0pt}{\isacharbrackleft}{\kern0pt}{\isadigit{1}}{\isacharcomma}{\kern0pt}exp\ {\isacharparenleft}{\kern0pt}{\isadigit{2}}{\isacharasterisk}{\kern0pt}{\isasymi}{\isacharasterisk}{\kern0pt}pi{\isacharasterisk}{\kern0pt}complex{\isacharunderscore}{\kern0pt}of{\isacharunderscore}{\kern0pt}nat\ {\isacharparenleft}{\kern0pt}j\ div\ {\isadigit{2}}{\isacharparenright}{\kern0pt}\ {\isacharslash}{\kern0pt}\ {\isadigit{2}}{\isacharcircum}{\kern0pt}n{\isacharparenright}{\kern0pt}{\isacharbrackright}{\kern0pt}{\isacharbrackright}{\kern0pt}{\isacharparenright}{\kern0pt}\ {\isachardollar}{\kern0pt}{\isachardollar}{\kern0pt}\ {\isacharparenleft}{\kern0pt}a{\isacharcomma}{\kern0pt}b{\isacharparenright}{\kern0pt}{\isachardoublequoteclose}\isanewline
\ \ \ \ \ \ \isacommand{proof}\isamarkupfalse%
\ {\isacharparenleft}{\kern0pt}rule\ disjE{\isacharparenright}{\kern0pt}\isanewline
\ \ \ \ \ \ \ \ \isacommand{show}\isamarkupfalse%
\ {\isachardoublequoteopen}a\ {\isacharequal}{\kern0pt}\ {\isadigit{0}}\ {\isasymor}\ a\ {\isacharequal}{\kern0pt}\ {\isadigit{1}}{\isachardoublequoteclose}\ \isacommand{using}\isamarkupfalse%
\ a{\isadigit{2}}\ \isacommand{by}\isamarkupfalse%
\ auto\isanewline
\ \ \ \ \ \ \isacommand{next}\isamarkupfalse%
\isanewline
\ \ \ \ \ \ \ \ \isacommand{assume}\isamarkupfalse%
\ a{\isadigit{0}}{\isacharcolon}{\kern0pt}{\isachardoublequoteopen}a\ {\isacharequal}{\kern0pt}\ {\isadigit{0}}{\isachardoublequoteclose}\isanewline
\ \ \ \ \ \ \ \ \isacommand{have}\isamarkupfalse%
\ {\isachardoublequoteopen}{\isacharparenleft}{\kern0pt}mat{\isacharunderscore}{\kern0pt}of{\isacharunderscore}{\kern0pt}cols{\isacharunderscore}{\kern0pt}list\ {\isadigit{2}}\ {\isacharbrackleft}{\kern0pt}{\isacharbrackleft}{\kern0pt}{\isadigit{1}}{\isacharcomma}{\kern0pt}{\isadigit{0}}{\isacharbrackright}{\kern0pt}{\isacharbrackright}{\kern0pt}\ {\isacharplus}{\kern0pt}\ \isanewline
\ \ \ \ \ \ \ \ \ \ \ \ \ \ \ mat{\isacharunderscore}{\kern0pt}of{\isacharunderscore}{\kern0pt}cols{\isacharunderscore}{\kern0pt}list\ {\isadigit{2}}\ {\isacharbrackleft}{\kern0pt}{\isacharbrackleft}{\kern0pt}{\isadigit{0}}{\isacharcomma}{\kern0pt}exp\ {\isacharparenleft}{\kern0pt}{\isadigit{2}}{\isacharasterisk}{\kern0pt}{\isasymi}{\isacharasterisk}{\kern0pt}pi{\isacharasterisk}{\kern0pt}complex{\isacharunderscore}{\kern0pt}of{\isacharunderscore}{\kern0pt}nat\ {\isacharparenleft}{\kern0pt}j\ div\ {\isadigit{2}}{\isacharparenright}{\kern0pt}\ {\isacharslash}{\kern0pt}\ {\isadigit{2}}{\isacharcircum}{\kern0pt}n{\isacharparenright}{\kern0pt}{\isacharbrackright}{\kern0pt}{\isacharbrackright}{\kern0pt}{\isacharparenright}{\kern0pt}\ {\isachardollar}{\kern0pt}{\isachardollar}{\kern0pt}\ {\isacharparenleft}{\kern0pt}{\isadigit{0}}{\isacharcomma}{\kern0pt}{\isadigit{0}}{\isacharparenright}{\kern0pt}\ {\isacharequal}{\kern0pt}\ \isanewline
\ \ \ \ \ \ \ \ \ \ \ \ \ \ {\isacharparenleft}{\kern0pt}mat{\isacharunderscore}{\kern0pt}of{\isacharunderscore}{\kern0pt}cols{\isacharunderscore}{\kern0pt}list\ {\isadigit{2}}\ {\isacharbrackleft}{\kern0pt}{\isacharbrackleft}{\kern0pt}{\isadigit{1}}{\isacharcomma}{\kern0pt}exp\ {\isacharparenleft}{\kern0pt}{\isadigit{2}}{\isacharasterisk}{\kern0pt}{\isasymi}{\isacharasterisk}{\kern0pt}pi{\isacharasterisk}{\kern0pt}complex{\isacharunderscore}{\kern0pt}of{\isacharunderscore}{\kern0pt}nat\ {\isacharparenleft}{\kern0pt}j\ div\ {\isadigit{2}}{\isacharparenright}{\kern0pt}\ {\isacharslash}{\kern0pt}\ {\isadigit{2}}{\isacharcircum}{\kern0pt}n{\isacharparenright}{\kern0pt}{\isacharbrackright}{\kern0pt}{\isacharbrackright}{\kern0pt}{\isacharparenright}{\kern0pt}\ {\isachardollar}{\kern0pt}{\isachardollar}{\kern0pt}\ {\isacharparenleft}{\kern0pt}{\isadigit{0}}{\isacharcomma}{\kern0pt}{\isadigit{0}}{\isacharparenright}{\kern0pt}{\isachardoublequoteclose}\isanewline
\ \ \ \ \ \ \ \ \ \ \isacommand{using}\isamarkupfalse%
\ index{\isacharunderscore}{\kern0pt}mat{\isacharunderscore}{\kern0pt}of{\isacharunderscore}{\kern0pt}cols{\isacharunderscore}{\kern0pt}list\ \isacommand{by}\isamarkupfalse%
\ {\isacharparenleft}{\kern0pt}simp\ add{\isacharcolon}{\kern0pt}\ Tensor{\isachardot}{\kern0pt}mat{\isacharunderscore}{\kern0pt}of{\isacharunderscore}{\kern0pt}cols{\isacharunderscore}{\kern0pt}list{\isacharunderscore}{\kern0pt}def{\isacharparenright}{\kern0pt}\isanewline
\ \ \ \ \ \ \ \ \isacommand{thus}\isamarkupfalse%
\ {\isachardoublequoteopen}{\isacharparenleft}{\kern0pt}mat{\isacharunderscore}{\kern0pt}of{\isacharunderscore}{\kern0pt}cols{\isacharunderscore}{\kern0pt}list\ {\isadigit{2}}\ {\isacharbrackleft}{\kern0pt}{\isacharbrackleft}{\kern0pt}{\isadigit{1}}{\isacharcomma}{\kern0pt}{\isadigit{0}}{\isacharbrackright}{\kern0pt}{\isacharbrackright}{\kern0pt}\ {\isacharplus}{\kern0pt}\ \isanewline
\ \ \ \ \ \ \ \ \ \ \ \ \ \ \ mat{\isacharunderscore}{\kern0pt}of{\isacharunderscore}{\kern0pt}cols{\isacharunderscore}{\kern0pt}list\ {\isadigit{2}}\ {\isacharbrackleft}{\kern0pt}{\isacharbrackleft}{\kern0pt}{\isadigit{0}}{\isacharcomma}{\kern0pt}exp\ {\isacharparenleft}{\kern0pt}{\isadigit{2}}{\isacharasterisk}{\kern0pt}{\isasymi}{\isacharasterisk}{\kern0pt}pi{\isacharasterisk}{\kern0pt}complex{\isacharunderscore}{\kern0pt}of{\isacharunderscore}{\kern0pt}nat\ {\isacharparenleft}{\kern0pt}j\ div\ {\isadigit{2}}{\isacharparenright}{\kern0pt}\ {\isacharslash}{\kern0pt}\ {\isadigit{2}}{\isacharcircum}{\kern0pt}n{\isacharparenright}{\kern0pt}{\isacharbrackright}{\kern0pt}{\isacharbrackright}{\kern0pt}{\isacharparenright}{\kern0pt}\ {\isachardollar}{\kern0pt}{\isachardollar}{\kern0pt}\ {\isacharparenleft}{\kern0pt}a{\isacharcomma}{\kern0pt}b{\isacharparenright}{\kern0pt}\ {\isacharequal}{\kern0pt}\ \isanewline
\ \ \ \ \ \ \ \ \ \ \ \ \ \ {\isacharparenleft}{\kern0pt}mat{\isacharunderscore}{\kern0pt}of{\isacharunderscore}{\kern0pt}cols{\isacharunderscore}{\kern0pt}list\ {\isadigit{2}}\ {\isacharbrackleft}{\kern0pt}{\isacharbrackleft}{\kern0pt}{\isadigit{1}}{\isacharcomma}{\kern0pt}exp\ {\isacharparenleft}{\kern0pt}{\isadigit{2}}{\isacharasterisk}{\kern0pt}{\isasymi}{\isacharasterisk}{\kern0pt}pi{\isacharasterisk}{\kern0pt}complex{\isacharunderscore}{\kern0pt}of{\isacharunderscore}{\kern0pt}nat\ {\isacharparenleft}{\kern0pt}j\ div\ {\isadigit{2}}{\isacharparenright}{\kern0pt}\ {\isacharslash}{\kern0pt}\ {\isadigit{2}}{\isacharcircum}{\kern0pt}n{\isacharparenright}{\kern0pt}{\isacharbrackright}{\kern0pt}{\isacharbrackright}{\kern0pt}{\isacharparenright}{\kern0pt}\ {\isachardollar}{\kern0pt}{\isachardollar}{\kern0pt}\ {\isacharparenleft}{\kern0pt}a{\isacharcomma}{\kern0pt}b{\isacharparenright}{\kern0pt}{\isachardoublequoteclose}\isanewline
\ \ \ \ \ \ \ \ \ \ \isacommand{using}\isamarkupfalse%
\ a{\isadigit{0}}\ b{\isadigit{0}}\ \isacommand{by}\isamarkupfalse%
\ simp\isanewline
\ \ \ \ \ \ \isacommand{next}\isamarkupfalse%
\isanewline
\ \ \ \ \ \ \ \ \isacommand{assume}\isamarkupfalse%
\ a{\isadigit{1}}{\isacharcolon}{\kern0pt}{\isachardoublequoteopen}a\ {\isacharequal}{\kern0pt}\ {\isadigit{1}}{\isachardoublequoteclose}\isanewline
\ \ \ \ \ \ \ \ \isacommand{show}\isamarkupfalse%
\ {\isachardoublequoteopen}{\isacharparenleft}{\kern0pt}mat{\isacharunderscore}{\kern0pt}of{\isacharunderscore}{\kern0pt}cols{\isacharunderscore}{\kern0pt}list\ {\isadigit{2}}\ {\isacharbrackleft}{\kern0pt}{\isacharbrackleft}{\kern0pt}{\isadigit{1}}{\isacharcomma}{\kern0pt}{\isadigit{0}}{\isacharbrackright}{\kern0pt}{\isacharbrackright}{\kern0pt}\ {\isacharplus}{\kern0pt}\ \isanewline
\ \ \ \ \ \ \ \ \ \ \ \ \ \ \ mat{\isacharunderscore}{\kern0pt}of{\isacharunderscore}{\kern0pt}cols{\isacharunderscore}{\kern0pt}list\ {\isadigit{2}}\ {\isacharbrackleft}{\kern0pt}{\isacharbrackleft}{\kern0pt}{\isadigit{0}}{\isacharcomma}{\kern0pt}exp\ {\isacharparenleft}{\kern0pt}{\isadigit{2}}{\isacharasterisk}{\kern0pt}{\isasymi}{\isacharasterisk}{\kern0pt}pi{\isacharasterisk}{\kern0pt}complex{\isacharunderscore}{\kern0pt}of{\isacharunderscore}{\kern0pt}nat\ {\isacharparenleft}{\kern0pt}j\ div\ {\isadigit{2}}{\isacharparenright}{\kern0pt}\ {\isacharslash}{\kern0pt}\ {\isadigit{2}}{\isacharcircum}{\kern0pt}n{\isacharparenright}{\kern0pt}{\isacharbrackright}{\kern0pt}{\isacharbrackright}{\kern0pt}{\isacharparenright}{\kern0pt}\ {\isachardollar}{\kern0pt}{\isachardollar}{\kern0pt}\ {\isacharparenleft}{\kern0pt}a{\isacharcomma}{\kern0pt}b{\isacharparenright}{\kern0pt}\ {\isacharequal}{\kern0pt}\ \isanewline
\ \ \ \ \ \ \ \ \ \ \ \ \ \ {\isacharparenleft}{\kern0pt}mat{\isacharunderscore}{\kern0pt}of{\isacharunderscore}{\kern0pt}cols{\isacharunderscore}{\kern0pt}list\ {\isadigit{2}}\ {\isacharbrackleft}{\kern0pt}{\isacharbrackleft}{\kern0pt}{\isadigit{1}}{\isacharcomma}{\kern0pt}exp\ {\isacharparenleft}{\kern0pt}{\isadigit{2}}{\isacharasterisk}{\kern0pt}{\isasymi}{\isacharasterisk}{\kern0pt}pi{\isacharasterisk}{\kern0pt}complex{\isacharunderscore}{\kern0pt}of{\isacharunderscore}{\kern0pt}nat\ {\isacharparenleft}{\kern0pt}j\ div\ {\isadigit{2}}{\isacharparenright}{\kern0pt}\ {\isacharslash}{\kern0pt}\ {\isadigit{2}}{\isacharcircum}{\kern0pt}n{\isacharparenright}{\kern0pt}{\isacharbrackright}{\kern0pt}{\isacharbrackright}{\kern0pt}{\isacharparenright}{\kern0pt}\ {\isachardollar}{\kern0pt}{\isachardollar}{\kern0pt}\ {\isacharparenleft}{\kern0pt}a{\isacharcomma}{\kern0pt}b{\isacharparenright}{\kern0pt}{\isachardoublequoteclose}\isanewline
\ \ \ \ \ \ \ \ \ \ \isacommand{using}\isamarkupfalse%
\ a{\isadigit{1}}\ b{\isadigit{0}}\ index{\isacharunderscore}{\kern0pt}mat{\isacharunderscore}{\kern0pt}of{\isacharunderscore}{\kern0pt}cols{\isacharunderscore}{\kern0pt}list\ mat{\isacharunderscore}{\kern0pt}of{\isacharunderscore}{\kern0pt}cols{\isacharunderscore}{\kern0pt}list{\isacharunderscore}{\kern0pt}def\ \isacommand{by}\isamarkupfalse%
\ simp\isanewline
\ \ \ \ \ \ \isacommand{qed}\isamarkupfalse%
\isanewline
\ \ \ \ \isacommand{next}\isamarkupfalse%
\isanewline
\ \ \ \ \ \ \isacommand{show}\isamarkupfalse%
\ {\isachardoublequoteopen}dim{\isacharunderscore}{\kern0pt}row\ {\isacharparenleft}{\kern0pt}Tensor{\isachardot}{\kern0pt}mat{\isacharunderscore}{\kern0pt}of{\isacharunderscore}{\kern0pt}cols{\isacharunderscore}{\kern0pt}list\ {\isadigit{2}}\ {\isacharbrackleft}{\kern0pt}{\isacharbrackleft}{\kern0pt}{\isadigit{1}}{\isacharcomma}{\kern0pt}\ {\isadigit{0}}{\isacharbrackright}{\kern0pt}{\isacharbrackright}{\kern0pt}\ {\isacharplus}{\kern0pt}\ Tensor{\isachardot}{\kern0pt}mat{\isacharunderscore}{\kern0pt}of{\isacharunderscore}{\kern0pt}cols{\isacharunderscore}{\kern0pt}list\ {\isadigit{2}}\isanewline
\ \ \ \ \ \ \ \ \ \ \ \ {\isacharbrackleft}{\kern0pt}{\isacharbrackleft}{\kern0pt}{\isadigit{0}}{\isacharcomma}{\kern0pt}\ exp\ {\isacharparenleft}{\kern0pt}{\isadigit{2}}\ {\isacharasterisk}{\kern0pt}\ {\isasymi}\ {\isacharasterisk}{\kern0pt}\ complex{\isacharunderscore}{\kern0pt}of{\isacharunderscore}{\kern0pt}real\ pi\ {\isacharasterisk}{\kern0pt}\ complex{\isacharunderscore}{\kern0pt}of{\isacharunderscore}{\kern0pt}nat\ {\isacharparenleft}{\kern0pt}j\ div\ {\isadigit{2}}{\isacharparenright}{\kern0pt}\ {\isacharslash}{\kern0pt}\ {\isadigit{2}}\ {\isacharcircum}{\kern0pt}\ n{\isacharparenright}{\kern0pt}{\isacharbrackright}{\kern0pt}{\isacharbrackright}{\kern0pt}{\isacharparenright}{\kern0pt}\ {\isacharequal}{\kern0pt}\isanewline
\ \ \ \ \ \ \ \ \ \ \ \ dim{\isacharunderscore}{\kern0pt}row\ {\isacharparenleft}{\kern0pt}Tensor{\isachardot}{\kern0pt}mat{\isacharunderscore}{\kern0pt}of{\isacharunderscore}{\kern0pt}cols{\isacharunderscore}{\kern0pt}list\ {\isadigit{2}}\ {\isacharbrackleft}{\kern0pt}{\isacharbrackleft}{\kern0pt}{\isadigit{1}}{\isacharcomma}{\kern0pt}\ exp\ {\isacharparenleft}{\kern0pt}{\isadigit{2}}\ {\isacharasterisk}{\kern0pt}\ {\isasymi}\ {\isacharasterisk}{\kern0pt}\ complex{\isacharunderscore}{\kern0pt}of{\isacharunderscore}{\kern0pt}real\ pi\ {\isacharasterisk}{\kern0pt}\isanewline
\ \ \ \ \ \ \ \ \ \ \ \ \ \ \ \ \ \ \ \ complex{\isacharunderscore}{\kern0pt}of{\isacharunderscore}{\kern0pt}nat\ {\isacharparenleft}{\kern0pt}j\ div\ {\isadigit{2}}{\isacharparenright}{\kern0pt}\ {\isacharslash}{\kern0pt}\ {\isadigit{2}}\ {\isacharcircum}{\kern0pt}\ n{\isacharparenright}{\kern0pt}{\isacharbrackright}{\kern0pt}{\isacharbrackright}{\kern0pt}{\isacharparenright}{\kern0pt}{\isachardoublequoteclose}\isanewline
\ \ \ \ \ \ \ \ \isacommand{by}\isamarkupfalse%
\ {\isacharparenleft}{\kern0pt}simp\ add{\isacharcolon}{\kern0pt}\ Tensor{\isachardot}{\kern0pt}mat{\isacharunderscore}{\kern0pt}of{\isacharunderscore}{\kern0pt}cols{\isacharunderscore}{\kern0pt}list{\isacharunderscore}{\kern0pt}def{\isacharparenright}{\kern0pt}\isanewline
\ \ \ \ \isacommand{next}\isamarkupfalse%
\ \isanewline
\ \ \ \ \ \ \isacommand{show}\isamarkupfalse%
\ {\isachardoublequoteopen}dim{\isacharunderscore}{\kern0pt}col\ {\isacharparenleft}{\kern0pt}Tensor{\isachardot}{\kern0pt}mat{\isacharunderscore}{\kern0pt}of{\isacharunderscore}{\kern0pt}cols{\isacharunderscore}{\kern0pt}list\ {\isadigit{2}}\ {\isacharbrackleft}{\kern0pt}{\isacharbrackleft}{\kern0pt}{\isadigit{1}}{\isacharcomma}{\kern0pt}\ {\isadigit{0}}{\isacharbrackright}{\kern0pt}{\isacharbrackright}{\kern0pt}\ {\isacharplus}{\kern0pt}\ Tensor{\isachardot}{\kern0pt}mat{\isacharunderscore}{\kern0pt}of{\isacharunderscore}{\kern0pt}cols{\isacharunderscore}{\kern0pt}list\ {\isadigit{2}}\isanewline
\ \ \ \ \ \ \ \ \ \ \ \ {\isacharbrackleft}{\kern0pt}{\isacharbrackleft}{\kern0pt}{\isadigit{0}}{\isacharcomma}{\kern0pt}\ exp\ {\isacharparenleft}{\kern0pt}{\isadigit{2}}\ {\isacharasterisk}{\kern0pt}\ {\isasymi}\ {\isacharasterisk}{\kern0pt}\ complex{\isacharunderscore}{\kern0pt}of{\isacharunderscore}{\kern0pt}real\ pi\ {\isacharasterisk}{\kern0pt}\ complex{\isacharunderscore}{\kern0pt}of{\isacharunderscore}{\kern0pt}nat\ {\isacharparenleft}{\kern0pt}j\ div\ {\isadigit{2}}{\isacharparenright}{\kern0pt}\ {\isacharslash}{\kern0pt}\ {\isadigit{2}}\ {\isacharcircum}{\kern0pt}\ n{\isacharparenright}{\kern0pt}{\isacharbrackright}{\kern0pt}{\isacharbrackright}{\kern0pt}{\isacharparenright}{\kern0pt}\ {\isacharequal}{\kern0pt}\isanewline
\ \ \ \ \ \ \ \ \ \ \ \ dim{\isacharunderscore}{\kern0pt}col\ {\isacharparenleft}{\kern0pt}Tensor{\isachardot}{\kern0pt}mat{\isacharunderscore}{\kern0pt}of{\isacharunderscore}{\kern0pt}cols{\isacharunderscore}{\kern0pt}list\ {\isadigit{2}}\ {\isacharbrackleft}{\kern0pt}{\isacharbrackleft}{\kern0pt}{\isadigit{1}}{\isacharcomma}{\kern0pt}\ exp\ {\isacharparenleft}{\kern0pt}{\isadigit{2}}\ {\isacharasterisk}{\kern0pt}\ {\isasymi}\ {\isacharasterisk}{\kern0pt}\ complex{\isacharunderscore}{\kern0pt}of{\isacharunderscore}{\kern0pt}real\ pi\ {\isacharasterisk}{\kern0pt}\isanewline
\ \ \ \ \ \ \ \ \ \ \ \ \ \ \ \ \ \ \ \ complex{\isacharunderscore}{\kern0pt}of{\isacharunderscore}{\kern0pt}nat\ {\isacharparenleft}{\kern0pt}j\ div\ {\isadigit{2}}{\isacharparenright}{\kern0pt}\ {\isacharslash}{\kern0pt}\ {\isadigit{2}}\ {\isacharcircum}{\kern0pt}\ n{\isacharparenright}{\kern0pt}{\isacharbrackright}{\kern0pt}{\isacharbrackright}{\kern0pt}{\isacharparenright}{\kern0pt}{\isachardoublequoteclose}\isanewline
\ \ \ \ \ \ \ \ \isacommand{by}\isamarkupfalse%
\ {\isacharparenleft}{\kern0pt}simp\ add{\isacharcolon}{\kern0pt}\ mat{\isacharunderscore}{\kern0pt}of{\isacharunderscore}{\kern0pt}cols{\isacharunderscore}{\kern0pt}list{\isacharunderscore}{\kern0pt}def{\isacharparenright}{\kern0pt}\isanewline
\ \ \ \ \isacommand{qed}\isamarkupfalse%
\isanewline
\ \ \ \ \isacommand{thus}\isamarkupfalse%
\ {\isacharquery}{\kern0pt}thesis\ \isacommand{by}\isamarkupfalse%
\ simp\isanewline
\ \ \isacommand{qed}\isamarkupfalse%
\isanewline
\ \ \isacommand{finally}\isamarkupfalse%
\ \isacommand{have}\isamarkupfalse%
\ {\isadigit{1}}{\isacharcolon}{\kern0pt}{\isachardoublequoteopen}R\ {\isacharparenleft}{\kern0pt}Suc\ n{\isacharparenright}{\kern0pt}\ {\isacharasterisk}{\kern0pt}\ {\isacharparenleft}{\kern0pt}\ {\isacharbar}{\kern0pt}Deutsch{\isachardot}{\kern0pt}zero{\isasymrangle}\ {\isacharplus}{\kern0pt}\ exp\ {\isacharparenleft}{\kern0pt}{\isadigit{2}}\ {\isacharasterisk}{\kern0pt}\ {\isasymi}\ {\isacharasterisk}{\kern0pt}\ complex{\isacharunderscore}{\kern0pt}of{\isacharunderscore}{\kern0pt}real\ pi\ {\isacharasterisk}{\kern0pt}\isanewline
\ \ \ \ \ \ \ \ \ \ \ \ \ \ \ \ \ \ complex{\isacharunderscore}{\kern0pt}of{\isacharunderscore}{\kern0pt}nat\ {\isacharparenleft}{\kern0pt}j\ div\ {\isadigit{2}}{\isacharparenright}{\kern0pt}\ {\isacharslash}{\kern0pt}\ {\isadigit{2}}\ {\isacharcircum}{\kern0pt}\ n{\isacharparenright}{\kern0pt}\ {\isasymcdot}\isactrlsub m\ {\isacharbar}{\kern0pt}Deutsch{\isachardot}{\kern0pt}one{\isasymrangle}{\isacharparenright}{\kern0pt}\ {\isacharequal}{\kern0pt}\isanewline
\ \ \ \ \ \ \ \ \ \ \ \ \ \ \ \ \ \ Tensor{\isachardot}{\kern0pt}mat{\isacharunderscore}{\kern0pt}of{\isacharunderscore}{\kern0pt}cols{\isacharunderscore}{\kern0pt}list\ {\isadigit{2}}\ {\isacharbrackleft}{\kern0pt}{\isacharbrackleft}{\kern0pt}{\isadigit{1}}{\isacharcomma}{\kern0pt}\ {\isadigit{0}}{\isacharbrackright}{\kern0pt}{\isacharcomma}{\kern0pt}\ {\isacharbrackleft}{\kern0pt}{\isadigit{0}}{\isacharcomma}{\kern0pt}\ exp\ {\isacharparenleft}{\kern0pt}complex{\isacharunderscore}{\kern0pt}of{\isacharunderscore}{\kern0pt}real\ {\isacharparenleft}{\kern0pt}{\isadigit{2}}\ {\isacharasterisk}{\kern0pt}\ pi{\isacharparenright}{\kern0pt}\ {\isacharasterisk}{\kern0pt}\ {\isasymi}\ {\isacharslash}{\kern0pt}\isanewline
\ \ \ \ \ \ \ \ \ \ \ \ \ \ \ \ \ \ {\isadigit{2}}\ {\isacharcircum}{\kern0pt}\ Suc\ n{\isacharparenright}{\kern0pt}{\isacharbrackright}{\kern0pt}{\isacharbrackright}{\kern0pt}\ {\isacharasterisk}{\kern0pt}\ Tensor{\isachardot}{\kern0pt}mat{\isacharunderscore}{\kern0pt}of{\isacharunderscore}{\kern0pt}cols{\isacharunderscore}{\kern0pt}list\ {\isadigit{2}}\ {\isacharbrackleft}{\kern0pt}{\isacharbrackleft}{\kern0pt}{\isadigit{1}}{\isacharcomma}{\kern0pt}\ exp\ {\isacharparenleft}{\kern0pt}{\isadigit{2}}\ {\isacharasterisk}{\kern0pt}\ {\isasymi}\ {\isacharasterisk}{\kern0pt}\ complex{\isacharunderscore}{\kern0pt}of{\isacharunderscore}{\kern0pt}real\ pi\ {\isacharasterisk}{\kern0pt}\isanewline
\ \ \ \ \ \ \ \ \ \ \ \ \ \ \ \ \ \ complex{\isacharunderscore}{\kern0pt}of{\isacharunderscore}{\kern0pt}nat\ {\isacharparenleft}{\kern0pt}j\ div\ {\isadigit{2}}{\isacharparenright}{\kern0pt}\ {\isacharslash}{\kern0pt}\ {\isadigit{2}}\ {\isacharcircum}{\kern0pt}\ n{\isacharparenright}{\kern0pt}{\isacharbrackright}{\kern0pt}{\isacharbrackright}{\kern0pt}{\isachardoublequoteclose}\isanewline
\ \ \ \ \isacommand{by}\isamarkupfalse%
\ this\isanewline
\ \ \isacommand{show}\isamarkupfalse%
\ {\isachardoublequoteopen}{\isacharparenleft}{\kern0pt}R\ {\isacharparenleft}{\kern0pt}Suc\ n{\isacharparenright}{\kern0pt}\ {\isacharasterisk}{\kern0pt}\ {\isacharparenleft}{\kern0pt}\ {\isacharbar}{\kern0pt}Deutsch{\isachardot}{\kern0pt}zero{\isasymrangle}\ {\isacharplus}{\kern0pt}\ exp\ {\isacharparenleft}{\kern0pt}{\isadigit{2}}\ {\isacharasterisk}{\kern0pt}\ {\isasymi}\ {\isacharasterisk}{\kern0pt}\ pi\ {\isacharasterisk}{\kern0pt}\ \ complex{\isacharunderscore}{\kern0pt}of{\isacharunderscore}{\kern0pt}nat\ {\isacharparenleft}{\kern0pt}j\ div\ {\isadigit{2}}{\isacharparenright}{\kern0pt}\ {\isacharslash}{\kern0pt}\ \ {\isadigit{2}}\ {\isacharcircum}{\kern0pt}\ n{\isacharparenright}{\kern0pt}\ {\isasymcdot}\isactrlsub m\isanewline
\ \ \ \ \ \ \ \ {\isacharbar}{\kern0pt}Deutsch{\isachardot}{\kern0pt}one{\isasymrangle}{\isacharparenright}{\kern0pt}{\isacharparenright}{\kern0pt}\ {\isachardollar}{\kern0pt}{\isachardollar}{\kern0pt}\ {\isacharparenleft}{\kern0pt}i{\isacharcomma}{\kern0pt}\ ja{\isacharparenright}{\kern0pt}\ {\isacharequal}{\kern0pt}\isanewline
\ \ \ \ \ \ \ {\isacharparenleft}{\kern0pt}\ {\isacharbar}{\kern0pt}Deutsch{\isachardot}{\kern0pt}zero{\isasymrangle}\ {\isacharplus}{\kern0pt}\ exp\ {\isacharparenleft}{\kern0pt}{\isadigit{2}}\ {\isacharasterisk}{\kern0pt}\ {\isasymi}\ {\isacharasterisk}{\kern0pt}\ complex{\isacharunderscore}{\kern0pt}of{\isacharunderscore}{\kern0pt}real\ pi\ {\isacharasterisk}{\kern0pt}\ complex{\isacharunderscore}{\kern0pt}of{\isacharunderscore}{\kern0pt}nat\ j\ {\isacharslash}{\kern0pt}\ {\isadigit{2}}\ {\isacharcircum}{\kern0pt}\ Suc\ n{\isacharparenright}{\kern0pt}\ {\isasymcdot}\isactrlsub m\isanewline
\ \ \ \ \ \ \ \ {\isacharbar}{\kern0pt}Deutsch{\isachardot}{\kern0pt}one{\isasymrangle}{\isacharparenright}{\kern0pt}\ {\isachardollar}{\kern0pt}{\isachardollar}{\kern0pt}\ {\isacharparenleft}{\kern0pt}i{\isacharcomma}{\kern0pt}\ ja{\isacharparenright}{\kern0pt}{\isachardoublequoteclose}\isanewline
\ \ \isacommand{proof}\isamarkupfalse%
\ {\isacharminus}{\kern0pt}\isanewline
\ \ \ \ \isacommand{have}\isamarkupfalse%
\ {\isachardoublequoteopen}{\isacharparenleft}{\kern0pt}{\isacharparenleft}{\kern0pt}R\ {\isacharparenleft}{\kern0pt}Suc\ n{\isacharparenright}{\kern0pt}\ {\isacharasterisk}{\kern0pt}\ {\isacharparenleft}{\kern0pt}\ {\isacharbar}{\kern0pt}Deutsch{\isachardot}{\kern0pt}zero{\isasymrangle}\ {\isacharplus}{\kern0pt}\ exp\ {\isacharparenleft}{\kern0pt}{\isadigit{2}}\ {\isacharasterisk}{\kern0pt}\ {\isasymi}\ {\isacharasterisk}{\kern0pt}\ pi\ {\isacharasterisk}{\kern0pt}\ \ complex{\isacharunderscore}{\kern0pt}of{\isacharunderscore}{\kern0pt}nat\ {\isacharparenleft}{\kern0pt}j\ div\ {\isadigit{2}}{\isacharparenright}{\kern0pt}\ {\isacharslash}{\kern0pt}\ \ {\isadigit{2}}\ {\isacharcircum}{\kern0pt}\ n{\isacharparenright}{\kern0pt}\ {\isasymcdot}\isactrlsub m\isanewline
\ \ \ \ \ \ \ \ \ \ {\isacharbar}{\kern0pt}Deutsch{\isachardot}{\kern0pt}one{\isasymrangle}{\isacharparenright}{\kern0pt}{\isacharparenright}{\kern0pt}{\isacharparenright}{\kern0pt}\ {\isachardollar}{\kern0pt}{\isachardollar}{\kern0pt}\ {\isacharparenleft}{\kern0pt}i{\isacharcomma}{\kern0pt}\ ja{\isacharparenright}{\kern0pt}\ {\isacharequal}{\kern0pt}\ \isanewline
\ \ \ \ \ \ \ \ \ \ {\isacharparenleft}{\kern0pt}Tensor{\isachardot}{\kern0pt}mat{\isacharunderscore}{\kern0pt}of{\isacharunderscore}{\kern0pt}cols{\isacharunderscore}{\kern0pt}list\ {\isadigit{2}}\ {\isacharbrackleft}{\kern0pt}{\isacharbrackleft}{\kern0pt}{\isadigit{1}}{\isacharcomma}{\kern0pt}\ {\isadigit{0}}{\isacharbrackright}{\kern0pt}{\isacharcomma}{\kern0pt}\ {\isacharbrackleft}{\kern0pt}{\isadigit{0}}{\isacharcomma}{\kern0pt}\ exp\ {\isacharparenleft}{\kern0pt}complex{\isacharunderscore}{\kern0pt}of{\isacharunderscore}{\kern0pt}real\ {\isacharparenleft}{\kern0pt}{\isadigit{2}}\ {\isacharasterisk}{\kern0pt}\ pi{\isacharparenright}{\kern0pt}\ {\isacharasterisk}{\kern0pt}\ {\isasymi}\ {\isacharslash}{\kern0pt}\isanewline
\ \ \ \ \ \ \ \ \ \ \ \ \ \ \ \ \ \ {\isadigit{2}}\ {\isacharcircum}{\kern0pt}\ Suc\ n{\isacharparenright}{\kern0pt}{\isacharbrackright}{\kern0pt}{\isacharbrackright}{\kern0pt}\ {\isacharasterisk}{\kern0pt}\ Tensor{\isachardot}{\kern0pt}mat{\isacharunderscore}{\kern0pt}of{\isacharunderscore}{\kern0pt}cols{\isacharunderscore}{\kern0pt}list\ {\isadigit{2}}\ {\isacharbrackleft}{\kern0pt}{\isacharbrackleft}{\kern0pt}{\isadigit{1}}{\isacharcomma}{\kern0pt}\ exp\ {\isacharparenleft}{\kern0pt}{\isadigit{2}}\ {\isacharasterisk}{\kern0pt}\ {\isasymi}\ {\isacharasterisk}{\kern0pt}\ complex{\isacharunderscore}{\kern0pt}of{\isacharunderscore}{\kern0pt}real\ pi\ {\isacharasterisk}{\kern0pt}\isanewline
\ \ \ \ \ \ \ \ \ \ \ \ \ \ \ \ \ \ complex{\isacharunderscore}{\kern0pt}of{\isacharunderscore}{\kern0pt}nat\ {\isacharparenleft}{\kern0pt}j\ div\ {\isadigit{2}}{\isacharparenright}{\kern0pt}\ {\isacharslash}{\kern0pt}\ {\isadigit{2}}\ {\isacharcircum}{\kern0pt}\ n{\isacharparenright}{\kern0pt}{\isacharbrackright}{\kern0pt}{\isacharbrackright}{\kern0pt}{\isacharparenright}{\kern0pt}\ {\isachardollar}{\kern0pt}{\isachardollar}{\kern0pt}\ {\isacharparenleft}{\kern0pt}i{\isacharcomma}{\kern0pt}ja{\isacharparenright}{\kern0pt}{\isachardoublequoteclose}\isanewline
\ \ \ \ \ \ \isacommand{using}\isamarkupfalse%
\ {\isadigit{1}}\ \isacommand{by}\isamarkupfalse%
\ simp\isanewline
\ \ \ \ \isacommand{also}\isamarkupfalse%
\ \isacommand{have}\isamarkupfalse%
\ {\isachardoublequoteopen}{\isasymdots}\ {\isacharequal}{\kern0pt}\ mat{\isacharunderscore}{\kern0pt}of{\isacharunderscore}{\kern0pt}cols{\isacharunderscore}{\kern0pt}list\ {\isadigit{2}}\ {\isacharbrackleft}{\kern0pt}{\isacharbrackleft}{\kern0pt}{\isadigit{1}}{\isacharcomma}{\kern0pt}exp\ {\isacharparenleft}{\kern0pt}{\isadigit{2}}{\isacharasterisk}{\kern0pt}{\isasymi}{\isacharasterisk}{\kern0pt}pi{\isacharasterisk}{\kern0pt}complex{\isacharunderscore}{\kern0pt}of{\isacharunderscore}{\kern0pt}nat\ j\ {\isacharslash}{\kern0pt}\ {\isadigit{2}}{\isacharcircum}{\kern0pt}Suc\ n{\isacharparenright}{\kern0pt}{\isacharbrackright}{\kern0pt}{\isacharbrackright}{\kern0pt}\ {\isachardollar}{\kern0pt}{\isachardollar}{\kern0pt}\ {\isacharparenleft}{\kern0pt}i{\isacharcomma}{\kern0pt}ja{\isacharparenright}{\kern0pt}{\isachardoublequoteclose}\isanewline
\ \ \ \ \isacommand{proof}\isamarkupfalse%
\ {\isacharparenleft}{\kern0pt}rule\ disjE{\isacharparenright}{\kern0pt}\isanewline
\ \ \ \ \ \ \isacommand{show}\isamarkupfalse%
\ {\isachardoublequoteopen}i\ {\isacharequal}{\kern0pt}\ {\isadigit{0}}\ {\isasymor}\ i\ {\isacharequal}{\kern0pt}\ {\isadigit{1}}{\isachardoublequoteclose}\ \isacommand{using}\isamarkupfalse%
\ il{\isadigit{2}}\ \isacommand{by}\isamarkupfalse%
\ auto\isanewline
\ \ \ \ \isacommand{next}\isamarkupfalse%
\isanewline
\ \ \ \ \ \ \isacommand{assume}\isamarkupfalse%
\ i{\isadigit{0}}{\isacharcolon}{\kern0pt}{\isachardoublequoteopen}i\ {\isacharequal}{\kern0pt}\ {\isadigit{0}}{\isachardoublequoteclose}\isanewline
\ \ \ \ \ \ \isacommand{have}\isamarkupfalse%
\ {\isachardoublequoteopen}{\isacharparenleft}{\kern0pt}Tensor{\isachardot}{\kern0pt}mat{\isacharunderscore}{\kern0pt}of{\isacharunderscore}{\kern0pt}cols{\isacharunderscore}{\kern0pt}list\ {\isadigit{2}}\ {\isacharbrackleft}{\kern0pt}{\isacharbrackleft}{\kern0pt}{\isadigit{1}}{\isacharcomma}{\kern0pt}\ {\isadigit{0}}{\isacharbrackright}{\kern0pt}{\isacharcomma}{\kern0pt}{\isacharbrackleft}{\kern0pt}{\isadigit{0}}{\isacharcomma}{\kern0pt}\ exp\ {\isacharparenleft}{\kern0pt}complex{\isacharunderscore}{\kern0pt}of{\isacharunderscore}{\kern0pt}real\ {\isacharparenleft}{\kern0pt}{\isadigit{2}}\ {\isacharasterisk}{\kern0pt}\ pi{\isacharparenright}{\kern0pt}\ {\isacharasterisk}{\kern0pt}\ {\isasymi}\ {\isacharslash}{\kern0pt}\ {\isadigit{2}}\ {\isacharcircum}{\kern0pt}\ Suc\ n{\isacharparenright}{\kern0pt}{\isacharbrackright}{\kern0pt}{\isacharbrackright}{\kern0pt}\isanewline
\ \ \ \ \ \ \ \ \ \ \ \ \ {\isacharasterisk}{\kern0pt}\ Tensor{\isachardot}{\kern0pt}mat{\isacharunderscore}{\kern0pt}of{\isacharunderscore}{\kern0pt}cols{\isacharunderscore}{\kern0pt}list\ {\isadigit{2}}\ {\isacharbrackleft}{\kern0pt}{\isacharbrackleft}{\kern0pt}{\isadigit{1}}{\isacharcomma}{\kern0pt}\ exp\ {\isacharparenleft}{\kern0pt}{\isadigit{2}}\ {\isacharasterisk}{\kern0pt}\ {\isasymi}\ {\isacharasterisk}{\kern0pt}\ complex{\isacharunderscore}{\kern0pt}of{\isacharunderscore}{\kern0pt}real\ pi\ {\isacharasterisk}{\kern0pt}\ \isanewline
\ \ \ \ \ \ \ \ \ \ \ \ \ complex{\isacharunderscore}{\kern0pt}of{\isacharunderscore}{\kern0pt}nat\ {\isacharparenleft}{\kern0pt}j\ div\ {\isadigit{2}}{\isacharparenright}{\kern0pt}\ {\isacharslash}{\kern0pt}\ {\isadigit{2}}\ {\isacharcircum}{\kern0pt}\ n{\isacharparenright}{\kern0pt}{\isacharbrackright}{\kern0pt}{\isacharbrackright}{\kern0pt}{\isacharparenright}{\kern0pt}\ {\isachardollar}{\kern0pt}{\isachardollar}{\kern0pt}\ {\isacharparenleft}{\kern0pt}{\isadigit{0}}{\isacharcomma}{\kern0pt}\ {\isadigit{0}}{\isacharparenright}{\kern0pt}\ {\isacharequal}{\kern0pt}\ \isanewline
\ \ \ \ \ \ \ \ \ \ \ {\isacharparenleft}{\kern0pt}{\isasymSum}k{\isacharless}{\kern0pt}{\isadigit{2}}{\isachardot}{\kern0pt}\ {\isacharparenleft}{\kern0pt}mat{\isacharunderscore}{\kern0pt}of{\isacharunderscore}{\kern0pt}cols{\isacharunderscore}{\kern0pt}list\ {\isadigit{2}}\ {\isacharbrackleft}{\kern0pt}{\isacharbrackleft}{\kern0pt}{\isadigit{1}}{\isacharcomma}{\kern0pt}\ {\isadigit{0}}{\isacharbrackright}{\kern0pt}{\isacharcomma}{\kern0pt}{\isacharbrackleft}{\kern0pt}{\isadigit{0}}{\isacharcomma}{\kern0pt}\ exp\ {\isacharparenleft}{\kern0pt}complex{\isacharunderscore}{\kern0pt}of{\isacharunderscore}{\kern0pt}real\ {\isacharparenleft}{\kern0pt}{\isadigit{2}}\ {\isacharasterisk}{\kern0pt}\ pi{\isacharparenright}{\kern0pt}\ {\isacharasterisk}{\kern0pt}\ {\isasymi}\ {\isacharslash}{\kern0pt}\ {\isadigit{2}}\ {\isacharcircum}{\kern0pt}\ Suc\ n{\isacharparenright}{\kern0pt}{\isacharbrackright}{\kern0pt}{\isacharbrackright}{\kern0pt}{\isacharparenright}{\kern0pt}\isanewline
\ \ \ \ \ \ \ \ \ \ \ \ {\isachardollar}{\kern0pt}{\isachardollar}{\kern0pt}\ {\isacharparenleft}{\kern0pt}{\isadigit{0}}{\isacharcomma}{\kern0pt}k{\isacharparenright}{\kern0pt}\ {\isacharasterisk}{\kern0pt}\ {\isacharparenleft}{\kern0pt}mat{\isacharunderscore}{\kern0pt}of{\isacharunderscore}{\kern0pt}cols{\isacharunderscore}{\kern0pt}list\ {\isadigit{2}}\ {\isacharbrackleft}{\kern0pt}{\isacharbrackleft}{\kern0pt}{\isadigit{1}}{\isacharcomma}{\kern0pt}\ exp\ {\isacharparenleft}{\kern0pt}{\isadigit{2}}\ {\isacharasterisk}{\kern0pt}\ {\isasymi}\ {\isacharasterisk}{\kern0pt}\ complex{\isacharunderscore}{\kern0pt}of{\isacharunderscore}{\kern0pt}real\ pi\ {\isacharasterisk}{\kern0pt}\ \isanewline
\ \ \ \ \ \ \ \ \ \ \ \ \ complex{\isacharunderscore}{\kern0pt}of{\isacharunderscore}{\kern0pt}nat\ {\isacharparenleft}{\kern0pt}j\ div\ {\isadigit{2}}{\isacharparenright}{\kern0pt}\ {\isacharslash}{\kern0pt}\ {\isadigit{2}}\ {\isacharcircum}{\kern0pt}\ n{\isacharparenright}{\kern0pt}{\isacharbrackright}{\kern0pt}{\isacharbrackright}{\kern0pt}{\isacharparenright}{\kern0pt}\ {\isachardollar}{\kern0pt}{\isachardollar}{\kern0pt}\ {\isacharparenleft}{\kern0pt}k{\isacharcomma}{\kern0pt}{\isadigit{0}}{\isacharparenright}{\kern0pt}{\isacharparenright}{\kern0pt}{\isachardoublequoteclose}\isanewline
\ \ \ \ \ \ \ \ \isacommand{using}\isamarkupfalse%
\ index{\isacharunderscore}{\kern0pt}mult{\isacharunderscore}{\kern0pt}mat\ mat{\isacharunderscore}{\kern0pt}of{\isacharunderscore}{\kern0pt}cols{\isacharunderscore}{\kern0pt}list{\isacharunderscore}{\kern0pt}def\ \isacommand{by}\isamarkupfalse%
\ auto\isanewline
\ \ \ \ \ \ \isacommand{also}\isamarkupfalse%
\ \isacommand{have}\isamarkupfalse%
\ {\isachardoublequoteopen}{\isasymdots}\ {\isacharequal}{\kern0pt}\ {\isacharparenleft}{\kern0pt}mat{\isacharunderscore}{\kern0pt}of{\isacharunderscore}{\kern0pt}cols{\isacharunderscore}{\kern0pt}list\ {\isadigit{2}}\ {\isacharbrackleft}{\kern0pt}{\isacharbrackleft}{\kern0pt}{\isadigit{1}}{\isacharcomma}{\kern0pt}\ {\isadigit{0}}{\isacharbrackright}{\kern0pt}{\isacharcomma}{\kern0pt}{\isacharbrackleft}{\kern0pt}{\isadigit{0}}{\isacharcomma}{\kern0pt}\ exp\ {\isacharparenleft}{\kern0pt}complex{\isacharunderscore}{\kern0pt}of{\isacharunderscore}{\kern0pt}real\ {\isacharparenleft}{\kern0pt}{\isadigit{2}}\ {\isacharasterisk}{\kern0pt}\ pi{\isacharparenright}{\kern0pt}\ {\isacharasterisk}{\kern0pt}\ {\isasymi}\ {\isacharslash}{\kern0pt}\ {\isadigit{2}}\ {\isacharcircum}{\kern0pt}\ Suc\ n{\isacharparenright}{\kern0pt}{\isacharbrackright}{\kern0pt}{\isacharbrackright}{\kern0pt}{\isacharparenright}{\kern0pt}\isanewline
\ \ \ \ \ \ \ \ \ \ \ \ \ \ \ \ \ \ \ \ \ \ {\isachardollar}{\kern0pt}{\isachardollar}{\kern0pt}\ {\isacharparenleft}{\kern0pt}{\isadigit{0}}{\isacharcomma}{\kern0pt}{\isadigit{0}}{\isacharparenright}{\kern0pt}\ {\isacharasterisk}{\kern0pt}\ {\isacharparenleft}{\kern0pt}mat{\isacharunderscore}{\kern0pt}of{\isacharunderscore}{\kern0pt}cols{\isacharunderscore}{\kern0pt}list\ {\isadigit{2}}\ {\isacharbrackleft}{\kern0pt}{\isacharbrackleft}{\kern0pt}{\isadigit{1}}{\isacharcomma}{\kern0pt}\ exp\ {\isacharparenleft}{\kern0pt}{\isadigit{2}}\ {\isacharasterisk}{\kern0pt}\ {\isasymi}\ {\isacharasterisk}{\kern0pt}\ complex{\isacharunderscore}{\kern0pt}of{\isacharunderscore}{\kern0pt}real\ pi\ {\isacharasterisk}{\kern0pt}\ \isanewline
\ \ \ \ \ \ \ \ \ \ \ \ \ \ \ \ \ \ \ \ \ \ \ complex{\isacharunderscore}{\kern0pt}of{\isacharunderscore}{\kern0pt}nat\ {\isacharparenleft}{\kern0pt}j\ div\ {\isadigit{2}}{\isacharparenright}{\kern0pt}\ {\isacharslash}{\kern0pt}\ {\isadigit{2}}\ {\isacharcircum}{\kern0pt}\ n{\isacharparenright}{\kern0pt}{\isacharbrackright}{\kern0pt}{\isacharbrackright}{\kern0pt}{\isacharparenright}{\kern0pt}\ {\isachardollar}{\kern0pt}{\isachardollar}{\kern0pt}\ {\isacharparenleft}{\kern0pt}{\isadigit{0}}{\isacharcomma}{\kern0pt}{\isadigit{0}}{\isacharparenright}{\kern0pt}\ {\isacharplus}{\kern0pt}\isanewline
\ \ \ \ \ \ \ \ \ \ \ \ \ \ \ \ \ \ \ \ \ \ {\isacharparenleft}{\kern0pt}mat{\isacharunderscore}{\kern0pt}of{\isacharunderscore}{\kern0pt}cols{\isacharunderscore}{\kern0pt}list\ {\isadigit{2}}\ {\isacharbrackleft}{\kern0pt}{\isacharbrackleft}{\kern0pt}{\isadigit{1}}{\isacharcomma}{\kern0pt}\ {\isadigit{0}}{\isacharbrackright}{\kern0pt}{\isacharcomma}{\kern0pt}{\isacharbrackleft}{\kern0pt}{\isadigit{0}}{\isacharcomma}{\kern0pt}\ exp\ {\isacharparenleft}{\kern0pt}complex{\isacharunderscore}{\kern0pt}of{\isacharunderscore}{\kern0pt}real\ {\isacharparenleft}{\kern0pt}{\isadigit{2}}\ {\isacharasterisk}{\kern0pt}\ pi{\isacharparenright}{\kern0pt}\ {\isacharasterisk}{\kern0pt}\ {\isasymi}\ {\isacharslash}{\kern0pt}\ {\isadigit{2}}\ {\isacharcircum}{\kern0pt}\ Suc\ n{\isacharparenright}{\kern0pt}{\isacharbrackright}{\kern0pt}{\isacharbrackright}{\kern0pt}{\isacharparenright}{\kern0pt}\isanewline
\ \ \ \ \ \ \ \ \ \ \ \ \ \ \ \ \ \ \ \ \ \ {\isachardollar}{\kern0pt}{\isachardollar}{\kern0pt}\ {\isacharparenleft}{\kern0pt}{\isadigit{0}}{\isacharcomma}{\kern0pt}{\isadigit{1}}{\isacharparenright}{\kern0pt}\ {\isacharasterisk}{\kern0pt}\ {\isacharparenleft}{\kern0pt}mat{\isacharunderscore}{\kern0pt}of{\isacharunderscore}{\kern0pt}cols{\isacharunderscore}{\kern0pt}list\ {\isadigit{2}}\ {\isacharbrackleft}{\kern0pt}{\isacharbrackleft}{\kern0pt}{\isadigit{1}}{\isacharcomma}{\kern0pt}\ exp\ {\isacharparenleft}{\kern0pt}{\isadigit{2}}\ {\isacharasterisk}{\kern0pt}\ {\isasymi}\ {\isacharasterisk}{\kern0pt}\ complex{\isacharunderscore}{\kern0pt}of{\isacharunderscore}{\kern0pt}real\ pi\ {\isacharasterisk}{\kern0pt}\ \isanewline
\ \ \ \ \ \ \ \ \ \ \ \ \ \ \ \ \ \ \ \ \ \ \ complex{\isacharunderscore}{\kern0pt}of{\isacharunderscore}{\kern0pt}nat\ {\isacharparenleft}{\kern0pt}j\ div\ {\isadigit{2}}{\isacharparenright}{\kern0pt}\ {\isacharslash}{\kern0pt}\ {\isadigit{2}}\ {\isacharcircum}{\kern0pt}\ n{\isacharparenright}{\kern0pt}{\isacharbrackright}{\kern0pt}{\isacharbrackright}{\kern0pt}{\isacharparenright}{\kern0pt}\ {\isachardollar}{\kern0pt}{\isachardollar}{\kern0pt}\ {\isacharparenleft}{\kern0pt}{\isadigit{1}}{\isacharcomma}{\kern0pt}{\isadigit{0}}{\isacharparenright}{\kern0pt}{\isachardoublequoteclose}\isanewline
\ \ \ \ \ \ \ \ \isacommand{by}\isamarkupfalse%
\ {\isacharparenleft}{\kern0pt}simp\ only{\isacharcolon}{\kern0pt}sumof{\isadigit{2}}{\isacharparenright}{\kern0pt}\isanewline
\ \ \ \ \ \ \isacommand{also}\isamarkupfalse%
\ \isacommand{have}\isamarkupfalse%
\ {\isachardoublequoteopen}{\isasymdots}\ {\isacharequal}{\kern0pt}\ {\isadigit{1}}{\isachardoublequoteclose}\ \isacommand{by}\isamarkupfalse%
\ auto\isanewline
\ \ \ \ \ \ \isacommand{also}\isamarkupfalse%
\ \isacommand{have}\isamarkupfalse%
\ {\isachardoublequoteopen}{\isasymdots}\ {\isacharequal}{\kern0pt}\ mat{\isacharunderscore}{\kern0pt}of{\isacharunderscore}{\kern0pt}cols{\isacharunderscore}{\kern0pt}list\ {\isadigit{2}}\ {\isacharbrackleft}{\kern0pt}{\isacharbrackleft}{\kern0pt}{\isadigit{1}}{\isacharcomma}{\kern0pt}exp\ {\isacharparenleft}{\kern0pt}{\isadigit{2}}{\isacharasterisk}{\kern0pt}{\isasymi}{\isacharasterisk}{\kern0pt}pi{\isacharasterisk}{\kern0pt}complex{\isacharunderscore}{\kern0pt}of{\isacharunderscore}{\kern0pt}nat\ j\ {\isacharslash}{\kern0pt}\ {\isadigit{2}}{\isacharcircum}{\kern0pt}Suc\ n{\isacharparenright}{\kern0pt}{\isacharbrackright}{\kern0pt}{\isacharbrackright}{\kern0pt}\ {\isachardollar}{\kern0pt}{\isachardollar}{\kern0pt}\ {\isacharparenleft}{\kern0pt}{\isadigit{0}}{\isacharcomma}{\kern0pt}{\isadigit{0}}{\isacharparenright}{\kern0pt}{\isachardoublequoteclose}\isanewline
\ \ \ \ \ \ \ \ \isacommand{using}\isamarkupfalse%
\ index{\isacharunderscore}{\kern0pt}mat{\isacharunderscore}{\kern0pt}of{\isacharunderscore}{\kern0pt}cols{\isacharunderscore}{\kern0pt}list\ \isacommand{by}\isamarkupfalse%
\ simp\isanewline
\ \ \ \ \ \ \isacommand{finally}\isamarkupfalse%
\ \isacommand{show}\isamarkupfalse%
\ {\isachardoublequoteopen}{\isacharparenleft}{\kern0pt}Tensor{\isachardot}{\kern0pt}mat{\isacharunderscore}{\kern0pt}of{\isacharunderscore}{\kern0pt}cols{\isacharunderscore}{\kern0pt}list\ {\isadigit{2}}\ {\isacharbrackleft}{\kern0pt}{\isacharbrackleft}{\kern0pt}{\isadigit{1}}{\isacharcomma}{\kern0pt}\ {\isadigit{0}}{\isacharbrackright}{\kern0pt}{\isacharcomma}{\kern0pt}{\isacharbrackleft}{\kern0pt}{\isadigit{0}}{\isacharcomma}{\kern0pt}\ exp\ {\isacharparenleft}{\kern0pt}complex{\isacharunderscore}{\kern0pt}of{\isacharunderscore}{\kern0pt}real\ {\isacharparenleft}{\kern0pt}{\isadigit{2}}\ {\isacharasterisk}{\kern0pt}\ pi{\isacharparenright}{\kern0pt}\ {\isacharasterisk}{\kern0pt}\ {\isasymi}\ {\isacharslash}{\kern0pt}\ \isanewline
\ \ \ \ \ \ \ \ \ \ \ \ \ \ \ \ \ \ \ \ {\isadigit{2}}\ {\isacharcircum}{\kern0pt}\ Suc\ n{\isacharparenright}{\kern0pt}{\isacharbrackright}{\kern0pt}{\isacharbrackright}{\kern0pt}\ {\isacharasterisk}{\kern0pt}\ Tensor{\isachardot}{\kern0pt}mat{\isacharunderscore}{\kern0pt}of{\isacharunderscore}{\kern0pt}cols{\isacharunderscore}{\kern0pt}list\ {\isadigit{2}}\ {\isacharbrackleft}{\kern0pt}{\isacharbrackleft}{\kern0pt}{\isadigit{1}}{\isacharcomma}{\kern0pt}\ exp\ {\isacharparenleft}{\kern0pt}{\isadigit{2}}\ {\isacharasterisk}{\kern0pt}\ {\isasymi}\ {\isacharasterisk}{\kern0pt}\ complex{\isacharunderscore}{\kern0pt}of{\isacharunderscore}{\kern0pt}real\ pi\ {\isacharasterisk}{\kern0pt}\ \isanewline
\ \ \ \ \ \ \ \ \ \ \ \ \ \ \ \ \ \ \ \ complex{\isacharunderscore}{\kern0pt}of{\isacharunderscore}{\kern0pt}nat\ {\isacharparenleft}{\kern0pt}j\ div\ {\isadigit{2}}{\isacharparenright}{\kern0pt}\ {\isacharslash}{\kern0pt}\ {\isadigit{2}}\ {\isacharcircum}{\kern0pt}\ n{\isacharparenright}{\kern0pt}{\isacharbrackright}{\kern0pt}{\isacharbrackright}{\kern0pt}{\isacharparenright}{\kern0pt}\ {\isachardollar}{\kern0pt}{\isachardollar}{\kern0pt}\ {\isacharparenleft}{\kern0pt}i{\isacharcomma}{\kern0pt}\ ja{\isacharparenright}{\kern0pt}\ {\isacharequal}{\kern0pt}\ \isanewline
\ \ \ \ \ \ \ \ \ \ \ \ \ \ \ \ \ \ \ \ {\isacharparenleft}{\kern0pt}mat{\isacharunderscore}{\kern0pt}of{\isacharunderscore}{\kern0pt}cols{\isacharunderscore}{\kern0pt}list\ {\isadigit{2}}\ {\isacharbrackleft}{\kern0pt}{\isacharbrackleft}{\kern0pt}{\isadigit{1}}{\isacharcomma}{\kern0pt}exp\ {\isacharparenleft}{\kern0pt}{\isadigit{2}}{\isacharasterisk}{\kern0pt}{\isasymi}{\isacharasterisk}{\kern0pt}pi{\isacharasterisk}{\kern0pt}complex{\isacharunderscore}{\kern0pt}of{\isacharunderscore}{\kern0pt}nat\ j\ {\isacharslash}{\kern0pt}\ {\isadigit{2}}{\isacharcircum}{\kern0pt}Suc\ n{\isacharparenright}{\kern0pt}{\isacharbrackright}{\kern0pt}{\isacharbrackright}{\kern0pt}{\isacharparenright}{\kern0pt}\ {\isachardollar}{\kern0pt}{\isachardollar}{\kern0pt}\ {\isacharparenleft}{\kern0pt}i{\isacharcomma}{\kern0pt}ja{\isacharparenright}{\kern0pt}{\isachardoublequoteclose}\isanewline
\ \ \ \ \ \ \ \ \isacommand{using}\isamarkupfalse%
\ i{\isadigit{0}}\ ja{\isadigit{0}}\ \isacommand{by}\isamarkupfalse%
\ simp\isanewline
\ \ \ \ \isacommand{next}\isamarkupfalse%
\isanewline
\ \ \ \ \ \ \isacommand{assume}\isamarkupfalse%
\ i{\isadigit{1}}{\isacharcolon}{\kern0pt}{\isachardoublequoteopen}i\ {\isacharequal}{\kern0pt}\ {\isadigit{1}}{\isachardoublequoteclose}\isanewline
\ \ \ \ \ \ \isacommand{have}\isamarkupfalse%
\ {\isachardoublequoteopen}{\isacharparenleft}{\kern0pt}Tensor{\isachardot}{\kern0pt}mat{\isacharunderscore}{\kern0pt}of{\isacharunderscore}{\kern0pt}cols{\isacharunderscore}{\kern0pt}list\ {\isadigit{2}}\ {\isacharbrackleft}{\kern0pt}{\isacharbrackleft}{\kern0pt}{\isadigit{1}}{\isacharcomma}{\kern0pt}\ {\isadigit{0}}{\isacharbrackright}{\kern0pt}{\isacharcomma}{\kern0pt}{\isacharbrackleft}{\kern0pt}{\isadigit{0}}{\isacharcomma}{\kern0pt}\ exp\ {\isacharparenleft}{\kern0pt}complex{\isacharunderscore}{\kern0pt}of{\isacharunderscore}{\kern0pt}real\ {\isacharparenleft}{\kern0pt}{\isadigit{2}}\ {\isacharasterisk}{\kern0pt}\ pi{\isacharparenright}{\kern0pt}\ {\isacharasterisk}{\kern0pt}\ {\isasymi}\ {\isacharslash}{\kern0pt}\ {\isadigit{2}}\ {\isacharcircum}{\kern0pt}\ Suc\ n{\isacharparenright}{\kern0pt}{\isacharbrackright}{\kern0pt}{\isacharbrackright}{\kern0pt}\isanewline
\ \ \ \ \ \ \ \ \ \ \ \ \ {\isacharasterisk}{\kern0pt}\ Tensor{\isachardot}{\kern0pt}mat{\isacharunderscore}{\kern0pt}of{\isacharunderscore}{\kern0pt}cols{\isacharunderscore}{\kern0pt}list\ {\isadigit{2}}\ {\isacharbrackleft}{\kern0pt}{\isacharbrackleft}{\kern0pt}{\isadigit{1}}{\isacharcomma}{\kern0pt}\ exp\ {\isacharparenleft}{\kern0pt}{\isadigit{2}}\ {\isacharasterisk}{\kern0pt}\ {\isasymi}\ {\isacharasterisk}{\kern0pt}\ complex{\isacharunderscore}{\kern0pt}of{\isacharunderscore}{\kern0pt}real\ pi\ {\isacharasterisk}{\kern0pt}\ \isanewline
\ \ \ \ \ \ \ \ \ \ \ \ \ complex{\isacharunderscore}{\kern0pt}of{\isacharunderscore}{\kern0pt}nat\ {\isacharparenleft}{\kern0pt}j\ div\ {\isadigit{2}}{\isacharparenright}{\kern0pt}\ {\isacharslash}{\kern0pt}\ {\isadigit{2}}\ {\isacharcircum}{\kern0pt}\ n{\isacharparenright}{\kern0pt}{\isacharbrackright}{\kern0pt}{\isacharbrackright}{\kern0pt}{\isacharparenright}{\kern0pt}\ {\isachardollar}{\kern0pt}{\isachardollar}{\kern0pt}\ {\isacharparenleft}{\kern0pt}{\isadigit{1}}{\isacharcomma}{\kern0pt}\ {\isadigit{0}}{\isacharparenright}{\kern0pt}\ {\isacharequal}{\kern0pt}\ \isanewline
\ \ \ \ \ \ \ \ \ \ \ {\isacharparenleft}{\kern0pt}{\isasymSum}k{\isacharless}{\kern0pt}{\isadigit{2}}{\isachardot}{\kern0pt}\ {\isacharparenleft}{\kern0pt}mat{\isacharunderscore}{\kern0pt}of{\isacharunderscore}{\kern0pt}cols{\isacharunderscore}{\kern0pt}list\ {\isadigit{2}}\ {\isacharbrackleft}{\kern0pt}{\isacharbrackleft}{\kern0pt}{\isadigit{1}}{\isacharcomma}{\kern0pt}\ {\isadigit{0}}{\isacharbrackright}{\kern0pt}{\isacharcomma}{\kern0pt}{\isacharbrackleft}{\kern0pt}{\isadigit{0}}{\isacharcomma}{\kern0pt}\ exp\ {\isacharparenleft}{\kern0pt}complex{\isacharunderscore}{\kern0pt}of{\isacharunderscore}{\kern0pt}real\ {\isacharparenleft}{\kern0pt}{\isadigit{2}}\ {\isacharasterisk}{\kern0pt}\ pi{\isacharparenright}{\kern0pt}\ {\isacharasterisk}{\kern0pt}\ {\isasymi}\ {\isacharslash}{\kern0pt}\ {\isadigit{2}}\ {\isacharcircum}{\kern0pt}\ Suc\ n{\isacharparenright}{\kern0pt}{\isacharbrackright}{\kern0pt}{\isacharbrackright}{\kern0pt}{\isacharparenright}{\kern0pt}\isanewline
\ \ \ \ \ \ \ \ \ \ \ \ {\isachardollar}{\kern0pt}{\isachardollar}{\kern0pt}\ {\isacharparenleft}{\kern0pt}{\isadigit{1}}{\isacharcomma}{\kern0pt}k{\isacharparenright}{\kern0pt}\ {\isacharasterisk}{\kern0pt}\ {\isacharparenleft}{\kern0pt}mat{\isacharunderscore}{\kern0pt}of{\isacharunderscore}{\kern0pt}cols{\isacharunderscore}{\kern0pt}list\ {\isadigit{2}}\ {\isacharbrackleft}{\kern0pt}{\isacharbrackleft}{\kern0pt}{\isadigit{1}}{\isacharcomma}{\kern0pt}\ exp\ {\isacharparenleft}{\kern0pt}{\isadigit{2}}\ {\isacharasterisk}{\kern0pt}\ {\isasymi}\ {\isacharasterisk}{\kern0pt}\ complex{\isacharunderscore}{\kern0pt}of{\isacharunderscore}{\kern0pt}real\ pi\ {\isacharasterisk}{\kern0pt}\ \isanewline
\ \ \ \ \ \ \ \ \ \ \ \ \ complex{\isacharunderscore}{\kern0pt}of{\isacharunderscore}{\kern0pt}nat\ {\isacharparenleft}{\kern0pt}j\ div\ {\isadigit{2}}{\isacharparenright}{\kern0pt}\ {\isacharslash}{\kern0pt}\ {\isadigit{2}}\ {\isacharcircum}{\kern0pt}\ n{\isacharparenright}{\kern0pt}{\isacharbrackright}{\kern0pt}{\isacharbrackright}{\kern0pt}{\isacharparenright}{\kern0pt}\ {\isachardollar}{\kern0pt}{\isachardollar}{\kern0pt}\ {\isacharparenleft}{\kern0pt}k{\isacharcomma}{\kern0pt}{\isadigit{0}}{\isacharparenright}{\kern0pt}{\isacharparenright}{\kern0pt}{\isachardoublequoteclose}\isanewline
\ \ \ \ \ \ \ \ \isacommand{using}\isamarkupfalse%
\ index{\isacharunderscore}{\kern0pt}mult{\isacharunderscore}{\kern0pt}mat\ mat{\isacharunderscore}{\kern0pt}of{\isacharunderscore}{\kern0pt}cols{\isacharunderscore}{\kern0pt}list{\isacharunderscore}{\kern0pt}def\ \isacommand{by}\isamarkupfalse%
\ auto\isanewline
\ \ \ \ \ \ \isacommand{also}\isamarkupfalse%
\ \isacommand{have}\isamarkupfalse%
\ {\isachardoublequoteopen}{\isasymdots}\ {\isacharequal}{\kern0pt}\ {\isacharparenleft}{\kern0pt}mat{\isacharunderscore}{\kern0pt}of{\isacharunderscore}{\kern0pt}cols{\isacharunderscore}{\kern0pt}list\ {\isadigit{2}}\ {\isacharbrackleft}{\kern0pt}{\isacharbrackleft}{\kern0pt}{\isadigit{1}}{\isacharcomma}{\kern0pt}\ {\isadigit{0}}{\isacharbrackright}{\kern0pt}{\isacharcomma}{\kern0pt}{\isacharbrackleft}{\kern0pt}{\isadigit{0}}{\isacharcomma}{\kern0pt}\ exp\ {\isacharparenleft}{\kern0pt}complex{\isacharunderscore}{\kern0pt}of{\isacharunderscore}{\kern0pt}real\ {\isacharparenleft}{\kern0pt}{\isadigit{2}}\ {\isacharasterisk}{\kern0pt}\ pi{\isacharparenright}{\kern0pt}\ {\isacharasterisk}{\kern0pt}\ {\isasymi}\ {\isacharslash}{\kern0pt}\ {\isadigit{2}}\ {\isacharcircum}{\kern0pt}\ Suc\ n{\isacharparenright}{\kern0pt}{\isacharbrackright}{\kern0pt}{\isacharbrackright}{\kern0pt}{\isacharparenright}{\kern0pt}\isanewline
\ \ \ \ \ \ \ \ \ \ \ \ \ \ \ \ \ \ \ \ \ \ {\isachardollar}{\kern0pt}{\isachardollar}{\kern0pt}\ {\isacharparenleft}{\kern0pt}{\isadigit{1}}{\isacharcomma}{\kern0pt}{\isadigit{0}}{\isacharparenright}{\kern0pt}\ {\isacharasterisk}{\kern0pt}\ {\isacharparenleft}{\kern0pt}mat{\isacharunderscore}{\kern0pt}of{\isacharunderscore}{\kern0pt}cols{\isacharunderscore}{\kern0pt}list\ {\isadigit{2}}\ {\isacharbrackleft}{\kern0pt}{\isacharbrackleft}{\kern0pt}{\isadigit{1}}{\isacharcomma}{\kern0pt}\ exp\ {\isacharparenleft}{\kern0pt}{\isadigit{2}}\ {\isacharasterisk}{\kern0pt}\ {\isasymi}\ {\isacharasterisk}{\kern0pt}\ complex{\isacharunderscore}{\kern0pt}of{\isacharunderscore}{\kern0pt}real\ pi\ {\isacharasterisk}{\kern0pt}\ \isanewline
\ \ \ \ \ \ \ \ \ \ \ \ \ \ \ \ \ \ \ \ \ \ \ complex{\isacharunderscore}{\kern0pt}of{\isacharunderscore}{\kern0pt}nat\ {\isacharparenleft}{\kern0pt}j\ div\ {\isadigit{2}}{\isacharparenright}{\kern0pt}\ {\isacharslash}{\kern0pt}\ {\isadigit{2}}\ {\isacharcircum}{\kern0pt}\ n{\isacharparenright}{\kern0pt}{\isacharbrackright}{\kern0pt}{\isacharbrackright}{\kern0pt}{\isacharparenright}{\kern0pt}\ {\isachardollar}{\kern0pt}{\isachardollar}{\kern0pt}\ {\isacharparenleft}{\kern0pt}{\isadigit{0}}{\isacharcomma}{\kern0pt}{\isadigit{0}}{\isacharparenright}{\kern0pt}\ {\isacharplus}{\kern0pt}\isanewline
\ \ \ \ \ \ \ \ \ \ \ \ \ \ \ \ \ \ \ \ \ \ {\isacharparenleft}{\kern0pt}mat{\isacharunderscore}{\kern0pt}of{\isacharunderscore}{\kern0pt}cols{\isacharunderscore}{\kern0pt}list\ {\isadigit{2}}\ {\isacharbrackleft}{\kern0pt}{\isacharbrackleft}{\kern0pt}{\isadigit{1}}{\isacharcomma}{\kern0pt}\ {\isadigit{0}}{\isacharbrackright}{\kern0pt}{\isacharcomma}{\kern0pt}{\isacharbrackleft}{\kern0pt}{\isadigit{0}}{\isacharcomma}{\kern0pt}\ exp\ {\isacharparenleft}{\kern0pt}complex{\isacharunderscore}{\kern0pt}of{\isacharunderscore}{\kern0pt}real\ {\isacharparenleft}{\kern0pt}{\isadigit{2}}\ {\isacharasterisk}{\kern0pt}\ pi{\isacharparenright}{\kern0pt}\ {\isacharasterisk}{\kern0pt}\ {\isasymi}\ {\isacharslash}{\kern0pt}\ {\isadigit{2}}\ {\isacharcircum}{\kern0pt}\ Suc\ n{\isacharparenright}{\kern0pt}{\isacharbrackright}{\kern0pt}{\isacharbrackright}{\kern0pt}{\isacharparenright}{\kern0pt}\isanewline
\ \ \ \ \ \ \ \ \ \ \ \ \ \ \ \ \ \ \ \ \ \ {\isachardollar}{\kern0pt}{\isachardollar}{\kern0pt}\ {\isacharparenleft}{\kern0pt}{\isadigit{1}}{\isacharcomma}{\kern0pt}{\isadigit{1}}{\isacharparenright}{\kern0pt}\ {\isacharasterisk}{\kern0pt}\ {\isacharparenleft}{\kern0pt}mat{\isacharunderscore}{\kern0pt}of{\isacharunderscore}{\kern0pt}cols{\isacharunderscore}{\kern0pt}list\ {\isadigit{2}}\ {\isacharbrackleft}{\kern0pt}{\isacharbrackleft}{\kern0pt}{\isadigit{1}}{\isacharcomma}{\kern0pt}\ exp\ {\isacharparenleft}{\kern0pt}{\isadigit{2}}\ {\isacharasterisk}{\kern0pt}\ {\isasymi}\ {\isacharasterisk}{\kern0pt}\ complex{\isacharunderscore}{\kern0pt}of{\isacharunderscore}{\kern0pt}real\ pi\ {\isacharasterisk}{\kern0pt}\ \isanewline
\ \ \ \ \ \ \ \ \ \ \ \ \ \ \ \ \ \ \ \ \ \ \ complex{\isacharunderscore}{\kern0pt}of{\isacharunderscore}{\kern0pt}nat\ {\isacharparenleft}{\kern0pt}j\ div\ {\isadigit{2}}{\isacharparenright}{\kern0pt}\ {\isacharslash}{\kern0pt}\ {\isadigit{2}}\ {\isacharcircum}{\kern0pt}\ n{\isacharparenright}{\kern0pt}{\isacharbrackright}{\kern0pt}{\isacharbrackright}{\kern0pt}{\isacharparenright}{\kern0pt}\ {\isachardollar}{\kern0pt}{\isachardollar}{\kern0pt}\ {\isacharparenleft}{\kern0pt}{\isadigit{1}}{\isacharcomma}{\kern0pt}{\isadigit{0}}{\isacharparenright}{\kern0pt}{\isachardoublequoteclose}\isanewline
\ \ \ \ \ \ \ \ \isacommand{by}\isamarkupfalse%
\ {\isacharparenleft}{\kern0pt}simp\ only{\isacharcolon}{\kern0pt}\ sumof{\isadigit{2}}{\isacharparenright}{\kern0pt}\isanewline
\ \ \ \ \ \ \isacommand{also}\isamarkupfalse%
\ \isacommand{have}\isamarkupfalse%
\ {\isachardoublequoteopen}{\isasymdots}\ {\isacharequal}{\kern0pt}\ exp\ {\isacharparenleft}{\kern0pt}complex{\isacharunderscore}{\kern0pt}of{\isacharunderscore}{\kern0pt}real\ {\isacharparenleft}{\kern0pt}{\isadigit{2}}\ {\isacharasterisk}{\kern0pt}\ pi{\isacharparenright}{\kern0pt}\ {\isacharasterisk}{\kern0pt}\ {\isasymi}\ {\isacharslash}{\kern0pt}\ {\isadigit{2}}\ {\isacharcircum}{\kern0pt}\ Suc\ n{\isacharparenright}{\kern0pt}\ {\isacharasterisk}{\kern0pt}\isanewline
\ \ \ \ \ \ \ \ \ \ \ \ \ \ \ \ \ \ \ \ \ \ exp\ {\isacharparenleft}{\kern0pt}{\isadigit{2}}\ {\isacharasterisk}{\kern0pt}\ {\isasymi}\ {\isacharasterisk}{\kern0pt}\ complex{\isacharunderscore}{\kern0pt}of{\isacharunderscore}{\kern0pt}real\ pi\ {\isacharasterisk}{\kern0pt}\ complex{\isacharunderscore}{\kern0pt}of{\isacharunderscore}{\kern0pt}nat\ {\isacharparenleft}{\kern0pt}j\ div\ {\isadigit{2}}{\isacharparenright}{\kern0pt}\ {\isacharslash}{\kern0pt}\ {\isadigit{2}}\ {\isacharcircum}{\kern0pt}\ n{\isacharparenright}{\kern0pt}{\isachardoublequoteclose}\isanewline
\ \ \ \ \ \ \ \ \isacommand{using}\isamarkupfalse%
\ index{\isacharunderscore}{\kern0pt}mat{\isacharunderscore}{\kern0pt}of{\isacharunderscore}{\kern0pt}cols{\isacharunderscore}{\kern0pt}list\ \isacommand{by}\isamarkupfalse%
\ auto\isanewline
\ \ \ \ \ \ \isacommand{also}\isamarkupfalse%
\ \isacommand{have}\isamarkupfalse%
\ {\isachardoublequoteopen}{\isasymdots}\ {\isacharequal}{\kern0pt}\ exp\ {\isacharparenleft}{\kern0pt}complex{\isacharunderscore}{\kern0pt}of{\isacharunderscore}{\kern0pt}real\ {\isacharparenleft}{\kern0pt}{\isadigit{2}}\ {\isacharasterisk}{\kern0pt}\ pi{\isacharparenright}{\kern0pt}\ {\isacharasterisk}{\kern0pt}\ {\isasymi}\ {\isacharslash}{\kern0pt}\ {\isadigit{2}}\ {\isacharcircum}{\kern0pt}\ Suc\ n\ {\isacharplus}{\kern0pt}\isanewline
\ \ \ \ \ \ \ \ \ \ \ \ \ \ \ \ \ \ \ \ \ \ \ \ \ \ {\isadigit{2}}\ {\isacharasterisk}{\kern0pt}\ {\isasymi}\ {\isacharasterisk}{\kern0pt}\ complex{\isacharunderscore}{\kern0pt}of{\isacharunderscore}{\kern0pt}real\ pi\ {\isacharasterisk}{\kern0pt}\ complex{\isacharunderscore}{\kern0pt}of{\isacharunderscore}{\kern0pt}nat\ {\isacharparenleft}{\kern0pt}j\ div\ {\isadigit{2}}{\isacharparenright}{\kern0pt}\ {\isacharslash}{\kern0pt}\ {\isadigit{2}}\ {\isacharcircum}{\kern0pt}\ n{\isacharparenright}{\kern0pt}{\isachardoublequoteclose}\isanewline
\ \ \ \ \ \ \ \ \isacommand{using}\isamarkupfalse%
\ mult{\isacharunderscore}{\kern0pt}exp{\isacharunderscore}{\kern0pt}exp\ \isacommand{by}\isamarkupfalse%
\ simp\isanewline
\ \ \ \ \ \ \isacommand{also}\isamarkupfalse%
\ \isacommand{have}\isamarkupfalse%
\ {\isachardoublequoteopen}{\isasymdots}\ {\isacharequal}{\kern0pt}\ exp\ {\isacharparenleft}{\kern0pt}{\isadigit{2}}\ {\isacharasterisk}{\kern0pt}\ {\isasymi}\ {\isacharasterisk}{\kern0pt}\ pi\ {\isacharslash}{\kern0pt}\ {\isadigit{2}}\ {\isacharcircum}{\kern0pt}\ Suc\ n\ {\isacharplus}{\kern0pt}\isanewline
\ \ \ \ \ \ \ \ \ \ \ \ \ \ \ \ \ \ \ \ \ \ \ \ \ \ {\isadigit{2}}\ {\isacharasterisk}{\kern0pt}\ {\isasymi}\ {\isacharasterisk}{\kern0pt}\ complex{\isacharunderscore}{\kern0pt}of{\isacharunderscore}{\kern0pt}real\ pi\ {\isacharasterisk}{\kern0pt}\ complex{\isacharunderscore}{\kern0pt}of{\isacharunderscore}{\kern0pt}nat\ {\isacharparenleft}{\kern0pt}j\ div\ {\isadigit{2}}{\isacharparenright}{\kern0pt}\ {\isacharslash}{\kern0pt}\ {\isadigit{2}}\ {\isacharcircum}{\kern0pt}\ n{\isacharparenright}{\kern0pt}{\isachardoublequoteclose}\isanewline
\ \ \ \ \ \ \ \ \isacommand{by}\isamarkupfalse%
\ {\isacharparenleft}{\kern0pt}simp\ add{\isacharcolon}{\kern0pt}\ mult{\isachardot}{\kern0pt}commute{\isacharparenright}{\kern0pt}\isanewline
\ \ \ \ \ \ \isacommand{also}\isamarkupfalse%
\ \isacommand{have}\isamarkupfalse%
\ {\isachardoublequoteopen}{\isasymdots}\ {\isacharequal}{\kern0pt}\ exp\ {\isacharparenleft}{\kern0pt}{\isadigit{2}}{\isacharasterisk}{\kern0pt}{\isasymi}{\isacharasterisk}{\kern0pt}pi{\isacharasterisk}{\kern0pt}{\isacharparenleft}{\kern0pt}{\isadigit{1}}{\isacharslash}{\kern0pt}{\isadigit{2}}{\isacharcircum}{\kern0pt}Suc\ n\ {\isacharplus}{\kern0pt}\ complex{\isacharunderscore}{\kern0pt}of{\isacharunderscore}{\kern0pt}nat\ {\isacharparenleft}{\kern0pt}j\ div\ {\isadigit{2}}{\isacharparenright}{\kern0pt}{\isacharslash}{\kern0pt}{\isadigit{2}}{\isacharcircum}{\kern0pt}n{\isacharparenright}{\kern0pt}{\isacharparenright}{\kern0pt}{\isachardoublequoteclose}\isanewline
\ \ \ \ \ \ \ \ \isacommand{by}\isamarkupfalse%
\ {\isacharparenleft}{\kern0pt}simp\ add{\isacharcolon}{\kern0pt}\ distrib{\isacharunderscore}{\kern0pt}left{\isacharparenright}{\kern0pt}\isanewline
\ \ \ \ \ \ \isacommand{also}\isamarkupfalse%
\ \isacommand{have}\isamarkupfalse%
\ {\isachardoublequoteopen}{\isasymdots}\ {\isacharequal}{\kern0pt}\ exp\ {\isacharparenleft}{\kern0pt}{\isadigit{2}}{\isacharasterisk}{\kern0pt}{\isasymi}{\isacharasterisk}{\kern0pt}pi{\isacharasterisk}{\kern0pt}{\isacharparenleft}{\kern0pt}{\isacharparenleft}{\kern0pt}{\isadigit{1}}\ {\isacharplus}{\kern0pt}\ {\isadigit{2}}{\isacharasterisk}{\kern0pt}{\isacharparenleft}{\kern0pt}j\ div\ {\isadigit{2}}{\isacharparenright}{\kern0pt}{\isacharparenright}{\kern0pt}{\isacharslash}{\kern0pt}{\isadigit{2}}{\isacharcircum}{\kern0pt}Suc\ n{\isacharparenright}{\kern0pt}{\isacharparenright}{\kern0pt}{\isachardoublequoteclose}\ \isanewline
\ \ \ \ \ \ \ \ \isacommand{by}\isamarkupfalse%
\ {\isacharparenleft}{\kern0pt}simp\ add{\isacharcolon}{\kern0pt}\ add{\isacharunderscore}{\kern0pt}divide{\isacharunderscore}{\kern0pt}distrib{\isacharparenright}{\kern0pt}\isanewline
\ \ \ \ \ \ \isacommand{also}\isamarkupfalse%
\ \isacommand{have}\isamarkupfalse%
\ {\isachardoublequoteopen}{\isasymdots}\ {\isacharequal}{\kern0pt}\ exp\ {\isacharparenleft}{\kern0pt}{\isadigit{2}}{\isacharasterisk}{\kern0pt}{\isasymi}{\isacharasterisk}{\kern0pt}pi{\isacharasterisk}{\kern0pt}{\isacharparenleft}{\kern0pt}j{\isacharparenright}{\kern0pt}{\isacharslash}{\kern0pt}{\isadigit{2}}{\isacharcircum}{\kern0pt}Suc\ n{\isacharparenright}{\kern0pt}{\isachardoublequoteclose}\isanewline
\ \ \ \ \ \ \ \ \isacommand{using}\isamarkupfalse%
\ assms\isanewline
\ \ \ \ \ \ \ \ \isacommand{by}\isamarkupfalse%
\ {\isacharparenleft}{\kern0pt}smt\ {\isacharparenleft}{\kern0pt}verit{\isacharcomma}{\kern0pt}\ ccfv{\isacharunderscore}{\kern0pt}threshold{\isacharparenright}{\kern0pt}\ Suc{\isacharunderscore}{\kern0pt}eq{\isacharunderscore}{\kern0pt}plus{\isadigit{1}}\ div{\isacharunderscore}{\kern0pt}mult{\isacharunderscore}{\kern0pt}mod{\isacharunderscore}{\kern0pt}eq\ mult{\isachardot}{\kern0pt}commute\ of{\isacharunderscore}{\kern0pt}real{\isacharunderscore}{\kern0pt}{\isadigit{1}}\ \isanewline
\ \ \ \ \ \ \ \ \ \ \ \ of{\isacharunderscore}{\kern0pt}real{\isacharunderscore}{\kern0pt}add\ of{\isacharunderscore}{\kern0pt}real{\isacharunderscore}{\kern0pt}divide\ of{\isacharunderscore}{\kern0pt}real{\isacharunderscore}{\kern0pt}of{\isacharunderscore}{\kern0pt}nat{\isacharunderscore}{\kern0pt}eq\ of{\isacharunderscore}{\kern0pt}real{\isacharunderscore}{\kern0pt}power\ one{\isacharunderscore}{\kern0pt}add{\isacharunderscore}{\kern0pt}one\ plus{\isacharunderscore}{\kern0pt}{\isadigit{1}}{\isacharunderscore}{\kern0pt}eq{\isacharunderscore}{\kern0pt}Suc\ \isanewline
\ \ \ \ \ \ \ \ \ \ \ \ times{\isacharunderscore}{\kern0pt}divide{\isacharunderscore}{\kern0pt}eq{\isacharunderscore}{\kern0pt}right{\isacharparenright}{\kern0pt}\isanewline
\ \ \ \ \ \ \isacommand{also}\isamarkupfalse%
\ \isacommand{have}\isamarkupfalse%
\ {\isachardoublequoteopen}{\isasymdots}\ {\isacharequal}{\kern0pt}\ {\isacharparenleft}{\kern0pt}mat{\isacharunderscore}{\kern0pt}of{\isacharunderscore}{\kern0pt}cols{\isacharunderscore}{\kern0pt}list\ {\isadigit{2}}\ {\isacharbrackleft}{\kern0pt}{\isacharbrackleft}{\kern0pt}{\isadigit{1}}{\isacharcomma}{\kern0pt}exp\ {\isacharparenleft}{\kern0pt}{\isadigit{2}}{\isacharasterisk}{\kern0pt}{\isasymi}{\isacharasterisk}{\kern0pt}pi{\isacharasterisk}{\kern0pt}complex{\isacharunderscore}{\kern0pt}of{\isacharunderscore}{\kern0pt}nat\ j\ {\isacharslash}{\kern0pt}\ {\isadigit{2}}{\isacharcircum}{\kern0pt}Suc\ n{\isacharparenright}{\kern0pt}{\isacharbrackright}{\kern0pt}{\isacharbrackright}{\kern0pt}{\isacharparenright}{\kern0pt}\ {\isachardollar}{\kern0pt}{\isachardollar}{\kern0pt}\ {\isacharparenleft}{\kern0pt}{\isadigit{1}}{\isacharcomma}{\kern0pt}{\isadigit{0}}{\isacharparenright}{\kern0pt}{\isachardoublequoteclose}\isanewline
\ \ \ \ \ \ \ \ \isacommand{using}\isamarkupfalse%
\ index{\isacharunderscore}{\kern0pt}mat{\isacharunderscore}{\kern0pt}of{\isacharunderscore}{\kern0pt}cols{\isacharunderscore}{\kern0pt}list\ \isacommand{by}\isamarkupfalse%
\ simp\isanewline
\ \ \ \ \ \ \isacommand{finally}\isamarkupfalse%
\ \isacommand{show}\isamarkupfalse%
\ {\isachardoublequoteopen}{\isacharparenleft}{\kern0pt}Tensor{\isachardot}{\kern0pt}mat{\isacharunderscore}{\kern0pt}of{\isacharunderscore}{\kern0pt}cols{\isacharunderscore}{\kern0pt}list\ {\isadigit{2}}\ {\isacharbrackleft}{\kern0pt}{\isacharbrackleft}{\kern0pt}{\isadigit{1}}{\isacharcomma}{\kern0pt}\ {\isadigit{0}}{\isacharbrackright}{\kern0pt}{\isacharcomma}{\kern0pt}{\isacharbrackleft}{\kern0pt}{\isadigit{0}}{\isacharcomma}{\kern0pt}\ exp\ {\isacharparenleft}{\kern0pt}complex{\isacharunderscore}{\kern0pt}of{\isacharunderscore}{\kern0pt}real\ {\isacharparenleft}{\kern0pt}{\isadigit{2}}\ {\isacharasterisk}{\kern0pt}\ pi{\isacharparenright}{\kern0pt}\ {\isacharasterisk}{\kern0pt}\ {\isasymi}\ {\isacharslash}{\kern0pt}\ \isanewline
\ \ \ \ \ \ \ \ \ \ \ \ \ \ \ \ \ \ \ \ {\isadigit{2}}\ {\isacharcircum}{\kern0pt}\ Suc\ n{\isacharparenright}{\kern0pt}{\isacharbrackright}{\kern0pt}{\isacharbrackright}{\kern0pt}\ {\isacharasterisk}{\kern0pt}\ Tensor{\isachardot}{\kern0pt}mat{\isacharunderscore}{\kern0pt}of{\isacharunderscore}{\kern0pt}cols{\isacharunderscore}{\kern0pt}list\ {\isadigit{2}}\ {\isacharbrackleft}{\kern0pt}{\isacharbrackleft}{\kern0pt}{\isadigit{1}}{\isacharcomma}{\kern0pt}\ exp\ {\isacharparenleft}{\kern0pt}{\isadigit{2}}\ {\isacharasterisk}{\kern0pt}\ {\isasymi}\ {\isacharasterisk}{\kern0pt}\ complex{\isacharunderscore}{\kern0pt}of{\isacharunderscore}{\kern0pt}real\ pi\ {\isacharasterisk}{\kern0pt}\ \isanewline
\ \ \ \ \ \ \ \ \ \ \ \ \ \ \ \ \ \ \ \ complex{\isacharunderscore}{\kern0pt}of{\isacharunderscore}{\kern0pt}nat\ {\isacharparenleft}{\kern0pt}j\ div\ {\isadigit{2}}{\isacharparenright}{\kern0pt}\ {\isacharslash}{\kern0pt}\ {\isadigit{2}}\ {\isacharcircum}{\kern0pt}\ n{\isacharparenright}{\kern0pt}{\isacharbrackright}{\kern0pt}{\isacharbrackright}{\kern0pt}{\isacharparenright}{\kern0pt}\ {\isachardollar}{\kern0pt}{\isachardollar}{\kern0pt}\ {\isacharparenleft}{\kern0pt}i{\isacharcomma}{\kern0pt}\ ja{\isacharparenright}{\kern0pt}\ {\isacharequal}{\kern0pt}\ \isanewline
\ \ \ \ \ \ \ \ \ \ \ \ \ \ \ \ \ \ \ \ {\isacharparenleft}{\kern0pt}mat{\isacharunderscore}{\kern0pt}of{\isacharunderscore}{\kern0pt}cols{\isacharunderscore}{\kern0pt}list\ {\isadigit{2}}\ {\isacharbrackleft}{\kern0pt}{\isacharbrackleft}{\kern0pt}{\isadigit{1}}{\isacharcomma}{\kern0pt}exp\ {\isacharparenleft}{\kern0pt}{\isadigit{2}}{\isacharasterisk}{\kern0pt}{\isasymi}{\isacharasterisk}{\kern0pt}pi{\isacharasterisk}{\kern0pt}complex{\isacharunderscore}{\kern0pt}of{\isacharunderscore}{\kern0pt}nat\ j\ {\isacharslash}{\kern0pt}\ {\isadigit{2}}{\isacharcircum}{\kern0pt}Suc\ n{\isacharparenright}{\kern0pt}{\isacharbrackright}{\kern0pt}{\isacharbrackright}{\kern0pt}{\isacharparenright}{\kern0pt}\ {\isachardollar}{\kern0pt}{\isachardollar}{\kern0pt}\ {\isacharparenleft}{\kern0pt}i{\isacharcomma}{\kern0pt}ja{\isacharparenright}{\kern0pt}{\isachardoublequoteclose}\isanewline
\ \ \ \ \ \ \ \ \isacommand{using}\isamarkupfalse%
\ i{\isadigit{1}}\ ja{\isadigit{0}}\ \isacommand{by}\isamarkupfalse%
\ simp\isanewline
\ \ \ \ \isacommand{qed}\isamarkupfalse%
\isanewline
\ \ \ \ \isacommand{also}\isamarkupfalse%
\ \isacommand{have}\isamarkupfalse%
\ {\isachardoublequoteopen}{\isasymdots}\ {\isacharequal}{\kern0pt}\ {\isacharparenleft}{\kern0pt}\ {\isacharbar}{\kern0pt}zero{\isasymrangle}\ {\isacharplus}{\kern0pt}\ exp\ {\isacharparenleft}{\kern0pt}{\isadigit{2}}{\isacharasterisk}{\kern0pt}{\isasymi}{\isacharasterisk}{\kern0pt}pi{\isacharasterisk}{\kern0pt}complex{\isacharunderscore}{\kern0pt}of{\isacharunderscore}{\kern0pt}nat\ j\ {\isacharslash}{\kern0pt}\ {\isadigit{2}}{\isacharcircum}{\kern0pt}Suc\ n{\isacharparenright}{\kern0pt}\ {\isasymcdot}\isactrlsub m\ {\isacharbar}{\kern0pt}one{\isasymrangle}{\isacharparenright}{\kern0pt}\ {\isachardollar}{\kern0pt}{\isachardollar}{\kern0pt}\ {\isacharparenleft}{\kern0pt}i{\isacharcomma}{\kern0pt}ja{\isacharparenright}{\kern0pt}{\isachardoublequoteclose}\isanewline
\ \ \ \ \isacommand{proof}\isamarkupfalse%
\ {\isacharparenleft}{\kern0pt}rule\ disjE{\isacharparenright}{\kern0pt}\isanewline
\ \ \ \ \ \ \isacommand{show}\isamarkupfalse%
\ {\isachardoublequoteopen}i\ {\isacharequal}{\kern0pt}\ {\isadigit{0}}\ {\isasymor}\ i\ {\isacharequal}{\kern0pt}\ {\isadigit{1}}{\isachardoublequoteclose}\ \isacommand{using}\isamarkupfalse%
\ il{\isadigit{2}}\ \isacommand{by}\isamarkupfalse%
\ auto\isanewline
\ \ \ \ \isacommand{next}\isamarkupfalse%
\isanewline
\ \ \ \ \ \ \isacommand{assume}\isamarkupfalse%
\ i{\isadigit{0}}{\isacharcolon}{\kern0pt}{\isachardoublequoteopen}i\ {\isacharequal}{\kern0pt}\ {\isadigit{0}}{\isachardoublequoteclose}\isanewline
\ \ \ \ \ \ \isacommand{have}\isamarkupfalse%
\ {\isachardoublequoteopen}{\isacharparenleft}{\kern0pt}mat{\isacharunderscore}{\kern0pt}of{\isacharunderscore}{\kern0pt}cols{\isacharunderscore}{\kern0pt}list\ {\isadigit{2}}\ {\isacharbrackleft}{\kern0pt}{\isacharbrackleft}{\kern0pt}{\isadigit{1}}{\isacharcomma}{\kern0pt}exp\ {\isacharparenleft}{\kern0pt}{\isadigit{2}}{\isacharasterisk}{\kern0pt}{\isasymi}{\isacharasterisk}{\kern0pt}pi{\isacharasterisk}{\kern0pt}complex{\isacharunderscore}{\kern0pt}of{\isacharunderscore}{\kern0pt}nat\ j\ {\isacharslash}{\kern0pt}\ {\isadigit{2}}{\isacharcircum}{\kern0pt}Suc\ n{\isacharparenright}{\kern0pt}{\isacharbrackright}{\kern0pt}{\isacharbrackright}{\kern0pt}{\isacharparenright}{\kern0pt}\ {\isachardollar}{\kern0pt}{\isachardollar}{\kern0pt}\ {\isacharparenleft}{\kern0pt}{\isadigit{0}}{\isacharcomma}{\kern0pt}{\isadigit{0}}{\isacharparenright}{\kern0pt}\ {\isacharequal}{\kern0pt}\ {\isadigit{1}}{\isachardoublequoteclose}\isanewline
\ \ \ \ \ \ \ \ \isacommand{by}\isamarkupfalse%
\ auto\isanewline
\ \ \ \ \ \ \isacommand{also}\isamarkupfalse%
\ \isacommand{have}\isamarkupfalse%
\ {\isachardoublequoteopen}{\isasymdots}\ {\isacharequal}{\kern0pt}\ {\isacharparenleft}{\kern0pt}\ {\isacharbar}{\kern0pt}zero{\isasymrangle}\ {\isacharplus}{\kern0pt}\ exp\ {\isacharparenleft}{\kern0pt}{\isadigit{2}}{\isacharasterisk}{\kern0pt}{\isasymi}{\isacharasterisk}{\kern0pt}pi{\isacharasterisk}{\kern0pt}complex{\isacharunderscore}{\kern0pt}of{\isacharunderscore}{\kern0pt}nat\ j\ {\isacharslash}{\kern0pt}\ {\isadigit{2}}{\isacharcircum}{\kern0pt}Suc\ n{\isacharparenright}{\kern0pt}\ {\isasymcdot}\isactrlsub m\ {\isacharbar}{\kern0pt}one{\isasymrangle}{\isacharparenright}{\kern0pt}\ {\isachardollar}{\kern0pt}{\isachardollar}{\kern0pt}\ {\isacharparenleft}{\kern0pt}{\isadigit{0}}{\isacharcomma}{\kern0pt}{\isadigit{0}}{\isacharparenright}{\kern0pt}{\isachardoublequoteclose}\isanewline
\ \ \ \ \ \ \isacommand{proof}\isamarkupfalse%
\ {\isacharminus}{\kern0pt}\isanewline
\ \ \ \ \ \ \ \ \isacommand{have}\isamarkupfalse%
\ {\isachardoublequoteopen}{\isacharbar}{\kern0pt}zero{\isasymrangle}\ {\isachardollar}{\kern0pt}{\isachardollar}{\kern0pt}\ {\isacharparenleft}{\kern0pt}{\isadigit{0}}{\isacharcomma}{\kern0pt}{\isadigit{0}}{\isacharparenright}{\kern0pt}\ {\isacharequal}{\kern0pt}\ {\isadigit{1}}{\isachardoublequoteclose}\ \isacommand{by}\isamarkupfalse%
\ auto\isanewline
\ \ \ \ \ \ \ \ \isacommand{moreover}\isamarkupfalse%
\ \isacommand{have}\isamarkupfalse%
\ {\isachardoublequoteopen}{\isacharparenleft}{\kern0pt}exp\ {\isacharparenleft}{\kern0pt}{\isadigit{2}}{\isacharasterisk}{\kern0pt}{\isasymi}{\isacharasterisk}{\kern0pt}pi{\isacharasterisk}{\kern0pt}complex{\isacharunderscore}{\kern0pt}of{\isacharunderscore}{\kern0pt}nat\ j\ {\isacharslash}{\kern0pt}\ {\isadigit{2}}{\isacharcircum}{\kern0pt}Suc\ n{\isacharparenright}{\kern0pt}\ {\isasymcdot}\isactrlsub m\ {\isacharbar}{\kern0pt}one{\isasymrangle}{\isacharparenright}{\kern0pt}\ {\isachardollar}{\kern0pt}{\isachardollar}{\kern0pt}\ {\isacharparenleft}{\kern0pt}{\isadigit{0}}{\isacharcomma}{\kern0pt}{\isadigit{0}}{\isacharparenright}{\kern0pt}\ {\isacharequal}{\kern0pt}\ {\isadigit{0}}{\isachardoublequoteclose}\isanewline
\ \ \ \ \ \ \ \ \isacommand{proof}\isamarkupfalse%
\ {\isacharminus}{\kern0pt}\isanewline
\ \ \ \ \ \ \ \ \ \ \isacommand{have}\isamarkupfalse%
\ {\isachardoublequoteopen}{\isacharparenleft}{\kern0pt}exp\ {\isacharparenleft}{\kern0pt}{\isadigit{2}}{\isacharasterisk}{\kern0pt}{\isasymi}{\isacharasterisk}{\kern0pt}pi{\isacharasterisk}{\kern0pt}complex{\isacharunderscore}{\kern0pt}of{\isacharunderscore}{\kern0pt}nat\ j\ {\isacharslash}{\kern0pt}\ {\isadigit{2}}{\isacharcircum}{\kern0pt}Suc\ n{\isacharparenright}{\kern0pt}\ {\isasymcdot}\isactrlsub m\ {\isacharbar}{\kern0pt}one{\isasymrangle}{\isacharparenright}{\kern0pt}\ {\isachardollar}{\kern0pt}{\isachardollar}{\kern0pt}\ {\isacharparenleft}{\kern0pt}{\isadigit{0}}{\isacharcomma}{\kern0pt}{\isadigit{0}}{\isacharparenright}{\kern0pt}\ {\isacharequal}{\kern0pt}\isanewline
\ \ \ \ \ \ \ \ \ \ \ \ \ \ \ \ exp\ {\isacharparenleft}{\kern0pt}{\isadigit{2}}{\isacharasterisk}{\kern0pt}{\isasymi}{\isacharasterisk}{\kern0pt}pi{\isacharasterisk}{\kern0pt}complex{\isacharunderscore}{\kern0pt}of{\isacharunderscore}{\kern0pt}nat\ j\ {\isacharslash}{\kern0pt}\ {\isadigit{2}}{\isacharcircum}{\kern0pt}Suc\ n{\isacharparenright}{\kern0pt}\ {\isacharasterisk}{\kern0pt}\ {\isacharbar}{\kern0pt}one{\isasymrangle}\ {\isachardollar}{\kern0pt}{\isachardollar}{\kern0pt}\ {\isacharparenleft}{\kern0pt}{\isadigit{0}}{\isacharcomma}{\kern0pt}{\isadigit{0}}{\isacharparenright}{\kern0pt}{\isachardoublequoteclose}\isanewline
\ \ \ \ \ \ \ \ \ \ \ \ \isacommand{using}\isamarkupfalse%
\ index{\isacharunderscore}{\kern0pt}smult{\isacharunderscore}{\kern0pt}mat\ \isacommand{using}\isamarkupfalse%
\ ket{\isacharunderscore}{\kern0pt}one{\isacharunderscore}{\kern0pt}is{\isacharunderscore}{\kern0pt}state\ state{\isacharunderscore}{\kern0pt}def\ \isacommand{by}\isamarkupfalse%
\ auto\isanewline
\ \ \ \ \ \ \ \ \ \ \isacommand{also}\isamarkupfalse%
\ \isacommand{have}\isamarkupfalse%
\ {\isachardoublequoteopen}{\isasymdots}\ {\isacharequal}{\kern0pt}\ {\isadigit{0}}{\isachardoublequoteclose}\ \isacommand{by}\isamarkupfalse%
\ auto\isanewline
\ \ \ \ \ \ \ \ \ \ \isacommand{finally}\isamarkupfalse%
\ \isacommand{show}\isamarkupfalse%
\ {\isacharquery}{\kern0pt}thesis\ \isacommand{by}\isamarkupfalse%
\ this\isanewline
\ \ \ \ \ \ \ \ \isacommand{qed}\isamarkupfalse%
\isanewline
\ \ \ \ \ \ \ \ \isacommand{ultimately}\isamarkupfalse%
\ \isacommand{show}\isamarkupfalse%
\ {\isacharquery}{\kern0pt}thesis\ \isacommand{by}\isamarkupfalse%
\ {\isacharparenleft}{\kern0pt}simp\ add{\isacharcolon}{\kern0pt}\ ket{\isacharunderscore}{\kern0pt}vec{\isacharunderscore}{\kern0pt}def{\isacharparenright}{\kern0pt}\isanewline
\ \ \ \ \ \ \isacommand{qed}\isamarkupfalse%
\isanewline
\ \ \ \ \ \ \isacommand{finally}\isamarkupfalse%
\ \isacommand{show}\isamarkupfalse%
\ {\isachardoublequoteopen}{\isacharparenleft}{\kern0pt}mat{\isacharunderscore}{\kern0pt}of{\isacharunderscore}{\kern0pt}cols{\isacharunderscore}{\kern0pt}list\ {\isadigit{2}}\ {\isacharbrackleft}{\kern0pt}{\isacharbrackleft}{\kern0pt}{\isadigit{1}}{\isacharcomma}{\kern0pt}exp\ {\isacharparenleft}{\kern0pt}{\isadigit{2}}{\isacharasterisk}{\kern0pt}{\isasymi}{\isacharasterisk}{\kern0pt}pi{\isacharasterisk}{\kern0pt}complex{\isacharunderscore}{\kern0pt}of{\isacharunderscore}{\kern0pt}nat\ j\ {\isacharslash}{\kern0pt}\ {\isadigit{2}}{\isacharcircum}{\kern0pt}Suc\ n{\isacharparenright}{\kern0pt}{\isacharbrackright}{\kern0pt}{\isacharbrackright}{\kern0pt}{\isacharparenright}{\kern0pt}\ {\isachardollar}{\kern0pt}{\isachardollar}{\kern0pt}\ {\isacharparenleft}{\kern0pt}i{\isacharcomma}{\kern0pt}ja{\isacharparenright}{\kern0pt}\ {\isacharequal}{\kern0pt}\isanewline
\ \ \ \ \ \ \ \ \ \ \ \ \ \ \ \ \ \ \ \ {\isacharparenleft}{\kern0pt}\ {\isacharbar}{\kern0pt}zero{\isasymrangle}\ {\isacharplus}{\kern0pt}\ exp\ {\isacharparenleft}{\kern0pt}{\isadigit{2}}{\isacharasterisk}{\kern0pt}{\isasymi}{\isacharasterisk}{\kern0pt}pi{\isacharasterisk}{\kern0pt}complex{\isacharunderscore}{\kern0pt}of{\isacharunderscore}{\kern0pt}nat\ j\ {\isacharslash}{\kern0pt}\ {\isadigit{2}}{\isacharcircum}{\kern0pt}Suc\ n{\isacharparenright}{\kern0pt}\ {\isasymcdot}\isactrlsub m\ {\isacharbar}{\kern0pt}one{\isasymrangle}{\isacharparenright}{\kern0pt}\ {\isachardollar}{\kern0pt}{\isachardollar}{\kern0pt}\ {\isacharparenleft}{\kern0pt}i{\isacharcomma}{\kern0pt}ja{\isacharparenright}{\kern0pt}{\isachardoublequoteclose}\isanewline
\ \ \ \ \ \ \ \ \isacommand{using}\isamarkupfalse%
\ i{\isadigit{0}}\ ja{\isadigit{0}}\ \isacommand{by}\isamarkupfalse%
\ simp\isanewline
\ \ \ \ \isacommand{next}\isamarkupfalse%
\isanewline
\ \ \ \ \ \ \isacommand{assume}\isamarkupfalse%
\ i{\isadigit{1}}{\isacharcolon}{\kern0pt}{\isachardoublequoteopen}i\ {\isacharequal}{\kern0pt}\ {\isadigit{1}}{\isachardoublequoteclose}\isanewline
\ \ \ \ \ \ \isacommand{have}\isamarkupfalse%
\ {\isachardoublequoteopen}{\isacharparenleft}{\kern0pt}mat{\isacharunderscore}{\kern0pt}of{\isacharunderscore}{\kern0pt}cols{\isacharunderscore}{\kern0pt}list\ {\isadigit{2}}\ {\isacharbrackleft}{\kern0pt}{\isacharbrackleft}{\kern0pt}{\isadigit{1}}{\isacharcomma}{\kern0pt}exp\ {\isacharparenleft}{\kern0pt}{\isadigit{2}}{\isacharasterisk}{\kern0pt}{\isasymi}{\isacharasterisk}{\kern0pt}pi{\isacharasterisk}{\kern0pt}complex{\isacharunderscore}{\kern0pt}of{\isacharunderscore}{\kern0pt}nat\ j\ {\isacharslash}{\kern0pt}\ {\isadigit{2}}{\isacharcircum}{\kern0pt}Suc\ n{\isacharparenright}{\kern0pt}{\isacharbrackright}{\kern0pt}{\isacharbrackright}{\kern0pt}{\isacharparenright}{\kern0pt}\ {\isachardollar}{\kern0pt}{\isachardollar}{\kern0pt}\ {\isacharparenleft}{\kern0pt}{\isadigit{1}}{\isacharcomma}{\kern0pt}{\isadigit{0}}{\isacharparenright}{\kern0pt}\ {\isacharequal}{\kern0pt}\isanewline
\ \ \ \ \ \ \ \ \ \ \ \ exp\ {\isacharparenleft}{\kern0pt}{\isadigit{2}}{\isacharasterisk}{\kern0pt}{\isasymi}{\isacharasterisk}{\kern0pt}pi{\isacharasterisk}{\kern0pt}complex{\isacharunderscore}{\kern0pt}of{\isacharunderscore}{\kern0pt}nat\ j\ {\isacharslash}{\kern0pt}\ {\isadigit{2}}{\isacharcircum}{\kern0pt}Suc\ n{\isacharparenright}{\kern0pt}{\isachardoublequoteclose}\ \isacommand{by}\isamarkupfalse%
\ auto\isanewline
\ \ \ \ \ \ \isacommand{also}\isamarkupfalse%
\ \isacommand{have}\isamarkupfalse%
\ {\isachardoublequoteopen}{\isasymdots}\ {\isacharequal}{\kern0pt}\ {\isacharparenleft}{\kern0pt}\ {\isacharbar}{\kern0pt}zero{\isasymrangle}\ {\isacharplus}{\kern0pt}\ exp\ {\isacharparenleft}{\kern0pt}{\isadigit{2}}{\isacharasterisk}{\kern0pt}{\isasymi}{\isacharasterisk}{\kern0pt}pi{\isacharasterisk}{\kern0pt}complex{\isacharunderscore}{\kern0pt}of{\isacharunderscore}{\kern0pt}nat\ j\ {\isacharslash}{\kern0pt}\ {\isadigit{2}}{\isacharcircum}{\kern0pt}Suc\ n{\isacharparenright}{\kern0pt}\ {\isasymcdot}\isactrlsub m\ {\isacharbar}{\kern0pt}one{\isasymrangle}{\isacharparenright}{\kern0pt}\ {\isachardollar}{\kern0pt}{\isachardollar}{\kern0pt}\ {\isacharparenleft}{\kern0pt}{\isadigit{1}}{\isacharcomma}{\kern0pt}{\isadigit{0}}{\isacharparenright}{\kern0pt}{\isachardoublequoteclose}\isanewline
\ \ \ \ \ \ \isacommand{proof}\isamarkupfalse%
\ {\isacharminus}{\kern0pt}\isanewline
\ \ \ \ \ \ \ \ \isacommand{have}\isamarkupfalse%
\ {\isachardoublequoteopen}{\isacharbar}{\kern0pt}zero{\isasymrangle}\ {\isachardollar}{\kern0pt}{\isachardollar}{\kern0pt}\ {\isacharparenleft}{\kern0pt}{\isadigit{1}}{\isacharcomma}{\kern0pt}{\isadigit{0}}{\isacharparenright}{\kern0pt}\ {\isacharequal}{\kern0pt}\ {\isadigit{0}}{\isachardoublequoteclose}\ \isacommand{by}\isamarkupfalse%
\ auto\isanewline
\ \ \ \ \ \ \ \ \isacommand{moreover}\isamarkupfalse%
\ \isacommand{have}\isamarkupfalse%
\ {\isachardoublequoteopen}{\isacharparenleft}{\kern0pt}exp\ {\isacharparenleft}{\kern0pt}{\isadigit{2}}{\isacharasterisk}{\kern0pt}{\isasymi}{\isacharasterisk}{\kern0pt}pi{\isacharasterisk}{\kern0pt}complex{\isacharunderscore}{\kern0pt}of{\isacharunderscore}{\kern0pt}nat\ j\ {\isacharslash}{\kern0pt}\ {\isadigit{2}}{\isacharcircum}{\kern0pt}Suc\ n{\isacharparenright}{\kern0pt}\ {\isasymcdot}\isactrlsub m\ {\isacharbar}{\kern0pt}one{\isasymrangle}{\isacharparenright}{\kern0pt}\ {\isachardollar}{\kern0pt}{\isachardollar}{\kern0pt}\ {\isacharparenleft}{\kern0pt}{\isadigit{1}}{\isacharcomma}{\kern0pt}{\isadigit{0}}{\isacharparenright}{\kern0pt}\ {\isacharequal}{\kern0pt}\isanewline
\ \ \ \ \ \ \ \ \ \ \ \ \ \ \ \ \ \ \ \ \ \ \ \ exp\ {\isacharparenleft}{\kern0pt}{\isadigit{2}}{\isacharasterisk}{\kern0pt}{\isasymi}{\isacharasterisk}{\kern0pt}pi{\isacharasterisk}{\kern0pt}complex{\isacharunderscore}{\kern0pt}of{\isacharunderscore}{\kern0pt}nat\ j\ {\isacharslash}{\kern0pt}\ {\isadigit{2}}{\isacharcircum}{\kern0pt}Suc\ n{\isacharparenright}{\kern0pt}{\isachardoublequoteclose}\isanewline
\ \ \ \ \ \ \ \ \isacommand{proof}\isamarkupfalse%
\ {\isacharminus}{\kern0pt}\isanewline
\ \ \ \ \ \ \ \ \ \ \isacommand{have}\isamarkupfalse%
\ {\isachardoublequoteopen}{\isacharparenleft}{\kern0pt}exp\ {\isacharparenleft}{\kern0pt}{\isadigit{2}}{\isacharasterisk}{\kern0pt}{\isasymi}{\isacharasterisk}{\kern0pt}pi{\isacharasterisk}{\kern0pt}complex{\isacharunderscore}{\kern0pt}of{\isacharunderscore}{\kern0pt}nat\ j\ {\isacharslash}{\kern0pt}\ {\isadigit{2}}{\isacharcircum}{\kern0pt}Suc\ n{\isacharparenright}{\kern0pt}\ {\isasymcdot}\isactrlsub m\ {\isacharbar}{\kern0pt}one{\isasymrangle}{\isacharparenright}{\kern0pt}\ {\isachardollar}{\kern0pt}{\isachardollar}{\kern0pt}\ {\isacharparenleft}{\kern0pt}{\isadigit{1}}{\isacharcomma}{\kern0pt}{\isadigit{0}}{\isacharparenright}{\kern0pt}\ {\isacharequal}{\kern0pt}\isanewline
\ \ \ \ \ \ \ \ \ \ \ \ \ \ \ \ exp\ {\isacharparenleft}{\kern0pt}{\isadigit{2}}{\isacharasterisk}{\kern0pt}{\isasymi}{\isacharasterisk}{\kern0pt}pi{\isacharasterisk}{\kern0pt}complex{\isacharunderscore}{\kern0pt}of{\isacharunderscore}{\kern0pt}nat\ j\ {\isacharslash}{\kern0pt}\ {\isadigit{2}}{\isacharcircum}{\kern0pt}Suc\ n{\isacharparenright}{\kern0pt}\ {\isacharasterisk}{\kern0pt}\ {\isacharbar}{\kern0pt}one{\isasymrangle}\ {\isachardollar}{\kern0pt}{\isachardollar}{\kern0pt}\ {\isacharparenleft}{\kern0pt}{\isadigit{1}}{\isacharcomma}{\kern0pt}{\isadigit{0}}{\isacharparenright}{\kern0pt}{\isachardoublequoteclose}\isanewline
\ \ \ \ \ \ \ \ \ \ \ \ \isacommand{using}\isamarkupfalse%
\ index{\isacharunderscore}{\kern0pt}smult{\isacharunderscore}{\kern0pt}mat\ ket{\isacharunderscore}{\kern0pt}one{\isacharunderscore}{\kern0pt}is{\isacharunderscore}{\kern0pt}state\ state{\isacharunderscore}{\kern0pt}def\ \isacommand{by}\isamarkupfalse%
\ auto\isanewline
\ \ \ \ \ \ \ \ \ \ \isacommand{also}\isamarkupfalse%
\ \isacommand{have}\isamarkupfalse%
\ {\isachardoublequoteopen}{\isasymdots}\ {\isacharequal}{\kern0pt}\ exp\ {\isacharparenleft}{\kern0pt}{\isadigit{2}}{\isacharasterisk}{\kern0pt}{\isasymi}{\isacharasterisk}{\kern0pt}pi{\isacharasterisk}{\kern0pt}complex{\isacharunderscore}{\kern0pt}of{\isacharunderscore}{\kern0pt}nat\ j\ {\isacharslash}{\kern0pt}\ {\isadigit{2}}{\isacharcircum}{\kern0pt}Suc\ n{\isacharparenright}{\kern0pt}{\isachardoublequoteclose}\ \isacommand{by}\isamarkupfalse%
\ auto\isanewline
\ \ \ \ \ \ \ \ \ \ \isacommand{finally}\isamarkupfalse%
\ \isacommand{show}\isamarkupfalse%
\ {\isacharquery}{\kern0pt}thesis\ \isacommand{by}\isamarkupfalse%
\ this\isanewline
\ \ \ \ \ \ \ \ \isacommand{qed}\isamarkupfalse%
\isanewline
\ \ \ \ \ \ \ \ \isacommand{ultimately}\isamarkupfalse%
\ \isacommand{show}\isamarkupfalse%
\ {\isacharquery}{\kern0pt}thesis\ \isacommand{by}\isamarkupfalse%
\ {\isacharparenleft}{\kern0pt}simp\ add{\isacharcolon}{\kern0pt}\ ket{\isacharunderscore}{\kern0pt}vec{\isacharunderscore}{\kern0pt}def{\isacharparenright}{\kern0pt}\isanewline
\ \ \ \ \ \ \isacommand{qed}\isamarkupfalse%
\isanewline
\ \ \ \ \ \ \isacommand{finally}\isamarkupfalse%
\ \isacommand{show}\isamarkupfalse%
\ {\isachardoublequoteopen}{\isacharparenleft}{\kern0pt}mat{\isacharunderscore}{\kern0pt}of{\isacharunderscore}{\kern0pt}cols{\isacharunderscore}{\kern0pt}list\ {\isadigit{2}}\ {\isacharbrackleft}{\kern0pt}{\isacharbrackleft}{\kern0pt}{\isadigit{1}}{\isacharcomma}{\kern0pt}exp\ {\isacharparenleft}{\kern0pt}{\isadigit{2}}{\isacharasterisk}{\kern0pt}{\isasymi}{\isacharasterisk}{\kern0pt}pi{\isacharasterisk}{\kern0pt}complex{\isacharunderscore}{\kern0pt}of{\isacharunderscore}{\kern0pt}nat\ j\ {\isacharslash}{\kern0pt}\ {\isadigit{2}}{\isacharcircum}{\kern0pt}Suc\ n{\isacharparenright}{\kern0pt}{\isacharbrackright}{\kern0pt}{\isacharbrackright}{\kern0pt}{\isacharparenright}{\kern0pt}\ {\isachardollar}{\kern0pt}{\isachardollar}{\kern0pt}\ {\isacharparenleft}{\kern0pt}i{\isacharcomma}{\kern0pt}ja{\isacharparenright}{\kern0pt}\ {\isacharequal}{\kern0pt}\isanewline
\ \ \ \ \ \ \ \ \ \ \ \ \ \ \ \ \ \ \ \ {\isacharparenleft}{\kern0pt}\ {\isacharbar}{\kern0pt}zero{\isasymrangle}\ {\isacharplus}{\kern0pt}\ exp\ {\isacharparenleft}{\kern0pt}{\isadigit{2}}{\isacharasterisk}{\kern0pt}{\isasymi}{\isacharasterisk}{\kern0pt}pi{\isacharasterisk}{\kern0pt}complex{\isacharunderscore}{\kern0pt}of{\isacharunderscore}{\kern0pt}nat\ j\ {\isacharslash}{\kern0pt}\ {\isadigit{2}}{\isacharcircum}{\kern0pt}Suc\ n{\isacharparenright}{\kern0pt}\ {\isasymcdot}\isactrlsub m\ {\isacharbar}{\kern0pt}one{\isasymrangle}{\isacharparenright}{\kern0pt}\ {\isachardollar}{\kern0pt}{\isachardollar}{\kern0pt}\ {\isacharparenleft}{\kern0pt}i{\isacharcomma}{\kern0pt}ja{\isacharparenright}{\kern0pt}{\isachardoublequoteclose}\isanewline
\ \ \ \ \ \ \ \ \isacommand{using}\isamarkupfalse%
\ i{\isadigit{1}}\ ja{\isadigit{0}}\ \isacommand{by}\isamarkupfalse%
\ simp\isanewline
\ \ \ \ \isacommand{qed}\isamarkupfalse%
\isanewline
\ \ \ \ \isacommand{finally}\isamarkupfalse%
\ \isacommand{show}\isamarkupfalse%
\ {\isacharquery}{\kern0pt}thesis\ \isacommand{by}\isamarkupfalse%
\ this\isanewline
\ \ \isacommand{qed}\isamarkupfalse%
\isanewline
\isacommand{next}\isamarkupfalse%
\isanewline
\ \ \isacommand{show}\isamarkupfalse%
\ {\isachardoublequoteopen}dim{\isacharunderscore}{\kern0pt}row\ {\isacharparenleft}{\kern0pt}R\ {\isacharparenleft}{\kern0pt}Suc\ n{\isacharparenright}{\kern0pt}\ {\isacharasterisk}{\kern0pt}\ {\isacharparenleft}{\kern0pt}\ {\isacharbar}{\kern0pt}Deutsch{\isachardot}{\kern0pt}zero{\isasymrangle}\ {\isacharplus}{\kern0pt}\ exp\ {\isacharparenleft}{\kern0pt}{\isadigit{2}}\ {\isacharasterisk}{\kern0pt}\ {\isasymi}\ {\isacharasterisk}{\kern0pt}\ complex{\isacharunderscore}{\kern0pt}of{\isacharunderscore}{\kern0pt}real\ pi\ {\isacharasterisk}{\kern0pt}\isanewline
\ \ \ \ \ \ \ \ complex{\isacharunderscore}{\kern0pt}of{\isacharunderscore}{\kern0pt}nat\ {\isacharparenleft}{\kern0pt}j\ div\ {\isadigit{2}}{\isacharparenright}{\kern0pt}\ {\isacharslash}{\kern0pt}\ {\isadigit{2}}\ {\isacharcircum}{\kern0pt}\ n{\isacharparenright}{\kern0pt}\ {\isasymcdot}\isactrlsub m\ {\isacharbar}{\kern0pt}Deutsch{\isachardot}{\kern0pt}one{\isasymrangle}{\isacharparenright}{\kern0pt}{\isacharparenright}{\kern0pt}\ {\isacharequal}{\kern0pt}\isanewline
\ \ \ \ \ \ \ \ dim{\isacharunderscore}{\kern0pt}row\ {\isacharparenleft}{\kern0pt}\ {\isacharbar}{\kern0pt}Deutsch{\isachardot}{\kern0pt}zero{\isasymrangle}\ {\isacharplus}{\kern0pt}\ exp\ {\isacharparenleft}{\kern0pt}{\isadigit{2}}\ {\isacharasterisk}{\kern0pt}\ {\isasymi}\ {\isacharasterisk}{\kern0pt}\ complex{\isacharunderscore}{\kern0pt}of{\isacharunderscore}{\kern0pt}real\ pi\ {\isacharasterisk}{\kern0pt}\ complex{\isacharunderscore}{\kern0pt}of{\isacharunderscore}{\kern0pt}nat\ j\ {\isacharslash}{\kern0pt}\ {\isadigit{2}}\ {\isacharcircum}{\kern0pt}\ Suc\ n{\isacharparenright}{\kern0pt}\ {\isasymcdot}\isactrlsub m\isanewline
\ \ \ \ \ \ \ \ {\isacharbar}{\kern0pt}Deutsch{\isachardot}{\kern0pt}one{\isasymrangle}{\isacharparenright}{\kern0pt}{\isachardoublequoteclose}\ \isanewline
\ \ \isacommand{by}\isamarkupfalse%
\ {\isacharparenleft}{\kern0pt}simp\ add{\isacharcolon}{\kern0pt}\ R{\isacharunderscore}{\kern0pt}def\ Tensor{\isachardot}{\kern0pt}mat{\isacharunderscore}{\kern0pt}of{\isacharunderscore}{\kern0pt}cols{\isacharunderscore}{\kern0pt}list{\isacharunderscore}{\kern0pt}def\ ket{\isacharunderscore}{\kern0pt}vec{\isacharunderscore}{\kern0pt}def{\isacharparenright}{\kern0pt}\isanewline
\isacommand{next}\isamarkupfalse%
\isanewline
\ \ \isacommand{show}\isamarkupfalse%
\ {\isachardoublequoteopen}dim{\isacharunderscore}{\kern0pt}col\ {\isacharparenleft}{\kern0pt}R\ {\isacharparenleft}{\kern0pt}Suc\ n{\isacharparenright}{\kern0pt}\ {\isacharasterisk}{\kern0pt}\ {\isacharparenleft}{\kern0pt}\ {\isacharbar}{\kern0pt}Deutsch{\isachardot}{\kern0pt}zero{\isasymrangle}\ {\isacharplus}{\kern0pt}\ exp\ {\isacharparenleft}{\kern0pt}{\isadigit{2}}\ {\isacharasterisk}{\kern0pt}\ {\isasymi}\ {\isacharasterisk}{\kern0pt}\ complex{\isacharunderscore}{\kern0pt}of{\isacharunderscore}{\kern0pt}real\ pi\ {\isacharasterisk}{\kern0pt}\isanewline
\ \ \ \ \ \ \ \ complex{\isacharunderscore}{\kern0pt}of{\isacharunderscore}{\kern0pt}nat\ {\isacharparenleft}{\kern0pt}j\ div\ {\isadigit{2}}{\isacharparenright}{\kern0pt}\ {\isacharslash}{\kern0pt}\ {\isadigit{2}}\ {\isacharcircum}{\kern0pt}\ n{\isacharparenright}{\kern0pt}\ {\isasymcdot}\isactrlsub m\ {\isacharbar}{\kern0pt}Deutsch{\isachardot}{\kern0pt}one{\isasymrangle}{\isacharparenright}{\kern0pt}{\isacharparenright}{\kern0pt}\ {\isacharequal}{\kern0pt}\isanewline
\ \ \ \ \ \ \ \ dim{\isacharunderscore}{\kern0pt}col\ {\isacharparenleft}{\kern0pt}\ {\isacharbar}{\kern0pt}Deutsch{\isachardot}{\kern0pt}zero{\isasymrangle}\ {\isacharplus}{\kern0pt}\ exp\ {\isacharparenleft}{\kern0pt}{\isadigit{2}}\ {\isacharasterisk}{\kern0pt}\ {\isasymi}\ {\isacharasterisk}{\kern0pt}\ complex{\isacharunderscore}{\kern0pt}of{\isacharunderscore}{\kern0pt}real\ pi\ {\isacharasterisk}{\kern0pt}\ complex{\isacharunderscore}{\kern0pt}of{\isacharunderscore}{\kern0pt}nat\ j\ {\isacharslash}{\kern0pt}\ {\isadigit{2}}\ {\isacharcircum}{\kern0pt}\ Suc\ n{\isacharparenright}{\kern0pt}\ {\isasymcdot}\isactrlsub m\isanewline
\ \ \ \ \ \ \ \ {\isacharbar}{\kern0pt}Deutsch{\isachardot}{\kern0pt}one{\isasymrangle}{\isacharparenright}{\kern0pt}{\isachardoublequoteclose}\isanewline
\ \ \ \ \isacommand{by}\isamarkupfalse%
\ {\isacharparenleft}{\kern0pt}simp\ add{\isacharcolon}{\kern0pt}\ R{\isacharunderscore}{\kern0pt}def\ Tensor{\isachardot}{\kern0pt}mat{\isacharunderscore}{\kern0pt}of{\isacharunderscore}{\kern0pt}cols{\isacharunderscore}{\kern0pt}list{\isacharunderscore}{\kern0pt}def\ ket{\isacharunderscore}{\kern0pt}vec{\isacharunderscore}{\kern0pt}def{\isacharparenright}{\kern0pt}\isanewline
\isacommand{qed}\isamarkupfalse%
%
\endisatagproof
{\isafoldproof}%
%
\isadelimproof
%
\endisadelimproof
%
\begin{isamarkuptext}%
Action of the SWAP cascades in the circuit%
\end{isamarkuptext}\isamarkuptrue%
\isacommand{lemma}\isamarkupfalse%
\ SWAP{\isacharunderscore}{\kern0pt}up{\isacharunderscore}{\kern0pt}action{\isacharcolon}{\kern0pt}\isanewline
\ \ {\isachardoublequoteopen}{\isasymforall}j{\isachardot}{\kern0pt}\ j\ {\isacharless}{\kern0pt}\ {\isadigit{2}}\ {\isacharcircum}{\kern0pt}{\isacharparenleft}{\kern0pt}Suc\ {\isacharparenleft}{\kern0pt}Suc\ n{\isacharparenright}{\kern0pt}{\isacharparenright}{\kern0pt}\ {\isasymlongrightarrow}\ \isanewline
\ \ \ \ SWAP{\isacharunderscore}{\kern0pt}up\ {\isacharparenleft}{\kern0pt}Suc\ {\isacharparenleft}{\kern0pt}Suc\ n{\isacharparenright}{\kern0pt}{\isacharparenright}{\kern0pt}\ {\isacharasterisk}{\kern0pt}\ {\isacharparenleft}{\kern0pt}\ {\isacharbar}{\kern0pt}state{\isacharunderscore}{\kern0pt}basis\ {\isacharparenleft}{\kern0pt}Suc\ n{\isacharparenright}{\kern0pt}\ {\isacharparenleft}{\kern0pt}j\ div\ {\isadigit{2}}{\isacharparenright}{\kern0pt}{\isasymrangle}\ {\isasymOtimes}\ {\isacharbar}{\kern0pt}state{\isacharunderscore}{\kern0pt}basis\ {\isadigit{1}}\ {\isacharparenleft}{\kern0pt}j\ mod\ {\isadigit{2}}{\isacharparenright}{\kern0pt}{\isasymrangle}{\isacharparenright}{\kern0pt}\ {\isacharequal}{\kern0pt}\isanewline
\ \ \ \ {\isacharbar}{\kern0pt}state{\isacharunderscore}{\kern0pt}basis\ {\isadigit{1}}\ {\isacharparenleft}{\kern0pt}j\ mod\ {\isadigit{2}}{\isacharparenright}{\kern0pt}{\isasymrangle}\ {\isasymOtimes}\ {\isacharbar}{\kern0pt}state{\isacharunderscore}{\kern0pt}basis\ {\isacharparenleft}{\kern0pt}Suc\ n{\isacharparenright}{\kern0pt}\ {\isacharparenleft}{\kern0pt}j\ div\ {\isadigit{2}}{\isacharparenright}{\kern0pt}{\isasymrangle}{\isachardoublequoteclose}\isanewline
%
\isadelimproof
%
\endisadelimproof
%
\isatagproof
\isacommand{proof}\isamarkupfalse%
\ {\isacharparenleft}{\kern0pt}induct\ n{\isacharparenright}{\kern0pt}\isanewline
\ \ \isacommand{case}\isamarkupfalse%
\ {\isadigit{0}}\isanewline
\ \ \isacommand{show}\isamarkupfalse%
\ {\isacharquery}{\kern0pt}case\isanewline
\ \ \isacommand{proof}\isamarkupfalse%
\isanewline
\ \ \ \ \isacommand{fix}\isamarkupfalse%
\ j\isanewline
\ \ \ \ \isacommand{show}\isamarkupfalse%
\ {\isachardoublequoteopen}j\ {\isacharless}{\kern0pt}\ {\isadigit{2}}\ {\isacharcircum}{\kern0pt}\ Suc\ {\isacharparenleft}{\kern0pt}Suc\ {\isadigit{0}}{\isacharparenright}{\kern0pt}\ {\isasymlongrightarrow}\ SWAP{\isacharunderscore}{\kern0pt}up\ {\isacharparenleft}{\kern0pt}Suc\ {\isacharparenleft}{\kern0pt}Suc\ {\isadigit{0}}{\isacharparenright}{\kern0pt}{\isacharparenright}{\kern0pt}\ {\isacharasterisk}{\kern0pt}\ {\isacharparenleft}{\kern0pt}\ {\isacharbar}{\kern0pt}state{\isacharunderscore}{\kern0pt}basis\ {\isacharparenleft}{\kern0pt}Suc\ {\isadigit{0}}{\isacharparenright}{\kern0pt}\ {\isacharparenleft}{\kern0pt}j\ div\ {\isadigit{2}}{\isacharparenright}{\kern0pt}{\isasymrangle}\ {\isasymOtimes}\isanewline
\ \ \ \ \ \ \ \ \ \ {\isacharbar}{\kern0pt}state{\isacharunderscore}{\kern0pt}basis\ {\isadigit{1}}\ {\isacharparenleft}{\kern0pt}j\ mod\ {\isadigit{2}}{\isacharparenright}{\kern0pt}{\isasymrangle}{\isacharparenright}{\kern0pt}\ {\isacharequal}{\kern0pt}\isanewline
\ \ \ \ \ \ \ \ \ \ {\isacharbar}{\kern0pt}state{\isacharunderscore}{\kern0pt}basis\ {\isadigit{1}}\ {\isacharparenleft}{\kern0pt}j\ mod\ {\isadigit{2}}{\isacharparenright}{\kern0pt}{\isasymrangle}\ {\isasymOtimes}\ {\isacharbar}{\kern0pt}state{\isacharunderscore}{\kern0pt}basis\ {\isacharparenleft}{\kern0pt}Suc\ {\isadigit{0}}{\isacharparenright}{\kern0pt}\ {\isacharparenleft}{\kern0pt}j\ div\ {\isadigit{2}}{\isacharparenright}{\kern0pt}{\isasymrangle}{\isachardoublequoteclose}\isanewline
\ \ \ \ \isacommand{proof}\isamarkupfalse%
\isanewline
\ \ \ \ \ \ \isacommand{assume}\isamarkupfalse%
\ {\isachardoublequoteopen}j\ {\isacharless}{\kern0pt}\ {\isadigit{2}}{\isacharcircum}{\kern0pt}\ Suc\ {\isacharparenleft}{\kern0pt}Suc\ {\isadigit{0}}{\isacharparenright}{\kern0pt}{\isachardoublequoteclose}\isanewline
\ \ \ \ \ \ \isacommand{show}\isamarkupfalse%
\ {\isachardoublequoteopen}SWAP{\isacharunderscore}{\kern0pt}up\ {\isacharparenleft}{\kern0pt}Suc\ {\isacharparenleft}{\kern0pt}Suc\ {\isadigit{0}}{\isacharparenright}{\kern0pt}{\isacharparenright}{\kern0pt}\ {\isacharasterisk}{\kern0pt}\ {\isacharparenleft}{\kern0pt}\ {\isacharbar}{\kern0pt}state{\isacharunderscore}{\kern0pt}basis\ {\isacharparenleft}{\kern0pt}Suc\ {\isadigit{0}}{\isacharparenright}{\kern0pt}\ {\isacharparenleft}{\kern0pt}j\ div\ {\isadigit{2}}{\isacharparenright}{\kern0pt}{\isasymrangle}\ {\isasymOtimes}\ {\isacharbar}{\kern0pt}state{\isacharunderscore}{\kern0pt}basis\ {\isadigit{1}}\ {\isacharparenleft}{\kern0pt}j\ mod\ {\isadigit{2}}{\isacharparenright}{\kern0pt}{\isasymrangle}{\isacharparenright}{\kern0pt}\ \isanewline
\ \ \ \ \ \ \ \ \ \ \ \ {\isacharequal}{\kern0pt}\ {\isacharbar}{\kern0pt}state{\isacharunderscore}{\kern0pt}basis\ {\isadigit{1}}\ {\isacharparenleft}{\kern0pt}j\ mod\ {\isadigit{2}}{\isacharparenright}{\kern0pt}{\isasymrangle}\ {\isasymOtimes}\ {\isacharbar}{\kern0pt}state{\isacharunderscore}{\kern0pt}basis\ {\isacharparenleft}{\kern0pt}Suc\ {\isadigit{0}}{\isacharparenright}{\kern0pt}\ {\isacharparenleft}{\kern0pt}j\ div\ {\isadigit{2}}{\isacharparenright}{\kern0pt}{\isasymrangle}{\isachardoublequoteclose}\isanewline
\ \ \ \ \ \ \isacommand{proof}\isamarkupfalse%
\ {\isacharminus}{\kern0pt}\isanewline
\ \ \ \ \ \ \ \ \isacommand{have}\isamarkupfalse%
\ {\isachardoublequoteopen}SWAP{\isacharunderscore}{\kern0pt}up\ {\isacharparenleft}{\kern0pt}Suc\ {\isacharparenleft}{\kern0pt}Suc\ {\isadigit{0}}{\isacharparenright}{\kern0pt}{\isacharparenright}{\kern0pt}{\isacharasterisk}{\kern0pt}{\isacharparenleft}{\kern0pt}\ {\isacharbar}{\kern0pt}state{\isacharunderscore}{\kern0pt}basis\ {\isacharparenleft}{\kern0pt}Suc\ {\isadigit{0}}{\isacharparenright}{\kern0pt}\ {\isacharparenleft}{\kern0pt}j\ div\ {\isadigit{2}}{\isacharparenright}{\kern0pt}{\isasymrangle}\ {\isasymOtimes}\ {\isacharbar}{\kern0pt}state{\isacharunderscore}{\kern0pt}basis\ {\isadigit{1}}\ {\isacharparenleft}{\kern0pt}j\ mod\ {\isadigit{2}}{\isacharparenright}{\kern0pt}{\isasymrangle}{\isacharparenright}{\kern0pt}\isanewline
\ \ \ \ \ \ \ \ \ \ \ \ \ \ {\isacharequal}{\kern0pt}\ SWAP\ {\isacharasterisk}{\kern0pt}\ {\isacharparenleft}{\kern0pt}\ {\isacharbar}{\kern0pt}state{\isacharunderscore}{\kern0pt}basis\ {\isacharparenleft}{\kern0pt}Suc\ {\isadigit{0}}{\isacharparenright}{\kern0pt}\ {\isacharparenleft}{\kern0pt}j\ div\ {\isadigit{2}}{\isacharparenright}{\kern0pt}{\isasymrangle}\ {\isasymOtimes}\ {\isacharbar}{\kern0pt}state{\isacharunderscore}{\kern0pt}basis\ {\isadigit{1}}\ {\isacharparenleft}{\kern0pt}j\ mod\ {\isadigit{2}}{\isacharparenright}{\kern0pt}{\isasymrangle}{\isacharparenright}{\kern0pt}{\isachardoublequoteclose}\isanewline
\ \ \ \ \ \ \ \ \ \ \isacommand{using}\isamarkupfalse%
\ SWAP{\isacharunderscore}{\kern0pt}up{\isachardot}{\kern0pt}simps\ \isacommand{by}\isamarkupfalse%
\ simp\isanewline
\ \ \ \ \ \ \ \ \isacommand{also}\isamarkupfalse%
\ \isacommand{have}\isamarkupfalse%
\ {\isachardoublequoteopen}{\isasymdots}\ {\isacharequal}{\kern0pt}\ {\isacharbar}{\kern0pt}state{\isacharunderscore}{\kern0pt}basis\ {\isadigit{1}}\ {\isacharparenleft}{\kern0pt}j\ mod\ {\isadigit{2}}{\isacharparenright}{\kern0pt}{\isasymrangle}\ {\isasymOtimes}\ {\isacharbar}{\kern0pt}state{\isacharunderscore}{\kern0pt}basis\ {\isacharparenleft}{\kern0pt}Suc\ {\isadigit{0}}{\isacharparenright}{\kern0pt}\ {\isacharparenleft}{\kern0pt}j\ div\ {\isadigit{2}}{\isacharparenright}{\kern0pt}{\isasymrangle}{\isachardoublequoteclose}\isanewline
\ \ \ \ \ \ \ \ \ \ \isacommand{using}\isamarkupfalse%
\ SWAP{\isacharunderscore}{\kern0pt}tensor\isanewline
\ \ \ \ \ \ \ \ \ \ \isacommand{by}\isamarkupfalse%
\ {\isacharparenleft}{\kern0pt}metis\ One{\isacharunderscore}{\kern0pt}nat{\isacharunderscore}{\kern0pt}def\ power{\isacharunderscore}{\kern0pt}one{\isacharunderscore}{\kern0pt}right\ state{\isacharunderscore}{\kern0pt}basis{\isacharunderscore}{\kern0pt}carrier{\isacharunderscore}{\kern0pt}mat{\isacharparenright}{\kern0pt}\isanewline
\ \ \ \ \ \ \ \ \isacommand{finally}\isamarkupfalse%
\ \isacommand{show}\isamarkupfalse%
\ {\isacharquery}{\kern0pt}thesis\ \isacommand{by}\isamarkupfalse%
\ this\isanewline
\ \ \ \ \ \ \isacommand{qed}\isamarkupfalse%
\isanewline
\ \ \ \ \isacommand{qed}\isamarkupfalse%
\isanewline
\ \ \isacommand{qed}\isamarkupfalse%
\isanewline
\isacommand{next}\isamarkupfalse%
\isanewline
\ \ \isacommand{case}\isamarkupfalse%
\ {\isacharparenleft}{\kern0pt}Suc\ n{\isacharparenright}{\kern0pt}\isanewline
\ \ \isacommand{assume}\isamarkupfalse%
\ HI{\isacharcolon}{\kern0pt}{\isachardoublequoteopen}{\isasymforall}j{\isacharless}{\kern0pt}{\isadigit{2}}\ {\isacharcircum}{\kern0pt}\ Suc\ {\isacharparenleft}{\kern0pt}Suc\ n{\isacharparenright}{\kern0pt}{\isachardot}{\kern0pt}\isanewline
\ \ \ \ \ \ \ \ \ \ \ \ SWAP{\isacharunderscore}{\kern0pt}up\ {\isacharparenleft}{\kern0pt}Suc\ {\isacharparenleft}{\kern0pt}Suc\ n{\isacharparenright}{\kern0pt}{\isacharparenright}{\kern0pt}\ {\isacharasterisk}{\kern0pt}\ {\isacharparenleft}{\kern0pt}\ {\isacharbar}{\kern0pt}state{\isacharunderscore}{\kern0pt}basis\ {\isacharparenleft}{\kern0pt}Suc\ n{\isacharparenright}{\kern0pt}\ {\isacharparenleft}{\kern0pt}j\ div\ {\isadigit{2}}{\isacharparenright}{\kern0pt}{\isasymrangle}\ {\isasymOtimes}\ {\isacharbar}{\kern0pt}state{\isacharunderscore}{\kern0pt}basis\ {\isadigit{1}}\ {\isacharparenleft}{\kern0pt}j\ mod\ {\isadigit{2}}{\isacharparenright}{\kern0pt}{\isasymrangle}{\isacharparenright}{\kern0pt}\isanewline
\ \ \ \ \ \ \ \ \ \ \ \ {\isacharequal}{\kern0pt}\ {\isacharbar}{\kern0pt}state{\isacharunderscore}{\kern0pt}basis\ {\isadigit{1}}\ {\isacharparenleft}{\kern0pt}j\ mod\ {\isadigit{2}}{\isacharparenright}{\kern0pt}{\isasymrangle}\ {\isasymOtimes}\ {\isacharbar}{\kern0pt}state{\isacharunderscore}{\kern0pt}basis\ {\isacharparenleft}{\kern0pt}Suc\ n{\isacharparenright}{\kern0pt}\ {\isacharparenleft}{\kern0pt}j\ div\ {\isadigit{2}}{\isacharparenright}{\kern0pt}{\isasymrangle}{\isachardoublequoteclose}\isanewline
\ \ \isacommand{show}\isamarkupfalse%
\ {\isachardoublequoteopen}{\isasymforall}j{\isacharless}{\kern0pt}{\isadigit{2}}\ {\isacharcircum}{\kern0pt}\ Suc\ {\isacharparenleft}{\kern0pt}Suc\ {\isacharparenleft}{\kern0pt}Suc\ n{\isacharparenright}{\kern0pt}{\isacharparenright}{\kern0pt}{\isachardot}{\kern0pt}\isanewline
\ \ \ \ \ \ \ \ \ SWAP{\isacharunderscore}{\kern0pt}up\ {\isacharparenleft}{\kern0pt}Suc\ {\isacharparenleft}{\kern0pt}Suc\ {\isacharparenleft}{\kern0pt}Suc\ n{\isacharparenright}{\kern0pt}{\isacharparenright}{\kern0pt}{\isacharparenright}{\kern0pt}\ {\isacharasterisk}{\kern0pt}\ {\isacharparenleft}{\kern0pt}\ {\isacharbar}{\kern0pt}state{\isacharunderscore}{\kern0pt}basis\ {\isacharparenleft}{\kern0pt}Suc\ {\isacharparenleft}{\kern0pt}Suc\ n{\isacharparenright}{\kern0pt}{\isacharparenright}{\kern0pt}\ {\isacharparenleft}{\kern0pt}j\ div\ {\isadigit{2}}{\isacharparenright}{\kern0pt}{\isasymrangle}\ {\isasymOtimes}\ \isanewline
\ \ \ \ \ \ \ \ \ {\isacharbar}{\kern0pt}state{\isacharunderscore}{\kern0pt}basis\ {\isadigit{1}}\ {\isacharparenleft}{\kern0pt}j\ mod\ {\isadigit{2}}{\isacharparenright}{\kern0pt}{\isasymrangle}{\isacharparenright}{\kern0pt}\ {\isacharequal}{\kern0pt}\isanewline
\ \ \ \ \ \ \ \ \ {\isacharbar}{\kern0pt}state{\isacharunderscore}{\kern0pt}basis\ {\isadigit{1}}\ {\isacharparenleft}{\kern0pt}j\ mod\ {\isadigit{2}}{\isacharparenright}{\kern0pt}{\isasymrangle}\ {\isasymOtimes}\ {\isacharbar}{\kern0pt}state{\isacharunderscore}{\kern0pt}basis\ {\isacharparenleft}{\kern0pt}Suc\ {\isacharparenleft}{\kern0pt}Suc\ n{\isacharparenright}{\kern0pt}{\isacharparenright}{\kern0pt}\ {\isacharparenleft}{\kern0pt}j\ div\ {\isadigit{2}}{\isacharparenright}{\kern0pt}{\isasymrangle}{\isachardoublequoteclose}\isanewline
\ \ \isacommand{proof}\isamarkupfalse%
\ \isanewline
\ \ \ \ \isacommand{fix}\isamarkupfalse%
\ j{\isacharcolon}{\kern0pt}{\isacharcolon}{\kern0pt}nat\isanewline
\ \ \ \ \isacommand{show}\isamarkupfalse%
\ {\isachardoublequoteopen}j\ {\isacharless}{\kern0pt}\ {\isadigit{2}}\ {\isacharcircum}{\kern0pt}\ Suc\ {\isacharparenleft}{\kern0pt}Suc\ {\isacharparenleft}{\kern0pt}Suc\ n{\isacharparenright}{\kern0pt}{\isacharparenright}{\kern0pt}\ {\isasymlongrightarrow}\isanewline
\ \ \ \ \ \ \ \ \ SWAP{\isacharunderscore}{\kern0pt}up\ {\isacharparenleft}{\kern0pt}Suc\ {\isacharparenleft}{\kern0pt}Suc\ {\isacharparenleft}{\kern0pt}Suc\ n{\isacharparenright}{\kern0pt}{\isacharparenright}{\kern0pt}{\isacharparenright}{\kern0pt}\ {\isacharasterisk}{\kern0pt}\ {\isacharparenleft}{\kern0pt}\ {\isacharbar}{\kern0pt}state{\isacharunderscore}{\kern0pt}basis\ {\isacharparenleft}{\kern0pt}Suc\ {\isacharparenleft}{\kern0pt}Suc\ n{\isacharparenright}{\kern0pt}{\isacharparenright}{\kern0pt}\ {\isacharparenleft}{\kern0pt}j\ div\ {\isadigit{2}}{\isacharparenright}{\kern0pt}{\isasymrangle}\ {\isasymOtimes}\isanewline
\ \ \ \ \ \ \ \ \ \ {\isacharbar}{\kern0pt}state{\isacharunderscore}{\kern0pt}basis\ {\isadigit{1}}\ {\isacharparenleft}{\kern0pt}j\ mod\ {\isadigit{2}}{\isacharparenright}{\kern0pt}{\isasymrangle}{\isacharparenright}{\kern0pt}\ {\isacharequal}{\kern0pt}\isanewline
\ \ \ \ \ \ \ \ \ {\isacharbar}{\kern0pt}state{\isacharunderscore}{\kern0pt}basis\ {\isadigit{1}}\ {\isacharparenleft}{\kern0pt}j\ mod\ {\isadigit{2}}{\isacharparenright}{\kern0pt}{\isasymrangle}\ {\isasymOtimes}\ {\isacharbar}{\kern0pt}state{\isacharunderscore}{\kern0pt}basis\ {\isacharparenleft}{\kern0pt}Suc\ {\isacharparenleft}{\kern0pt}Suc\ n{\isacharparenright}{\kern0pt}{\isacharparenright}{\kern0pt}\ {\isacharparenleft}{\kern0pt}j\ div\ {\isadigit{2}}{\isacharparenright}{\kern0pt}{\isasymrangle}{\isachardoublequoteclose}\isanewline
\ \ \ \ \isacommand{proof}\isamarkupfalse%
\ \isanewline
\ \ \ \ \ \ \isacommand{assume}\isamarkupfalse%
\ jl{\isacharcolon}{\kern0pt}{\isachardoublequoteopen}j\ {\isacharless}{\kern0pt}\ {\isadigit{2}}\ {\isacharcircum}{\kern0pt}\ Suc\ {\isacharparenleft}{\kern0pt}Suc\ {\isacharparenleft}{\kern0pt}Suc\ n{\isacharparenright}{\kern0pt}{\isacharparenright}{\kern0pt}{\isachardoublequoteclose}\isanewline
\ \ \ \ \ \ \isacommand{show}\isamarkupfalse%
\ {\isachardoublequoteopen}SWAP{\isacharunderscore}{\kern0pt}up\ {\isacharparenleft}{\kern0pt}Suc\ {\isacharparenleft}{\kern0pt}Suc\ {\isacharparenleft}{\kern0pt}Suc\ n{\isacharparenright}{\kern0pt}{\isacharparenright}{\kern0pt}{\isacharparenright}{\kern0pt}\ {\isacharasterisk}{\kern0pt}\ {\isacharparenleft}{\kern0pt}\ {\isacharbar}{\kern0pt}state{\isacharunderscore}{\kern0pt}basis\ {\isacharparenleft}{\kern0pt}Suc\ {\isacharparenleft}{\kern0pt}Suc\ n{\isacharparenright}{\kern0pt}{\isacharparenright}{\kern0pt}\ {\isacharparenleft}{\kern0pt}j\ div\ {\isadigit{2}}{\isacharparenright}{\kern0pt}{\isasymrangle}\ {\isasymOtimes}\isanewline
\ \ \ \ \ \ \ \ \ \ \ \ {\isacharbar}{\kern0pt}state{\isacharunderscore}{\kern0pt}basis\ {\isadigit{1}}\ {\isacharparenleft}{\kern0pt}j\ mod\ {\isadigit{2}}{\isacharparenright}{\kern0pt}{\isasymrangle}{\isacharparenright}{\kern0pt}\ {\isacharequal}{\kern0pt}\isanewline
\ \ \ \ \ \ \ \ \ \ \ \ {\isacharbar}{\kern0pt}state{\isacharunderscore}{\kern0pt}basis\ {\isadigit{1}}\ {\isacharparenleft}{\kern0pt}j\ mod\ {\isadigit{2}}{\isacharparenright}{\kern0pt}{\isasymrangle}\ {\isasymOtimes}\ {\isacharbar}{\kern0pt}state{\isacharunderscore}{\kern0pt}basis\ {\isacharparenleft}{\kern0pt}Suc\ {\isacharparenleft}{\kern0pt}Suc\ n{\isacharparenright}{\kern0pt}{\isacharparenright}{\kern0pt}\ {\isacharparenleft}{\kern0pt}j\ div\ {\isadigit{2}}{\isacharparenright}{\kern0pt}{\isasymrangle}{\isachardoublequoteclose}\isanewline
\ \ \ \ \ \ \isacommand{proof}\isamarkupfalse%
\ {\isacharminus}{\kern0pt}\isanewline
\ \ \ \ \ \ \ \ \isacommand{have}\isamarkupfalse%
\ {\isachardoublequoteopen}SWAP{\isacharunderscore}{\kern0pt}up\ {\isacharparenleft}{\kern0pt}Suc\ {\isacharparenleft}{\kern0pt}Suc\ {\isacharparenleft}{\kern0pt}Suc\ n{\isacharparenright}{\kern0pt}{\isacharparenright}{\kern0pt}{\isacharparenright}{\kern0pt}\ {\isacharasterisk}{\kern0pt}\ {\isacharparenleft}{\kern0pt}\ {\isacharbar}{\kern0pt}state{\isacharunderscore}{\kern0pt}basis\ {\isacharparenleft}{\kern0pt}Suc\ {\isacharparenleft}{\kern0pt}Suc\ n{\isacharparenright}{\kern0pt}{\isacharparenright}{\kern0pt}\ {\isacharparenleft}{\kern0pt}j\ div\ {\isadigit{2}}{\isacharparenright}{\kern0pt}{\isasymrangle}\ {\isasymOtimes}\isanewline
\ \ \ \ \ \ \ \ \ \ \ \ \ \ {\isacharbar}{\kern0pt}state{\isacharunderscore}{\kern0pt}basis\ {\isadigit{1}}\ {\isacharparenleft}{\kern0pt}j\ mod\ {\isadigit{2}}{\isacharparenright}{\kern0pt}{\isasymrangle}{\isacharparenright}{\kern0pt}\ {\isacharequal}{\kern0pt}\isanewline
\ \ \ \ \ \ \ \ \ \ \ \ \ \ {\isacharparenleft}{\kern0pt}{\isacharparenleft}{\kern0pt}SWAP\ {\isasymOtimes}\ {\isacharparenleft}{\kern0pt}{\isadigit{1}}\isactrlsub m\ {\isacharparenleft}{\kern0pt}{\isadigit{2}}{\isacharcircum}{\kern0pt}{\isacharparenleft}{\kern0pt}Suc\ n{\isacharparenright}{\kern0pt}{\isacharparenright}{\kern0pt}{\isacharparenright}{\kern0pt}{\isacharparenright}{\kern0pt}\ {\isacharasterisk}{\kern0pt}\ {\isacharparenleft}{\kern0pt}{\isacharparenleft}{\kern0pt}{\isadigit{1}}\isactrlsub m\ {\isadigit{2}}{\isacharparenright}{\kern0pt}\ {\isasymOtimes}\ {\isacharparenleft}{\kern0pt}SWAP{\isacharunderscore}{\kern0pt}up\ {\isacharparenleft}{\kern0pt}Suc\ {\isacharparenleft}{\kern0pt}Suc\ n{\isacharparenright}{\kern0pt}{\isacharparenright}{\kern0pt}{\isacharparenright}{\kern0pt}{\isacharparenright}{\kern0pt}{\isacharparenright}{\kern0pt}\ {\isacharasterisk}{\kern0pt}\isanewline
\ \ \ \ \ \ \ \ \ \ \ \ \ \ {\isacharparenleft}{\kern0pt}\ {\isacharbar}{\kern0pt}state{\isacharunderscore}{\kern0pt}basis\ {\isacharparenleft}{\kern0pt}Suc\ {\isacharparenleft}{\kern0pt}Suc\ n{\isacharparenright}{\kern0pt}{\isacharparenright}{\kern0pt}\ {\isacharparenleft}{\kern0pt}j\ div\ {\isadigit{2}}{\isacharparenright}{\kern0pt}{\isasymrangle}\ {\isasymOtimes}\ {\isacharbar}{\kern0pt}state{\isacharunderscore}{\kern0pt}basis\ {\isadigit{1}}\ {\isacharparenleft}{\kern0pt}j\ mod\ {\isadigit{2}}{\isacharparenright}{\kern0pt}{\isasymrangle}{\isacharparenright}{\kern0pt}{\isachardoublequoteclose}\isanewline
\ \ \ \ \ \ \ \ \ \ \isacommand{using}\isamarkupfalse%
\ SWAP{\isacharunderscore}{\kern0pt}up{\isachardot}{\kern0pt}simps\ \isacommand{by}\isamarkupfalse%
\ simp\isanewline
\ \ \ \ \ \ \ \ \isacommand{also}\isamarkupfalse%
\ \isacommand{have}\isamarkupfalse%
\ {\isachardoublequoteopen}{\isasymdots}\ {\isacharequal}{\kern0pt}\ {\isacharparenleft}{\kern0pt}SWAP\ {\isasymOtimes}\ {\isacharparenleft}{\kern0pt}{\isadigit{1}}\isactrlsub m\ {\isacharparenleft}{\kern0pt}{\isadigit{2}}{\isacharcircum}{\kern0pt}{\isacharparenleft}{\kern0pt}Suc\ n{\isacharparenright}{\kern0pt}{\isacharparenright}{\kern0pt}{\isacharparenright}{\kern0pt}{\isacharparenright}{\kern0pt}\ {\isacharasterisk}{\kern0pt}\ {\isacharparenleft}{\kern0pt}{\isacharparenleft}{\kern0pt}{\isacharparenleft}{\kern0pt}{\isadigit{1}}\isactrlsub m\ {\isadigit{2}}{\isacharparenright}{\kern0pt}\ {\isasymOtimes}\ {\isacharparenleft}{\kern0pt}SWAP{\isacharunderscore}{\kern0pt}up\ {\isacharparenleft}{\kern0pt}Suc\ {\isacharparenleft}{\kern0pt}Suc\ n{\isacharparenright}{\kern0pt}{\isacharparenright}{\kern0pt}{\isacharparenright}{\kern0pt}{\isacharparenright}{\kern0pt}\ {\isacharasterisk}{\kern0pt}\isanewline
\ \ \ \ \ \ \ \ \ \ \ \ \ \ \ \ \ \ \ \ \ \ \ \ {\isacharparenleft}{\kern0pt}\ {\isacharbar}{\kern0pt}state{\isacharunderscore}{\kern0pt}basis\ {\isacharparenleft}{\kern0pt}Suc\ {\isacharparenleft}{\kern0pt}Suc\ n{\isacharparenright}{\kern0pt}{\isacharparenright}{\kern0pt}\ {\isacharparenleft}{\kern0pt}j\ div\ {\isadigit{2}}{\isacharparenright}{\kern0pt}{\isasymrangle}\ {\isasymOtimes}\ {\isacharbar}{\kern0pt}state{\isacharunderscore}{\kern0pt}basis\ {\isadigit{1}}\ {\isacharparenleft}{\kern0pt}j\ mod\ {\isadigit{2}}{\isacharparenright}{\kern0pt}{\isasymrangle}{\isacharparenright}{\kern0pt}{\isacharparenright}{\kern0pt}{\isachardoublequoteclose}\isanewline
\ \ \ \ \ \ \ \ \ \ \isacommand{using}\isamarkupfalse%
\ assoc{\isacharunderscore}{\kern0pt}mult{\isacharunderscore}{\kern0pt}mat\isanewline
\ \ \ \ \ \ \ \ \ \ \isacommand{by}\isamarkupfalse%
\ {\isacharparenleft}{\kern0pt}smt\ {\isacharparenleft}{\kern0pt}verit{\isacharcomma}{\kern0pt}\ ccfv{\isacharunderscore}{\kern0pt}threshold{\isacharparenright}{\kern0pt}\ Groups{\isachardot}{\kern0pt}mult{\isacharunderscore}{\kern0pt}ac{\isacharparenleft}{\kern0pt}{\isadigit{2}}{\isacharparenright}{\kern0pt}\ Groups{\isachardot}{\kern0pt}mult{\isacharunderscore}{\kern0pt}ac{\isacharparenleft}{\kern0pt}{\isadigit{3}}{\isacharparenright}{\kern0pt}\ One{\isacharunderscore}{\kern0pt}nat{\isacharunderscore}{\kern0pt}def\ \isanewline
\ \ \ \ \ \ \ \ \ \ \ \ \ \ SWAP{\isacharunderscore}{\kern0pt}up{\isachardot}{\kern0pt}simps{\isacharparenleft}{\kern0pt}{\isadigit{3}}{\isacharparenright}{\kern0pt}\ SWAP{\isacharunderscore}{\kern0pt}up{\isacharunderscore}{\kern0pt}carrier{\isacharunderscore}{\kern0pt}mat\ carrier{\isacharunderscore}{\kern0pt}matD{\isacharparenleft}{\kern0pt}{\isadigit{2}}{\isacharparenright}{\kern0pt}\ carrier{\isacharunderscore}{\kern0pt}matI\ dim{\isacharunderscore}{\kern0pt}col{\isacharunderscore}{\kern0pt}tensor{\isacharunderscore}{\kern0pt}mat\ \isanewline
\ \ \ \ \ \ \ \ \ \ \ \ \ \ dim{\isacharunderscore}{\kern0pt}row{\isacharunderscore}{\kern0pt}mat{\isacharparenleft}{\kern0pt}{\isadigit{1}}{\isacharparenright}{\kern0pt}\ dim{\isacharunderscore}{\kern0pt}row{\isacharunderscore}{\kern0pt}tensor{\isacharunderscore}{\kern0pt}mat\ index{\isacharunderscore}{\kern0pt}mult{\isacharunderscore}{\kern0pt}mat{\isacharparenleft}{\kern0pt}{\isadigit{2}}{\isacharparenright}{\kern0pt}\ index{\isacharunderscore}{\kern0pt}one{\isacharunderscore}{\kern0pt}mat{\isacharparenleft}{\kern0pt}{\isadigit{3}}{\isacharparenright}{\kern0pt}\ \isanewline
\ \ \ \ \ \ \ \ \ \ \ \ \ \ index{\isacharunderscore}{\kern0pt}unit{\isacharunderscore}{\kern0pt}vec{\isacharparenleft}{\kern0pt}{\isadigit{3}}{\isacharparenright}{\kern0pt}\ ket{\isacharunderscore}{\kern0pt}vec{\isacharunderscore}{\kern0pt}def\ left{\isacharunderscore}{\kern0pt}mult{\isacharunderscore}{\kern0pt}one{\isacharunderscore}{\kern0pt}mat\ power{\isacharunderscore}{\kern0pt}Suc{\isadigit{2}}\ power{\isacharunderscore}{\kern0pt}one{\isacharunderscore}{\kern0pt}right\ \isanewline
\ \ \ \ \ \ \ \ \ \ \ \ \ \ state{\isacharunderscore}{\kern0pt}basis{\isacharunderscore}{\kern0pt}def{\isacharparenright}{\kern0pt}\isanewline
\ \ \ \ \ \ \ \ \isacommand{also}\isamarkupfalse%
\ \isacommand{have}\isamarkupfalse%
\ {\isachardoublequoteopen}{\isasymdots}\ {\isacharequal}{\kern0pt}\ {\isacharparenleft}{\kern0pt}SWAP\ {\isasymOtimes}\ {\isacharparenleft}{\kern0pt}{\isadigit{1}}\isactrlsub m\ {\isacharparenleft}{\kern0pt}{\isadigit{2}}{\isacharcircum}{\kern0pt}{\isacharparenleft}{\kern0pt}Suc\ n{\isacharparenright}{\kern0pt}{\isacharparenright}{\kern0pt}{\isacharparenright}{\kern0pt}{\isacharparenright}{\kern0pt}\ {\isacharasterisk}{\kern0pt}\ {\isacharparenleft}{\kern0pt}{\isacharparenleft}{\kern0pt}{\isacharparenleft}{\kern0pt}{\isadigit{1}}\isactrlsub m\ {\isadigit{2}}{\isacharparenright}{\kern0pt}\ {\isasymOtimes}\ {\isacharparenleft}{\kern0pt}SWAP{\isacharunderscore}{\kern0pt}up\ {\isacharparenleft}{\kern0pt}Suc\ {\isacharparenleft}{\kern0pt}Suc\ n{\isacharparenright}{\kern0pt}{\isacharparenright}{\kern0pt}{\isacharparenright}{\kern0pt}{\isacharparenright}{\kern0pt}\ {\isacharasterisk}{\kern0pt}\isanewline
\ \ \ \ \ \ \ \ \ \ \ \ \ \ \ \ \ \ \ \ \ \ \ \ {\isacharparenleft}{\kern0pt}{\isacharparenleft}{\kern0pt}\ {\isacharbar}{\kern0pt}state{\isacharunderscore}{\kern0pt}basis\ {\isadigit{1}}\ {\isacharparenleft}{\kern0pt}{\isacharparenleft}{\kern0pt}j\ div\ {\isadigit{2}}{\isacharparenright}{\kern0pt}\ div\ {\isadigit{2}}{\isacharcircum}{\kern0pt}Suc\ n{\isacharparenright}{\kern0pt}{\isasymrangle}\ {\isasymOtimes}\ \isanewline
\ \ \ \ \ \ \ \ \ \ \ \ \ \ \ \ \ \ \ \ \ \ \ \ \ \ \ {\isacharbar}{\kern0pt}state{\isacharunderscore}{\kern0pt}basis\ {\isacharparenleft}{\kern0pt}Suc\ n{\isacharparenright}{\kern0pt}\ {\isacharparenleft}{\kern0pt}{\isacharparenleft}{\kern0pt}j\ div\ {\isadigit{2}}{\isacharparenright}{\kern0pt}\ mod\ {\isadigit{2}}{\isacharcircum}{\kern0pt}Suc\ n{\isacharparenright}{\kern0pt}{\isasymrangle}{\isacharparenright}{\kern0pt}\isanewline
\ \ \ \ \ \ \ \ \ \ \ \ \ \ \ \ \ \ \ \ \ \ \ \ \ {\isasymOtimes}\ {\isacharbar}{\kern0pt}state{\isacharunderscore}{\kern0pt}basis\ {\isadigit{1}}\ {\isacharparenleft}{\kern0pt}j\ mod\ {\isadigit{2}}{\isacharparenright}{\kern0pt}{\isasymrangle}{\isacharparenright}{\kern0pt}{\isacharparenright}{\kern0pt}{\isachardoublequoteclose}\isanewline
\ \ \ \ \ \ \ \ \ \ \isacommand{using}\isamarkupfalse%
\ state{\isacharunderscore}{\kern0pt}basis{\isacharunderscore}{\kern0pt}dec\isanewline
\ \ \ \ \ \ \ \ \ \ \isacommand{by}\isamarkupfalse%
\ {\isacharparenleft}{\kern0pt}metis\ jl\ less{\isacharunderscore}{\kern0pt}mult{\isacharunderscore}{\kern0pt}imp{\isacharunderscore}{\kern0pt}div{\isacharunderscore}{\kern0pt}less\ power{\isacharunderscore}{\kern0pt}Suc{\isadigit{2}}{\isacharparenright}{\kern0pt}\isanewline
\ \ \ \ \ \ \ \ \isacommand{also}\isamarkupfalse%
\ \isacommand{have}\isamarkupfalse%
\ {\isachardoublequoteopen}{\isasymdots}\ {\isacharequal}{\kern0pt}\ {\isacharparenleft}{\kern0pt}SWAP\ {\isasymOtimes}\ {\isacharparenleft}{\kern0pt}{\isadigit{1}}\isactrlsub m\ {\isacharparenleft}{\kern0pt}{\isadigit{2}}{\isacharcircum}{\kern0pt}{\isacharparenleft}{\kern0pt}Suc\ n{\isacharparenright}{\kern0pt}{\isacharparenright}{\kern0pt}{\isacharparenright}{\kern0pt}{\isacharparenright}{\kern0pt}\ {\isacharasterisk}{\kern0pt}\ {\isacharparenleft}{\kern0pt}{\isacharparenleft}{\kern0pt}{\isacharparenleft}{\kern0pt}{\isadigit{1}}\isactrlsub m\ {\isadigit{2}}{\isacharparenright}{\kern0pt}\ {\isasymOtimes}\ {\isacharparenleft}{\kern0pt}SWAP{\isacharunderscore}{\kern0pt}up\ {\isacharparenleft}{\kern0pt}Suc\ {\isacharparenleft}{\kern0pt}Suc\ n{\isacharparenright}{\kern0pt}{\isacharparenright}{\kern0pt}{\isacharparenright}{\kern0pt}{\isacharparenright}{\kern0pt}\ {\isacharasterisk}{\kern0pt}\isanewline
\ \ \ \ \ \ \ \ \ \ \ \ \ \ \ \ \ \ \ \ \ \ \ \ {\isacharparenleft}{\kern0pt}\ {\isacharbar}{\kern0pt}state{\isacharunderscore}{\kern0pt}basis\ {\isadigit{1}}\ {\isacharparenleft}{\kern0pt}{\isacharparenleft}{\kern0pt}j\ div\ {\isadigit{2}}{\isacharparenright}{\kern0pt}\ div\ {\isadigit{2}}{\isacharcircum}{\kern0pt}Suc\ n{\isacharparenright}{\kern0pt}{\isasymrangle}\ {\isasymOtimes}\ \isanewline
\ \ \ \ \ \ \ \ \ \ \ \ \ \ \ \ \ \ \ \ \ \ \ \ \ {\isacharparenleft}{\kern0pt}\ {\isacharbar}{\kern0pt}state{\isacharunderscore}{\kern0pt}basis\ {\isacharparenleft}{\kern0pt}Suc\ n{\isacharparenright}{\kern0pt}\ {\isacharparenleft}{\kern0pt}{\isacharparenleft}{\kern0pt}j\ div\ {\isadigit{2}}{\isacharparenright}{\kern0pt}\ mod\ {\isadigit{2}}{\isacharcircum}{\kern0pt}Suc\ n{\isacharparenright}{\kern0pt}{\isasymrangle}\isanewline
\ \ \ \ \ \ \ \ \ \ \ \ \ \ \ \ \ \ \ \ \ \ \ \ \ {\isasymOtimes}\ {\isacharbar}{\kern0pt}state{\isacharunderscore}{\kern0pt}basis\ {\isadigit{1}}\ {\isacharparenleft}{\kern0pt}j\ mod\ {\isadigit{2}}{\isacharparenright}{\kern0pt}{\isasymrangle}{\isacharparenright}{\kern0pt}{\isacharparenright}{\kern0pt}{\isacharparenright}{\kern0pt}{\isachardoublequoteclose}\isanewline
\ \ \ \ \ \ \ \ \ \ \isacommand{using}\isamarkupfalse%
\ tensor{\isacharunderscore}{\kern0pt}mat{\isacharunderscore}{\kern0pt}is{\isacharunderscore}{\kern0pt}assoc\ state{\isacharunderscore}{\kern0pt}basis{\isacharunderscore}{\kern0pt}carrier{\isacharunderscore}{\kern0pt}mat\ \isacommand{by}\isamarkupfalse%
\ auto\isanewline
\ \ \ \ \ \ \ \ \isacommand{also}\isamarkupfalse%
\ \isacommand{have}\isamarkupfalse%
\ {\isachardoublequoteopen}{\isasymdots}\ {\isacharequal}{\kern0pt}\ {\isacharparenleft}{\kern0pt}SWAP\ {\isasymOtimes}\ {\isacharparenleft}{\kern0pt}{\isadigit{1}}\isactrlsub m\ {\isacharparenleft}{\kern0pt}{\isadigit{2}}{\isacharcircum}{\kern0pt}{\isacharparenleft}{\kern0pt}Suc\ n{\isacharparenright}{\kern0pt}{\isacharparenright}{\kern0pt}{\isacharparenright}{\kern0pt}{\isacharparenright}{\kern0pt}\ {\isacharasterisk}{\kern0pt}\ {\isacharparenleft}{\kern0pt}{\isacharparenleft}{\kern0pt}{\isacharparenleft}{\kern0pt}{\isadigit{1}}\isactrlsub m\ {\isadigit{2}}{\isacharparenright}{\kern0pt}\ {\isasymOtimes}\ {\isacharparenleft}{\kern0pt}SWAP{\isacharunderscore}{\kern0pt}up\ {\isacharparenleft}{\kern0pt}Suc\ {\isacharparenleft}{\kern0pt}Suc\ n{\isacharparenright}{\kern0pt}{\isacharparenright}{\kern0pt}{\isacharparenright}{\kern0pt}{\isacharparenright}{\kern0pt}\ {\isacharasterisk}{\kern0pt}\isanewline
\ \ \ \ \ \ \ \ \ \ \ \ \ \ \ \ \ \ \ \ \ \ \ \ {\isacharparenleft}{\kern0pt}\ {\isacharbar}{\kern0pt}state{\isacharunderscore}{\kern0pt}basis\ {\isadigit{1}}\ {\isacharparenleft}{\kern0pt}{\isacharparenleft}{\kern0pt}j\ div\ {\isadigit{2}}{\isacharparenright}{\kern0pt}\ div\ {\isadigit{2}}{\isacharcircum}{\kern0pt}Suc\ n{\isacharparenright}{\kern0pt}{\isasymrangle}\ {\isasymOtimes}\ \isanewline
\ \ \ \ \ \ \ \ \ \ \ \ \ \ \ \ \ \ \ \ \ \ \ \ {\isacharparenleft}{\kern0pt}\ {\isacharbar}{\kern0pt}state{\isacharunderscore}{\kern0pt}basis\ {\isacharparenleft}{\kern0pt}Suc\ n{\isacharparenright}{\kern0pt}\ {\isacharparenleft}{\kern0pt}{\isacharparenleft}{\kern0pt}j\ mod\ {\isadigit{2}}{\isacharcircum}{\kern0pt}Suc\ {\isacharparenleft}{\kern0pt}Suc\ n{\isacharparenright}{\kern0pt}{\isacharparenright}{\kern0pt}\ div\ {\isadigit{2}}{\isacharparenright}{\kern0pt}{\isasymrangle}\isanewline
\ \ \ \ \ \ \ \ \ \ \ \ \ \ \ \ \ \ \ \ \ \ \ \ {\isasymOtimes}\ {\isacharbar}{\kern0pt}state{\isacharunderscore}{\kern0pt}basis\ {\isadigit{1}}\ {\isacharparenleft}{\kern0pt}{\isacharparenleft}{\kern0pt}j\ mod\ {\isadigit{2}}{\isacharcircum}{\kern0pt}Suc\ {\isacharparenleft}{\kern0pt}Suc\ n{\isacharparenright}{\kern0pt}{\isacharparenright}{\kern0pt}\ mod\ {\isadigit{2}}{\isacharparenright}{\kern0pt}{\isasymrangle}{\isacharparenright}{\kern0pt}{\isacharparenright}{\kern0pt}{\isacharparenright}{\kern0pt}{\isachardoublequoteclose}\isanewline
\ \ \ \ \ \ \ \ \ \ \isacommand{using}\isamarkupfalse%
\ jl\ power{\isacharunderscore}{\kern0pt}Suc\ power{\isacharunderscore}{\kern0pt}add\ power{\isacharunderscore}{\kern0pt}one{\isacharunderscore}{\kern0pt}right\isanewline
\ \ \ \ \ \ \ \ \ \ \isacommand{by}\isamarkupfalse%
\ {\isacharparenleft}{\kern0pt}smt\ {\isacharparenleft}{\kern0pt}z{\isadigit{3}}{\isacharparenright}{\kern0pt}\ Suc{\isacharunderscore}{\kern0pt}{\isadigit{1}}\ add{\isacharunderscore}{\kern0pt}{\isadigit{0}}\ div{\isacharunderscore}{\kern0pt}Suc\ div{\isacharunderscore}{\kern0pt}exp{\isacharunderscore}{\kern0pt}mod{\isacharunderscore}{\kern0pt}exp{\isacharunderscore}{\kern0pt}eq\ lessI\ mod{\isacharunderscore}{\kern0pt}less\ mod{\isacharunderscore}{\kern0pt}mod{\isacharunderscore}{\kern0pt}cancel\ \isanewline
\ \ \ \ \ \ \ \ \ \ \ \ \ \ mod{\isacharunderscore}{\kern0pt}mult{\isacharunderscore}{\kern0pt}self{\isadigit{2}}\ n{\isacharunderscore}{\kern0pt}not{\isacharunderscore}{\kern0pt}Suc{\isacharunderscore}{\kern0pt}n\ odd{\isacharunderscore}{\kern0pt}Suc{\isacharunderscore}{\kern0pt}div{\isacharunderscore}{\kern0pt}two\ plus{\isacharunderscore}{\kern0pt}{\isadigit{1}}{\isacharunderscore}{\kern0pt}eq{\isacharunderscore}{\kern0pt}Suc{\isacharparenright}{\kern0pt}\isanewline
\ \ \ \ \ \ \ \ \isacommand{also}\isamarkupfalse%
\ \isacommand{have}\isamarkupfalse%
\ {\isachardoublequoteopen}{\isasymdots}\ {\isacharequal}{\kern0pt}\ {\isacharparenleft}{\kern0pt}SWAP\ {\isasymOtimes}\ {\isacharparenleft}{\kern0pt}{\isadigit{1}}\isactrlsub m\ {\isacharparenleft}{\kern0pt}{\isadigit{2}}{\isacharcircum}{\kern0pt}{\isacharparenleft}{\kern0pt}Suc\ n{\isacharparenright}{\kern0pt}{\isacharparenright}{\kern0pt}{\isacharparenright}{\kern0pt}{\isacharparenright}{\kern0pt}\ {\isacharasterisk}{\kern0pt}\isanewline
\ \ \ \ \ \ \ \ \ \ \ \ \ \ \ \ \ \ \ \ \ \ \ \ {\isacharparenleft}{\kern0pt}{\isacharparenleft}{\kern0pt}{\isacharparenleft}{\kern0pt}{\isadigit{1}}\isactrlsub m\ {\isadigit{2}}{\isacharparenright}{\kern0pt}\ {\isacharasterisk}{\kern0pt}\ {\isacharbar}{\kern0pt}state{\isacharunderscore}{\kern0pt}basis\ {\isadigit{1}}\ {\isacharparenleft}{\kern0pt}{\isacharparenleft}{\kern0pt}j\ div\ {\isadigit{2}}{\isacharparenright}{\kern0pt}\ div\ {\isadigit{2}}{\isacharcircum}{\kern0pt}Suc\ n{\isacharparenright}{\kern0pt}{\isasymrangle}{\isacharparenright}{\kern0pt}\ {\isasymOtimes}\isanewline
\ \ \ \ \ \ \ \ \ \ \ \ \ \ \ \ \ \ \ \ \ \ \ \ {\isacharparenleft}{\kern0pt}{\isacharparenleft}{\kern0pt}SWAP{\isacharunderscore}{\kern0pt}up\ {\isacharparenleft}{\kern0pt}Suc\ {\isacharparenleft}{\kern0pt}Suc\ n{\isacharparenright}{\kern0pt}{\isacharparenright}{\kern0pt}{\isacharparenright}{\kern0pt}{\isacharparenright}{\kern0pt}\ {\isacharasterisk}{\kern0pt}\isanewline
\ \ \ \ \ \ \ \ \ \ \ \ \ \ \ \ \ \ \ \ \ \ \ \ {\isacharparenleft}{\kern0pt}\ {\isacharbar}{\kern0pt}state{\isacharunderscore}{\kern0pt}basis\ {\isacharparenleft}{\kern0pt}Suc\ n{\isacharparenright}{\kern0pt}\ {\isacharparenleft}{\kern0pt}{\isacharparenleft}{\kern0pt}j\ mod\ {\isadigit{2}}{\isacharcircum}{\kern0pt}Suc\ {\isacharparenleft}{\kern0pt}Suc\ n{\isacharparenright}{\kern0pt}{\isacharparenright}{\kern0pt}\ div\ {\isadigit{2}}{\isacharparenright}{\kern0pt}{\isasymrangle}\isanewline
\ \ \ \ \ \ \ \ \ \ \ \ \ \ \ \ \ \ \ \ \ \ \ \ {\isasymOtimes}\ {\isacharbar}{\kern0pt}state{\isacharunderscore}{\kern0pt}basis\ {\isadigit{1}}\ {\isacharparenleft}{\kern0pt}{\isacharparenleft}{\kern0pt}j\ mod\ {\isadigit{2}}{\isacharcircum}{\kern0pt}Suc\ {\isacharparenleft}{\kern0pt}Suc\ n{\isacharparenright}{\kern0pt}{\isacharparenright}{\kern0pt}\ mod\ {\isadigit{2}}{\isacharparenright}{\kern0pt}{\isasymrangle}{\isacharparenright}{\kern0pt}{\isacharparenright}{\kern0pt}{\isachardoublequoteclose}\isanewline
\ \ \ \ \ \ \ \ \ \ \isacommand{using}\isamarkupfalse%
\ mult{\isacharunderscore}{\kern0pt}distr{\isacharunderscore}{\kern0pt}tensor\isanewline
\ \ \ \ \ \ \ \ \ \ \isacommand{by}\isamarkupfalse%
\ {\isacharparenleft}{\kern0pt}metis\ SWAP{\isacharunderscore}{\kern0pt}up{\isacharunderscore}{\kern0pt}carrier{\isacharunderscore}{\kern0pt}mat\ carrier{\isacharunderscore}{\kern0pt}matD{\isacharparenleft}{\kern0pt}{\isadigit{1}}{\isacharparenright}{\kern0pt}\ carrier{\isacharunderscore}{\kern0pt}matD{\isacharparenleft}{\kern0pt}{\isadigit{2}}{\isacharparenright}{\kern0pt}\ index{\isacharunderscore}{\kern0pt}one{\isacharunderscore}{\kern0pt}mat{\isacharparenleft}{\kern0pt}{\isadigit{3}}{\isacharparenright}{\kern0pt}\ \isanewline
\ \ \ \ \ \ \ \ \ \ \ \ \ \ less{\isacharunderscore}{\kern0pt}numeral{\isacharunderscore}{\kern0pt}extra{\isacharparenleft}{\kern0pt}{\isadigit{1}}{\isacharparenright}{\kern0pt}\ mod{\isacharunderscore}{\kern0pt}less{\isacharunderscore}{\kern0pt}divisor\ pos{\isadigit{2}}\ power{\isacharunderscore}{\kern0pt}one{\isacharunderscore}{\kern0pt}right\ state{\isacharunderscore}{\kern0pt}basis{\isacharunderscore}{\kern0pt}carrier{\isacharunderscore}{\kern0pt}mat\ \isanewline
\ \ \ \ \ \ \ \ \ \ \ \ \ \ state{\isacharunderscore}{\kern0pt}basis{\isacharunderscore}{\kern0pt}dec{\isacharprime}{\kern0pt}\ zero{\isacharunderscore}{\kern0pt}less{\isacharunderscore}{\kern0pt}power{\isacharparenright}{\kern0pt}\isanewline
\ \ \ \ \ \ \ \ \isacommand{also}\isamarkupfalse%
\ \isacommand{have}\isamarkupfalse%
\ {\isachardoublequoteopen}{\isasymdots}\ {\isacharequal}{\kern0pt}\ {\isacharparenleft}{\kern0pt}SWAP\ {\isasymOtimes}\ {\isacharparenleft}{\kern0pt}{\isadigit{1}}\isactrlsub m\ {\isacharparenleft}{\kern0pt}{\isadigit{2}}{\isacharcircum}{\kern0pt}{\isacharparenleft}{\kern0pt}Suc\ n{\isacharparenright}{\kern0pt}{\isacharparenright}{\kern0pt}{\isacharparenright}{\kern0pt}{\isacharparenright}{\kern0pt}\ {\isacharasterisk}{\kern0pt}\isanewline
\ \ \ \ \ \ \ \ \ \ \ \ \ \ \ \ \ \ \ \ \ \ \ \ {\isacharparenleft}{\kern0pt}\ {\isacharbar}{\kern0pt}state{\isacharunderscore}{\kern0pt}basis\ {\isadigit{1}}\ {\isacharparenleft}{\kern0pt}{\isacharparenleft}{\kern0pt}j\ div\ {\isadigit{2}}{\isacharparenright}{\kern0pt}\ div\ {\isadigit{2}}{\isacharcircum}{\kern0pt}Suc\ n{\isacharparenright}{\kern0pt}{\isasymrangle}\ {\isasymOtimes}\isanewline
\ \ \ \ \ \ \ \ \ \ \ \ \ \ \ \ \ \ \ \ \ \ \ \ {\isacharparenleft}{\kern0pt}\ {\isacharbar}{\kern0pt}state{\isacharunderscore}{\kern0pt}basis\ {\isadigit{1}}\ {\isacharparenleft}{\kern0pt}{\isacharparenleft}{\kern0pt}j\ mod\ {\isadigit{2}}{\isacharcircum}{\kern0pt}Suc\ {\isacharparenleft}{\kern0pt}Suc\ n{\isacharparenright}{\kern0pt}{\isacharparenright}{\kern0pt}\ mod\ {\isadigit{2}}{\isacharparenright}{\kern0pt}{\isasymrangle}\ {\isasymOtimes}\isanewline
\ \ \ \ \ \ \ \ \ \ \ \ \ \ \ \ \ \ \ \ \ \ \ \ \ \ {\isacharbar}{\kern0pt}state{\isacharunderscore}{\kern0pt}basis\ {\isacharparenleft}{\kern0pt}Suc\ n{\isacharparenright}{\kern0pt}\ {\isacharparenleft}{\kern0pt}{\isacharparenleft}{\kern0pt}j\ mod\ {\isadigit{2}}{\isacharcircum}{\kern0pt}Suc\ {\isacharparenleft}{\kern0pt}Suc\ n{\isacharparenright}{\kern0pt}{\isacharparenright}{\kern0pt}\ div\ {\isadigit{2}}{\isacharparenright}{\kern0pt}{\isasymrangle}{\isacharparenright}{\kern0pt}{\isacharparenright}{\kern0pt}{\isachardoublequoteclose}\isanewline
\ \ \ \ \ \ \ \ \ \ \isacommand{using}\isamarkupfalse%
\ HI\isanewline
\ \ \ \ \ \ \ \ \ \ \isacommand{by}\isamarkupfalse%
\ {\isacharparenleft}{\kern0pt}metis\ left{\isacharunderscore}{\kern0pt}mult{\isacharunderscore}{\kern0pt}one{\isacharunderscore}{\kern0pt}mat\ mod{\isacharunderscore}{\kern0pt}less{\isacharunderscore}{\kern0pt}divisor\ pos{\isadigit{2}}\ power{\isacharunderscore}{\kern0pt}one{\isacharunderscore}{\kern0pt}right\ state{\isacharunderscore}{\kern0pt}basis{\isacharunderscore}{\kern0pt}carrier{\isacharunderscore}{\kern0pt}mat\isanewline
\ \ \ \ \ \ \ \ \ \ \ \ \ \ zero{\isacharunderscore}{\kern0pt}less{\isacharunderscore}{\kern0pt}power{\isacharparenright}{\kern0pt}\isanewline
\ \ \ \ \ \ \ \ \isacommand{also}\isamarkupfalse%
\ \isacommand{have}\isamarkupfalse%
\ {\isachardoublequoteopen}{\isasymdots}\ {\isacharequal}{\kern0pt}\ {\isacharparenleft}{\kern0pt}SWAP\ {\isasymOtimes}\ {\isacharparenleft}{\kern0pt}{\isadigit{1}}\isactrlsub m\ {\isacharparenleft}{\kern0pt}{\isadigit{2}}{\isacharcircum}{\kern0pt}{\isacharparenleft}{\kern0pt}Suc\ n{\isacharparenright}{\kern0pt}{\isacharparenright}{\kern0pt}{\isacharparenright}{\kern0pt}{\isacharparenright}{\kern0pt}\ {\isacharasterisk}{\kern0pt}\isanewline
\ \ \ \ \ \ \ \ \ \ \ \ \ \ \ \ \ \ \ \ \ \ \ \ {\isacharparenleft}{\kern0pt}{\isacharparenleft}{\kern0pt}\ {\isacharbar}{\kern0pt}state{\isacharunderscore}{\kern0pt}basis\ {\isadigit{1}}\ {\isacharparenleft}{\kern0pt}{\isacharparenleft}{\kern0pt}j\ div\ {\isadigit{2}}{\isacharparenright}{\kern0pt}\ div\ {\isadigit{2}}{\isacharcircum}{\kern0pt}Suc\ n{\isacharparenright}{\kern0pt}{\isasymrangle}\ {\isasymOtimes}\isanewline
\ \ \ \ \ \ \ \ \ \ \ \ \ \ \ \ \ \ \ \ \ \ \ \ \ \ \ {\isacharbar}{\kern0pt}state{\isacharunderscore}{\kern0pt}basis\ {\isadigit{1}}\ {\isacharparenleft}{\kern0pt}{\isacharparenleft}{\kern0pt}j\ mod\ {\isadigit{2}}{\isacharcircum}{\kern0pt}Suc\ {\isacharparenleft}{\kern0pt}Suc\ n{\isacharparenright}{\kern0pt}{\isacharparenright}{\kern0pt}\ mod\ {\isadigit{2}}{\isacharparenright}{\kern0pt}{\isasymrangle}{\isacharparenright}{\kern0pt}\ {\isasymOtimes}\isanewline
\ \ \ \ \ \ \ \ \ \ \ \ \ \ \ \ \ \ \ \ \ \ \ \ \ \ \ {\isacharbar}{\kern0pt}state{\isacharunderscore}{\kern0pt}basis\ {\isacharparenleft}{\kern0pt}Suc\ n{\isacharparenright}{\kern0pt}\ {\isacharparenleft}{\kern0pt}{\isacharparenleft}{\kern0pt}j\ mod\ {\isadigit{2}}{\isacharcircum}{\kern0pt}Suc\ {\isacharparenleft}{\kern0pt}Suc\ n{\isacharparenright}{\kern0pt}{\isacharparenright}{\kern0pt}\ div\ {\isadigit{2}}{\isacharparenright}{\kern0pt}{\isasymrangle}{\isacharparenright}{\kern0pt}{\isachardoublequoteclose}\isanewline
\ \ \ \ \ \ \ \ \ \ \isacommand{using}\isamarkupfalse%
\ tensor{\isacharunderscore}{\kern0pt}mat{\isacharunderscore}{\kern0pt}is{\isacharunderscore}{\kern0pt}assoc\ \isacommand{by}\isamarkupfalse%
\ simp\isanewline
\ \ \ \ \ \ \ \ \isacommand{also}\isamarkupfalse%
\ \isacommand{have}\isamarkupfalse%
\ {\isachardoublequoteopen}{\isasymdots}\ {\isacharequal}{\kern0pt}\ {\isacharparenleft}{\kern0pt}SWAP\ {\isacharasterisk}{\kern0pt}\ {\isacharparenleft}{\kern0pt}\ {\isacharbar}{\kern0pt}state{\isacharunderscore}{\kern0pt}basis\ {\isadigit{1}}\ {\isacharparenleft}{\kern0pt}{\isacharparenleft}{\kern0pt}j\ div\ {\isadigit{2}}{\isacharparenright}{\kern0pt}\ div\ {\isadigit{2}}{\isacharcircum}{\kern0pt}Suc\ n{\isacharparenright}{\kern0pt}{\isasymrangle}\ {\isasymOtimes}\isanewline
\ \ \ \ \ \ \ \ \ \ \ \ \ \ \ \ \ \ \ \ \ \ \ \ \ \ \ \ \ \ \ \ \ \ {\isacharbar}{\kern0pt}state{\isacharunderscore}{\kern0pt}basis\ {\isadigit{1}}\ {\isacharparenleft}{\kern0pt}{\isacharparenleft}{\kern0pt}j\ mod\ {\isadigit{2}}{\isacharcircum}{\kern0pt}Suc\ {\isacharparenleft}{\kern0pt}Suc\ n{\isacharparenright}{\kern0pt}{\isacharparenright}{\kern0pt}\ mod\ {\isadigit{2}}{\isacharparenright}{\kern0pt}{\isasymrangle}{\isacharparenright}{\kern0pt}{\isacharparenright}{\kern0pt}\ {\isasymOtimes}\isanewline
\ \ \ \ \ \ \ \ \ \ \ \ \ \ \ \ \ \ \ \ \ \ \ \ {\isacharparenleft}{\kern0pt}{\isacharparenleft}{\kern0pt}{\isadigit{1}}\isactrlsub m\ {\isacharparenleft}{\kern0pt}{\isadigit{2}}{\isacharcircum}{\kern0pt}{\isacharparenleft}{\kern0pt}Suc\ n{\isacharparenright}{\kern0pt}{\isacharparenright}{\kern0pt}{\isacharparenright}{\kern0pt}\ {\isacharasterisk}{\kern0pt}\ {\isacharbar}{\kern0pt}state{\isacharunderscore}{\kern0pt}basis\ {\isacharparenleft}{\kern0pt}Suc\ n{\isacharparenright}{\kern0pt}\ {\isacharparenleft}{\kern0pt}{\isacharparenleft}{\kern0pt}j\ mod\ {\isadigit{2}}{\isacharcircum}{\kern0pt}Suc\ {\isacharparenleft}{\kern0pt}Suc\ n{\isacharparenright}{\kern0pt}{\isacharparenright}{\kern0pt}\ div\ {\isadigit{2}}{\isacharparenright}{\kern0pt}{\isasymrangle}{\isacharparenright}{\kern0pt}{\isachardoublequoteclose}\isanewline
\ \ \ \ \ \ \ \ \ \ \isacommand{using}\isamarkupfalse%
\ mult{\isacharunderscore}{\kern0pt}distr{\isacharunderscore}{\kern0pt}tensor\ \isanewline
\ \ \ \ \ \ \ \ \ \ \isacommand{by}\isamarkupfalse%
\ {\isacharparenleft}{\kern0pt}smt\ {\isacharparenleft}{\kern0pt}verit{\isacharcomma}{\kern0pt}\ del{\isacharunderscore}{\kern0pt}insts{\isacharparenright}{\kern0pt}\ One{\isacharunderscore}{\kern0pt}nat{\isacharunderscore}{\kern0pt}def\ SWAP{\isacharunderscore}{\kern0pt}ncols\ SWAP{\isacharunderscore}{\kern0pt}nrows\ SWAP{\isacharunderscore}{\kern0pt}tensor\ carrier{\isacharunderscore}{\kern0pt}matD{\isacharparenleft}{\kern0pt}{\isadigit{2}}{\isacharparenright}{\kern0pt}\ \isanewline
\ \ \ \ \ \ \ \ \ \ \ \ \ \ dim{\isacharunderscore}{\kern0pt}col{\isacharunderscore}{\kern0pt}tensor{\isacharunderscore}{\kern0pt}mat\ dim{\isacharunderscore}{\kern0pt}row{\isacharunderscore}{\kern0pt}mat{\isacharparenleft}{\kern0pt}{\isadigit{1}}{\isacharparenright}{\kern0pt}\ dim{\isacharunderscore}{\kern0pt}row{\isacharunderscore}{\kern0pt}tensor{\isacharunderscore}{\kern0pt}mat\ index{\isacharunderscore}{\kern0pt}mult{\isacharunderscore}{\kern0pt}mat{\isacharparenleft}{\kern0pt}{\isadigit{2}}{\isacharparenright}{\kern0pt}\ \isanewline
\ \ \ \ \ \ \ \ \ \ \ \ \ \ index{\isacharunderscore}{\kern0pt}one{\isacharunderscore}{\kern0pt}mat{\isacharparenleft}{\kern0pt}{\isadigit{3}}{\isacharparenright}{\kern0pt}\ index{\isacharunderscore}{\kern0pt}unit{\isacharunderscore}{\kern0pt}vec{\isacharparenleft}{\kern0pt}{\isadigit{3}}{\isacharparenright}{\kern0pt}\ ket{\isacharunderscore}{\kern0pt}vec{\isacharunderscore}{\kern0pt}def\ lessI\ one{\isacharunderscore}{\kern0pt}power{\isadigit{2}}\ pos{\isadigit{2}}\ power{\isacharunderscore}{\kern0pt}Suc{\isadigit{2}}\ \isanewline
\ \ \ \ \ \ \ \ \ \ \ \ \ \ power{\isacharunderscore}{\kern0pt}one{\isacharunderscore}{\kern0pt}right\ state{\isacharunderscore}{\kern0pt}basis{\isacharunderscore}{\kern0pt}carrier{\isacharunderscore}{\kern0pt}mat\ state{\isacharunderscore}{\kern0pt}basis{\isacharunderscore}{\kern0pt}def\ zero{\isacharunderscore}{\kern0pt}less{\isacharunderscore}{\kern0pt}power{\isacharparenright}{\kern0pt}\isanewline
\ \ \ \ \ \ \ \ \isacommand{also}\isamarkupfalse%
\ \isacommand{have}\isamarkupfalse%
\ {\isachardoublequoteopen}{\isasymdots}\ {\isacharequal}{\kern0pt}\ {\isacharparenleft}{\kern0pt}\ {\isacharbar}{\kern0pt}state{\isacharunderscore}{\kern0pt}basis\ {\isadigit{1}}\ {\isacharparenleft}{\kern0pt}{\isacharparenleft}{\kern0pt}j\ mod\ {\isadigit{2}}{\isacharcircum}{\kern0pt}Suc\ {\isacharparenleft}{\kern0pt}Suc\ n{\isacharparenright}{\kern0pt}{\isacharparenright}{\kern0pt}\ mod\ {\isadigit{2}}{\isacharparenright}{\kern0pt}{\isasymrangle}\ {\isasymOtimes}\isanewline
\ \ \ \ \ \ \ \ \ \ \ \ \ \ \ \ \ \ \ \ \ \ \ \ \ \ {\isacharbar}{\kern0pt}state{\isacharunderscore}{\kern0pt}basis\ {\isadigit{1}}\ {\isacharparenleft}{\kern0pt}{\isacharparenleft}{\kern0pt}j\ div\ {\isadigit{2}}{\isacharparenright}{\kern0pt}\ div\ {\isadigit{2}}{\isacharcircum}{\kern0pt}Suc\ n{\isacharparenright}{\kern0pt}{\isasymrangle}{\isacharparenright}{\kern0pt}\ {\isasymOtimes}\isanewline
\ \ \ \ \ \ \ \ \ \ \ \ \ \ \ \ \ \ \ \ \ \ \ \ \ \ {\isacharbar}{\kern0pt}state{\isacharunderscore}{\kern0pt}basis\ {\isacharparenleft}{\kern0pt}Suc\ n{\isacharparenright}{\kern0pt}\ {\isacharparenleft}{\kern0pt}{\isacharparenleft}{\kern0pt}j\ mod\ {\isadigit{2}}{\isacharcircum}{\kern0pt}Suc\ {\isacharparenleft}{\kern0pt}Suc\ n{\isacharparenright}{\kern0pt}{\isacharparenright}{\kern0pt}\ div\ {\isadigit{2}}{\isacharparenright}{\kern0pt}{\isasymrangle}{\isachardoublequoteclose}\isanewline
\ \ \ \ \ \ \ \ \ \ \isacommand{using}\isamarkupfalse%
\ SWAP{\isacharunderscore}{\kern0pt}tensor\isanewline
\ \ \ \ \ \ \ \ \ \ \isacommand{by}\isamarkupfalse%
\ {\isacharparenleft}{\kern0pt}metis\ left{\isacharunderscore}{\kern0pt}mult{\isacharunderscore}{\kern0pt}one{\isacharunderscore}{\kern0pt}mat\ power{\isacharunderscore}{\kern0pt}one{\isacharunderscore}{\kern0pt}right\ state{\isacharunderscore}{\kern0pt}basis{\isacharunderscore}{\kern0pt}carrier{\isacharunderscore}{\kern0pt}mat{\isacharparenright}{\kern0pt}\isanewline
\ \ \ \ \ \ \ \ \isacommand{also}\isamarkupfalse%
\ \isacommand{have}\isamarkupfalse%
\ {\isachardoublequoteopen}{\isasymdots}\ {\isacharequal}{\kern0pt}\ {\isacharbar}{\kern0pt}state{\isacharunderscore}{\kern0pt}basis\ {\isadigit{1}}\ {\isacharparenleft}{\kern0pt}{\isacharparenleft}{\kern0pt}j\ mod\ {\isadigit{2}}{\isacharcircum}{\kern0pt}Suc\ {\isacharparenleft}{\kern0pt}Suc\ n{\isacharparenright}{\kern0pt}{\isacharparenright}{\kern0pt}\ mod\ {\isadigit{2}}{\isacharparenright}{\kern0pt}{\isasymrangle}\ {\isasymOtimes}\isanewline
\ \ \ \ \ \ \ \ \ \ \ \ \ \ \ \ \ \ \ \ \ \ {\isacharparenleft}{\kern0pt}\ {\isacharbar}{\kern0pt}state{\isacharunderscore}{\kern0pt}basis\ {\isadigit{1}}\ {\isacharparenleft}{\kern0pt}{\isacharparenleft}{\kern0pt}j\ div\ {\isadigit{2}}{\isacharparenright}{\kern0pt}\ div\ {\isadigit{2}}{\isacharcircum}{\kern0pt}Suc\ n{\isacharparenright}{\kern0pt}{\isasymrangle}\ {\isasymOtimes}\isanewline
\ \ \ \ \ \ \ \ \ \ \ \ \ \ \ \ \ \ \ \ \ \ \ \ {\isacharbar}{\kern0pt}state{\isacharunderscore}{\kern0pt}basis\ {\isacharparenleft}{\kern0pt}Suc\ n{\isacharparenright}{\kern0pt}\ {\isacharparenleft}{\kern0pt}{\isacharparenleft}{\kern0pt}j\ mod\ {\isadigit{2}}{\isacharcircum}{\kern0pt}Suc\ {\isacharparenleft}{\kern0pt}Suc\ n{\isacharparenright}{\kern0pt}{\isacharparenright}{\kern0pt}\ div\ {\isadigit{2}}{\isacharparenright}{\kern0pt}{\isasymrangle}{\isacharparenright}{\kern0pt}{\isachardoublequoteclose}\isanewline
\ \ \ \ \ \ \ \ \ \ \isacommand{using}\isamarkupfalse%
\ tensor{\isacharunderscore}{\kern0pt}mat{\isacharunderscore}{\kern0pt}is{\isacharunderscore}{\kern0pt}assoc\ \isacommand{by}\isamarkupfalse%
\ simp\isanewline
\ \ \ \ \ \ \ \ \isacommand{also}\isamarkupfalse%
\ \isacommand{have}\isamarkupfalse%
\ {\isachardoublequoteopen}{\isasymdots}\ {\isacharequal}{\kern0pt}\ {\isacharbar}{\kern0pt}state{\isacharunderscore}{\kern0pt}basis\ {\isadigit{1}}\ {\isacharparenleft}{\kern0pt}j\ mod\ {\isadigit{2}}{\isacharparenright}{\kern0pt}{\isasymrangle}\ {\isasymOtimes}\isanewline
\ \ \ \ \ \ \ \ \ \ \ \ \ \ \ \ \ \ \ \ \ \ {\isacharparenleft}{\kern0pt}\ {\isacharbar}{\kern0pt}state{\isacharunderscore}{\kern0pt}basis\ {\isadigit{1}}\ {\isacharparenleft}{\kern0pt}{\isacharparenleft}{\kern0pt}j\ div\ {\isadigit{2}}{\isacharparenright}{\kern0pt}\ div\ {\isadigit{2}}{\isacharcircum}{\kern0pt}Suc\ n{\isacharparenright}{\kern0pt}{\isasymrangle}\ {\isasymOtimes}\isanewline
\ \ \ \ \ \ \ \ \ \ \ \ \ \ \ \ \ \ \ \ \ \ \ \ {\isacharbar}{\kern0pt}state{\isacharunderscore}{\kern0pt}basis\ {\isacharparenleft}{\kern0pt}Suc\ n{\isacharparenright}{\kern0pt}\ {\isacharparenleft}{\kern0pt}{\isacharparenleft}{\kern0pt}j\ div\ {\isadigit{2}}{\isacharparenright}{\kern0pt}\ mod\ {\isadigit{2}}{\isacharcircum}{\kern0pt}Suc\ n{\isacharparenright}{\kern0pt}{\isasymrangle}{\isacharparenright}{\kern0pt}{\isachardoublequoteclose}\ \isanewline
\ \ \ \ \ \ \ \ \isacommand{proof}\isamarkupfalse%
\ {\isacharminus}{\kern0pt}\isanewline
\ \ \ \ \ \ \ \ \ \ \isacommand{have}\isamarkupfalse%
\ f{\isadigit{1}}{\isacharcolon}{\kern0pt}\ {\isachardoublequoteopen}{\isasymforall}n\ na{\isachardot}{\kern0pt}\ {\isacharparenleft}{\kern0pt}n{\isacharcolon}{\kern0pt}{\isacharcolon}{\kern0pt}nat{\isacharparenright}{\kern0pt}\ {\isacharcircum}{\kern0pt}\ {\isacharparenleft}{\kern0pt}{\isadigit{1}}\ {\isacharplus}{\kern0pt}\ na{\isacharparenright}{\kern0pt}\ {\isacharequal}{\kern0pt}\ n\ {\isacharcircum}{\kern0pt}\ Suc\ na{\isachardoublequoteclose}\isanewline
\ \ \ \ \ \ \ \ \ \ \ \ \isacommand{by}\isamarkupfalse%
\ simp\isanewline
\ \ \ \ \ \ \ \ \ \ \isacommand{have}\isamarkupfalse%
\ {\isachardoublequoteopen}{\isasymforall}n\ na{\isachardot}{\kern0pt}\ {\isacharparenleft}{\kern0pt}n{\isacharcolon}{\kern0pt}{\isacharcolon}{\kern0pt}nat{\isacharparenright}{\kern0pt}\ dvd\ n\ {\isacharcircum}{\kern0pt}\ Suc\ na{\isachardoublequoteclose}\isanewline
\ \ \ \ \ \ \ \ \ \ \ \ \isacommand{by}\isamarkupfalse%
\ simp\isanewline
\ \ \ \ \ \ \ \ \ \ \isacommand{then}\isamarkupfalse%
\ \isacommand{show}\isamarkupfalse%
\ {\isacharquery}{\kern0pt}thesis\isanewline
\ \ \ \ \ \ \ \ \ \ \ \ \isacommand{using}\isamarkupfalse%
\ f{\isadigit{1}}\ \isacommand{by}\isamarkupfalse%
\ {\isacharparenleft}{\kern0pt}smt\ {\isacharparenleft}{\kern0pt}z{\isadigit{3}}{\isacharparenright}{\kern0pt}\ div{\isacharunderscore}{\kern0pt}exp{\isacharunderscore}{\kern0pt}mod{\isacharunderscore}{\kern0pt}exp{\isacharunderscore}{\kern0pt}eq\ mod{\isacharunderscore}{\kern0pt}mod{\isacharunderscore}{\kern0pt}cancel\ power{\isacharunderscore}{\kern0pt}one{\isacharunderscore}{\kern0pt}right{\isacharparenright}{\kern0pt}\isanewline
\ \ \ \ \ \ \ \ \isacommand{qed}\isamarkupfalse%
\isanewline
\ \ \ \ \ \ \ \ \isacommand{also}\isamarkupfalse%
\ \isacommand{have}\isamarkupfalse%
\ {\isachardoublequoteopen}{\isasymdots}\ {\isacharequal}{\kern0pt}\ {\isacharbar}{\kern0pt}state{\isacharunderscore}{\kern0pt}basis\ {\isadigit{1}}\ {\isacharparenleft}{\kern0pt}j\ mod\ {\isadigit{2}}{\isacharparenright}{\kern0pt}{\isasymrangle}\ {\isasymOtimes}\ {\isacharbar}{\kern0pt}state{\isacharunderscore}{\kern0pt}basis\ {\isacharparenleft}{\kern0pt}Suc\ {\isacharparenleft}{\kern0pt}Suc\ n{\isacharparenright}{\kern0pt}{\isacharparenright}{\kern0pt}\ {\isacharparenleft}{\kern0pt}j\ div\ {\isadigit{2}}{\isacharparenright}{\kern0pt}{\isasymrangle}{\isachardoublequoteclose}\isanewline
\ \ \ \ \ \ \ \ \ \ \isacommand{using}\isamarkupfalse%
\ state{\isacharunderscore}{\kern0pt}basis{\isacharunderscore}{\kern0pt}dec\ jl\isanewline
\ \ \ \ \ \ \ \ \ \ \isacommand{by}\isamarkupfalse%
\ {\isacharparenleft}{\kern0pt}metis\ less{\isacharunderscore}{\kern0pt}mult{\isacharunderscore}{\kern0pt}imp{\isacharunderscore}{\kern0pt}div{\isacharunderscore}{\kern0pt}less\ power{\isacharunderscore}{\kern0pt}Suc{\isadigit{2}}{\isacharparenright}{\kern0pt}\isanewline
\ \ \ \ \ \ \ \ \isacommand{finally}\isamarkupfalse%
\ \isacommand{show}\isamarkupfalse%
\ {\isacharquery}{\kern0pt}thesis\ \isacommand{by}\isamarkupfalse%
\ this\isanewline
\ \ \ \ \ \ \isacommand{qed}\isamarkupfalse%
\isanewline
\ \ \ \ \isacommand{qed}\isamarkupfalse%
\isanewline
\ \ \isacommand{qed}\isamarkupfalse%
\isanewline
\isacommand{qed}\isamarkupfalse%
%
\endisatagproof
{\isafoldproof}%
%
\isadelimproof
\isanewline
%
\endisadelimproof
\isanewline
\isanewline
\isanewline
\isacommand{lemma}\isamarkupfalse%
\ SWAP{\isacharunderscore}{\kern0pt}down{\isacharunderscore}{\kern0pt}action{\isacharcolon}{\kern0pt}\isanewline
\ \ {\isachardoublequoteopen}{\isasymforall}j{\isachardot}{\kern0pt}\ j\ {\isacharless}{\kern0pt}\ {\isadigit{2}}\ {\isacharcircum}{\kern0pt}Suc\ {\isacharparenleft}{\kern0pt}Suc\ n{\isacharparenright}{\kern0pt}\ {\isasymlongrightarrow}\ \isanewline
\ \ \ \ SWAP{\isacharunderscore}{\kern0pt}down\ {\isacharparenleft}{\kern0pt}Suc\ {\isacharparenleft}{\kern0pt}Suc\ n{\isacharparenright}{\kern0pt}{\isacharparenright}{\kern0pt}\ {\isacharasterisk}{\kern0pt}\ {\isacharparenleft}{\kern0pt}\ {\isacharbar}{\kern0pt}state{\isacharunderscore}{\kern0pt}basis\ {\isadigit{1}}\ {\isacharparenleft}{\kern0pt}j\ mod\ {\isadigit{2}}{\isacharparenright}{\kern0pt}{\isasymrangle}\ {\isasymOtimes}\ {\isacharbar}{\kern0pt}state{\isacharunderscore}{\kern0pt}basis\ {\isacharparenleft}{\kern0pt}Suc\ n{\isacharparenright}{\kern0pt}\ {\isacharparenleft}{\kern0pt}j\ div\ {\isadigit{2}}{\isacharparenright}{\kern0pt}{\isasymrangle}{\isacharparenright}{\kern0pt}\ {\isacharequal}{\kern0pt}\isanewline
\ \ \ \ {\isacharbar}{\kern0pt}state{\isacharunderscore}{\kern0pt}basis\ {\isacharparenleft}{\kern0pt}Suc\ n{\isacharparenright}{\kern0pt}\ {\isacharparenleft}{\kern0pt}j\ div\ {\isadigit{2}}{\isacharparenright}{\kern0pt}{\isasymrangle}\ {\isasymOtimes}\ {\isacharbar}{\kern0pt}state{\isacharunderscore}{\kern0pt}basis\ {\isadigit{1}}\ {\isacharparenleft}{\kern0pt}j\ mod\ {\isadigit{2}}{\isacharparenright}{\kern0pt}{\isasymrangle}{\isachardoublequoteclose}\isanewline
%
\isadelimproof
%
\endisadelimproof
%
\isatagproof
\isacommand{proof}\isamarkupfalse%
\ {\isacharparenleft}{\kern0pt}induct\ n{\isacharparenright}{\kern0pt}\isanewline
\ \ \isacommand{case}\isamarkupfalse%
\ {\isadigit{0}}\isanewline
\ \ \isacommand{show}\isamarkupfalse%
\ {\isacharquery}{\kern0pt}case\isanewline
\ \ \isacommand{proof}\isamarkupfalse%
\isanewline
\ \ \ \ \isacommand{fix}\isamarkupfalse%
\ j{\isacharcolon}{\kern0pt}{\isacharcolon}{\kern0pt}nat\isanewline
\ \ \ \ \isacommand{show}\isamarkupfalse%
\ {\isachardoublequoteopen}j\ {\isacharless}{\kern0pt}\ {\isadigit{2}}\ {\isacharcircum}{\kern0pt}\ Suc\ {\isacharparenleft}{\kern0pt}Suc\ {\isadigit{0}}{\isacharparenright}{\kern0pt}\ {\isasymlongrightarrow}\isanewline
\ \ \ \ \ \ \ \ \ SWAP{\isacharunderscore}{\kern0pt}down\ {\isacharparenleft}{\kern0pt}Suc\ {\isacharparenleft}{\kern0pt}Suc\ {\isadigit{0}}{\isacharparenright}{\kern0pt}{\isacharparenright}{\kern0pt}\ {\isacharasterisk}{\kern0pt}\ {\isacharparenleft}{\kern0pt}\ {\isacharbar}{\kern0pt}state{\isacharunderscore}{\kern0pt}basis\ {\isadigit{1}}\ {\isacharparenleft}{\kern0pt}j\ mod\ {\isadigit{2}}{\isacharparenright}{\kern0pt}{\isasymrangle}\ {\isasymOtimes}\ {\isacharbar}{\kern0pt}state{\isacharunderscore}{\kern0pt}basis\ {\isacharparenleft}{\kern0pt}Suc\ {\isadigit{0}}{\isacharparenright}{\kern0pt}\ {\isacharparenleft}{\kern0pt}j\ div\ {\isadigit{2}}{\isacharparenright}{\kern0pt}{\isasymrangle}{\isacharparenright}{\kern0pt}\ {\isacharequal}{\kern0pt}\isanewline
\ \ \ \ \ \ \ \ \ {\isacharbar}{\kern0pt}state{\isacharunderscore}{\kern0pt}basis\ {\isacharparenleft}{\kern0pt}Suc\ {\isadigit{0}}{\isacharparenright}{\kern0pt}\ {\isacharparenleft}{\kern0pt}j\ div\ {\isadigit{2}}{\isacharparenright}{\kern0pt}{\isasymrangle}\ {\isasymOtimes}\ {\isacharbar}{\kern0pt}state{\isacharunderscore}{\kern0pt}basis\ {\isadigit{1}}\ {\isacharparenleft}{\kern0pt}j\ mod\ {\isadigit{2}}{\isacharparenright}{\kern0pt}{\isasymrangle}{\isachardoublequoteclose}\isanewline
\ \ \ \ \isacommand{proof}\isamarkupfalse%
\isanewline
\ \ \ \ \ \ \isacommand{assume}\isamarkupfalse%
\ {\isachardoublequoteopen}j\ {\isacharless}{\kern0pt}\ {\isadigit{2}}\ {\isacharcircum}{\kern0pt}\ Suc\ {\isacharparenleft}{\kern0pt}Suc\ {\isadigit{0}}{\isacharparenright}{\kern0pt}{\isachardoublequoteclose}\isanewline
\ \ \ \ \ \ \isacommand{show}\isamarkupfalse%
\ {\isachardoublequoteopen}SWAP{\isacharunderscore}{\kern0pt}down\ {\isacharparenleft}{\kern0pt}Suc\ {\isacharparenleft}{\kern0pt}Suc\ {\isadigit{0}}{\isacharparenright}{\kern0pt}{\isacharparenright}{\kern0pt}{\isacharasterisk}{\kern0pt}{\isacharparenleft}{\kern0pt}\ {\isacharbar}{\kern0pt}state{\isacharunderscore}{\kern0pt}basis\ {\isadigit{1}}\ {\isacharparenleft}{\kern0pt}j\ mod\ {\isadigit{2}}{\isacharparenright}{\kern0pt}{\isasymrangle}\ {\isasymOtimes}\ {\isacharbar}{\kern0pt}state{\isacharunderscore}{\kern0pt}basis\ {\isacharparenleft}{\kern0pt}Suc\ {\isadigit{0}}{\isacharparenright}{\kern0pt}\ {\isacharparenleft}{\kern0pt}j\ div\ {\isadigit{2}}{\isacharparenright}{\kern0pt}{\isasymrangle}{\isacharparenright}{\kern0pt}\isanewline
\ \ \ \ \ \ \ \ \ {\isacharequal}{\kern0pt}\ {\isacharbar}{\kern0pt}state{\isacharunderscore}{\kern0pt}basis\ {\isacharparenleft}{\kern0pt}Suc\ {\isadigit{0}}{\isacharparenright}{\kern0pt}\ {\isacharparenleft}{\kern0pt}j\ div\ {\isadigit{2}}{\isacharparenright}{\kern0pt}{\isasymrangle}\ {\isasymOtimes}\ {\isacharbar}{\kern0pt}state{\isacharunderscore}{\kern0pt}basis\ {\isadigit{1}}\ {\isacharparenleft}{\kern0pt}j\ mod\ {\isadigit{2}}{\isacharparenright}{\kern0pt}{\isasymrangle}{\isachardoublequoteclose}\isanewline
\ \ \ \ \ \ \isacommand{proof}\isamarkupfalse%
\ {\isacharminus}{\kern0pt}\isanewline
\ \ \ \ \ \ \ \ \isacommand{have}\isamarkupfalse%
\ {\isachardoublequoteopen}SWAP{\isacharunderscore}{\kern0pt}down\ {\isacharparenleft}{\kern0pt}Suc\ {\isacharparenleft}{\kern0pt}Suc\ {\isadigit{0}}{\isacharparenright}{\kern0pt}{\isacharparenright}{\kern0pt}{\isacharasterisk}{\kern0pt}{\isacharparenleft}{\kern0pt}\ {\isacharbar}{\kern0pt}state{\isacharunderscore}{\kern0pt}basis\ {\isadigit{1}}\ {\isacharparenleft}{\kern0pt}j\ mod\ {\isadigit{2}}{\isacharparenright}{\kern0pt}{\isasymrangle}{\isasymOtimes}{\isacharbar}{\kern0pt}state{\isacharunderscore}{\kern0pt}basis\ {\isacharparenleft}{\kern0pt}Suc\ {\isadigit{0}}{\isacharparenright}{\kern0pt}\ {\isacharparenleft}{\kern0pt}j\ div\ {\isadigit{2}}{\isacharparenright}{\kern0pt}{\isasymrangle}{\isacharparenright}{\kern0pt}\isanewline
\ \ \ \ \ \ \ \ \ \ \ \ {\isacharequal}{\kern0pt}\ SWAP\ {\isacharasterisk}{\kern0pt}\ {\isacharparenleft}{\kern0pt}\ {\isacharbar}{\kern0pt}state{\isacharunderscore}{\kern0pt}basis\ {\isadigit{1}}\ {\isacharparenleft}{\kern0pt}j\ mod\ {\isadigit{2}}{\isacharparenright}{\kern0pt}{\isasymrangle}\ {\isasymOtimes}\ {\isacharbar}{\kern0pt}state{\isacharunderscore}{\kern0pt}basis\ {\isacharparenleft}{\kern0pt}Suc\ {\isadigit{0}}{\isacharparenright}{\kern0pt}\ {\isacharparenleft}{\kern0pt}j\ div\ {\isadigit{2}}{\isacharparenright}{\kern0pt}{\isasymrangle}{\isacharparenright}{\kern0pt}{\isachardoublequoteclose}\ \isanewline
\ \ \ \ \ \ \ \ \ \ \isacommand{using}\isamarkupfalse%
\ SWAP{\isacharunderscore}{\kern0pt}down{\isachardot}{\kern0pt}simps\ \isacommand{by}\isamarkupfalse%
\ simp\isanewline
\ \ \ \ \ \ \ \ \isacommand{also}\isamarkupfalse%
\ \isacommand{have}\isamarkupfalse%
\ {\isachardoublequoteopen}{\isasymdots}\ {\isacharequal}{\kern0pt}\ {\isacharbar}{\kern0pt}state{\isacharunderscore}{\kern0pt}basis\ {\isacharparenleft}{\kern0pt}Suc\ {\isadigit{0}}{\isacharparenright}{\kern0pt}\ {\isacharparenleft}{\kern0pt}j\ div\ {\isadigit{2}}{\isacharparenright}{\kern0pt}{\isasymrangle}\ {\isasymOtimes}\ {\isacharbar}{\kern0pt}state{\isacharunderscore}{\kern0pt}basis\ {\isadigit{1}}\ {\isacharparenleft}{\kern0pt}j\ mod\ {\isadigit{2}}{\isacharparenright}{\kern0pt}{\isasymrangle}{\isachardoublequoteclose}\isanewline
\ \ \ \ \ \ \ \ \ \ \isacommand{using}\isamarkupfalse%
\ SWAP{\isacharunderscore}{\kern0pt}tensor\ state{\isacharunderscore}{\kern0pt}basis{\isacharunderscore}{\kern0pt}carrier{\isacharunderscore}{\kern0pt}mat\ \isanewline
\ \ \ \ \ \ \ \ \ \ \isacommand{by}\isamarkupfalse%
\ {\isacharparenleft}{\kern0pt}metis\ One{\isacharunderscore}{\kern0pt}nat{\isacharunderscore}{\kern0pt}def\ power{\isacharunderscore}{\kern0pt}one{\isacharunderscore}{\kern0pt}right{\isacharparenright}{\kern0pt}\isanewline
\ \ \ \ \ \ \ \ \isacommand{finally}\isamarkupfalse%
\ \isacommand{show}\isamarkupfalse%
\ {\isacharquery}{\kern0pt}thesis\ \isacommand{by}\isamarkupfalse%
\ this\isanewline
\ \ \ \ \ \ \isacommand{qed}\isamarkupfalse%
\isanewline
\ \ \ \ \isacommand{qed}\isamarkupfalse%
\isanewline
\ \ \isacommand{qed}\isamarkupfalse%
\isanewline
\isacommand{next}\isamarkupfalse%
\isanewline
\ \ \isacommand{case}\isamarkupfalse%
\ {\isacharparenleft}{\kern0pt}Suc\ n{\isacharparenright}{\kern0pt}\isanewline
\ \ \isacommand{assume}\isamarkupfalse%
\ HI{\isacharcolon}{\kern0pt}{\isachardoublequoteopen}{\isasymforall}j{\isacharless}{\kern0pt}{\isadigit{2}}\ {\isacharcircum}{\kern0pt}\ Suc\ {\isacharparenleft}{\kern0pt}Suc\ n{\isacharparenright}{\kern0pt}{\isachardot}{\kern0pt}\isanewline
\ \ \ \ \ \ \ \ \ \ \ \ SWAP{\isacharunderscore}{\kern0pt}down\ {\isacharparenleft}{\kern0pt}Suc\ {\isacharparenleft}{\kern0pt}Suc\ n{\isacharparenright}{\kern0pt}{\isacharparenright}{\kern0pt}{\isacharasterisk}{\kern0pt}{\isacharparenleft}{\kern0pt}\ {\isacharbar}{\kern0pt}state{\isacharunderscore}{\kern0pt}basis\ {\isadigit{1}}\ {\isacharparenleft}{\kern0pt}j\ mod\ {\isadigit{2}}{\isacharparenright}{\kern0pt}{\isasymrangle}\ {\isasymOtimes}\ {\isacharbar}{\kern0pt}state{\isacharunderscore}{\kern0pt}basis\ {\isacharparenleft}{\kern0pt}Suc\ n{\isacharparenright}{\kern0pt}\ {\isacharparenleft}{\kern0pt}j\ div\ {\isadigit{2}}{\isacharparenright}{\kern0pt}{\isasymrangle}{\isacharparenright}{\kern0pt}\isanewline
\ \ \ \ \ \ \ \ \ \ {\isacharequal}{\kern0pt}\ {\isacharbar}{\kern0pt}state{\isacharunderscore}{\kern0pt}basis\ {\isacharparenleft}{\kern0pt}Suc\ n{\isacharparenright}{\kern0pt}\ {\isacharparenleft}{\kern0pt}j\ div\ {\isadigit{2}}{\isacharparenright}{\kern0pt}{\isasymrangle}\ {\isasymOtimes}\ {\isacharbar}{\kern0pt}state{\isacharunderscore}{\kern0pt}basis\ {\isadigit{1}}\ {\isacharparenleft}{\kern0pt}j\ mod\ {\isadigit{2}}{\isacharparenright}{\kern0pt}{\isasymrangle}{\isachardoublequoteclose}\isanewline
\ \ \isacommand{show}\isamarkupfalse%
\ {\isachardoublequoteopen}{\isasymforall}j{\isacharless}{\kern0pt}{\isadigit{2}}\ {\isacharcircum}{\kern0pt}\ Suc\ {\isacharparenleft}{\kern0pt}Suc\ {\isacharparenleft}{\kern0pt}Suc\ n{\isacharparenright}{\kern0pt}{\isacharparenright}{\kern0pt}{\isachardot}{\kern0pt}\isanewline
\ \ \ \ \ \ \ \ \ \ \ \ SWAP{\isacharunderscore}{\kern0pt}down\ {\isacharparenleft}{\kern0pt}Suc\ {\isacharparenleft}{\kern0pt}Suc\ {\isacharparenleft}{\kern0pt}Suc\ n{\isacharparenright}{\kern0pt}{\isacharparenright}{\kern0pt}{\isacharparenright}{\kern0pt}{\isacharasterisk}{\kern0pt}{\isacharparenleft}{\kern0pt}\ {\isacharbar}{\kern0pt}state{\isacharunderscore}{\kern0pt}basis\ {\isadigit{1}}\ {\isacharparenleft}{\kern0pt}j\ mod\ {\isadigit{2}}{\isacharparenright}{\kern0pt}{\isasymrangle}\ {\isasymOtimes}\ \isanewline
\ \ \ \ \ \ \ \ \ \ \ \ {\isacharbar}{\kern0pt}state{\isacharunderscore}{\kern0pt}basis\ {\isacharparenleft}{\kern0pt}Suc\ {\isacharparenleft}{\kern0pt}Suc\ n{\isacharparenright}{\kern0pt}{\isacharparenright}{\kern0pt}\ {\isacharparenleft}{\kern0pt}j\ div\ {\isadigit{2}}{\isacharparenright}{\kern0pt}{\isasymrangle}{\isacharparenright}{\kern0pt}\isanewline
\ \ \ \ \ \ \ \ \ \ {\isacharequal}{\kern0pt}\ {\isacharbar}{\kern0pt}state{\isacharunderscore}{\kern0pt}basis\ {\isacharparenleft}{\kern0pt}Suc\ {\isacharparenleft}{\kern0pt}Suc\ n{\isacharparenright}{\kern0pt}{\isacharparenright}{\kern0pt}\ {\isacharparenleft}{\kern0pt}j\ div\ {\isadigit{2}}{\isacharparenright}{\kern0pt}{\isasymrangle}\ {\isasymOtimes}\ {\isacharbar}{\kern0pt}state{\isacharunderscore}{\kern0pt}basis\ {\isadigit{1}}\ {\isacharparenleft}{\kern0pt}j\ mod\ {\isadigit{2}}{\isacharparenright}{\kern0pt}{\isasymrangle}{\isachardoublequoteclose}\isanewline
\ \ \isacommand{proof}\isamarkupfalse%
\isanewline
\ \ \ \ \isacommand{fix}\isamarkupfalse%
\ j{\isacharcolon}{\kern0pt}{\isacharcolon}{\kern0pt}nat\isanewline
\ \ \ \ \isacommand{show}\isamarkupfalse%
\ {\isachardoublequoteopen}j\ {\isacharless}{\kern0pt}\ {\isadigit{2}}\ {\isacharcircum}{\kern0pt}\ Suc\ {\isacharparenleft}{\kern0pt}Suc\ {\isacharparenleft}{\kern0pt}Suc\ n{\isacharparenright}{\kern0pt}{\isacharparenright}{\kern0pt}\ {\isasymlongrightarrow}\isanewline
\ \ \ \ \ \ \ \ \ SWAP{\isacharunderscore}{\kern0pt}down\ {\isacharparenleft}{\kern0pt}Suc\ {\isacharparenleft}{\kern0pt}Suc\ {\isacharparenleft}{\kern0pt}Suc\ n{\isacharparenright}{\kern0pt}{\isacharparenright}{\kern0pt}{\isacharparenright}{\kern0pt}\ {\isacharasterisk}{\kern0pt}\ {\isacharparenleft}{\kern0pt}\ {\isacharbar}{\kern0pt}state{\isacharunderscore}{\kern0pt}basis\ {\isadigit{1}}\ {\isacharparenleft}{\kern0pt}j\ mod\ {\isadigit{2}}{\isacharparenright}{\kern0pt}{\isasymrangle}\ {\isasymOtimes}\ {\isacharbar}{\kern0pt}state{\isacharunderscore}{\kern0pt}basis\ {\isacharparenleft}{\kern0pt}Suc\ {\isacharparenleft}{\kern0pt}Suc\ n{\isacharparenright}{\kern0pt}{\isacharparenright}{\kern0pt}\isanewline
\ \ \ \ \ \ \ \ \ \ \ \ {\isacharparenleft}{\kern0pt}j\ div\ {\isadigit{2}}{\isacharparenright}{\kern0pt}{\isasymrangle}{\isacharparenright}{\kern0pt}\ {\isacharequal}{\kern0pt}\isanewline
\ \ \ \ \ \ \ \ \ {\isacharbar}{\kern0pt}state{\isacharunderscore}{\kern0pt}basis\ {\isacharparenleft}{\kern0pt}Suc\ {\isacharparenleft}{\kern0pt}Suc\ n{\isacharparenright}{\kern0pt}{\isacharparenright}{\kern0pt}\ {\isacharparenleft}{\kern0pt}j\ div\ {\isadigit{2}}{\isacharparenright}{\kern0pt}{\isasymrangle}\ {\isasymOtimes}\ {\isacharbar}{\kern0pt}state{\isacharunderscore}{\kern0pt}basis\ {\isadigit{1}}\ {\isacharparenleft}{\kern0pt}j\ mod\ {\isadigit{2}}{\isacharparenright}{\kern0pt}{\isasymrangle}{\isachardoublequoteclose}\isanewline
\ \ \ \ \isacommand{proof}\isamarkupfalse%
\isanewline
\ \ \ \ \ \ \isacommand{assume}\isamarkupfalse%
\ jl{\isacharcolon}{\kern0pt}{\isachardoublequoteopen}j\ {\isacharless}{\kern0pt}\ {\isadigit{2}}\ {\isacharcircum}{\kern0pt}\ Suc\ {\isacharparenleft}{\kern0pt}Suc\ {\isacharparenleft}{\kern0pt}Suc\ n{\isacharparenright}{\kern0pt}{\isacharparenright}{\kern0pt}{\isachardoublequoteclose}\isanewline
\ \ \ \ \ \ \isacommand{show}\isamarkupfalse%
\ {\isachardoublequoteopen}SWAP{\isacharunderscore}{\kern0pt}down\ {\isacharparenleft}{\kern0pt}Suc\ {\isacharparenleft}{\kern0pt}Suc\ {\isacharparenleft}{\kern0pt}Suc\ n{\isacharparenright}{\kern0pt}{\isacharparenright}{\kern0pt}{\isacharparenright}{\kern0pt}\ {\isacharasterisk}{\kern0pt}\ {\isacharparenleft}{\kern0pt}\ {\isacharbar}{\kern0pt}state{\isacharunderscore}{\kern0pt}basis\ {\isadigit{1}}\ {\isacharparenleft}{\kern0pt}j\ mod\ {\isadigit{2}}{\isacharparenright}{\kern0pt}{\isasymrangle}\ {\isasymOtimes}\ \isanewline
\ \ \ \ \ \ \ \ \ \ \ \ {\isacharbar}{\kern0pt}state{\isacharunderscore}{\kern0pt}basis\ {\isacharparenleft}{\kern0pt}Suc\ {\isacharparenleft}{\kern0pt}Suc\ n{\isacharparenright}{\kern0pt}{\isacharparenright}{\kern0pt}\ {\isacharparenleft}{\kern0pt}j\ div\ {\isadigit{2}}{\isacharparenright}{\kern0pt}{\isasymrangle}{\isacharparenright}{\kern0pt}\ {\isacharequal}{\kern0pt}\isanewline
\ \ \ \ \ \ \ \ \ \ \ \ {\isacharbar}{\kern0pt}state{\isacharunderscore}{\kern0pt}basis\ {\isacharparenleft}{\kern0pt}Suc\ {\isacharparenleft}{\kern0pt}Suc\ n{\isacharparenright}{\kern0pt}{\isacharparenright}{\kern0pt}\ {\isacharparenleft}{\kern0pt}j\ div\ {\isadigit{2}}{\isacharparenright}{\kern0pt}{\isasymrangle}\ {\isasymOtimes}\ {\isacharbar}{\kern0pt}state{\isacharunderscore}{\kern0pt}basis\ {\isadigit{1}}\ {\isacharparenleft}{\kern0pt}j\ mod\ {\isadigit{2}}{\isacharparenright}{\kern0pt}{\isasymrangle}{\isachardoublequoteclose}\isanewline
\ \ \ \ \ \ \isacommand{proof}\isamarkupfalse%
\ {\isacharminus}{\kern0pt}\isanewline
\ \ \ \ \ \ \ \ \isacommand{define}\isamarkupfalse%
\ x\ \isakeyword{where}\ {\isachardoublequoteopen}x\ {\isacharequal}{\kern0pt}\ {\isadigit{2}}{\isacharasterisk}{\kern0pt}{\isacharparenleft}{\kern0pt}{\isacharparenleft}{\kern0pt}j\ div\ {\isadigit{2}}{\isacharparenright}{\kern0pt}\ div\ {\isadigit{2}}{\isacharparenright}{\kern0pt}\ {\isacharplus}{\kern0pt}\ {\isacharparenleft}{\kern0pt}j\ mod\ {\isadigit{2}}{\isacharparenright}{\kern0pt}{\isachardoublequoteclose}\isanewline
\ \ \ \ \ \ \ \ \isacommand{have}\isamarkupfalse%
\ xl{\isacharcolon}{\kern0pt}{\isachardoublequoteopen}x\ {\isacharless}{\kern0pt}\ {\isadigit{2}}{\isacharcircum}{\kern0pt}Suc\ {\isacharparenleft}{\kern0pt}Suc\ n{\isacharparenright}{\kern0pt}{\isachardoublequoteclose}\isanewline
\ \ \ \ \ \ \ \ \isacommand{proof}\isamarkupfalse%
\ {\isacharminus}{\kern0pt}\isanewline
\ \ \ \ \ \ \ \ \ \ \isacommand{have}\isamarkupfalse%
\ {\isachardoublequoteopen}j\ mod\ {\isadigit{2}}\ {\isacharless}{\kern0pt}\ {\isadigit{2}}{\isachardoublequoteclose}\ \isacommand{by}\isamarkupfalse%
\ auto\isanewline
\ \ \ \ \ \ \ \ \ \ \isacommand{moreover}\isamarkupfalse%
\ \isacommand{have}\isamarkupfalse%
\ {\isadigit{0}}{\isacharcolon}{\kern0pt}{\isachardoublequoteopen}{\isacharparenleft}{\kern0pt}j\ div\ {\isadigit{2}}{\isacharparenright}{\kern0pt}\ div\ {\isadigit{2}}\ {\isacharless}{\kern0pt}\ {\isadigit{2}}{\isacharcircum}{\kern0pt}Suc\ n{\isachardoublequoteclose}\ \isacommand{using}\isamarkupfalse%
\ jl\ \isacommand{by}\isamarkupfalse%
\ auto\isanewline
\ \ \ \ \ \ \ \ \ \ \isacommand{moreover}\isamarkupfalse%
\ \isacommand{have}\isamarkupfalse%
\ {\isachardoublequoteopen}{\isadigit{2}}{\isacharasterisk}{\kern0pt}{\isacharparenleft}{\kern0pt}{\isacharparenleft}{\kern0pt}j\ div\ {\isadigit{2}}{\isacharparenright}{\kern0pt}\ div\ {\isadigit{2}}{\isacharparenright}{\kern0pt}\ {\isacharless}{\kern0pt}\ {\isadigit{2}}{\isacharcircum}{\kern0pt}Suc\ {\isacharparenleft}{\kern0pt}Suc\ n{\isacharparenright}{\kern0pt}{\isachardoublequoteclose}\ \isacommand{using}\isamarkupfalse%
\ {\isadigit{0}}\ \isacommand{by}\isamarkupfalse%
\ auto\isanewline
\ \ \ \ \ \ \ \ \ \ \isacommand{ultimately}\isamarkupfalse%
\ \isacommand{show}\isamarkupfalse%
\ {\isacharquery}{\kern0pt}thesis\ \isacommand{using}\isamarkupfalse%
\ x{\isacharunderscore}{\kern0pt}def\isanewline
\ \ \ \ \ \ \ \ \ \ \ \ \isacommand{by}\isamarkupfalse%
\ {\isacharparenleft}{\kern0pt}metis\ {\isacharparenleft}{\kern0pt}no{\isacharunderscore}{\kern0pt}types{\isacharcomma}{\kern0pt}\ lifting{\isacharparenright}{\kern0pt}\ Suc{\isacharunderscore}{\kern0pt}double{\isacharunderscore}{\kern0pt}not{\isacharunderscore}{\kern0pt}eq{\isacharunderscore}{\kern0pt}double\ add{\isachardot}{\kern0pt}right{\isacharunderscore}{\kern0pt}neutral\ add{\isacharunderscore}{\kern0pt}Suc{\isacharunderscore}{\kern0pt}right\ \isanewline
\ \ \ \ \ \ \ \ \ \ \ \ \ \ \ \ less{\isacharunderscore}{\kern0pt}{\isadigit{2}}{\isacharunderscore}{\kern0pt}cases{\isacharunderscore}{\kern0pt}iff\ linorder{\isacharunderscore}{\kern0pt}neqE{\isacharunderscore}{\kern0pt}nat\ not{\isacharunderscore}{\kern0pt}less{\isacharunderscore}{\kern0pt}eq\ power{\isacharunderscore}{\kern0pt}Suc{\isacharparenright}{\kern0pt}\isanewline
\ \ \ \ \ \ \ \ \isacommand{qed}\isamarkupfalse%
\isanewline
\ \ \ \ \ \ \ \ \isacommand{have}\isamarkupfalse%
\ xm{\isacharcolon}{\kern0pt}{\isachardoublequoteopen}x\ mod\ {\isadigit{2}}\ {\isacharequal}{\kern0pt}\ j\ mod\ {\isadigit{2}}{\isachardoublequoteclose}\ \isacommand{using}\isamarkupfalse%
\ x{\isacharunderscore}{\kern0pt}def\ \isacommand{by}\isamarkupfalse%
\ auto\isanewline
\ \ \ \ \ \ \ \ \isacommand{have}\isamarkupfalse%
\ xd{\isacharcolon}{\kern0pt}{\isachardoublequoteopen}x\ div\ {\isadigit{2}}\ {\isacharequal}{\kern0pt}\ j\ div\ {\isadigit{2}}\ div\ {\isadigit{2}}{\isachardoublequoteclose}\ \isacommand{using}\isamarkupfalse%
\ x{\isacharunderscore}{\kern0pt}def\ \isacommand{by}\isamarkupfalse%
\ auto\isanewline
\ \ \ \ \ \ \ \ \isacommand{have}\isamarkupfalse%
\ {\isachardoublequoteopen}SWAP{\isacharunderscore}{\kern0pt}down\ {\isacharparenleft}{\kern0pt}Suc\ {\isacharparenleft}{\kern0pt}Suc\ {\isacharparenleft}{\kern0pt}Suc\ n{\isacharparenright}{\kern0pt}{\isacharparenright}{\kern0pt}{\isacharparenright}{\kern0pt}\ {\isacharasterisk}{\kern0pt}\ {\isacharparenleft}{\kern0pt}\ {\isacharbar}{\kern0pt}state{\isacharunderscore}{\kern0pt}basis\ {\isadigit{1}}\ {\isacharparenleft}{\kern0pt}j\ mod\ {\isadigit{2}}{\isacharparenright}{\kern0pt}{\isasymrangle}\ {\isasymOtimes}\ \isanewline
\ \ \ \ \ \ \ \ \ \ \ \ \ \ {\isacharbar}{\kern0pt}state{\isacharunderscore}{\kern0pt}basis\ {\isacharparenleft}{\kern0pt}Suc\ {\isacharparenleft}{\kern0pt}Suc\ n{\isacharparenright}{\kern0pt}{\isacharparenright}{\kern0pt}\ {\isacharparenleft}{\kern0pt}j\ div\ {\isadigit{2}}{\isacharparenright}{\kern0pt}{\isasymrangle}{\isacharparenright}{\kern0pt}\ {\isacharequal}{\kern0pt}\isanewline
\ \ \ \ \ \ \ \ \ \ \ \ \ \ {\isacharparenleft}{\kern0pt}{\isacharparenleft}{\kern0pt}{\isacharparenleft}{\kern0pt}{\isadigit{1}}\isactrlsub m\ {\isacharparenleft}{\kern0pt}{\isadigit{2}}{\isacharcircum}{\kern0pt}{\isacharparenleft}{\kern0pt}Suc\ n{\isacharparenright}{\kern0pt}{\isacharparenright}{\kern0pt}{\isacharparenright}{\kern0pt}\ {\isasymOtimes}\ SWAP{\isacharparenright}{\kern0pt}\ {\isacharasterisk}{\kern0pt}\ {\isacharparenleft}{\kern0pt}{\isacharparenleft}{\kern0pt}SWAP{\isacharunderscore}{\kern0pt}down\ {\isacharparenleft}{\kern0pt}Suc\ {\isacharparenleft}{\kern0pt}Suc\ n{\isacharparenright}{\kern0pt}{\isacharparenright}{\kern0pt}{\isacharparenright}{\kern0pt}\ {\isasymOtimes}\ {\isacharparenleft}{\kern0pt}{\isadigit{1}}\isactrlsub m\ {\isadigit{2}}{\isacharparenright}{\kern0pt}{\isacharparenright}{\kern0pt}{\isacharparenright}{\kern0pt}\ {\isacharasterisk}{\kern0pt}\isanewline
\ \ \ \ \ \ \ \ \ \ \ \ {\isacharparenleft}{\kern0pt}\ {\isacharbar}{\kern0pt}state{\isacharunderscore}{\kern0pt}basis\ {\isadigit{1}}\ {\isacharparenleft}{\kern0pt}j\ mod\ {\isadigit{2}}{\isacharparenright}{\kern0pt}{\isasymrangle}\ {\isasymOtimes}\ {\isacharbar}{\kern0pt}state{\isacharunderscore}{\kern0pt}basis\ {\isacharparenleft}{\kern0pt}Suc\ {\isacharparenleft}{\kern0pt}Suc\ n{\isacharparenright}{\kern0pt}{\isacharparenright}{\kern0pt}\ {\isacharparenleft}{\kern0pt}j\ div\ {\isadigit{2}}{\isacharparenright}{\kern0pt}{\isasymrangle}{\isacharparenright}{\kern0pt}{\isachardoublequoteclose}\isanewline
\ \ \ \ \ \ \ \ \ \ \isacommand{using}\isamarkupfalse%
\ SWAP{\isacharunderscore}{\kern0pt}down{\isachardot}{\kern0pt}simps\ \isacommand{by}\isamarkupfalse%
\ simp\isanewline
\ \ \ \ \ \ \ \ \isacommand{also}\isamarkupfalse%
\ \isacommand{have}\isamarkupfalse%
\ {\isachardoublequoteopen}{\isasymdots}\ {\isacharequal}{\kern0pt}\ {\isacharparenleft}{\kern0pt}{\isacharparenleft}{\kern0pt}{\isadigit{1}}\isactrlsub m\ {\isacharparenleft}{\kern0pt}{\isadigit{2}}{\isacharcircum}{\kern0pt}{\isacharparenleft}{\kern0pt}Suc\ n{\isacharparenright}{\kern0pt}{\isacharparenright}{\kern0pt}{\isacharparenright}{\kern0pt}\ {\isasymOtimes}\ SWAP{\isacharparenright}{\kern0pt}\ {\isacharasterisk}{\kern0pt}\ {\isacharparenleft}{\kern0pt}{\isacharparenleft}{\kern0pt}{\isacharparenleft}{\kern0pt}SWAP{\isacharunderscore}{\kern0pt}down\ {\isacharparenleft}{\kern0pt}Suc\ {\isacharparenleft}{\kern0pt}Suc\ n{\isacharparenright}{\kern0pt}{\isacharparenright}{\kern0pt}{\isacharparenright}{\kern0pt}\ {\isasymOtimes}\ {\isacharparenleft}{\kern0pt}{\isadigit{1}}\isactrlsub m\ {\isadigit{2}}{\isacharparenright}{\kern0pt}{\isacharparenright}{\kern0pt}\ {\isacharasterisk}{\kern0pt}\isanewline
\ \ \ \ \ \ \ \ \ \ \ \ \ \ \ \ \ \ \ \ \ \ \ \ {\isacharparenleft}{\kern0pt}\ {\isacharbar}{\kern0pt}state{\isacharunderscore}{\kern0pt}basis\ {\isadigit{1}}\ {\isacharparenleft}{\kern0pt}j\ mod\ {\isadigit{2}}{\isacharparenright}{\kern0pt}{\isasymrangle}\ {\isasymOtimes}\ {\isacharbar}{\kern0pt}state{\isacharunderscore}{\kern0pt}basis\ {\isacharparenleft}{\kern0pt}Suc\ {\isacharparenleft}{\kern0pt}Suc\ n{\isacharparenright}{\kern0pt}{\isacharparenright}{\kern0pt}\ {\isacharparenleft}{\kern0pt}j\ div\ {\isadigit{2}}{\isacharparenright}{\kern0pt}{\isasymrangle}{\isacharparenright}{\kern0pt}{\isacharparenright}{\kern0pt}{\isachardoublequoteclose}\isanewline
\ \ \ \ \ \ \ \ \isacommand{proof}\isamarkupfalse%
\ {\isacharparenleft}{\kern0pt}rule\ assoc{\isacharunderscore}{\kern0pt}mult{\isacharunderscore}{\kern0pt}mat{\isacharparenright}{\kern0pt}\isanewline
\ \ \ \ \ \ \ \ \ \ \isacommand{show}\isamarkupfalse%
\ {\isachardoublequoteopen}{\isadigit{1}}\isactrlsub m\ {\isacharparenleft}{\kern0pt}{\isadigit{2}}\ {\isacharcircum}{\kern0pt}\ Suc\ n{\isacharparenright}{\kern0pt}\ {\isasymOtimes}\ SWAP\ {\isasymin}\ carrier{\isacharunderscore}{\kern0pt}mat\ {\isacharparenleft}{\kern0pt}{\isadigit{2}}{\isacharcircum}{\kern0pt}Suc\ {\isacharparenleft}{\kern0pt}Suc\ {\isacharparenleft}{\kern0pt}Suc\ n{\isacharparenright}{\kern0pt}{\isacharparenright}{\kern0pt}{\isacharparenright}{\kern0pt}\ {\isacharparenleft}{\kern0pt}{\isadigit{2}}{\isacharcircum}{\kern0pt}Suc\ {\isacharparenleft}{\kern0pt}Suc\ {\isacharparenleft}{\kern0pt}Suc\ n{\isacharparenright}{\kern0pt}{\isacharparenright}{\kern0pt}{\isacharparenright}{\kern0pt}{\isachardoublequoteclose}\isanewline
\ \ \ \ \ \ \ \ \ \ \ \ \isacommand{by}\isamarkupfalse%
\ {\isacharparenleft}{\kern0pt}simp\ add{\isacharcolon}{\kern0pt}\ SWAP{\isacharunderscore}{\kern0pt}ncols\ SWAP{\isacharunderscore}{\kern0pt}nrows\ carrier{\isacharunderscore}{\kern0pt}matI{\isacharparenright}{\kern0pt}\isanewline
\ \ \ \ \ \ \ \ \ \ \isacommand{show}\isamarkupfalse%
\ {\isachardoublequoteopen}SWAP{\isacharunderscore}{\kern0pt}down\ {\isacharparenleft}{\kern0pt}Suc\ {\isacharparenleft}{\kern0pt}Suc\ n{\isacharparenright}{\kern0pt}{\isacharparenright}{\kern0pt}\ {\isasymOtimes}\ {\isadigit{1}}\isactrlsub m\ {\isadigit{2}}\isanewline
\ \ \ \ \ \ \ \ \ \ \ \ \ \ \ \ {\isasymin}\ carrier{\isacharunderscore}{\kern0pt}mat\ {\isacharparenleft}{\kern0pt}{\isadigit{2}}\ {\isacharcircum}{\kern0pt}\ Suc\ {\isacharparenleft}{\kern0pt}Suc\ {\isacharparenleft}{\kern0pt}Suc\ n{\isacharparenright}{\kern0pt}{\isacharparenright}{\kern0pt}{\isacharparenright}{\kern0pt}\ {\isacharparenleft}{\kern0pt}{\isadigit{2}}\ {\isacharcircum}{\kern0pt}\ Suc\ {\isacharparenleft}{\kern0pt}Suc\ {\isacharparenleft}{\kern0pt}Suc\ n{\isacharparenright}{\kern0pt}{\isacharparenright}{\kern0pt}{\isacharparenright}{\kern0pt}{\isachardoublequoteclose}\isanewline
\ \ \ \ \ \ \ \ \ \ \ \ \isacommand{by}\isamarkupfalse%
\ {\isacharparenleft}{\kern0pt}metis\ One{\isacharunderscore}{\kern0pt}nat{\isacharunderscore}{\kern0pt}def\ SWAP{\isacharunderscore}{\kern0pt}down{\isachardot}{\kern0pt}simps{\isacharparenleft}{\kern0pt}{\isadigit{2}}{\isacharparenright}{\kern0pt}\ SWAP{\isacharunderscore}{\kern0pt}down{\isacharunderscore}{\kern0pt}carrier{\isacharunderscore}{\kern0pt}mat\ power{\isacharunderscore}{\kern0pt}Suc{\isadigit{2}}\isanewline
\ \ \ \ \ \ \ \ \ \ \ \ \ \ \ \ power{\isacharunderscore}{\kern0pt}one{\isacharunderscore}{\kern0pt}right\ tensor{\isacharunderscore}{\kern0pt}carrier{\isacharunderscore}{\kern0pt}mat{\isacharparenright}{\kern0pt}\isanewline
\ \ \ \ \ \ \ \ \ \ \isacommand{show}\isamarkupfalse%
\ {\isachardoublequoteopen}{\isacharbar}{\kern0pt}state{\isacharunderscore}{\kern0pt}basis\ {\isadigit{1}}\ {\isacharparenleft}{\kern0pt}j\ mod\ {\isadigit{2}}{\isacharparenright}{\kern0pt}{\isasymrangle}\ {\isasymOtimes}\ {\isacharbar}{\kern0pt}state{\isacharunderscore}{\kern0pt}basis\ {\isacharparenleft}{\kern0pt}Suc\ {\isacharparenleft}{\kern0pt}Suc\ n{\isacharparenright}{\kern0pt}{\isacharparenright}{\kern0pt}\ {\isacharparenleft}{\kern0pt}j\ div\ {\isadigit{2}}{\isacharparenright}{\kern0pt}{\isasymrangle}\isanewline
\ \ \ \ \ \ \ \ \ \ \ \ \ \ \ \ {\isasymin}\ carrier{\isacharunderscore}{\kern0pt}mat\ {\isacharparenleft}{\kern0pt}{\isadigit{2}}\ {\isacharcircum}{\kern0pt}\ Suc\ {\isacharparenleft}{\kern0pt}Suc\ {\isacharparenleft}{\kern0pt}Suc\ n{\isacharparenright}{\kern0pt}{\isacharparenright}{\kern0pt}{\isacharparenright}{\kern0pt}\ {\isadigit{1}}{\isachardoublequoteclose}\isanewline
\ \ \ \ \ \ \ \ \ \ \ \ \isacommand{by}\isamarkupfalse%
\ {\isacharparenleft}{\kern0pt}metis\ Suc{\isacharunderscore}{\kern0pt}{\isadigit{1}}\ one{\isacharunderscore}{\kern0pt}power{\isadigit{2}}\ power{\isacharunderscore}{\kern0pt}Suc\ power{\isacharunderscore}{\kern0pt}one{\isacharunderscore}{\kern0pt}right\ state{\isacharunderscore}{\kern0pt}basis{\isacharunderscore}{\kern0pt}carrier{\isacharunderscore}{\kern0pt}mat\ \isanewline
\ \ \ \ \ \ \ \ \ \ \ \ \ \ \ \ tensor{\isacharunderscore}{\kern0pt}carrier{\isacharunderscore}{\kern0pt}mat{\isacharparenright}{\kern0pt}\isanewline
\ \ \ \ \ \ \ \ \isacommand{qed}\isamarkupfalse%
\isanewline
\ \ \ \ \ \ \ \ \isacommand{also}\isamarkupfalse%
\ \isacommand{have}\isamarkupfalse%
\ {\isachardoublequoteopen}{\isasymdots}\ {\isacharequal}{\kern0pt}\ {\isacharparenleft}{\kern0pt}{\isacharparenleft}{\kern0pt}{\isadigit{1}}\isactrlsub m\ {\isacharparenleft}{\kern0pt}{\isadigit{2}}{\isacharcircum}{\kern0pt}{\isacharparenleft}{\kern0pt}Suc\ n{\isacharparenright}{\kern0pt}{\isacharparenright}{\kern0pt}{\isacharparenright}{\kern0pt}\ {\isasymOtimes}\ SWAP{\isacharparenright}{\kern0pt}\ {\isacharasterisk}{\kern0pt}\ {\isacharparenleft}{\kern0pt}{\isacharparenleft}{\kern0pt}{\isacharparenleft}{\kern0pt}SWAP{\isacharunderscore}{\kern0pt}down\ {\isacharparenleft}{\kern0pt}Suc\ {\isacharparenleft}{\kern0pt}Suc\ n{\isacharparenright}{\kern0pt}{\isacharparenright}{\kern0pt}{\isacharparenright}{\kern0pt}\ {\isasymOtimes}\ {\isacharparenleft}{\kern0pt}{\isadigit{1}}\isactrlsub m\ {\isadigit{2}}{\isacharparenright}{\kern0pt}{\isacharparenright}{\kern0pt}\ {\isacharasterisk}{\kern0pt}\isanewline
\ \ \ \ \ \ \ \ \ \ \ \ \ \ \ \ \ \ \ \ \ \ \ \ {\isacharparenleft}{\kern0pt}\ {\isacharbar}{\kern0pt}state{\isacharunderscore}{\kern0pt}basis\ {\isadigit{1}}\ {\isacharparenleft}{\kern0pt}j\ mod\ {\isadigit{2}}{\isacharparenright}{\kern0pt}{\isasymrangle}\ {\isasymOtimes}\ \isanewline
\ \ \ \ \ \ \ \ \ \ \ \ \ \ \ \ \ \ \ \ \ \ \ \ {\isacharparenleft}{\kern0pt}\ {\isacharbar}{\kern0pt}state{\isacharunderscore}{\kern0pt}basis\ {\isacharparenleft}{\kern0pt}Suc\ n{\isacharparenright}{\kern0pt}\ {\isacharparenleft}{\kern0pt}{\isacharparenleft}{\kern0pt}j\ div\ {\isadigit{2}}{\isacharparenright}{\kern0pt}\ div\ {\isadigit{2}}{\isacharparenright}{\kern0pt}{\isasymrangle}\ {\isasymOtimes}\isanewline
\ \ \ \ \ \ \ \ \ \ \ \ \ \ \ \ \ \ \ \ \ \ \ \ \ \ {\isacharbar}{\kern0pt}state{\isacharunderscore}{\kern0pt}basis\ {\isadigit{1}}\ {\isacharparenleft}{\kern0pt}{\isacharparenleft}{\kern0pt}j\ div\ {\isadigit{2}}{\isacharparenright}{\kern0pt}\ mod\ {\isadigit{2}}{\isacharparenright}{\kern0pt}{\isasymrangle}{\isacharparenright}{\kern0pt}{\isacharparenright}{\kern0pt}{\isacharparenright}{\kern0pt}{\isachardoublequoteclose}\isanewline
\ \ \ \ \ \ \ \ \ \ \isacommand{using}\isamarkupfalse%
\ state{\isacharunderscore}{\kern0pt}basis{\isacharunderscore}{\kern0pt}dec{\isacharprime}{\kern0pt}\ jl\ \isanewline
\ \ \ \ \ \ \ \ \ \ \isacommand{by}\isamarkupfalse%
\ {\isacharparenleft}{\kern0pt}metis\ less{\isacharunderscore}{\kern0pt}mult{\isacharunderscore}{\kern0pt}imp{\isacharunderscore}{\kern0pt}div{\isacharunderscore}{\kern0pt}less\ power{\isacharunderscore}{\kern0pt}Suc{\isadigit{2}}{\isacharparenright}{\kern0pt}\isanewline
\ \ \ \ \ \ \ \ \isacommand{also}\isamarkupfalse%
\ \isacommand{have}\isamarkupfalse%
\ {\isachardoublequoteopen}{\isasymdots}\ {\isacharequal}{\kern0pt}\ {\isacharparenleft}{\kern0pt}{\isacharparenleft}{\kern0pt}{\isadigit{1}}\isactrlsub m\ {\isacharparenleft}{\kern0pt}{\isadigit{2}}{\isacharcircum}{\kern0pt}{\isacharparenleft}{\kern0pt}Suc\ n{\isacharparenright}{\kern0pt}{\isacharparenright}{\kern0pt}{\isacharparenright}{\kern0pt}\ {\isasymOtimes}\ SWAP{\isacharparenright}{\kern0pt}\ {\isacharasterisk}{\kern0pt}\ {\isacharparenleft}{\kern0pt}{\isacharparenleft}{\kern0pt}{\isacharparenleft}{\kern0pt}SWAP{\isacharunderscore}{\kern0pt}down\ {\isacharparenleft}{\kern0pt}Suc\ {\isacharparenleft}{\kern0pt}Suc\ n{\isacharparenright}{\kern0pt}{\isacharparenright}{\kern0pt}{\isacharparenright}{\kern0pt}\ {\isasymOtimes}\ {\isacharparenleft}{\kern0pt}{\isadigit{1}}\isactrlsub m\ {\isadigit{2}}{\isacharparenright}{\kern0pt}{\isacharparenright}{\kern0pt}\ {\isacharasterisk}{\kern0pt}\isanewline
\ \ \ \ \ \ \ \ \ \ \ \ \ \ \ \ \ \ \ \ \ \ \ \ {\isacharparenleft}{\kern0pt}{\isacharparenleft}{\kern0pt}\ {\isacharbar}{\kern0pt}state{\isacharunderscore}{\kern0pt}basis\ {\isadigit{1}}\ {\isacharparenleft}{\kern0pt}j\ mod\ {\isadigit{2}}{\isacharparenright}{\kern0pt}{\isasymrangle}\ {\isasymOtimes}\ \isanewline
\ \ \ \ \ \ \ \ \ \ \ \ \ \ \ \ \ \ \ \ \ \ \ \ \ \ {\isacharbar}{\kern0pt}state{\isacharunderscore}{\kern0pt}basis\ {\isacharparenleft}{\kern0pt}Suc\ n{\isacharparenright}{\kern0pt}\ {\isacharparenleft}{\kern0pt}{\isacharparenleft}{\kern0pt}j\ div\ {\isadigit{2}}{\isacharparenright}{\kern0pt}\ div\ {\isadigit{2}}{\isacharparenright}{\kern0pt}{\isasymrangle}{\isacharparenright}{\kern0pt}\ {\isasymOtimes}\isanewline
\ \ \ \ \ \ \ \ \ \ \ \ \ \ \ \ \ \ \ \ \ \ \ \ \ \ {\isacharbar}{\kern0pt}state{\isacharunderscore}{\kern0pt}basis\ {\isadigit{1}}\ {\isacharparenleft}{\kern0pt}{\isacharparenleft}{\kern0pt}j\ div\ {\isadigit{2}}{\isacharparenright}{\kern0pt}\ mod\ {\isadigit{2}}{\isacharparenright}{\kern0pt}{\isasymrangle}{\isacharparenright}{\kern0pt}{\isacharparenright}{\kern0pt}{\isachardoublequoteclose}\isanewline
\ \ \ \ \ \ \ \ \ \ \isacommand{using}\isamarkupfalse%
\ tensor{\isacharunderscore}{\kern0pt}mat{\isacharunderscore}{\kern0pt}is{\isacharunderscore}{\kern0pt}assoc\ \isacommand{by}\isamarkupfalse%
\ simp\isanewline
\ \ \ \ \ \ \ \ \isacommand{also}\isamarkupfalse%
\ \isacommand{have}\isamarkupfalse%
\ {\isachardoublequoteopen}{\isasymdots}\ {\isacharequal}{\kern0pt}\ {\isacharparenleft}{\kern0pt}{\isacharparenleft}{\kern0pt}{\isadigit{1}}\isactrlsub m\ {\isacharparenleft}{\kern0pt}{\isadigit{2}}{\isacharcircum}{\kern0pt}{\isacharparenleft}{\kern0pt}Suc\ n{\isacharparenright}{\kern0pt}{\isacharparenright}{\kern0pt}{\isacharparenright}{\kern0pt}\ {\isasymOtimes}\ SWAP{\isacharparenright}{\kern0pt}\ {\isacharasterisk}{\kern0pt}\ \isanewline
\ \ \ \ \ \ \ \ \ \ \ \ \ \ \ \ \ \ \ \ \ \ \ \ {\isacharparenleft}{\kern0pt}{\isacharparenleft}{\kern0pt}{\isacharparenleft}{\kern0pt}SWAP{\isacharunderscore}{\kern0pt}down\ {\isacharparenleft}{\kern0pt}Suc\ {\isacharparenleft}{\kern0pt}Suc\ n{\isacharparenright}{\kern0pt}{\isacharparenright}{\kern0pt}{\isacharparenright}{\kern0pt}\ {\isacharasterisk}{\kern0pt}\ {\isacharparenleft}{\kern0pt}\ {\isacharbar}{\kern0pt}state{\isacharunderscore}{\kern0pt}basis\ {\isadigit{1}}\ {\isacharparenleft}{\kern0pt}j\ mod\ {\isadigit{2}}{\isacharparenright}{\kern0pt}{\isasymrangle}\ {\isasymOtimes}\ \isanewline
\ \ \ \ \ \ \ \ \ \ \ \ \ \ \ \ \ \ \ \ \ \ \ \ \ \ {\isacharbar}{\kern0pt}state{\isacharunderscore}{\kern0pt}basis\ {\isacharparenleft}{\kern0pt}Suc\ n{\isacharparenright}{\kern0pt}\ {\isacharparenleft}{\kern0pt}{\isacharparenleft}{\kern0pt}j\ div\ {\isadigit{2}}{\isacharparenright}{\kern0pt}\ div\ {\isadigit{2}}{\isacharparenright}{\kern0pt}{\isasymrangle}{\isacharparenright}{\kern0pt}{\isacharparenright}{\kern0pt}\ {\isasymOtimes}\isanewline
\ \ \ \ \ \ \ \ \ \ \ \ \ \ \ \ \ \ \ \ \ \ \ \ {\isacharparenleft}{\kern0pt}{\isacharparenleft}{\kern0pt}{\isadigit{1}}\isactrlsub m\ {\isadigit{2}}{\isacharparenright}{\kern0pt}\ {\isacharasterisk}{\kern0pt}\ {\isacharbar}{\kern0pt}state{\isacharunderscore}{\kern0pt}basis\ {\isadigit{1}}\ {\isacharparenleft}{\kern0pt}{\isacharparenleft}{\kern0pt}j\ div\ {\isadigit{2}}{\isacharparenright}{\kern0pt}\ mod\ {\isadigit{2}}{\isacharparenright}{\kern0pt}{\isasymrangle}{\isacharparenright}{\kern0pt}{\isacharparenright}{\kern0pt}{\isachardoublequoteclose}\isanewline
\ \ \ \ \ \ \ \ \ \ \isacommand{using}\isamarkupfalse%
\ mult{\isacharunderscore}{\kern0pt}distr{\isacharunderscore}{\kern0pt}tensor\isanewline
\ \ \ \ \ \ \ \ \ \ \isacommand{by}\isamarkupfalse%
\ {\isacharparenleft}{\kern0pt}smt\ {\isacharparenleft}{\kern0pt}verit{\isacharcomma}{\kern0pt}\ ccfv{\isacharunderscore}{\kern0pt}threshold{\isacharparenright}{\kern0pt}\ SWAP{\isacharunderscore}{\kern0pt}down{\isacharunderscore}{\kern0pt}carrier{\isacharunderscore}{\kern0pt}mat\ carrier{\isacharunderscore}{\kern0pt}matD{\isacharparenleft}{\kern0pt}{\isadigit{1}}{\isacharparenright}{\kern0pt}\ carrier{\isacharunderscore}{\kern0pt}matD{\isacharparenleft}{\kern0pt}{\isadigit{2}}{\isacharparenright}{\kern0pt}\isanewline
\ \ \ \ \ \ \ \ \ \ \ \ \ \ dim{\isacharunderscore}{\kern0pt}col{\isacharunderscore}{\kern0pt}tensor{\isacharunderscore}{\kern0pt}mat\ dim{\isacharunderscore}{\kern0pt}row{\isacharunderscore}{\kern0pt}tensor{\isacharunderscore}{\kern0pt}mat\ index{\isacharunderscore}{\kern0pt}one{\isacharunderscore}{\kern0pt}mat{\isacharparenleft}{\kern0pt}{\isadigit{3}}{\isacharparenright}{\kern0pt}\ mult{\isachardot}{\kern0pt}right{\isacharunderscore}{\kern0pt}neutral\ \isanewline
\ \ \ \ \ \ \ \ \ \ \ \ \ \ nat{\isacharunderscore}{\kern0pt}zero{\isacharunderscore}{\kern0pt}less{\isacharunderscore}{\kern0pt}power{\isacharunderscore}{\kern0pt}iff\ pos{\isadigit{2}}\ power{\isacharunderscore}{\kern0pt}Suc{\isadigit{2}}\ power{\isacharunderscore}{\kern0pt}commutes\ power{\isacharunderscore}{\kern0pt}one{\isacharunderscore}{\kern0pt}right\ \isanewline
\ \ \ \ \ \ \ \ \ \ \ \ \ \ state{\isacharunderscore}{\kern0pt}basis{\isacharunderscore}{\kern0pt}carrier{\isacharunderscore}{\kern0pt}mat\ zero{\isacharunderscore}{\kern0pt}less{\isacharunderscore}{\kern0pt}one{\isacharunderscore}{\kern0pt}class{\isachardot}{\kern0pt}zero{\isacharunderscore}{\kern0pt}less{\isacharunderscore}{\kern0pt}one{\isacharparenright}{\kern0pt}\isanewline
\ \ \ \ \ \ \ \ \isacommand{also}\isamarkupfalse%
\ \isacommand{have}\isamarkupfalse%
\ {\isachardoublequoteopen}{\isasymdots}\ {\isacharequal}{\kern0pt}\ {\isacharparenleft}{\kern0pt}{\isacharparenleft}{\kern0pt}{\isadigit{1}}\isactrlsub m\ {\isacharparenleft}{\kern0pt}{\isadigit{2}}{\isacharcircum}{\kern0pt}{\isacharparenleft}{\kern0pt}Suc\ n{\isacharparenright}{\kern0pt}{\isacharparenright}{\kern0pt}{\isacharparenright}{\kern0pt}\ {\isasymOtimes}\ SWAP{\isacharparenright}{\kern0pt}\ {\isacharasterisk}{\kern0pt}\ \isanewline
\ \ \ \ \ \ \ \ \ \ \ \ \ \ \ \ \ \ \ \ \ \ \ \ {\isacharparenleft}{\kern0pt}{\isacharparenleft}{\kern0pt}{\isacharparenleft}{\kern0pt}SWAP{\isacharunderscore}{\kern0pt}down\ {\isacharparenleft}{\kern0pt}Suc\ {\isacharparenleft}{\kern0pt}Suc\ n{\isacharparenright}{\kern0pt}{\isacharparenright}{\kern0pt}{\isacharparenright}{\kern0pt}\ {\isacharasterisk}{\kern0pt}\ {\isacharparenleft}{\kern0pt}\ {\isacharbar}{\kern0pt}state{\isacharunderscore}{\kern0pt}basis\ {\isadigit{1}}\ {\isacharparenleft}{\kern0pt}x\ mod\ {\isadigit{2}}{\isacharparenright}{\kern0pt}{\isasymrangle}\ {\isasymOtimes}\ \isanewline
\ \ \ \ \ \ \ \ \ \ \ \ \ \ \ \ \ \ \ \ \ \ \ \ \ \ {\isacharbar}{\kern0pt}state{\isacharunderscore}{\kern0pt}basis\ {\isacharparenleft}{\kern0pt}Suc\ n{\isacharparenright}{\kern0pt}\ {\isacharparenleft}{\kern0pt}x\ div\ {\isadigit{2}}{\isacharparenright}{\kern0pt}{\isasymrangle}{\isacharparenright}{\kern0pt}{\isacharparenright}{\kern0pt}\ {\isasymOtimes}\isanewline
\ \ \ \ \ \ \ \ \ \ \ \ \ \ \ \ \ \ \ \ \ \ \ \ {\isacharparenleft}{\kern0pt}{\isacharparenleft}{\kern0pt}{\isadigit{1}}\isactrlsub m\ {\isadigit{2}}{\isacharparenright}{\kern0pt}\ {\isacharasterisk}{\kern0pt}\ {\isacharbar}{\kern0pt}state{\isacharunderscore}{\kern0pt}basis\ {\isadigit{1}}\ {\isacharparenleft}{\kern0pt}{\isacharparenleft}{\kern0pt}j\ div\ {\isadigit{2}}{\isacharparenright}{\kern0pt}\ mod\ {\isadigit{2}}{\isacharparenright}{\kern0pt}{\isasymrangle}{\isacharparenright}{\kern0pt}{\isacharparenright}{\kern0pt}{\isachardoublequoteclose}\isanewline
\ \ \ \ \ \ \ \ \ \ \isacommand{using}\isamarkupfalse%
\ xm\ xd\ \isacommand{by}\isamarkupfalse%
\ simp\isanewline
\ \ \ \ \ \ \ \ \isacommand{also}\isamarkupfalse%
\ \isacommand{have}\isamarkupfalse%
\ {\isachardoublequoteopen}{\isasymdots}\ {\isacharequal}{\kern0pt}\ {\isacharparenleft}{\kern0pt}{\isacharparenleft}{\kern0pt}{\isadigit{1}}\isactrlsub m\ {\isacharparenleft}{\kern0pt}{\isadigit{2}}{\isacharcircum}{\kern0pt}{\isacharparenleft}{\kern0pt}Suc\ n{\isacharparenright}{\kern0pt}{\isacharparenright}{\kern0pt}{\isacharparenright}{\kern0pt}\ {\isasymOtimes}\ SWAP{\isacharparenright}{\kern0pt}\ {\isacharasterisk}{\kern0pt}\isanewline
\ \ \ \ \ \ \ \ \ \ \ \ \ \ \ \ \ \ \ \ \ \ \ \ {\isacharparenleft}{\kern0pt}{\isacharparenleft}{\kern0pt}\ {\isacharbar}{\kern0pt}state{\isacharunderscore}{\kern0pt}basis\ {\isacharparenleft}{\kern0pt}Suc\ n{\isacharparenright}{\kern0pt}\ {\isacharparenleft}{\kern0pt}x\ div\ {\isadigit{2}}{\isacharparenright}{\kern0pt}{\isasymrangle}\ {\isasymOtimes}\ {\isacharbar}{\kern0pt}state{\isacharunderscore}{\kern0pt}basis\ {\isadigit{1}}\ {\isacharparenleft}{\kern0pt}x\ mod\ {\isadigit{2}}{\isacharparenright}{\kern0pt}{\isasymrangle}{\isacharparenright}{\kern0pt}\ {\isasymOtimes}\isanewline
\ \ \ \ \ \ \ \ \ \ \ \ \ \ \ \ \ \ \ \ \ \ \ \ \ \ \ {\isacharbar}{\kern0pt}state{\isacharunderscore}{\kern0pt}basis\ {\isadigit{1}}\ {\isacharparenleft}{\kern0pt}{\isacharparenleft}{\kern0pt}j\ div\ {\isadigit{2}}{\isacharparenright}{\kern0pt}\ mod\ {\isadigit{2}}{\isacharparenright}{\kern0pt}{\isasymrangle}{\isacharparenright}{\kern0pt}{\isachardoublequoteclose}\isanewline
\ \ \ \ \ \ \ \ \ \ \isacommand{using}\isamarkupfalse%
\ HI\isanewline
\ \ \ \ \ \ \ \ \ \ \isacommand{by}\isamarkupfalse%
\ {\isacharparenleft}{\kern0pt}metis\ dim{\isacharunderscore}{\kern0pt}row{\isacharunderscore}{\kern0pt}mat{\isacharparenleft}{\kern0pt}{\isadigit{1}}{\isacharparenright}{\kern0pt}\ index{\isacharunderscore}{\kern0pt}unit{\isacharunderscore}{\kern0pt}vec{\isacharparenleft}{\kern0pt}{\isadigit{3}}{\isacharparenright}{\kern0pt}\ ket{\isacharunderscore}{\kern0pt}vec{\isacharunderscore}{\kern0pt}def\ left{\isacharunderscore}{\kern0pt}mult{\isacharunderscore}{\kern0pt}one{\isacharunderscore}{\kern0pt}mat{\isacharprime}{\kern0pt}\ power{\isacharunderscore}{\kern0pt}one{\isacharunderscore}{\kern0pt}right\ \isanewline
\ \ \ \ \ \ \ \ \ \ \ \ \ \ state{\isacharunderscore}{\kern0pt}basis{\isacharunderscore}{\kern0pt}def\ xl{\isacharparenright}{\kern0pt}\isanewline
\ \ \ \ \ \ \ \ \isacommand{also}\isamarkupfalse%
\ \isacommand{have}\isamarkupfalse%
\ {\isachardoublequoteopen}{\isasymdots}\ {\isacharequal}{\kern0pt}\ {\isacharparenleft}{\kern0pt}{\isacharparenleft}{\kern0pt}{\isadigit{1}}\isactrlsub m\ {\isacharparenleft}{\kern0pt}{\isadigit{2}}{\isacharcircum}{\kern0pt}{\isacharparenleft}{\kern0pt}Suc\ n{\isacharparenright}{\kern0pt}{\isacharparenright}{\kern0pt}{\isacharparenright}{\kern0pt}\ {\isasymOtimes}\ SWAP{\isacharparenright}{\kern0pt}\ {\isacharasterisk}{\kern0pt}\isanewline
\ \ \ \ \ \ \ \ \ \ \ \ \ \ \ \ \ \ \ \ \ \ \ \ {\isacharparenleft}{\kern0pt}\ {\isacharbar}{\kern0pt}state{\isacharunderscore}{\kern0pt}basis\ {\isacharparenleft}{\kern0pt}Suc\ n{\isacharparenright}{\kern0pt}\ {\isacharparenleft}{\kern0pt}x\ div\ {\isadigit{2}}{\isacharparenright}{\kern0pt}{\isasymrangle}\ {\isasymOtimes}\ {\isacharparenleft}{\kern0pt}\ {\isacharbar}{\kern0pt}state{\isacharunderscore}{\kern0pt}basis\ {\isadigit{1}}\ {\isacharparenleft}{\kern0pt}x\ mod\ {\isadigit{2}}{\isacharparenright}{\kern0pt}{\isasymrangle}\ {\isasymOtimes}\isanewline
\ \ \ \ \ \ \ \ \ \ \ \ \ \ \ \ \ \ \ \ \ \ \ \ \ \ \ {\isacharbar}{\kern0pt}state{\isacharunderscore}{\kern0pt}basis\ {\isadigit{1}}\ {\isacharparenleft}{\kern0pt}{\isacharparenleft}{\kern0pt}j\ div\ {\isadigit{2}}{\isacharparenright}{\kern0pt}\ mod\ {\isadigit{2}}{\isacharparenright}{\kern0pt}{\isasymrangle}{\isacharparenright}{\kern0pt}{\isacharparenright}{\kern0pt}{\isachardoublequoteclose}\isanewline
\ \ \ \ \ \ \ \ \ \ \isacommand{using}\isamarkupfalse%
\ tensor{\isacharunderscore}{\kern0pt}mat{\isacharunderscore}{\kern0pt}is{\isacharunderscore}{\kern0pt}assoc\ \isacommand{by}\isamarkupfalse%
\ force\isanewline
\ \ \ \ \ \ \ \ \isacommand{also}\isamarkupfalse%
\ \isacommand{have}\isamarkupfalse%
\ {\isachardoublequoteopen}{\isasymdots}\ {\isacharequal}{\kern0pt}\ {\isacharparenleft}{\kern0pt}{\isacharparenleft}{\kern0pt}{\isadigit{1}}\isactrlsub m\ {\isacharparenleft}{\kern0pt}{\isadigit{2}}{\isacharcircum}{\kern0pt}{\isacharparenleft}{\kern0pt}Suc\ n{\isacharparenright}{\kern0pt}{\isacharparenright}{\kern0pt}{\isacharparenright}{\kern0pt}\ {\isacharasterisk}{\kern0pt}\ {\isacharbar}{\kern0pt}state{\isacharunderscore}{\kern0pt}basis\ {\isacharparenleft}{\kern0pt}Suc\ n{\isacharparenright}{\kern0pt}\ {\isacharparenleft}{\kern0pt}x\ div\ {\isadigit{2}}{\isacharparenright}{\kern0pt}{\isasymrangle}{\isacharparenright}{\kern0pt}\ {\isasymOtimes}\isanewline
\ \ \ \ \ \ \ \ \ \ \ \ \ \ \ \ \ \ \ \ \ \ \ \ {\isacharparenleft}{\kern0pt}SWAP\ {\isacharasterisk}{\kern0pt}\ {\isacharparenleft}{\kern0pt}\ {\isacharbar}{\kern0pt}state{\isacharunderscore}{\kern0pt}basis\ {\isadigit{1}}\ {\isacharparenleft}{\kern0pt}x\ mod\ {\isadigit{2}}{\isacharparenright}{\kern0pt}{\isasymrangle}\ {\isasymOtimes}\ {\isacharbar}{\kern0pt}state{\isacharunderscore}{\kern0pt}basis\ {\isadigit{1}}\ {\isacharparenleft}{\kern0pt}{\isacharparenleft}{\kern0pt}j\ div\ {\isadigit{2}}{\isacharparenright}{\kern0pt}\ mod\ {\isadigit{2}}{\isacharparenright}{\kern0pt}{\isasymrangle}{\isacharparenright}{\kern0pt}{\isacharparenright}{\kern0pt}{\isachardoublequoteclose}\isanewline
\ \ \ \ \ \ \ \ \ \ \isacommand{using}\isamarkupfalse%
\ mult{\isacharunderscore}{\kern0pt}distr{\isacharunderscore}{\kern0pt}tensor\ state{\isacharunderscore}{\kern0pt}basis{\isacharunderscore}{\kern0pt}carrier{\isacharunderscore}{\kern0pt}mat\ SWAP{\isacharunderscore}{\kern0pt}carrier{\isacharunderscore}{\kern0pt}mat\isanewline
\ \ \ \ \ \ \ \ \ \ \isacommand{by}\isamarkupfalse%
\ {\isacharparenleft}{\kern0pt}smt\ {\isacharparenleft}{\kern0pt}verit{\isacharcomma}{\kern0pt}\ del{\isacharunderscore}{\kern0pt}insts{\isacharparenright}{\kern0pt}\ SWAP{\isacharunderscore}{\kern0pt}tensor\ carrier{\isacharunderscore}{\kern0pt}matD{\isacharparenleft}{\kern0pt}{\isadigit{1}}{\isacharparenright}{\kern0pt}\ carrier{\isacharunderscore}{\kern0pt}matD{\isacharparenleft}{\kern0pt}{\isadigit{2}}{\isacharparenright}{\kern0pt}\ dim{\isacharunderscore}{\kern0pt}col{\isacharunderscore}{\kern0pt}tensor{\isacharunderscore}{\kern0pt}mat\isanewline
\ \ \ \ \ \ \ \ \ \ \ \ \ \ index{\isacharunderscore}{\kern0pt}mult{\isacharunderscore}{\kern0pt}mat{\isacharparenleft}{\kern0pt}{\isadigit{2}}{\isacharparenright}{\kern0pt}\ index{\isacharunderscore}{\kern0pt}one{\isacharunderscore}{\kern0pt}mat{\isacharparenleft}{\kern0pt}{\isadigit{3}}{\isacharparenright}{\kern0pt}\ nat{\isacharunderscore}{\kern0pt}{\isadigit{0}}{\isacharunderscore}{\kern0pt}less{\isacharunderscore}{\kern0pt}mult{\isacharunderscore}{\kern0pt}iff\ power{\isacharunderscore}{\kern0pt}one{\isacharunderscore}{\kern0pt}right\ \isanewline
\ \ \ \ \ \ \ \ \ \ \ \ \ \ tensor{\isacharunderscore}{\kern0pt}mat{\isacharunderscore}{\kern0pt}is{\isacharunderscore}{\kern0pt}assoc\ zero{\isacharunderscore}{\kern0pt}less{\isacharunderscore}{\kern0pt}numeral\ zero{\isacharunderscore}{\kern0pt}less{\isacharunderscore}{\kern0pt}one{\isacharunderscore}{\kern0pt}class{\isachardot}{\kern0pt}zero{\isacharunderscore}{\kern0pt}less{\isacharunderscore}{\kern0pt}one\ \isanewline
\ \ \ \ \ \ \ \ \ \ \ \ \ \ zero{\isacharunderscore}{\kern0pt}less{\isacharunderscore}{\kern0pt}power{\isacharparenright}{\kern0pt}\ \isanewline
\ \ \ \ \ \ \ \ \isacommand{also}\isamarkupfalse%
\ \isacommand{have}\isamarkupfalse%
\ {\isachardoublequoteopen}{\isasymdots}\ {\isacharequal}{\kern0pt}\ {\isacharbar}{\kern0pt}state{\isacharunderscore}{\kern0pt}basis\ {\isacharparenleft}{\kern0pt}Suc\ n{\isacharparenright}{\kern0pt}\ {\isacharparenleft}{\kern0pt}x\ div\ {\isadigit{2}}{\isacharparenright}{\kern0pt}{\isasymrangle}\ {\isasymOtimes}\isanewline
\ \ \ \ \ \ \ \ \ \ \ \ \ \ \ \ \ \ \ \ \ \ {\isacharparenleft}{\kern0pt}\ {\isacharbar}{\kern0pt}state{\isacharunderscore}{\kern0pt}basis\ {\isadigit{1}}\ {\isacharparenleft}{\kern0pt}{\isacharparenleft}{\kern0pt}j\ div\ {\isadigit{2}}{\isacharparenright}{\kern0pt}\ mod\ {\isadigit{2}}{\isacharparenright}{\kern0pt}{\isasymrangle}\ {\isasymOtimes}\ {\isacharbar}{\kern0pt}state{\isacharunderscore}{\kern0pt}basis\ {\isadigit{1}}\ {\isacharparenleft}{\kern0pt}x\ mod\ {\isadigit{2}}{\isacharparenright}{\kern0pt}{\isasymrangle}{\isacharparenright}{\kern0pt}{\isachardoublequoteclose}\isanewline
\ \ \ \ \ \ \ \ \ \ \isacommand{using}\isamarkupfalse%
\ SWAP{\isacharunderscore}{\kern0pt}tensor\isanewline
\ \ \ \ \ \ \ \ \ \ \isacommand{by}\isamarkupfalse%
\ {\isacharparenleft}{\kern0pt}metis\ left{\isacharunderscore}{\kern0pt}mult{\isacharunderscore}{\kern0pt}one{\isacharunderscore}{\kern0pt}mat\ power{\isacharunderscore}{\kern0pt}one{\isacharunderscore}{\kern0pt}right\ state{\isacharunderscore}{\kern0pt}basis{\isacharunderscore}{\kern0pt}carrier{\isacharunderscore}{\kern0pt}mat{\isacharparenright}{\kern0pt}\isanewline
\ \ \ \ \ \ \ \ \isacommand{also}\isamarkupfalse%
\ \isacommand{have}\isamarkupfalse%
\ {\isachardoublequoteopen}{\isasymdots}\ {\isacharequal}{\kern0pt}\ {\isacharparenleft}{\kern0pt}\ {\isacharbar}{\kern0pt}state{\isacharunderscore}{\kern0pt}basis\ {\isacharparenleft}{\kern0pt}Suc\ n{\isacharparenright}{\kern0pt}\ {\isacharparenleft}{\kern0pt}x\ div\ {\isadigit{2}}{\isacharparenright}{\kern0pt}{\isasymrangle}\ {\isasymOtimes}\ {\isacharbar}{\kern0pt}state{\isacharunderscore}{\kern0pt}basis\ {\isadigit{1}}\ {\isacharparenleft}{\kern0pt}{\isacharparenleft}{\kern0pt}j\ div\ {\isadigit{2}}{\isacharparenright}{\kern0pt}\ mod\ {\isadigit{2}}{\isacharparenright}{\kern0pt}{\isasymrangle}{\isacharparenright}{\kern0pt}\ {\isasymOtimes}\isanewline
\ \ \ \ \ \ \ \ \ \ \ \ \ \ \ \ \ \ \ \ \ \ \ \ \ \ {\isacharbar}{\kern0pt}state{\isacharunderscore}{\kern0pt}basis\ {\isadigit{1}}\ {\isacharparenleft}{\kern0pt}x\ mod\ {\isadigit{2}}{\isacharparenright}{\kern0pt}{\isasymrangle}{\isachardoublequoteclose}\isanewline
\ \ \ \ \ \ \ \ \ \ \isacommand{using}\isamarkupfalse%
\ assoc{\isacharunderscore}{\kern0pt}mult{\isacharunderscore}{\kern0pt}mat\ tensor{\isacharunderscore}{\kern0pt}mat{\isacharunderscore}{\kern0pt}is{\isacharunderscore}{\kern0pt}assoc\ \isacommand{by}\isamarkupfalse%
\ presburger\isanewline
\ \ \ \ \ \ \ \ \isacommand{also}\isamarkupfalse%
\ \isacommand{have}\isamarkupfalse%
\ {\isachardoublequoteopen}{\isasymdots}\ {\isacharequal}{\kern0pt}\ {\isacharbar}{\kern0pt}state{\isacharunderscore}{\kern0pt}basis\ {\isacharparenleft}{\kern0pt}Suc\ {\isacharparenleft}{\kern0pt}Suc\ n{\isacharparenright}{\kern0pt}{\isacharparenright}{\kern0pt}\ {\isacharparenleft}{\kern0pt}j\ div\ {\isadigit{2}}{\isacharparenright}{\kern0pt}{\isasymrangle}\ {\isasymOtimes}\ {\isacharbar}{\kern0pt}state{\isacharunderscore}{\kern0pt}basis\ {\isadigit{1}}\ {\isacharparenleft}{\kern0pt}j\ mod\ {\isadigit{2}}{\isacharparenright}{\kern0pt}{\isasymrangle}{\isachardoublequoteclose}\isanewline
\ \ \ \ \ \ \ \ \ \ \isacommand{using}\isamarkupfalse%
\ state{\isacharunderscore}{\kern0pt}basis{\isacharunderscore}{\kern0pt}dec{\isacharprime}{\kern0pt}\ xd\ xm\isanewline
\ \ \ \ \ \ \ \ \ \ \isacommand{by}\isamarkupfalse%
\ {\isacharparenleft}{\kern0pt}metis\ jl\ less{\isacharunderscore}{\kern0pt}mult{\isacharunderscore}{\kern0pt}imp{\isacharunderscore}{\kern0pt}div{\isacharunderscore}{\kern0pt}less\ power{\isacharunderscore}{\kern0pt}Suc{\isadigit{2}}{\isacharparenright}{\kern0pt}\isanewline
\ \ \ \ \ \ \ \ \isacommand{finally}\isamarkupfalse%
\ \isacommand{show}\isamarkupfalse%
\ {\isacharquery}{\kern0pt}thesis\ \isacommand{by}\isamarkupfalse%
\ this\isanewline
\ \ \ \ \ \ \isacommand{qed}\isamarkupfalse%
\isanewline
\ \ \ \ \isacommand{qed}\isamarkupfalse%
\isanewline
\ \ \isacommand{qed}\isamarkupfalse%
\isanewline
\isacommand{qed}\isamarkupfalse%
%
\endisatagproof
{\isafoldproof}%
%
\isadelimproof
%
\endisadelimproof
%
\begin{isamarkuptext}%
Action of the controlled-R gates in the circuit%
\end{isamarkuptext}\isamarkuptrue%
\isacommand{lemma}\isamarkupfalse%
\ controlR{\isacharunderscore}{\kern0pt}action{\isacharcolon}{\kern0pt}\isanewline
\ \ \isakeyword{assumes}\ {\isachardoublequoteopen}j\ {\isacharless}{\kern0pt}\ {\isadigit{2}}\ {\isacharcircum}{\kern0pt}\ Suc\ {\isacharparenleft}{\kern0pt}Suc\ n{\isacharparenright}{\kern0pt}{\isachardoublequoteclose}\isanewline
\ \ \isakeyword{shows}\ {\isachardoublequoteopen}{\isacharparenleft}{\kern0pt}control\ {\isacharparenleft}{\kern0pt}Suc\ {\isacharparenleft}{\kern0pt}Suc\ n{\isacharparenright}{\kern0pt}{\isacharparenright}{\kern0pt}\ {\isacharparenleft}{\kern0pt}R\ {\isacharparenleft}{\kern0pt}Suc\ {\isacharparenleft}{\kern0pt}Suc\ n{\isacharparenright}{\kern0pt}{\isacharparenright}{\kern0pt}{\isacharparenright}{\kern0pt}{\isacharparenright}{\kern0pt}\ {\isacharasterisk}{\kern0pt}\isanewline
\ \ \ \ \ \ \ \ \ {\isacharparenleft}{\kern0pt}{\isacharparenleft}{\kern0pt}\ {\isacharbar}{\kern0pt}zero{\isasymrangle}\ {\isacharplus}{\kern0pt}\ exp\ {\isacharparenleft}{\kern0pt}{\isadigit{2}}{\isacharasterisk}{\kern0pt}{\isasymi}{\isacharasterisk}{\kern0pt}pi{\isacharasterisk}{\kern0pt}complex{\isacharunderscore}{\kern0pt}of{\isacharunderscore}{\kern0pt}nat\ {\isacharparenleft}{\kern0pt}j\ div\ {\isadigit{2}}{\isacharparenright}{\kern0pt}\ {\isacharslash}{\kern0pt}\ {\isadigit{2}}{\isacharcircum}{\kern0pt}{\isacharparenleft}{\kern0pt}Suc\ n{\isacharparenright}{\kern0pt}{\isacharparenright}{\kern0pt}\ {\isasymcdot}\isactrlsub m\ {\isacharbar}{\kern0pt}one{\isasymrangle}{\isacharparenright}{\kern0pt}\ {\isasymOtimes}\isanewline
\ \ \ \ \ \ \ \ \ \ {\isacharbar}{\kern0pt}state{\isacharunderscore}{\kern0pt}basis\ n\ {\isacharparenleft}{\kern0pt}{\isacharparenleft}{\kern0pt}j\ mod\ {\isadigit{2}}{\isacharcircum}{\kern0pt}{\isacharparenleft}{\kern0pt}Suc\ n{\isacharparenright}{\kern0pt}{\isacharparenright}{\kern0pt}\ div\ {\isadigit{2}}{\isacharparenright}{\kern0pt}{\isasymrangle}\ {\isasymOtimes}\ {\isacharbar}{\kern0pt}state{\isacharunderscore}{\kern0pt}basis\ {\isadigit{1}}\ {\isacharparenleft}{\kern0pt}j\ mod\ {\isadigit{2}}{\isacharparenright}{\kern0pt}{\isasymrangle}{\isacharparenright}{\kern0pt}\ {\isacharequal}{\kern0pt}\isanewline
\ \ \ \ \ \ \ \ \ \ {\isacharparenleft}{\kern0pt}\ {\isacharbar}{\kern0pt}zero{\isasymrangle}\ {\isacharplus}{\kern0pt}\ exp\ {\isacharparenleft}{\kern0pt}{\isadigit{2}}{\isacharasterisk}{\kern0pt}{\isasymi}{\isacharasterisk}{\kern0pt}pi{\isacharasterisk}{\kern0pt}complex{\isacharunderscore}{\kern0pt}of{\isacharunderscore}{\kern0pt}nat\ j\ {\isacharslash}{\kern0pt}\ {\isadigit{2}}{\isacharcircum}{\kern0pt}{\isacharparenleft}{\kern0pt}Suc\ {\isacharparenleft}{\kern0pt}Suc\ n{\isacharparenright}{\kern0pt}{\isacharparenright}{\kern0pt}{\isacharparenright}{\kern0pt}\ {\isasymcdot}\isactrlsub m\ {\isacharbar}{\kern0pt}one{\isasymrangle}{\isacharparenright}{\kern0pt}\ {\isasymOtimes}\isanewline
\ \ \ \ \ \ \ \ \ \ {\isacharbar}{\kern0pt}state{\isacharunderscore}{\kern0pt}basis\ n\ {\isacharparenleft}{\kern0pt}{\isacharparenleft}{\kern0pt}j\ mod\ {\isadigit{2}}{\isacharcircum}{\kern0pt}{\isacharparenleft}{\kern0pt}Suc\ n{\isacharparenright}{\kern0pt}{\isacharparenright}{\kern0pt}\ div\ {\isadigit{2}}{\isacharparenright}{\kern0pt}{\isasymrangle}\ {\isasymOtimes}\ {\isacharbar}{\kern0pt}state{\isacharunderscore}{\kern0pt}basis\ {\isadigit{1}}\ {\isacharparenleft}{\kern0pt}j\ mod\ {\isadigit{2}}{\isacharparenright}{\kern0pt}{\isasymrangle}{\isachardoublequoteclose}\isanewline
%
\isadelimproof
%
\endisadelimproof
%
\isatagproof
\isacommand{proof}\isamarkupfalse%
\ {\isacharparenleft}{\kern0pt}cases\ n{\isacharparenright}{\kern0pt}\isanewline
\ \ \isacommand{case}\isamarkupfalse%
\ {\isadigit{0}}\isanewline
\ \ \isacommand{then}\isamarkupfalse%
\ \isacommand{show}\isamarkupfalse%
\ {\isacharquery}{\kern0pt}thesis\isanewline
\ \ \isacommand{proof}\isamarkupfalse%
\ {\isacharminus}{\kern0pt}\isanewline
\ \ \ \ \isacommand{assume}\isamarkupfalse%
\ n{\isadigit{0}}{\isacharcolon}{\kern0pt}{\isachardoublequoteopen}n\ {\isacharequal}{\kern0pt}\ {\isadigit{0}}{\isachardoublequoteclose}\isanewline
\ \ \ \ \isacommand{show}\isamarkupfalse%
\ {\isachardoublequoteopen}control\ {\isacharparenleft}{\kern0pt}Suc\ {\isacharparenleft}{\kern0pt}Suc\ n{\isacharparenright}{\kern0pt}{\isacharparenright}{\kern0pt}\ {\isacharparenleft}{\kern0pt}R\ {\isacharparenleft}{\kern0pt}Suc\ {\isacharparenleft}{\kern0pt}Suc\ n{\isacharparenright}{\kern0pt}{\isacharparenright}{\kern0pt}{\isacharparenright}{\kern0pt}\ {\isacharasterisk}{\kern0pt}\isanewline
\ \ \ \ \ \ \ \ \ \ {\isacharparenleft}{\kern0pt}\ {\isacharbar}{\kern0pt}Deutsch{\isachardot}{\kern0pt}zero{\isasymrangle}\ {\isacharplus}{\kern0pt}\ exp\ {\isacharparenleft}{\kern0pt}{\isadigit{2}}\ {\isacharasterisk}{\kern0pt}\ {\isasymi}\ {\isacharasterisk}{\kern0pt}\ complex{\isacharunderscore}{\kern0pt}of{\isacharunderscore}{\kern0pt}real\ pi\ {\isacharasterisk}{\kern0pt}\ complex{\isacharunderscore}{\kern0pt}of{\isacharunderscore}{\kern0pt}nat\ {\isacharparenleft}{\kern0pt}j\ div\ {\isadigit{2}}{\isacharparenright}{\kern0pt}\ {\isacharslash}{\kern0pt}\ {\isadigit{2}}\ {\isacharcircum}{\kern0pt}\ Suc\ n{\isacharparenright}{\kern0pt}\isanewline
\ \ \ \ \ \ \ \ \ \ {\isasymcdot}\isactrlsub m\ {\isacharbar}{\kern0pt}Deutsch{\isachardot}{\kern0pt}one{\isasymrangle}\ {\isasymOtimes}\ {\isacharbar}{\kern0pt}state{\isacharunderscore}{\kern0pt}basis\ n\ {\isacharparenleft}{\kern0pt}j\ mod\ {\isadigit{2}}\ {\isacharcircum}{\kern0pt}\ Suc\ n\ div\ {\isadigit{2}}{\isacharparenright}{\kern0pt}{\isasymrangle}\ {\isasymOtimes}\ {\isacharbar}{\kern0pt}state{\isacharunderscore}{\kern0pt}basis\ {\isadigit{1}}\ {\isacharparenleft}{\kern0pt}j\ mod\ {\isadigit{2}}{\isacharparenright}{\kern0pt}{\isasymrangle}{\isacharparenright}{\kern0pt}\ {\isacharequal}{\kern0pt}\isanewline
\ \ \ \ \ \ \ \ \ \ {\isacharbar}{\kern0pt}Deutsch{\isachardot}{\kern0pt}zero{\isasymrangle}\ {\isacharplus}{\kern0pt}\ exp\ {\isacharparenleft}{\kern0pt}{\isadigit{2}}\ {\isacharasterisk}{\kern0pt}\ {\isasymi}\ {\isacharasterisk}{\kern0pt}\ complex{\isacharunderscore}{\kern0pt}of{\isacharunderscore}{\kern0pt}real\ pi\ {\isacharasterisk}{\kern0pt}\ complex{\isacharunderscore}{\kern0pt}of{\isacharunderscore}{\kern0pt}nat\ j\ {\isacharslash}{\kern0pt}\ {\isadigit{2}}\ {\isacharcircum}{\kern0pt}\ Suc\ {\isacharparenleft}{\kern0pt}Suc\ n{\isacharparenright}{\kern0pt}{\isacharparenright}{\kern0pt}\ {\isasymcdot}\isactrlsub m\isanewline
\ \ \ \ \ \ \ \ \ \ {\isacharbar}{\kern0pt}Deutsch{\isachardot}{\kern0pt}one{\isasymrangle}\ {\isasymOtimes}\ {\isacharbar}{\kern0pt}state{\isacharunderscore}{\kern0pt}basis\ n\ {\isacharparenleft}{\kern0pt}j\ mod\ {\isadigit{2}}\ {\isacharcircum}{\kern0pt}\ Suc\ n\ div\ {\isadigit{2}}{\isacharparenright}{\kern0pt}{\isasymrangle}\ {\isasymOtimes}\ {\isacharbar}{\kern0pt}state{\isacharunderscore}{\kern0pt}basis\ {\isadigit{1}}\ {\isacharparenleft}{\kern0pt}j\ mod\ {\isadigit{2}}{\isacharparenright}{\kern0pt}{\isasymrangle}{\isachardoublequoteclose}\isanewline
\ \ \ \ \isacommand{proof}\isamarkupfalse%
\ {\isacharminus}{\kern0pt}\isanewline
\ \ \ \ \ \ \isacommand{have}\isamarkupfalse%
\ {\isachardoublequoteopen}control\ {\isacharparenleft}{\kern0pt}Suc\ {\isacharparenleft}{\kern0pt}Suc\ {\isadigit{0}}{\isacharparenright}{\kern0pt}{\isacharparenright}{\kern0pt}\ {\isacharparenleft}{\kern0pt}R\ {\isacharparenleft}{\kern0pt}Suc\ {\isacharparenleft}{\kern0pt}Suc\ {\isadigit{0}}{\isacharparenright}{\kern0pt}{\isacharparenright}{\kern0pt}{\isacharparenright}{\kern0pt}\ {\isacharasterisk}{\kern0pt}\isanewline
\ \ \ \ \ \ \ \ \ \ {\isacharparenleft}{\kern0pt}\ {\isacharbar}{\kern0pt}Deutsch{\isachardot}{\kern0pt}zero{\isasymrangle}\ {\isacharplus}{\kern0pt}\ exp\ {\isacharparenleft}{\kern0pt}{\isadigit{2}}\ {\isacharasterisk}{\kern0pt}\ {\isasymi}\ {\isacharasterisk}{\kern0pt}\ complex{\isacharunderscore}{\kern0pt}of{\isacharunderscore}{\kern0pt}real\ pi\ {\isacharasterisk}{\kern0pt}\ complex{\isacharunderscore}{\kern0pt}of{\isacharunderscore}{\kern0pt}nat\ {\isacharparenleft}{\kern0pt}j\ div\ {\isadigit{2}}{\isacharparenright}{\kern0pt}\ {\isacharslash}{\kern0pt}\ {\isadigit{2}}\ {\isacharcircum}{\kern0pt}\ Suc\ {\isadigit{0}}{\isacharparenright}{\kern0pt}\isanewline
\ \ \ \ \ \ \ \ \ \ {\isasymcdot}\isactrlsub m\ {\isacharbar}{\kern0pt}Deutsch{\isachardot}{\kern0pt}one{\isasymrangle}\ {\isasymOtimes}\ {\isacharbar}{\kern0pt}state{\isacharunderscore}{\kern0pt}basis\ {\isadigit{0}}\ {\isacharparenleft}{\kern0pt}j\ mod\ {\isadigit{2}}\ {\isacharcircum}{\kern0pt}\ Suc\ {\isadigit{0}}\ div\ {\isadigit{2}}{\isacharparenright}{\kern0pt}{\isasymrangle}\ {\isasymOtimes}\ {\isacharbar}{\kern0pt}state{\isacharunderscore}{\kern0pt}basis\ {\isadigit{1}}\ {\isacharparenleft}{\kern0pt}j\ mod\ {\isadigit{2}}{\isacharparenright}{\kern0pt}{\isasymrangle}{\isacharparenright}{\kern0pt}\ {\isacharequal}{\kern0pt}\isanewline
\ \ \ \ \ \ \ \ \ \ control{\isadigit{2}}\ {\isacharparenleft}{\kern0pt}R\ {\isadigit{2}}{\isacharparenright}{\kern0pt}\ {\isacharasterisk}{\kern0pt}\isanewline
\ \ \ \ \ \ \ \ \ \ {\isacharparenleft}{\kern0pt}\ {\isacharbar}{\kern0pt}Deutsch{\isachardot}{\kern0pt}zero{\isasymrangle}\ {\isacharplus}{\kern0pt}\ exp\ {\isacharparenleft}{\kern0pt}{\isadigit{2}}\ {\isacharasterisk}{\kern0pt}\ {\isasymi}\ {\isacharasterisk}{\kern0pt}\ complex{\isacharunderscore}{\kern0pt}of{\isacharunderscore}{\kern0pt}real\ pi\ {\isacharasterisk}{\kern0pt}\ complex{\isacharunderscore}{\kern0pt}of{\isacharunderscore}{\kern0pt}nat\ {\isacharparenleft}{\kern0pt}j\ div\ {\isadigit{2}}{\isacharparenright}{\kern0pt}\ {\isacharslash}{\kern0pt}\ {\isadigit{2}}\ {\isacharcircum}{\kern0pt}\ Suc\ {\isadigit{0}}{\isacharparenright}{\kern0pt}\isanewline
\ \ \ \ \ \ \ \ \ \ {\isasymcdot}\isactrlsub m\ {\isacharbar}{\kern0pt}Deutsch{\isachardot}{\kern0pt}one{\isasymrangle}\ {\isasymOtimes}\ {\isacharbar}{\kern0pt}state{\isacharunderscore}{\kern0pt}basis\ {\isadigit{0}}\ {\isacharparenleft}{\kern0pt}j\ mod\ {\isadigit{2}}\ {\isacharcircum}{\kern0pt}\ Suc\ {\isadigit{0}}\ div\ {\isadigit{2}}{\isacharparenright}{\kern0pt}{\isasymrangle}\ {\isasymOtimes}\ {\isacharbar}{\kern0pt}state{\isacharunderscore}{\kern0pt}basis\ {\isadigit{1}}\ {\isacharparenleft}{\kern0pt}j\ mod\ {\isadigit{2}}{\isacharparenright}{\kern0pt}{\isasymrangle}{\isacharparenright}{\kern0pt}{\isachardoublequoteclose}\isanewline
\ \ \ \ \ \ \ \ \isacommand{using}\isamarkupfalse%
\ control{\isachardot}{\kern0pt}simps\ \isacommand{by}\isamarkupfalse%
\ {\isacharparenleft}{\kern0pt}metis\ One{\isacharunderscore}{\kern0pt}nat{\isacharunderscore}{\kern0pt}def\ Suc{\isacharunderscore}{\kern0pt}{\isadigit{1}}{\isacharparenright}{\kern0pt}\isanewline
\ \ \ \ \ \ \isacommand{also}\isamarkupfalse%
\ \isacommand{have}\isamarkupfalse%
\ {\isachardoublequoteopen}{\isasymdots}\ {\isacharequal}{\kern0pt}\ control{\isadigit{2}}\ {\isacharparenleft}{\kern0pt}R\ {\isadigit{2}}{\isacharparenright}{\kern0pt}\ {\isacharasterisk}{\kern0pt}\isanewline
\ \ \ \ \ \ \ \ \ \ {\isacharparenleft}{\kern0pt}\ {\isacharbar}{\kern0pt}Deutsch{\isachardot}{\kern0pt}zero{\isasymrangle}\ {\isacharplus}{\kern0pt}\ exp\ {\isacharparenleft}{\kern0pt}{\isadigit{2}}\ {\isacharasterisk}{\kern0pt}\ {\isasymi}\ {\isacharasterisk}{\kern0pt}\ complex{\isacharunderscore}{\kern0pt}of{\isacharunderscore}{\kern0pt}real\ pi\ {\isacharasterisk}{\kern0pt}\ complex{\isacharunderscore}{\kern0pt}of{\isacharunderscore}{\kern0pt}nat\ {\isacharparenleft}{\kern0pt}j\ div\ {\isadigit{2}}{\isacharparenright}{\kern0pt}\ {\isacharslash}{\kern0pt}\ {\isadigit{2}}\ {\isacharcircum}{\kern0pt}\ Suc\ {\isadigit{0}}{\isacharparenright}{\kern0pt}\isanewline
\ \ \ \ \ \ \ \ \ \ {\isasymcdot}\isactrlsub m\ {\isacharbar}{\kern0pt}Deutsch{\isachardot}{\kern0pt}one{\isasymrangle}\ {\isasymOtimes}\ {\isacharbar}{\kern0pt}state{\isacharunderscore}{\kern0pt}basis\ {\isadigit{1}}\ {\isacharparenleft}{\kern0pt}j\ mod\ {\isadigit{2}}{\isacharparenright}{\kern0pt}{\isasymrangle}{\isacharparenright}{\kern0pt}{\isachardoublequoteclose}\isanewline
\ \ \ \ \ \ \ \ \isacommand{using}\isamarkupfalse%
\ state{\isacharunderscore}{\kern0pt}basis{\isacharunderscore}{\kern0pt}def\ unit{\isacharunderscore}{\kern0pt}vec{\isacharunderscore}{\kern0pt}def\ ket{\isacharunderscore}{\kern0pt}vec{\isacharunderscore}{\kern0pt}def\isanewline
\ \ \ \ \ \ \ \ \isacommand{by}\isamarkupfalse%
\ {\isacharparenleft}{\kern0pt}smt\ {\isacharparenleft}{\kern0pt}verit{\isacharcomma}{\kern0pt}\ del{\isacharunderscore}{\kern0pt}insts{\isacharparenright}{\kern0pt}\ H{\isacharunderscore}{\kern0pt}inv\ H{\isacharunderscore}{\kern0pt}is{\isacharunderscore}{\kern0pt}gate\ One{\isacharunderscore}{\kern0pt}nat{\isacharunderscore}{\kern0pt}def\ gate{\isacharunderscore}{\kern0pt}def\ index{\isacharunderscore}{\kern0pt}mult{\isacharunderscore}{\kern0pt}mat{\isacharparenleft}{\kern0pt}{\isadigit{2}}{\isacharparenright}{\kern0pt}\ \isanewline
\ \ \ \ \ \ \ \ \ \ \ \ index{\isacharunderscore}{\kern0pt}one{\isacharunderscore}{\kern0pt}mat{\isacharparenleft}{\kern0pt}{\isadigit{2}}{\isacharparenright}{\kern0pt}\ mod{\isacharunderscore}{\kern0pt}less{\isacharunderscore}{\kern0pt}divisor\ mod{\isacharunderscore}{\kern0pt}mod{\isacharunderscore}{\kern0pt}trivial\ pos{\isadigit{2}}\ state{\isacharunderscore}{\kern0pt}basis{\isacharunderscore}{\kern0pt}dec{\isacharprime}{\kern0pt}\ \isanewline
\ \ \ \ \ \ \ \ \ \ \ \ tensor{\isacharunderscore}{\kern0pt}mat{\isacharunderscore}{\kern0pt}is{\isacharunderscore}{\kern0pt}assoc{\isacharparenright}{\kern0pt}\isanewline
\ \ \ \ \ \ \isacommand{also}\isamarkupfalse%
\ \isacommand{have}\isamarkupfalse%
\ {\isachardoublequoteopen}{\isasymdots}\ {\isacharequal}{\kern0pt}\ {\isacharparenleft}{\kern0pt}\ {\isacharbar}{\kern0pt}zero{\isasymrangle}\ {\isacharplus}{\kern0pt}\ exp\ {\isacharparenleft}{\kern0pt}{\isadigit{2}}{\isacharasterisk}{\kern0pt}{\isasymi}{\isacharasterisk}{\kern0pt}pi{\isacharasterisk}{\kern0pt}complex{\isacharunderscore}{\kern0pt}of{\isacharunderscore}{\kern0pt}nat\ j\ {\isacharslash}{\kern0pt}\ {\isadigit{2}}{\isacharcircum}{\kern0pt}{\isacharparenleft}{\kern0pt}Suc\ {\isacharparenleft}{\kern0pt}Suc\ {\isadigit{0}}{\isacharparenright}{\kern0pt}{\isacharparenright}{\kern0pt}{\isacharparenright}{\kern0pt}\ {\isasymcdot}\isactrlsub m\ {\isacharbar}{\kern0pt}one{\isasymrangle}{\isacharparenright}{\kern0pt}\ {\isasymOtimes}\isanewline
\ \ \ \ \ \ \ \ \ \ \ \ \ \ \ \ \ \ \ \ \ \ {\isacharbar}{\kern0pt}state{\isacharunderscore}{\kern0pt}basis\ {\isadigit{1}}\ {\isacharparenleft}{\kern0pt}j\ mod\ {\isadigit{2}}{\isacharparenright}{\kern0pt}{\isasymrangle}{\isachardoublequoteclose}\isanewline
\ \ \ \ \ \ \isacommand{proof}\isamarkupfalse%
\ {\isacharparenleft}{\kern0pt}rule\ disjE{\isacharparenright}{\kern0pt}\isanewline
\ \ \ \ \ \ \ \ \isacommand{show}\isamarkupfalse%
\ {\isachardoublequoteopen}j\ mod\ {\isadigit{2}}\ {\isacharequal}{\kern0pt}\ {\isadigit{0}}\ {\isasymor}\ j\ mod\ {\isadigit{2}}\ {\isacharequal}{\kern0pt}\ {\isadigit{1}}{\isachardoublequoteclose}\ \isacommand{by}\isamarkupfalse%
\ auto\isanewline
\ \ \ \ \ \ \isacommand{next}\isamarkupfalse%
\isanewline
\ \ \ \ \ \ \ \ \isacommand{assume}\isamarkupfalse%
\ jm{\isadigit{0}}{\isacharcolon}{\kern0pt}{\isachardoublequoteopen}j\ mod\ {\isadigit{2}}\ {\isacharequal}{\kern0pt}\ {\isadigit{0}}{\isachardoublequoteclose}\isanewline
\ \ \ \ \ \ \ \ \isacommand{hence}\isamarkupfalse%
\ jdj{\isacharcolon}{\kern0pt}{\isachardoublequoteopen}j\ div\ {\isadigit{2}}\ {\isacharequal}{\kern0pt}\ j{\isacharslash}{\kern0pt}{\isadigit{2}}{\isachardoublequoteclose}\ \isacommand{by}\isamarkupfalse%
\ auto\isanewline
\ \ \ \ \ \ \ \ \isacommand{have}\isamarkupfalse%
\ {\isachardoublequoteopen}control{\isadigit{2}}\ {\isacharparenleft}{\kern0pt}R\ {\isadigit{2}}{\isacharparenright}{\kern0pt}\ {\isacharasterisk}{\kern0pt}\isanewline
\ \ \ \ \ \ \ \ \ \ {\isacharparenleft}{\kern0pt}\ {\isacharbar}{\kern0pt}Deutsch{\isachardot}{\kern0pt}zero{\isasymrangle}\ {\isacharplus}{\kern0pt}\ exp\ {\isacharparenleft}{\kern0pt}{\isadigit{2}}\ {\isacharasterisk}{\kern0pt}\ {\isasymi}\ {\isacharasterisk}{\kern0pt}\ complex{\isacharunderscore}{\kern0pt}of{\isacharunderscore}{\kern0pt}real\ pi\ {\isacharasterisk}{\kern0pt}\ complex{\isacharunderscore}{\kern0pt}of{\isacharunderscore}{\kern0pt}nat\ {\isacharparenleft}{\kern0pt}j\ div\ {\isadigit{2}}{\isacharparenright}{\kern0pt}\ {\isacharslash}{\kern0pt}\ {\isadigit{2}}\ {\isacharcircum}{\kern0pt}\ Suc\ {\isadigit{0}}{\isacharparenright}{\kern0pt}\isanewline
\ \ \ \ \ \ \ \ \ \ {\isasymcdot}\isactrlsub m\ {\isacharbar}{\kern0pt}Deutsch{\isachardot}{\kern0pt}one{\isasymrangle}\ {\isasymOtimes}\ {\isacharbar}{\kern0pt}state{\isacharunderscore}{\kern0pt}basis\ {\isadigit{1}}\ {\isacharparenleft}{\kern0pt}j\ mod\ {\isadigit{2}}{\isacharparenright}{\kern0pt}{\isasymrangle}{\isacharparenright}{\kern0pt}\ {\isacharequal}{\kern0pt}\isanewline
\ \ \ \ \ \ \ \ \ \ control{\isadigit{2}}\ {\isacharparenleft}{\kern0pt}R\ {\isadigit{2}}{\isacharparenright}{\kern0pt}\ {\isacharasterisk}{\kern0pt}\isanewline
\ \ \ \ \ \ \ \ \ \ {\isacharparenleft}{\kern0pt}\ {\isacharbar}{\kern0pt}Deutsch{\isachardot}{\kern0pt}zero{\isasymrangle}\ {\isacharplus}{\kern0pt}\ exp\ {\isacharparenleft}{\kern0pt}{\isadigit{2}}\ {\isacharasterisk}{\kern0pt}\ {\isasymi}\ {\isacharasterisk}{\kern0pt}\ complex{\isacharunderscore}{\kern0pt}of{\isacharunderscore}{\kern0pt}real\ pi\ {\isacharasterisk}{\kern0pt}\ complex{\isacharunderscore}{\kern0pt}of{\isacharunderscore}{\kern0pt}nat\ {\isacharparenleft}{\kern0pt}j\ div\ {\isadigit{2}}{\isacharparenright}{\kern0pt}\ {\isacharslash}{\kern0pt}\ {\isadigit{2}}\ {\isacharcircum}{\kern0pt}\ Suc\ {\isadigit{0}}{\isacharparenright}{\kern0pt}\isanewline
\ \ \ \ \ \ \ \ \ \ {\isasymcdot}\isactrlsub m\ {\isacharbar}{\kern0pt}Deutsch{\isachardot}{\kern0pt}one{\isasymrangle}\ {\isasymOtimes}\ {\isacharbar}{\kern0pt}zero{\isasymrangle}{\isacharparenright}{\kern0pt}{\isachardoublequoteclose}\isanewline
\ \ \ \ \ \ \ \ \ \ \isacommand{using}\isamarkupfalse%
\ jm{\isadigit{0}}\ state{\isacharunderscore}{\kern0pt}basis{\isacharunderscore}{\kern0pt}def\ mat{\isacharunderscore}{\kern0pt}of{\isacharunderscore}{\kern0pt}cols{\isacharunderscore}{\kern0pt}list{\isacharunderscore}{\kern0pt}def\ \isacommand{by}\isamarkupfalse%
\ fastforce\isanewline
\ \ \ \ \ \ \ \ \isacommand{also}\isamarkupfalse%
\ \isacommand{have}\isamarkupfalse%
\ {\isachardoublequoteopen}{\isasymdots}\ {\isacharequal}{\kern0pt}\ {\isacharbar}{\kern0pt}Deutsch{\isachardot}{\kern0pt}zero{\isasymrangle}\ {\isacharplus}{\kern0pt}\ exp\ {\isacharparenleft}{\kern0pt}{\isadigit{2}}{\isacharasterisk}{\kern0pt}{\isasymi}{\isacharasterisk}{\kern0pt}pi{\isacharasterisk}{\kern0pt}\ complex{\isacharunderscore}{\kern0pt}of{\isacharunderscore}{\kern0pt}nat\ {\isacharparenleft}{\kern0pt}j\ div\ {\isadigit{2}}{\isacharparenright}{\kern0pt}\ {\isacharslash}{\kern0pt}\ {\isadigit{2}}\ {\isacharcircum}{\kern0pt}\ Suc\ {\isadigit{0}}{\isacharparenright}{\kern0pt}\isanewline
\ \ \ \ \ \ \ \ \ \ \ \ \ \ \ \ \ \ \ \ \ \ \ \ {\isasymcdot}\isactrlsub m\ {\isacharbar}{\kern0pt}Deutsch{\isachardot}{\kern0pt}one{\isasymrangle}\ {\isasymOtimes}\ {\isacharbar}{\kern0pt}zero{\isasymrangle}{\isachardoublequoteclose}\isanewline
\ \ \ \ \ \ \ \ \ \ \isacommand{using}\isamarkupfalse%
\ control{\isadigit{2}}{\isacharunderscore}{\kern0pt}zero\ \isacommand{by}\isamarkupfalse%
\ {\isacharparenleft}{\kern0pt}simp\ add{\isacharcolon}{\kern0pt}\ ket{\isacharunderscore}{\kern0pt}vec{\isacharunderscore}{\kern0pt}def{\isacharparenright}{\kern0pt}\isanewline
\ \ \ \ \ \ \ \ \isacommand{also}\isamarkupfalse%
\ \isacommand{have}\isamarkupfalse%
\ {\isachardoublequoteopen}{\isasymdots}\ {\isacharequal}{\kern0pt}\ {\isacharbar}{\kern0pt}Deutsch{\isachardot}{\kern0pt}zero{\isasymrangle}\ {\isacharplus}{\kern0pt}\ exp\ {\isacharparenleft}{\kern0pt}{\isadigit{2}}\ {\isacharasterisk}{\kern0pt}\ {\isasymi}\ {\isacharasterisk}{\kern0pt}\ complex{\isacharunderscore}{\kern0pt}of{\isacharunderscore}{\kern0pt}real\ pi\ {\isacharasterisk}{\kern0pt}\isanewline
\ \ \ \ \ \ \ \ \ \ \ \ \ \ \ \ \ \ \ \ \ \ \ \ complex{\isacharunderscore}{\kern0pt}of{\isacharunderscore}{\kern0pt}nat\ j\ {\isacharslash}{\kern0pt}\ {\isadigit{2}}\ {\isacharcircum}{\kern0pt}\ Suc\ {\isacharparenleft}{\kern0pt}Suc\ {\isadigit{0}}{\isacharparenright}{\kern0pt}{\isacharparenright}{\kern0pt}\ {\isasymcdot}\isactrlsub m\ {\isacharbar}{\kern0pt}Deutsch{\isachardot}{\kern0pt}one{\isasymrangle}\ {\isasymOtimes}\isanewline
\ \ \ \ \ \ \ \ \ \ \ \ \ \ \ \ \ \ \ \ \ \ \ \ {\isacharbar}{\kern0pt}state{\isacharunderscore}{\kern0pt}basis\ {\isadigit{1}}\ {\isacharparenleft}{\kern0pt}j\ mod\ {\isadigit{2}}{\isacharparenright}{\kern0pt}{\isasymrangle}{\isachardoublequoteclose}\ \isanewline
\ \ \ \ \ \ \ \ \ \ \isacommand{using}\isamarkupfalse%
\ jm{\isadigit{0}}\ state{\isacharunderscore}{\kern0pt}basis{\isacharunderscore}{\kern0pt}def\ mat{\isacharunderscore}{\kern0pt}of{\isacharunderscore}{\kern0pt}cols{\isacharunderscore}{\kern0pt}list{\isacharunderscore}{\kern0pt}def\ jdj\ \isanewline
\ \ \ \ \ \ \ \ \ \ \isacommand{by}\isamarkupfalse%
\ {\isacharparenleft}{\kern0pt}smt\ {\isacharparenleft}{\kern0pt}verit{\isacharcomma}{\kern0pt}\ best{\isacharparenright}{\kern0pt}\ Euclidean{\isacharunderscore}{\kern0pt}Rings{\isachardot}{\kern0pt}div{\isacharunderscore}{\kern0pt}eq{\isacharunderscore}{\kern0pt}{\isadigit{0}}{\isacharunderscore}{\kern0pt}iff\ One{\isacharunderscore}{\kern0pt}nat{\isacharunderscore}{\kern0pt}def\ Suc{\isacharunderscore}{\kern0pt}{\isadigit{1}}\ assms\ \isanewline
\ \ \ \ \ \ \ \ \ \ \ \ \ \ divide{\isacharunderscore}{\kern0pt}divide{\isacharunderscore}{\kern0pt}eq{\isacharunderscore}{\kern0pt}left\ divide{\isacharunderscore}{\kern0pt}eq{\isacharunderscore}{\kern0pt}{\isadigit{0}}{\isacharunderscore}{\kern0pt}iff\ less{\isacharunderscore}{\kern0pt}{\isadigit{2}}{\isacharunderscore}{\kern0pt}cases{\isacharunderscore}{\kern0pt}iff\ less{\isacharunderscore}{\kern0pt}power{\isacharunderscore}{\kern0pt}add{\isacharunderscore}{\kern0pt}imp{\isacharunderscore}{\kern0pt}div{\isacharunderscore}{\kern0pt}less\ n{\isadigit{0}}\isanewline
\ \ \ \ \ \ \ \ \ \ \ \ \ \ neq{\isacharunderscore}{\kern0pt}imp{\isacharunderscore}{\kern0pt}neq{\isacharunderscore}{\kern0pt}div{\isacharunderscore}{\kern0pt}or{\isacharunderscore}{\kern0pt}mod\ of{\isacharunderscore}{\kern0pt}nat{\isacharunderscore}{\kern0pt}{\isadigit{0}}\ of{\isacharunderscore}{\kern0pt}nat{\isacharunderscore}{\kern0pt}{\isadigit{1}}\ of{\isacharunderscore}{\kern0pt}nat{\isacharunderscore}{\kern0pt}Suc\ of{\isacharunderscore}{\kern0pt}nat{\isacharunderscore}{\kern0pt}numeral\ of{\isacharunderscore}{\kern0pt}real{\isacharunderscore}{\kern0pt}{\isadigit{1}}\ \isanewline
\ \ \ \ \ \ \ \ \ \ \ \ \ \ of{\isacharunderscore}{\kern0pt}real{\isacharunderscore}{\kern0pt}divide\ of{\isacharunderscore}{\kern0pt}real{\isacharunderscore}{\kern0pt}numeral\ power{\isacharunderscore}{\kern0pt}Suc\ power{\isacharunderscore}{\kern0pt}one{\isacharunderscore}{\kern0pt}right\ times{\isacharunderscore}{\kern0pt}divide{\isacharunderscore}{\kern0pt}eq{\isacharunderscore}{\kern0pt}right\ \isanewline
\ \ \ \ \ \ \ \ \ \ \ \ \ \ two{\isacharunderscore}{\kern0pt}div{\isacharunderscore}{\kern0pt}two\ two{\isacharunderscore}{\kern0pt}mod{\isacharunderscore}{\kern0pt}two{\isacharparenright}{\kern0pt}\isanewline
\ \ \ \ \ \ \ \ \isacommand{finally}\isamarkupfalse%
\ \isacommand{show}\isamarkupfalse%
\ {\isacharquery}{\kern0pt}thesis\ \isacommand{by}\isamarkupfalse%
\ this\isanewline
\ \ \ \ \ \ \isacommand{next}\isamarkupfalse%
\isanewline
\ \ \ \ \ \ \ \ \isacommand{assume}\isamarkupfalse%
\ jm{\isadigit{1}}{\isacharcolon}{\kern0pt}{\isachardoublequoteopen}j\ mod\ {\isadigit{2}}\ {\isacharequal}{\kern0pt}\ {\isadigit{1}}{\isachardoublequoteclose}\isanewline
\ \ \ \ \ \ \ \ \isacommand{have}\isamarkupfalse%
\ {\isachardoublequoteopen}control{\isadigit{2}}\ {\isacharparenleft}{\kern0pt}R\ {\isadigit{2}}{\isacharparenright}{\kern0pt}\ {\isacharasterisk}{\kern0pt}\isanewline
\ \ \ \ \ \ \ \ \ \ {\isacharparenleft}{\kern0pt}\ {\isacharbar}{\kern0pt}Deutsch{\isachardot}{\kern0pt}zero{\isasymrangle}\ {\isacharplus}{\kern0pt}\ exp\ {\isacharparenleft}{\kern0pt}{\isadigit{2}}\ {\isacharasterisk}{\kern0pt}\ {\isasymi}\ {\isacharasterisk}{\kern0pt}\ complex{\isacharunderscore}{\kern0pt}of{\isacharunderscore}{\kern0pt}real\ pi\ {\isacharasterisk}{\kern0pt}\ complex{\isacharunderscore}{\kern0pt}of{\isacharunderscore}{\kern0pt}nat\ {\isacharparenleft}{\kern0pt}j\ div\ {\isadigit{2}}{\isacharparenright}{\kern0pt}\ {\isacharslash}{\kern0pt}\ {\isadigit{2}}\ {\isacharcircum}{\kern0pt}\ Suc\ {\isadigit{0}}{\isacharparenright}{\kern0pt}\isanewline
\ \ \ \ \ \ \ \ \ \ {\isasymcdot}\isactrlsub m\ {\isacharbar}{\kern0pt}Deutsch{\isachardot}{\kern0pt}one{\isasymrangle}\ {\isasymOtimes}\ {\isacharbar}{\kern0pt}state{\isacharunderscore}{\kern0pt}basis\ {\isadigit{1}}\ {\isacharparenleft}{\kern0pt}j\ mod\ {\isadigit{2}}{\isacharparenright}{\kern0pt}{\isasymrangle}{\isacharparenright}{\kern0pt}\ {\isacharequal}{\kern0pt}\isanewline
\ \ \ \ \ \ \ \ \ \ control{\isadigit{2}}\ {\isacharparenleft}{\kern0pt}R\ {\isadigit{2}}{\isacharparenright}{\kern0pt}\ {\isacharasterisk}{\kern0pt}\isanewline
\ \ \ \ \ \ \ \ \ \ {\isacharparenleft}{\kern0pt}\ {\isacharbar}{\kern0pt}Deutsch{\isachardot}{\kern0pt}zero{\isasymrangle}\ {\isacharplus}{\kern0pt}\ exp\ {\isacharparenleft}{\kern0pt}{\isadigit{2}}\ {\isacharasterisk}{\kern0pt}\ {\isasymi}\ {\isacharasterisk}{\kern0pt}\ complex{\isacharunderscore}{\kern0pt}of{\isacharunderscore}{\kern0pt}real\ pi\ {\isacharasterisk}{\kern0pt}\ complex{\isacharunderscore}{\kern0pt}of{\isacharunderscore}{\kern0pt}nat\ {\isacharparenleft}{\kern0pt}j\ div\ {\isadigit{2}}{\isacharparenright}{\kern0pt}\ {\isacharslash}{\kern0pt}\ {\isadigit{2}}\ {\isacharcircum}{\kern0pt}\ Suc\ {\isadigit{0}}{\isacharparenright}{\kern0pt}\isanewline
\ \ \ \ \ \ \ \ \ \ {\isasymcdot}\isactrlsub m\ {\isacharbar}{\kern0pt}Deutsch{\isachardot}{\kern0pt}one{\isasymrangle}\ {\isasymOtimes}\ {\isacharbar}{\kern0pt}one{\isasymrangle}{\isacharparenright}{\kern0pt}{\isachardoublequoteclose}\isanewline
\ \ \ \ \ \ \ \ \ \ \isacommand{using}\isamarkupfalse%
\ jm{\isadigit{1}}\ \isacommand{by}\isamarkupfalse%
\ {\isacharparenleft}{\kern0pt}simp\ add{\isacharcolon}{\kern0pt}\ state{\isacharunderscore}{\kern0pt}basis{\isacharunderscore}{\kern0pt}def{\isacharparenright}{\kern0pt}\isanewline
\ \ \ \ \ \ \ \ \isacommand{also}\isamarkupfalse%
\ \isacommand{have}\isamarkupfalse%
\ {\isachardoublequoteopen}{\isasymdots}\ {\isacharequal}{\kern0pt}\ {\isacharparenleft}{\kern0pt}{\isacharparenleft}{\kern0pt}R\ {\isadigit{2}}{\isacharparenright}{\kern0pt}\ {\isacharasterisk}{\kern0pt}\ \isanewline
\ \ \ \ \ \ \ \ \ \ \ {\isacharparenleft}{\kern0pt}\ {\isacharbar}{\kern0pt}Deutsch{\isachardot}{\kern0pt}zero{\isasymrangle}\ {\isacharplus}{\kern0pt}\ exp\ {\isacharparenleft}{\kern0pt}{\isadigit{2}}\ {\isacharasterisk}{\kern0pt}\ {\isasymi}\ {\isacharasterisk}{\kern0pt}\ complex{\isacharunderscore}{\kern0pt}of{\isacharunderscore}{\kern0pt}real\ pi\ {\isacharasterisk}{\kern0pt}\ complex{\isacharunderscore}{\kern0pt}of{\isacharunderscore}{\kern0pt}nat\ {\isacharparenleft}{\kern0pt}j\ div\ {\isadigit{2}}{\isacharparenright}{\kern0pt}\ {\isacharslash}{\kern0pt}\ {\isadigit{2}}\ {\isacharcircum}{\kern0pt}\ Suc\ {\isadigit{0}}{\isacharparenright}{\kern0pt}\isanewline
\ \ \ \ \ \ \ \ \ \ \ \ {\isasymcdot}\isactrlsub m\ {\isacharbar}{\kern0pt}Deutsch{\isachardot}{\kern0pt}one{\isasymrangle}{\isacharparenright}{\kern0pt}{\isacharparenright}{\kern0pt}\ {\isasymOtimes}\ {\isacharbar}{\kern0pt}one{\isasymrangle}{\isachardoublequoteclose}\isanewline
\ \ \ \ \ \ \ \ \ \ \isacommand{using}\isamarkupfalse%
\ control{\isadigit{2}}{\isacharunderscore}{\kern0pt}one\ ket{\isacharunderscore}{\kern0pt}vec{\isacharunderscore}{\kern0pt}def\ R{\isacharunderscore}{\kern0pt}def\ mat{\isacharunderscore}{\kern0pt}of{\isacharunderscore}{\kern0pt}cols{\isacharunderscore}{\kern0pt}list{\isacharunderscore}{\kern0pt}def\ \isacommand{by}\isamarkupfalse%
\ simp\isanewline
\ \ \ \ \ \ \ \ \isacommand{also}\isamarkupfalse%
\ \isacommand{have}\isamarkupfalse%
\ {\isachardoublequoteopen}{\isasymdots}\ {\isacharequal}{\kern0pt}\ {\isacharparenleft}{\kern0pt}\ {\isacharbar}{\kern0pt}zero{\isasymrangle}\ {\isacharplus}{\kern0pt}\ exp\ {\isacharparenleft}{\kern0pt}{\isadigit{2}}{\isacharasterisk}{\kern0pt}{\isasymi}{\isacharasterisk}{\kern0pt}pi{\isacharasterisk}{\kern0pt}complex{\isacharunderscore}{\kern0pt}of{\isacharunderscore}{\kern0pt}nat\ j{\isacharslash}{\kern0pt}{\isadigit{2}}{\isacharcircum}{\kern0pt}Suc\ {\isacharparenleft}{\kern0pt}Suc\ {\isadigit{0}}{\isacharparenright}{\kern0pt}{\isacharparenright}{\kern0pt}\ {\isasymcdot}\isactrlsub m\ {\isacharbar}{\kern0pt}one{\isasymrangle}{\isacharparenright}{\kern0pt}\ {\isasymOtimes}\ {\isacharbar}{\kern0pt}one{\isasymrangle}{\isachardoublequoteclose}\isanewline
\ \ \ \ \ \ \ \ \ \ \isacommand{using}\isamarkupfalse%
\ R{\isacharunderscore}{\kern0pt}action\ jm{\isadigit{1}}\ assms\ \isacommand{by}\isamarkupfalse%
\ {\isacharparenleft}{\kern0pt}metis\ One{\isacharunderscore}{\kern0pt}nat{\isacharunderscore}{\kern0pt}def\ Suc{\isacharunderscore}{\kern0pt}{\isadigit{1}}\ n{\isadigit{0}}{\isacharparenright}{\kern0pt}\isanewline
\ \ \ \ \ \ \ \ \isacommand{finally}\isamarkupfalse%
\ \isacommand{show}\isamarkupfalse%
\ {\isacharquery}{\kern0pt}thesis\ \isacommand{by}\isamarkupfalse%
\ {\isacharparenleft}{\kern0pt}metis\ jm{\isadigit{1}}\ power{\isacharunderscore}{\kern0pt}one{\isacharunderscore}{\kern0pt}right\ state{\isacharunderscore}{\kern0pt}basis{\isacharunderscore}{\kern0pt}def{\isacharparenright}{\kern0pt}\isanewline
\ \ \ \ \ \ \isacommand{qed}\isamarkupfalse%
\isanewline
\ \ \ \ \ \ \isacommand{finally}\isamarkupfalse%
\ \isacommand{show}\isamarkupfalse%
\ {\isacharquery}{\kern0pt}thesis\isanewline
\ \ \ \ \ \ \ \ \isacommand{by}\isamarkupfalse%
\ {\isacharparenleft}{\kern0pt}smt\ {\isacharparenleft}{\kern0pt}verit{\isacharcomma}{\kern0pt}\ best{\isacharparenright}{\kern0pt}\ Euclidean{\isacharunderscore}{\kern0pt}Rings{\isachardot}{\kern0pt}div{\isacharunderscore}{\kern0pt}eq{\isacharunderscore}{\kern0pt}{\isadigit{0}}{\isacharunderscore}{\kern0pt}iff\ Suc{\isacharunderscore}{\kern0pt}{\isadigit{1}}\ mod{\isacharunderscore}{\kern0pt}less{\isacharunderscore}{\kern0pt}divisor\ n{\isadigit{0}}\ \isanewline
\ \ \ \ \ \ \ \ \ \ \ \ not{\isacharunderscore}{\kern0pt}mod{\isadigit{2}}{\isacharunderscore}{\kern0pt}eq{\isacharunderscore}{\kern0pt}Suc{\isacharunderscore}{\kern0pt}{\isadigit{0}}{\isacharunderscore}{\kern0pt}eq{\isacharunderscore}{\kern0pt}{\isadigit{0}}\ one{\isacharunderscore}{\kern0pt}mod{\isacharunderscore}{\kern0pt}two{\isacharunderscore}{\kern0pt}eq{\isacharunderscore}{\kern0pt}one\ pos{\isadigit{2}}\ power{\isacharunderscore}{\kern0pt}{\isadigit{0}}\ power{\isacharunderscore}{\kern0pt}one{\isacharunderscore}{\kern0pt}right\ state{\isacharunderscore}{\kern0pt}basis{\isacharunderscore}{\kern0pt}dec{\isacharprime}{\kern0pt}\ \isanewline
\ \ \ \ \ \ \ \ \ \ \ \ tensor{\isacharunderscore}{\kern0pt}mat{\isacharunderscore}{\kern0pt}is{\isacharunderscore}{\kern0pt}assoc{\isacharparenright}{\kern0pt}\isanewline
\ \ \ \ \isacommand{qed}\isamarkupfalse%
\isanewline
\ \ \isacommand{qed}\isamarkupfalse%
\isanewline
\isacommand{next}\isamarkupfalse%
\isanewline
\ \ \isacommand{case}\isamarkupfalse%
\ {\isacharparenleft}{\kern0pt}Suc\ nat{\isacharparenright}{\kern0pt}\isanewline
\ \ \isacommand{then}\isamarkupfalse%
\ \isacommand{show}\isamarkupfalse%
\ {\isacharquery}{\kern0pt}thesis\isanewline
\ \ \isacommand{proof}\isamarkupfalse%
\ {\isacharminus}{\kern0pt}\isanewline
\ \ \ \ \isacommand{assume}\isamarkupfalse%
\ {\isachardoublequoteopen}n\ {\isacharequal}{\kern0pt}\ Suc\ nat{\isachardoublequoteclose}\isanewline
\ \ \ \ \isacommand{define}\isamarkupfalse%
\ jd{\isadigit{2}}\ \isakeyword{where}\ {\isachardoublequoteopen}jd{\isadigit{2}}\ {\isacharequal}{\kern0pt}\ j\ div\ {\isadigit{2}}{\isachardoublequoteclose}\isanewline
\ \ \ \ \isacommand{define}\isamarkupfalse%
\ jm{\isadigit{2}}\ \isakeyword{where}\ {\isachardoublequoteopen}jm{\isadigit{2}}\ {\isacharequal}{\kern0pt}\ j\ mod\ {\isadigit{2}}{\isachardoublequoteclose}\isanewline
\ \ \ \ \isacommand{define}\isamarkupfalse%
\ jm{\isadigit{2}}sn\ \isakeyword{where}\ {\isachardoublequoteopen}jm{\isadigit{2}}sn\ {\isacharequal}{\kern0pt}\ j\ mod\ {\isadigit{2}}{\isacharcircum}{\kern0pt}Suc\ n{\isachardoublequoteclose}\isanewline
\ \ \ \ \isacommand{have}\isamarkupfalse%
\ jeq{\isacharcolon}{\kern0pt}{\isachardoublequoteopen}jm{\isadigit{2}}sn\ mod\ {\isadigit{2}}\ {\isacharequal}{\kern0pt}\ j\ mod\ {\isadigit{2}}{\isachardoublequoteclose}\ \isacommand{using}\isamarkupfalse%
\ jm{\isadigit{2}}sn{\isacharunderscore}{\kern0pt}def\ \isanewline
\ \ \ \ \ \ \isacommand{by}\isamarkupfalse%
\ {\isacharparenleft}{\kern0pt}metis\ One{\isacharunderscore}{\kern0pt}nat{\isacharunderscore}{\kern0pt}def\ Suc{\isacharunderscore}{\kern0pt}le{\isacharunderscore}{\kern0pt}mono\ mod{\isacharunderscore}{\kern0pt}mod{\isacharunderscore}{\kern0pt}power{\isacharunderscore}{\kern0pt}cancel\ power{\isacharunderscore}{\kern0pt}one{\isacharunderscore}{\kern0pt}right\ zero{\isacharunderscore}{\kern0pt}order{\isacharparenleft}{\kern0pt}{\isadigit{1}}{\isacharparenright}{\kern0pt}{\isacharparenright}{\kern0pt}\isanewline
\ \ \ \ \isacommand{have}\isamarkupfalse%
\ {\isachardoublequoteopen}{\isacharparenleft}{\kern0pt}control\ {\isacharparenleft}{\kern0pt}Suc\ {\isacharparenleft}{\kern0pt}Suc\ n{\isacharparenright}{\kern0pt}{\isacharparenright}{\kern0pt}\ {\isacharparenleft}{\kern0pt}R\ {\isacharparenleft}{\kern0pt}Suc\ {\isacharparenleft}{\kern0pt}Suc\ n{\isacharparenright}{\kern0pt}{\isacharparenright}{\kern0pt}{\isacharparenright}{\kern0pt}{\isacharparenright}{\kern0pt}\ {\isacharasterisk}{\kern0pt}\ {\isacharparenleft}{\kern0pt}\ {\isacharbar}{\kern0pt}Deutsch{\isachardot}{\kern0pt}zero{\isasymrangle}\ {\isacharplus}{\kern0pt}\ \isanewline
\ \ \ \ \ \ \ \ \ \ exp\ {\isacharparenleft}{\kern0pt}{\isadigit{2}}\ {\isacharasterisk}{\kern0pt}\ {\isasymi}\ {\isacharasterisk}{\kern0pt}\ complex{\isacharunderscore}{\kern0pt}of{\isacharunderscore}{\kern0pt}real\ pi\ {\isacharasterisk}{\kern0pt}\ complex{\isacharunderscore}{\kern0pt}of{\isacharunderscore}{\kern0pt}nat\ {\isacharparenleft}{\kern0pt}j\ div\ {\isadigit{2}}{\isacharparenright}{\kern0pt}\ {\isacharslash}{\kern0pt}\ {\isadigit{2}}\ {\isacharcircum}{\kern0pt}\ Suc\ n{\isacharparenright}{\kern0pt}\ {\isasymcdot}\isactrlsub m\ {\isacharbar}{\kern0pt}Deutsch{\isachardot}{\kern0pt}one{\isasymrangle}\ {\isasymOtimes}\isanewline
\ \ \ \ \ \ \ \ \ \ {\isacharbar}{\kern0pt}state{\isacharunderscore}{\kern0pt}basis\ n\ {\isacharparenleft}{\kern0pt}j\ mod\ {\isadigit{2}}\ {\isacharcircum}{\kern0pt}\ Suc\ n\ div\ {\isadigit{2}}{\isacharparenright}{\kern0pt}{\isasymrangle}\ {\isasymOtimes}\ {\isacharbar}{\kern0pt}state{\isacharunderscore}{\kern0pt}basis\ {\isadigit{1}}\ {\isacharparenleft}{\kern0pt}j\ mod\ {\isadigit{2}}{\isacharparenright}{\kern0pt}{\isasymrangle}{\isacharparenright}{\kern0pt}\ {\isacharequal}{\kern0pt}\ \isanewline
\ \ \ \ \ \ \ \ \ \ {\isacharparenleft}{\kern0pt}{\isacharparenleft}{\kern0pt}{\isacharparenleft}{\kern0pt}{\isadigit{1}}\isactrlsub m\ {\isadigit{2}}{\isacharparenright}{\kern0pt}\ {\isasymOtimes}\ SWAP{\isacharunderscore}{\kern0pt}down\ {\isacharparenleft}{\kern0pt}Suc\ n{\isacharparenright}{\kern0pt}{\isacharparenright}{\kern0pt}\ {\isacharasterisk}{\kern0pt}\ {\isacharparenleft}{\kern0pt}control{\isadigit{2}}\ {\isacharparenleft}{\kern0pt}R\ {\isacharparenleft}{\kern0pt}Suc\ {\isacharparenleft}{\kern0pt}Suc\ n{\isacharparenright}{\kern0pt}{\isacharparenright}{\kern0pt}{\isacharparenright}{\kern0pt}\ {\isasymOtimes}\ {\isacharparenleft}{\kern0pt}{\isadigit{1}}\isactrlsub m\ {\isacharparenleft}{\kern0pt}{\isadigit{2}}{\isacharcircum}{\kern0pt}n{\isacharparenright}{\kern0pt}{\isacharparenright}{\kern0pt}{\isacharparenright}{\kern0pt}\ {\isacharasterisk}{\kern0pt}\ \isanewline
\ \ \ \ \ \ \ \ \ \ {\isacharparenleft}{\kern0pt}{\isacharparenleft}{\kern0pt}{\isadigit{1}}\isactrlsub m\ {\isadigit{2}}{\isacharparenright}{\kern0pt}\ {\isasymOtimes}\ SWAP{\isacharunderscore}{\kern0pt}up\ {\isacharparenleft}{\kern0pt}Suc\ n{\isacharparenright}{\kern0pt}{\isacharparenright}{\kern0pt}{\isacharparenright}{\kern0pt}\ {\isacharasterisk}{\kern0pt}\ {\isacharparenleft}{\kern0pt}\ {\isacharbar}{\kern0pt}Deutsch{\isachardot}{\kern0pt}zero{\isasymrangle}\ {\isacharplus}{\kern0pt}\ \isanewline
\ \ \ \ \ \ \ \ \ \ exp\ {\isacharparenleft}{\kern0pt}{\isadigit{2}}\ {\isacharasterisk}{\kern0pt}\ {\isasymi}\ {\isacharasterisk}{\kern0pt}\ complex{\isacharunderscore}{\kern0pt}of{\isacharunderscore}{\kern0pt}real\ pi\ {\isacharasterisk}{\kern0pt}\ complex{\isacharunderscore}{\kern0pt}of{\isacharunderscore}{\kern0pt}nat\ {\isacharparenleft}{\kern0pt}j\ div\ {\isadigit{2}}{\isacharparenright}{\kern0pt}\ {\isacharslash}{\kern0pt}\ {\isadigit{2}}\ {\isacharcircum}{\kern0pt}\ Suc\ n{\isacharparenright}{\kern0pt}\ {\isasymcdot}\isactrlsub m\ {\isacharbar}{\kern0pt}Deutsch{\isachardot}{\kern0pt}one{\isasymrangle}\ {\isasymOtimes}\isanewline
\ \ \ \ \ \ \ \ \ \ {\isacharbar}{\kern0pt}state{\isacharunderscore}{\kern0pt}basis\ n\ {\isacharparenleft}{\kern0pt}j\ mod\ {\isadigit{2}}\ {\isacharcircum}{\kern0pt}\ Suc\ n\ div\ {\isadigit{2}}{\isacharparenright}{\kern0pt}{\isasymrangle}\ {\isasymOtimes}\ {\isacharbar}{\kern0pt}state{\isacharunderscore}{\kern0pt}basis\ {\isadigit{1}}\ {\isacharparenleft}{\kern0pt}j\ mod\ {\isadigit{2}}{\isacharparenright}{\kern0pt}{\isasymrangle}{\isacharparenright}{\kern0pt}{\isachardoublequoteclose}\isanewline
\ \ \ \ \ \ \isacommand{using}\isamarkupfalse%
\ control{\isachardot}{\kern0pt}simps\ Suc\ \isacommand{by}\isamarkupfalse%
\ presburger\isanewline
\ \ \ \ \isacommand{also}\isamarkupfalse%
\ \isacommand{have}\isamarkupfalse%
\ {\isachardoublequoteopen}{\isasymdots}\ {\isacharequal}{\kern0pt}\ {\isacharparenleft}{\kern0pt}{\isacharparenleft}{\kern0pt}{\isacharparenleft}{\kern0pt}{\isadigit{1}}\isactrlsub m\ {\isadigit{2}}{\isacharparenright}{\kern0pt}\ {\isasymOtimes}\ SWAP{\isacharunderscore}{\kern0pt}down\ {\isacharparenleft}{\kern0pt}Suc\ n{\isacharparenright}{\kern0pt}{\isacharparenright}{\kern0pt}\ {\isacharasterisk}{\kern0pt}\ {\isacharparenleft}{\kern0pt}control{\isadigit{2}}\ {\isacharparenleft}{\kern0pt}R\ {\isacharparenleft}{\kern0pt}Suc\ {\isacharparenleft}{\kern0pt}Suc\ n{\isacharparenright}{\kern0pt}{\isacharparenright}{\kern0pt}{\isacharparenright}{\kern0pt}\ {\isasymOtimes}\ {\isacharparenleft}{\kern0pt}{\isadigit{1}}\isactrlsub m\ {\isacharparenleft}{\kern0pt}{\isadigit{2}}{\isacharcircum}{\kern0pt}n{\isacharparenright}{\kern0pt}{\isacharparenright}{\kern0pt}{\isacharparenright}{\kern0pt}{\isacharparenright}{\kern0pt}\ {\isacharasterisk}{\kern0pt}\ \isanewline
\ \ \ \ \ \ \ \ \ \ {\isacharparenleft}{\kern0pt}{\isacharparenleft}{\kern0pt}{\isacharparenleft}{\kern0pt}{\isadigit{1}}\isactrlsub m\ {\isadigit{2}}{\isacharparenright}{\kern0pt}\ {\isasymOtimes}\ SWAP{\isacharunderscore}{\kern0pt}up\ {\isacharparenleft}{\kern0pt}Suc\ n{\isacharparenright}{\kern0pt}{\isacharparenright}{\kern0pt}\ {\isacharasterisk}{\kern0pt}\ {\isacharparenleft}{\kern0pt}\ {\isacharbar}{\kern0pt}Deutsch{\isachardot}{\kern0pt}zero{\isasymrangle}\ {\isacharplus}{\kern0pt}\ \isanewline
\ \ \ \ \ \ \ \ \ \ exp\ {\isacharparenleft}{\kern0pt}{\isadigit{2}}\ {\isacharasterisk}{\kern0pt}\ {\isasymi}\ {\isacharasterisk}{\kern0pt}\ complex{\isacharunderscore}{\kern0pt}of{\isacharunderscore}{\kern0pt}real\ pi\ {\isacharasterisk}{\kern0pt}\ complex{\isacharunderscore}{\kern0pt}of{\isacharunderscore}{\kern0pt}nat\ {\isacharparenleft}{\kern0pt}j\ div\ {\isadigit{2}}{\isacharparenright}{\kern0pt}\ {\isacharslash}{\kern0pt}\ {\isadigit{2}}\ {\isacharcircum}{\kern0pt}\ Suc\ n{\isacharparenright}{\kern0pt}\ {\isasymcdot}\isactrlsub m\ {\isacharbar}{\kern0pt}Deutsch{\isachardot}{\kern0pt}one{\isasymrangle}\ {\isasymOtimes}\isanewline
\ \ \ \ \ \ \ \ \ \ {\isacharbar}{\kern0pt}state{\isacharunderscore}{\kern0pt}basis\ n\ {\isacharparenleft}{\kern0pt}j\ mod\ {\isadigit{2}}\ {\isacharcircum}{\kern0pt}\ Suc\ n\ div\ {\isadigit{2}}{\isacharparenright}{\kern0pt}{\isasymrangle}\ {\isasymOtimes}\ {\isacharbar}{\kern0pt}state{\isacharunderscore}{\kern0pt}basis\ {\isadigit{1}}\ {\isacharparenleft}{\kern0pt}j\ mod\ {\isadigit{2}}{\isacharparenright}{\kern0pt}{\isasymrangle}{\isacharparenright}{\kern0pt}{\isacharparenright}{\kern0pt}{\isachardoublequoteclose}\isanewline
\ \ \ \ \isacommand{proof}\isamarkupfalse%
\ {\isacharparenleft}{\kern0pt}rule\ assoc{\isacharunderscore}{\kern0pt}mult{\isacharunderscore}{\kern0pt}mat{\isacharparenright}{\kern0pt}\isanewline
\ \ \ \ \ \ \isacommand{show}\isamarkupfalse%
\ {\isachardoublequoteopen}{\isacharparenleft}{\kern0pt}{\isadigit{1}}\isactrlsub m\ {\isadigit{2}}\ {\isasymOtimes}\ SWAP{\isacharunderscore}{\kern0pt}down\ {\isacharparenleft}{\kern0pt}Suc\ n{\isacharparenright}{\kern0pt}{\isacharparenright}{\kern0pt}\ {\isacharasterisk}{\kern0pt}\ {\isacharparenleft}{\kern0pt}control{\isadigit{2}}\ {\isacharparenleft}{\kern0pt}R\ {\isacharparenleft}{\kern0pt}Suc\ {\isacharparenleft}{\kern0pt}Suc\ n{\isacharparenright}{\kern0pt}{\isacharparenright}{\kern0pt}{\isacharparenright}{\kern0pt}\ {\isasymOtimes}\ {\isadigit{1}}\isactrlsub m\ {\isacharparenleft}{\kern0pt}{\isadigit{2}}\ {\isacharcircum}{\kern0pt}\ n{\isacharparenright}{\kern0pt}{\isacharparenright}{\kern0pt}\isanewline
\ \ \ \ \ \ \ \ \ \ \ \ {\isasymin}\ carrier{\isacharunderscore}{\kern0pt}mat\ {\isacharparenleft}{\kern0pt}{\isadigit{2}}{\isacharcircum}{\kern0pt}Suc\ {\isacharparenleft}{\kern0pt}Suc\ n{\isacharparenright}{\kern0pt}{\isacharparenright}{\kern0pt}\ {\isacharparenleft}{\kern0pt}{\isadigit{2}}{\isacharcircum}{\kern0pt}Suc\ {\isacharparenleft}{\kern0pt}Suc\ n{\isacharparenright}{\kern0pt}{\isacharparenright}{\kern0pt}{\isachardoublequoteclose}\isanewline
\ \ \ \ \ \ \ \ \isacommand{using}\isamarkupfalse%
\ SWAP{\isacharunderscore}{\kern0pt}down{\isacharunderscore}{\kern0pt}carrier{\isacharunderscore}{\kern0pt}mat\ SWAP{\isacharunderscore}{\kern0pt}up{\isacharunderscore}{\kern0pt}carrier{\isacharunderscore}{\kern0pt}mat\ control{\isadigit{2}}{\isacharunderscore}{\kern0pt}carrier{\isacharunderscore}{\kern0pt}mat\ \isanewline
\ \ \ \ \ \ \ \ \isacommand{by}\isamarkupfalse%
\ {\isacharparenleft}{\kern0pt}smt\ {\isacharparenleft}{\kern0pt}verit{\isacharparenright}{\kern0pt}\ Suc\ carrier{\isacharunderscore}{\kern0pt}matD{\isacharparenleft}{\kern0pt}{\isadigit{1}}{\isacharparenright}{\kern0pt}\ carrier{\isacharunderscore}{\kern0pt}matD{\isacharparenleft}{\kern0pt}{\isadigit{2}}{\isacharparenright}{\kern0pt}\ carrier{\isacharunderscore}{\kern0pt}matI\ control{\isachardot}{\kern0pt}simps{\isacharparenleft}{\kern0pt}{\isadigit{4}}{\isacharparenright}{\kern0pt}\ \isanewline
\ \ \ \ \ \ \ \ \ \ \ \ control{\isacharunderscore}{\kern0pt}carrier{\isacharunderscore}{\kern0pt}mat\ dim{\isacharunderscore}{\kern0pt}col{\isacharunderscore}{\kern0pt}tensor{\isacharunderscore}{\kern0pt}mat\ index{\isacharunderscore}{\kern0pt}mult{\isacharunderscore}{\kern0pt}mat{\isacharparenleft}{\kern0pt}{\isadigit{2}}{\isacharparenright}{\kern0pt}\ index{\isacharunderscore}{\kern0pt}mult{\isacharunderscore}{\kern0pt}mat{\isacharparenleft}{\kern0pt}{\isadigit{3}}{\isacharparenright}{\kern0pt}\ \isanewline
\ \ \ \ \ \ \ \ \ \ \ \ index{\isacharunderscore}{\kern0pt}one{\isacharunderscore}{\kern0pt}mat{\isacharparenleft}{\kern0pt}{\isadigit{3}}{\isacharparenright}{\kern0pt}\ mult{\isacharunderscore}{\kern0pt}numeral{\isacharunderscore}{\kern0pt}left{\isacharunderscore}{\kern0pt}semiring{\isacharunderscore}{\kern0pt}numeral\ num{\isacharunderscore}{\kern0pt}double\ power{\isacharunderscore}{\kern0pt}Suc{\isacharparenright}{\kern0pt}\isanewline
\ \ \ \ \ \ \isacommand{show}\isamarkupfalse%
\ {\isachardoublequoteopen}{\isadigit{1}}\isactrlsub m\ {\isadigit{2}}\ {\isasymOtimes}\ SWAP{\isacharunderscore}{\kern0pt}up\ {\isacharparenleft}{\kern0pt}Suc\ n{\isacharparenright}{\kern0pt}\ {\isasymin}\ carrier{\isacharunderscore}{\kern0pt}mat\ {\isacharparenleft}{\kern0pt}{\isadigit{2}}\ {\isacharcircum}{\kern0pt}\ Suc\ {\isacharparenleft}{\kern0pt}Suc\ n{\isacharparenright}{\kern0pt}{\isacharparenright}{\kern0pt}\ {\isacharparenleft}{\kern0pt}{\isadigit{2}}\ {\isacharcircum}{\kern0pt}\ Suc\ {\isacharparenleft}{\kern0pt}Suc\ n{\isacharparenright}{\kern0pt}{\isacharparenright}{\kern0pt}{\isachardoublequoteclose}\isanewline
\ \ \ \ \ \ \ \ \isacommand{using}\isamarkupfalse%
\ SWAP{\isacharunderscore}{\kern0pt}up{\isacharunderscore}{\kern0pt}carrier{\isacharunderscore}{\kern0pt}mat\isanewline
\ \ \ \ \ \ \ \ \isacommand{by}\isamarkupfalse%
\ {\isacharparenleft}{\kern0pt}metis\ One{\isacharunderscore}{\kern0pt}nat{\isacharunderscore}{\kern0pt}def\ SWAP{\isacharunderscore}{\kern0pt}up{\isachardot}{\kern0pt}simps{\isacharparenleft}{\kern0pt}{\isadigit{2}}{\isacharparenright}{\kern0pt}\ power{\isacharunderscore}{\kern0pt}Suc\ power{\isacharunderscore}{\kern0pt}one{\isacharunderscore}{\kern0pt}right\ tensor{\isacharunderscore}{\kern0pt}carrier{\isacharunderscore}{\kern0pt}mat{\isacharparenright}{\kern0pt}\isanewline
\ \ \ \ \ \ \isacommand{show}\isamarkupfalse%
\ {\isachardoublequoteopen}{\isacharbar}{\kern0pt}Deutsch{\isachardot}{\kern0pt}zero{\isasymrangle}\ {\isacharplus}{\kern0pt}\ exp\ {\isacharparenleft}{\kern0pt}{\isadigit{2}}\ {\isacharasterisk}{\kern0pt}\ {\isasymi}\ {\isacharasterisk}{\kern0pt}\ complex{\isacharunderscore}{\kern0pt}of{\isacharunderscore}{\kern0pt}real\ pi\ {\isacharasterisk}{\kern0pt}\ \ complex{\isacharunderscore}{\kern0pt}of{\isacharunderscore}{\kern0pt}nat\ {\isacharparenleft}{\kern0pt}j\ div\ {\isadigit{2}}{\isacharparenright}{\kern0pt}\ {\isacharslash}{\kern0pt}\isanewline
\ \ \ \ \ \ \ \ \ \ \ \ {\isadigit{2}}\ {\isacharcircum}{\kern0pt}\ Suc\ n{\isacharparenright}{\kern0pt}\ {\isasymcdot}\isactrlsub m\ {\isacharbar}{\kern0pt}Deutsch{\isachardot}{\kern0pt}one{\isasymrangle}\ {\isasymOtimes}\ {\isacharbar}{\kern0pt}state{\isacharunderscore}{\kern0pt}basis\ n\ {\isacharparenleft}{\kern0pt}j\ mod\ {\isadigit{2}}\ {\isacharcircum}{\kern0pt}\ Suc\ n\ div\ {\isadigit{2}}{\isacharparenright}{\kern0pt}{\isasymrangle}\ {\isasymOtimes}\isanewline
\ \ \ \ \ \ \ \ \ \ \ \ {\isacharbar}{\kern0pt}state{\isacharunderscore}{\kern0pt}basis\ {\isadigit{1}}\ {\isacharparenleft}{\kern0pt}j\ mod\ {\isadigit{2}}{\isacharparenright}{\kern0pt}{\isasymrangle}\ {\isasymin}\ carrier{\isacharunderscore}{\kern0pt}mat\ {\isacharparenleft}{\kern0pt}{\isadigit{2}}\ {\isacharcircum}{\kern0pt}\ Suc\ {\isacharparenleft}{\kern0pt}Suc\ n{\isacharparenright}{\kern0pt}{\isacharparenright}{\kern0pt}\ {\isadigit{1}}{\isachardoublequoteclose}\isanewline
\ \ \ \ \ \ \ \ \isacommand{using}\isamarkupfalse%
\ ket{\isacharunderscore}{\kern0pt}vec{\isacharunderscore}{\kern0pt}def\ state{\isacharunderscore}{\kern0pt}basis{\isacharunderscore}{\kern0pt}carrier{\isacharunderscore}{\kern0pt}mat\isanewline
\ \ \ \ \ \ \ \ \isacommand{by}\isamarkupfalse%
\ {\isacharparenleft}{\kern0pt}simp\ add{\isacharcolon}{\kern0pt}\ carrier{\isacharunderscore}{\kern0pt}matI\ index{\isacharunderscore}{\kern0pt}unit{\isacharunderscore}{\kern0pt}vec{\isacharparenleft}{\kern0pt}{\isadigit{3}}{\isacharparenright}{\kern0pt}\ state{\isacharunderscore}{\kern0pt}basis{\isacharunderscore}{\kern0pt}def{\isacharparenright}{\kern0pt}\isanewline
\ \ \ \ \isacommand{qed}\isamarkupfalse%
\isanewline
\ \ \ \ \isacommand{also}\isamarkupfalse%
\ \isacommand{have}\isamarkupfalse%
\ {\isachardoublequoteopen}{\isasymdots}\ {\isacharequal}{\kern0pt}\ {\isacharparenleft}{\kern0pt}{\isacharparenleft}{\kern0pt}{\isacharparenleft}{\kern0pt}{\isadigit{1}}\isactrlsub m\ {\isadigit{2}}{\isacharparenright}{\kern0pt}\ {\isasymOtimes}\ SWAP{\isacharunderscore}{\kern0pt}down\ {\isacharparenleft}{\kern0pt}Suc\ n{\isacharparenright}{\kern0pt}{\isacharparenright}{\kern0pt}\ {\isacharasterisk}{\kern0pt}\ {\isacharparenleft}{\kern0pt}control{\isadigit{2}}\ {\isacharparenleft}{\kern0pt}R\ {\isacharparenleft}{\kern0pt}Suc\ {\isacharparenleft}{\kern0pt}Suc\ n{\isacharparenright}{\kern0pt}{\isacharparenright}{\kern0pt}{\isacharparenright}{\kern0pt}\ {\isasymOtimes}\ {\isacharparenleft}{\kern0pt}{\isadigit{1}}\isactrlsub m\ {\isacharparenleft}{\kern0pt}{\isadigit{2}}{\isacharcircum}{\kern0pt}n{\isacharparenright}{\kern0pt}{\isacharparenright}{\kern0pt}{\isacharparenright}{\kern0pt}{\isacharparenright}{\kern0pt}\ {\isacharasterisk}{\kern0pt}\ \isanewline
\ \ \ \ \ \ \ \ \ \ {\isacharparenleft}{\kern0pt}{\isacharparenleft}{\kern0pt}{\isacharparenleft}{\kern0pt}{\isadigit{1}}\isactrlsub m\ {\isadigit{2}}{\isacharparenright}{\kern0pt}\ {\isasymOtimes}\ SWAP{\isacharunderscore}{\kern0pt}up\ {\isacharparenleft}{\kern0pt}Suc\ n{\isacharparenright}{\kern0pt}{\isacharparenright}{\kern0pt}\ {\isacharasterisk}{\kern0pt}\ {\isacharparenleft}{\kern0pt}\ {\isacharbar}{\kern0pt}Deutsch{\isachardot}{\kern0pt}zero{\isasymrangle}\ {\isacharplus}{\kern0pt}\ \isanewline
\ \ \ \ \ \ \ \ \ \ exp\ {\isacharparenleft}{\kern0pt}{\isadigit{2}}\ {\isacharasterisk}{\kern0pt}\ {\isasymi}\ {\isacharasterisk}{\kern0pt}\ complex{\isacharunderscore}{\kern0pt}of{\isacharunderscore}{\kern0pt}real\ pi\ {\isacharasterisk}{\kern0pt}\ complex{\isacharunderscore}{\kern0pt}of{\isacharunderscore}{\kern0pt}nat\ {\isacharparenleft}{\kern0pt}j\ div\ {\isadigit{2}}{\isacharparenright}{\kern0pt}\ {\isacharslash}{\kern0pt}\ {\isadigit{2}}\ {\isacharcircum}{\kern0pt}\ Suc\ n{\isacharparenright}{\kern0pt}\ {\isasymcdot}\isactrlsub m\ {\isacharbar}{\kern0pt}Deutsch{\isachardot}{\kern0pt}one{\isasymrangle}\ {\isasymOtimes}\isanewline
\ \ \ \ \ \ \ \ \ \ {\isacharparenleft}{\kern0pt}\ {\isacharbar}{\kern0pt}state{\isacharunderscore}{\kern0pt}basis\ n\ {\isacharparenleft}{\kern0pt}j\ mod\ {\isadigit{2}}\ {\isacharcircum}{\kern0pt}\ Suc\ n\ div\ {\isadigit{2}}{\isacharparenright}{\kern0pt}{\isasymrangle}\ {\isasymOtimes}\ {\isacharbar}{\kern0pt}state{\isacharunderscore}{\kern0pt}basis\ {\isadigit{1}}\ {\isacharparenleft}{\kern0pt}j\ mod\ {\isadigit{2}}{\isacharparenright}{\kern0pt}{\isasymrangle}{\isacharparenright}{\kern0pt}{\isacharparenright}{\kern0pt}{\isacharparenright}{\kern0pt}{\isachardoublequoteclose}\isanewline
\ \ \ \ \ \ \isacommand{using}\isamarkupfalse%
\ tensor{\isacharunderscore}{\kern0pt}mat{\isacharunderscore}{\kern0pt}is{\isacharunderscore}{\kern0pt}assoc\ \isacommand{by}\isamarkupfalse%
\ presburger\isanewline
\ \ \ \ \isacommand{also}\isamarkupfalse%
\ \isacommand{have}\isamarkupfalse%
\ {\isachardoublequoteopen}{\isasymdots}\ {\isacharequal}{\kern0pt}\ {\isacharparenleft}{\kern0pt}{\isacharparenleft}{\kern0pt}{\isacharparenleft}{\kern0pt}{\isadigit{1}}\isactrlsub m\ {\isadigit{2}}{\isacharparenright}{\kern0pt}\ {\isasymOtimes}\ SWAP{\isacharunderscore}{\kern0pt}down\ {\isacharparenleft}{\kern0pt}Suc\ n{\isacharparenright}{\kern0pt}{\isacharparenright}{\kern0pt}\ {\isacharasterisk}{\kern0pt}\ {\isacharparenleft}{\kern0pt}control{\isadigit{2}}\ {\isacharparenleft}{\kern0pt}R\ {\isacharparenleft}{\kern0pt}Suc\ {\isacharparenleft}{\kern0pt}Suc\ n{\isacharparenright}{\kern0pt}{\isacharparenright}{\kern0pt}{\isacharparenright}{\kern0pt}\ {\isasymOtimes}\ {\isacharparenleft}{\kern0pt}{\isadigit{1}}\isactrlsub m\ {\isacharparenleft}{\kern0pt}{\isadigit{2}}{\isacharcircum}{\kern0pt}n{\isacharparenright}{\kern0pt}{\isacharparenright}{\kern0pt}{\isacharparenright}{\kern0pt}{\isacharparenright}{\kern0pt}\ {\isacharasterisk}{\kern0pt}\isanewline
\ \ \ \ \ \ \ \ \ \ {\isacharparenleft}{\kern0pt}{\isacharparenleft}{\kern0pt}{\isacharparenleft}{\kern0pt}{\isadigit{1}}\isactrlsub m\ {\isadigit{2}}{\isacharparenright}{\kern0pt}\ {\isacharasterisk}{\kern0pt}\ {\isacharparenleft}{\kern0pt}\ {\isacharbar}{\kern0pt}Deutsch{\isachardot}{\kern0pt}zero{\isasymrangle}\ {\isacharplus}{\kern0pt}\ exp\ {\isacharparenleft}{\kern0pt}{\isadigit{2}}\ {\isacharasterisk}{\kern0pt}\ {\isasymi}\ {\isacharasterisk}{\kern0pt}\ pi\ {\isacharasterisk}{\kern0pt}\ complex{\isacharunderscore}{\kern0pt}of{\isacharunderscore}{\kern0pt}nat\ {\isacharparenleft}{\kern0pt}j\ div\ {\isadigit{2}}{\isacharparenright}{\kern0pt}\ {\isacharslash}{\kern0pt}\ {\isadigit{2}}\ {\isacharcircum}{\kern0pt}\ Suc\ n{\isacharparenright}{\kern0pt}\ {\isasymcdot}\isactrlsub m\ \isanewline
\ \ \ \ \ \ \ \ \ \ \ \ {\isacharbar}{\kern0pt}Deutsch{\isachardot}{\kern0pt}one{\isasymrangle}{\isacharparenright}{\kern0pt}{\isacharparenright}{\kern0pt}\ {\isasymOtimes}\ {\isacharparenleft}{\kern0pt}{\isacharparenleft}{\kern0pt}SWAP{\isacharunderscore}{\kern0pt}up\ {\isacharparenleft}{\kern0pt}Suc\ n{\isacharparenright}{\kern0pt}{\isacharparenright}{\kern0pt}\ {\isacharasterisk}{\kern0pt}\ {\isacharparenleft}{\kern0pt}\ {\isacharbar}{\kern0pt}state{\isacharunderscore}{\kern0pt}basis\ n\ {\isacharparenleft}{\kern0pt}j\ mod\ {\isadigit{2}}\ {\isacharcircum}{\kern0pt}\ Suc\ n\ div\ {\isadigit{2}}{\isacharparenright}{\kern0pt}{\isasymrangle}\ {\isasymOtimes}\ \isanewline
\ \ \ \ \ \ \ \ \ \ \ \ {\isacharbar}{\kern0pt}state{\isacharunderscore}{\kern0pt}basis\ {\isadigit{1}}\ {\isacharparenleft}{\kern0pt}j\ mod\ {\isadigit{2}}{\isacharparenright}{\kern0pt}{\isasymrangle}{\isacharparenright}{\kern0pt}{\isacharparenright}{\kern0pt}{\isacharparenright}{\kern0pt}{\isachardoublequoteclose}\isanewline
\ \ \ \ \ \ \isacommand{using}\isamarkupfalse%
\ mult{\isacharunderscore}{\kern0pt}distr{\isacharunderscore}{\kern0pt}tensor\isanewline
\ \ \ \ \ \ \isacommand{by}\isamarkupfalse%
\ {\isacharparenleft}{\kern0pt}smt\ {\isacharparenleft}{\kern0pt}verit{\isacharcomma}{\kern0pt}\ del{\isacharunderscore}{\kern0pt}insts{\isacharparenright}{\kern0pt}\ SWAP{\isacharunderscore}{\kern0pt}up{\isacharunderscore}{\kern0pt}carrier{\isacharunderscore}{\kern0pt}mat\ carrier{\isacharunderscore}{\kern0pt}matD{\isacharparenleft}{\kern0pt}{\isadigit{2}}{\isacharparenright}{\kern0pt}\ dim{\isacharunderscore}{\kern0pt}col{\isacharunderscore}{\kern0pt}mat{\isacharparenleft}{\kern0pt}{\isadigit{1}}{\isacharparenright}{\kern0pt}\ \isanewline
\ \ \ \ \ \ \ \ \ \ dim{\isacharunderscore}{\kern0pt}col{\isacharunderscore}{\kern0pt}tensor{\isacharunderscore}{\kern0pt}mat\ dim{\isacharunderscore}{\kern0pt}row{\isacharunderscore}{\kern0pt}mat{\isacharparenleft}{\kern0pt}{\isadigit{1}}{\isacharparenright}{\kern0pt}\ dim{\isacharunderscore}{\kern0pt}row{\isacharunderscore}{\kern0pt}tensor{\isacharunderscore}{\kern0pt}mat\ index{\isacharunderscore}{\kern0pt}add{\isacharunderscore}{\kern0pt}mat{\isacharparenleft}{\kern0pt}{\isadigit{2}}{\isacharparenright}{\kern0pt}\ index{\isacharunderscore}{\kern0pt}add{\isacharunderscore}{\kern0pt}mat{\isacharparenleft}{\kern0pt}{\isadigit{3}}{\isacharparenright}{\kern0pt}\ \isanewline
\ \ \ \ \ \ \ \ \ \ index{\isacharunderscore}{\kern0pt}one{\isacharunderscore}{\kern0pt}mat{\isacharparenleft}{\kern0pt}{\isadigit{3}}{\isacharparenright}{\kern0pt}\ index{\isacharunderscore}{\kern0pt}smult{\isacharunderscore}{\kern0pt}mat{\isacharparenleft}{\kern0pt}{\isadigit{2}}{\isacharparenright}{\kern0pt}\ index{\isacharunderscore}{\kern0pt}smult{\isacharunderscore}{\kern0pt}mat{\isacharparenleft}{\kern0pt}{\isadigit{3}}{\isacharparenright}{\kern0pt}\ index{\isacharunderscore}{\kern0pt}unit{\isacharunderscore}{\kern0pt}vec{\isacharparenleft}{\kern0pt}{\isadigit{3}}{\isacharparenright}{\kern0pt}\ ket{\isacharunderscore}{\kern0pt}vec{\isacharunderscore}{\kern0pt}def\ \isanewline
\ \ \ \ \ \ \ \ \ \ one{\isacharunderscore}{\kern0pt}power{\isadigit{2}}\ pos{\isadigit{2}}\ power{\isacharunderscore}{\kern0pt}Suc{\isadigit{2}}\ power{\isacharunderscore}{\kern0pt}one{\isacharunderscore}{\kern0pt}right\ state{\isacharunderscore}{\kern0pt}basis{\isacharunderscore}{\kern0pt}def\ \isanewline
\ \ \ \ \ \ \ \ \ \ zero{\isacharunderscore}{\kern0pt}less{\isacharunderscore}{\kern0pt}one{\isacharunderscore}{\kern0pt}class{\isachardot}{\kern0pt}zero{\isacharunderscore}{\kern0pt}less{\isacharunderscore}{\kern0pt}one\ zero{\isacharunderscore}{\kern0pt}less{\isacharunderscore}{\kern0pt}power{\isacharparenright}{\kern0pt}\ \ \ \isanewline
\ \ \ \ \isacommand{also}\isamarkupfalse%
\ \isacommand{have}\isamarkupfalse%
\ {\isachardoublequoteopen}{\isasymdots}\ {\isacharequal}{\kern0pt}\ {\isacharparenleft}{\kern0pt}{\isacharparenleft}{\kern0pt}{\isacharparenleft}{\kern0pt}{\isadigit{1}}\isactrlsub m\ {\isadigit{2}}{\isacharparenright}{\kern0pt}\ {\isasymOtimes}\ SWAP{\isacharunderscore}{\kern0pt}down\ {\isacharparenleft}{\kern0pt}Suc\ n{\isacharparenright}{\kern0pt}{\isacharparenright}{\kern0pt}\ {\isacharasterisk}{\kern0pt}\ {\isacharparenleft}{\kern0pt}control{\isadigit{2}}\ {\isacharparenleft}{\kern0pt}R\ {\isacharparenleft}{\kern0pt}Suc\ {\isacharparenleft}{\kern0pt}Suc\ n{\isacharparenright}{\kern0pt}{\isacharparenright}{\kern0pt}{\isacharparenright}{\kern0pt}\ {\isasymOtimes}\ {\isacharparenleft}{\kern0pt}{\isadigit{1}}\isactrlsub m\ {\isacharparenleft}{\kern0pt}{\isadigit{2}}{\isacharcircum}{\kern0pt}n{\isacharparenright}{\kern0pt}{\isacharparenright}{\kern0pt}{\isacharparenright}{\kern0pt}{\isacharparenright}{\kern0pt}\ {\isacharasterisk}{\kern0pt}\isanewline
\ \ \ \ \ \ \ \ \ \ {\isacharparenleft}{\kern0pt}{\isacharparenleft}{\kern0pt}\ {\isacharbar}{\kern0pt}Deutsch{\isachardot}{\kern0pt}zero{\isasymrangle}\ {\isacharplus}{\kern0pt}\ exp\ {\isacharparenleft}{\kern0pt}{\isadigit{2}}\ {\isacharasterisk}{\kern0pt}\ {\isasymi}\ {\isacharasterisk}{\kern0pt}\ pi\ {\isacharasterisk}{\kern0pt}\ complex{\isacharunderscore}{\kern0pt}of{\isacharunderscore}{\kern0pt}nat\ {\isacharparenleft}{\kern0pt}j\ div\ {\isadigit{2}}{\isacharparenright}{\kern0pt}\ {\isacharslash}{\kern0pt}\ {\isadigit{2}}\ {\isacharcircum}{\kern0pt}\ Suc\ n{\isacharparenright}{\kern0pt}\ {\isasymcdot}\isactrlsub m\ \isanewline
\ \ \ \ \ \ \ \ \ \ \ \ {\isacharbar}{\kern0pt}one{\isasymrangle}{\isacharparenright}{\kern0pt}\ {\isasymOtimes}\ {\isacharparenleft}{\kern0pt}\ {\isacharbar}{\kern0pt}state{\isacharunderscore}{\kern0pt}basis\ {\isadigit{1}}\ {\isacharparenleft}{\kern0pt}j\ mod\ {\isadigit{2}}{\isacharparenright}{\kern0pt}{\isasymrangle}\ {\isasymOtimes}\ {\isacharbar}{\kern0pt}state{\isacharunderscore}{\kern0pt}basis\ n\ {\isacharparenleft}{\kern0pt}j\ mod\ {\isadigit{2}}\ {\isacharcircum}{\kern0pt}\ Suc\ n\ div\ {\isadigit{2}}{\isacharparenright}{\kern0pt}{\isasymrangle}{\isacharparenright}{\kern0pt}{\isacharparenright}{\kern0pt}{\isachardoublequoteclose}\isanewline
\ \ \ \ \ \ \isacommand{using}\isamarkupfalse%
\ SWAP{\isacharunderscore}{\kern0pt}up{\isacharunderscore}{\kern0pt}action\ jeq\ \isanewline
\ \ \ \ \ \ \isacommand{by}\isamarkupfalse%
\ {\isacharparenleft}{\kern0pt}smt\ {\isacharparenleft}{\kern0pt}verit{\isacharcomma}{\kern0pt}\ best{\isacharparenright}{\kern0pt}\ Suc\ index{\isacharunderscore}{\kern0pt}add{\isacharunderscore}{\kern0pt}mat{\isacharparenleft}{\kern0pt}{\isadigit{2}}{\isacharparenright}{\kern0pt}\ index{\isacharunderscore}{\kern0pt}smult{\isacharunderscore}{\kern0pt}mat{\isacharparenleft}{\kern0pt}{\isadigit{2}}{\isacharparenright}{\kern0pt}\ jm{\isadigit{2}}sn{\isacharunderscore}{\kern0pt}def\ ket{\isacharunderscore}{\kern0pt}one{\isacharunderscore}{\kern0pt}is{\isacharunderscore}{\kern0pt}state\ \isanewline
\ \ \ \ \ \ \ \ \ \ left{\isacharunderscore}{\kern0pt}mult{\isacharunderscore}{\kern0pt}one{\isacharunderscore}{\kern0pt}mat{\isacharprime}{\kern0pt}\ mod{\isacharunderscore}{\kern0pt}less{\isacharunderscore}{\kern0pt}divisor\ pos{\isadigit{2}}\ power{\isacharunderscore}{\kern0pt}one{\isacharunderscore}{\kern0pt}right\ state{\isachardot}{\kern0pt}dim{\isacharunderscore}{\kern0pt}row\ zero{\isacharunderscore}{\kern0pt}less{\isacharunderscore}{\kern0pt}power{\isacharparenright}{\kern0pt}\isanewline
\ \ \ \ \isacommand{also}\isamarkupfalse%
\ \isacommand{have}\isamarkupfalse%
\ {\isachardoublequoteopen}{\isasymdots}\ {\isacharequal}{\kern0pt}\ {\isacharparenleft}{\kern0pt}{\isacharparenleft}{\kern0pt}{\isacharparenleft}{\kern0pt}{\isadigit{1}}\isactrlsub m\ {\isadigit{2}}{\isacharparenright}{\kern0pt}\ {\isasymOtimes}\ SWAP{\isacharunderscore}{\kern0pt}down\ {\isacharparenleft}{\kern0pt}Suc\ n{\isacharparenright}{\kern0pt}{\isacharparenright}{\kern0pt}\ {\isacharasterisk}{\kern0pt}\ {\isacharparenleft}{\kern0pt}control{\isadigit{2}}\ {\isacharparenleft}{\kern0pt}R\ {\isacharparenleft}{\kern0pt}Suc\ {\isacharparenleft}{\kern0pt}Suc\ n{\isacharparenright}{\kern0pt}{\isacharparenright}{\kern0pt}{\isacharparenright}{\kern0pt}\ {\isasymOtimes}\ {\isacharparenleft}{\kern0pt}{\isadigit{1}}\isactrlsub m\ {\isacharparenleft}{\kern0pt}{\isadigit{2}}{\isacharcircum}{\kern0pt}n{\isacharparenright}{\kern0pt}{\isacharparenright}{\kern0pt}{\isacharparenright}{\kern0pt}{\isacharparenright}{\kern0pt}\ {\isacharasterisk}{\kern0pt}\isanewline
\ \ \ \ \ \ \ \ \ \ {\isacharparenleft}{\kern0pt}{\isacharparenleft}{\kern0pt}{\isacharparenleft}{\kern0pt}\ {\isacharbar}{\kern0pt}Deutsch{\isachardot}{\kern0pt}zero{\isasymrangle}\ {\isacharplus}{\kern0pt}\ exp\ {\isacharparenleft}{\kern0pt}{\isadigit{2}}\ {\isacharasterisk}{\kern0pt}\ {\isasymi}\ {\isacharasterisk}{\kern0pt}\ pi\ {\isacharasterisk}{\kern0pt}\ complex{\isacharunderscore}{\kern0pt}of{\isacharunderscore}{\kern0pt}nat\ {\isacharparenleft}{\kern0pt}j\ div\ {\isadigit{2}}{\isacharparenright}{\kern0pt}\ {\isacharslash}{\kern0pt}\ {\isadigit{2}}\ {\isacharcircum}{\kern0pt}\ Suc\ n{\isacharparenright}{\kern0pt}\ {\isasymcdot}\isactrlsub m\ \isanewline
\ \ \ \ \ \ \ \ \ \ \ \ {\isacharbar}{\kern0pt}one{\isasymrangle}{\isacharparenright}{\kern0pt}\ {\isasymOtimes}\ \ {\isacharbar}{\kern0pt}state{\isacharunderscore}{\kern0pt}basis\ {\isadigit{1}}\ {\isacharparenleft}{\kern0pt}j\ mod\ {\isadigit{2}}{\isacharparenright}{\kern0pt}{\isasymrangle}{\isacharparenright}{\kern0pt}\ {\isasymOtimes}\ {\isacharbar}{\kern0pt}state{\isacharunderscore}{\kern0pt}basis\ n\ {\isacharparenleft}{\kern0pt}j\ mod\ {\isadigit{2}}\ {\isacharcircum}{\kern0pt}\ Suc\ n\ div\ {\isadigit{2}}{\isacharparenright}{\kern0pt}{\isasymrangle}{\isacharparenright}{\kern0pt}{\isachardoublequoteclose}\isanewline
\ \ \ \ \ \ \isacommand{using}\isamarkupfalse%
\ tensor{\isacharunderscore}{\kern0pt}mat{\isacharunderscore}{\kern0pt}is{\isacharunderscore}{\kern0pt}assoc\ \isacommand{by}\isamarkupfalse%
\ presburger\isanewline
\ \ \ \ \isacommand{also}\isamarkupfalse%
\ \isacommand{have}\isamarkupfalse%
\ {\isachardoublequoteopen}{\isasymdots}\ {\isacharequal}{\kern0pt}\ {\isacharparenleft}{\kern0pt}{\isacharparenleft}{\kern0pt}{\isadigit{1}}\isactrlsub m\ {\isadigit{2}}{\isacharparenright}{\kern0pt}\ {\isasymOtimes}\ SWAP{\isacharunderscore}{\kern0pt}down\ {\isacharparenleft}{\kern0pt}Suc\ n{\isacharparenright}{\kern0pt}{\isacharparenright}{\kern0pt}\ {\isacharasterisk}{\kern0pt}\ {\isacharparenleft}{\kern0pt}{\isacharparenleft}{\kern0pt}{\isacharparenleft}{\kern0pt}control{\isadigit{2}}\ {\isacharparenleft}{\kern0pt}R\ {\isacharparenleft}{\kern0pt}Suc\ {\isacharparenleft}{\kern0pt}Suc\ n{\isacharparenright}{\kern0pt}{\isacharparenright}{\kern0pt}{\isacharparenright}{\kern0pt}\ {\isasymOtimes}\ {\isacharparenleft}{\kern0pt}{\isadigit{1}}\isactrlsub m\ {\isacharparenleft}{\kern0pt}{\isadigit{2}}{\isacharcircum}{\kern0pt}n{\isacharparenright}{\kern0pt}{\isacharparenright}{\kern0pt}{\isacharparenright}{\kern0pt}{\isacharparenright}{\kern0pt}\ {\isacharasterisk}{\kern0pt}\isanewline
\ \ \ \ \ \ \ \ \ \ {\isacharparenleft}{\kern0pt}{\isacharparenleft}{\kern0pt}{\isacharparenleft}{\kern0pt}\ {\isacharbar}{\kern0pt}Deutsch{\isachardot}{\kern0pt}zero{\isasymrangle}\ {\isacharplus}{\kern0pt}\ exp\ {\isacharparenleft}{\kern0pt}{\isadigit{2}}\ {\isacharasterisk}{\kern0pt}\ {\isasymi}\ {\isacharasterisk}{\kern0pt}\ pi\ {\isacharasterisk}{\kern0pt}\ complex{\isacharunderscore}{\kern0pt}of{\isacharunderscore}{\kern0pt}nat\ {\isacharparenleft}{\kern0pt}j\ div\ {\isadigit{2}}{\isacharparenright}{\kern0pt}\ {\isacharslash}{\kern0pt}\ {\isadigit{2}}\ {\isacharcircum}{\kern0pt}\ Suc\ n{\isacharparenright}{\kern0pt}\ {\isasymcdot}\isactrlsub m\ \isanewline
\ \ \ \ \ \ \ \ \ \ \ \ {\isacharbar}{\kern0pt}one{\isasymrangle}{\isacharparenright}{\kern0pt}\ {\isasymOtimes}\ \ {\isacharbar}{\kern0pt}state{\isacharunderscore}{\kern0pt}basis\ {\isadigit{1}}\ {\isacharparenleft}{\kern0pt}j\ mod\ {\isadigit{2}}{\isacharparenright}{\kern0pt}{\isasymrangle}{\isacharparenright}{\kern0pt}\ {\isasymOtimes}\ {\isacharbar}{\kern0pt}state{\isacharunderscore}{\kern0pt}basis\ n\ {\isacharparenleft}{\kern0pt}j\ mod\ {\isadigit{2}}\ {\isacharcircum}{\kern0pt}\ Suc\ n\ div\ {\isadigit{2}}{\isacharparenright}{\kern0pt}{\isasymrangle}{\isacharparenright}{\kern0pt}{\isacharparenright}{\kern0pt}{\isachardoublequoteclose}\isanewline
\ \ \ \ \isacommand{proof}\isamarkupfalse%
\ {\isacharparenleft}{\kern0pt}rule\ assoc{\isacharunderscore}{\kern0pt}mult{\isacharunderscore}{\kern0pt}mat{\isacharparenright}{\kern0pt}\isanewline
\ \ \ \ \ \ \isacommand{show}\isamarkupfalse%
\ {\isachardoublequoteopen}{\isadigit{1}}\isactrlsub m\ {\isadigit{2}}\ {\isasymOtimes}\ SWAP{\isacharunderscore}{\kern0pt}down\ {\isacharparenleft}{\kern0pt}Suc\ n{\isacharparenright}{\kern0pt}\ {\isasymin}\ carrier{\isacharunderscore}{\kern0pt}mat\ {\isacharparenleft}{\kern0pt}{\isadigit{2}}{\isacharcircum}{\kern0pt}Suc\ {\isacharparenleft}{\kern0pt}Suc\ n{\isacharparenright}{\kern0pt}{\isacharparenright}{\kern0pt}\ {\isacharparenleft}{\kern0pt}{\isadigit{2}}{\isacharcircum}{\kern0pt}Suc\ {\isacharparenleft}{\kern0pt}Suc\ n{\isacharparenright}{\kern0pt}{\isacharparenright}{\kern0pt}{\isachardoublequoteclose}\isanewline
\ \ \ \ \ \ \ \ \isacommand{using}\isamarkupfalse%
\ SWAP{\isacharunderscore}{\kern0pt}down{\isacharunderscore}{\kern0pt}carrier{\isacharunderscore}{\kern0pt}mat\ \isanewline
\ \ \ \ \ \ \ \ \isacommand{by}\isamarkupfalse%
\ {\isacharparenleft}{\kern0pt}metis\ One{\isacharunderscore}{\kern0pt}nat{\isacharunderscore}{\kern0pt}def\ SWAP{\isacharunderscore}{\kern0pt}down{\isachardot}{\kern0pt}simps{\isacharparenleft}{\kern0pt}{\isadigit{2}}{\isacharparenright}{\kern0pt}\ power{\isacharunderscore}{\kern0pt}Suc\ power{\isacharunderscore}{\kern0pt}one{\isacharunderscore}{\kern0pt}right\ tensor{\isacharunderscore}{\kern0pt}carrier{\isacharunderscore}{\kern0pt}mat{\isacharparenright}{\kern0pt}\isanewline
\ \ \ \ \ \ \isacommand{show}\isamarkupfalse%
\ {\isachardoublequoteopen}control{\isadigit{2}}\ {\isacharparenleft}{\kern0pt}R\ {\isacharparenleft}{\kern0pt}Suc\ {\isacharparenleft}{\kern0pt}Suc\ n{\isacharparenright}{\kern0pt}{\isacharparenright}{\kern0pt}{\isacharparenright}{\kern0pt}\ {\isasymOtimes}\ {\isadigit{1}}\isactrlsub m\ {\isacharparenleft}{\kern0pt}{\isadigit{2}}\ {\isacharcircum}{\kern0pt}\ n{\isacharparenright}{\kern0pt}\ {\isasymin}\ carrier{\isacharunderscore}{\kern0pt}mat\ {\isacharparenleft}{\kern0pt}{\isadigit{2}}{\isacharcircum}{\kern0pt}Suc\ {\isacharparenleft}{\kern0pt}Suc\ n{\isacharparenright}{\kern0pt}{\isacharparenright}{\kern0pt}\ {\isacharparenleft}{\kern0pt}{\isadigit{2}}{\isacharcircum}{\kern0pt}Suc\ {\isacharparenleft}{\kern0pt}Suc\ n{\isacharparenright}{\kern0pt}{\isacharparenright}{\kern0pt}{\isachardoublequoteclose}\isanewline
\ \ \ \ \ \ \ \ \isacommand{using}\isamarkupfalse%
\ control{\isadigit{2}}{\isacharunderscore}{\kern0pt}carrier{\isacharunderscore}{\kern0pt}mat\ \isacommand{by}\isamarkupfalse%
\ simp\isanewline
\ \ \ \ \ \ \isacommand{show}\isamarkupfalse%
\ {\isachardoublequoteopen}{\isacharbar}{\kern0pt}Deutsch{\isachardot}{\kern0pt}zero{\isasymrangle}\ {\isacharplus}{\kern0pt}\ exp\ {\isacharparenleft}{\kern0pt}{\isadigit{2}}\ {\isacharasterisk}{\kern0pt}\ {\isasymi}\ {\isacharasterisk}{\kern0pt}\ complex{\isacharunderscore}{\kern0pt}of{\isacharunderscore}{\kern0pt}real\ pi\ {\isacharasterisk}{\kern0pt}\ complex{\isacharunderscore}{\kern0pt}of{\isacharunderscore}{\kern0pt}nat\ {\isacharparenleft}{\kern0pt}j\ div\ {\isadigit{2}}{\isacharparenright}{\kern0pt}\ {\isacharslash}{\kern0pt}\ {\isadigit{2}}\ {\isacharcircum}{\kern0pt}\ Suc\ n{\isacharparenright}{\kern0pt}\isanewline
\ \ \ \ \ \ \ \ \ \ \ \ \ {\isasymcdot}\isactrlsub m\ {\isacharbar}{\kern0pt}Deutsch{\isachardot}{\kern0pt}one{\isasymrangle}\ {\isasymOtimes}\ {\isacharbar}{\kern0pt}state{\isacharunderscore}{\kern0pt}basis\ {\isadigit{1}}\ {\isacharparenleft}{\kern0pt}j\ mod\ {\isadigit{2}}{\isacharparenright}{\kern0pt}{\isasymrangle}\ {\isasymOtimes}\ {\isacharbar}{\kern0pt}state{\isacharunderscore}{\kern0pt}basis\ n\ {\isacharparenleft}{\kern0pt}j\ mod\ {\isadigit{2}}\ {\isacharcircum}{\kern0pt}\ Suc\ n\ div\ {\isadigit{2}}{\isacharparenright}{\kern0pt}{\isasymrangle}\isanewline
\ \ \ \ \ \ \ \ \ \ \ \ {\isasymin}\ carrier{\isacharunderscore}{\kern0pt}mat\ {\isacharparenleft}{\kern0pt}{\isadigit{2}}\ {\isacharcircum}{\kern0pt}\ Suc\ {\isacharparenleft}{\kern0pt}Suc\ n{\isacharparenright}{\kern0pt}{\isacharparenright}{\kern0pt}\ {\isadigit{1}}{\isachardoublequoteclose}\isanewline
\ \ \ \ \ \ \ \ \isacommand{using}\isamarkupfalse%
\ state{\isacharunderscore}{\kern0pt}basis{\isacharunderscore}{\kern0pt}carrier{\isacharunderscore}{\kern0pt}mat\ ket{\isacharunderscore}{\kern0pt}vec{\isacharunderscore}{\kern0pt}def\isanewline
\ \ \ \ \ \ \ \ \isacommand{by}\isamarkupfalse%
\ {\isacharparenleft}{\kern0pt}simp\ add{\isacharcolon}{\kern0pt}\ carrier{\isacharunderscore}{\kern0pt}matI\ state{\isacharunderscore}{\kern0pt}basis{\isacharunderscore}{\kern0pt}def{\isacharparenright}{\kern0pt}\isanewline
\ \ \ \ \isacommand{qed}\isamarkupfalse%
\isanewline
\ \ \ \ \isacommand{also}\isamarkupfalse%
\ \isacommand{have}\isamarkupfalse%
\ {\isachardoublequoteopen}{\isasymdots}\ {\isacharequal}{\kern0pt}\ {\isacharparenleft}{\kern0pt}{\isacharparenleft}{\kern0pt}{\isadigit{1}}\isactrlsub m\ {\isadigit{2}}{\isacharparenright}{\kern0pt}\ {\isasymOtimes}\ SWAP{\isacharunderscore}{\kern0pt}down\ {\isacharparenleft}{\kern0pt}Suc\ n{\isacharparenright}{\kern0pt}{\isacharparenright}{\kern0pt}\ {\isacharasterisk}{\kern0pt}\ {\isacharparenleft}{\kern0pt}{\isacharparenleft}{\kern0pt}{\isacharparenleft}{\kern0pt}control{\isadigit{2}}\ {\isacharparenleft}{\kern0pt}R\ {\isacharparenleft}{\kern0pt}Suc\ {\isacharparenleft}{\kern0pt}Suc\ n{\isacharparenright}{\kern0pt}{\isacharparenright}{\kern0pt}{\isacharparenright}{\kern0pt}{\isacharparenright}{\kern0pt}\ {\isacharasterisk}{\kern0pt}\isanewline
\ \ \ \ \ \ \ \ \ \ \ \ \ \ {\isacharparenleft}{\kern0pt}{\isacharparenleft}{\kern0pt}\ {\isacharbar}{\kern0pt}Deutsch{\isachardot}{\kern0pt}zero{\isasymrangle}\ {\isacharplus}{\kern0pt}\ exp\ {\isacharparenleft}{\kern0pt}{\isadigit{2}}\ {\isacharasterisk}{\kern0pt}\ {\isasymi}\ {\isacharasterisk}{\kern0pt}\ pi\ {\isacharasterisk}{\kern0pt}\ complex{\isacharunderscore}{\kern0pt}of{\isacharunderscore}{\kern0pt}nat\ {\isacharparenleft}{\kern0pt}j\ div\ {\isadigit{2}}{\isacharparenright}{\kern0pt}\ {\isacharslash}{\kern0pt}\ {\isadigit{2}}\ {\isacharcircum}{\kern0pt}\ Suc\ n{\isacharparenright}{\kern0pt}\ {\isasymcdot}\isactrlsub m\ {\isacharbar}{\kern0pt}one{\isasymrangle}{\isacharparenright}{\kern0pt}\isanewline
\ \ \ \ \ \ \ \ \ \ \ \ {\isasymOtimes}\ {\isacharbar}{\kern0pt}state{\isacharunderscore}{\kern0pt}basis\ {\isadigit{1}}\ {\isacharparenleft}{\kern0pt}j\ mod\ {\isadigit{2}}{\isacharparenright}{\kern0pt}{\isasymrangle}{\isacharparenright}{\kern0pt}{\isacharparenright}{\kern0pt}\ {\isasymOtimes}\ {\isacharparenleft}{\kern0pt}{\isacharparenleft}{\kern0pt}{\isadigit{1}}\isactrlsub m\ {\isacharparenleft}{\kern0pt}{\isadigit{2}}{\isacharcircum}{\kern0pt}n{\isacharparenright}{\kern0pt}{\isacharparenright}{\kern0pt}\ {\isacharasterisk}{\kern0pt}\ {\isacharbar}{\kern0pt}state{\isacharunderscore}{\kern0pt}basis\ n\ {\isacharparenleft}{\kern0pt}j\ mod\ {\isadigit{2}}\ {\isacharcircum}{\kern0pt}\ Suc\ n\ div\ {\isadigit{2}}{\isacharparenright}{\kern0pt}{\isasymrangle}{\isacharparenright}{\kern0pt}{\isacharparenright}{\kern0pt}{\isachardoublequoteclose}\isanewline
\ \ \ \ \ \ \isacommand{using}\isamarkupfalse%
\ mult{\isacharunderscore}{\kern0pt}distr{\isacharunderscore}{\kern0pt}tensor\ \isanewline
\ \ \ \ \ \ \isacommand{by}\isamarkupfalse%
\ {\isacharparenleft}{\kern0pt}smt\ {\isacharparenleft}{\kern0pt}verit{\isacharcomma}{\kern0pt}\ del{\isacharunderscore}{\kern0pt}insts{\isacharparenright}{\kern0pt}\ SWAP{\isacharunderscore}{\kern0pt}nrows\ SWAP{\isacharunderscore}{\kern0pt}tensor\ carrier{\isacharunderscore}{\kern0pt}matD{\isacharparenleft}{\kern0pt}{\isadigit{1}}{\isacharparenright}{\kern0pt}\ carrier{\isacharunderscore}{\kern0pt}matD{\isacharparenleft}{\kern0pt}{\isadigit{2}}{\isacharparenright}{\kern0pt}\ \isanewline
\ \ \ \ \ \ \ \ \ \ carrier{\isacharunderscore}{\kern0pt}matI\ control{\isadigit{2}}{\isacharunderscore}{\kern0pt}carrier{\isacharunderscore}{\kern0pt}mat\ dim{\isacharunderscore}{\kern0pt}col{\isacharunderscore}{\kern0pt}tensor{\isacharunderscore}{\kern0pt}mat\ index{\isacharunderscore}{\kern0pt}add{\isacharunderscore}{\kern0pt}mat{\isacharparenleft}{\kern0pt}{\isadigit{2}}{\isacharparenright}{\kern0pt}\ index{\isacharunderscore}{\kern0pt}add{\isacharunderscore}{\kern0pt}mat{\isacharparenleft}{\kern0pt}{\isadigit{3}}{\isacharparenright}{\kern0pt}\ \isanewline
\ \ \ \ \ \ \ \ \ \ index{\isacharunderscore}{\kern0pt}mult{\isacharunderscore}{\kern0pt}mat{\isacharparenleft}{\kern0pt}{\isadigit{2}}{\isacharparenright}{\kern0pt}\ index{\isacharunderscore}{\kern0pt}one{\isacharunderscore}{\kern0pt}mat{\isacharparenleft}{\kern0pt}{\isadigit{3}}{\isacharparenright}{\kern0pt}\ index{\isacharunderscore}{\kern0pt}smult{\isacharunderscore}{\kern0pt}mat{\isacharparenleft}{\kern0pt}{\isadigit{2}}{\isacharparenright}{\kern0pt}\ index{\isacharunderscore}{\kern0pt}smult{\isacharunderscore}{\kern0pt}mat{\isacharparenleft}{\kern0pt}{\isadigit{3}}{\isacharparenright}{\kern0pt}\ ket{\isacharunderscore}{\kern0pt}one{\isacharunderscore}{\kern0pt}is{\isacharunderscore}{\kern0pt}state\isanewline
\ \ \ \ \ \ \ \ \ \ less{\isacharunderscore}{\kern0pt}numeral{\isacharunderscore}{\kern0pt}extra{\isacharparenleft}{\kern0pt}{\isadigit{1}}{\isacharparenright}{\kern0pt}\ one{\isacharunderscore}{\kern0pt}power{\isadigit{2}}\ power{\isacharunderscore}{\kern0pt}Suc{\isadigit{2}}\ power{\isacharunderscore}{\kern0pt}one{\isacharunderscore}{\kern0pt}right\ state{\isacharunderscore}{\kern0pt}basis{\isacharunderscore}{\kern0pt}carrier{\isacharunderscore}{\kern0pt}mat\isanewline
\ \ \ \ \ \ \ \ \ \ state{\isacharunderscore}{\kern0pt}def\ zero{\isacharunderscore}{\kern0pt}less{\isacharunderscore}{\kern0pt}numeral\ zero{\isacharunderscore}{\kern0pt}less{\isacharunderscore}{\kern0pt}power{\isacharparenright}{\kern0pt}\isanewline
\ \ \ \ \isacommand{also}\isamarkupfalse%
\ \isacommand{have}\isamarkupfalse%
\ {\isachardoublequoteopen}{\isasymdots}\ {\isacharequal}{\kern0pt}\ {\isacharparenleft}{\kern0pt}{\isacharparenleft}{\kern0pt}{\isadigit{1}}\isactrlsub m\ {\isadigit{2}}{\isacharparenright}{\kern0pt}\ {\isasymOtimes}\ SWAP{\isacharunderscore}{\kern0pt}down\ {\isacharparenleft}{\kern0pt}Suc\ n{\isacharparenright}{\kern0pt}{\isacharparenright}{\kern0pt}\ {\isacharasterisk}{\kern0pt}\ \isanewline
\ \ \ \ \ \ \ \ \ \ \ \ \ \ \ {\isacharparenleft}{\kern0pt}{\isacharparenleft}{\kern0pt}\ {\isacharbar}{\kern0pt}zero{\isasymrangle}\ {\isacharplus}{\kern0pt}\ exp\ {\isacharparenleft}{\kern0pt}{\isadigit{2}}\ {\isacharasterisk}{\kern0pt}\ {\isasymi}\ {\isacharasterisk}{\kern0pt}\ pi\ {\isacharasterisk}{\kern0pt}\ complex{\isacharunderscore}{\kern0pt}of{\isacharunderscore}{\kern0pt}nat\ j\ {\isacharslash}{\kern0pt}\ {\isadigit{2}}\ {\isacharcircum}{\kern0pt}\ Suc\ {\isacharparenleft}{\kern0pt}Suc\ n{\isacharparenright}{\kern0pt}{\isacharparenright}{\kern0pt}\ {\isasymcdot}\isactrlsub m\ {\isacharbar}{\kern0pt}one{\isasymrangle}{\isacharparenright}{\kern0pt}\ {\isasymOtimes}\isanewline
\ \ \ \ \ \ \ \ \ \ \ \ \ \ \ {\isacharbar}{\kern0pt}state{\isacharunderscore}{\kern0pt}basis\ {\isadigit{1}}\ {\isacharparenleft}{\kern0pt}j\ mod\ {\isadigit{2}}{\isacharparenright}{\kern0pt}{\isasymrangle}\ {\isasymOtimes}\ {\isacharparenleft}{\kern0pt}{\isacharparenleft}{\kern0pt}{\isadigit{1}}\isactrlsub m\ {\isacharparenleft}{\kern0pt}{\isadigit{2}}{\isacharcircum}{\kern0pt}n{\isacharparenright}{\kern0pt}{\isacharparenright}{\kern0pt}\ {\isacharasterisk}{\kern0pt}\ {\isacharbar}{\kern0pt}state{\isacharunderscore}{\kern0pt}basis\ n\ {\isacharparenleft}{\kern0pt}j\ mod\ {\isadigit{2}}\ {\isacharcircum}{\kern0pt}\ Suc\ n\ div\ {\isadigit{2}}{\isacharparenright}{\kern0pt}{\isasymrangle}{\isacharparenright}{\kern0pt}{\isacharparenright}{\kern0pt}{\isachardoublequoteclose}\isanewline
\ \ \ \ \isacommand{proof}\isamarkupfalse%
\ {\isacharparenleft}{\kern0pt}rule\ disjE{\isacharparenright}{\kern0pt}\isanewline
\ \ \ \ \ \ \isacommand{show}\isamarkupfalse%
\ {\isachardoublequoteopen}j\ mod\ {\isadigit{2}}\ {\isacharequal}{\kern0pt}\ {\isadigit{0}}\ {\isasymor}\ j\ mod\ {\isadigit{2}}\ {\isacharequal}{\kern0pt}\ {\isadigit{1}}{\isachardoublequoteclose}\ \isacommand{by}\isamarkupfalse%
\ auto\isanewline
\ \ \ \ \isacommand{next}\isamarkupfalse%
\isanewline
\ \ \ \ \ \ \isacommand{assume}\isamarkupfalse%
\ jm{\isadigit{0}}{\isacharcolon}{\kern0pt}{\isachardoublequoteopen}j\ mod\ {\isadigit{2}}\ {\isacharequal}{\kern0pt}\ {\isadigit{0}}{\isachardoublequoteclose}\isanewline
\ \ \ \ \ \ \isacommand{hence}\isamarkupfalse%
\ jid{\isacharcolon}{\kern0pt}{\isachardoublequoteopen}j\ {\isacharslash}{\kern0pt}\ {\isadigit{2}}\ {\isacharequal}{\kern0pt}\ j\ div\ {\isadigit{2}}{\isachardoublequoteclose}\ \isacommand{by}\isamarkupfalse%
\ auto\isanewline
\ \ \ \ \ \ \isacommand{have}\isamarkupfalse%
\ {\isachardoublequoteopen}{\isacharparenleft}{\kern0pt}control{\isadigit{2}}\ {\isacharparenleft}{\kern0pt}R\ {\isacharparenleft}{\kern0pt}Suc\ {\isacharparenleft}{\kern0pt}Suc\ n{\isacharparenright}{\kern0pt}{\isacharparenright}{\kern0pt}{\isacharparenright}{\kern0pt}{\isacharparenright}{\kern0pt}\ {\isacharasterisk}{\kern0pt}\isanewline
\ \ \ \ \ \ \ \ \ \ \ \ \ \ {\isacharparenleft}{\kern0pt}{\isacharparenleft}{\kern0pt}\ {\isacharbar}{\kern0pt}Deutsch{\isachardot}{\kern0pt}zero{\isasymrangle}\ {\isacharplus}{\kern0pt}\ exp\ {\isacharparenleft}{\kern0pt}{\isadigit{2}}\ {\isacharasterisk}{\kern0pt}\ {\isasymi}\ {\isacharasterisk}{\kern0pt}\ pi\ {\isacharasterisk}{\kern0pt}\ complex{\isacharunderscore}{\kern0pt}of{\isacharunderscore}{\kern0pt}nat\ {\isacharparenleft}{\kern0pt}j\ div\ {\isadigit{2}}{\isacharparenright}{\kern0pt}\ {\isacharslash}{\kern0pt}\ {\isadigit{2}}\ {\isacharcircum}{\kern0pt}\ Suc\ n{\isacharparenright}{\kern0pt}\ {\isasymcdot}\isactrlsub m\ {\isacharbar}{\kern0pt}one{\isasymrangle}{\isacharparenright}{\kern0pt}\isanewline
\ \ \ \ \ \ \ \ \ \ \ \ \ \ {\isasymOtimes}\ {\isacharbar}{\kern0pt}state{\isacharunderscore}{\kern0pt}basis\ {\isadigit{1}}\ {\isacharparenleft}{\kern0pt}j\ mod\ {\isadigit{2}}{\isacharparenright}{\kern0pt}{\isasymrangle}{\isacharparenright}{\kern0pt}\ {\isacharequal}{\kern0pt}\ \isanewline
\ \ \ \ \ \ \ \ \ \ \ \ {\isacharparenleft}{\kern0pt}control{\isadigit{2}}\ {\isacharparenleft}{\kern0pt}R\ {\isacharparenleft}{\kern0pt}Suc\ {\isacharparenleft}{\kern0pt}Suc\ n{\isacharparenright}{\kern0pt}{\isacharparenright}{\kern0pt}{\isacharparenright}{\kern0pt}{\isacharparenright}{\kern0pt}\ {\isacharasterisk}{\kern0pt}\isanewline
\ \ \ \ \ \ \ \ \ \ \ \ \ \ {\isacharparenleft}{\kern0pt}{\isacharparenleft}{\kern0pt}\ {\isacharbar}{\kern0pt}Deutsch{\isachardot}{\kern0pt}zero{\isasymrangle}\ {\isacharplus}{\kern0pt}\ exp\ {\isacharparenleft}{\kern0pt}{\isadigit{2}}\ {\isacharasterisk}{\kern0pt}\ {\isasymi}\ {\isacharasterisk}{\kern0pt}\ pi\ {\isacharasterisk}{\kern0pt}\ complex{\isacharunderscore}{\kern0pt}of{\isacharunderscore}{\kern0pt}nat\ {\isacharparenleft}{\kern0pt}j\ div\ {\isadigit{2}}{\isacharparenright}{\kern0pt}\ {\isacharslash}{\kern0pt}\ {\isadigit{2}}\ {\isacharcircum}{\kern0pt}\ Suc\ n{\isacharparenright}{\kern0pt}\ {\isasymcdot}\isactrlsub m\ {\isacharbar}{\kern0pt}one{\isasymrangle}{\isacharparenright}{\kern0pt}\isanewline
\ \ \ \ \ \ \ \ \ \ \ \ \ \ {\isasymOtimes}\ {\isacharbar}{\kern0pt}zero{\isasymrangle}{\isacharparenright}{\kern0pt}{\isachardoublequoteclose}\isanewline
\ \ \ \ \ \ \ \ \isacommand{using}\isamarkupfalse%
\ state{\isacharunderscore}{\kern0pt}basis{\isacharunderscore}{\kern0pt}def\ jm{\isadigit{0}}\ \isacommand{by}\isamarkupfalse%
\ fastforce\isanewline
\ \ \ \ \ \ \isacommand{also}\isamarkupfalse%
\ \isacommand{have}\isamarkupfalse%
\ {\isachardoublequoteopen}{\isasymdots}\ {\isacharequal}{\kern0pt}\ {\isacharparenleft}{\kern0pt}{\isacharparenleft}{\kern0pt}\ {\isacharbar}{\kern0pt}zero{\isasymrangle}\ {\isacharplus}{\kern0pt}\ exp\ {\isacharparenleft}{\kern0pt}{\isadigit{2}}\ {\isacharasterisk}{\kern0pt}\ {\isasymi}\ {\isacharasterisk}{\kern0pt}\ pi\ {\isacharasterisk}{\kern0pt}\ complex{\isacharunderscore}{\kern0pt}of{\isacharunderscore}{\kern0pt}nat\ {\isacharparenleft}{\kern0pt}j\ div\ {\isadigit{2}}{\isacharparenright}{\kern0pt}\ {\isacharslash}{\kern0pt}\ {\isadigit{2}}\ {\isacharcircum}{\kern0pt}\ Suc\ n{\isacharparenright}{\kern0pt}\ {\isasymcdot}\isactrlsub m\ {\isacharbar}{\kern0pt}one{\isasymrangle}{\isacharparenright}{\kern0pt}\isanewline
\ \ \ \ \ \ \ \ \ \ \ \ \ \ {\isasymOtimes}\ {\isacharbar}{\kern0pt}zero{\isasymrangle}{\isacharparenright}{\kern0pt}{\isachardoublequoteclose}\isanewline
\ \ \ \ \ \ \ \ \isacommand{using}\isamarkupfalse%
\ control{\isadigit{2}}{\isacharunderscore}{\kern0pt}zero\ \isacommand{by}\isamarkupfalse%
\ {\isacharparenleft}{\kern0pt}simp\ add{\isacharcolon}{\kern0pt}\ ket{\isacharunderscore}{\kern0pt}vec{\isacharunderscore}{\kern0pt}def{\isacharparenright}{\kern0pt}\isanewline
\ \ \ \ \ \ \isacommand{also}\isamarkupfalse%
\ \isacommand{have}\isamarkupfalse%
\ {\isachardoublequoteopen}{\isasymdots}\ {\isacharequal}{\kern0pt}\ {\isacharparenleft}{\kern0pt}\ {\isacharbar}{\kern0pt}zero{\isasymrangle}\ {\isacharplus}{\kern0pt}\ exp\ {\isacharparenleft}{\kern0pt}{\isadigit{2}}\ {\isacharasterisk}{\kern0pt}\ {\isasymi}\ {\isacharasterisk}{\kern0pt}\ pi\ {\isacharasterisk}{\kern0pt}\ complex{\isacharunderscore}{\kern0pt}of{\isacharunderscore}{\kern0pt}nat\ j\ {\isacharslash}{\kern0pt}\ {\isadigit{2}}\ {\isacharcircum}{\kern0pt}\ Suc\ {\isacharparenleft}{\kern0pt}Suc\ n{\isacharparenright}{\kern0pt}{\isacharparenright}{\kern0pt}\ {\isasymcdot}\isactrlsub m\ {\isacharbar}{\kern0pt}one{\isasymrangle}{\isacharparenright}{\kern0pt}\ {\isasymOtimes}\isanewline
\ \ \ \ \ \ \ \ \ \ \ \ \ \ \ \ \ \ \ \ \ \ \ \ {\isacharbar}{\kern0pt}zero{\isasymrangle}{\isachardoublequoteclose}\isanewline
\ \ \ \ \ \ \ \ \isacommand{using}\isamarkupfalse%
\ jid\ \isanewline
\ \ \ \ \ \ \ \ \isacommand{by}\isamarkupfalse%
\ {\isacharparenleft}{\kern0pt}smt\ {\isacharparenleft}{\kern0pt}verit{\isacharcomma}{\kern0pt}\ del{\isacharunderscore}{\kern0pt}insts{\isacharparenright}{\kern0pt}\ dbl{\isacharunderscore}{\kern0pt}simps{\isacharparenleft}{\kern0pt}{\isadigit{3}}{\isacharparenright}{\kern0pt}\ dbl{\isacharunderscore}{\kern0pt}simps{\isacharparenleft}{\kern0pt}{\isadigit{5}}{\isacharparenright}{\kern0pt}\ divide{\isacharunderscore}{\kern0pt}divide{\isacharunderscore}{\kern0pt}eq{\isacharunderscore}{\kern0pt}left\ numerals{\isacharparenleft}{\kern0pt}{\isadigit{1}}{\isacharparenright}{\kern0pt}\ \isanewline
\ \ \ \ \ \ \ \ \ \ \ \ of{\isacharunderscore}{\kern0pt}nat{\isacharunderscore}{\kern0pt}{\isadigit{1}}\ of{\isacharunderscore}{\kern0pt}nat{\isacharunderscore}{\kern0pt}numeral\ of{\isacharunderscore}{\kern0pt}real{\isacharunderscore}{\kern0pt}divide\ of{\isacharunderscore}{\kern0pt}real{\isacharunderscore}{\kern0pt}of{\isacharunderscore}{\kern0pt}nat{\isacharunderscore}{\kern0pt}eq\ power{\isacharunderscore}{\kern0pt}Suc\isanewline
\ \ \ \ \ \ \ \ \ \ \ \ times{\isacharunderscore}{\kern0pt}divide{\isacharunderscore}{\kern0pt}eq{\isacharunderscore}{\kern0pt}right{\isacharparenright}{\kern0pt}\isanewline
\ \ \ \ \ \ \isacommand{finally}\isamarkupfalse%
\ \isacommand{show}\isamarkupfalse%
\ {\isachardoublequoteopen}{\isacharparenleft}{\kern0pt}{\isadigit{1}}\isactrlsub m\ {\isadigit{2}}\ {\isasymOtimes}\ SWAP{\isacharunderscore}{\kern0pt}down\ {\isacharparenleft}{\kern0pt}Suc\ n{\isacharparenright}{\kern0pt}{\isacharparenright}{\kern0pt}\ {\isacharasterisk}{\kern0pt}\ {\isacharparenleft}{\kern0pt}control{\isadigit{2}}\ {\isacharparenleft}{\kern0pt}R\ {\isacharparenleft}{\kern0pt}Suc\ {\isacharparenleft}{\kern0pt}Suc\ n{\isacharparenright}{\kern0pt}{\isacharparenright}{\kern0pt}{\isacharparenright}{\kern0pt}\ {\isacharasterisk}{\kern0pt}\ {\isacharparenleft}{\kern0pt}\ {\isacharbar}{\kern0pt}Deutsch{\isachardot}{\kern0pt}zero{\isasymrangle}\ {\isacharplus}{\kern0pt}\isanewline
\ \ \ \ \ \ \ \ \ \ \ \ \ \ \ \ \ \ \ \ exp\ {\isacharparenleft}{\kern0pt}{\isadigit{2}}\ {\isacharasterisk}{\kern0pt}\ {\isasymi}\ {\isacharasterisk}{\kern0pt}\ complex{\isacharunderscore}{\kern0pt}of{\isacharunderscore}{\kern0pt}real\ pi\ {\isacharasterisk}{\kern0pt}\ complex{\isacharunderscore}{\kern0pt}of{\isacharunderscore}{\kern0pt}nat\ {\isacharparenleft}{\kern0pt}j\ div\ {\isadigit{2}}{\isacharparenright}{\kern0pt}\ {\isacharslash}{\kern0pt}\ {\isadigit{2}}\ {\isacharcircum}{\kern0pt}\ Suc\ n{\isacharparenright}{\kern0pt}\ {\isasymcdot}\isactrlsub m\isanewline
\ \ \ \ \ \ \ \ \ \ \ \ \ \ \ \ \ \ \ \ {\isacharbar}{\kern0pt}Deutsch{\isachardot}{\kern0pt}one{\isasymrangle}\ {\isasymOtimes}\ {\isacharbar}{\kern0pt}state{\isacharunderscore}{\kern0pt}basis\ {\isadigit{1}}\ {\isacharparenleft}{\kern0pt}j\ mod\ {\isadigit{2}}{\isacharparenright}{\kern0pt}{\isasymrangle}{\isacharparenright}{\kern0pt}\ {\isasymOtimes}\ {\isadigit{1}}\isactrlsub m\ {\isacharparenleft}{\kern0pt}{\isadigit{2}}\ {\isacharcircum}{\kern0pt}\ n{\isacharparenright}{\kern0pt}\ {\isacharasterisk}{\kern0pt}\isanewline
\ \ \ \ \ \ \ \ \ \ \ \ \ \ \ \ \ \ \ \ {\isacharbar}{\kern0pt}state{\isacharunderscore}{\kern0pt}basis\ n\ {\isacharparenleft}{\kern0pt}j\ mod\ {\isadigit{2}}\ {\isacharcircum}{\kern0pt}\ Suc\ n\ div\ {\isadigit{2}}{\isacharparenright}{\kern0pt}{\isasymrangle}{\isacharparenright}{\kern0pt}\ {\isacharequal}{\kern0pt}\ {\isacharparenleft}{\kern0pt}{\isadigit{1}}\isactrlsub m\ {\isadigit{2}}\ {\isasymOtimes}\ SWAP{\isacharunderscore}{\kern0pt}down\ {\isacharparenleft}{\kern0pt}Suc\ n{\isacharparenright}{\kern0pt}{\isacharparenright}{\kern0pt}\ {\isacharasterisk}{\kern0pt}\isanewline
\ \ \ \ \ \ \ \ \ \ \ \ \ \ \ \ \ \ {\isacharparenleft}{\kern0pt}\ {\isacharbar}{\kern0pt}Deutsch{\isachardot}{\kern0pt}zero{\isasymrangle}\ {\isacharplus}{\kern0pt}\ exp\ {\isacharparenleft}{\kern0pt}{\isadigit{2}}\ {\isacharasterisk}{\kern0pt}\ {\isasymi}\ {\isacharasterisk}{\kern0pt}\ complex{\isacharunderscore}{\kern0pt}of{\isacharunderscore}{\kern0pt}real\ pi\ {\isacharasterisk}{\kern0pt}\ complex{\isacharunderscore}{\kern0pt}of{\isacharunderscore}{\kern0pt}nat\ j\ {\isacharslash}{\kern0pt}\isanewline
\ \ \ \ \ \ \ \ \ \ \ \ \ \ \ \ \ \ \ \ {\isadigit{2}}\ {\isacharcircum}{\kern0pt}\ Suc\ {\isacharparenleft}{\kern0pt}Suc\ n{\isacharparenright}{\kern0pt}{\isacharparenright}{\kern0pt}\ {\isasymcdot}\isactrlsub m\ {\isacharbar}{\kern0pt}Deutsch{\isachardot}{\kern0pt}one{\isasymrangle}\ {\isasymOtimes}\ {\isacharbar}{\kern0pt}state{\isacharunderscore}{\kern0pt}basis\ {\isadigit{1}}\ {\isacharparenleft}{\kern0pt}j\ mod\ {\isadigit{2}}{\isacharparenright}{\kern0pt}{\isasymrangle}\ {\isasymOtimes}\ {\isadigit{1}}\isactrlsub m\ {\isacharparenleft}{\kern0pt}{\isadigit{2}}\ {\isacharcircum}{\kern0pt}\ n{\isacharparenright}{\kern0pt}\ {\isacharasterisk}{\kern0pt}\isanewline
\ \ \ \ \ \ \ \ \ \ \ \ \ \ \ \ \ \ \ \ {\isacharbar}{\kern0pt}state{\isacharunderscore}{\kern0pt}basis\ n\ {\isacharparenleft}{\kern0pt}j\ mod\ {\isadigit{2}}\ {\isacharcircum}{\kern0pt}\ Suc\ n\ div\ {\isadigit{2}}{\isacharparenright}{\kern0pt}{\isasymrangle}{\isacharparenright}{\kern0pt}{\isachardoublequoteclose}\ \isanewline
\ \ \ \ \ \ \ \ \isacommand{by}\isamarkupfalse%
\ {\isacharparenleft}{\kern0pt}metis\ jm{\isadigit{0}}\ power{\isacharunderscore}{\kern0pt}one{\isacharunderscore}{\kern0pt}right\ state{\isacharunderscore}{\kern0pt}basis{\isacharunderscore}{\kern0pt}def{\isacharparenright}{\kern0pt}\isanewline
\ \ \ \ \isacommand{next}\isamarkupfalse%
\isanewline
\ \ \ \ \ \ \isacommand{assume}\isamarkupfalse%
\ jm{\isadigit{1}}{\isacharcolon}{\kern0pt}{\isachardoublequoteopen}j\ mod\ {\isadigit{2}}\ {\isacharequal}{\kern0pt}\ {\isadigit{1}}{\isachardoublequoteclose}\isanewline
\ \ \ \ \ \ \isacommand{have}\isamarkupfalse%
\ {\isachardoublequoteopen}{\isacharparenleft}{\kern0pt}control{\isadigit{2}}\ {\isacharparenleft}{\kern0pt}R\ {\isacharparenleft}{\kern0pt}Suc\ {\isacharparenleft}{\kern0pt}Suc\ n{\isacharparenright}{\kern0pt}{\isacharparenright}{\kern0pt}{\isacharparenright}{\kern0pt}{\isacharparenright}{\kern0pt}\ {\isacharasterisk}{\kern0pt}\isanewline
\ \ \ \ \ \ \ \ \ \ \ \ \ \ {\isacharparenleft}{\kern0pt}{\isacharparenleft}{\kern0pt}\ {\isacharbar}{\kern0pt}Deutsch{\isachardot}{\kern0pt}zero{\isasymrangle}\ {\isacharplus}{\kern0pt}\ exp\ {\isacharparenleft}{\kern0pt}{\isadigit{2}}\ {\isacharasterisk}{\kern0pt}\ {\isasymi}\ {\isacharasterisk}{\kern0pt}\ pi\ {\isacharasterisk}{\kern0pt}\ complex{\isacharunderscore}{\kern0pt}of{\isacharunderscore}{\kern0pt}nat\ {\isacharparenleft}{\kern0pt}j\ div\ {\isadigit{2}}{\isacharparenright}{\kern0pt}\ {\isacharslash}{\kern0pt}\ {\isadigit{2}}\ {\isacharcircum}{\kern0pt}\ Suc\ n{\isacharparenright}{\kern0pt}\ {\isasymcdot}\isactrlsub m\ {\isacharbar}{\kern0pt}one{\isasymrangle}{\isacharparenright}{\kern0pt}\isanewline
\ \ \ \ \ \ \ \ \ \ \ \ \ \ {\isasymOtimes}\ {\isacharbar}{\kern0pt}state{\isacharunderscore}{\kern0pt}basis\ {\isadigit{1}}\ {\isacharparenleft}{\kern0pt}j\ mod\ {\isadigit{2}}{\isacharparenright}{\kern0pt}{\isasymrangle}{\isacharparenright}{\kern0pt}\ {\isacharequal}{\kern0pt}\ \isanewline
\ \ \ \ \ \ \ \ \ \ \ \ {\isacharparenleft}{\kern0pt}control{\isadigit{2}}\ {\isacharparenleft}{\kern0pt}R\ {\isacharparenleft}{\kern0pt}Suc\ {\isacharparenleft}{\kern0pt}Suc\ n{\isacharparenright}{\kern0pt}{\isacharparenright}{\kern0pt}{\isacharparenright}{\kern0pt}{\isacharparenright}{\kern0pt}\ {\isacharasterisk}{\kern0pt}\isanewline
\ \ \ \ \ \ \ \ \ \ \ \ \ \ {\isacharparenleft}{\kern0pt}{\isacharparenleft}{\kern0pt}\ {\isacharbar}{\kern0pt}Deutsch{\isachardot}{\kern0pt}zero{\isasymrangle}\ {\isacharplus}{\kern0pt}\ exp\ {\isacharparenleft}{\kern0pt}{\isadigit{2}}\ {\isacharasterisk}{\kern0pt}\ {\isasymi}\ {\isacharasterisk}{\kern0pt}\ pi\ {\isacharasterisk}{\kern0pt}\ complex{\isacharunderscore}{\kern0pt}of{\isacharunderscore}{\kern0pt}nat\ {\isacharparenleft}{\kern0pt}j\ div\ {\isadigit{2}}{\isacharparenright}{\kern0pt}\ {\isacharslash}{\kern0pt}\ {\isadigit{2}}\ {\isacharcircum}{\kern0pt}\ Suc\ n{\isacharparenright}{\kern0pt}\ {\isasymcdot}\isactrlsub m\ {\isacharbar}{\kern0pt}one{\isasymrangle}{\isacharparenright}{\kern0pt}\isanewline
\ \ \ \ \ \ \ \ \ \ \ \ \ \ {\isasymOtimes}\ {\isacharbar}{\kern0pt}one{\isasymrangle}{\isacharparenright}{\kern0pt}{\isachardoublequoteclose}\isanewline
\ \ \ \ \ \ \ \ \isacommand{using}\isamarkupfalse%
\ jm{\isadigit{1}}\ state{\isacharunderscore}{\kern0pt}basis{\isacharunderscore}{\kern0pt}def\ \isacommand{by}\isamarkupfalse%
\ fastforce\isanewline
\ \ \ \ \ \ \isacommand{also}\isamarkupfalse%
\ \isacommand{have}\isamarkupfalse%
\ {\isachardoublequoteopen}{\isasymdots}\ {\isacharequal}{\kern0pt}\ {\isacharparenleft}{\kern0pt}{\isacharparenleft}{\kern0pt}R\ {\isacharparenleft}{\kern0pt}Suc\ {\isacharparenleft}{\kern0pt}Suc\ n{\isacharparenright}{\kern0pt}{\isacharparenright}{\kern0pt}{\isacharparenright}{\kern0pt}\ {\isacharasterisk}{\kern0pt}\ \isanewline
\ \ \ \ \ \ \ \ \ \ \ \ \ \ \ \ \ \ \ \ \ \ {\isacharparenleft}{\kern0pt}\ {\isacharbar}{\kern0pt}zero{\isasymrangle}\ {\isacharplus}{\kern0pt}\ exp\ {\isacharparenleft}{\kern0pt}{\isadigit{2}}\ {\isacharasterisk}{\kern0pt}\ {\isasymi}\ {\isacharasterisk}{\kern0pt}\ pi\ {\isacharasterisk}{\kern0pt}\ complex{\isacharunderscore}{\kern0pt}of{\isacharunderscore}{\kern0pt}nat\ {\isacharparenleft}{\kern0pt}j\ div\ {\isadigit{2}}{\isacharparenright}{\kern0pt}\ {\isacharslash}{\kern0pt}\ {\isadigit{2}}\ {\isacharcircum}{\kern0pt}\ Suc\ n{\isacharparenright}{\kern0pt}\ {\isasymcdot}\isactrlsub m\ {\isacharbar}{\kern0pt}one{\isasymrangle}{\isacharparenright}{\kern0pt}{\isacharparenright}{\kern0pt}\isanewline
\ \ \ \ \ \ \ \ \ \ \ \ \ \ \ \ \ \ \ \ \ \ {\isasymOtimes}\ {\isacharbar}{\kern0pt}one{\isasymrangle}{\isachardoublequoteclose}\isanewline
\ \ \ \ \ \ \ \ \isacommand{using}\isamarkupfalse%
\ control{\isadigit{2}}{\isacharunderscore}{\kern0pt}one\ \isacommand{by}\isamarkupfalse%
\ {\isacharparenleft}{\kern0pt}simp\ add{\isacharcolon}{\kern0pt}\ ket{\isacharunderscore}{\kern0pt}vec{\isacharunderscore}{\kern0pt}def\ R{\isacharunderscore}{\kern0pt}def\ mat{\isacharunderscore}{\kern0pt}of{\isacharunderscore}{\kern0pt}cols{\isacharunderscore}{\kern0pt}list{\isacharunderscore}{\kern0pt}def{\isacharparenright}{\kern0pt}\isanewline
\ \ \ \ \ \ \isacommand{also}\isamarkupfalse%
\ \isacommand{have}\isamarkupfalse%
\ {\isachardoublequoteopen}{\isasymdots}\ {\isacharequal}{\kern0pt}\ {\isacharparenleft}{\kern0pt}\ {\isacharbar}{\kern0pt}zero{\isasymrangle}\ {\isacharplus}{\kern0pt}\ exp\ {\isacharparenleft}{\kern0pt}{\isadigit{2}}{\isacharasterisk}{\kern0pt}{\isasymi}{\isacharasterisk}{\kern0pt}pi{\isacharasterisk}{\kern0pt}complex{\isacharunderscore}{\kern0pt}of{\isacharunderscore}{\kern0pt}nat\ j\ {\isacharslash}{\kern0pt}\ {\isadigit{2}}{\isacharcircum}{\kern0pt}{\isacharparenleft}{\kern0pt}Suc\ {\isacharparenleft}{\kern0pt}Suc\ n{\isacharparenright}{\kern0pt}{\isacharparenright}{\kern0pt}{\isacharparenright}{\kern0pt}\ {\isasymcdot}\isactrlsub m\ {\isacharbar}{\kern0pt}one{\isasymrangle}{\isacharparenright}{\kern0pt}\ {\isasymOtimes}\ {\isacharbar}{\kern0pt}one{\isasymrangle}{\isachardoublequoteclose}\isanewline
\ \ \ \ \ \ \ \ \isacommand{using}\isamarkupfalse%
\ R{\isacharunderscore}{\kern0pt}action\isanewline
\ \ \ \ \ \ \ \ \isacommand{by}\isamarkupfalse%
\ {\isacharparenleft}{\kern0pt}metis\ assms\ jm{\isadigit{1}}{\isacharparenright}{\kern0pt}\isanewline
\ \ \ \ \ \ \isacommand{finally}\isamarkupfalse%
\ \isacommand{show}\isamarkupfalse%
\ {\isachardoublequoteopen}{\isacharparenleft}{\kern0pt}{\isadigit{1}}\isactrlsub m\ {\isadigit{2}}\ {\isasymOtimes}\ SWAP{\isacharunderscore}{\kern0pt}down\ {\isacharparenleft}{\kern0pt}Suc\ n{\isacharparenright}{\kern0pt}{\isacharparenright}{\kern0pt}\ {\isacharasterisk}{\kern0pt}\ {\isacharparenleft}{\kern0pt}control{\isadigit{2}}\ {\isacharparenleft}{\kern0pt}R\ {\isacharparenleft}{\kern0pt}Suc\ {\isacharparenleft}{\kern0pt}Suc\ n{\isacharparenright}{\kern0pt}{\isacharparenright}{\kern0pt}{\isacharparenright}{\kern0pt}\ {\isacharasterisk}{\kern0pt}\ {\isacharparenleft}{\kern0pt}\ {\isacharbar}{\kern0pt}Deutsch{\isachardot}{\kern0pt}zero{\isasymrangle}\ {\isacharplus}{\kern0pt}\isanewline
\ \ \ \ \ \ \ \ \ \ \ \ \ \ \ \ \ \ \ \ exp\ {\isacharparenleft}{\kern0pt}{\isadigit{2}}\ {\isacharasterisk}{\kern0pt}\ {\isasymi}\ {\isacharasterisk}{\kern0pt}\ complex{\isacharunderscore}{\kern0pt}of{\isacharunderscore}{\kern0pt}real\ pi\ {\isacharasterisk}{\kern0pt}\ complex{\isacharunderscore}{\kern0pt}of{\isacharunderscore}{\kern0pt}nat\ {\isacharparenleft}{\kern0pt}j\ div\ {\isadigit{2}}{\isacharparenright}{\kern0pt}\ {\isacharslash}{\kern0pt}\ {\isadigit{2}}\ {\isacharcircum}{\kern0pt}\ Suc\ n{\isacharparenright}{\kern0pt}\ {\isasymcdot}\isactrlsub m\isanewline
\ \ \ \ \ \ \ \ \ \ \ \ \ \ \ \ \ \ \ \ {\isacharbar}{\kern0pt}Deutsch{\isachardot}{\kern0pt}one{\isasymrangle}\ {\isasymOtimes}\ {\isacharbar}{\kern0pt}state{\isacharunderscore}{\kern0pt}basis\ {\isadigit{1}}\ {\isacharparenleft}{\kern0pt}j\ mod\ {\isadigit{2}}{\isacharparenright}{\kern0pt}{\isasymrangle}{\isacharparenright}{\kern0pt}\ {\isasymOtimes}\ {\isadigit{1}}\isactrlsub m\ {\isacharparenleft}{\kern0pt}{\isadigit{2}}\ {\isacharcircum}{\kern0pt}\ n{\isacharparenright}{\kern0pt}\ {\isacharasterisk}{\kern0pt}\isanewline
\ \ \ \ \ \ \ \ \ \ \ \ \ \ \ \ \ \ \ \ {\isacharbar}{\kern0pt}state{\isacharunderscore}{\kern0pt}basis\ n\ {\isacharparenleft}{\kern0pt}j\ mod\ {\isadigit{2}}\ {\isacharcircum}{\kern0pt}\ Suc\ n\ div\ {\isadigit{2}}{\isacharparenright}{\kern0pt}{\isasymrangle}{\isacharparenright}{\kern0pt}\ {\isacharequal}{\kern0pt}\isanewline
\ \ \ \ \ \ \ \ \ \ \ \ \ \ \ \ \ \ \ \ {\isacharparenleft}{\kern0pt}{\isadigit{1}}\isactrlsub m\ {\isadigit{2}}\ {\isasymOtimes}\ SWAP{\isacharunderscore}{\kern0pt}down\ {\isacharparenleft}{\kern0pt}Suc\ n{\isacharparenright}{\kern0pt}{\isacharparenright}{\kern0pt}\ {\isacharasterisk}{\kern0pt}\ {\isacharparenleft}{\kern0pt}\ {\isacharbar}{\kern0pt}Deutsch{\isachardot}{\kern0pt}zero{\isasymrangle}\ {\isacharplus}{\kern0pt}\ exp\ {\isacharparenleft}{\kern0pt}{\isadigit{2}}\ {\isacharasterisk}{\kern0pt}\ {\isasymi}\ {\isacharasterisk}{\kern0pt}\ \isanewline
\ \ \ \ \ \ \ \ \ \ \ \ \ \ \ \ \ \ \ \ complex{\isacharunderscore}{\kern0pt}of{\isacharunderscore}{\kern0pt}real\ pi\ {\isacharasterisk}{\kern0pt}\ complex{\isacharunderscore}{\kern0pt}of{\isacharunderscore}{\kern0pt}nat\ j\ {\isacharslash}{\kern0pt}\ {\isadigit{2}}\ {\isacharcircum}{\kern0pt}\ Suc\ {\isacharparenleft}{\kern0pt}Suc\ n{\isacharparenright}{\kern0pt}{\isacharparenright}{\kern0pt}\ {\isasymcdot}\isactrlsub m\ {\isacharbar}{\kern0pt}Deutsch{\isachardot}{\kern0pt}one{\isasymrangle}\ {\isasymOtimes}\isanewline
\ \ \ \ \ \ \ \ \ \ \ \ \ \ \ \ \ {\isacharbar}{\kern0pt}state{\isacharunderscore}{\kern0pt}basis\ {\isadigit{1}}\ {\isacharparenleft}{\kern0pt}j\ mod\ {\isadigit{2}}{\isacharparenright}{\kern0pt}{\isasymrangle}\ {\isasymOtimes}\ {\isadigit{1}}\isactrlsub m\ {\isacharparenleft}{\kern0pt}{\isadigit{2}}\ {\isacharcircum}{\kern0pt}\ n{\isacharparenright}{\kern0pt}\ {\isacharasterisk}{\kern0pt}\ {\isacharbar}{\kern0pt}state{\isacharunderscore}{\kern0pt}basis\ n\ {\isacharparenleft}{\kern0pt}j\ mod\ {\isadigit{2}}\ {\isacharcircum}{\kern0pt}\ Suc\ n\ div\ {\isadigit{2}}{\isacharparenright}{\kern0pt}{\isasymrangle}{\isacharparenright}{\kern0pt}{\isachardoublequoteclose}\ \isanewline
\ \ \ \ \ \ \ \ \isacommand{by}\isamarkupfalse%
\ {\isacharparenleft}{\kern0pt}metis\ jm{\isadigit{1}}\ power{\isacharunderscore}{\kern0pt}one{\isacharunderscore}{\kern0pt}right\ state{\isacharunderscore}{\kern0pt}basis{\isacharunderscore}{\kern0pt}def{\isacharparenright}{\kern0pt}\isanewline
\ \ \ \ \isacommand{qed}\isamarkupfalse%
\isanewline
\ \ \ \ \isacommand{also}\isamarkupfalse%
\ \isacommand{have}\isamarkupfalse%
\ {\isachardoublequoteopen}{\isasymdots}\ {\isacharequal}{\kern0pt}\ {\isacharparenleft}{\kern0pt}{\isacharparenleft}{\kern0pt}{\isadigit{1}}\isactrlsub m\ {\isadigit{2}}{\isacharparenright}{\kern0pt}\ {\isasymOtimes}\ SWAP{\isacharunderscore}{\kern0pt}down\ {\isacharparenleft}{\kern0pt}Suc\ n{\isacharparenright}{\kern0pt}{\isacharparenright}{\kern0pt}\ {\isacharasterisk}{\kern0pt}\ \isanewline
\ \ \ \ \ \ \ \ \ \ \ \ \ \ \ \ \ \ \ \ {\isacharparenleft}{\kern0pt}{\isacharparenleft}{\kern0pt}\ {\isacharbar}{\kern0pt}zero{\isasymrangle}\ {\isacharplus}{\kern0pt}\ exp\ {\isacharparenleft}{\kern0pt}{\isadigit{2}}\ {\isacharasterisk}{\kern0pt}\ {\isasymi}\ {\isacharasterisk}{\kern0pt}\ pi\ {\isacharasterisk}{\kern0pt}\ complex{\isacharunderscore}{\kern0pt}of{\isacharunderscore}{\kern0pt}nat\ j\ {\isacharslash}{\kern0pt}\ {\isadigit{2}}\ {\isacharcircum}{\kern0pt}\ Suc\ {\isacharparenleft}{\kern0pt}Suc\ n{\isacharparenright}{\kern0pt}{\isacharparenright}{\kern0pt}\ {\isasymcdot}\isactrlsub m\ {\isacharbar}{\kern0pt}one{\isasymrangle}{\isacharparenright}{\kern0pt}\ {\isasymOtimes}\isanewline
\ \ \ \ \ \ \ \ \ \ \ \ \ \ \ \ \ \ {\isacharparenleft}{\kern0pt}\ {\isacharbar}{\kern0pt}state{\isacharunderscore}{\kern0pt}basis\ {\isadigit{1}}\ {\isacharparenleft}{\kern0pt}j\ mod\ {\isadigit{2}}{\isacharparenright}{\kern0pt}{\isasymrangle}\ {\isasymOtimes}\ {\isacharparenleft}{\kern0pt}{\isacharparenleft}{\kern0pt}{\isadigit{1}}\isactrlsub m\ {\isacharparenleft}{\kern0pt}{\isadigit{2}}{\isacharcircum}{\kern0pt}n{\isacharparenright}{\kern0pt}{\isacharparenright}{\kern0pt}\ {\isacharasterisk}{\kern0pt}\ \isanewline
\ \ \ \ \ \ \ \ \ \ \ \ \ \ \ \ \ \ \ \ {\isacharbar}{\kern0pt}state{\isacharunderscore}{\kern0pt}basis\ n\ {\isacharparenleft}{\kern0pt}j\ mod\ {\isadigit{2}}\ {\isacharcircum}{\kern0pt}\ Suc\ n\ div\ {\isadigit{2}}{\isacharparenright}{\kern0pt}{\isasymrangle}{\isacharparenright}{\kern0pt}{\isacharparenright}{\kern0pt}{\isacharparenright}{\kern0pt}{\isachardoublequoteclose}\isanewline
\ \ \ \ \ \ \isacommand{using}\isamarkupfalse%
\ tensor{\isacharunderscore}{\kern0pt}mat{\isacharunderscore}{\kern0pt}is{\isacharunderscore}{\kern0pt}assoc\ ket{\isacharunderscore}{\kern0pt}vec{\isacharunderscore}{\kern0pt}def\ \isacommand{by}\isamarkupfalse%
\ auto\isanewline
\ \ \ \ \isacommand{also}\isamarkupfalse%
\ \isacommand{have}\isamarkupfalse%
\ {\isachardoublequoteopen}{\isasymdots}\ {\isacharequal}{\kern0pt}\ {\isacharparenleft}{\kern0pt}\ {\isacharbar}{\kern0pt}zero{\isasymrangle}\ {\isacharplus}{\kern0pt}\ exp\ {\isacharparenleft}{\kern0pt}{\isadigit{2}}\ {\isacharasterisk}{\kern0pt}\ {\isasymi}\ {\isacharasterisk}{\kern0pt}\ pi\ {\isacharasterisk}{\kern0pt}\ complex{\isacharunderscore}{\kern0pt}of{\isacharunderscore}{\kern0pt}nat\ j\ {\isacharslash}{\kern0pt}\ {\isadigit{2}}\ {\isacharcircum}{\kern0pt}\ Suc\ {\isacharparenleft}{\kern0pt}Suc\ n{\isacharparenright}{\kern0pt}{\isacharparenright}{\kern0pt}\ {\isasymcdot}\isactrlsub m\ {\isacharbar}{\kern0pt}one{\isasymrangle}{\isacharparenright}{\kern0pt}\ {\isasymOtimes}\isanewline
\ \ \ \ \ \ \ \ \ \ \ \ \ \ \ \ \ \ \ \ {\isacharparenleft}{\kern0pt}{\isacharparenleft}{\kern0pt}SWAP{\isacharunderscore}{\kern0pt}down\ {\isacharparenleft}{\kern0pt}Suc\ n{\isacharparenright}{\kern0pt}{\isacharparenright}{\kern0pt}\ {\isacharasterisk}{\kern0pt}\ {\isacharparenleft}{\kern0pt}\ {\isacharbar}{\kern0pt}state{\isacharunderscore}{\kern0pt}basis\ {\isadigit{1}}\ {\isacharparenleft}{\kern0pt}j\ mod\ {\isadigit{2}}{\isacharparenright}{\kern0pt}{\isasymrangle}\ {\isasymOtimes}\ {\isacharparenleft}{\kern0pt}{\isacharparenleft}{\kern0pt}{\isadigit{1}}\isactrlsub m\ {\isacharparenleft}{\kern0pt}{\isadigit{2}}{\isacharcircum}{\kern0pt}n{\isacharparenright}{\kern0pt}{\isacharparenright}{\kern0pt}\ {\isacharasterisk}{\kern0pt}\ \isanewline
\ \ \ \ \ \ \ \ \ \ \ \ \ \ \ \ \ \ \ \ {\isacharbar}{\kern0pt}state{\isacharunderscore}{\kern0pt}basis\ n\ {\isacharparenleft}{\kern0pt}j\ mod\ {\isadigit{2}}\ {\isacharcircum}{\kern0pt}\ Suc\ n\ div\ {\isadigit{2}}{\isacharparenright}{\kern0pt}{\isasymrangle}{\isacharparenright}{\kern0pt}{\isacharparenright}{\kern0pt}{\isacharparenright}{\kern0pt}{\isachardoublequoteclose}\isanewline
\ \ \ \ \ \ \isacommand{using}\isamarkupfalse%
\ mult{\isacharunderscore}{\kern0pt}distr{\isacharunderscore}{\kern0pt}tensor\isanewline
\ \ \ \ \ \ \isacommand{by}\isamarkupfalse%
\ {\isacharparenleft}{\kern0pt}smt\ {\isacharparenleft}{\kern0pt}verit{\isacharcomma}{\kern0pt}\ del{\isacharunderscore}{\kern0pt}insts{\isacharparenright}{\kern0pt}\ SWAP{\isacharunderscore}{\kern0pt}down{\isacharunderscore}{\kern0pt}carrier{\isacharunderscore}{\kern0pt}mat\ carrier{\isacharunderscore}{\kern0pt}matD{\isacharparenleft}{\kern0pt}{\isadigit{1}}{\isacharparenright}{\kern0pt}\ carrier{\isacharunderscore}{\kern0pt}matD{\isacharparenleft}{\kern0pt}{\isadigit{2}}{\isacharparenright}{\kern0pt}\ \isanewline
\ \ \ \ \ \ \ \ \ \ dim{\isacharunderscore}{\kern0pt}col{\isacharunderscore}{\kern0pt}tensor{\isacharunderscore}{\kern0pt}mat\ dim{\isacharunderscore}{\kern0pt}row{\isacharunderscore}{\kern0pt}tensor{\isacharunderscore}{\kern0pt}mat\ index{\isacharunderscore}{\kern0pt}add{\isacharunderscore}{\kern0pt}mat{\isacharparenleft}{\kern0pt}{\isadigit{2}}{\isacharparenright}{\kern0pt}\ index{\isacharunderscore}{\kern0pt}add{\isacharunderscore}{\kern0pt}mat{\isacharparenleft}{\kern0pt}{\isadigit{3}}{\isacharparenright}{\kern0pt}\ index{\isacharunderscore}{\kern0pt}one{\isacharunderscore}{\kern0pt}mat{\isacharparenleft}{\kern0pt}{\isadigit{3}}{\isacharparenright}{\kern0pt}\isanewline
\ \ \ \ \ \ \ \ \ \ index{\isacharunderscore}{\kern0pt}smult{\isacharunderscore}{\kern0pt}mat{\isacharparenleft}{\kern0pt}{\isadigit{2}}{\isacharparenright}{\kern0pt}\ index{\isacharunderscore}{\kern0pt}smult{\isacharunderscore}{\kern0pt}mat{\isacharparenleft}{\kern0pt}{\isadigit{3}}{\isacharparenright}{\kern0pt}\ ket{\isacharunderscore}{\kern0pt}one{\isacharunderscore}{\kern0pt}is{\isacharunderscore}{\kern0pt}state\ left{\isacharunderscore}{\kern0pt}mult{\isacharunderscore}{\kern0pt}one{\isacharunderscore}{\kern0pt}mat{\isacharprime}{\kern0pt}\ one{\isacharunderscore}{\kern0pt}power{\isadigit{2}}\ pos{\isadigit{2}}\isanewline
\ \ \ \ \ \ \ \ \ \ power{\isachardot}{\kern0pt}simps{\isacharparenleft}{\kern0pt}{\isadigit{2}}{\isacharparenright}{\kern0pt}\ power{\isacharunderscore}{\kern0pt}one{\isacharunderscore}{\kern0pt}right\ state{\isacharunderscore}{\kern0pt}basis{\isacharunderscore}{\kern0pt}carrier{\isacharunderscore}{\kern0pt}mat\ state{\isacharunderscore}{\kern0pt}def\ \isanewline
\ \ \ \ \ \ \ \ \ \ zero{\isacharunderscore}{\kern0pt}less{\isacharunderscore}{\kern0pt}one{\isacharunderscore}{\kern0pt}class{\isachardot}{\kern0pt}zero{\isacharunderscore}{\kern0pt}less{\isacharunderscore}{\kern0pt}one\ zero{\isacharunderscore}{\kern0pt}less{\isacharunderscore}{\kern0pt}power{\isacharparenright}{\kern0pt}\isanewline
\ \ \ \ \isacommand{also}\isamarkupfalse%
\ \isacommand{have}\isamarkupfalse%
\ {\isachardoublequoteopen}{\isasymdots}\ {\isacharequal}{\kern0pt}\ {\isacharparenleft}{\kern0pt}\ {\isacharbar}{\kern0pt}zero{\isasymrangle}\ {\isacharplus}{\kern0pt}\ exp\ {\isacharparenleft}{\kern0pt}{\isadigit{2}}\ {\isacharasterisk}{\kern0pt}\ {\isasymi}\ {\isacharasterisk}{\kern0pt}\ pi\ {\isacharasterisk}{\kern0pt}\ complex{\isacharunderscore}{\kern0pt}of{\isacharunderscore}{\kern0pt}nat\ j\ {\isacharslash}{\kern0pt}\ {\isadigit{2}}\ {\isacharcircum}{\kern0pt}\ Suc\ {\isacharparenleft}{\kern0pt}Suc\ n{\isacharparenright}{\kern0pt}{\isacharparenright}{\kern0pt}\ {\isasymcdot}\isactrlsub m\ {\isacharbar}{\kern0pt}one{\isasymrangle}{\isacharparenright}{\kern0pt}\ {\isasymOtimes}\isanewline
\ \ \ \ \ \ \ \ \ \ \ \ \ \ \ \ \ \ \ \ {\isacharparenleft}{\kern0pt}\ {\isacharbar}{\kern0pt}state{\isacharunderscore}{\kern0pt}basis\ n\ {\isacharparenleft}{\kern0pt}j\ mod\ {\isadigit{2}}\ {\isacharcircum}{\kern0pt}\ Suc\ n\ div\ {\isadigit{2}}{\isacharparenright}{\kern0pt}{\isasymrangle}\ {\isasymOtimes}\ {\isacharbar}{\kern0pt}state{\isacharunderscore}{\kern0pt}basis\ {\isadigit{1}}\ {\isacharparenleft}{\kern0pt}j\ mod\ {\isadigit{2}}{\isacharparenright}{\kern0pt}{\isasymrangle}{\isacharparenright}{\kern0pt}{\isachardoublequoteclose}\isanewline
\ \ \ \ \ \ \isacommand{using}\isamarkupfalse%
\ SWAP{\isacharunderscore}{\kern0pt}down{\isacharunderscore}{\kern0pt}action\ jeq\ \isanewline
\ \ \ \ \ \ \isacommand{by}\isamarkupfalse%
\ {\isacharparenleft}{\kern0pt}metis\ Suc\ dim{\isacharunderscore}{\kern0pt}row{\isacharunderscore}{\kern0pt}mat{\isacharparenleft}{\kern0pt}{\isadigit{1}}{\isacharparenright}{\kern0pt}\ index{\isacharunderscore}{\kern0pt}unit{\isacharunderscore}{\kern0pt}vec{\isacharparenleft}{\kern0pt}{\isadigit{3}}{\isacharparenright}{\kern0pt}\ jm{\isadigit{2}}sn{\isacharunderscore}{\kern0pt}def\ ket{\isacharunderscore}{\kern0pt}vec{\isacharunderscore}{\kern0pt}def\ left{\isacharunderscore}{\kern0pt}mult{\isacharunderscore}{\kern0pt}one{\isacharunderscore}{\kern0pt}mat{\isacharprime}{\kern0pt}\ \isanewline
\ \ \ \ \ \ \ \ \ \ mod{\isacharunderscore}{\kern0pt}less{\isacharunderscore}{\kern0pt}divisor\ pos{\isadigit{2}}\ state{\isacharunderscore}{\kern0pt}basis{\isacharunderscore}{\kern0pt}def\ zero{\isacharunderscore}{\kern0pt}less{\isacharunderscore}{\kern0pt}power{\isacharparenright}{\kern0pt}\isanewline
\ \ \ \ \isacommand{finally}\isamarkupfalse%
\ \isacommand{show}\isamarkupfalse%
\ {\isachardoublequoteopen}control\ {\isacharparenleft}{\kern0pt}Suc\ {\isacharparenleft}{\kern0pt}Suc\ n{\isacharparenright}{\kern0pt}{\isacharparenright}{\kern0pt}\ {\isacharparenleft}{\kern0pt}R\ {\isacharparenleft}{\kern0pt}Suc\ {\isacharparenleft}{\kern0pt}Suc\ n{\isacharparenright}{\kern0pt}{\isacharparenright}{\kern0pt}{\isacharparenright}{\kern0pt}\ {\isacharasterisk}{\kern0pt}\ {\isacharparenleft}{\kern0pt}\ {\isacharbar}{\kern0pt}Deutsch{\isachardot}{\kern0pt}zero{\isasymrangle}\ {\isacharplus}{\kern0pt}\ exp\ {\isacharparenleft}{\kern0pt}{\isadigit{2}}\ {\isacharasterisk}{\kern0pt}\ {\isasymi}\ {\isacharasterisk}{\kern0pt}\isanewline
\ \ \ \ \ \ \ \ \ \ \ \ \ \ \ \ \ \ complex{\isacharunderscore}{\kern0pt}of{\isacharunderscore}{\kern0pt}real\ pi\ {\isacharasterisk}{\kern0pt}\ complex{\isacharunderscore}{\kern0pt}of{\isacharunderscore}{\kern0pt}nat\ {\isacharparenleft}{\kern0pt}j\ div\ {\isadigit{2}}{\isacharparenright}{\kern0pt}\ {\isacharslash}{\kern0pt}\ {\isadigit{2}}\ {\isacharcircum}{\kern0pt}\ Suc\ n{\isacharparenright}{\kern0pt}\ {\isasymcdot}\isactrlsub m\ {\isacharbar}{\kern0pt}Deutsch{\isachardot}{\kern0pt}one{\isasymrangle}\ {\isasymOtimes}\isanewline
\ \ \ \ \ \ \ \ \ \ \ \ \ \ \ \ \ \ {\isacharbar}{\kern0pt}state{\isacharunderscore}{\kern0pt}basis\ n\ {\isacharparenleft}{\kern0pt}j\ mod\ {\isadigit{2}}\ {\isacharcircum}{\kern0pt}\ Suc\ n\ div\ {\isadigit{2}}{\isacharparenright}{\kern0pt}{\isasymrangle}\ {\isasymOtimes}\ {\isacharbar}{\kern0pt}state{\isacharunderscore}{\kern0pt}basis\ {\isadigit{1}}\ {\isacharparenleft}{\kern0pt}j\ mod\ {\isadigit{2}}{\isacharparenright}{\kern0pt}{\isasymrangle}{\isacharparenright}{\kern0pt}\ {\isacharequal}{\kern0pt}\isanewline
\ \ \ \ \ \ \ \ \ \ \ \ \ \ \ \ \ \ {\isacharbar}{\kern0pt}Deutsch{\isachardot}{\kern0pt}zero{\isasymrangle}\ {\isacharplus}{\kern0pt}\ exp\ {\isacharparenleft}{\kern0pt}{\isadigit{2}}\ {\isacharasterisk}{\kern0pt}\ {\isasymi}\ {\isacharasterisk}{\kern0pt}\ complex{\isacharunderscore}{\kern0pt}of{\isacharunderscore}{\kern0pt}real\ pi\ {\isacharasterisk}{\kern0pt}\ complex{\isacharunderscore}{\kern0pt}of{\isacharunderscore}{\kern0pt}nat\ j\ {\isacharslash}{\kern0pt}\isanewline
\ \ \ \ \ \ \ \ \ \ \ \ \ \ \ \ \ \ {\isadigit{2}}\ {\isacharcircum}{\kern0pt}\ Suc\ {\isacharparenleft}{\kern0pt}Suc\ n{\isacharparenright}{\kern0pt}{\isacharparenright}{\kern0pt}\ {\isasymcdot}\isactrlsub m\ {\isacharbar}{\kern0pt}Deutsch{\isachardot}{\kern0pt}one{\isasymrangle}\ {\isasymOtimes}\ {\isacharbar}{\kern0pt}state{\isacharunderscore}{\kern0pt}basis\ n\ {\isacharparenleft}{\kern0pt}j\ mod\ {\isadigit{2}}\ {\isacharcircum}{\kern0pt}\ Suc\ n\ div\ {\isadigit{2}}{\isacharparenright}{\kern0pt}{\isasymrangle}\ {\isasymOtimes}\isanewline
\ \ \ \ \ \ \ \ \ \ \ \ \ \ \ \ \ \ {\isacharbar}{\kern0pt}state{\isacharunderscore}{\kern0pt}basis\ {\isadigit{1}}\ {\isacharparenleft}{\kern0pt}j\ mod\ {\isadigit{2}}{\isacharparenright}{\kern0pt}{\isasymrangle}{\isachardoublequoteclose}\isanewline
\ \ \ \ \ \ \isacommand{using}\isamarkupfalse%
\ tensor{\isacharunderscore}{\kern0pt}mat{\isacharunderscore}{\kern0pt}is{\isacharunderscore}{\kern0pt}assoc\ ket{\isacharunderscore}{\kern0pt}vec{\isacharunderscore}{\kern0pt}def\ \isacommand{by}\isamarkupfalse%
\ auto\isanewline
\ \ \isacommand{qed}\isamarkupfalse%
\isanewline
\isacommand{qed}\isamarkupfalse%
%
\endisatagproof
{\isafoldproof}%
%
\isadelimproof
%
\endisadelimproof
%
\begin{isamarkuptext}%
Action of the controlled rotations subcircuit%
\end{isamarkuptext}\isamarkuptrue%
\isacommand{lemma}\isamarkupfalse%
\ controlled{\isacharunderscore}{\kern0pt}rotations{\isacharunderscore}{\kern0pt}ind{\isacharcolon}{\kern0pt}\isanewline
\ \ {\isachardoublequoteopen}{\isasymforall}j{\isachardot}{\kern0pt}\ j\ {\isacharless}{\kern0pt}\ {\isadigit{2}}\ {\isacharcircum}{\kern0pt}\ Suc\ n\ {\isasymlongrightarrow}\ \isanewline
\ \ controlled{\isacharunderscore}{\kern0pt}rotations\ {\isacharparenleft}{\kern0pt}Suc\ n{\isacharparenright}{\kern0pt}\ {\isacharasterisk}{\kern0pt}\ \isanewline
\ \ {\isacharparenleft}{\kern0pt}{\isacharparenleft}{\kern0pt}\ {\isacharbar}{\kern0pt}zero{\isasymrangle}\ {\isacharplus}{\kern0pt}\ exp{\isacharparenleft}{\kern0pt}{\isadigit{2}}{\isacharasterisk}{\kern0pt}{\isasymi}{\isacharasterisk}{\kern0pt}pi{\isacharasterisk}{\kern0pt}{\isacharparenleft}{\kern0pt}complex{\isacharunderscore}{\kern0pt}of{\isacharunderscore}{\kern0pt}nat\ {\isacharparenleft}{\kern0pt}j\ div\ {\isadigit{2}}{\isacharcircum}{\kern0pt}n{\isacharparenright}{\kern0pt}{\isacharparenright}{\kern0pt}{\isacharslash}{\kern0pt}{\isadigit{2}}{\isacharparenright}{\kern0pt}\ {\isasymcdot}\isactrlsub m\ {\isacharbar}{\kern0pt}one{\isasymrangle}{\isacharparenright}{\kern0pt}\ {\isasymOtimes}\ {\isacharbar}{\kern0pt}state{\isacharunderscore}{\kern0pt}basis\ n\ {\isacharparenleft}{\kern0pt}j\ mod\ {\isadigit{2}}{\isacharcircum}{\kern0pt}n{\isacharparenright}{\kern0pt}{\isasymrangle}{\isacharparenright}{\kern0pt}\ {\isacharequal}{\kern0pt}\isanewline
\ \ {\isacharparenleft}{\kern0pt}\ {\isacharbar}{\kern0pt}zero{\isasymrangle}\ {\isacharplus}{\kern0pt}\ exp{\isacharparenleft}{\kern0pt}{\isadigit{2}}{\isacharasterisk}{\kern0pt}{\isasymi}{\isacharasterisk}{\kern0pt}pi{\isacharasterisk}{\kern0pt}j{\isacharslash}{\kern0pt}{\isacharparenleft}{\kern0pt}{\isadigit{2}}{\isacharcircum}{\kern0pt}{\isacharparenleft}{\kern0pt}Suc\ n{\isacharparenright}{\kern0pt}{\isacharparenright}{\kern0pt}{\isacharparenright}{\kern0pt}\ {\isasymcdot}\isactrlsub m\ {\isacharbar}{\kern0pt}one{\isasymrangle}{\isacharparenright}{\kern0pt}\ {\isasymOtimes}\ {\isacharbar}{\kern0pt}state{\isacharunderscore}{\kern0pt}basis\ n\ {\isacharparenleft}{\kern0pt}j\ mod\ {\isadigit{2}}{\isacharcircum}{\kern0pt}n{\isacharparenright}{\kern0pt}{\isasymrangle}{\isachardoublequoteclose}\ \isanewline
%
\isadelimproof
%
\endisadelimproof
%
\isatagproof
\isacommand{proof}\isamarkupfalse%
\ {\isacharparenleft}{\kern0pt}induct\ n{\isacharparenright}{\kern0pt}\isanewline
\ \ \isacommand{case}\isamarkupfalse%
\ {\isadigit{0}}\isanewline
\ \ \isacommand{then}\isamarkupfalse%
\ \isacommand{show}\isamarkupfalse%
\ {\isacharquery}{\kern0pt}case\isanewline
\ \ \isacommand{proof}\isamarkupfalse%
\ {\isacharparenleft}{\kern0pt}rule\ allI{\isacharparenright}{\kern0pt}\isanewline
\ \ \ \ \isacommand{fix}\isamarkupfalse%
\ j{\isacharcolon}{\kern0pt}{\isacharcolon}{\kern0pt}nat\isanewline
\ \ \ \ \isacommand{show}\isamarkupfalse%
\ {\isachardoublequoteopen}j\ {\isacharless}{\kern0pt}\ {\isadigit{2}}\ {\isacharcircum}{\kern0pt}\ Suc\ {\isadigit{0}}\ {\isasymlongrightarrow}\isanewline
\ \ \ \ \ \ \ \ \ controlled{\isacharunderscore}{\kern0pt}rotations\ {\isacharparenleft}{\kern0pt}Suc\ {\isadigit{0}}{\isacharparenright}{\kern0pt}\ {\isacharasterisk}{\kern0pt}\ {\isacharparenleft}{\kern0pt}\ {\isacharbar}{\kern0pt}zero{\isasymrangle}\ {\isacharplus}{\kern0pt}\isanewline
\ \ \ \ \ \ \ \ \ \ exp\ {\isacharparenleft}{\kern0pt}{\isadigit{2}}\ {\isacharasterisk}{\kern0pt}\ {\isasymi}\ {\isacharasterisk}{\kern0pt}\ complex{\isacharunderscore}{\kern0pt}of{\isacharunderscore}{\kern0pt}real\ pi\ {\isacharasterisk}{\kern0pt}\ complex{\isacharunderscore}{\kern0pt}of{\isacharunderscore}{\kern0pt}nat\ {\isacharparenleft}{\kern0pt}j\ div\ {\isadigit{2}}\ {\isacharcircum}{\kern0pt}\ {\isadigit{0}}{\isacharparenright}{\kern0pt}\ {\isacharslash}{\kern0pt}\ {\isadigit{2}}{\isacharparenright}{\kern0pt}\ {\isasymcdot}\isactrlsub m\ {\isacharbar}{\kern0pt}one{\isasymrangle}\ {\isasymOtimes}\isanewline
\ \ \ \ \ \ \ \ \ \ {\isacharbar}{\kern0pt}state{\isacharunderscore}{\kern0pt}basis\ {\isadigit{0}}\ {\isacharparenleft}{\kern0pt}j\ mod\ {\isadigit{2}}\ {\isacharcircum}{\kern0pt}\ {\isadigit{0}}{\isacharparenright}{\kern0pt}{\isasymrangle}{\isacharparenright}{\kern0pt}\ {\isacharequal}{\kern0pt}\isanewline
\ \ \ \ \ \ \ \ \ {\isacharbar}{\kern0pt}zero{\isasymrangle}\ {\isacharplus}{\kern0pt}\ exp\ {\isacharparenleft}{\kern0pt}{\isadigit{2}}\ {\isacharasterisk}{\kern0pt}\ {\isasymi}\ {\isacharasterisk}{\kern0pt}\ complex{\isacharunderscore}{\kern0pt}of{\isacharunderscore}{\kern0pt}real\ pi\ {\isacharasterisk}{\kern0pt}\ complex{\isacharunderscore}{\kern0pt}of{\isacharunderscore}{\kern0pt}nat\ j\ {\isacharslash}{\kern0pt}\ {\isadigit{2}}\ {\isacharcircum}{\kern0pt}\ Suc\ {\isadigit{0}}{\isacharparenright}{\kern0pt}\ {\isasymcdot}\isactrlsub m\ {\isacharbar}{\kern0pt}one{\isasymrangle}\ {\isasymOtimes}\isanewline
\ \ \ \ \ \ \ \ \ {\isacharbar}{\kern0pt}state{\isacharunderscore}{\kern0pt}basis\ {\isadigit{0}}\ {\isacharparenleft}{\kern0pt}j\ mod\ {\isadigit{2}}\ {\isacharcircum}{\kern0pt}\ {\isadigit{0}}{\isacharparenright}{\kern0pt}{\isasymrangle}{\isachardoublequoteclose}\isanewline
\ \ \ \ \isacommand{proof}\isamarkupfalse%
\isanewline
\ \ \ \ \ \ \isacommand{assume}\isamarkupfalse%
\ {\isachardoublequoteopen}j\ {\isacharless}{\kern0pt}\ {\isadigit{2}}\ {\isacharcircum}{\kern0pt}\ Suc\ {\isadigit{0}}{\isachardoublequoteclose}\isanewline
\ \ \ \ \ \ \isacommand{hence}\isamarkupfalse%
\ j{\isadigit{2}}{\isacharcolon}{\kern0pt}{\isachardoublequoteopen}j\ {\isacharless}{\kern0pt}\ {\isadigit{2}}{\isachardoublequoteclose}\ \isacommand{by}\isamarkupfalse%
\ auto\isanewline
\ \ \ \ \ \ \isacommand{have}\isamarkupfalse%
\ {\isachardoublequoteopen}controlled{\isacharunderscore}{\kern0pt}rotations\ {\isacharparenleft}{\kern0pt}Suc\ {\isadigit{0}}{\isacharparenright}{\kern0pt}\ {\isacharasterisk}{\kern0pt}\ {\isacharparenleft}{\kern0pt}\ {\isacharbar}{\kern0pt}zero{\isasymrangle}\ {\isacharplus}{\kern0pt}\isanewline
\ \ \ \ \ \ \ \ \ \ \ \ exp\ {\isacharparenleft}{\kern0pt}{\isadigit{2}}\ {\isacharasterisk}{\kern0pt}\ {\isasymi}\ {\isacharasterisk}{\kern0pt}\ complex{\isacharunderscore}{\kern0pt}of{\isacharunderscore}{\kern0pt}real\ pi\ {\isacharasterisk}{\kern0pt}\ complex{\isacharunderscore}{\kern0pt}of{\isacharunderscore}{\kern0pt}nat\ {\isacharparenleft}{\kern0pt}j\ div\ {\isadigit{2}}\ {\isacharcircum}{\kern0pt}\ {\isadigit{0}}{\isacharparenright}{\kern0pt}\ {\isacharslash}{\kern0pt}\ {\isadigit{2}}{\isacharparenright}{\kern0pt}\ {\isasymcdot}\isactrlsub m\ {\isacharbar}{\kern0pt}one{\isasymrangle}\ {\isasymOtimes}\isanewline
\ \ \ \ \ \ \ \ \ \ \ \ {\isacharbar}{\kern0pt}state{\isacharunderscore}{\kern0pt}basis\ {\isadigit{0}}\ {\isacharparenleft}{\kern0pt}j\ mod\ {\isadigit{2}}\ {\isacharcircum}{\kern0pt}\ {\isadigit{0}}{\isacharparenright}{\kern0pt}{\isasymrangle}{\isacharparenright}{\kern0pt}\ {\isacharequal}{\kern0pt}\ \isanewline
\ \ \ \ \ \ \ \ \ \ \ \ {\isacharparenleft}{\kern0pt}{\isadigit{1}}\isactrlsub m\ {\isadigit{2}}{\isacharparenright}{\kern0pt}\ \ {\isacharasterisk}{\kern0pt}\ {\isacharparenleft}{\kern0pt}\ {\isacharbar}{\kern0pt}zero{\isasymrangle}\ {\isacharplus}{\kern0pt}\isanewline
\ \ \ \ \ \ \ \ \ \ \ \ exp\ {\isacharparenleft}{\kern0pt}{\isadigit{2}}\ {\isacharasterisk}{\kern0pt}\ {\isasymi}\ {\isacharasterisk}{\kern0pt}\ complex{\isacharunderscore}{\kern0pt}of{\isacharunderscore}{\kern0pt}real\ pi\ {\isacharasterisk}{\kern0pt}\ complex{\isacharunderscore}{\kern0pt}of{\isacharunderscore}{\kern0pt}nat\ {\isacharparenleft}{\kern0pt}j\ div\ {\isadigit{2}}\ {\isacharcircum}{\kern0pt}\ {\isadigit{0}}{\isacharparenright}{\kern0pt}\ {\isacharslash}{\kern0pt}\ {\isadigit{2}}{\isacharparenright}{\kern0pt}\ {\isasymcdot}\isactrlsub m\ {\isacharbar}{\kern0pt}one{\isasymrangle}\ {\isasymOtimes}\isanewline
\ \ \ \ \ \ \ \ \ \ \ \ {\isacharbar}{\kern0pt}state{\isacharunderscore}{\kern0pt}basis\ {\isadigit{0}}\ {\isacharparenleft}{\kern0pt}j\ mod\ {\isadigit{2}}\ {\isacharcircum}{\kern0pt}\ {\isadigit{0}}{\isacharparenright}{\kern0pt}{\isasymrangle}{\isacharparenright}{\kern0pt}{\isachardoublequoteclose}\isanewline
\ \ \ \ \ \ \ \ \isacommand{using}\isamarkupfalse%
\ controlled{\isacharunderscore}{\kern0pt}rotations{\isachardot}{\kern0pt}simps\ \isacommand{by}\isamarkupfalse%
\ simp\isanewline
\ \ \ \ \ \ \isacommand{also}\isamarkupfalse%
\ \isacommand{have}\isamarkupfalse%
\ {\isachardoublequoteopen}{\isasymdots}\ {\isacharequal}{\kern0pt}\ {\isacharbar}{\kern0pt}zero{\isasymrangle}\ {\isacharplus}{\kern0pt}\isanewline
\ \ \ \ \ \ \ \ \ \ \ \ \ \ \ \ \ \ \ \ \ \ exp\ {\isacharparenleft}{\kern0pt}{\isadigit{2}}\ {\isacharasterisk}{\kern0pt}\ {\isasymi}\ {\isacharasterisk}{\kern0pt}\ complex{\isacharunderscore}{\kern0pt}of{\isacharunderscore}{\kern0pt}real\ pi\ {\isacharasterisk}{\kern0pt}\ complex{\isacharunderscore}{\kern0pt}of{\isacharunderscore}{\kern0pt}nat\ {\isacharparenleft}{\kern0pt}j\ div\ {\isadigit{2}}\ {\isacharcircum}{\kern0pt}\ {\isadigit{0}}{\isacharparenright}{\kern0pt}\ {\isacharslash}{\kern0pt}\ {\isadigit{2}}{\isacharparenright}{\kern0pt}\ {\isasymcdot}\isactrlsub m\ {\isacharbar}{\kern0pt}one{\isasymrangle}\ {\isasymOtimes}\isanewline
\ \ \ \ \ \ \ \ \ \ \ \ \ \ \ \ \ \ \ \ \ \ {\isacharbar}{\kern0pt}state{\isacharunderscore}{\kern0pt}basis\ {\isadigit{0}}\ {\isacharparenleft}{\kern0pt}j\ mod\ {\isadigit{2}}\ {\isacharcircum}{\kern0pt}\ {\isadigit{0}}{\isacharparenright}{\kern0pt}{\isasymrangle}{\isachardoublequoteclose}\isanewline
\ \ \ \ \ \ \ \ \isacommand{using}\isamarkupfalse%
\ left{\isacharunderscore}{\kern0pt}mult{\isacharunderscore}{\kern0pt}one{\isacharunderscore}{\kern0pt}mat\ \isacommand{by}\isamarkupfalse%
\ {\isacharparenleft}{\kern0pt}simp\ add{\isacharcolon}{\kern0pt}\ ket{\isacharunderscore}{\kern0pt}vec{\isacharunderscore}{\kern0pt}def\ state{\isacharunderscore}{\kern0pt}basis{\isacharunderscore}{\kern0pt}def{\isacharparenright}{\kern0pt}\isanewline
\ \ \ \ \ \ \isacommand{also}\isamarkupfalse%
\ \isacommand{have}\isamarkupfalse%
\ {\isachardoublequoteopen}{\isasymdots}\ {\isacharequal}{\kern0pt}\ {\isacharbar}{\kern0pt}zero{\isasymrangle}\ {\isacharplus}{\kern0pt}\isanewline
\ \ \ \ \ \ \ \ \ \ \ \ \ \ \ \ \ \ \ \ \ \ exp\ {\isacharparenleft}{\kern0pt}{\isadigit{2}}\ {\isacharasterisk}{\kern0pt}\ {\isasymi}\ {\isacharasterisk}{\kern0pt}\ complex{\isacharunderscore}{\kern0pt}of{\isacharunderscore}{\kern0pt}real\ pi\ {\isacharasterisk}{\kern0pt}\ complex{\isacharunderscore}{\kern0pt}of{\isacharunderscore}{\kern0pt}nat\ j\ {\isacharslash}{\kern0pt}\ {\isadigit{2}}{\isacharcircum}{\kern0pt}Suc\ {\isadigit{0}}{\isacharparenright}{\kern0pt}\ {\isasymcdot}\isactrlsub m\ {\isacharbar}{\kern0pt}one{\isasymrangle}\ {\isasymOtimes}\isanewline
\ \ \ \ \ \ \ \ \ \ \ \ \ \ \ \ \ \ \ \ \ \ {\isacharbar}{\kern0pt}state{\isacharunderscore}{\kern0pt}basis\ {\isadigit{0}}\ {\isacharparenleft}{\kern0pt}j\ mod\ {\isadigit{2}}\ {\isacharcircum}{\kern0pt}\ {\isadigit{0}}{\isacharparenright}{\kern0pt}{\isasymrangle}{\isachardoublequoteclose}\isanewline
\ \ \ \ \ \ \ \ \isacommand{by}\isamarkupfalse%
\ auto\isanewline
\ \ \ \ \ \ \isacommand{finally}\isamarkupfalse%
\ \isacommand{show}\isamarkupfalse%
\ {\isachardoublequoteopen}controlled{\isacharunderscore}{\kern0pt}rotations\ {\isacharparenleft}{\kern0pt}Suc\ {\isadigit{0}}{\isacharparenright}{\kern0pt}\ {\isacharasterisk}{\kern0pt}\ {\isacharparenleft}{\kern0pt}\ {\isacharbar}{\kern0pt}zero{\isasymrangle}\ {\isacharplus}{\kern0pt}\isanewline
\ \ \ \ \ \ \ \ \ \ \ \ \ \ \ \ \ \ \ \ exp\ {\isacharparenleft}{\kern0pt}{\isadigit{2}}\ {\isacharasterisk}{\kern0pt}\ {\isasymi}\ {\isacharasterisk}{\kern0pt}\ complex{\isacharunderscore}{\kern0pt}of{\isacharunderscore}{\kern0pt}real\ pi\ {\isacharasterisk}{\kern0pt}\ complex{\isacharunderscore}{\kern0pt}of{\isacharunderscore}{\kern0pt}nat\ {\isacharparenleft}{\kern0pt}j\ div\ {\isadigit{2}}\ {\isacharcircum}{\kern0pt}\ {\isadigit{0}}{\isacharparenright}{\kern0pt}\ {\isacharslash}{\kern0pt}\ {\isadigit{2}}{\isacharparenright}{\kern0pt}\ {\isasymcdot}\isactrlsub m\ {\isacharbar}{\kern0pt}one{\isasymrangle}\ {\isasymOtimes}\isanewline
\ \ \ \ \ \ \ \ \ \ \ \ \ \ \ \ \ \ \ \ {\isacharbar}{\kern0pt}state{\isacharunderscore}{\kern0pt}basis\ {\isadigit{0}}\ {\isacharparenleft}{\kern0pt}j\ mod\ {\isadigit{2}}\ {\isacharcircum}{\kern0pt}\ {\isadigit{0}}{\isacharparenright}{\kern0pt}{\isasymrangle}{\isacharparenright}{\kern0pt}\ {\isacharequal}{\kern0pt}\isanewline
\ \ \ \ \ \ \ \ \ \ \ \ \ \ \ \ \ \ \ \ {\isacharbar}{\kern0pt}zero{\isasymrangle}\ {\isacharplus}{\kern0pt}\ exp\ {\isacharparenleft}{\kern0pt}{\isadigit{2}}\ {\isacharasterisk}{\kern0pt}\ {\isasymi}\ {\isacharasterisk}{\kern0pt}\ complex{\isacharunderscore}{\kern0pt}of{\isacharunderscore}{\kern0pt}real\ pi\ {\isacharasterisk}{\kern0pt}\ complex{\isacharunderscore}{\kern0pt}of{\isacharunderscore}{\kern0pt}nat\ j\ {\isacharslash}{\kern0pt}\ {\isadigit{2}}\ {\isacharcircum}{\kern0pt}\ Suc\ {\isadigit{0}}{\isacharparenright}{\kern0pt}\ {\isasymcdot}\isactrlsub m\ {\isacharbar}{\kern0pt}one{\isasymrangle}\isanewline
\ \ \ \ \ \ \ \ \ \ \ \ \ \ \ \ \ \ \ \ {\isasymOtimes}\ {\isacharbar}{\kern0pt}state{\isacharunderscore}{\kern0pt}basis\ {\isadigit{0}}\ {\isacharparenleft}{\kern0pt}j\ mod\ {\isadigit{2}}\ {\isacharcircum}{\kern0pt}\ {\isadigit{0}}{\isacharparenright}{\kern0pt}{\isasymrangle}{\isachardoublequoteclose}\isanewline
\ \ \ \ \ \ \ \ \isacommand{by}\isamarkupfalse%
\ this\isanewline
\ \ \ \ \isacommand{qed}\isamarkupfalse%
\isanewline
\ \ \isacommand{qed}\isamarkupfalse%
\isanewline
\isacommand{next}\isamarkupfalse%
\isanewline
\ \ \isacommand{case}\isamarkupfalse%
\ {\isacharparenleft}{\kern0pt}Suc\ n{\isacharprime}{\kern0pt}{\isacharparenright}{\kern0pt}\isanewline
\ \ \isacommand{define}\isamarkupfalse%
\ n\ \isakeyword{where}\ {\isachardoublequoteopen}n\ {\isacharequal}{\kern0pt}\ Suc\ n{\isacharprime}{\kern0pt}{\isachardoublequoteclose}\isanewline
\ \ \isacommand{assume}\isamarkupfalse%
\ HI{\isacharcolon}{\kern0pt}{\isachardoublequoteopen}\ {\isasymforall}j{\isacharless}{\kern0pt}{\isadigit{2}}\ {\isacharcircum}{\kern0pt}\ Suc\ n{\isacharprime}{\kern0pt}{\isachardot}{\kern0pt}\ controlled{\isacharunderscore}{\kern0pt}rotations\ {\isacharparenleft}{\kern0pt}Suc\ n{\isacharprime}{\kern0pt}{\isacharparenright}{\kern0pt}\ {\isacharasterisk}{\kern0pt}\ {\isacharparenleft}{\kern0pt}\ {\isacharbar}{\kern0pt}zero{\isasymrangle}\ {\isacharplus}{\kern0pt}\isanewline
\ \ \ \ \ \ \ \ \ \ \ \ \ exp\ {\isacharparenleft}{\kern0pt}{\isadigit{2}}\ {\isacharasterisk}{\kern0pt}\ {\isasymi}\ {\isacharasterisk}{\kern0pt}\ complex{\isacharunderscore}{\kern0pt}of{\isacharunderscore}{\kern0pt}real\ pi\ {\isacharasterisk}{\kern0pt}\ complex{\isacharunderscore}{\kern0pt}of{\isacharunderscore}{\kern0pt}nat\ {\isacharparenleft}{\kern0pt}j\ div\ {\isadigit{2}}\ {\isacharcircum}{\kern0pt}\ n{\isacharprime}{\kern0pt}{\isacharparenright}{\kern0pt}\ {\isacharslash}{\kern0pt}\ {\isadigit{2}}{\isacharparenright}{\kern0pt}\ {\isasymcdot}\isactrlsub m\ {\isacharbar}{\kern0pt}one{\isasymrangle}\ {\isasymOtimes}\isanewline
\ \ \ \ \ \ \ \ \ \ \ \ \ {\isacharbar}{\kern0pt}state{\isacharunderscore}{\kern0pt}basis\ n{\isacharprime}{\kern0pt}\ {\isacharparenleft}{\kern0pt}j\ mod\ {\isadigit{2}}\ {\isacharcircum}{\kern0pt}\ n{\isacharprime}{\kern0pt}{\isacharparenright}{\kern0pt}{\isasymrangle}{\isacharparenright}{\kern0pt}\ {\isacharequal}{\kern0pt}\isanewline
\ \ \ \ \ \ \ \ \ \ \ \ {\isacharbar}{\kern0pt}Deutsch{\isachardot}{\kern0pt}zero{\isasymrangle}\ {\isacharplus}{\kern0pt}\ exp\ {\isacharparenleft}{\kern0pt}{\isadigit{2}}\ {\isacharasterisk}{\kern0pt}\ {\isasymi}\ {\isacharasterisk}{\kern0pt}\ complex{\isacharunderscore}{\kern0pt}of{\isacharunderscore}{\kern0pt}real\ pi\ {\isacharasterisk}{\kern0pt}\ complex{\isacharunderscore}{\kern0pt}of{\isacharunderscore}{\kern0pt}nat\ j\ {\isacharslash}{\kern0pt}\ {\isadigit{2}}\ {\isacharcircum}{\kern0pt}\ Suc\ n{\isacharprime}{\kern0pt}{\isacharparenright}{\kern0pt}\ {\isasymcdot}\isactrlsub m\isanewline
\ \ \ \ \ \ \ \ \ \ \ \ {\isacharbar}{\kern0pt}Deutsch{\isachardot}{\kern0pt}one{\isasymrangle}\ {\isasymOtimes}\ {\isacharbar}{\kern0pt}state{\isacharunderscore}{\kern0pt}basis\ n{\isacharprime}{\kern0pt}\ {\isacharparenleft}{\kern0pt}j\ mod\ {\isadigit{2}}\ {\isacharcircum}{\kern0pt}\ n{\isacharprime}{\kern0pt}{\isacharparenright}{\kern0pt}{\isasymrangle}{\isachardoublequoteclose}\isanewline
\ \ \isacommand{show}\isamarkupfalse%
\ {\isachardoublequoteopen}{\isasymforall}j{\isacharless}{\kern0pt}{\isadigit{2}}\ {\isacharcircum}{\kern0pt}\ Suc\ {\isacharparenleft}{\kern0pt}Suc\ n{\isacharprime}{\kern0pt}{\isacharparenright}{\kern0pt}{\isachardot}{\kern0pt}\isanewline
\ \ \ \ \ \ \ \ \ \ \ \ controlled{\isacharunderscore}{\kern0pt}rotations\ {\isacharparenleft}{\kern0pt}Suc\ {\isacharparenleft}{\kern0pt}Suc\ n{\isacharprime}{\kern0pt}{\isacharparenright}{\kern0pt}{\isacharparenright}{\kern0pt}\ {\isacharasterisk}{\kern0pt}\isanewline
\ \ \ \ \ \ \ \ \ \ \ \ {\isacharparenleft}{\kern0pt}\ {\isacharbar}{\kern0pt}Deutsch{\isachardot}{\kern0pt}zero{\isasymrangle}\ {\isacharplus}{\kern0pt}\ exp\ {\isacharparenleft}{\kern0pt}{\isadigit{2}}\ {\isacharasterisk}{\kern0pt}\ {\isasymi}\ {\isacharasterisk}{\kern0pt}\ complex{\isacharunderscore}{\kern0pt}of{\isacharunderscore}{\kern0pt}real\ pi\ {\isacharasterisk}{\kern0pt}\ \isanewline
\ \ \ \ \ \ \ \ \ \ \ \ \ \ complex{\isacharunderscore}{\kern0pt}of{\isacharunderscore}{\kern0pt}nat\ {\isacharparenleft}{\kern0pt}j\ div\ {\isadigit{2}}\ {\isacharcircum}{\kern0pt}\ Suc\ n{\isacharprime}{\kern0pt}{\isacharparenright}{\kern0pt}\ {\isacharslash}{\kern0pt}\ {\isadigit{2}}{\isacharparenright}{\kern0pt}\ {\isasymcdot}\isactrlsub m\ {\isacharbar}{\kern0pt}Deutsch{\isachardot}{\kern0pt}one{\isasymrangle}\ {\isasymOtimes}\isanewline
\ \ \ \ \ \ \ \ \ \ \ \ \ {\isacharbar}{\kern0pt}state{\isacharunderscore}{\kern0pt}basis\ {\isacharparenleft}{\kern0pt}Suc\ n{\isacharprime}{\kern0pt}{\isacharparenright}{\kern0pt}\ {\isacharparenleft}{\kern0pt}j\ mod\ {\isadigit{2}}\ {\isacharcircum}{\kern0pt}\ Suc\ n{\isacharprime}{\kern0pt}{\isacharparenright}{\kern0pt}{\isasymrangle}{\isacharparenright}{\kern0pt}\ {\isacharequal}{\kern0pt}\isanewline
\ \ \ \ \ \ \ \ \ \ \ \ {\isacharbar}{\kern0pt}Deutsch{\isachardot}{\kern0pt}zero{\isasymrangle}\ {\isacharplus}{\kern0pt}\ exp\ {\isacharparenleft}{\kern0pt}{\isadigit{2}}\ {\isacharasterisk}{\kern0pt}\ {\isasymi}\ {\isacharasterisk}{\kern0pt}\ complex{\isacharunderscore}{\kern0pt}of{\isacharunderscore}{\kern0pt}real\ pi\ {\isacharasterisk}{\kern0pt}\ \isanewline
\ \ \ \ \ \ \ \ \ \ \ \ complex{\isacharunderscore}{\kern0pt}of{\isacharunderscore}{\kern0pt}nat\ j\ {\isacharslash}{\kern0pt}\ {\isadigit{2}}\ {\isacharcircum}{\kern0pt}\ Suc\ {\isacharparenleft}{\kern0pt}Suc\ n{\isacharprime}{\kern0pt}{\isacharparenright}{\kern0pt}{\isacharparenright}{\kern0pt}\ {\isasymcdot}\isactrlsub m\ {\isacharbar}{\kern0pt}Deutsch{\isachardot}{\kern0pt}one{\isasymrangle}\ {\isasymOtimes}\isanewline
\ \ \ \ \ \ \ \ \ \ \ \ {\isacharbar}{\kern0pt}state{\isacharunderscore}{\kern0pt}basis\ {\isacharparenleft}{\kern0pt}Suc\ n{\isacharprime}{\kern0pt}{\isacharparenright}{\kern0pt}\ {\isacharparenleft}{\kern0pt}j\ mod\ {\isadigit{2}}\ {\isacharcircum}{\kern0pt}\ Suc\ n{\isacharprime}{\kern0pt}{\isacharparenright}{\kern0pt}{\isasymrangle}{\isachardoublequoteclose}\isanewline
\ \ \isacommand{proof}\isamarkupfalse%
\ {\isacharparenleft}{\kern0pt}rule\ allI{\isacharparenright}{\kern0pt}\isanewline
\ \ \ \ \isacommand{fix}\isamarkupfalse%
\ j{\isacharcolon}{\kern0pt}{\isacharcolon}{\kern0pt}nat\isanewline
\ \ \ \ \isacommand{show}\isamarkupfalse%
\ {\isachardoublequoteopen}j\ {\isacharless}{\kern0pt}\ {\isadigit{2}}\ {\isacharcircum}{\kern0pt}\ Suc\ {\isacharparenleft}{\kern0pt}Suc\ n{\isacharprime}{\kern0pt}{\isacharparenright}{\kern0pt}\ {\isasymlongrightarrow}\isanewline
\ \ \ \ \ \ \ \ \ controlled{\isacharunderscore}{\kern0pt}rotations\ {\isacharparenleft}{\kern0pt}Suc\ {\isacharparenleft}{\kern0pt}Suc\ n{\isacharprime}{\kern0pt}{\isacharparenright}{\kern0pt}{\isacharparenright}{\kern0pt}\ {\isacharasterisk}{\kern0pt}\ {\isacharparenleft}{\kern0pt}\ {\isacharbar}{\kern0pt}Deutsch{\isachardot}{\kern0pt}zero{\isasymrangle}\ {\isacharplus}{\kern0pt}\isanewline
\ \ \ \ \ \ \ \ \ \ exp\ {\isacharparenleft}{\kern0pt}{\isadigit{2}}\ {\isacharasterisk}{\kern0pt}\ {\isasymi}\ {\isacharasterisk}{\kern0pt}\ complex{\isacharunderscore}{\kern0pt}of{\isacharunderscore}{\kern0pt}real\ pi\ {\isacharasterisk}{\kern0pt}\ complex{\isacharunderscore}{\kern0pt}of{\isacharunderscore}{\kern0pt}nat\ {\isacharparenleft}{\kern0pt}j\ div\ {\isadigit{2}}\ {\isacharcircum}{\kern0pt}\ Suc\ n{\isacharprime}{\kern0pt}{\isacharparenright}{\kern0pt}\ {\isacharslash}{\kern0pt}\ {\isadigit{2}}{\isacharparenright}{\kern0pt}\ {\isasymcdot}\isactrlsub m\isanewline
\ \ \ \ \ \ \ \ \ \ {\isacharbar}{\kern0pt}Deutsch{\isachardot}{\kern0pt}one{\isasymrangle}\ {\isasymOtimes}\ {\isacharbar}{\kern0pt}state{\isacharunderscore}{\kern0pt}basis\ {\isacharparenleft}{\kern0pt}Suc\ n{\isacharprime}{\kern0pt}{\isacharparenright}{\kern0pt}\ {\isacharparenleft}{\kern0pt}j\ mod\ {\isadigit{2}}\ {\isacharcircum}{\kern0pt}\ Suc\ n{\isacharprime}{\kern0pt}{\isacharparenright}{\kern0pt}{\isasymrangle}{\isacharparenright}{\kern0pt}\ {\isacharequal}{\kern0pt}\isanewline
\ \ \ \ \ \ \ \ \ {\isacharbar}{\kern0pt}Deutsch{\isachardot}{\kern0pt}zero{\isasymrangle}\ {\isacharplus}{\kern0pt}\ exp\ {\isacharparenleft}{\kern0pt}{\isadigit{2}}\ {\isacharasterisk}{\kern0pt}\ {\isasymi}\ {\isacharasterisk}{\kern0pt}\ complex{\isacharunderscore}{\kern0pt}of{\isacharunderscore}{\kern0pt}real\ pi\ {\isacharasterisk}{\kern0pt}\ complex{\isacharunderscore}{\kern0pt}of{\isacharunderscore}{\kern0pt}nat\ j\ {\isacharslash}{\kern0pt}\ {\isadigit{2}}\ {\isacharcircum}{\kern0pt}\ Suc\ {\isacharparenleft}{\kern0pt}Suc\ n{\isacharprime}{\kern0pt}{\isacharparenright}{\kern0pt}{\isacharparenright}{\kern0pt}\ {\isasymcdot}\isactrlsub m\isanewline
\ \ \ \ \ \ \ \ \ {\isacharbar}{\kern0pt}Deutsch{\isachardot}{\kern0pt}one{\isasymrangle}\ {\isasymOtimes}\ {\isacharbar}{\kern0pt}state{\isacharunderscore}{\kern0pt}basis\ {\isacharparenleft}{\kern0pt}Suc\ n{\isacharprime}{\kern0pt}{\isacharparenright}{\kern0pt}\ {\isacharparenleft}{\kern0pt}j\ mod\ {\isadigit{2}}\ {\isacharcircum}{\kern0pt}\ Suc\ n{\isacharprime}{\kern0pt}{\isacharparenright}{\kern0pt}{\isasymrangle}{\isachardoublequoteclose}\isanewline
\ \ \ \ \isacommand{proof}\isamarkupfalse%
\ \isanewline
\ \ \ \ \ \ \isacommand{assume}\isamarkupfalse%
\ jass{\isacharcolon}{\kern0pt}{\isachardoublequoteopen}j\ {\isacharless}{\kern0pt}\ {\isadigit{2}}\ {\isacharcircum}{\kern0pt}\ Suc\ {\isacharparenleft}{\kern0pt}Suc\ n{\isacharprime}{\kern0pt}{\isacharparenright}{\kern0pt}{\isachardoublequoteclose}\isanewline
\ \ \ \ \ \ \isacommand{show}\isamarkupfalse%
\ {\isachardoublequoteopen}controlled{\isacharunderscore}{\kern0pt}rotations\ {\isacharparenleft}{\kern0pt}Suc\ {\isacharparenleft}{\kern0pt}Suc\ n{\isacharprime}{\kern0pt}{\isacharparenright}{\kern0pt}{\isacharparenright}{\kern0pt}\ {\isacharasterisk}{\kern0pt}\ {\isacharparenleft}{\kern0pt}\ {\isacharbar}{\kern0pt}Deutsch{\isachardot}{\kern0pt}zero{\isasymrangle}\ {\isacharplus}{\kern0pt}\isanewline
\ \ \ \ \ \ \ \ \ \ \ \ exp\ {\isacharparenleft}{\kern0pt}{\isadigit{2}}\ {\isacharasterisk}{\kern0pt}\ {\isasymi}\ {\isacharasterisk}{\kern0pt}\ complex{\isacharunderscore}{\kern0pt}of{\isacharunderscore}{\kern0pt}real\ pi\ {\isacharasterisk}{\kern0pt}\ complex{\isacharunderscore}{\kern0pt}of{\isacharunderscore}{\kern0pt}nat\ {\isacharparenleft}{\kern0pt}j\ div\ {\isadigit{2}}\ {\isacharcircum}{\kern0pt}\ Suc\ n{\isacharprime}{\kern0pt}{\isacharparenright}{\kern0pt}\ {\isacharslash}{\kern0pt}\ {\isadigit{2}}{\isacharparenright}{\kern0pt}\ {\isasymcdot}\isactrlsub m\isanewline
\ \ \ \ \ \ \ \ \ \ \ \ {\isacharbar}{\kern0pt}Deutsch{\isachardot}{\kern0pt}one{\isasymrangle}\ {\isasymOtimes}\ {\isacharbar}{\kern0pt}state{\isacharunderscore}{\kern0pt}basis\ {\isacharparenleft}{\kern0pt}Suc\ n{\isacharprime}{\kern0pt}{\isacharparenright}{\kern0pt}\ {\isacharparenleft}{\kern0pt}j\ mod\ {\isadigit{2}}\ {\isacharcircum}{\kern0pt}\ Suc\ n{\isacharprime}{\kern0pt}{\isacharparenright}{\kern0pt}{\isasymrangle}{\isacharparenright}{\kern0pt}\ {\isacharequal}{\kern0pt}\isanewline
\ \ \ \ \ \ \ \ \ \ \ \ {\isacharbar}{\kern0pt}Deutsch{\isachardot}{\kern0pt}zero{\isasymrangle}\ {\isacharplus}{\kern0pt}\ exp\ {\isacharparenleft}{\kern0pt}{\isadigit{2}}\ {\isacharasterisk}{\kern0pt}\ {\isasymi}\ {\isacharasterisk}{\kern0pt}\ complex{\isacharunderscore}{\kern0pt}of{\isacharunderscore}{\kern0pt}real\ pi\ {\isacharasterisk}{\kern0pt}\ complex{\isacharunderscore}{\kern0pt}of{\isacharunderscore}{\kern0pt}nat\ j\ {\isacharslash}{\kern0pt}\ {\isadigit{2}}\ {\isacharcircum}{\kern0pt}\ Suc\ {\isacharparenleft}{\kern0pt}Suc\ n{\isacharprime}{\kern0pt}{\isacharparenright}{\kern0pt}{\isacharparenright}{\kern0pt}{\isasymcdot}\isactrlsub m\isanewline
\ \ \ \ \ \ \ \ \ \ \ \ {\isacharbar}{\kern0pt}Deutsch{\isachardot}{\kern0pt}one{\isasymrangle}\ {\isasymOtimes}\ {\isacharbar}{\kern0pt}state{\isacharunderscore}{\kern0pt}basis\ {\isacharparenleft}{\kern0pt}Suc\ n{\isacharprime}{\kern0pt}{\isacharparenright}{\kern0pt}\ {\isacharparenleft}{\kern0pt}j\ mod\ {\isadigit{2}}\ {\isacharcircum}{\kern0pt}\ Suc\ n{\isacharprime}{\kern0pt}{\isacharparenright}{\kern0pt}{\isasymrangle}{\isachardoublequoteclose}\isanewline
\ \ \ \ \ \ \isacommand{proof}\isamarkupfalse%
\ {\isacharminus}{\kern0pt}\isanewline
\ \ \ \ \ \ \ \ \isacommand{define}\isamarkupfalse%
\ jd{\isadigit{2}}n\ jm{\isadigit{2}}n\ \isakeyword{where}\ {\isachardoublequoteopen}jd{\isadigit{2}}n\ {\isacharequal}{\kern0pt}\ j\ div\ {\isadigit{2}}{\isacharcircum}{\kern0pt}n{\isachardoublequoteclose}\ \isakeyword{and}\ {\isachardoublequoteopen}jm{\isadigit{2}}n\ {\isacharequal}{\kern0pt}\ j\ mod\ {\isadigit{2}}{\isacharcircum}{\kern0pt}n{\isachardoublequoteclose}\isanewline
\ \ \ \ \ \ \ \ \isacommand{define}\isamarkupfalse%
\ jlast\ \isakeyword{where}\ {\isachardoublequoteopen}jlast\ {\isacharequal}{\kern0pt}\ jm{\isadigit{2}}n\ mod\ {\isadigit{2}}{\isachardoublequoteclose}\isanewline
\ \ \ \ \ \ \ \ \isacommand{define}\isamarkupfalse%
\ jmm\ \isakeyword{where}\ {\isachardoublequoteopen}jmm\ {\isacharequal}{\kern0pt}\ jm{\isadigit{2}}n\ div\ {\isadigit{2}}{\isachardoublequoteclose}\isanewline
\ \ \ \ \ \ \ \ \isacommand{define}\isamarkupfalse%
\ jd{\isadigit{2}}\ \isakeyword{where}\ {\isachardoublequoteopen}jd{\isadigit{2}}\ {\isacharequal}{\kern0pt}\ j\ div\ {\isadigit{2}}{\isachardoublequoteclose}\isanewline
\ \ \ \ \ \ \ \ \isacommand{have}\isamarkupfalse%
\ jlastj{\isacharcolon}{\kern0pt}{\isachardoublequoteopen}jlast\ {\isacharequal}{\kern0pt}\ j\ mod\ {\isadigit{2}}{\isachardoublequoteclose}\ \isacommand{using}\isamarkupfalse%
\ jlast{\isacharunderscore}{\kern0pt}def\ jm{\isadigit{2}}n{\isacharunderscore}{\kern0pt}def\ \isanewline
\ \ \ \ \ \ \ \ \ \ \isacommand{by}\isamarkupfalse%
\ {\isacharparenleft}{\kern0pt}metis\ less{\isacharunderscore}{\kern0pt}Suc{\isacharunderscore}{\kern0pt}eq{\isacharunderscore}{\kern0pt}{\isadigit{0}}{\isacharunderscore}{\kern0pt}disj\ less{\isacharunderscore}{\kern0pt}Suc{\isacharunderscore}{\kern0pt}eq{\isacharunderscore}{\kern0pt}le\ mod{\isacharunderscore}{\kern0pt}mod{\isacharunderscore}{\kern0pt}power{\isacharunderscore}{\kern0pt}cancel\ n{\isacharunderscore}{\kern0pt}def\ power{\isacharunderscore}{\kern0pt}Suc{\isadigit{0}}{\isacharunderscore}{\kern0pt}right{\isacharparenright}{\kern0pt}\isanewline
\ \ \ \ \ \ \ \ \isacommand{have}\isamarkupfalse%
\ {\isachardoublequoteopen}controlled{\isacharunderscore}{\kern0pt}rotations\ {\isacharparenleft}{\kern0pt}Suc\ n{\isacharparenright}{\kern0pt}\ {\isacharasterisk}{\kern0pt}\ {\isacharparenleft}{\kern0pt}\ {\isacharbar}{\kern0pt}Deutsch{\isachardot}{\kern0pt}zero{\isasymrangle}\ {\isacharplus}{\kern0pt}\isanewline
\ \ \ \ \ \ \ \ \ \ \ \ exp\ {\isacharparenleft}{\kern0pt}{\isadigit{2}}\ {\isacharasterisk}{\kern0pt}\ {\isasymi}\ {\isacharasterisk}{\kern0pt}\ complex{\isacharunderscore}{\kern0pt}of{\isacharunderscore}{\kern0pt}real\ pi\ {\isacharasterisk}{\kern0pt}\ complex{\isacharunderscore}{\kern0pt}of{\isacharunderscore}{\kern0pt}nat\ jd{\isadigit{2}}n\ {\isacharslash}{\kern0pt}\ {\isadigit{2}}{\isacharparenright}{\kern0pt}\ {\isasymcdot}\isactrlsub m\isanewline
\ \ \ \ \ \ \ \ \ \ \ \ {\isacharbar}{\kern0pt}Deutsch{\isachardot}{\kern0pt}one{\isasymrangle}\ {\isasymOtimes}\ {\isacharbar}{\kern0pt}state{\isacharunderscore}{\kern0pt}basis\ n\ jm{\isadigit{2}}n{\isasymrangle}{\isacharparenright}{\kern0pt}\ {\isacharequal}{\kern0pt}\ \isanewline
\ \ \ \ \ \ \ \ \ \ \ \ {\isacharparenleft}{\kern0pt}{\isacharparenleft}{\kern0pt}control\ {\isacharparenleft}{\kern0pt}Suc\ n{\isacharparenright}{\kern0pt}\ {\isacharparenleft}{\kern0pt}R\ {\isacharparenleft}{\kern0pt}Suc\ n{\isacharparenright}{\kern0pt}{\isacharparenright}{\kern0pt}{\isacharparenright}{\kern0pt}\ {\isacharasterisk}{\kern0pt}\ {\isacharparenleft}{\kern0pt}{\isacharparenleft}{\kern0pt}controlled{\isacharunderscore}{\kern0pt}rotations\ n{\isacharparenright}{\kern0pt}\ {\isasymOtimes}\ {\isacharparenleft}{\kern0pt}{\isadigit{1}}\isactrlsub m\ {\isadigit{2}}{\isacharparenright}{\kern0pt}{\isacharparenright}{\kern0pt}{\isacharparenright}{\kern0pt}\ {\isacharasterisk}{\kern0pt}\ {\isacharparenleft}{\kern0pt}\ {\isacharbar}{\kern0pt}zero{\isasymrangle}\ {\isacharplus}{\kern0pt}\isanewline
\ \ \ \ \ \ \ \ \ \ \ \ exp\ {\isacharparenleft}{\kern0pt}{\isadigit{2}}\ {\isacharasterisk}{\kern0pt}\ {\isasymi}\ {\isacharasterisk}{\kern0pt}\ complex{\isacharunderscore}{\kern0pt}of{\isacharunderscore}{\kern0pt}real\ pi\ {\isacharasterisk}{\kern0pt}\ complex{\isacharunderscore}{\kern0pt}of{\isacharunderscore}{\kern0pt}nat\ jd{\isadigit{2}}n\ {\isacharslash}{\kern0pt}\ {\isadigit{2}}{\isacharparenright}{\kern0pt}\ {\isasymcdot}\isactrlsub m\isanewline
\ \ \ \ \ \ \ \ \ \ \ \ {\isacharbar}{\kern0pt}Deutsch{\isachardot}{\kern0pt}one{\isasymrangle}\ {\isasymOtimes}\ {\isacharbar}{\kern0pt}state{\isacharunderscore}{\kern0pt}basis\ n\ jm{\isadigit{2}}n{\isasymrangle}{\isacharparenright}{\kern0pt}{\isachardoublequoteclose}\isanewline
\ \ \ \ \ \ \ \ \ \ \isacommand{using}\isamarkupfalse%
\ controlled{\isacharunderscore}{\kern0pt}rotations{\isachardot}{\kern0pt}simps\ n{\isacharunderscore}{\kern0pt}def\ \isacommand{by}\isamarkupfalse%
\ simp\isanewline
\ \ \ \ \ \ \ \ \isacommand{also}\isamarkupfalse%
\ \isacommand{have}\isamarkupfalse%
\ {\isachardoublequoteopen}{\isasymdots}\ {\isacharequal}{\kern0pt}\ {\isacharparenleft}{\kern0pt}{\isacharparenleft}{\kern0pt}control\ {\isacharparenleft}{\kern0pt}Suc\ n{\isacharparenright}{\kern0pt}\ {\isacharparenleft}{\kern0pt}R\ {\isacharparenleft}{\kern0pt}Suc\ n{\isacharparenright}{\kern0pt}{\isacharparenright}{\kern0pt}{\isacharparenright}{\kern0pt}\ {\isacharasterisk}{\kern0pt}\ {\isacharparenleft}{\kern0pt}{\isacharparenleft}{\kern0pt}controlled{\isacharunderscore}{\kern0pt}rotations\ n{\isacharparenright}{\kern0pt}\ {\isasymOtimes}\ {\isacharparenleft}{\kern0pt}{\isadigit{1}}\isactrlsub m\ {\isadigit{2}}{\isacharparenright}{\kern0pt}{\isacharparenright}{\kern0pt}{\isacharparenright}{\kern0pt}\ {\isacharasterisk}{\kern0pt}\ \isanewline
\ \ \ \ \ \ \ \ \ \ \ \ {\isacharparenleft}{\kern0pt}\ {\isacharbar}{\kern0pt}zero{\isasymrangle}\ {\isacharplus}{\kern0pt}\ exp\ {\isacharparenleft}{\kern0pt}{\isadigit{2}}\ {\isacharasterisk}{\kern0pt}\ {\isasymi}\ {\isacharasterisk}{\kern0pt}\ complex{\isacharunderscore}{\kern0pt}of{\isacharunderscore}{\kern0pt}real\ pi\ {\isacharasterisk}{\kern0pt}\ complex{\isacharunderscore}{\kern0pt}of{\isacharunderscore}{\kern0pt}nat\ jd{\isadigit{2}}n\ {\isacharslash}{\kern0pt}\ {\isadigit{2}}{\isacharparenright}{\kern0pt}\ {\isasymcdot}\isactrlsub m\ \ {\isacharbar}{\kern0pt}one{\isasymrangle}\ {\isasymOtimes}\ \isanewline
\ \ \ \ \ \ \ \ \ \ \ \ {\isacharparenleft}{\kern0pt}\ {\isacharbar}{\kern0pt}state{\isacharunderscore}{\kern0pt}basis\ n{\isacharprime}{\kern0pt}\ jmm{\isasymrangle}\ {\isasymOtimes}\ {\isacharbar}{\kern0pt}state{\isacharunderscore}{\kern0pt}basis\ {\isadigit{1}}\ jlast{\isasymrangle}{\isacharparenright}{\kern0pt}{\isacharparenright}{\kern0pt}{\isachardoublequoteclose}\isanewline
\ \ \ \ \ \ \ \ \ \ \isacommand{using}\isamarkupfalse%
\ state{\isacharunderscore}{\kern0pt}basis{\isacharunderscore}{\kern0pt}dec{\isacharprime}{\kern0pt}\ jass\ n{\isacharunderscore}{\kern0pt}def\ jlast{\isacharunderscore}{\kern0pt}def\ jmm{\isacharunderscore}{\kern0pt}def\ jm{\isadigit{2}}n{\isacharunderscore}{\kern0pt}def\ mod{\isacharunderscore}{\kern0pt}less{\isacharunderscore}{\kern0pt}divisor\ pos{\isadigit{2}}\isanewline
\ \ \ \ \ \ \ \ \ \ \isacommand{by}\isamarkupfalse%
\ presburger\isanewline
\ \ \ \ \ \ \ \ \isacommand{also}\isamarkupfalse%
\ \isacommand{have}\isamarkupfalse%
\ {\isachardoublequoteopen}{\isasymdots}\ {\isacharequal}{\kern0pt}\ {\isacharparenleft}{\kern0pt}control\ {\isacharparenleft}{\kern0pt}Suc\ n{\isacharparenright}{\kern0pt}\ {\isacharparenleft}{\kern0pt}R\ {\isacharparenleft}{\kern0pt}Suc\ n{\isacharparenright}{\kern0pt}{\isacharparenright}{\kern0pt}{\isacharparenright}{\kern0pt}\ {\isacharasterisk}{\kern0pt}\ {\isacharparenleft}{\kern0pt}{\isacharparenleft}{\kern0pt}{\isacharparenleft}{\kern0pt}{\isacharparenleft}{\kern0pt}controlled{\isacharunderscore}{\kern0pt}rotations\ n{\isacharparenright}{\kern0pt}\ {\isasymOtimes}\ {\isacharparenleft}{\kern0pt}{\isadigit{1}}\isactrlsub m\ {\isadigit{2}}{\isacharparenright}{\kern0pt}{\isacharparenright}{\kern0pt}{\isacharparenright}{\kern0pt}\ {\isacharasterisk}{\kern0pt}\ \isanewline
\ \ \ \ \ \ \ \ \ \ \ \ {\isacharparenleft}{\kern0pt}\ {\isacharbar}{\kern0pt}zero{\isasymrangle}\ {\isacharplus}{\kern0pt}\ exp\ {\isacharparenleft}{\kern0pt}{\isadigit{2}}\ {\isacharasterisk}{\kern0pt}\ {\isasymi}\ {\isacharasterisk}{\kern0pt}\ complex{\isacharunderscore}{\kern0pt}of{\isacharunderscore}{\kern0pt}real\ pi\ {\isacharasterisk}{\kern0pt}\ complex{\isacharunderscore}{\kern0pt}of{\isacharunderscore}{\kern0pt}nat\ jd{\isadigit{2}}n\ {\isacharslash}{\kern0pt}\ {\isadigit{2}}{\isacharparenright}{\kern0pt}\ {\isasymcdot}\isactrlsub m\ \ {\isacharbar}{\kern0pt}one{\isasymrangle}\ {\isasymOtimes}\ \isanewline
\ \ \ \ \ \ \ \ \ \ \ \ {\isacharparenleft}{\kern0pt}\ {\isacharbar}{\kern0pt}state{\isacharunderscore}{\kern0pt}basis\ n{\isacharprime}{\kern0pt}\ jmm{\isasymrangle}\ {\isasymOtimes}\ {\isacharbar}{\kern0pt}state{\isacharunderscore}{\kern0pt}basis\ {\isadigit{1}}\ jlast{\isasymrangle}{\isacharparenright}{\kern0pt}{\isacharparenright}{\kern0pt}{\isacharparenright}{\kern0pt}{\isachardoublequoteclose}\isanewline
\ \ \ \ \ \ \ \ \isacommand{proof}\isamarkupfalse%
\ {\isacharparenleft}{\kern0pt}rule\ assoc{\isacharunderscore}{\kern0pt}mult{\isacharunderscore}{\kern0pt}mat{\isacharparenright}{\kern0pt}\isanewline
\ \ \ \ \ \ \ \ \ \ \isacommand{show}\isamarkupfalse%
\ {\isachardoublequoteopen}control\ {\isacharparenleft}{\kern0pt}Suc\ n{\isacharparenright}{\kern0pt}\ {\isacharparenleft}{\kern0pt}R\ {\isacharparenleft}{\kern0pt}Suc\ n{\isacharparenright}{\kern0pt}{\isacharparenright}{\kern0pt}\ {\isasymin}\ carrier{\isacharunderscore}{\kern0pt}mat\ {\isacharparenleft}{\kern0pt}{\isadigit{2}}{\isacharcircum}{\kern0pt}{\isacharparenleft}{\kern0pt}Suc\ n{\isacharparenright}{\kern0pt}{\isacharparenright}{\kern0pt}\ {\isacharparenleft}{\kern0pt}{\isadigit{2}}{\isacharcircum}{\kern0pt}{\isacharparenleft}{\kern0pt}Suc\ n{\isacharparenright}{\kern0pt}{\isacharparenright}{\kern0pt}{\isachardoublequoteclose}\isanewline
\ \ \ \ \ \ \ \ \ \ \ \ \isacommand{using}\isamarkupfalse%
\ control{\isacharunderscore}{\kern0pt}carrier{\isacharunderscore}{\kern0pt}mat\ n{\isacharunderscore}{\kern0pt}def\ \isacommand{by}\isamarkupfalse%
\ blast\isanewline
\ \ \ \ \ \ \ \ \ \ \isacommand{show}\isamarkupfalse%
\ {\isachardoublequoteopen}controlled{\isacharunderscore}{\kern0pt}rotations\ n\ {\isasymOtimes}\ {\isadigit{1}}\isactrlsub m\ {\isadigit{2}}\ {\isasymin}\ carrier{\isacharunderscore}{\kern0pt}mat\ {\isacharparenleft}{\kern0pt}{\isadigit{2}}\ {\isacharcircum}{\kern0pt}\ Suc\ n{\isacharparenright}{\kern0pt}\ {\isacharparenleft}{\kern0pt}{\isadigit{2}}\ {\isacharcircum}{\kern0pt}\ Suc\ n{\isacharparenright}{\kern0pt}{\isachardoublequoteclose}\isanewline
\ \ \ \ \ \ \ \ \ \ \ \ \isacommand{using}\isamarkupfalse%
\ controlled{\isacharunderscore}{\kern0pt}rotations{\isacharunderscore}{\kern0pt}carrier{\isacharunderscore}{\kern0pt}mat\ n{\isacharunderscore}{\kern0pt}def\isanewline
\ \ \ \ \ \ \ \ \ \ \ \ \isacommand{by}\isamarkupfalse%
\ {\isacharparenleft}{\kern0pt}metis\ One{\isacharunderscore}{\kern0pt}nat{\isacharunderscore}{\kern0pt}def\ controlled{\isacharunderscore}{\kern0pt}rotations{\isachardot}{\kern0pt}simps{\isacharparenleft}{\kern0pt}{\isadigit{2}}{\isacharparenright}{\kern0pt}\ power{\isacharunderscore}{\kern0pt}Suc{\isadigit{2}}\ power{\isacharunderscore}{\kern0pt}one{\isacharunderscore}{\kern0pt}right\ \isanewline
\ \ \ \ \ \ \ \ \ \ \ \ \ \ tensor{\isacharunderscore}{\kern0pt}carrier{\isacharunderscore}{\kern0pt}mat{\isacharparenright}{\kern0pt}\isanewline
\ \ \ \ \ \ \ \ \ \ \isacommand{show}\isamarkupfalse%
\ {\isachardoublequoteopen}{\isacharbar}{\kern0pt}zero{\isasymrangle}\ {\isacharplus}{\kern0pt}\ exp\ {\isacharparenleft}{\kern0pt}{\isadigit{2}}{\isacharasterisk}{\kern0pt}{\isasymi}{\isacharasterisk}{\kern0pt}pi{\isacharasterisk}{\kern0pt}complex{\isacharunderscore}{\kern0pt}of{\isacharunderscore}{\kern0pt}nat\ jd{\isadigit{2}}n\ {\isacharslash}{\kern0pt}{\isadigit{2}}{\isacharparenright}{\kern0pt}\ {\isasymcdot}\isactrlsub m\ {\isacharbar}{\kern0pt}one{\isasymrangle}\ {\isasymOtimes}\ {\isacharparenleft}{\kern0pt}\ {\isacharbar}{\kern0pt}state{\isacharunderscore}{\kern0pt}basis\ n{\isacharprime}{\kern0pt}\ jmm{\isasymrangle}\ {\isasymOtimes}\isanewline
\ \ \ \ \ \ \ \ \ \ \ \ \ \ \ \ {\isacharbar}{\kern0pt}state{\isacharunderscore}{\kern0pt}basis\ {\isadigit{1}}\ jlast{\isasymrangle}{\isacharparenright}{\kern0pt}\ {\isasymin}\ carrier{\isacharunderscore}{\kern0pt}mat\ {\isacharparenleft}{\kern0pt}{\isadigit{2}}\ {\isacharcircum}{\kern0pt}\ Suc\ n{\isacharparenright}{\kern0pt}\ {\isadigit{1}}{\isachardoublequoteclose}\isanewline
\ \ \ \ \ \ \ \ \ \ \ \ \isacommand{using}\isamarkupfalse%
\ state{\isacharunderscore}{\kern0pt}basis{\isacharunderscore}{\kern0pt}carrier{\isacharunderscore}{\kern0pt}mat\ ket{\isacharunderscore}{\kern0pt}vec{\isacharunderscore}{\kern0pt}def\ \isanewline
\ \ \ \ \ \ \ \ \ \ \ \ \isacommand{by}\isamarkupfalse%
\ {\isacharparenleft}{\kern0pt}simp\ add{\isacharcolon}{\kern0pt}\ carrier{\isacharunderscore}{\kern0pt}matI\ n{\isacharunderscore}{\kern0pt}def\ state{\isacharunderscore}{\kern0pt}basis{\isacharunderscore}{\kern0pt}def{\isacharparenright}{\kern0pt}\isanewline
\ \ \ \ \ \ \ \ \isacommand{qed}\isamarkupfalse%
\isanewline
\ \ \ \ \ \ \ \ \isacommand{also}\isamarkupfalse%
\ \isacommand{have}\isamarkupfalse%
\ {\isachardoublequoteopen}{\isasymdots}\ {\isacharequal}{\kern0pt}\ {\isacharparenleft}{\kern0pt}control\ {\isacharparenleft}{\kern0pt}Suc\ n{\isacharparenright}{\kern0pt}\ {\isacharparenleft}{\kern0pt}R\ {\isacharparenleft}{\kern0pt}Suc\ n{\isacharparenright}{\kern0pt}{\isacharparenright}{\kern0pt}{\isacharparenright}{\kern0pt}\ {\isacharasterisk}{\kern0pt}\ {\isacharparenleft}{\kern0pt}{\isacharparenleft}{\kern0pt}{\isacharparenleft}{\kern0pt}{\isacharparenleft}{\kern0pt}controlled{\isacharunderscore}{\kern0pt}rotations\ n{\isacharparenright}{\kern0pt}\ {\isasymOtimes}\ {\isacharparenleft}{\kern0pt}{\isadigit{1}}\isactrlsub m\ {\isadigit{2}}{\isacharparenright}{\kern0pt}{\isacharparenright}{\kern0pt}{\isacharparenright}{\kern0pt}\ {\isacharasterisk}{\kern0pt}\ \isanewline
\ \ \ \ \ \ \ \ \ \ \ \ {\isacharparenleft}{\kern0pt}{\isacharparenleft}{\kern0pt}\ {\isacharbar}{\kern0pt}zero{\isasymrangle}\ {\isacharplus}{\kern0pt}\ exp\ {\isacharparenleft}{\kern0pt}{\isadigit{2}}\ {\isacharasterisk}{\kern0pt}\ {\isasymi}\ {\isacharasterisk}{\kern0pt}\ complex{\isacharunderscore}{\kern0pt}of{\isacharunderscore}{\kern0pt}real\ pi\ {\isacharasterisk}{\kern0pt}\ complex{\isacharunderscore}{\kern0pt}of{\isacharunderscore}{\kern0pt}nat\ jd{\isadigit{2}}n\ {\isacharslash}{\kern0pt}\ {\isadigit{2}}{\isacharparenright}{\kern0pt}\ {\isasymcdot}\isactrlsub m\ \ {\isacharbar}{\kern0pt}one{\isasymrangle}\ {\isasymOtimes}\ \isanewline
\ \ \ \ \ \ \ \ \ \ \ \ \ {\isacharbar}{\kern0pt}state{\isacharunderscore}{\kern0pt}basis\ n{\isacharprime}{\kern0pt}\ jmm{\isasymrangle}{\isacharparenright}{\kern0pt}\ {\isasymOtimes}\ {\isacharbar}{\kern0pt}state{\isacharunderscore}{\kern0pt}basis\ {\isadigit{1}}\ jlast{\isasymrangle}{\isacharparenright}{\kern0pt}{\isacharparenright}{\kern0pt}{\isachardoublequoteclose}\isanewline
\ \ \ \ \ \ \ \ \ \ \isacommand{using}\isamarkupfalse%
\ tensor{\isacharunderscore}{\kern0pt}mat{\isacharunderscore}{\kern0pt}is{\isacharunderscore}{\kern0pt}assoc\ control{\isacharunderscore}{\kern0pt}carrier{\isacharunderscore}{\kern0pt}mat\ n{\isacharunderscore}{\kern0pt}def\ controlled{\isacharunderscore}{\kern0pt}rotations{\isacharunderscore}{\kern0pt}carrier{\isacharunderscore}{\kern0pt}mat\isanewline
\ \ \ \ \ \ \ \ \ \ \ \ state{\isacharunderscore}{\kern0pt}basis{\isacharunderscore}{\kern0pt}carrier{\isacharunderscore}{\kern0pt}mat\ ket{\isacharunderscore}{\kern0pt}vec{\isacharunderscore}{\kern0pt}def\ \isacommand{by}\isamarkupfalse%
\ simp\isanewline
\ \ \ \ \ \ \ \ \isacommand{also}\isamarkupfalse%
\ \isacommand{have}\isamarkupfalse%
\ {\isachardoublequoteopen}{\isasymdots}\ {\isacharequal}{\kern0pt}\ {\isacharparenleft}{\kern0pt}control\ {\isacharparenleft}{\kern0pt}Suc\ n{\isacharparenright}{\kern0pt}\ {\isacharparenleft}{\kern0pt}R\ {\isacharparenleft}{\kern0pt}Suc\ n{\isacharparenright}{\kern0pt}{\isacharparenright}{\kern0pt}{\isacharparenright}{\kern0pt}\ {\isacharasterisk}{\kern0pt}\ {\isacharparenleft}{\kern0pt}{\isacharparenleft}{\kern0pt}{\isacharparenleft}{\kern0pt}controlled{\isacharunderscore}{\kern0pt}rotations\ n{\isacharparenright}{\kern0pt}\ {\isacharasterisk}{\kern0pt}\isanewline
\ \ \ \ \ \ \ \ \ \ \ \ \ \ \ \ \ \ \ \ \ \ \ \ {\isacharparenleft}{\kern0pt}{\isacharparenleft}{\kern0pt}\ {\isacharbar}{\kern0pt}zero{\isasymrangle}\ {\isacharplus}{\kern0pt}\ exp\ {\isacharparenleft}{\kern0pt}{\isadigit{2}}\ {\isacharasterisk}{\kern0pt}\ {\isasymi}\ {\isacharasterisk}{\kern0pt}\ pi\ {\isacharasterisk}{\kern0pt}\ complex{\isacharunderscore}{\kern0pt}of{\isacharunderscore}{\kern0pt}nat\ jd{\isadigit{2}}n\ {\isacharslash}{\kern0pt}\ {\isadigit{2}}{\isacharparenright}{\kern0pt}\ {\isasymcdot}\isactrlsub m\ \ {\isacharbar}{\kern0pt}one{\isasymrangle}{\isacharparenright}{\kern0pt}\ {\isasymOtimes}\isanewline
\ \ \ \ \ \ \ \ \ \ \ \ \ \ \ \ \ \ \ \ \ \ \ \ {\isacharbar}{\kern0pt}state{\isacharunderscore}{\kern0pt}basis\ n{\isacharprime}{\kern0pt}\ jmm{\isasymrangle}{\isacharparenright}{\kern0pt}{\isacharparenright}{\kern0pt}\ {\isasymOtimes}\ {\isacharparenleft}{\kern0pt}{\isacharparenleft}{\kern0pt}{\isadigit{1}}\isactrlsub m\ {\isadigit{2}}{\isacharparenright}{\kern0pt}\ {\isacharasterisk}{\kern0pt}\ {\isacharbar}{\kern0pt}state{\isacharunderscore}{\kern0pt}basis\ {\isadigit{1}}\ jlast{\isasymrangle}{\isacharparenright}{\kern0pt}{\isacharparenright}{\kern0pt}{\isachardoublequoteclose}\isanewline
\ \ \ \ \ \ \ \ \ \ \isacommand{using}\isamarkupfalse%
\ mult{\isacharunderscore}{\kern0pt}distr{\isacharunderscore}{\kern0pt}tensor\ control{\isacharunderscore}{\kern0pt}carrier{\isacharunderscore}{\kern0pt}mat\ n{\isacharunderscore}{\kern0pt}def\ controlled{\isacharunderscore}{\kern0pt}rotations{\isacharunderscore}{\kern0pt}carrier{\isacharunderscore}{\kern0pt}mat\isanewline
\ \ \ \ \ \ \ \ \ \ \ \ state{\isacharunderscore}{\kern0pt}basis{\isacharunderscore}{\kern0pt}carrier{\isacharunderscore}{\kern0pt}mat\ ket{\isacharunderscore}{\kern0pt}vec{\isacharunderscore}{\kern0pt}def\ \isanewline
\ \ \ \ \ \ \ \ \ \ \isacommand{by}\isamarkupfalse%
\ {\isacharparenleft}{\kern0pt}smt\ {\isacharparenleft}{\kern0pt}verit{\isacharparenright}{\kern0pt}\ carrier{\isacharunderscore}{\kern0pt}matD{\isacharparenleft}{\kern0pt}{\isadigit{1}}{\isacharparenright}{\kern0pt}\ carrier{\isacharunderscore}{\kern0pt}matD{\isacharparenleft}{\kern0pt}{\isadigit{2}}{\isacharparenright}{\kern0pt}\ dim{\isacharunderscore}{\kern0pt}col{\isacharunderscore}{\kern0pt}tensor{\isacharunderscore}{\kern0pt}mat\ dim{\isacharunderscore}{\kern0pt}row{\isacharunderscore}{\kern0pt}tensor{\isacharunderscore}{\kern0pt}mat\isanewline
\ \ \ \ \ \ \ \ \ \ \ \ \ \ index{\isacharunderscore}{\kern0pt}add{\isacharunderscore}{\kern0pt}mat{\isacharparenleft}{\kern0pt}{\isadigit{2}}{\isacharparenright}{\kern0pt}\ index{\isacharunderscore}{\kern0pt}add{\isacharunderscore}{\kern0pt}mat{\isacharparenleft}{\kern0pt}{\isadigit{3}}{\isacharparenright}{\kern0pt}\ index{\isacharunderscore}{\kern0pt}one{\isacharunderscore}{\kern0pt}mat{\isacharparenleft}{\kern0pt}{\isadigit{3}}{\isacharparenright}{\kern0pt}\ index{\isacharunderscore}{\kern0pt}smult{\isacharunderscore}{\kern0pt}mat{\isacharparenleft}{\kern0pt}{\isadigit{2}}{\isacharparenright}{\kern0pt}\ \isanewline
\ \ \ \ \ \ \ \ \ \ \ \ \ \ index{\isacharunderscore}{\kern0pt}smult{\isacharunderscore}{\kern0pt}mat{\isacharparenleft}{\kern0pt}{\isadigit{3}}{\isacharparenright}{\kern0pt}\ ket{\isacharunderscore}{\kern0pt}one{\isacharunderscore}{\kern0pt}is{\isacharunderscore}{\kern0pt}state\ one{\isacharunderscore}{\kern0pt}power{\isadigit{2}}\ pos{\isadigit{2}}\ power{\isacharunderscore}{\kern0pt}Suc\ power{\isacharunderscore}{\kern0pt}one{\isacharunderscore}{\kern0pt}right\ \isanewline
\ \ \ \ \ \ \ \ \ \ \ \ \ \ state{\isacharunderscore}{\kern0pt}def\ zero{\isacharunderscore}{\kern0pt}less{\isacharunderscore}{\kern0pt}one{\isacharunderscore}{\kern0pt}class{\isachardot}{\kern0pt}zero{\isacharunderscore}{\kern0pt}less{\isacharunderscore}{\kern0pt}one\ zero{\isacharunderscore}{\kern0pt}less{\isacharunderscore}{\kern0pt}power{\isacharparenright}{\kern0pt}\isanewline
\ \ \ \ \ \ \ \ \isacommand{also}\isamarkupfalse%
\ \isacommand{have}\isamarkupfalse%
\ {\isachardoublequoteopen}{\isasymdots}\ {\isacharequal}{\kern0pt}\ {\isacharparenleft}{\kern0pt}control\ {\isacharparenleft}{\kern0pt}Suc\ n{\isacharparenright}{\kern0pt}\ {\isacharparenleft}{\kern0pt}R\ {\isacharparenleft}{\kern0pt}Suc\ n{\isacharparenright}{\kern0pt}{\isacharparenright}{\kern0pt}{\isacharparenright}{\kern0pt}\ {\isacharasterisk}{\kern0pt}\ \isanewline
\ \ \ \ \ \ \ \ \ \ \ \ \ \ \ \ \ \ \ \ \ \ \ \ {\isacharparenleft}{\kern0pt}{\isacharparenleft}{\kern0pt}\ {\isacharbar}{\kern0pt}zero{\isasymrangle}\ {\isacharplus}{\kern0pt}\ exp\ {\isacharparenleft}{\kern0pt}{\isadigit{2}}{\isacharasterisk}{\kern0pt}{\isasymi}{\isacharasterisk}{\kern0pt}pi{\isacharasterisk}{\kern0pt}complex{\isacharunderscore}{\kern0pt}of{\isacharunderscore}{\kern0pt}nat\ jd{\isadigit{2}}\ {\isacharslash}{\kern0pt}\ {\isadigit{2}}{\isacharcircum}{\kern0pt}n{\isacharparenright}{\kern0pt}\ {\isasymcdot}\isactrlsub m\isanewline
\ \ \ \ \ \ \ \ \ \ \ \ \ \ \ \ \ \ \ \ \ \ \ \ {\isacharbar}{\kern0pt}one{\isasymrangle}\ {\isasymOtimes}\ {\isacharbar}{\kern0pt}state{\isacharunderscore}{\kern0pt}basis\ n{\isacharprime}{\kern0pt}\ {\isacharparenleft}{\kern0pt}jd{\isadigit{2}}\ mod\ {\isadigit{2}}\ {\isacharcircum}{\kern0pt}\ n{\isacharprime}{\kern0pt}{\isacharparenright}{\kern0pt}{\isasymrangle}{\isacharparenright}{\kern0pt}\ {\isasymOtimes}\ \isanewline
\ \ \ \ \ \ \ \ \ \ \ \ \ \ \ \ \ \ \ \ \ \ \ \ {\isacharparenleft}{\kern0pt}{\isacharparenleft}{\kern0pt}{\isadigit{1}}\isactrlsub m\ {\isadigit{2}}{\isacharparenright}{\kern0pt}\ {\isacharasterisk}{\kern0pt}\ {\isacharbar}{\kern0pt}state{\isacharunderscore}{\kern0pt}basis\ {\isadigit{1}}\ jlast{\isasymrangle}{\isacharparenright}{\kern0pt}{\isacharparenright}{\kern0pt}{\isachardoublequoteclose}\isanewline
\ \ \ \ \ \ \ \ \ \ \isacommand{using}\isamarkupfalse%
\ HI\ jd{\isadigit{2}}{\isacharunderscore}{\kern0pt}def\ n{\isacharunderscore}{\kern0pt}def\isanewline
\ \ \ \ \ \ \ \ \ \ \isacommand{by}\isamarkupfalse%
\ {\isacharparenleft}{\kern0pt}smt\ {\isacharparenleft}{\kern0pt}verit{\isacharcomma}{\kern0pt}\ del{\isacharunderscore}{\kern0pt}insts{\isacharparenright}{\kern0pt}\ Suc{\isacharunderscore}{\kern0pt}eq{\isacharunderscore}{\kern0pt}plus{\isadigit{1}}\ div{\isacharunderscore}{\kern0pt}exp{\isacharunderscore}{\kern0pt}eq\ div{\isacharunderscore}{\kern0pt}exp{\isacharunderscore}{\kern0pt}mod{\isacharunderscore}{\kern0pt}exp{\isacharunderscore}{\kern0pt}eq\ jass\ jd{\isadigit{2}}n{\isacharunderscore}{\kern0pt}def\ \isanewline
\ \ \ \ \ \ \ \ \ \ \ \ \ \ jm{\isadigit{2}}n{\isacharunderscore}{\kern0pt}def\ jmm{\isacharunderscore}{\kern0pt}def\ less{\isacharunderscore}{\kern0pt}power{\isacharunderscore}{\kern0pt}add{\isacharunderscore}{\kern0pt}imp{\isacharunderscore}{\kern0pt}div{\isacharunderscore}{\kern0pt}less\ plus{\isacharunderscore}{\kern0pt}{\isadigit{1}}{\isacharunderscore}{\kern0pt}eq{\isacharunderscore}{\kern0pt}Suc\ power{\isacharunderscore}{\kern0pt}one{\isacharunderscore}{\kern0pt}right{\isacharparenright}{\kern0pt}\isanewline
\ \ \ \ \ \ \ \ \isacommand{also}\isamarkupfalse%
\ \isacommand{have}\isamarkupfalse%
\ {\isachardoublequoteopen}{\isasymdots}\ {\isacharequal}{\kern0pt}\ {\isacharparenleft}{\kern0pt}control\ {\isacharparenleft}{\kern0pt}Suc\ n{\isacharparenright}{\kern0pt}\ {\isacharparenleft}{\kern0pt}R\ {\isacharparenleft}{\kern0pt}Suc\ n{\isacharparenright}{\kern0pt}{\isacharparenright}{\kern0pt}{\isacharparenright}{\kern0pt}\ {\isacharasterisk}{\kern0pt}\ \isanewline
\ \ \ \ \ \ \ \ \ \ \ \ \ \ \ \ \ \ \ \ \ \ \ \ {\isacharparenleft}{\kern0pt}{\isacharparenleft}{\kern0pt}\ {\isacharbar}{\kern0pt}zero{\isasymrangle}\ {\isacharplus}{\kern0pt}\ exp\ {\isacharparenleft}{\kern0pt}{\isadigit{2}}{\isacharasterisk}{\kern0pt}{\isasymi}{\isacharasterisk}{\kern0pt}pi{\isacharasterisk}{\kern0pt}complex{\isacharunderscore}{\kern0pt}of{\isacharunderscore}{\kern0pt}nat\ jd{\isadigit{2}}\ {\isacharslash}{\kern0pt}\ {\isadigit{2}}{\isacharcircum}{\kern0pt}n{\isacharparenright}{\kern0pt}\ {\isasymcdot}\isactrlsub m\isanewline
\ \ \ \ \ \ \ \ \ \ \ \ \ \ \ \ \ \ \ \ \ \ \ \ {\isacharbar}{\kern0pt}one{\isasymrangle}\ {\isasymOtimes}\ {\isacharbar}{\kern0pt}state{\isacharunderscore}{\kern0pt}basis\ n{\isacharprime}{\kern0pt}\ jmm{\isasymrangle}{\isacharparenright}{\kern0pt}\ {\isasymOtimes}\ \isanewline
\ \ \ \ \ \ \ \ \ \ \ \ \ \ \ \ \ \ \ \ \ \ \ \ {\isacharbar}{\kern0pt}state{\isacharunderscore}{\kern0pt}basis\ {\isadigit{1}}\ jlast{\isasymrangle}{\isacharparenright}{\kern0pt}{\isachardoublequoteclose}\isanewline
\ \ \ \ \ \ \ \ \ \ \isacommand{using}\isamarkupfalse%
\ jmm{\isacharunderscore}{\kern0pt}def\ jd{\isadigit{2}}{\isacharunderscore}{\kern0pt}def\ \isanewline
\ \ \ \ \ \ \ \ \ \ \isacommand{by}\isamarkupfalse%
\ {\isacharparenleft}{\kern0pt}metis\ div{\isacharunderscore}{\kern0pt}exp{\isacharunderscore}{\kern0pt}mod{\isacharunderscore}{\kern0pt}exp{\isacharunderscore}{\kern0pt}eq\ jm{\isadigit{2}}n{\isacharunderscore}{\kern0pt}def\ left{\isacharunderscore}{\kern0pt}mult{\isacharunderscore}{\kern0pt}one{\isacharunderscore}{\kern0pt}mat\ n{\isacharunderscore}{\kern0pt}def\ plus{\isacharunderscore}{\kern0pt}{\isadigit{1}}{\isacharunderscore}{\kern0pt}eq{\isacharunderscore}{\kern0pt}Suc\isanewline
\ \ \ \ \ \ \ \ \ \ \ \ \ \ power{\isacharunderscore}{\kern0pt}one{\isacharunderscore}{\kern0pt}right\ state{\isacharunderscore}{\kern0pt}basis{\isacharunderscore}{\kern0pt}carrier{\isacharunderscore}{\kern0pt}mat{\isacharparenright}{\kern0pt}\isanewline
\ \ \ \ \ \ \ \ \isacommand{also}\isamarkupfalse%
\ \isacommand{have}\isamarkupfalse%
\ {\isachardoublequoteopen}{\isasymdots}\ {\isacharequal}{\kern0pt}\ {\isacharparenleft}{\kern0pt}\ {\isacharbar}{\kern0pt}zero{\isasymrangle}\ {\isacharplus}{\kern0pt}\ exp\ {\isacharparenleft}{\kern0pt}{\isadigit{2}}{\isacharasterisk}{\kern0pt}{\isasymi}{\isacharasterisk}{\kern0pt}pi{\isacharasterisk}{\kern0pt}complex{\isacharunderscore}{\kern0pt}of{\isacharunderscore}{\kern0pt}nat\ j\ {\isacharslash}{\kern0pt}\ {\isadigit{2}}{\isacharcircum}{\kern0pt}Suc\ n{\isacharparenright}{\kern0pt}\ {\isasymcdot}\isactrlsub m\ {\isacharbar}{\kern0pt}one{\isasymrangle}{\isacharparenright}{\kern0pt}\ {\isasymOtimes}\isanewline
\ \ \ \ \ \ \ \ \ \ \ \ \ \ \ \ \ \ \ \ \ \ \ \ {\isacharbar}{\kern0pt}state{\isacharunderscore}{\kern0pt}basis\ n{\isacharprime}{\kern0pt}\ jmm{\isasymrangle}\ {\isasymOtimes}\ {\isacharbar}{\kern0pt}state{\isacharunderscore}{\kern0pt}basis\ {\isadigit{1}}\ jlast{\isasymrangle}{\isachardoublequoteclose}\isanewline
\ \ \ \ \ \ \ \ \ \ \isacommand{using}\isamarkupfalse%
\ controlR{\isacharunderscore}{\kern0pt}action\ jmm{\isacharunderscore}{\kern0pt}def\ jlast{\isacharunderscore}{\kern0pt}def\ jd{\isadigit{2}}{\isacharunderscore}{\kern0pt}def\ n{\isacharunderscore}{\kern0pt}def\ jm{\isadigit{2}}n{\isacharunderscore}{\kern0pt}def\ jass\ jlastj\ \isacommand{by}\isamarkupfalse%
\ presburger\isanewline
\ \ \ \ \ \ \ \ \isacommand{also}\isamarkupfalse%
\ \isacommand{have}\isamarkupfalse%
\ {\isachardoublequoteopen}{\isasymdots}\ {\isacharequal}{\kern0pt}\ {\isacharparenleft}{\kern0pt}\ {\isacharbar}{\kern0pt}zero{\isasymrangle}\ {\isacharplus}{\kern0pt}\ exp\ {\isacharparenleft}{\kern0pt}{\isadigit{2}}{\isacharasterisk}{\kern0pt}{\isasymi}{\isacharasterisk}{\kern0pt}pi{\isacharasterisk}{\kern0pt}complex{\isacharunderscore}{\kern0pt}of{\isacharunderscore}{\kern0pt}nat\ j\ {\isacharslash}{\kern0pt}\ {\isadigit{2}}{\isacharcircum}{\kern0pt}Suc\ n{\isacharparenright}{\kern0pt}\ {\isasymcdot}\isactrlsub m\ {\isacharbar}{\kern0pt}one{\isasymrangle}{\isacharparenright}{\kern0pt}\ {\isasymOtimes}\isanewline
\ \ \ \ \ \ \ \ \ \ \ \ \ \ \ \ \ \ \ \ \ \ \ \ {\isacharbar}{\kern0pt}state{\isacharunderscore}{\kern0pt}basis\ n\ jm{\isadigit{2}}n{\isasymrangle}{\isachardoublequoteclose}\isanewline
\ \ \ \ \ \ \ \ \ \ \isacommand{using}\isamarkupfalse%
\ state{\isacharunderscore}{\kern0pt}basis{\isacharunderscore}{\kern0pt}dec{\isacharprime}{\kern0pt}\ jm{\isadigit{2}}n{\isacharunderscore}{\kern0pt}def\ jmm{\isacharunderscore}{\kern0pt}def\ jlast{\isacharunderscore}{\kern0pt}def\isanewline
\ \ \ \ \ \ \ \ \ \ \isacommand{by}\isamarkupfalse%
\ {\isacharparenleft}{\kern0pt}metis\ mod{\isacharunderscore}{\kern0pt}less{\isacharunderscore}{\kern0pt}divisor\ n{\isacharunderscore}{\kern0pt}def\ pos{\isadigit{2}}\ tensor{\isacharunderscore}{\kern0pt}mat{\isacharunderscore}{\kern0pt}is{\isacharunderscore}{\kern0pt}assoc\ zero{\isacharunderscore}{\kern0pt}less{\isacharunderscore}{\kern0pt}power{\isacharparenright}{\kern0pt}\isanewline
\ \ \ \ \ \ \ \ \isacommand{finally}\isamarkupfalse%
\ \isacommand{show}\isamarkupfalse%
\ {\isacharquery}{\kern0pt}thesis\ \isacommand{using}\isamarkupfalse%
\ jm{\isadigit{2}}n{\isacharunderscore}{\kern0pt}def\ n{\isacharunderscore}{\kern0pt}def\ jd{\isadigit{2}}n{\isacharunderscore}{\kern0pt}def\ \isacommand{by}\isamarkupfalse%
\ meson\isanewline
\ \ \ \ \ \ \isacommand{qed}\isamarkupfalse%
\isanewline
\ \ \ \ \isacommand{qed}\isamarkupfalse%
\isanewline
\ \ \isacommand{qed}\isamarkupfalse%
\isanewline
\isacommand{qed}\isamarkupfalse%
%
\endisatagproof
{\isafoldproof}%
%
\isadelimproof
\isanewline
%
\endisadelimproof
\isanewline
\isanewline
\isacommand{lemma}\isamarkupfalse%
\ controlled{\isacharunderscore}{\kern0pt}rotations{\isacharunderscore}{\kern0pt}on{\isacharunderscore}{\kern0pt}first{\isacharunderscore}{\kern0pt}qubit{\isacharcolon}{\kern0pt}\isanewline
\ \ \isakeyword{assumes}\ {\isachardoublequoteopen}j\ {\isacharless}{\kern0pt}\ {\isadigit{2}}\ {\isacharcircum}{\kern0pt}\ Suc\ n{\isachardoublequoteclose}\isanewline
\ \ \isakeyword{shows}\ {\isachardoublequoteopen}controlled{\isacharunderscore}{\kern0pt}rotations\ {\isacharparenleft}{\kern0pt}Suc\ n{\isacharparenright}{\kern0pt}\ {\isacharasterisk}{\kern0pt}\isanewline
\ \ \ \ \ \ \ \ {\isacharparenleft}{\kern0pt}{\isadigit{1}}{\isacharslash}{\kern0pt}sqrt\ {\isadigit{2}}\ {\isasymcdot}\isactrlsub m\ {\isacharparenleft}{\kern0pt}\ {\isacharbar}{\kern0pt}zero{\isasymrangle}\ {\isacharplus}{\kern0pt}\ exp{\isacharparenleft}{\kern0pt}{\isadigit{2}}{\isacharasterisk}{\kern0pt}{\isasymi}{\isacharasterisk}{\kern0pt}pi{\isacharasterisk}{\kern0pt}{\isacharparenleft}{\kern0pt}complex{\isacharunderscore}{\kern0pt}of{\isacharunderscore}{\kern0pt}nat\ {\isacharparenleft}{\kern0pt}j\ div\ {\isadigit{2}}{\isacharcircum}{\kern0pt}n{\isacharparenright}{\kern0pt}{\isacharparenright}{\kern0pt}{\isacharslash}{\kern0pt}{\isadigit{2}}{\isacharparenright}{\kern0pt}\ {\isasymcdot}\isactrlsub m\ {\isacharbar}{\kern0pt}one{\isasymrangle}{\isacharparenright}{\kern0pt}\ {\isasymOtimes}\ \isanewline
\ \ \ \ \ \ \ \ {\isacharbar}{\kern0pt}state{\isacharunderscore}{\kern0pt}basis\ n\ {\isacharparenleft}{\kern0pt}j\ mod\ {\isadigit{2}}{\isacharcircum}{\kern0pt}n{\isacharparenright}{\kern0pt}{\isasymrangle}{\isacharparenright}{\kern0pt}\ {\isacharequal}{\kern0pt}\isanewline
\ \ \ \ \ \ \ \ {\isacharparenleft}{\kern0pt}{\isadigit{1}}{\isacharslash}{\kern0pt}sqrt\ {\isadigit{2}}\ {\isasymcdot}\isactrlsub m\ {\isacharparenleft}{\kern0pt}{\isacharparenleft}{\kern0pt}\ {\isacharbar}{\kern0pt}zero{\isasymrangle}\ {\isacharplus}{\kern0pt}\ exp{\isacharparenleft}{\kern0pt}{\isadigit{2}}{\isacharasterisk}{\kern0pt}{\isasymi}{\isacharasterisk}{\kern0pt}pi{\isacharasterisk}{\kern0pt}j{\isacharslash}{\kern0pt}{\isacharparenleft}{\kern0pt}{\isadigit{2}}{\isacharcircum}{\kern0pt}{\isacharparenleft}{\kern0pt}Suc\ n{\isacharparenright}{\kern0pt}{\isacharparenright}{\kern0pt}{\isacharparenright}{\kern0pt}\ {\isasymcdot}\isactrlsub m\ {\isacharbar}{\kern0pt}one{\isasymrangle}{\isacharparenright}{\kern0pt}{\isacharparenright}{\kern0pt}\ {\isasymOtimes}\ {\isacharbar}{\kern0pt}state{\isacharunderscore}{\kern0pt}basis\ n\ {\isacharparenleft}{\kern0pt}j\ mod\ {\isadigit{2}}{\isacharcircum}{\kern0pt}n{\isacharparenright}{\kern0pt}{\isasymrangle}{\isacharparenright}{\kern0pt}{\isachardoublequoteclose}\isanewline
%
\isadelimproof
%
\endisadelimproof
%
\isatagproof
\isacommand{proof}\isamarkupfalse%
\ {\isacharminus}{\kern0pt}\isanewline
\ \ \isacommand{have}\isamarkupfalse%
\ {\isachardoublequoteopen}controlled{\isacharunderscore}{\kern0pt}rotations\ {\isacharparenleft}{\kern0pt}Suc\ n{\isacharparenright}{\kern0pt}\ {\isacharasterisk}{\kern0pt}\isanewline
\ \ \ \ \ \ \ \ {\isacharparenleft}{\kern0pt}{\isadigit{1}}{\isacharslash}{\kern0pt}sqrt\ {\isadigit{2}}\ {\isasymcdot}\isactrlsub m\ {\isacharparenleft}{\kern0pt}\ {\isacharbar}{\kern0pt}zero{\isasymrangle}\ {\isacharplus}{\kern0pt}\ exp{\isacharparenleft}{\kern0pt}{\isadigit{2}}{\isacharasterisk}{\kern0pt}{\isasymi}{\isacharasterisk}{\kern0pt}pi{\isacharasterisk}{\kern0pt}{\isacharparenleft}{\kern0pt}complex{\isacharunderscore}{\kern0pt}of{\isacharunderscore}{\kern0pt}nat\ {\isacharparenleft}{\kern0pt}j\ div\ {\isadigit{2}}{\isacharcircum}{\kern0pt}n{\isacharparenright}{\kern0pt}{\isacharparenright}{\kern0pt}{\isacharslash}{\kern0pt}{\isadigit{2}}{\isacharparenright}{\kern0pt}\ {\isasymcdot}\isactrlsub m\ {\isacharbar}{\kern0pt}one{\isasymrangle}{\isacharparenright}{\kern0pt}\ {\isasymOtimes}\ \isanewline
\ \ \ \ \ \ \ \ {\isacharbar}{\kern0pt}state{\isacharunderscore}{\kern0pt}basis\ n\ {\isacharparenleft}{\kern0pt}j\ mod\ {\isadigit{2}}{\isacharcircum}{\kern0pt}n{\isacharparenright}{\kern0pt}{\isasymrangle}{\isacharparenright}{\kern0pt}\ {\isacharequal}{\kern0pt}\ \isanewline
\ \ \ \ \ \ \ \ controlled{\isacharunderscore}{\kern0pt}rotations\ {\isacharparenleft}{\kern0pt}Suc\ n{\isacharparenright}{\kern0pt}\ {\isacharasterisk}{\kern0pt}\isanewline
\ \ \ \ \ \ \ \ {\isacharparenleft}{\kern0pt}{\isadigit{1}}{\isacharslash}{\kern0pt}sqrt\ {\isadigit{2}}\ {\isasymcdot}\isactrlsub m\ {\isacharparenleft}{\kern0pt}{\isacharparenleft}{\kern0pt}\ {\isacharbar}{\kern0pt}zero{\isasymrangle}\ {\isacharplus}{\kern0pt}\ exp{\isacharparenleft}{\kern0pt}{\isadigit{2}}{\isacharasterisk}{\kern0pt}{\isasymi}{\isacharasterisk}{\kern0pt}pi{\isacharasterisk}{\kern0pt}{\isacharparenleft}{\kern0pt}complex{\isacharunderscore}{\kern0pt}of{\isacharunderscore}{\kern0pt}nat\ {\isacharparenleft}{\kern0pt}j\ div\ {\isadigit{2}}{\isacharcircum}{\kern0pt}n{\isacharparenright}{\kern0pt}{\isacharparenright}{\kern0pt}{\isacharslash}{\kern0pt}{\isadigit{2}}{\isacharparenright}{\kern0pt}\ {\isasymcdot}\isactrlsub m\ {\isacharbar}{\kern0pt}one{\isasymrangle}{\isacharparenright}{\kern0pt}\ {\isasymOtimes}\ \isanewline
\ \ \ \ \ \ \ \ {\isacharbar}{\kern0pt}state{\isacharunderscore}{\kern0pt}basis\ n\ {\isacharparenleft}{\kern0pt}j\ mod\ {\isadigit{2}}{\isacharcircum}{\kern0pt}n{\isacharparenright}{\kern0pt}{\isasymrangle}{\isacharparenright}{\kern0pt}{\isacharparenright}{\kern0pt}{\isachardoublequoteclose}\isanewline
\ \ \ \ \isacommand{using}\isamarkupfalse%
\ smult{\isacharunderscore}{\kern0pt}mat{\isacharunderscore}{\kern0pt}def\ tensor{\isacharunderscore}{\kern0pt}mat{\isacharunderscore}{\kern0pt}def\ \isanewline
\ \ \ \ \isacommand{by}\isamarkupfalse%
\ {\isacharparenleft}{\kern0pt}smt\ {\isacharparenleft}{\kern0pt}verit{\isacharparenright}{\kern0pt}\ One{\isacharunderscore}{\kern0pt}nat{\isacharunderscore}{\kern0pt}def\ carrier{\isacharunderscore}{\kern0pt}matD{\isacharparenleft}{\kern0pt}{\isadigit{2}}{\isacharparenright}{\kern0pt}\ index{\isacharunderscore}{\kern0pt}add{\isacharunderscore}{\kern0pt}mat{\isacharparenleft}{\kern0pt}{\isadigit{3}}{\isacharparenright}{\kern0pt}\ index{\isacharunderscore}{\kern0pt}smult{\isacharunderscore}{\kern0pt}mat{\isacharparenleft}{\kern0pt}{\isadigit{3}}{\isacharparenright}{\kern0pt}\ lessI\ power{\isacharunderscore}{\kern0pt}one{\isacharunderscore}{\kern0pt}right\ smult{\isacharunderscore}{\kern0pt}tensor{\isadigit{1}}\ state{\isacharunderscore}{\kern0pt}basis{\isacharunderscore}{\kern0pt}carrier{\isacharunderscore}{\kern0pt}mat\ state{\isacharunderscore}{\kern0pt}basis{\isacharunderscore}{\kern0pt}def{\isacharparenright}{\kern0pt}\isanewline
\ \ \isacommand{also}\isamarkupfalse%
\ \isacommand{have}\isamarkupfalse%
\ {\isachardoublequoteopen}{\isasymdots}\ {\isacharequal}{\kern0pt}\ {\isadigit{1}}{\isacharslash}{\kern0pt}sqrt\ {\isadigit{2}}\ {\isasymcdot}\isactrlsub m\ {\isacharparenleft}{\kern0pt}controlled{\isacharunderscore}{\kern0pt}rotations\ {\isacharparenleft}{\kern0pt}Suc\ n{\isacharparenright}{\kern0pt}\ {\isacharasterisk}{\kern0pt}\ \isanewline
\ \ \ \ \ \ \ \ \ \ \ \ \ \ \ \ \ \ {\isacharparenleft}{\kern0pt}{\isacharparenleft}{\kern0pt}\ {\isacharbar}{\kern0pt}zero{\isasymrangle}\ {\isacharplus}{\kern0pt}\ exp{\isacharparenleft}{\kern0pt}{\isadigit{2}}{\isacharasterisk}{\kern0pt}{\isasymi}{\isacharasterisk}{\kern0pt}pi{\isacharasterisk}{\kern0pt}{\isacharparenleft}{\kern0pt}complex{\isacharunderscore}{\kern0pt}of{\isacharunderscore}{\kern0pt}nat\ {\isacharparenleft}{\kern0pt}j\ div\ {\isadigit{2}}{\isacharcircum}{\kern0pt}n{\isacharparenright}{\kern0pt}{\isacharparenright}{\kern0pt}{\isacharslash}{\kern0pt}{\isadigit{2}}{\isacharparenright}{\kern0pt}\ {\isasymcdot}\isactrlsub m\ {\isacharbar}{\kern0pt}one{\isasymrangle}{\isacharparenright}{\kern0pt}\ {\isasymOtimes}\ \isanewline
\ \ \ \ \ \ \ \ \ \ \ \ \ \ \ \ \ \ {\isacharbar}{\kern0pt}state{\isacharunderscore}{\kern0pt}basis\ n\ {\isacharparenleft}{\kern0pt}j\ mod\ {\isadigit{2}}{\isacharcircum}{\kern0pt}n{\isacharparenright}{\kern0pt}{\isasymrangle}{\isacharparenright}{\kern0pt}{\isacharparenright}{\kern0pt}{\isachardoublequoteclose}\isanewline
\ \ \ \ \isacommand{using}\isamarkupfalse%
\ mult{\isacharunderscore}{\kern0pt}smult{\isacharunderscore}{\kern0pt}distrib\ controlled{\isacharunderscore}{\kern0pt}rotations{\isacharunderscore}{\kern0pt}carrier{\isacharunderscore}{\kern0pt}mat\ state{\isacharunderscore}{\kern0pt}basis{\isacharunderscore}{\kern0pt}carrier{\isacharunderscore}{\kern0pt}mat\isanewline
\ \ \ \ \isacommand{by}\isamarkupfalse%
\ {\isacharparenleft}{\kern0pt}smt\ {\isacharparenleft}{\kern0pt}verit{\isacharparenright}{\kern0pt}\ carrier{\isacharunderscore}{\kern0pt}matI\ dim{\isacharunderscore}{\kern0pt}row{\isacharunderscore}{\kern0pt}mat{\isacharparenleft}{\kern0pt}{\isadigit{1}}{\isacharparenright}{\kern0pt}\ dim{\isacharunderscore}{\kern0pt}row{\isacharunderscore}{\kern0pt}tensor{\isacharunderscore}{\kern0pt}mat\ index{\isacharunderscore}{\kern0pt}add{\isacharunderscore}{\kern0pt}mat{\isacharparenleft}{\kern0pt}{\isadigit{2}}{\isacharparenright}{\kern0pt}\ \isanewline
\ \ \ \ \ \ \ \ index{\isacharunderscore}{\kern0pt}smult{\isacharunderscore}{\kern0pt}mat{\isacharparenleft}{\kern0pt}{\isadigit{2}}{\isacharparenright}{\kern0pt}\ index{\isacharunderscore}{\kern0pt}unit{\isacharunderscore}{\kern0pt}vec{\isacharparenleft}{\kern0pt}{\isadigit{3}}{\isacharparenright}{\kern0pt}\ ket{\isacharunderscore}{\kern0pt}vec{\isacharunderscore}{\kern0pt}def\ power{\isacharunderscore}{\kern0pt}Suc\ state{\isacharunderscore}{\kern0pt}basis{\isacharunderscore}{\kern0pt}def{\isacharparenright}{\kern0pt}\isanewline
\ \ \isacommand{also}\isamarkupfalse%
\ \isacommand{have}\isamarkupfalse%
\ {\isachardoublequoteopen}{\isasymdots}\ {\isacharequal}{\kern0pt}\ {\isacharparenleft}{\kern0pt}{\isadigit{1}}{\isacharslash}{\kern0pt}sqrt\ {\isadigit{2}}\ {\isasymcdot}\isactrlsub m\ \isanewline
\ \ \ \ \ \ \ \ \ \ \ \ \ \ \ \ \ \ {\isacharparenleft}{\kern0pt}{\isacharparenleft}{\kern0pt}\ {\isacharbar}{\kern0pt}zero{\isasymrangle}\ {\isacharplus}{\kern0pt}\ exp{\isacharparenleft}{\kern0pt}{\isadigit{2}}{\isacharasterisk}{\kern0pt}{\isasymi}{\isacharasterisk}{\kern0pt}pi{\isacharasterisk}{\kern0pt}j{\isacharslash}{\kern0pt}{\isacharparenleft}{\kern0pt}{\isadigit{2}}{\isacharcircum}{\kern0pt}{\isacharparenleft}{\kern0pt}Suc\ n{\isacharparenright}{\kern0pt}{\isacharparenright}{\kern0pt}{\isacharparenright}{\kern0pt}\ {\isasymcdot}\isactrlsub m\ {\isacharbar}{\kern0pt}one{\isasymrangle}{\isacharparenright}{\kern0pt}{\isacharparenright}{\kern0pt}\ {\isasymOtimes}\ {\isacharbar}{\kern0pt}state{\isacharunderscore}{\kern0pt}basis\ n\ {\isacharparenleft}{\kern0pt}j\ mod\ {\isadigit{2}}{\isacharcircum}{\kern0pt}n{\isacharparenright}{\kern0pt}{\isasymrangle}{\isacharparenright}{\kern0pt}{\isachardoublequoteclose}\isanewline
\ \ \ \ \isacommand{using}\isamarkupfalse%
\ assms\ controlled{\isacharunderscore}{\kern0pt}rotations{\isacharunderscore}{\kern0pt}ind\ ket{\isacharunderscore}{\kern0pt}vec{\isacharunderscore}{\kern0pt}def\ \isacommand{by}\isamarkupfalse%
\ simp\isanewline
\ \ \isacommand{finally}\isamarkupfalse%
\ \isacommand{show}\isamarkupfalse%
\ {\isacharquery}{\kern0pt}thesis\ \isacommand{by}\isamarkupfalse%
\ this\isanewline
\isacommand{qed}\isamarkupfalse%
%
\endisatagproof
{\isafoldproof}%
%
\isadelimproof
%
\endisadelimproof
%
\begin{isamarkuptext}%
More useful lemmas:%
\end{isamarkuptext}\isamarkuptrue%
\isacommand{lemma}\isamarkupfalse%
\ exp{\isacharunderscore}{\kern0pt}j{\isacharcolon}{\kern0pt}\isanewline
\ \ \isakeyword{assumes}\ {\isachardoublequoteopen}l\ {\isacharless}{\kern0pt}\ Suc\ n{\isachardoublequoteclose}\isanewline
\ \ \isakeyword{shows}\ {\isachardoublequoteopen}exp\ {\isacharparenleft}{\kern0pt}{\isadigit{2}}{\isacharasterisk}{\kern0pt}{\isasymi}{\isacharasterisk}{\kern0pt}pi{\isacharasterisk}{\kern0pt}j{\isacharslash}{\kern0pt}{\isacharparenleft}{\kern0pt}{\isadigit{2}}{\isacharcircum}{\kern0pt}l{\isacharparenright}{\kern0pt}{\isacharparenright}{\kern0pt}\ {\isacharequal}{\kern0pt}\ exp\ {\isacharparenleft}{\kern0pt}{\isadigit{2}}{\isacharasterisk}{\kern0pt}{\isasymi}{\isacharasterisk}{\kern0pt}pi{\isacharasterisk}{\kern0pt}{\isacharparenleft}{\kern0pt}j\ mod\ {\isadigit{2}}{\isacharcircum}{\kern0pt}n{\isacharparenright}{\kern0pt}{\isacharslash}{\kern0pt}{\isacharparenleft}{\kern0pt}{\isadigit{2}}{\isacharcircum}{\kern0pt}l{\isacharparenright}{\kern0pt}{\isacharparenright}{\kern0pt}{\isachardoublequoteclose}\isanewline
%
\isadelimproof
%
\endisadelimproof
%
\isatagproof
\isacommand{proof}\isamarkupfalse%
\ {\isacharminus}{\kern0pt}\isanewline
\ \ \isacommand{define}\isamarkupfalse%
\ jd\ jm\ \isakeyword{where}\ {\isachardoublequoteopen}jd\ {\isacharequal}{\kern0pt}\ j\ div\ {\isadigit{2}}{\isacharcircum}{\kern0pt}n{\isachardoublequoteclose}\ \isakeyword{and}\ {\isachardoublequoteopen}jm\ {\isacharequal}{\kern0pt}\ j\ mod\ {\isadigit{2}}{\isacharcircum}{\kern0pt}n{\isachardoublequoteclose}\isanewline
\ \ \isacommand{have}\isamarkupfalse%
\ {\isadigit{0}}{\isacharcolon}{\kern0pt}{\isachardoublequoteopen}real\ {\isacharparenleft}{\kern0pt}{\isadigit{2}}{\isacharcircum}{\kern0pt}n{\isacharparenright}{\kern0pt}{\isacharslash}{\kern0pt}{\isacharparenleft}{\kern0pt}{\isadigit{2}}{\isacharcircum}{\kern0pt}l{\isacharparenright}{\kern0pt}\ {\isacharequal}{\kern0pt}\ {\isacharparenleft}{\kern0pt}{\isadigit{2}}{\isacharcircum}{\kern0pt}{\isacharparenleft}{\kern0pt}n{\isacharminus}{\kern0pt}l{\isacharparenright}{\kern0pt}{\isacharparenright}{\kern0pt}{\isachardoublequoteclose}\isanewline
\ \ \isacommand{proof}\isamarkupfalse%
\ {\isacharminus}{\kern0pt}\isanewline
\ \ \ \ \isacommand{have}\isamarkupfalse%
\ {\isadigit{1}}{\isacharcolon}{\kern0pt}{\isachardoublequoteopen}{\isacharparenleft}{\kern0pt}{\isadigit{2}}{\isacharcolon}{\kern0pt}{\isacharcolon}{\kern0pt}nat{\isacharparenright}{\kern0pt}\ {\isasymnoteq}\ {\isadigit{0}}{\isachardoublequoteclose}\ \isacommand{by}\isamarkupfalse%
\ simp\isanewline
\ \ \ \ \isacommand{have}\isamarkupfalse%
\ {\isadigit{2}}{\isacharcolon}{\kern0pt}{\isachardoublequoteopen}l\ {\isasymle}\ n{\isachardoublequoteclose}\ \isacommand{using}\isamarkupfalse%
\ assms\ \isacommand{by}\isamarkupfalse%
\ simp\isanewline
\ \ \ \ \isacommand{show}\isamarkupfalse%
\ {\isacharquery}{\kern0pt}thesis\isanewline
\ \ \ \ \ \ \isacommand{using}\isamarkupfalse%
\ {\isadigit{1}}\ {\isadigit{2}}\ power{\isacharunderscore}{\kern0pt}diff\isanewline
\ \ \ \ \ \ \isacommand{by}\isamarkupfalse%
\ {\isacharparenleft}{\kern0pt}metis\ numeral{\isacharunderscore}{\kern0pt}power{\isacharunderscore}{\kern0pt}eq{\isacharunderscore}{\kern0pt}of{\isacharunderscore}{\kern0pt}nat{\isacharunderscore}{\kern0pt}cancel{\isacharunderscore}{\kern0pt}iff\ zero{\isacharunderscore}{\kern0pt}neq{\isacharunderscore}{\kern0pt}numeral{\isacharparenright}{\kern0pt}\isanewline
\ \ \isacommand{qed}\isamarkupfalse%
\isanewline
\ \ \isacommand{have}\isamarkupfalse%
\ {\isachardoublequoteopen}j\ {\isacharequal}{\kern0pt}\ jd{\isacharasterisk}{\kern0pt}{\isacharparenleft}{\kern0pt}{\isadigit{2}}{\isacharcircum}{\kern0pt}n{\isacharparenright}{\kern0pt}\ {\isacharplus}{\kern0pt}\ jm{\isachardoublequoteclose}\ \isacommand{using}\isamarkupfalse%
\ jd{\isacharunderscore}{\kern0pt}def\ jm{\isacharunderscore}{\kern0pt}def\ \isacommand{by}\isamarkupfalse%
\ presburger\isanewline
\ \ \isacommand{hence}\isamarkupfalse%
\ {\isachardoublequoteopen}exp\ {\isacharparenleft}{\kern0pt}{\isadigit{2}}{\isacharasterisk}{\kern0pt}{\isasymi}{\isacharasterisk}{\kern0pt}pi{\isacharasterisk}{\kern0pt}j{\isacharslash}{\kern0pt}{\isacharparenleft}{\kern0pt}{\isadigit{2}}{\isacharcircum}{\kern0pt}l{\isacharparenright}{\kern0pt}{\isacharparenright}{\kern0pt}\ {\isacharequal}{\kern0pt}\ exp\ {\isacharparenleft}{\kern0pt}{\isadigit{2}}{\isacharasterisk}{\kern0pt}pi{\isacharasterisk}{\kern0pt}{\isasymi}{\isacharasterisk}{\kern0pt}{\isacharparenleft}{\kern0pt}jd{\isacharasterisk}{\kern0pt}{\isacharparenleft}{\kern0pt}{\isadigit{2}}{\isacharcircum}{\kern0pt}n{\isacharparenright}{\kern0pt}\ {\isacharplus}{\kern0pt}\ jm{\isacharparenright}{\kern0pt}{\isacharslash}{\kern0pt}{\isacharparenleft}{\kern0pt}{\isadigit{2}}{\isacharcircum}{\kern0pt}l{\isacharparenright}{\kern0pt}{\isacharparenright}{\kern0pt}{\isachardoublequoteclose}\isanewline
\ \ \ \ \isacommand{by}\isamarkupfalse%
\ {\isacharparenleft}{\kern0pt}simp\ add{\isacharcolon}{\kern0pt}\ mult{\isachardot}{\kern0pt}commute\ mult{\isachardot}{\kern0pt}left{\isacharunderscore}{\kern0pt}commute{\isacharparenright}{\kern0pt}\isanewline
\ \ \isacommand{also}\isamarkupfalse%
\ \isacommand{have}\isamarkupfalse%
\ {\isachardoublequoteopen}{\isasymdots}\ {\isacharequal}{\kern0pt}\ exp\ {\isacharparenleft}{\kern0pt}{\isadigit{2}}{\isacharasterisk}{\kern0pt}pi{\isacharasterisk}{\kern0pt}{\isasymi}{\isacharasterisk}{\kern0pt}{\isacharparenleft}{\kern0pt}jd{\isacharasterisk}{\kern0pt}{\isacharparenleft}{\kern0pt}{\isadigit{2}}{\isacharcircum}{\kern0pt}n{\isacharparenright}{\kern0pt}{\isacharparenright}{\kern0pt}{\isacharslash}{\kern0pt}{\isacharparenleft}{\kern0pt}{\isadigit{2}}{\isacharcircum}{\kern0pt}l{\isacharparenright}{\kern0pt}\ {\isacharplus}{\kern0pt}\ {\isadigit{2}}{\isacharasterisk}{\kern0pt}{\isasymi}{\isacharasterisk}{\kern0pt}pi{\isacharasterisk}{\kern0pt}jm{\isacharslash}{\kern0pt}{\isacharparenleft}{\kern0pt}{\isadigit{2}}{\isacharcircum}{\kern0pt}l{\isacharparenright}{\kern0pt}{\isacharparenright}{\kern0pt}{\isachardoublequoteclose}\isanewline
\ \ \ \ \isacommand{by}\isamarkupfalse%
\ {\isacharparenleft}{\kern0pt}simp\ add{\isacharcolon}{\kern0pt}\ add{\isacharunderscore}{\kern0pt}divide{\isacharunderscore}{\kern0pt}distrib\ distrib{\isacharunderscore}{\kern0pt}left\ mult{\isachardot}{\kern0pt}left{\isacharunderscore}{\kern0pt}commute\ semigroup{\isacharunderscore}{\kern0pt}mult{\isacharunderscore}{\kern0pt}class{\isachardot}{\kern0pt}mult{\isachardot}{\kern0pt}assoc{\isacharparenright}{\kern0pt}\isanewline
\ \ \isacommand{also}\isamarkupfalse%
\ \isacommand{have}\isamarkupfalse%
\ {\isachardoublequoteopen}{\isasymdots}\ {\isacharequal}{\kern0pt}\ exp\ {\isacharparenleft}{\kern0pt}{\isadigit{2}}{\isacharasterisk}{\kern0pt}pi{\isacharasterisk}{\kern0pt}{\isasymi}{\isacharasterisk}{\kern0pt}{\isacharparenleft}{\kern0pt}jd{\isacharasterisk}{\kern0pt}{\isacharparenleft}{\kern0pt}{\isadigit{2}}{\isacharcircum}{\kern0pt}n{\isacharparenright}{\kern0pt}{\isacharparenright}{\kern0pt}{\isacharslash}{\kern0pt}{\isacharparenleft}{\kern0pt}{\isadigit{2}}{\isacharcircum}{\kern0pt}l{\isacharparenright}{\kern0pt}{\isacharparenright}{\kern0pt}\ {\isacharasterisk}{\kern0pt}\ exp\ {\isacharparenleft}{\kern0pt}{\isadigit{2}}{\isacharasterisk}{\kern0pt}{\isasymi}{\isacharasterisk}{\kern0pt}pi{\isacharasterisk}{\kern0pt}jm{\isacharslash}{\kern0pt}{\isacharparenleft}{\kern0pt}{\isadigit{2}}{\isacharcircum}{\kern0pt}l{\isacharparenright}{\kern0pt}{\isacharparenright}{\kern0pt}{\isachardoublequoteclose}\ \isacommand{using}\isamarkupfalse%
\ exp{\isacharunderscore}{\kern0pt}add\ \isacommand{by}\isamarkupfalse%
\ blast\isanewline
\ \ \isacommand{also}\isamarkupfalse%
\ \isacommand{have}\isamarkupfalse%
\ {\isachardoublequoteopen}{\isasymdots}\ {\isacharequal}{\kern0pt}\ exp\ {\isacharparenleft}{\kern0pt}{\isadigit{2}}{\isacharasterisk}{\kern0pt}pi{\isacharasterisk}{\kern0pt}{\isasymi}{\isacharasterisk}{\kern0pt}jd{\isacharasterisk}{\kern0pt}{\isacharparenleft}{\kern0pt}{\isacharparenleft}{\kern0pt}{\isadigit{2}}{\isacharcircum}{\kern0pt}n{\isacharparenright}{\kern0pt}{\isacharslash}{\kern0pt}{\isacharparenleft}{\kern0pt}{\isadigit{2}}{\isacharcircum}{\kern0pt}l{\isacharparenright}{\kern0pt}{\isacharparenright}{\kern0pt}{\isacharparenright}{\kern0pt}\ {\isacharasterisk}{\kern0pt}\ exp\ {\isacharparenleft}{\kern0pt}{\isadigit{2}}{\isacharasterisk}{\kern0pt}{\isasymi}{\isacharasterisk}{\kern0pt}pi{\isacharasterisk}{\kern0pt}jm{\isacharslash}{\kern0pt}{\isacharparenleft}{\kern0pt}{\isadigit{2}}{\isacharcircum}{\kern0pt}l{\isacharparenright}{\kern0pt}{\isacharparenright}{\kern0pt}{\isachardoublequoteclose}\isanewline
\ \ \ \ \isacommand{by}\isamarkupfalse%
\ {\isacharparenleft}{\kern0pt}simp\ add{\isacharcolon}{\kern0pt}\ semigroup{\isacharunderscore}{\kern0pt}mult{\isacharunderscore}{\kern0pt}class{\isachardot}{\kern0pt}mult{\isachardot}{\kern0pt}assoc{\isacharparenright}{\kern0pt}\isanewline
\ \ \isacommand{also}\isamarkupfalse%
\ \isacommand{have}\isamarkupfalse%
\ {\isachardoublequoteopen}{\isasymdots}\ {\isacharequal}{\kern0pt}\ exp\ {\isacharparenleft}{\kern0pt}{\isadigit{2}}{\isacharasterisk}{\kern0pt}pi{\isacharasterisk}{\kern0pt}{\isasymi}{\isacharasterisk}{\kern0pt}jd{\isacharasterisk}{\kern0pt}{\isacharparenleft}{\kern0pt}{\isacharparenleft}{\kern0pt}{\isadigit{2}}{\isacharcircum}{\kern0pt}{\isacharparenleft}{\kern0pt}n{\isacharminus}{\kern0pt}l{\isacharparenright}{\kern0pt}{\isacharparenright}{\kern0pt}{\isacharparenright}{\kern0pt}{\isacharparenright}{\kern0pt}\ {\isacharasterisk}{\kern0pt}\ exp\ {\isacharparenleft}{\kern0pt}{\isadigit{2}}{\isacharasterisk}{\kern0pt}{\isasymi}{\isacharasterisk}{\kern0pt}pi{\isacharasterisk}{\kern0pt}jm{\isacharslash}{\kern0pt}{\isacharparenleft}{\kern0pt}{\isadigit{2}}{\isacharcircum}{\kern0pt}l{\isacharparenright}{\kern0pt}{\isacharparenright}{\kern0pt}{\isachardoublequoteclose}\ \isanewline
\ \ \ \ \isacommand{using}\isamarkupfalse%
\ {\isadigit{0}}\ \isacommand{by}\isamarkupfalse%
\ {\isacharparenleft}{\kern0pt}smt\ {\isacharparenleft}{\kern0pt}verit{\isacharparenright}{\kern0pt}\ dbl{\isacharunderscore}{\kern0pt}simps{\isacharparenleft}{\kern0pt}{\isadigit{3}}{\isacharparenright}{\kern0pt}\ dbl{\isacharunderscore}{\kern0pt}simps{\isacharparenleft}{\kern0pt}{\isadigit{5}}{\isacharparenright}{\kern0pt}\ numerals{\isacharparenleft}{\kern0pt}{\isadigit{1}}{\isacharparenright}{\kern0pt}\ of{\isacharunderscore}{\kern0pt}nat{\isacharunderscore}{\kern0pt}{\isadigit{1}}\ of{\isacharunderscore}{\kern0pt}nat{\isacharunderscore}{\kern0pt}numeral\ \isanewline
\ \ \ \ \ \ \ \ of{\isacharunderscore}{\kern0pt}nat{\isacharunderscore}{\kern0pt}power\ of{\isacharunderscore}{\kern0pt}real{\isacharunderscore}{\kern0pt}divide\ of{\isacharunderscore}{\kern0pt}real{\isacharunderscore}{\kern0pt}of{\isacharunderscore}{\kern0pt}nat{\isacharunderscore}{\kern0pt}eq{\isacharparenright}{\kern0pt}\isanewline
\ \ \isacommand{also}\isamarkupfalse%
\ \isacommand{have}\isamarkupfalse%
\ {\isachardoublequoteopen}{\isasymdots}\ {\isacharequal}{\kern0pt}\ exp\ {\isacharparenleft}{\kern0pt}{\isacharparenleft}{\kern0pt}{\isadigit{2}}{\isacharasterisk}{\kern0pt}pi{\isacharasterisk}{\kern0pt}{\isasymi}{\isacharasterisk}{\kern0pt}jd{\isacharparenright}{\kern0pt}{\isacharasterisk}{\kern0pt}{\isacharparenleft}{\kern0pt}of{\isacharunderscore}{\kern0pt}nat\ {\isacharparenleft}{\kern0pt}{\isadigit{2}}{\isacharcircum}{\kern0pt}{\isacharparenleft}{\kern0pt}n{\isacharminus}{\kern0pt}l{\isacharparenright}{\kern0pt}{\isacharparenright}{\kern0pt}{\isacharparenright}{\kern0pt}{\isacharparenright}{\kern0pt}\ {\isacharasterisk}{\kern0pt}\ exp\ {\isacharparenleft}{\kern0pt}{\isadigit{2}}{\isacharasterisk}{\kern0pt}{\isasymi}{\isacharasterisk}{\kern0pt}pi{\isacharasterisk}{\kern0pt}jm{\isacharslash}{\kern0pt}{\isacharparenleft}{\kern0pt}{\isadigit{2}}{\isacharcircum}{\kern0pt}l{\isacharparenright}{\kern0pt}{\isacharparenright}{\kern0pt}{\isachardoublequoteclose}\ \isacommand{by}\isamarkupfalse%
\ auto\isanewline
\ \ \isacommand{also}\isamarkupfalse%
\ \isacommand{have}\isamarkupfalse%
\ {\isachardoublequoteopen}{\isasymdots}\ {\isacharequal}{\kern0pt}\ {\isacharparenleft}{\kern0pt}exp\ {\isacharparenleft}{\kern0pt}{\isadigit{2}}{\isacharasterisk}{\kern0pt}pi{\isacharasterisk}{\kern0pt}{\isasymi}{\isacharparenright}{\kern0pt}{\isacharparenright}{\kern0pt}{\isacharcircum}{\kern0pt}{\isacharparenleft}{\kern0pt}{\isadigit{2}}{\isacharcircum}{\kern0pt}{\isacharparenleft}{\kern0pt}n{\isacharminus}{\kern0pt}l{\isacharparenright}{\kern0pt}{\isacharparenright}{\kern0pt}\ {\isacharasterisk}{\kern0pt}\ exp\ {\isacharparenleft}{\kern0pt}{\isadigit{2}}{\isacharasterisk}{\kern0pt}{\isasymi}{\isacharasterisk}{\kern0pt}pi{\isacharasterisk}{\kern0pt}jm{\isacharslash}{\kern0pt}{\isacharparenleft}{\kern0pt}{\isadigit{2}}{\isacharcircum}{\kern0pt}l{\isacharparenright}{\kern0pt}{\isacharparenright}{\kern0pt}{\isachardoublequoteclose}\ \isanewline
\ \ \ \ \isacommand{using}\isamarkupfalse%
\ exp{\isacharunderscore}{\kern0pt}of{\isacharunderscore}{\kern0pt}nat{\isadigit{2}}{\isacharunderscore}{\kern0pt}mult\ \isacommand{by}\isamarkupfalse%
\ {\isacharparenleft}{\kern0pt}smt\ {\isacharparenleft}{\kern0pt}verit{\isacharcomma}{\kern0pt}\ best{\isacharparenright}{\kern0pt}\ cis{\isacharunderscore}{\kern0pt}{\isadigit{2}}pi\ cis{\isacharunderscore}{\kern0pt}conv{\isacharunderscore}{\kern0pt}exp\ exp{\isacharunderscore}{\kern0pt}power{\isacharunderscore}{\kern0pt}int\ exp{\isacharunderscore}{\kern0pt}zero\ \isanewline
\ \ \ \ \ \ \ \ mult{\isachardot}{\kern0pt}commute\ mult{\isacharunderscore}{\kern0pt}zero{\isacharunderscore}{\kern0pt}right{\isacharparenright}{\kern0pt}\isanewline
\ \ \isacommand{also}\isamarkupfalse%
\ \isacommand{have}\isamarkupfalse%
\ {\isachardoublequoteopen}{\isasymdots}\ {\isacharequal}{\kern0pt}\ {\isadigit{1}}{\isacharcircum}{\kern0pt}{\isacharparenleft}{\kern0pt}{\isadigit{2}}{\isacharcircum}{\kern0pt}{\isacharparenleft}{\kern0pt}n{\isacharminus}{\kern0pt}l{\isacharparenright}{\kern0pt}{\isacharparenright}{\kern0pt}\ {\isacharasterisk}{\kern0pt}\ exp\ {\isacharparenleft}{\kern0pt}{\isadigit{2}}{\isacharasterisk}{\kern0pt}{\isasymi}{\isacharasterisk}{\kern0pt}pi{\isacharasterisk}{\kern0pt}jm{\isacharslash}{\kern0pt}{\isacharparenleft}{\kern0pt}{\isadigit{2}}{\isacharcircum}{\kern0pt}l{\isacharparenright}{\kern0pt}{\isacharparenright}{\kern0pt}{\isachardoublequoteclose}\ \isacommand{using}\isamarkupfalse%
\ exp{\isacharunderscore}{\kern0pt}two{\isacharunderscore}{\kern0pt}pi{\isacharunderscore}{\kern0pt}i\ \isacommand{by}\isamarkupfalse%
\ auto\isanewline
\ \ \isacommand{also}\isamarkupfalse%
\ \isacommand{have}\isamarkupfalse%
\ {\isachardoublequoteopen}{\isasymdots}\ {\isacharequal}{\kern0pt}\ exp\ {\isacharparenleft}{\kern0pt}{\isadigit{2}}{\isacharasterisk}{\kern0pt}{\isasymi}{\isacharasterisk}{\kern0pt}pi{\isacharasterisk}{\kern0pt}jm{\isacharslash}{\kern0pt}{\isacharparenleft}{\kern0pt}{\isadigit{2}}{\isacharcircum}{\kern0pt}l{\isacharparenright}{\kern0pt}{\isacharparenright}{\kern0pt}{\isachardoublequoteclose}\ \isacommand{by}\isamarkupfalse%
\ auto\isanewline
\ \ \isacommand{finally}\isamarkupfalse%
\ \isacommand{show}\isamarkupfalse%
\ {\isacharquery}{\kern0pt}thesis\ \isacommand{using}\isamarkupfalse%
\ jd{\isacharunderscore}{\kern0pt}def\ jm{\isacharunderscore}{\kern0pt}def\ \isacommand{by}\isamarkupfalse%
\ simp\isanewline
\isacommand{qed}\isamarkupfalse%
%
\endisatagproof
{\isafoldproof}%
%
\isadelimproof
\isanewline
%
\endisadelimproof
\ \ \isanewline
\isanewline
\isanewline
\isacommand{lemma}\isamarkupfalse%
\ kron{\isacharunderscore}{\kern0pt}list{\isacharunderscore}{\kern0pt}fun{\isacharbrackleft}{\kern0pt}simp{\isacharbrackright}{\kern0pt}{\isacharcolon}{\kern0pt}\isanewline
\ \ {\isachardoublequoteopen}{\isasymforall}x{\isachardot}{\kern0pt}\ List{\isachardot}{\kern0pt}member\ xs\ x\ {\isasymlongrightarrow}\ f\ x\ {\isacharequal}{\kern0pt}\ g\ x\ {\isasymLongrightarrow}\ kron\ f\ xs\ {\isacharequal}{\kern0pt}\ kron\ g\ xs{\isachardoublequoteclose}\isanewline
%
\isadelimproof
%
\endisadelimproof
%
\isatagproof
\isacommand{proof}\isamarkupfalse%
\ {\isacharparenleft}{\kern0pt}induct\ xs{\isacharparenright}{\kern0pt}\isanewline
\ \ \isacommand{case}\isamarkupfalse%
\ Nil\isanewline
\ \ \isacommand{show}\isamarkupfalse%
\ {\isachardoublequoteopen}kron\ f\ {\isacharbrackleft}{\kern0pt}{\isacharbrackright}{\kern0pt}\ {\isacharequal}{\kern0pt}\ kron\ g\ {\isacharbrackleft}{\kern0pt}{\isacharbrackright}{\kern0pt}{\isachardoublequoteclose}\ \isacommand{by}\isamarkupfalse%
\ simp\isanewline
\isacommand{next}\isamarkupfalse%
\isanewline
\ \ \isacommand{fix}\isamarkupfalse%
\ a\ xs\isanewline
\ \ \isacommand{assume}\isamarkupfalse%
\ HI{\isacharcolon}{\kern0pt}{\isachardoublequoteopen}{\isacharparenleft}{\kern0pt}{\isasymforall}x{\isachardot}{\kern0pt}\ List{\isachardot}{\kern0pt}member\ xs\ x\ {\isasymlongrightarrow}\ f\ x\ {\isacharequal}{\kern0pt}\ g\ x\ {\isasymLongrightarrow}\ kron\ f\ xs\ {\isacharequal}{\kern0pt}\ kron\ g\ xs{\isacharparenright}{\kern0pt}{\isachardoublequoteclose}\isanewline
\ \ \isacommand{show}\isamarkupfalse%
\ {\isachardoublequoteopen}{\isasymforall}x{\isachardot}{\kern0pt}\ List{\isachardot}{\kern0pt}member\ {\isacharparenleft}{\kern0pt}a\ {\isacharhash}{\kern0pt}\ xs{\isacharparenright}{\kern0pt}\ x\ {\isasymlongrightarrow}\ f\ x\ {\isacharequal}{\kern0pt}\ g\ x\ {\isasymLongrightarrow}\ kron\ f\ {\isacharparenleft}{\kern0pt}a\ {\isacharhash}{\kern0pt}\ xs{\isacharparenright}{\kern0pt}\ {\isacharequal}{\kern0pt}\ kron\ g\ {\isacharparenleft}{\kern0pt}a\ {\isacharhash}{\kern0pt}\ xs{\isacharparenright}{\kern0pt}{\isachardoublequoteclose}\isanewline
\ \ \isacommand{proof}\isamarkupfalse%
\ {\isacharminus}{\kern0pt}\isanewline
\ \ \ \ \isacommand{assume}\isamarkupfalse%
\ {\isadigit{1}}{\isacharcolon}{\kern0pt}{\isachardoublequoteopen}{\isasymforall}x{\isachardot}{\kern0pt}\ List{\isachardot}{\kern0pt}member\ {\isacharparenleft}{\kern0pt}a\ {\isacharhash}{\kern0pt}\ xs{\isacharparenright}{\kern0pt}\ x\ {\isasymlongrightarrow}\ f\ x\ {\isacharequal}{\kern0pt}\ g\ x{\isachardoublequoteclose}\isanewline
\ \ \ \ \isacommand{show}\isamarkupfalse%
\ {\isachardoublequoteopen}kron\ f\ {\isacharparenleft}{\kern0pt}a\ {\isacharhash}{\kern0pt}\ xs{\isacharparenright}{\kern0pt}\ {\isacharequal}{\kern0pt}\ kron\ g\ {\isacharparenleft}{\kern0pt}a\ {\isacharhash}{\kern0pt}\ xs{\isacharparenright}{\kern0pt}{\isachardoublequoteclose}\isanewline
\ \ \ \ \isacommand{proof}\isamarkupfalse%
\ {\isacharminus}{\kern0pt}\isanewline
\ \ \ \ \ \ \isacommand{from}\isamarkupfalse%
\ {\isadigit{1}}\ \isacommand{have}\isamarkupfalse%
\ {\isachardoublequoteopen}List{\isachardot}{\kern0pt}member\ {\isacharparenleft}{\kern0pt}a\ {\isacharhash}{\kern0pt}\ xs{\isacharparenright}{\kern0pt}\ a\ {\isasymlongrightarrow}\ f\ a\ {\isacharequal}{\kern0pt}\ g\ a{\isachardoublequoteclose}\ \isacommand{by}\isamarkupfalse%
\ auto\isanewline
\ \ \ \ \ \ \isacommand{moreover}\isamarkupfalse%
\ \isacommand{have}\isamarkupfalse%
\ {\isachardoublequoteopen}List{\isachardot}{\kern0pt}member\ {\isacharparenleft}{\kern0pt}a\ {\isacharhash}{\kern0pt}\ xs{\isacharparenright}{\kern0pt}\ a{\isachardoublequoteclose}\ \isacommand{by}\isamarkupfalse%
\ {\isacharparenleft}{\kern0pt}simp\ add{\isacharcolon}{\kern0pt}\ List{\isachardot}{\kern0pt}member{\isacharunderscore}{\kern0pt}rec{\isacharparenleft}{\kern0pt}{\isadigit{1}}{\isacharparenright}{\kern0pt}{\isacharparenright}{\kern0pt}\isanewline
\ \ \ \ \ \ \isacommand{ultimately}\isamarkupfalse%
\ \isacommand{have}\isamarkupfalse%
\ {\isadigit{2}}{\isacharcolon}{\kern0pt}{\isachardoublequoteopen}f\ a\ {\isacharequal}{\kern0pt}\ g\ a{\isachardoublequoteclose}\ \isacommand{by}\isamarkupfalse%
\ auto\isanewline
\ \ \ \ \ \ \isacommand{have}\isamarkupfalse%
\ {\isachardoublequoteopen}kron\ f\ {\isacharparenleft}{\kern0pt}a{\isacharhash}{\kern0pt}xs{\isacharparenright}{\kern0pt}\ {\isacharequal}{\kern0pt}\ f\ a\ {\isasymOtimes}\ kron\ f\ xs{\isachardoublequoteclose}\ \isacommand{by}\isamarkupfalse%
\ simp\isanewline
\ \ \ \ \ \ \isacommand{also}\isamarkupfalse%
\ \isacommand{have}\isamarkupfalse%
\ {\isachardoublequoteopen}{\isasymdots}\ {\isacharequal}{\kern0pt}\ g\ a\ {\isasymOtimes}\ kron\ f\ xs{\isachardoublequoteclose}\ \isacommand{using}\isamarkupfalse%
\ {\isadigit{2}}\ \isacommand{by}\isamarkupfalse%
\ simp\isanewline
\ \ \ \ \ \ \isacommand{also}\isamarkupfalse%
\ \isacommand{have}\isamarkupfalse%
\ {\isachardoublequoteopen}{\isasymdots}\ {\isacharequal}{\kern0pt}\ g\ a\ {\isasymOtimes}\ kron\ g\ xs{\isachardoublequoteclose}\ \isacommand{using}\isamarkupfalse%
\ HI\ {\isadigit{1}}\ \isacommand{by}\isamarkupfalse%
\ {\isacharparenleft}{\kern0pt}simp\ add{\isacharcolon}{\kern0pt}\ member{\isacharunderscore}{\kern0pt}rec{\isacharparenleft}{\kern0pt}{\isadigit{1}}{\isacharparenright}{\kern0pt}{\isacharparenright}{\kern0pt}\isanewline
\ \ \ \ \ \ \isacommand{also}\isamarkupfalse%
\ \isacommand{have}\isamarkupfalse%
\ {\isachardoublequoteopen}{\isasymdots}\ {\isacharequal}{\kern0pt}\ kron\ g\ {\isacharparenleft}{\kern0pt}a{\isacharhash}{\kern0pt}xs{\isacharparenright}{\kern0pt}{\isachardoublequoteclose}\ \isacommand{using}\isamarkupfalse%
\ kron{\isachardot}{\kern0pt}simps{\isacharparenleft}{\kern0pt}{\isadigit{2}}{\isacharparenright}{\kern0pt}\ \isacommand{by}\isamarkupfalse%
\ simp\isanewline
\ \ \ \ \ \ \isacommand{finally}\isamarkupfalse%
\ \isacommand{show}\isamarkupfalse%
\ {\isacharquery}{\kern0pt}thesis\ \isacommand{by}\isamarkupfalse%
\ this\isanewline
\ \ \ \ \isacommand{qed}\isamarkupfalse%
\isanewline
\ \ \isacommand{qed}\isamarkupfalse%
\isanewline
\isacommand{qed}\isamarkupfalse%
%
\endisatagproof
{\isafoldproof}%
%
\isadelimproof
\isanewline
%
\endisadelimproof
\ \ \ \ \ \isanewline
\isanewline
\isacommand{lemma}\isamarkupfalse%
\ member{\isacharunderscore}{\kern0pt}rev{\isacharcolon}{\kern0pt}\isanewline
\ \ \isakeyword{shows}\ {\isachardoublequoteopen}List{\isachardot}{\kern0pt}member\ {\isacharparenleft}{\kern0pt}rev\ xs{\isacharparenright}{\kern0pt}\ x\ {\isacharequal}{\kern0pt}\ List{\isachardot}{\kern0pt}member\ xs\ x{\isachardoublequoteclose}\isanewline
%
\isadelimproof
%
\endisadelimproof
%
\isatagproof
\isacommand{proof}\isamarkupfalse%
\ {\isacharparenleft}{\kern0pt}induct\ xs{\isacharparenright}{\kern0pt}\isanewline
\ \ \isacommand{show}\isamarkupfalse%
\ {\isachardoublequoteopen}List{\isachardot}{\kern0pt}member\ {\isacharparenleft}{\kern0pt}rev\ {\isacharbrackleft}{\kern0pt}{\isacharbrackright}{\kern0pt}{\isacharparenright}{\kern0pt}\ x\ {\isacharequal}{\kern0pt}\ List{\isachardot}{\kern0pt}member\ {\isacharbrackleft}{\kern0pt}{\isacharbrackright}{\kern0pt}\ x{\isachardoublequoteclose}\ \isacommand{by}\isamarkupfalse%
\ simp\isanewline
\isacommand{next}\isamarkupfalse%
\isanewline
\ \ \isacommand{case}\isamarkupfalse%
\ {\isacharparenleft}{\kern0pt}Cons\ a\ xs{\isacharparenright}{\kern0pt}\isanewline
\ \ \isacommand{assume}\isamarkupfalse%
\ HI{\isacharcolon}{\kern0pt}{\isachardoublequoteopen}List{\isachardot}{\kern0pt}member\ {\isacharparenleft}{\kern0pt}rev\ xs{\isacharparenright}{\kern0pt}\ x\ {\isacharequal}{\kern0pt}\ List{\isachardot}{\kern0pt}member\ xs\ x{\isachardoublequoteclose}\isanewline
\ \ \isacommand{have}\isamarkupfalse%
\ {\isachardoublequoteopen}List{\isachardot}{\kern0pt}member\ {\isacharparenleft}{\kern0pt}rev\ {\isacharparenleft}{\kern0pt}a{\isacharhash}{\kern0pt}xs{\isacharparenright}{\kern0pt}{\isacharparenright}{\kern0pt}\ x\ {\isacharequal}{\kern0pt}\ List{\isachardot}{\kern0pt}member\ {\isacharparenleft}{\kern0pt}{\isacharparenleft}{\kern0pt}rev\ xs{\isacharparenright}{\kern0pt}{\isacharat}{\kern0pt}{\isacharbrackleft}{\kern0pt}a{\isacharbrackright}{\kern0pt}{\isacharparenright}{\kern0pt}\ x{\isachardoublequoteclose}\ \isacommand{using}\isamarkupfalse%
\ rev{\isacharunderscore}{\kern0pt}append\ \isacommand{by}\isamarkupfalse%
\ auto\isanewline
\ \ \isacommand{also}\isamarkupfalse%
\ \isacommand{have}\isamarkupfalse%
\ {\isachardoublequoteopen}{\isasymdots}\ {\isacharequal}{\kern0pt}\ {\isacharparenleft}{\kern0pt}x\ {\isasymin}\ set\ {\isacharparenleft}{\kern0pt}{\isacharparenleft}{\kern0pt}rev\ xs{\isacharparenright}{\kern0pt}\ {\isacharat}{\kern0pt}\ {\isacharbrackleft}{\kern0pt}a{\isacharbrackright}{\kern0pt}{\isacharparenright}{\kern0pt}{\isacharparenright}{\kern0pt}{\isachardoublequoteclose}\ \isacommand{using}\isamarkupfalse%
\ List{\isachardot}{\kern0pt}member{\isacharunderscore}{\kern0pt}def\ \isacommand{by}\isamarkupfalse%
\ metis\isanewline
\ \ \isacommand{also}\isamarkupfalse%
\ \isacommand{have}\isamarkupfalse%
\ {\isachardoublequoteopen}{\isasymdots}\ {\isacharequal}{\kern0pt}\ {\isacharparenleft}{\kern0pt}x\ {\isasymin}\ set\ {\isacharparenleft}{\kern0pt}rev\ xs{\isacharparenright}{\kern0pt}\ {\isasymunion}\ set\ {\isacharbrackleft}{\kern0pt}a{\isacharbrackright}{\kern0pt}{\isacharparenright}{\kern0pt}{\isachardoublequoteclose}\ \isacommand{using}\isamarkupfalse%
\ set{\isacharunderscore}{\kern0pt}append\ \isacommand{by}\isamarkupfalse%
\ metis\isanewline
\ \ \isacommand{also}\isamarkupfalse%
\ \isacommand{have}\isamarkupfalse%
\ {\isachardoublequoteopen}{\isasymdots}\ {\isacharequal}{\kern0pt}\ {\isacharparenleft}{\kern0pt}x\ {\isasymin}\ set\ {\isacharbrackleft}{\kern0pt}a{\isacharbrackright}{\kern0pt}\ {\isasymor}\ x\ {\isasymin}\ set\ {\isacharparenleft}{\kern0pt}rev\ xs{\isacharparenright}{\kern0pt}{\isacharparenright}{\kern0pt}{\isachardoublequoteclose}\ \isacommand{by}\isamarkupfalse%
\ blast\isanewline
\ \ \isacommand{also}\isamarkupfalse%
\ \isacommand{have}\isamarkupfalse%
\ {\isachardoublequoteopen}{\isasymdots}\ {\isacharequal}{\kern0pt}\ {\isacharparenleft}{\kern0pt}x\ {\isacharequal}{\kern0pt}\ a\ {\isasymor}\ List{\isachardot}{\kern0pt}member\ {\isacharparenleft}{\kern0pt}rev\ xs{\isacharparenright}{\kern0pt}\ x{\isacharparenright}{\kern0pt}{\isachardoublequoteclose}\ \isacommand{using}\isamarkupfalse%
\ List{\isachardot}{\kern0pt}member{\isacharunderscore}{\kern0pt}def\ \isacommand{by}\isamarkupfalse%
\ fastforce\isanewline
\ \ \isacommand{also}\isamarkupfalse%
\ \isacommand{have}\isamarkupfalse%
\ {\isachardoublequoteopen}{\isasymdots}\ {\isacharequal}{\kern0pt}\ {\isacharparenleft}{\kern0pt}x\ {\isacharequal}{\kern0pt}\ a\ {\isasymor}\ List{\isachardot}{\kern0pt}member\ xs\ x{\isacharparenright}{\kern0pt}{\isachardoublequoteclose}\ \isacommand{using}\isamarkupfalse%
\ HI\ \isacommand{by}\isamarkupfalse%
\ metis\isanewline
\ \ \isacommand{also}\isamarkupfalse%
\ \isacommand{have}\isamarkupfalse%
\ {\isachardoublequoteopen}{\isasymdots}\ {\isacharequal}{\kern0pt}\ List{\isachardot}{\kern0pt}member\ {\isacharparenleft}{\kern0pt}a{\isacharhash}{\kern0pt}xs{\isacharparenright}{\kern0pt}\ x{\isachardoublequoteclose}\ \isacommand{using}\isamarkupfalse%
\ List{\isachardot}{\kern0pt}member{\isacharunderscore}{\kern0pt}rec{\isacharparenleft}{\kern0pt}{\isadigit{1}}{\isacharparenright}{\kern0pt}\ \isacommand{by}\isamarkupfalse%
\ metis\isanewline
\ \ \isacommand{finally}\isamarkupfalse%
\ \isacommand{show}\isamarkupfalse%
\ {\isachardoublequoteopen}List{\isachardot}{\kern0pt}member\ {\isacharparenleft}{\kern0pt}rev\ {\isacharparenleft}{\kern0pt}a{\isacharhash}{\kern0pt}xs{\isacharparenright}{\kern0pt}{\isacharparenright}{\kern0pt}\ x\ {\isacharequal}{\kern0pt}\ List{\isachardot}{\kern0pt}member\ {\isacharparenleft}{\kern0pt}a{\isacharhash}{\kern0pt}xs{\isacharparenright}{\kern0pt}\ x{\isachardoublequoteclose}\ \isacommand{by}\isamarkupfalse%
\ this\isanewline
\isacommand{qed}\isamarkupfalse%
%
\endisatagproof
{\isafoldproof}%
%
\isadelimproof
\isanewline
%
\endisadelimproof
\isanewline
\isanewline
\isacommand{lemma}\isamarkupfalse%
\ kron{\isacharunderscore}{\kern0pt}j{\isacharcolon}{\kern0pt}\isanewline
\ \ \isakeyword{shows}\ {\isachardoublequoteopen}kron\ {\isacharparenleft}{\kern0pt}{\isasymlambda}{\isacharparenleft}{\kern0pt}l{\isacharcolon}{\kern0pt}{\isacharcolon}{\kern0pt}nat{\isacharparenright}{\kern0pt}{\isachardot}{\kern0pt}\ {\isacharbar}{\kern0pt}zero{\isasymrangle}\ {\isacharplus}{\kern0pt}\ exp\ {\isacharparenleft}{\kern0pt}{\isadigit{2}}{\isacharasterisk}{\kern0pt}{\isasymi}{\isacharasterisk}{\kern0pt}pi{\isacharasterisk}{\kern0pt}j{\isacharslash}{\kern0pt}{\isacharparenleft}{\kern0pt}{\isadigit{2}}{\isacharcircum}{\kern0pt}l{\isacharparenright}{\kern0pt}{\isacharparenright}{\kern0pt}\ {\isasymcdot}\isactrlsub m\ {\isacharbar}{\kern0pt}one{\isasymrangle}{\isacharparenright}{\kern0pt}\ {\isacharparenleft}{\kern0pt}map\ nat\ {\isacharparenleft}{\kern0pt}rev\ {\isacharbrackleft}{\kern0pt}{\isadigit{1}}{\isachardot}{\kern0pt}{\isachardot}{\kern0pt}n{\isacharbrackright}{\kern0pt}{\isacharparenright}{\kern0pt}{\isacharparenright}{\kern0pt}\ {\isacharequal}{\kern0pt}\isanewline
\ \ \ \ \ \ \ \ \ kron\ {\isacharparenleft}{\kern0pt}{\isasymlambda}{\isacharparenleft}{\kern0pt}l{\isacharcolon}{\kern0pt}{\isacharcolon}{\kern0pt}nat{\isacharparenright}{\kern0pt}{\isachardot}{\kern0pt}\ {\isacharbar}{\kern0pt}zero{\isasymrangle}\ {\isacharplus}{\kern0pt}\ exp\ {\isacharparenleft}{\kern0pt}{\isadigit{2}}{\isacharasterisk}{\kern0pt}{\isasymi}{\isacharasterisk}{\kern0pt}pi{\isacharasterisk}{\kern0pt}{\isacharparenleft}{\kern0pt}complex{\isacharunderscore}{\kern0pt}of{\isacharunderscore}{\kern0pt}nat\ {\isacharparenleft}{\kern0pt}j\ mod\ {\isadigit{2}}{\isacharcircum}{\kern0pt}n{\isacharparenright}{\kern0pt}{\isacharparenright}{\kern0pt}{\isacharslash}{\kern0pt}{\isacharparenleft}{\kern0pt}{\isadigit{2}}{\isacharcircum}{\kern0pt}l{\isacharparenright}{\kern0pt}{\isacharparenright}{\kern0pt}\ {\isasymcdot}\isactrlsub m\ {\isacharbar}{\kern0pt}one{\isasymrangle}{\isacharparenright}{\kern0pt}\ \isanewline
\ \ \ \ \ \ \ \ \ {\isacharparenleft}{\kern0pt}map\ nat\ {\isacharparenleft}{\kern0pt}rev\ {\isacharbrackleft}{\kern0pt}{\isadigit{1}}{\isachardot}{\kern0pt}{\isachardot}{\kern0pt}n{\isacharbrackright}{\kern0pt}{\isacharparenright}{\kern0pt}{\isacharparenright}{\kern0pt}{\isachardoublequoteclose}\isanewline
%
\isadelimproof
%
\endisadelimproof
%
\isatagproof
\isacommand{proof}\isamarkupfalse%
\ {\isacharminus}{\kern0pt}\isanewline
\ \ \isacommand{define}\isamarkupfalse%
\ fj\ fjm\ \isakeyword{where}\ {\isachardoublequoteopen}fj\ {\isacharequal}{\kern0pt}\ {\isacharparenleft}{\kern0pt}{\isasymlambda}{\isacharparenleft}{\kern0pt}l{\isacharcolon}{\kern0pt}{\isacharcolon}{\kern0pt}nat{\isacharparenright}{\kern0pt}{\isachardot}{\kern0pt}\ {\isacharbar}{\kern0pt}zero{\isasymrangle}\ {\isacharplus}{\kern0pt}\ exp\ {\isacharparenleft}{\kern0pt}{\isadigit{2}}{\isacharasterisk}{\kern0pt}{\isasymi}{\isacharasterisk}{\kern0pt}pi{\isacharasterisk}{\kern0pt}j{\isacharslash}{\kern0pt}{\isacharparenleft}{\kern0pt}{\isadigit{2}}{\isacharcircum}{\kern0pt}l{\isacharparenright}{\kern0pt}{\isacharparenright}{\kern0pt}\ {\isasymcdot}\isactrlsub m\ {\isacharbar}{\kern0pt}one{\isasymrangle}{\isacharparenright}{\kern0pt}{\isachardoublequoteclose}\isanewline
\ \ \ \ \isakeyword{and}\ {\isachardoublequoteopen}fjm\ {\isacharequal}{\kern0pt}\ {\isacharparenleft}{\kern0pt}{\isasymlambda}{\isacharparenleft}{\kern0pt}l{\isacharcolon}{\kern0pt}{\isacharcolon}{\kern0pt}nat{\isacharparenright}{\kern0pt}{\isachardot}{\kern0pt}\ {\isacharbar}{\kern0pt}zero{\isasymrangle}\ {\isacharplus}{\kern0pt}\ exp\ {\isacharparenleft}{\kern0pt}{\isadigit{2}}{\isacharasterisk}{\kern0pt}{\isasymi}{\isacharasterisk}{\kern0pt}pi{\isacharasterisk}{\kern0pt}{\isacharparenleft}{\kern0pt}complex{\isacharunderscore}{\kern0pt}of{\isacharunderscore}{\kern0pt}nat\ {\isacharparenleft}{\kern0pt}j\ mod\ {\isadigit{2}}{\isacharcircum}{\kern0pt}n{\isacharparenright}{\kern0pt}{\isacharparenright}{\kern0pt}{\isacharslash}{\kern0pt}{\isacharparenleft}{\kern0pt}{\isadigit{2}}{\isacharcircum}{\kern0pt}l{\isacharparenright}{\kern0pt}{\isacharparenright}{\kern0pt}\ {\isasymcdot}\isactrlsub m\ {\isacharbar}{\kern0pt}one{\isasymrangle}{\isacharparenright}{\kern0pt}{\isachardoublequoteclose}\ \isanewline
\ \ \isacommand{have}\isamarkupfalse%
\ {\isachardoublequoteopen}{\isasymforall}x{\isachardot}{\kern0pt}\ {\isacharparenleft}{\kern0pt}{\isacharparenleft}{\kern0pt}List{\isachardot}{\kern0pt}member\ {\isacharparenleft}{\kern0pt}map\ nat\ {\isacharparenleft}{\kern0pt}rev\ {\isacharbrackleft}{\kern0pt}{\isadigit{1}}{\isachardot}{\kern0pt}{\isachardot}{\kern0pt}n{\isacharbrackright}{\kern0pt}{\isacharparenright}{\kern0pt}{\isacharparenright}{\kern0pt}\ x{\isacharparenright}{\kern0pt}\ {\isasymlongrightarrow}\ {\isacharparenleft}{\kern0pt}x\ {\isacharless}{\kern0pt}\ Suc\ n{\isacharparenright}{\kern0pt}{\isacharparenright}{\kern0pt}{\isachardoublequoteclose}\isanewline
\ \ \isacommand{proof}\isamarkupfalse%
\ {\isacharparenleft}{\kern0pt}rule\ allI{\isacharparenright}{\kern0pt}\isanewline
\ \ \ \ \isacommand{fix}\isamarkupfalse%
\ x\isanewline
\ \ \ \ \isacommand{show}\isamarkupfalse%
\ {\isachardoublequoteopen}List{\isachardot}{\kern0pt}member\ {\isacharparenleft}{\kern0pt}map\ nat\ {\isacharparenleft}{\kern0pt}rev\ {\isacharbrackleft}{\kern0pt}{\isadigit{1}}{\isachardot}{\kern0pt}{\isachardot}{\kern0pt}int\ n{\isacharbrackright}{\kern0pt}{\isacharparenright}{\kern0pt}{\isacharparenright}{\kern0pt}\ x\ {\isasymlongrightarrow}\ x\ {\isacharless}{\kern0pt}\ Suc\ n{\isachardoublequoteclose}\isanewline
\ \ \ \ \isacommand{proof}\isamarkupfalse%
\isanewline
\ \ \ \ \ \ \isacommand{assume}\isamarkupfalse%
\ {\isachardoublequoteopen}List{\isachardot}{\kern0pt}member\ {\isacharparenleft}{\kern0pt}map\ nat\ {\isacharparenleft}{\kern0pt}rev\ {\isacharbrackleft}{\kern0pt}{\isadigit{1}}{\isachardot}{\kern0pt}{\isachardot}{\kern0pt}int\ n{\isacharbrackright}{\kern0pt}{\isacharparenright}{\kern0pt}{\isacharparenright}{\kern0pt}\ x{\isachardoublequoteclose}\isanewline
\ \ \ \ \ \ \isacommand{hence}\isamarkupfalse%
\ {\isachardoublequoteopen}List{\isachardot}{\kern0pt}member\ {\isacharparenleft}{\kern0pt}rev\ {\isacharparenleft}{\kern0pt}map\ nat\ {\isacharbrackleft}{\kern0pt}{\isadigit{1}}{\isachardot}{\kern0pt}{\isachardot}{\kern0pt}int\ n{\isacharbrackright}{\kern0pt}{\isacharparenright}{\kern0pt}{\isacharparenright}{\kern0pt}\ x{\isachardoublequoteclose}\ \isacommand{using}\isamarkupfalse%
\ rev{\isacharunderscore}{\kern0pt}map\ \isacommand{by}\isamarkupfalse%
\ metis\isanewline
\ \ \ \ \ \ \isacommand{hence}\isamarkupfalse%
\ {\isachardoublequoteopen}List{\isachardot}{\kern0pt}member\ {\isacharparenleft}{\kern0pt}map\ nat\ {\isacharbrackleft}{\kern0pt}{\isadigit{1}}{\isachardot}{\kern0pt}{\isachardot}{\kern0pt}int\ n{\isacharbrackright}{\kern0pt}{\isacharparenright}{\kern0pt}\ x{\isachardoublequoteclose}\ \isacommand{using}\isamarkupfalse%
\ member{\isacharunderscore}{\kern0pt}rev\ \isacommand{by}\isamarkupfalse%
\ metis\isanewline
\ \ \ \ \ \ \isacommand{hence}\isamarkupfalse%
\ {\isachardoublequoteopen}x\ {\isasymin}\ set\ {\isacharparenleft}{\kern0pt}map\ nat\ {\isacharbrackleft}{\kern0pt}{\isadigit{1}}{\isachardot}{\kern0pt}{\isachardot}{\kern0pt}int\ n{\isacharbrackright}{\kern0pt}{\isacharparenright}{\kern0pt}{\isachardoublequoteclose}\ \isacommand{using}\isamarkupfalse%
\ List{\isachardot}{\kern0pt}member{\isacharunderscore}{\kern0pt}def\ \isacommand{by}\isamarkupfalse%
\ metis\isanewline
\ \ \ \ \ \ \isacommand{hence}\isamarkupfalse%
\ {\isachardoublequoteopen}x\ {\isasymin}\ {\isacharbraceleft}{\kern0pt}{\isadigit{1}}{\isachardot}{\kern0pt}{\isachardot}{\kern0pt}n{\isacharbraceright}{\kern0pt}{\isachardoublequoteclose}\ \isacommand{by}\isamarkupfalse%
\ auto\isanewline
\ \ \ \ \ \ \isacommand{thus}\isamarkupfalse%
\ {\isachardoublequoteopen}x\ {\isacharless}{\kern0pt}\ Suc\ n{\isachardoublequoteclose}\ \isacommand{by}\isamarkupfalse%
\ auto\isanewline
\ \ \ \ \isacommand{qed}\isamarkupfalse%
\isanewline
\ \ \isacommand{qed}\isamarkupfalse%
\isanewline
\ \ \isacommand{hence}\isamarkupfalse%
\ {\isachardoublequoteopen}{\isasymforall}x{\isachardot}{\kern0pt}\ {\isacharparenleft}{\kern0pt}{\isacharparenleft}{\kern0pt}List{\isachardot}{\kern0pt}member\ {\isacharparenleft}{\kern0pt}map\ nat\ {\isacharparenleft}{\kern0pt}rev\ {\isacharbrackleft}{\kern0pt}{\isadigit{1}}{\isachardot}{\kern0pt}{\isachardot}{\kern0pt}n{\isacharbrackright}{\kern0pt}{\isacharparenright}{\kern0pt}{\isacharparenright}{\kern0pt}\ x{\isacharparenright}{\kern0pt}\ {\isasymlongrightarrow}\ \isanewline
\ \ \ \ \ \ \ \ \ \ \ \ \ {\isacharparenleft}{\kern0pt}exp\ {\isacharparenleft}{\kern0pt}{\isadigit{2}}{\isacharasterisk}{\kern0pt}{\isasymi}{\isacharasterisk}{\kern0pt}pi{\isacharasterisk}{\kern0pt}j{\isacharslash}{\kern0pt}{\isacharparenleft}{\kern0pt}{\isadigit{2}}{\isacharcircum}{\kern0pt}x{\isacharparenright}{\kern0pt}{\isacharparenright}{\kern0pt}\ {\isacharequal}{\kern0pt}\ exp\ {\isacharparenleft}{\kern0pt}{\isadigit{2}}{\isacharasterisk}{\kern0pt}{\isasymi}{\isacharasterisk}{\kern0pt}pi{\isacharasterisk}{\kern0pt}{\isacharparenleft}{\kern0pt}j\ mod\ {\isadigit{2}}{\isacharcircum}{\kern0pt}n{\isacharparenright}{\kern0pt}{\isacharslash}{\kern0pt}{\isacharparenleft}{\kern0pt}{\isadigit{2}}{\isacharcircum}{\kern0pt}x{\isacharparenright}{\kern0pt}{\isacharparenright}{\kern0pt}{\isacharparenright}{\kern0pt}{\isacharparenright}{\kern0pt}{\isachardoublequoteclose}\isanewline
\ \ \ \ \isacommand{using}\isamarkupfalse%
\ exp{\isacharunderscore}{\kern0pt}j\isanewline
\ \ \ \ \isacommand{by}\isamarkupfalse%
\ {\isacharparenleft}{\kern0pt}metis\ {\isacharparenleft}{\kern0pt}mono{\isacharunderscore}{\kern0pt}tags{\isacharcomma}{\kern0pt}\ lifting{\isacharparenright}{\kern0pt}\ of{\isacharunderscore}{\kern0pt}int{\isacharunderscore}{\kern0pt}of{\isacharunderscore}{\kern0pt}nat{\isacharunderscore}{\kern0pt}eq\ of{\isacharunderscore}{\kern0pt}nat{\isacharunderscore}{\kern0pt}numeral\ of{\isacharunderscore}{\kern0pt}nat{\isacharunderscore}{\kern0pt}power\ zmod{\isacharunderscore}{\kern0pt}int{\isacharparenright}{\kern0pt}\isanewline
\ \ \isacommand{hence}\isamarkupfalse%
\ {\isachardoublequoteopen}{\isasymforall}x{\isachardot}{\kern0pt}\ {\isacharparenleft}{\kern0pt}{\isacharparenleft}{\kern0pt}List{\isachardot}{\kern0pt}member\ {\isacharparenleft}{\kern0pt}map\ nat\ {\isacharparenleft}{\kern0pt}rev\ {\isacharbrackleft}{\kern0pt}{\isadigit{1}}{\isachardot}{\kern0pt}{\isachardot}{\kern0pt}n{\isacharbrackright}{\kern0pt}{\isacharparenright}{\kern0pt}{\isacharparenright}{\kern0pt}\ x{\isacharparenright}{\kern0pt}\ {\isasymlongrightarrow}\ {\isacharparenleft}{\kern0pt}fj\ x\ {\isacharequal}{\kern0pt}\ fjm\ x{\isacharparenright}{\kern0pt}{\isacharparenright}{\kern0pt}{\isachardoublequoteclose}\isanewline
\ \ \ \ \isacommand{using}\isamarkupfalse%
\ fj{\isacharunderscore}{\kern0pt}def\ fjm{\isacharunderscore}{\kern0pt}def\ \isacommand{by}\isamarkupfalse%
\ presburger\isanewline
\ \ \isacommand{hence}\isamarkupfalse%
\ {\isachardoublequoteopen}kron\ fj\ {\isacharparenleft}{\kern0pt}map\ nat\ {\isacharparenleft}{\kern0pt}rev\ {\isacharbrackleft}{\kern0pt}{\isadigit{1}}{\isachardot}{\kern0pt}{\isachardot}{\kern0pt}n{\isacharbrackright}{\kern0pt}{\isacharparenright}{\kern0pt}{\isacharparenright}{\kern0pt}\ {\isacharequal}{\kern0pt}\ kron\ fjm\ {\isacharparenleft}{\kern0pt}map\ nat\ {\isacharparenleft}{\kern0pt}rev\ {\isacharbrackleft}{\kern0pt}{\isadigit{1}}{\isachardot}{\kern0pt}{\isachardot}{\kern0pt}n{\isacharbrackright}{\kern0pt}{\isacharparenright}{\kern0pt}{\isacharparenright}{\kern0pt}{\isachardoublequoteclose}\isanewline
\ \ \ \ \isacommand{by}\isamarkupfalse%
\ simp\isanewline
\ \ \isacommand{thus}\isamarkupfalse%
\ {\isacharquery}{\kern0pt}thesis\ \isacommand{using}\isamarkupfalse%
\ fj{\isacharunderscore}{\kern0pt}def\ fjm{\isacharunderscore}{\kern0pt}def\ \isacommand{by}\isamarkupfalse%
\ auto\isanewline
\isacommand{qed}\isamarkupfalse%
%
\endisatagproof
{\isafoldproof}%
%
\isadelimproof
%
\endisadelimproof
%
\begin{isamarkuptext}%
We proof that the QFT circuit is correct:%
\end{isamarkuptext}\isamarkuptrue%
\isacommand{theorem}\isamarkupfalse%
\ QFT{\isacharunderscore}{\kern0pt}is{\isacharunderscore}{\kern0pt}correct{\isacharcolon}{\kern0pt}\isanewline
\ \ \isakeyword{shows}\ {\isachardoublequoteopen}{\isasymforall}j{\isachardot}{\kern0pt}\ j\ {\isacharless}{\kern0pt}\ {\isadigit{2}}{\isacharcircum}{\kern0pt}n\ {\isasymlongrightarrow}\ {\isacharparenleft}{\kern0pt}QFT\ n{\isacharparenright}{\kern0pt}\ {\isacharasterisk}{\kern0pt}\ {\isacharbar}{\kern0pt}state{\isacharunderscore}{\kern0pt}basis\ n\ j{\isasymrangle}\ {\isacharequal}{\kern0pt}\ reverse{\isacharunderscore}{\kern0pt}QFT{\isacharunderscore}{\kern0pt}product{\isacharunderscore}{\kern0pt}representation\ j\ n{\isachardoublequoteclose}\isanewline
%
\isadelimproof
%
\endisadelimproof
%
\isatagproof
\isacommand{proof}\isamarkupfalse%
\ {\isacharparenleft}{\kern0pt}induct\ n\ rule{\isacharcolon}{\kern0pt}\ QFT{\isachardot}{\kern0pt}induct{\isacharparenright}{\kern0pt}\isanewline
\ \ \isacommand{case}\isamarkupfalse%
\ {\isadigit{1}}\isanewline
\ \ \isacommand{thus}\isamarkupfalse%
\ {\isacharquery}{\kern0pt}case\isanewline
\ \ \isacommand{proof}\isamarkupfalse%
\ {\isacharparenleft}{\kern0pt}rule\ allI{\isacharparenright}{\kern0pt}\isanewline
\ \ \ \ \isacommand{fix}\isamarkupfalse%
\ j{\isacharcolon}{\kern0pt}{\isacharcolon}{\kern0pt}nat\isanewline
\ \ \ \ \isacommand{show}\isamarkupfalse%
\ {\isachardoublequoteopen}j\ {\isacharless}{\kern0pt}\ {\isadigit{2}}\ {\isacharcircum}{\kern0pt}\ {\isadigit{0}}\ {\isasymlongrightarrow}\ QFT\ {\isadigit{0}}\ {\isacharasterisk}{\kern0pt}\ {\isacharbar}{\kern0pt}state{\isacharunderscore}{\kern0pt}basis\ {\isadigit{0}}\ j{\isasymrangle}\ {\isacharequal}{\kern0pt}\ reverse{\isacharunderscore}{\kern0pt}QFT{\isacharunderscore}{\kern0pt}product{\isacharunderscore}{\kern0pt}representation\ j\ {\isadigit{0}}{\isachardoublequoteclose}\isanewline
\ \ \ \ \isacommand{proof}\isamarkupfalse%
\ \isanewline
\ \ \ \ \ \ \isacommand{assume}\isamarkupfalse%
\ {\isachardoublequoteopen}j\ {\isacharless}{\kern0pt}\ {\isadigit{2}}\ {\isacharcircum}{\kern0pt}\ {\isadigit{0}}{\isachardoublequoteclose}\isanewline
\ \ \ \ \ \ \isacommand{hence}\isamarkupfalse%
\ j{\isadigit{0}}{\isacharcolon}{\kern0pt}{\isachardoublequoteopen}j\ {\isacharequal}{\kern0pt}\ {\isadigit{0}}{\isachardoublequoteclose}\ \isacommand{by}\isamarkupfalse%
\ auto\isanewline
\ \ \ \ \ \ \isacommand{have}\isamarkupfalse%
\ {\isachardoublequoteopen}QFT\ {\isadigit{0}}\ {\isacharasterisk}{\kern0pt}\ {\isacharbar}{\kern0pt}state{\isacharunderscore}{\kern0pt}basis\ {\isadigit{0}}\ j{\isasymrangle}\ {\isacharequal}{\kern0pt}\ {\isacharparenleft}{\kern0pt}{\isadigit{1}}\isactrlsub m\ {\isadigit{1}}{\isacharparenright}{\kern0pt}\ {\isacharasterisk}{\kern0pt}\ {\isacharbar}{\kern0pt}state{\isacharunderscore}{\kern0pt}basis\ {\isadigit{0}}\ j{\isasymrangle}{\isachardoublequoteclose}\ \isacommand{using}\isamarkupfalse%
\ QFT{\isachardot}{\kern0pt}simps\ \isacommand{by}\isamarkupfalse%
\ simp\isanewline
\ \ \ \ \ \ \isacommand{also}\isamarkupfalse%
\ \isacommand{have}\isamarkupfalse%
\ {\isachardoublequoteopen}{\isasymdots}\ {\isacharequal}{\kern0pt}\ {\isacharbar}{\kern0pt}unit{\isacharunderscore}{\kern0pt}vec\ {\isadigit{1}}\ j{\isasymrangle}{\isachardoublequoteclose}\ \isacommand{using}\isamarkupfalse%
\ state{\isacharunderscore}{\kern0pt}basis{\isacharunderscore}{\kern0pt}def\isanewline
\ \ \ \ \ \ \ \ \isacommand{by}\isamarkupfalse%
\ {\isacharparenleft}{\kern0pt}metis\ left{\isacharunderscore}{\kern0pt}mult{\isacharunderscore}{\kern0pt}one{\isacharunderscore}{\kern0pt}mat\ power{\isacharunderscore}{\kern0pt}{\isadigit{0}}\ state{\isacharunderscore}{\kern0pt}basis{\isacharunderscore}{\kern0pt}carrier{\isacharunderscore}{\kern0pt}mat{\isacharparenright}{\kern0pt}\isanewline
\ \ \ \ \ \ \isacommand{also}\isamarkupfalse%
\ \isacommand{have}\isamarkupfalse%
\ {\isachardoublequoteopen}{\isasymdots}\ {\isacharequal}{\kern0pt}\ {\isacharparenleft}{\kern0pt}{\isadigit{1}}\isactrlsub m\ {\isadigit{1}}{\isacharparenright}{\kern0pt}{\isachardoublequoteclose}\ \isacommand{using}\isamarkupfalse%
\ unit{\isacharunderscore}{\kern0pt}vec{\isacharunderscore}{\kern0pt}def\ unit{\isacharunderscore}{\kern0pt}vec{\isacharunderscore}{\kern0pt}carrier\ ket{\isacharunderscore}{\kern0pt}vec{\isacharunderscore}{\kern0pt}def\ j{\isadigit{0}}\ \isacommand{by}\isamarkupfalse%
\ auto\isanewline
\ \ \ \ \ \ \isacommand{also}\isamarkupfalse%
\ \isacommand{have}\isamarkupfalse%
\ {\isachardoublequoteopen}{\isasymdots}\ {\isacharequal}{\kern0pt}\ reverse{\isacharunderscore}{\kern0pt}QFT{\isacharunderscore}{\kern0pt}product{\isacharunderscore}{\kern0pt}representation\ j\ {\isadigit{0}}{\isachardoublequoteclose}\isanewline
\ \ \ \ \ \ \ \ \isacommand{using}\isamarkupfalse%
\ reverse{\isacharunderscore}{\kern0pt}QFT{\isacharunderscore}{\kern0pt}product{\isacharunderscore}{\kern0pt}representation{\isacharunderscore}{\kern0pt}def\ \isacommand{by}\isamarkupfalse%
\ auto\isanewline
\ \ \ \ \ \ \isacommand{finally}\isamarkupfalse%
\ \isacommand{show}\isamarkupfalse%
\ {\isachardoublequoteopen}QFT\ {\isadigit{0}}\ {\isacharasterisk}{\kern0pt}\ {\isacharbar}{\kern0pt}state{\isacharunderscore}{\kern0pt}basis\ {\isadigit{0}}\ j{\isasymrangle}\ {\isacharequal}{\kern0pt}\ reverse{\isacharunderscore}{\kern0pt}QFT{\isacharunderscore}{\kern0pt}product{\isacharunderscore}{\kern0pt}representation\ j\ {\isadigit{0}}{\isachardoublequoteclose}\ \isacommand{by}\isamarkupfalse%
\ this\isanewline
\ \ \ \ \isacommand{qed}\isamarkupfalse%
\isanewline
\ \ \isacommand{qed}\isamarkupfalse%
\isanewline
\isacommand{next}\isamarkupfalse%
\isanewline
\ \ \isacommand{case}\isamarkupfalse%
\ {\isadigit{2}}\isanewline
\ \ \isacommand{thus}\isamarkupfalse%
\ {\isacharquery}{\kern0pt}case\isanewline
\ \ \isacommand{proof}\isamarkupfalse%
\ {\isacharparenleft}{\kern0pt}rule\ allI{\isacharparenright}{\kern0pt}\isanewline
\ \ \ \ \isacommand{fix}\isamarkupfalse%
\ j{\isacharcolon}{\kern0pt}{\isacharcolon}{\kern0pt}nat\isanewline
\ \ \ \ \isacommand{show}\isamarkupfalse%
\ {\isachardoublequoteopen}j\ {\isacharless}{\kern0pt}\ {\isadigit{2}}\ {\isacharcircum}{\kern0pt}\ Suc\ {\isadigit{0}}\ {\isasymlongrightarrow}\isanewline
\ \ \ \ \ \ \ \ \ QFT\ {\isacharparenleft}{\kern0pt}Suc\ {\isadigit{0}}{\isacharparenright}{\kern0pt}\ {\isacharasterisk}{\kern0pt}\isanewline
\ \ \ \ \ \ \ \ \ {\isacharbar}{\kern0pt}state{\isacharunderscore}{\kern0pt}basis\ {\isacharparenleft}{\kern0pt}Suc\ {\isadigit{0}}{\isacharparenright}{\kern0pt}\ j{\isasymrangle}\ {\isacharequal}{\kern0pt}\isanewline
\ \ \ \ \ \ \ \ \ reverse{\isacharunderscore}{\kern0pt}QFT{\isacharunderscore}{\kern0pt}product{\isacharunderscore}{\kern0pt}representation\ j\isanewline
\ \ \ \ \ \ \ \ \ \ {\isacharparenleft}{\kern0pt}Suc\ {\isadigit{0}}{\isacharparenright}{\kern0pt}{\isachardoublequoteclose}\isanewline
\ \ \ \ \isacommand{proof}\isamarkupfalse%
\ \isanewline
\ \ \ \ \ \ \isacommand{assume}\isamarkupfalse%
\ a{\isadigit{1}}{\isacharcolon}{\kern0pt}{\isachardoublequoteopen}j\ {\isacharless}{\kern0pt}\ {\isadigit{2}}{\isacharcircum}{\kern0pt}Suc\ {\isadigit{0}}{\isachardoublequoteclose}\isanewline
\ \ \ \ \ \ \isacommand{then}\isamarkupfalse%
\ \isacommand{show}\isamarkupfalse%
\ {\isachardoublequoteopen}QFT\ {\isacharparenleft}{\kern0pt}Suc\ {\isadigit{0}}{\isacharparenright}{\kern0pt}\ {\isacharasterisk}{\kern0pt}\ {\isacharbar}{\kern0pt}state{\isacharunderscore}{\kern0pt}basis\ {\isacharparenleft}{\kern0pt}Suc\ {\isadigit{0}}{\isacharparenright}{\kern0pt}\ j{\isasymrangle}\ {\isacharequal}{\kern0pt}\ \isanewline
\ \ \ \ \ \ \ \ \ \ \ \ \ \ \ \ \ reverse{\isacharunderscore}{\kern0pt}QFT{\isacharunderscore}{\kern0pt}product{\isacharunderscore}{\kern0pt}representation\ j\ {\isacharparenleft}{\kern0pt}Suc\ {\isadigit{0}}{\isacharparenright}{\kern0pt}{\isachardoublequoteclose}\isanewline
\ \ \ \ \ \ \isacommand{proof}\isamarkupfalse%
\ {\isacharminus}{\kern0pt}\isanewline
\ \ \ \ \ \ \ \ \isacommand{have}\isamarkupfalse%
\ {\isachardoublequoteopen}QFT\ {\isacharparenleft}{\kern0pt}Suc\ {\isadigit{0}}{\isacharparenright}{\kern0pt}\ {\isacharasterisk}{\kern0pt}\ {\isacharbar}{\kern0pt}state{\isacharunderscore}{\kern0pt}basis\ {\isacharparenleft}{\kern0pt}Suc\ {\isadigit{0}}{\isacharparenright}{\kern0pt}\ j{\isasymrangle}\ {\isacharequal}{\kern0pt}\ H\ {\isacharasterisk}{\kern0pt}\ {\isacharbar}{\kern0pt}unit{\isacharunderscore}{\kern0pt}vec\ {\isacharparenleft}{\kern0pt}{\isadigit{2}}{\isacharcircum}{\kern0pt}{\isacharparenleft}{\kern0pt}Suc\ {\isadigit{0}}{\isacharparenright}{\kern0pt}{\isacharparenright}{\kern0pt}\ j{\isasymrangle}{\isachardoublequoteclose}\isanewline
\ \ \ \ \ \ \ \ \ \ \isacommand{using}\isamarkupfalse%
\ QFT{\isachardot}{\kern0pt}simps{\isacharparenleft}{\kern0pt}{\isadigit{2}}{\isacharparenright}{\kern0pt}\ state{\isacharunderscore}{\kern0pt}basis{\isacharunderscore}{\kern0pt}def\ \isacommand{by}\isamarkupfalse%
\ auto\isanewline
\ \ \ \ \ \ \ \ \isacommand{also}\isamarkupfalse%
\ \isacommand{have}\isamarkupfalse%
\ {\isachardoublequoteopen}{\isasymdots}\ {\isacharequal}{\kern0pt}\ reverse{\isacharunderscore}{\kern0pt}QFT{\isacharunderscore}{\kern0pt}product{\isacharunderscore}{\kern0pt}representation\ j\ {\isacharparenleft}{\kern0pt}Suc\ {\isadigit{0}}{\isacharparenright}{\kern0pt}{\isachardoublequoteclose}\isanewline
\ \ \ \ \ \ \ \ \isacommand{proof}\isamarkupfalse%
\ {\isacharparenleft}{\kern0pt}rule\ disjE{\isacharparenright}{\kern0pt}\isanewline
\ \ \ \ \ \ \ \ \ \ \isacommand{show}\isamarkupfalse%
\ {\isachardoublequoteopen}j{\isacharequal}{\kern0pt}{\isadigit{0}}\ {\isasymor}\ j{\isacharequal}{\kern0pt}{\isadigit{1}}{\isachardoublequoteclose}\ \isacommand{using}\isamarkupfalse%
\ a{\isadigit{1}}\ \isacommand{by}\isamarkupfalse%
\ auto\isanewline
\ \ \ \ \ \ \ \ \isacommand{next}\isamarkupfalse%
\isanewline
\ \ \ \ \ \ \ \ \ \ \isacommand{assume}\isamarkupfalse%
\ j{\isadigit{0}}{\isacharcolon}{\kern0pt}{\isachardoublequoteopen}j{\isacharequal}{\kern0pt}{\isadigit{0}}{\isachardoublequoteclose}\isanewline
\ \ \ \ \ \ \ \ \ \ \isacommand{hence}\isamarkupfalse%
\ {\isachardoublequoteopen}H\ {\isacharasterisk}{\kern0pt}\ {\isacharbar}{\kern0pt}unit{\isacharunderscore}{\kern0pt}vec\ {\isacharparenleft}{\kern0pt}{\isadigit{2}}{\isacharcircum}{\kern0pt}{\isacharparenleft}{\kern0pt}Suc\ {\isadigit{0}}{\isacharparenright}{\kern0pt}{\isacharparenright}{\kern0pt}\ j{\isasymrangle}\ {\isacharequal}{\kern0pt}\ H\ {\isacharasterisk}{\kern0pt}\ {\isacharbar}{\kern0pt}unit{\isacharunderscore}{\kern0pt}vec\ {\isacharparenleft}{\kern0pt}{\isadigit{2}}{\isacharcircum}{\kern0pt}{\isacharparenleft}{\kern0pt}Suc\ {\isadigit{0}}{\isacharparenright}{\kern0pt}{\isacharparenright}{\kern0pt}\ {\isadigit{0}}{\isasymrangle}{\isachardoublequoteclose}\ \isacommand{by}\isamarkupfalse%
\ simp\isanewline
\ \ \ \ \ \ \ \ \ \ \isacommand{also}\isamarkupfalse%
\ \isacommand{have}\isamarkupfalse%
\ {\isachardoublequoteopen}{\isasymdots}\ {\isacharequal}{\kern0pt}\ H\ {\isacharasterisk}{\kern0pt}\ {\isacharbar}{\kern0pt}zero{\isasymrangle}{\isachardoublequoteclose}\ \isacommand{by}\isamarkupfalse%
\ auto\isanewline
\ \ \ \ \ \ \ \ \ \ \isacommand{also}\isamarkupfalse%
\ \isacommand{have}\isamarkupfalse%
\ {\isachardoublequoteopen}{\isasymdots}\ {\isacharequal}{\kern0pt}\ mat{\isacharunderscore}{\kern0pt}of{\isacharunderscore}{\kern0pt}cols{\isacharunderscore}{\kern0pt}list\ {\isadigit{2}}\ {\isacharbrackleft}{\kern0pt}{\isacharbrackleft}{\kern0pt}{\isadigit{1}}{\isacharslash}{\kern0pt}sqrt{\isacharparenleft}{\kern0pt}{\isadigit{2}}{\isacharparenright}{\kern0pt}{\isacharcomma}{\kern0pt}{\isadigit{1}}{\isacharslash}{\kern0pt}sqrt{\isacharparenleft}{\kern0pt}{\isadigit{2}}{\isacharparenright}{\kern0pt}{\isacharbrackright}{\kern0pt}{\isacharbrackright}{\kern0pt}{\isachardoublequoteclose}\isanewline
\ \ \ \ \ \ \ \ \ \ \ \ \isacommand{using}\isamarkupfalse%
\ H{\isacharunderscore}{\kern0pt}on{\isacharunderscore}{\kern0pt}ket{\isacharunderscore}{\kern0pt}zero\ \isacommand{by}\isamarkupfalse%
\ simp\isanewline
\ \ \ \ \ \ \ \ \ \ \isacommand{also}\isamarkupfalse%
\ \isacommand{have}\isamarkupfalse%
\ {\isachardoublequoteopen}{\isasymdots}\ {\isacharequal}{\kern0pt}\ {\isadigit{1}}{\isacharslash}{\kern0pt}sqrt{\isacharparenleft}{\kern0pt}{\isadigit{2}}{\isacharparenright}{\kern0pt}\ {\isasymcdot}\isactrlsub m\ {\isacharparenleft}{\kern0pt}mat{\isacharunderscore}{\kern0pt}of{\isacharunderscore}{\kern0pt}cols{\isacharunderscore}{\kern0pt}list\ {\isadigit{2}}\ {\isacharbrackleft}{\kern0pt}{\isacharbrackleft}{\kern0pt}{\isadigit{1}}{\isacharcomma}{\kern0pt}{\isadigit{1}}{\isacharbrackright}{\kern0pt}{\isacharbrackright}{\kern0pt}{\isacharparenright}{\kern0pt}{\isachardoublequoteclose}\isanewline
\ \ \ \ \ \ \ \ \ \ \isacommand{proof}\isamarkupfalse%
\ \isanewline
\ \ \ \ \ \ \ \ \ \ \ \ \isacommand{fix}\isamarkupfalse%
\ i\ j{\isacharcolon}{\kern0pt}{\isacharcolon}{\kern0pt}nat\isanewline
\ \ \ \ \ \ \ \ \ \ \ \ \isacommand{define}\isamarkupfalse%
\ {\isasympsi}{\isadigit{1}}\ {\isasympsi}{\isadigit{2}}\ \isakeyword{where}\ {\isachardoublequoteopen}{\isasympsi}{\isadigit{1}}\ {\isacharequal}{\kern0pt}\ mat{\isacharunderscore}{\kern0pt}of{\isacharunderscore}{\kern0pt}cols{\isacharunderscore}{\kern0pt}list\ {\isadigit{2}}\ {\isacharbrackleft}{\kern0pt}{\isacharbrackleft}{\kern0pt}{\isadigit{1}}{\isacharslash}{\kern0pt}sqrt{\isacharparenleft}{\kern0pt}{\isadigit{2}}{\isacharparenright}{\kern0pt}{\isacharcomma}{\kern0pt}{\isadigit{1}}{\isacharslash}{\kern0pt}sqrt{\isacharparenleft}{\kern0pt}{\isadigit{2}}{\isacharparenright}{\kern0pt}{\isacharbrackright}{\kern0pt}{\isacharbrackright}{\kern0pt}{\isachardoublequoteclose}\ \isakeyword{and}\ \isanewline
\ \ \ \ \ \ \ \ \ \ \ \ \ \ \ \ \ \ \ \ \ \ \ \ \ \ \ \ \ \ \ {\isachardoublequoteopen}{\isasympsi}{\isadigit{2}}\ {\isacharequal}{\kern0pt}\ {\isadigit{1}}{\isacharslash}{\kern0pt}sqrt{\isacharparenleft}{\kern0pt}{\isadigit{2}}{\isacharparenright}{\kern0pt}\ {\isasymcdot}\isactrlsub m\ {\isacharparenleft}{\kern0pt}mat{\isacharunderscore}{\kern0pt}of{\isacharunderscore}{\kern0pt}cols{\isacharunderscore}{\kern0pt}list\ {\isadigit{2}}\ {\isacharbrackleft}{\kern0pt}{\isacharbrackleft}{\kern0pt}{\isadigit{1}}{\isacharcomma}{\kern0pt}{\isadigit{1}}{\isacharbrackright}{\kern0pt}{\isacharbrackright}{\kern0pt}{\isacharparenright}{\kern0pt}{\isachardoublequoteclose}\isanewline
\ \ \ \ \ \ \ \ \ \ \ \ \isacommand{assume}\isamarkupfalse%
\ {\isachardoublequoteopen}i\ {\isacharless}{\kern0pt}\ dim{\isacharunderscore}{\kern0pt}row\ {\isasympsi}{\isadigit{2}}{\isachardoublequoteclose}\ \isakeyword{and}\ {\isachardoublequoteopen}j\ {\isacharless}{\kern0pt}\ dim{\isacharunderscore}{\kern0pt}col\ {\isasympsi}{\isadigit{2}}{\isachardoublequoteclose}\isanewline
\ \ \ \ \ \ \ \ \ \ \ \ \isacommand{hence}\isamarkupfalse%
\ a{\isadigit{2}}{\isacharcolon}{\kern0pt}{\isachardoublequoteopen}i\ {\isasymin}\ {\isacharbraceleft}{\kern0pt}{\isadigit{0}}{\isacharcomma}{\kern0pt}{\isadigit{1}}{\isacharbraceright}{\kern0pt}\ {\isasymand}\ j{\isacharequal}{\kern0pt}{\isadigit{0}}{\isachardoublequoteclose}\isanewline
\ \ \ \ \ \ \ \ \ \ \ \ \ \ \isacommand{by}\isamarkupfalse%
\ {\isacharparenleft}{\kern0pt}simp\ add{\isacharcolon}{\kern0pt}\ Tensor{\isachardot}{\kern0pt}mat{\isacharunderscore}{\kern0pt}of{\isacharunderscore}{\kern0pt}cols{\isacharunderscore}{\kern0pt}list{\isacharunderscore}{\kern0pt}def\ {\isasympsi}{\isadigit{2}}{\isacharunderscore}{\kern0pt}def\ less{\isacharunderscore}{\kern0pt}Suc{\isacharunderscore}{\kern0pt}eq{\isacharunderscore}{\kern0pt}{\isadigit{0}}{\isacharunderscore}{\kern0pt}disj\ numerals{\isacharparenleft}{\kern0pt}{\isadigit{2}}{\isacharparenright}{\kern0pt}{\isacharparenright}{\kern0pt}\isanewline
\ \ \ \ \ \ \ \ \ \ \ \ \isacommand{have}\isamarkupfalse%
\ {\isachardoublequoteopen}{\isasympsi}{\isadigit{1}}\ {\isachardollar}{\kern0pt}{\isachardollar}{\kern0pt}\ {\isacharparenleft}{\kern0pt}{\isadigit{0}}{\isacharcomma}{\kern0pt}{\isadigit{0}}{\isacharparenright}{\kern0pt}\ {\isacharequal}{\kern0pt}\ {\isadigit{1}}{\isacharslash}{\kern0pt}sqrt\ {\isadigit{2}}{\isachardoublequoteclose}\ \isacommand{using}\isamarkupfalse%
\ mat{\isacharunderscore}{\kern0pt}of{\isacharunderscore}{\kern0pt}cols{\isacharunderscore}{\kern0pt}list{\isacharunderscore}{\kern0pt}def\ {\isasympsi}{\isadigit{1}}{\isacharunderscore}{\kern0pt}def\ \isacommand{by}\isamarkupfalse%
\ simp\isanewline
\ \ \ \ \ \ \ \ \ \ \ \ \isacommand{moreover}\isamarkupfalse%
\ \isacommand{have}\isamarkupfalse%
\ {\isachardoublequoteopen}{\isasympsi}{\isadigit{1}}\ {\isachardollar}{\kern0pt}{\isachardollar}{\kern0pt}\ {\isacharparenleft}{\kern0pt}{\isadigit{1}}{\isacharcomma}{\kern0pt}{\isadigit{0}}{\isacharparenright}{\kern0pt}\ {\isacharequal}{\kern0pt}\ {\isadigit{1}}{\isacharslash}{\kern0pt}sqrt\ {\isadigit{2}}{\isachardoublequoteclose}\ \isacommand{using}\isamarkupfalse%
\ mat{\isacharunderscore}{\kern0pt}of{\isacharunderscore}{\kern0pt}cols{\isacharunderscore}{\kern0pt}list{\isacharunderscore}{\kern0pt}def\ {\isasympsi}{\isadigit{1}}{\isacharunderscore}{\kern0pt}def\ \isacommand{by}\isamarkupfalse%
\ simp\isanewline
\ \ \ \ \ \ \ \ \ \ \ \ \isacommand{moreover}\isamarkupfalse%
\ \isacommand{have}\isamarkupfalse%
\ {\isachardoublequoteopen}{\isasympsi}{\isadigit{2}}\ {\isachardollar}{\kern0pt}{\isachardollar}{\kern0pt}\ {\isacharparenleft}{\kern0pt}{\isadigit{0}}{\isacharcomma}{\kern0pt}{\isadigit{0}}{\isacharparenright}{\kern0pt}\ {\isacharequal}{\kern0pt}\ {\isadigit{1}}{\isacharslash}{\kern0pt}sqrt\ {\isadigit{2}}{\isachardoublequoteclose}\ \isanewline
\ \ \ \ \ \ \ \ \ \ \ \ \ \ \isacommand{using}\isamarkupfalse%
\ smult{\isacharunderscore}{\kern0pt}mat{\isacharunderscore}{\kern0pt}def\ mat{\isacharunderscore}{\kern0pt}of{\isacharunderscore}{\kern0pt}cols{\isacharunderscore}{\kern0pt}list{\isacharunderscore}{\kern0pt}def\ {\isasympsi}{\isadigit{2}}{\isacharunderscore}{\kern0pt}def\ \isacommand{by}\isamarkupfalse%
\ simp\isanewline
\ \ \ \ \ \ \ \ \ \ \ \ \isacommand{moreover}\isamarkupfalse%
\ \isacommand{have}\isamarkupfalse%
\ {\isachardoublequoteopen}{\isasympsi}{\isadigit{2}}\ {\isachardollar}{\kern0pt}{\isachardollar}{\kern0pt}\ {\isacharparenleft}{\kern0pt}{\isadigit{1}}{\isacharcomma}{\kern0pt}{\isadigit{0}}{\isacharparenright}{\kern0pt}\ {\isacharequal}{\kern0pt}\ {\isadigit{1}}{\isacharslash}{\kern0pt}sqrt\ {\isadigit{2}}{\isachardoublequoteclose}\ \isanewline
\ \ \ \ \ \ \ \ \ \ \ \ \ \ \isacommand{using}\isamarkupfalse%
\ smult{\isacharunderscore}{\kern0pt}mat{\isacharunderscore}{\kern0pt}def\ mat{\isacharunderscore}{\kern0pt}of{\isacharunderscore}{\kern0pt}cols{\isacharunderscore}{\kern0pt}list{\isacharunderscore}{\kern0pt}def\ {\isasympsi}{\isadigit{2}}{\isacharunderscore}{\kern0pt}def\ \isacommand{by}\isamarkupfalse%
\ simp\isanewline
\ \ \ \ \ \ \ \ \ \ \ \ \isacommand{ultimately}\isamarkupfalse%
\ \isacommand{show}\isamarkupfalse%
\ {\isachardoublequoteopen}{\isasympsi}{\isadigit{1}}\ {\isachardollar}{\kern0pt}{\isachardollar}{\kern0pt}\ {\isacharparenleft}{\kern0pt}i{\isacharcomma}{\kern0pt}j{\isacharparenright}{\kern0pt}\ {\isacharequal}{\kern0pt}\ {\isasympsi}{\isadigit{2}}\ {\isachardollar}{\kern0pt}{\isachardollar}{\kern0pt}\ {\isacharparenleft}{\kern0pt}i{\isacharcomma}{\kern0pt}j{\isacharparenright}{\kern0pt}{\isachardoublequoteclose}\ \isacommand{using}\isamarkupfalse%
\ a{\isadigit{2}}\ \isacommand{by}\isamarkupfalse%
\ auto\isanewline
\ \ \ \ \ \ \ \ \ \ \isacommand{next}\isamarkupfalse%
\isanewline
\ \ \ \ \ \ \ \ \ \ \ \ \isacommand{define}\isamarkupfalse%
\ {\isasympsi}{\isadigit{1}}\ {\isasympsi}{\isadigit{2}}\ \isakeyword{where}\ {\isachardoublequoteopen}{\isasympsi}{\isadigit{1}}\ {\isacharequal}{\kern0pt}\ mat{\isacharunderscore}{\kern0pt}of{\isacharunderscore}{\kern0pt}cols{\isacharunderscore}{\kern0pt}list\ {\isadigit{2}}\ {\isacharbrackleft}{\kern0pt}{\isacharbrackleft}{\kern0pt}{\isadigit{1}}{\isacharslash}{\kern0pt}sqrt{\isacharparenleft}{\kern0pt}{\isadigit{2}}{\isacharparenright}{\kern0pt}{\isacharcomma}{\kern0pt}{\isadigit{1}}{\isacharslash}{\kern0pt}sqrt{\isacharparenleft}{\kern0pt}{\isadigit{2}}{\isacharparenright}{\kern0pt}{\isacharbrackright}{\kern0pt}{\isacharbrackright}{\kern0pt}{\isachardoublequoteclose}\ \isakeyword{and}\ \isanewline
\ \ \ \ \ \ \ \ \ \ \ \ \ \ \ \ \ \ \ \ \ \ \ \ \ \ \ \ \ \ \ {\isachardoublequoteopen}{\isasympsi}{\isadigit{2}}\ {\isacharequal}{\kern0pt}\ {\isadigit{1}}{\isacharslash}{\kern0pt}sqrt{\isacharparenleft}{\kern0pt}{\isadigit{2}}{\isacharparenright}{\kern0pt}\ {\isasymcdot}\isactrlsub m\ {\isacharparenleft}{\kern0pt}mat{\isacharunderscore}{\kern0pt}of{\isacharunderscore}{\kern0pt}cols{\isacharunderscore}{\kern0pt}list\ {\isadigit{2}}\ {\isacharbrackleft}{\kern0pt}{\isacharbrackleft}{\kern0pt}{\isadigit{1}}{\isacharcomma}{\kern0pt}{\isadigit{1}}{\isacharbrackright}{\kern0pt}{\isacharbrackright}{\kern0pt}{\isacharparenright}{\kern0pt}{\isachardoublequoteclose}\isanewline
\ \ \ \ \ \ \ \ \ \ \ \ \isacommand{show}\isamarkupfalse%
\ {\isachardoublequoteopen}dim{\isacharunderscore}{\kern0pt}row\ {\isasympsi}{\isadigit{1}}\ {\isacharequal}{\kern0pt}\ dim{\isacharunderscore}{\kern0pt}row\ {\isasympsi}{\isadigit{2}}{\isachardoublequoteclose}\ \isacommand{using}\isamarkupfalse%
\ {\isasympsi}{\isadigit{1}}{\isacharunderscore}{\kern0pt}def\ {\isasympsi}{\isadigit{2}}{\isacharunderscore}{\kern0pt}def\ Tensor{\isachardot}{\kern0pt}mat{\isacharunderscore}{\kern0pt}of{\isacharunderscore}{\kern0pt}cols{\isacharunderscore}{\kern0pt}list{\isacharunderscore}{\kern0pt}def\ \isacommand{by}\isamarkupfalse%
\ simp\isanewline
\ \ \ \ \ \ \ \ \ \ \isacommand{next}\isamarkupfalse%
\isanewline
\ \ \ \ \ \ \ \ \ \ \ \ \isacommand{define}\isamarkupfalse%
\ {\isasympsi}{\isadigit{1}}\ {\isasympsi}{\isadigit{2}}\ \isakeyword{where}\ {\isachardoublequoteopen}{\isasympsi}{\isadigit{1}}\ {\isacharequal}{\kern0pt}\ mat{\isacharunderscore}{\kern0pt}of{\isacharunderscore}{\kern0pt}cols{\isacharunderscore}{\kern0pt}list\ {\isadigit{2}}\ {\isacharbrackleft}{\kern0pt}{\isacharbrackleft}{\kern0pt}{\isadigit{1}}{\isacharslash}{\kern0pt}sqrt{\isacharparenleft}{\kern0pt}{\isadigit{2}}{\isacharparenright}{\kern0pt}{\isacharcomma}{\kern0pt}{\isadigit{1}}{\isacharslash}{\kern0pt}sqrt{\isacharparenleft}{\kern0pt}{\isadigit{2}}{\isacharparenright}{\kern0pt}{\isacharbrackright}{\kern0pt}{\isacharbrackright}{\kern0pt}{\isachardoublequoteclose}\ \isakeyword{and}\ \isanewline
\ \ \ \ \ \ \ \ \ \ \ \ \ \ \ \ \ \ \ \ \ \ \ \ \ \ \ \ \ \ \ {\isachardoublequoteopen}{\isasympsi}{\isadigit{2}}\ {\isacharequal}{\kern0pt}\ {\isadigit{1}}{\isacharslash}{\kern0pt}sqrt{\isacharparenleft}{\kern0pt}{\isadigit{2}}{\isacharparenright}{\kern0pt}\ {\isasymcdot}\isactrlsub m\ {\isacharparenleft}{\kern0pt}mat{\isacharunderscore}{\kern0pt}of{\isacharunderscore}{\kern0pt}cols{\isacharunderscore}{\kern0pt}list\ {\isadigit{2}}\ {\isacharbrackleft}{\kern0pt}{\isacharbrackleft}{\kern0pt}{\isadigit{1}}{\isacharcomma}{\kern0pt}{\isadigit{1}}{\isacharbrackright}{\kern0pt}{\isacharbrackright}{\kern0pt}{\isacharparenright}{\kern0pt}{\isachardoublequoteclose}\isanewline
\ \ \ \ \ \ \ \ \ \ \ \ \isacommand{show}\isamarkupfalse%
\ {\isachardoublequoteopen}dim{\isacharunderscore}{\kern0pt}col\ {\isasympsi}{\isadigit{1}}\ {\isacharequal}{\kern0pt}\ dim{\isacharunderscore}{\kern0pt}col\ {\isasympsi}{\isadigit{2}}{\isachardoublequoteclose}\ \isacommand{using}\isamarkupfalse%
\ {\isasympsi}{\isadigit{1}}{\isacharunderscore}{\kern0pt}def\ {\isasympsi}{\isadigit{2}}{\isacharunderscore}{\kern0pt}def\ Tensor{\isachardot}{\kern0pt}mat{\isacharunderscore}{\kern0pt}of{\isacharunderscore}{\kern0pt}cols{\isacharunderscore}{\kern0pt}list{\isacharunderscore}{\kern0pt}def\ \isacommand{by}\isamarkupfalse%
\ simp\isanewline
\ \ \ \ \ \ \ \ \ \ \isacommand{qed}\isamarkupfalse%
\isanewline
\ \ \ \ \ \ \ \ \ \ \isacommand{also}\isamarkupfalse%
\ \isacommand{have}\isamarkupfalse%
\ {\isachardoublequoteopen}{\isasymdots}\ {\isacharequal}{\kern0pt}\ {\isadigit{1}}{\isacharslash}{\kern0pt}sqrt\ {\isadigit{2}}\ {\isasymcdot}\isactrlsub m\ {\isacharparenleft}{\kern0pt}\ {\isacharbar}{\kern0pt}zero{\isasymrangle}\ {\isacharplus}{\kern0pt}\ {\isacharbar}{\kern0pt}one{\isasymrangle}{\isacharparenright}{\kern0pt}{\isachardoublequoteclose}\isanewline
\ \ \ \ \ \ \ \ \ \ \isacommand{proof}\isamarkupfalse%
\ {\isacharminus}{\kern0pt}\isanewline
\ \ \ \ \ \ \ \ \ \ \ \ \isacommand{have}\isamarkupfalse%
\ {\isachardoublequoteopen}mat{\isacharunderscore}{\kern0pt}of{\isacharunderscore}{\kern0pt}cols{\isacharunderscore}{\kern0pt}list\ {\isadigit{2}}\ {\isacharbrackleft}{\kern0pt}{\isacharbrackleft}{\kern0pt}{\isadigit{1}}{\isacharcomma}{\kern0pt}{\isadigit{1}}{\isacharbrackright}{\kern0pt}{\isacharbrackright}{\kern0pt}\ {\isacharequal}{\kern0pt}\ {\isacharbar}{\kern0pt}zero{\isasymrangle}\ {\isacharplus}{\kern0pt}\ {\isacharbar}{\kern0pt}one{\isasymrangle}{\isachardoublequoteclose}\isanewline
\ \ \ \ \ \ \ \ \ \ \ \ \isacommand{proof}\isamarkupfalse%
\ \isanewline
\ \ \ \ \ \ \ \ \ \ \ \ \ \ \isacommand{fix}\isamarkupfalse%
\ i\ j{\isacharcolon}{\kern0pt}{\isacharcolon}{\kern0pt}nat\ \isanewline
\ \ \ \ \ \ \ \ \ \ \ \ \ \ \isacommand{define}\isamarkupfalse%
\ s{\isadigit{1}}\ s{\isadigit{2}}\ \isakeyword{where}\ {\isachardoublequoteopen}s{\isadigit{1}}\ {\isacharequal}{\kern0pt}\ mat{\isacharunderscore}{\kern0pt}of{\isacharunderscore}{\kern0pt}cols{\isacharunderscore}{\kern0pt}list\ {\isadigit{2}}\ {\isacharbrackleft}{\kern0pt}{\isacharbrackleft}{\kern0pt}{\isadigit{1}}{\isacharcomma}{\kern0pt}{\isadigit{1}}{\isacharbrackright}{\kern0pt}{\isacharbrackright}{\kern0pt}{\isachardoublequoteclose}\ \isakeyword{and}\ {\isachardoublequoteopen}s{\isadigit{2}}\ {\isacharequal}{\kern0pt}\ {\isacharbar}{\kern0pt}zero{\isasymrangle}\ {\isacharplus}{\kern0pt}\ {\isacharbar}{\kern0pt}one{\isasymrangle}{\isachardoublequoteclose}\isanewline
\ \ \ \ \ \ \ \ \ \ \ \ \ \ \isacommand{assume}\isamarkupfalse%
\ {\isachardoublequoteopen}i\ {\isacharless}{\kern0pt}\ dim{\isacharunderscore}{\kern0pt}row\ s{\isadigit{2}}{\isachardoublequoteclose}\ \isakeyword{and}\ {\isachardoublequoteopen}j\ {\isacharless}{\kern0pt}\ dim{\isacharunderscore}{\kern0pt}col\ s{\isadigit{2}}{\isachardoublequoteclose}\isanewline
\ \ \ \ \ \ \ \ \ \ \ \ \ \ \isacommand{hence}\isamarkupfalse%
\ {\isachardoublequoteopen}i\ {\isasymin}\ {\isacharbraceleft}{\kern0pt}{\isadigit{0}}{\isacharcomma}{\kern0pt}{\isadigit{1}}{\isacharbraceright}{\kern0pt}\ {\isasymand}\ j\ {\isacharequal}{\kern0pt}\ {\isadigit{0}}{\isachardoublequoteclose}\ \isacommand{using}\isamarkupfalse%
\ index{\isacharunderscore}{\kern0pt}add{\isacharunderscore}{\kern0pt}mat\ \isanewline
\ \ \ \ \ \ \ \ \ \ \ \ \ \ \ \ \isacommand{by}\isamarkupfalse%
\ {\isacharparenleft}{\kern0pt}simp\ add{\isacharcolon}{\kern0pt}\ ket{\isacharunderscore}{\kern0pt}vec{\isacharunderscore}{\kern0pt}def\ less{\isacharunderscore}{\kern0pt}Suc{\isacharunderscore}{\kern0pt}eq\ numerals{\isacharparenleft}{\kern0pt}{\isadigit{2}}{\isacharparenright}{\kern0pt}\ s{\isadigit{2}}{\isacharunderscore}{\kern0pt}def{\isacharparenright}{\kern0pt}\isanewline
\ \ \ \ \ \ \ \ \ \ \ \ \ \ \isacommand{thus}\isamarkupfalse%
\ {\isachardoublequoteopen}s{\isadigit{1}}\ {\isachardollar}{\kern0pt}{\isachardollar}{\kern0pt}\ {\isacharparenleft}{\kern0pt}i{\isacharcomma}{\kern0pt}j{\isacharparenright}{\kern0pt}\ {\isacharequal}{\kern0pt}\ s{\isadigit{2}}\ {\isachardollar}{\kern0pt}{\isachardollar}{\kern0pt}\ {\isacharparenleft}{\kern0pt}i{\isacharcomma}{\kern0pt}j{\isacharparenright}{\kern0pt}{\isachardoublequoteclose}\ \isacommand{using}\isamarkupfalse%
\ s{\isadigit{1}}{\isacharunderscore}{\kern0pt}def\ s{\isadigit{2}}{\isacharunderscore}{\kern0pt}def\ mat{\isacharunderscore}{\kern0pt}of{\isacharunderscore}{\kern0pt}cols{\isacharunderscore}{\kern0pt}list{\isacharunderscore}{\kern0pt}def\ \isanewline
\ \ \ \ \ \ \ \ \ \ \ \ \ \ \ \ \ \ \ \ {\isacartoucheopen}i\ {\isacharless}{\kern0pt}\ dim{\isacharunderscore}{\kern0pt}row\ s{\isadigit{2}}{\isacartoucheclose}\ ket{\isacharunderscore}{\kern0pt}one{\isacharunderscore}{\kern0pt}to{\isacharunderscore}{\kern0pt}mat{\isacharunderscore}{\kern0pt}of{\isacharunderscore}{\kern0pt}cols{\isacharunderscore}{\kern0pt}list\ \isacommand{by}\isamarkupfalse%
\ force\isanewline
\ \ \ \ \ \ \ \ \ \ \ \ \isacommand{next}\isamarkupfalse%
\isanewline
\ \ \ \ \ \ \ \ \ \ \ \ \ \ \isacommand{define}\isamarkupfalse%
\ s{\isadigit{1}}\ s{\isadigit{2}}\ \isakeyword{where}\ {\isachardoublequoteopen}s{\isadigit{1}}\ {\isacharequal}{\kern0pt}\ mat{\isacharunderscore}{\kern0pt}of{\isacharunderscore}{\kern0pt}cols{\isacharunderscore}{\kern0pt}list\ {\isadigit{2}}\ {\isacharbrackleft}{\kern0pt}{\isacharbrackleft}{\kern0pt}{\isadigit{1}}{\isacharcomma}{\kern0pt}{\isadigit{1}}{\isacharbrackright}{\kern0pt}{\isacharbrackright}{\kern0pt}{\isachardoublequoteclose}\ \isakeyword{and}\ {\isachardoublequoteopen}s{\isadigit{2}}\ {\isacharequal}{\kern0pt}\ {\isacharbar}{\kern0pt}zero{\isasymrangle}\ {\isacharplus}{\kern0pt}\ {\isacharbar}{\kern0pt}one{\isasymrangle}{\isachardoublequoteclose}\isanewline
\ \ \ \ \ \ \ \ \ \ \ \ \ \ \isacommand{thus}\isamarkupfalse%
\ {\isachardoublequoteopen}dim{\isacharunderscore}{\kern0pt}row\ s{\isadigit{1}}\ {\isacharequal}{\kern0pt}\ dim{\isacharunderscore}{\kern0pt}row\ s{\isadigit{2}}{\isachardoublequoteclose}\ \isacommand{using}\isamarkupfalse%
\ mat{\isacharunderscore}{\kern0pt}of{\isacharunderscore}{\kern0pt}cols{\isacharunderscore}{\kern0pt}list{\isacharunderscore}{\kern0pt}def\ \isacommand{by}\isamarkupfalse%
\ {\isacharparenleft}{\kern0pt}simp\ add{\isacharcolon}{\kern0pt}\ ket{\isacharunderscore}{\kern0pt}vec{\isacharunderscore}{\kern0pt}def{\isacharparenright}{\kern0pt}\isanewline
\ \ \ \ \ \ \ \ \ \ \ \ \isacommand{next}\isamarkupfalse%
\isanewline
\ \ \ \ \ \ \ \ \ \ \ \ \ \ \isacommand{define}\isamarkupfalse%
\ s{\isadigit{1}}\ s{\isadigit{2}}\ \isakeyword{where}\ {\isachardoublequoteopen}s{\isadigit{1}}\ {\isacharequal}{\kern0pt}\ mat{\isacharunderscore}{\kern0pt}of{\isacharunderscore}{\kern0pt}cols{\isacharunderscore}{\kern0pt}list\ {\isadigit{2}}\ {\isacharbrackleft}{\kern0pt}{\isacharbrackleft}{\kern0pt}{\isadigit{1}}{\isacharcomma}{\kern0pt}{\isadigit{1}}{\isacharbrackright}{\kern0pt}{\isacharbrackright}{\kern0pt}{\isachardoublequoteclose}\ \isakeyword{and}\ {\isachardoublequoteopen}s{\isadigit{2}}\ {\isacharequal}{\kern0pt}\ {\isacharbar}{\kern0pt}zero{\isasymrangle}\ {\isacharplus}{\kern0pt}\ {\isacharbar}{\kern0pt}one{\isasymrangle}{\isachardoublequoteclose}\isanewline
\ \ \ \ \ \ \ \ \ \ \ \ \ \ \isacommand{thus}\isamarkupfalse%
\ {\isachardoublequoteopen}dim{\isacharunderscore}{\kern0pt}col\ s{\isadigit{1}}\ {\isacharequal}{\kern0pt}\ dim{\isacharunderscore}{\kern0pt}col\ s{\isadigit{2}}{\isachardoublequoteclose}\ \isacommand{using}\isamarkupfalse%
\ mat{\isacharunderscore}{\kern0pt}of{\isacharunderscore}{\kern0pt}cols{\isacharunderscore}{\kern0pt}list{\isacharunderscore}{\kern0pt}def\ \isacommand{by}\isamarkupfalse%
\ {\isacharparenleft}{\kern0pt}simp\ add{\isacharcolon}{\kern0pt}\ ket{\isacharunderscore}{\kern0pt}vec{\isacharunderscore}{\kern0pt}def{\isacharparenright}{\kern0pt}\isanewline
\ \ \ \ \ \ \ \ \ \ \ \ \isacommand{qed}\isamarkupfalse%
\isanewline
\ \ \ \ \ \ \ \ \ \ \ \ \isacommand{thus}\isamarkupfalse%
\ {\isacharquery}{\kern0pt}thesis\ \isacommand{by}\isamarkupfalse%
\ simp\isanewline
\ \ \ \ \ \ \ \ \ \ \isacommand{qed}\isamarkupfalse%
\isanewline
\ \ \ \ \ \ \ \ \ \ \isacommand{also}\isamarkupfalse%
\ \isacommand{have}\isamarkupfalse%
\ {\isachardoublequoteopen}{\isasymdots}\ {\isacharequal}{\kern0pt}\ {\isadigit{1}}{\isacharslash}{\kern0pt}sqrt\ {\isadigit{2}}\ {\isasymcdot}\isactrlsub m\ {\isacharparenleft}{\kern0pt}kron\ {\isacharparenleft}{\kern0pt}{\isasymlambda}\ l{\isachardot}{\kern0pt}\ {\isacharbar}{\kern0pt}zero{\isasymrangle}\ {\isacharplus}{\kern0pt}\ {\isacharbar}{\kern0pt}one{\isasymrangle}{\isacharparenright}{\kern0pt}\ {\isacharbrackleft}{\kern0pt}{\isadigit{1}}{\isacharbrackright}{\kern0pt}{\isacharparenright}{\kern0pt}{\isachardoublequoteclose}\ \isacommand{using}\isamarkupfalse%
\ kron{\isachardot}{\kern0pt}simps\ \isacommand{by}\isamarkupfalse%
\ auto\isanewline
\ \ \ \ \ \ \ \ \ \ \isacommand{also}\isamarkupfalse%
\ \isacommand{have}\isamarkupfalse%
\ {\isachardoublequoteopen}{\isasymdots}\ {\isacharequal}{\kern0pt}\ {\isadigit{1}}{\isacharslash}{\kern0pt}sqrt\ {\isadigit{2}}\ {\isasymcdot}\isactrlsub m\ {\isacharparenleft}{\kern0pt}kron\ {\isacharparenleft}{\kern0pt}{\isasymlambda}\ l{\isachardot}{\kern0pt}\ {\isacharbar}{\kern0pt}zero{\isasymrangle}\ {\isacharplus}{\kern0pt}\ exp\ {\isacharparenleft}{\kern0pt}{\isadigit{2}}{\isacharasterisk}{\kern0pt}{\isasymi}{\isacharasterisk}{\kern0pt}pi{\isacharasterisk}{\kern0pt}{\isadigit{0}}{\isacharslash}{\kern0pt}{\isacharparenleft}{\kern0pt}{\isadigit{2}}{\isacharcircum}{\kern0pt}l{\isacharparenright}{\kern0pt}{\isacharparenright}{\kern0pt}\ {\isasymcdot}\isactrlsub m\ {\isacharbar}{\kern0pt}one{\isasymrangle}{\isacharparenright}{\kern0pt}\ {\isacharbrackleft}{\kern0pt}{\isadigit{1}}{\isacharbrackright}{\kern0pt}{\isacharparenright}{\kern0pt}{\isachardoublequoteclose}\isanewline
\ \ \ \ \ \ \ \ \ \ \ \ \isacommand{using}\isamarkupfalse%
\ exp{\isacharunderscore}{\kern0pt}zero\ smult{\isacharunderscore}{\kern0pt}mat{\isacharunderscore}{\kern0pt}def\ \isacommand{by}\isamarkupfalse%
\ auto\isanewline
\ \ \ \ \ \ \ \ \ \ \isacommand{also}\isamarkupfalse%
\ \isacommand{have}\isamarkupfalse%
\ {\isachardoublequoteopen}{\isasymdots}\ {\isacharequal}{\kern0pt}\ reverse{\isacharunderscore}{\kern0pt}QFT{\isacharunderscore}{\kern0pt}product{\isacharunderscore}{\kern0pt}representation\ {\isadigit{0}}\ {\isacharparenleft}{\kern0pt}Suc\ {\isadigit{0}}{\isacharparenright}{\kern0pt}{\isachardoublequoteclose}\isanewline
\ \ \ \ \ \ \ \ \ \ \ \ \isacommand{using}\isamarkupfalse%
\ reverse{\isacharunderscore}{\kern0pt}QFT{\isacharunderscore}{\kern0pt}product{\isacharunderscore}{\kern0pt}representation{\isacharunderscore}{\kern0pt}def\ rev{\isacharunderscore}{\kern0pt}def\ map{\isacharunderscore}{\kern0pt}def\ \isacommand{by}\isamarkupfalse%
\ auto\isanewline
\ \ \ \ \ \ \ \ \ \ \isacommand{finally}\isamarkupfalse%
\ \isacommand{show}\isamarkupfalse%
\ {\isachardoublequoteopen}H\ {\isacharasterisk}{\kern0pt}\ {\isacharbar}{\kern0pt}unit{\isacharunderscore}{\kern0pt}vec\ {\isacharparenleft}{\kern0pt}{\isadigit{2}}\ {\isacharcircum}{\kern0pt}\ Suc\ {\isadigit{0}}{\isacharparenright}{\kern0pt}\ j{\isasymrangle}\ {\isacharequal}{\kern0pt}\ reverse{\isacharunderscore}{\kern0pt}QFT{\isacharunderscore}{\kern0pt}product{\isacharunderscore}{\kern0pt}representation\ j\ {\isacharparenleft}{\kern0pt}Suc\ {\isadigit{0}}{\isacharparenright}{\kern0pt}{\isachardoublequoteclose}\isanewline
\ \ \ \ \ \ \ \ \ \ \ \ \isacommand{using}\isamarkupfalse%
\ j{\isadigit{0}}\ \isacommand{by}\isamarkupfalse%
\ simp\isanewline
\ \ \ \ \ \ \ \ \isacommand{next}\isamarkupfalse%
\isanewline
\ \ \ \ \ \ \ \ \ \ \isacommand{assume}\isamarkupfalse%
\ j{\isadigit{1}}{\isacharcolon}{\kern0pt}{\isachardoublequoteopen}j\ {\isacharequal}{\kern0pt}\ {\isadigit{1}}{\isachardoublequoteclose}\isanewline
\ \ \ \ \ \ \ \ \ \ \isacommand{hence}\isamarkupfalse%
\ {\isachardoublequoteopen}H\ {\isacharasterisk}{\kern0pt}\ {\isacharbar}{\kern0pt}unit{\isacharunderscore}{\kern0pt}vec\ {\isacharparenleft}{\kern0pt}{\isadigit{2}}\ {\isacharcircum}{\kern0pt}\ Suc\ {\isadigit{0}}{\isacharparenright}{\kern0pt}\ j{\isasymrangle}\ {\isacharequal}{\kern0pt}\ H\ {\isacharasterisk}{\kern0pt}\ {\isacharbar}{\kern0pt}one{\isasymrangle}{\isachardoublequoteclose}\ \isacommand{by}\isamarkupfalse%
\ simp\isanewline
\ \ \ \ \ \ \ \ \ \ \isacommand{also}\isamarkupfalse%
\ \isacommand{have}\isamarkupfalse%
\ {\isachardoublequoteopen}{\isasymdots}\ {\isacharequal}{\kern0pt}\ mat{\isacharunderscore}{\kern0pt}of{\isacharunderscore}{\kern0pt}cols{\isacharunderscore}{\kern0pt}list\ {\isadigit{2}}\ {\isacharbrackleft}{\kern0pt}{\isacharbrackleft}{\kern0pt}{\isadigit{1}}{\isacharslash}{\kern0pt}sqrt{\isacharparenleft}{\kern0pt}{\isadigit{2}}{\isacharparenright}{\kern0pt}{\isacharcomma}{\kern0pt}\ {\isacharminus}{\kern0pt}{\isadigit{1}}{\isacharslash}{\kern0pt}sqrt{\isacharparenleft}{\kern0pt}{\isadigit{2}}{\isacharparenright}{\kern0pt}{\isacharbrackright}{\kern0pt}{\isacharbrackright}{\kern0pt}{\isachardoublequoteclose}\ \isacommand{using}\isamarkupfalse%
\ H{\isacharunderscore}{\kern0pt}on{\isacharunderscore}{\kern0pt}ket{\isacharunderscore}{\kern0pt}one\ \isacommand{by}\isamarkupfalse%
\ simp\isanewline
\ \ \ \ \ \ \ \ \ \ \isacommand{also}\isamarkupfalse%
\ \isacommand{have}\isamarkupfalse%
\ {\isachardoublequoteopen}{\isasymdots}\ {\isacharequal}{\kern0pt}\ {\isadigit{1}}{\isacharslash}{\kern0pt}sqrt\ {\isadigit{2}}\ {\isasymcdot}\isactrlsub m\ {\isacharparenleft}{\kern0pt}mat{\isacharunderscore}{\kern0pt}of{\isacharunderscore}{\kern0pt}cols{\isacharunderscore}{\kern0pt}list\ {\isadigit{2}}\ {\isacharbrackleft}{\kern0pt}{\isacharbrackleft}{\kern0pt}{\isadigit{1}}{\isacharcomma}{\kern0pt}{\isacharminus}{\kern0pt}{\isadigit{1}}{\isacharbrackright}{\kern0pt}{\isacharbrackright}{\kern0pt}{\isacharparenright}{\kern0pt}{\isachardoublequoteclose}\isanewline
\ \ \ \ \ \ \ \ \ \ \isacommand{proof}\isamarkupfalse%
\isanewline
\ \ \ \ \ \ \ \ \ \ \ \ \isacommand{fix}\isamarkupfalse%
\ i\ j{\isacharcolon}{\kern0pt}{\isacharcolon}{\kern0pt}nat\isanewline
\ \ \ \ \ \ \ \ \ \ \ \ \isacommand{define}\isamarkupfalse%
\ {\isasymphi}{\isadigit{1}}\ {\isasymphi}{\isadigit{2}}\ \isakeyword{where}\ {\isachardoublequoteopen}{\isasymphi}{\isadigit{1}}\ {\isacharequal}{\kern0pt}\ mat{\isacharunderscore}{\kern0pt}of{\isacharunderscore}{\kern0pt}cols{\isacharunderscore}{\kern0pt}list\ {\isadigit{2}}\ {\isacharbrackleft}{\kern0pt}{\isacharbrackleft}{\kern0pt}{\isadigit{1}}{\isacharslash}{\kern0pt}sqrt{\isacharparenleft}{\kern0pt}{\isadigit{2}}{\isacharparenright}{\kern0pt}{\isacharcomma}{\kern0pt}\ {\isacharminus}{\kern0pt}{\isadigit{1}}{\isacharslash}{\kern0pt}sqrt{\isacharparenleft}{\kern0pt}{\isadigit{2}}{\isacharparenright}{\kern0pt}{\isacharbrackright}{\kern0pt}{\isacharbrackright}{\kern0pt}{\isachardoublequoteclose}\ \isakeyword{and}\isanewline
\ \ \ \ \ \ \ \ \ \ \ \ \ \ \ \ \ \ \ \ \ \ \ \ \ \ \ \ \ \ \ {\isachardoublequoteopen}{\isasymphi}{\isadigit{2}}\ {\isacharequal}{\kern0pt}\ {\isadigit{1}}{\isacharslash}{\kern0pt}sqrt\ {\isadigit{2}}\ {\isasymcdot}\isactrlsub m\ {\isacharparenleft}{\kern0pt}mat{\isacharunderscore}{\kern0pt}of{\isacharunderscore}{\kern0pt}cols{\isacharunderscore}{\kern0pt}list\ {\isadigit{2}}\ {\isacharbrackleft}{\kern0pt}{\isacharbrackleft}{\kern0pt}{\isadigit{1}}{\isacharcomma}{\kern0pt}{\isacharminus}{\kern0pt}{\isadigit{1}}{\isacharbrackright}{\kern0pt}{\isacharbrackright}{\kern0pt}{\isacharparenright}{\kern0pt}{\isachardoublequoteclose}\isanewline
\ \ \ \ \ \ \ \ \ \ \ \ \isacommand{assume}\isamarkupfalse%
\ {\isachardoublequoteopen}i\ {\isacharless}{\kern0pt}\ dim{\isacharunderscore}{\kern0pt}row\ {\isasymphi}{\isadigit{2}}{\isachardoublequoteclose}\ \isakeyword{and}\ {\isachardoublequoteopen}j\ {\isacharless}{\kern0pt}\ dim{\isacharunderscore}{\kern0pt}col\ {\isasymphi}{\isadigit{2}}{\isachardoublequoteclose}\isanewline
\ \ \ \ \ \ \ \ \ \ \ \ \isacommand{hence}\isamarkupfalse%
\ a{\isadigit{3}}{\isacharcolon}{\kern0pt}{\isachardoublequoteopen}i\ {\isasymin}\ {\isacharbraceleft}{\kern0pt}{\isadigit{0}}{\isacharcomma}{\kern0pt}{\isadigit{1}}{\isacharbraceright}{\kern0pt}\ {\isasymand}\ j\ {\isacharequal}{\kern0pt}\ {\isadigit{0}}{\isachardoublequoteclose}\ \isanewline
\ \ \ \ \ \ \ \ \ \ \ \ \ \ \isacommand{using}\isamarkupfalse%
\ {\isasymphi}{\isadigit{2}}{\isacharunderscore}{\kern0pt}def\ mat{\isacharunderscore}{\kern0pt}of{\isacharunderscore}{\kern0pt}cols{\isacharunderscore}{\kern0pt}list{\isacharunderscore}{\kern0pt}def\ numerals{\isacharparenleft}{\kern0pt}{\isadigit{2}}{\isacharparenright}{\kern0pt}\ less{\isacharunderscore}{\kern0pt}{\isadigit{2}}{\isacharunderscore}{\kern0pt}cases\ \isacommand{by}\isamarkupfalse%
\ simp\isanewline
\ \ \ \ \ \ \ \ \ \ \ \ \isacommand{have}\isamarkupfalse%
\ {\isachardoublequoteopen}{\isasymphi}{\isadigit{1}}\ {\isachardollar}{\kern0pt}{\isachardollar}{\kern0pt}\ {\isacharparenleft}{\kern0pt}{\isadigit{0}}{\isacharcomma}{\kern0pt}{\isadigit{0}}{\isacharparenright}{\kern0pt}\ {\isacharequal}{\kern0pt}\ {\isasymphi}{\isadigit{2}}\ {\isachardollar}{\kern0pt}{\isachardollar}{\kern0pt}\ {\isacharparenleft}{\kern0pt}{\isadigit{0}}{\isacharcomma}{\kern0pt}{\isadigit{0}}{\isacharparenright}{\kern0pt}{\isachardoublequoteclose}\isanewline
\ \ \ \ \ \ \ \ \ \ \ \ \ \ \isacommand{using}\isamarkupfalse%
\ {\isasymphi}{\isadigit{1}}{\isacharunderscore}{\kern0pt}def\ {\isasymphi}{\isadigit{2}}{\isacharunderscore}{\kern0pt}def\ smult{\isacharunderscore}{\kern0pt}def\ mat{\isacharunderscore}{\kern0pt}of{\isacharunderscore}{\kern0pt}cols{\isacharunderscore}{\kern0pt}list{\isacharunderscore}{\kern0pt}def\ \isacommand{by}\isamarkupfalse%
\ simp\isanewline
\ \ \ \ \ \ \ \ \ \ \ \ \isacommand{moreover}\isamarkupfalse%
\ \isacommand{have}\isamarkupfalse%
\ {\isachardoublequoteopen}{\isasymphi}{\isadigit{1}}\ {\isachardollar}{\kern0pt}{\isachardollar}{\kern0pt}\ {\isacharparenleft}{\kern0pt}{\isadigit{1}}{\isacharcomma}{\kern0pt}{\isadigit{0}}{\isacharparenright}{\kern0pt}\ {\isacharequal}{\kern0pt}\ {\isasymphi}{\isadigit{2}}\ {\isachardollar}{\kern0pt}{\isachardollar}{\kern0pt}\ {\isacharparenleft}{\kern0pt}{\isadigit{1}}{\isacharcomma}{\kern0pt}{\isadigit{0}}{\isacharparenright}{\kern0pt}{\isachardoublequoteclose}\isanewline
\ \ \ \ \ \ \ \ \ \ \ \ \ \ \isacommand{using}\isamarkupfalse%
\ {\isasymphi}{\isadigit{1}}{\isacharunderscore}{\kern0pt}def\ {\isasymphi}{\isadigit{2}}{\isacharunderscore}{\kern0pt}def\ smult{\isacharunderscore}{\kern0pt}def\ mat{\isacharunderscore}{\kern0pt}of{\isacharunderscore}{\kern0pt}cols{\isacharunderscore}{\kern0pt}list{\isacharunderscore}{\kern0pt}def\ \isacommand{by}\isamarkupfalse%
\ simp\isanewline
\ \ \ \ \ \ \ \ \ \ \ \ \isacommand{ultimately}\isamarkupfalse%
\ \isacommand{show}\isamarkupfalse%
\ {\isachardoublequoteopen}{\isasymphi}{\isadigit{1}}\ {\isachardollar}{\kern0pt}{\isachardollar}{\kern0pt}\ {\isacharparenleft}{\kern0pt}i{\isacharcomma}{\kern0pt}j{\isacharparenright}{\kern0pt}\ {\isacharequal}{\kern0pt}\ {\isasymphi}{\isadigit{2}}\ {\isachardollar}{\kern0pt}{\isachardollar}{\kern0pt}\ {\isacharparenleft}{\kern0pt}i{\isacharcomma}{\kern0pt}j{\isacharparenright}{\kern0pt}{\isachardoublequoteclose}\ \isacommand{using}\isamarkupfalse%
\ a{\isadigit{3}}\ \isacommand{by}\isamarkupfalse%
\ auto\isanewline
\ \ \ \ \ \ \ \ \ \ \isacommand{next}\isamarkupfalse%
\isanewline
\ \ \ \ \ \ \ \ \ \ \ \ \isacommand{define}\isamarkupfalse%
\ {\isasymphi}{\isadigit{1}}\ {\isasymphi}{\isadigit{2}}\ \isakeyword{where}\ {\isachardoublequoteopen}{\isasymphi}{\isadigit{1}}\ {\isacharequal}{\kern0pt}\ mat{\isacharunderscore}{\kern0pt}of{\isacharunderscore}{\kern0pt}cols{\isacharunderscore}{\kern0pt}list\ {\isadigit{2}}\ {\isacharbrackleft}{\kern0pt}{\isacharbrackleft}{\kern0pt}{\isadigit{1}}{\isacharslash}{\kern0pt}sqrt{\isacharparenleft}{\kern0pt}{\isadigit{2}}{\isacharparenright}{\kern0pt}{\isacharcomma}{\kern0pt}\ {\isacharminus}{\kern0pt}{\isadigit{1}}{\isacharslash}{\kern0pt}sqrt{\isacharparenleft}{\kern0pt}{\isadigit{2}}{\isacharparenright}{\kern0pt}{\isacharbrackright}{\kern0pt}{\isacharbrackright}{\kern0pt}{\isachardoublequoteclose}\ \isakeyword{and}\isanewline
\ \ \ \ \ \ \ \ \ \ \ \ \ \ \ \ \ \ \ \ \ \ \ \ \ \ \ \ \ \ \ {\isachardoublequoteopen}{\isasymphi}{\isadigit{2}}\ {\isacharequal}{\kern0pt}\ {\isadigit{1}}{\isacharslash}{\kern0pt}sqrt\ {\isadigit{2}}\ {\isasymcdot}\isactrlsub m\ {\isacharparenleft}{\kern0pt}mat{\isacharunderscore}{\kern0pt}of{\isacharunderscore}{\kern0pt}cols{\isacharunderscore}{\kern0pt}list\ {\isadigit{2}}\ {\isacharbrackleft}{\kern0pt}{\isacharbrackleft}{\kern0pt}{\isadigit{1}}{\isacharcomma}{\kern0pt}{\isacharminus}{\kern0pt}{\isadigit{1}}{\isacharbrackright}{\kern0pt}{\isacharbrackright}{\kern0pt}{\isacharparenright}{\kern0pt}{\isachardoublequoteclose}\isanewline
\ \ \ \ \ \ \ \ \ \ \ \ \isacommand{then}\isamarkupfalse%
\ \isacommand{show}\isamarkupfalse%
\ {\isachardoublequoteopen}dim{\isacharunderscore}{\kern0pt}row\ {\isasymphi}{\isadigit{1}}\ {\isacharequal}{\kern0pt}\ dim{\isacharunderscore}{\kern0pt}row\ {\isasymphi}{\isadigit{2}}{\isachardoublequoteclose}\ \isacommand{using}\isamarkupfalse%
\ smult{\isacharunderscore}{\kern0pt}def\ mat{\isacharunderscore}{\kern0pt}of{\isacharunderscore}{\kern0pt}cols{\isacharunderscore}{\kern0pt}list{\isacharunderscore}{\kern0pt}def\ \isacommand{by}\isamarkupfalse%
\ simp\isanewline
\ \ \ \ \ \ \ \ \ \ \isacommand{next}\isamarkupfalse%
\isanewline
\ \ \ \ \ \ \ \ \ \ \ \ \isacommand{define}\isamarkupfalse%
\ {\isasymphi}{\isadigit{1}}\ {\isasymphi}{\isadigit{2}}\ \isakeyword{where}\ {\isachardoublequoteopen}{\isasymphi}{\isadigit{1}}\ {\isacharequal}{\kern0pt}\ mat{\isacharunderscore}{\kern0pt}of{\isacharunderscore}{\kern0pt}cols{\isacharunderscore}{\kern0pt}list\ {\isadigit{2}}\ {\isacharbrackleft}{\kern0pt}{\isacharbrackleft}{\kern0pt}{\isadigit{1}}{\isacharslash}{\kern0pt}sqrt{\isacharparenleft}{\kern0pt}{\isadigit{2}}{\isacharparenright}{\kern0pt}{\isacharcomma}{\kern0pt}\ {\isacharminus}{\kern0pt}{\isadigit{1}}{\isacharslash}{\kern0pt}sqrt{\isacharparenleft}{\kern0pt}{\isadigit{2}}{\isacharparenright}{\kern0pt}{\isacharbrackright}{\kern0pt}{\isacharbrackright}{\kern0pt}{\isachardoublequoteclose}\ \isakeyword{and}\isanewline
\ \ \ \ \ \ \ \ \ \ \ \ \ \ \ \ \ \ \ \ \ \ \ \ \ \ \ \ \ \ \ {\isachardoublequoteopen}{\isasymphi}{\isadigit{2}}\ {\isacharequal}{\kern0pt}\ {\isadigit{1}}{\isacharslash}{\kern0pt}sqrt\ {\isadigit{2}}\ {\isasymcdot}\isactrlsub m\ {\isacharparenleft}{\kern0pt}mat{\isacharunderscore}{\kern0pt}of{\isacharunderscore}{\kern0pt}cols{\isacharunderscore}{\kern0pt}list\ {\isadigit{2}}\ {\isacharbrackleft}{\kern0pt}{\isacharbrackleft}{\kern0pt}{\isadigit{1}}{\isacharcomma}{\kern0pt}{\isacharminus}{\kern0pt}{\isadigit{1}}{\isacharbrackright}{\kern0pt}{\isacharbrackright}{\kern0pt}{\isacharparenright}{\kern0pt}{\isachardoublequoteclose}\isanewline
\ \ \ \ \ \ \ \ \ \ \ \ \isacommand{then}\isamarkupfalse%
\ \isacommand{show}\isamarkupfalse%
\ {\isachardoublequoteopen}dim{\isacharunderscore}{\kern0pt}col\ {\isasymphi}{\isadigit{1}}\ {\isacharequal}{\kern0pt}\ dim{\isacharunderscore}{\kern0pt}col\ {\isasymphi}{\isadigit{2}}{\isachardoublequoteclose}\ \isacommand{using}\isamarkupfalse%
\ smult{\isacharunderscore}{\kern0pt}def\ mat{\isacharunderscore}{\kern0pt}of{\isacharunderscore}{\kern0pt}cols{\isacharunderscore}{\kern0pt}list{\isacharunderscore}{\kern0pt}def\ \isacommand{by}\isamarkupfalse%
\ simp\isanewline
\ \ \ \ \ \ \ \ \ \ \isacommand{qed}\isamarkupfalse%
\isanewline
\ \ \ \ \ \ \ \ \ \ \isacommand{also}\isamarkupfalse%
\ \isacommand{have}\isamarkupfalse%
\ {\isachardoublequoteopen}{\isasymdots}\ {\isacharequal}{\kern0pt}\ {\isadigit{1}}{\isacharslash}{\kern0pt}sqrt\ {\isadigit{2}}\ {\isasymcdot}\isactrlsub m\ {\isacharparenleft}{\kern0pt}\ {\isacharbar}{\kern0pt}zero{\isasymrangle}\ {\isacharminus}{\kern0pt}\ {\isacharbar}{\kern0pt}one{\isasymrangle}{\isacharparenright}{\kern0pt}{\isachardoublequoteclose}\isanewline
\ \ \ \ \ \ \ \ \ \ \isacommand{proof}\isamarkupfalse%
\ {\isacharminus}{\kern0pt}\isanewline
\ \ \ \ \ \ \ \ \ \ \ \ \isacommand{have}\isamarkupfalse%
\ {\isachardoublequoteopen}mat{\isacharunderscore}{\kern0pt}of{\isacharunderscore}{\kern0pt}cols{\isacharunderscore}{\kern0pt}list\ {\isadigit{2}}\ {\isacharbrackleft}{\kern0pt}{\isacharbrackleft}{\kern0pt}{\isadigit{1}}{\isacharcomma}{\kern0pt}{\isacharminus}{\kern0pt}{\isadigit{1}}{\isacharbrackright}{\kern0pt}{\isacharbrackright}{\kern0pt}\ {\isacharequal}{\kern0pt}\ {\isacharbar}{\kern0pt}zero{\isasymrangle}\ {\isacharminus}{\kern0pt}\ {\isacharbar}{\kern0pt}one{\isasymrangle}{\isachardoublequoteclose}\isanewline
\ \ \ \ \ \ \ \ \ \ \ \ \isacommand{proof}\isamarkupfalse%
\isanewline
\ \ \ \ \ \ \ \ \ \ \ \ \ \ \isacommand{fix}\isamarkupfalse%
\ i\ j{\isacharcolon}{\kern0pt}{\isacharcolon}{\kern0pt}nat\isanewline
\ \ \ \ \ \ \ \ \ \ \ \ \ \ \isacommand{define}\isamarkupfalse%
\ r{\isadigit{1}}\ r{\isadigit{2}}\ \isakeyword{where}\ {\isachardoublequoteopen}r{\isadigit{1}}\ {\isacharequal}{\kern0pt}\ mat{\isacharunderscore}{\kern0pt}of{\isacharunderscore}{\kern0pt}cols{\isacharunderscore}{\kern0pt}list\ {\isadigit{2}}\ {\isacharbrackleft}{\kern0pt}{\isacharbrackleft}{\kern0pt}{\isadigit{1}}{\isacharcomma}{\kern0pt}{\isacharminus}{\kern0pt}{\isadigit{1}}{\isacharbrackright}{\kern0pt}{\isacharbrackright}{\kern0pt}{\isachardoublequoteclose}\ \isakeyword{and}\ {\isachardoublequoteopen}r{\isadigit{2}}\ {\isacharequal}{\kern0pt}\ {\isacharbar}{\kern0pt}zero{\isasymrangle}\ {\isacharminus}{\kern0pt}\ {\isacharbar}{\kern0pt}one{\isasymrangle}{\isachardoublequoteclose}\isanewline
\ \ \ \ \ \ \ \ \ \ \ \ \ \ \isacommand{assume}\isamarkupfalse%
\ {\isachardoublequoteopen}i\ {\isacharless}{\kern0pt}\ dim{\isacharunderscore}{\kern0pt}row\ r{\isadigit{2}}{\isachardoublequoteclose}\ \isakeyword{and}\ {\isachardoublequoteopen}j\ {\isacharless}{\kern0pt}\ dim{\isacharunderscore}{\kern0pt}col\ r{\isadigit{2}}{\isachardoublequoteclose}\isanewline
\ \ \ \ \ \ \ \ \ \ \ \ \ \ \isacommand{hence}\isamarkupfalse%
\ a{\isadigit{4}}{\isacharcolon}{\kern0pt}{\isachardoublequoteopen}i\ {\isasymin}\ {\isacharbraceleft}{\kern0pt}{\isadigit{0}}{\isacharcomma}{\kern0pt}{\isadigit{1}}{\isacharbraceright}{\kern0pt}\ {\isasymand}\ j{\isacharequal}{\kern0pt}{\isadigit{0}}{\isachardoublequoteclose}\ \isanewline
\ \ \ \ \ \ \ \ \ \ \ \ \ \ \ \ \isacommand{using}\isamarkupfalse%
\ ket{\isacharunderscore}{\kern0pt}vec{\isacharunderscore}{\kern0pt}def\ index{\isacharunderscore}{\kern0pt}add{\isacharunderscore}{\kern0pt}mat\ \isacommand{by}\isamarkupfalse%
\ {\isacharparenleft}{\kern0pt}simp\ add{\isacharcolon}{\kern0pt}\ less{\isacharunderscore}{\kern0pt}{\isadigit{2}}{\isacharunderscore}{\kern0pt}cases\ r{\isadigit{2}}{\isacharunderscore}{\kern0pt}def{\isacharparenright}{\kern0pt}\isanewline
\ \ \ \ \ \ \ \ \ \ \ \ \ \ \isacommand{have}\isamarkupfalse%
\ {\isachardoublequoteopen}r{\isadigit{1}}\ {\isachardollar}{\kern0pt}{\isachardollar}{\kern0pt}\ {\isacharparenleft}{\kern0pt}{\isadigit{0}}{\isacharcomma}{\kern0pt}{\isadigit{0}}{\isacharparenright}{\kern0pt}\ {\isacharequal}{\kern0pt}\ r{\isadigit{2}}\ {\isachardollar}{\kern0pt}{\isachardollar}{\kern0pt}\ {\isacharparenleft}{\kern0pt}{\isadigit{0}}{\isacharcomma}{\kern0pt}{\isadigit{0}}{\isacharparenright}{\kern0pt}{\isachardoublequoteclose}\ \isacommand{using}\isamarkupfalse%
\ r{\isadigit{1}}{\isacharunderscore}{\kern0pt}def\ r{\isadigit{2}}{\isacharunderscore}{\kern0pt}def\ mat{\isacharunderscore}{\kern0pt}of{\isacharunderscore}{\kern0pt}cols{\isacharunderscore}{\kern0pt}list{\isacharunderscore}{\kern0pt}def\isanewline
\ \ \ \ \ \ \ \ \ \ \ \ \ \ \ \ \isacommand{by}\isamarkupfalse%
\ {\isacharparenleft}{\kern0pt}smt\ {\isacharparenleft}{\kern0pt}verit{\isacharcomma}{\kern0pt}\ ccfv{\isacharunderscore}{\kern0pt}threshold{\isacharparenright}{\kern0pt}\ One{\isacharunderscore}{\kern0pt}nat{\isacharunderscore}{\kern0pt}def\ add{\isachardot}{\kern0pt}commute\ diff{\isacharunderscore}{\kern0pt}zero\ dim{\isacharunderscore}{\kern0pt}row{\isacharunderscore}{\kern0pt}mat{\isacharparenleft}{\kern0pt}{\isadigit{1}}{\isacharparenright}{\kern0pt}\ \isanewline
\ \ \ \ \ \ \ \ \ \ \ \ \ \ \ \ \ \ \ \ index{\isacharunderscore}{\kern0pt}mat{\isacharparenleft}{\kern0pt}{\isadigit{1}}{\isacharparenright}{\kern0pt}\ index{\isacharunderscore}{\kern0pt}mat{\isacharunderscore}{\kern0pt}of{\isacharunderscore}{\kern0pt}cols{\isacharunderscore}{\kern0pt}list\ ket{\isacharunderscore}{\kern0pt}one{\isacharunderscore}{\kern0pt}is{\isacharunderscore}{\kern0pt}state\ ket{\isacharunderscore}{\kern0pt}one{\isacharunderscore}{\kern0pt}to{\isacharunderscore}{\kern0pt}mat{\isacharunderscore}{\kern0pt}of{\isacharunderscore}{\kern0pt}cols{\isacharunderscore}{\kern0pt}list\ \isanewline
\ \ \ \ \ \ \ \ \ \ \ \ \ \ \ \ \ \ \ \ ket{\isacharunderscore}{\kern0pt}zero{\isacharunderscore}{\kern0pt}to{\isacharunderscore}{\kern0pt}mat{\isacharunderscore}{\kern0pt}of{\isacharunderscore}{\kern0pt}cols{\isacharunderscore}{\kern0pt}list\ list{\isachardot}{\kern0pt}size{\isacharparenleft}{\kern0pt}{\isadigit{3}}{\isacharparenright}{\kern0pt}\ list{\isachardot}{\kern0pt}size{\isacharparenleft}{\kern0pt}{\isadigit{4}}{\isacharparenright}{\kern0pt}\ minus{\isacharunderscore}{\kern0pt}mat{\isacharunderscore}{\kern0pt}def\ nth{\isacharunderscore}{\kern0pt}Cons{\isacharunderscore}{\kern0pt}{\isadigit{0}}\ \isanewline
\ \ \ \ \ \ \ \ \ \ \ \ \ \ \ \ \ \ \ \ plus{\isacharunderscore}{\kern0pt}{\isadigit{1}}{\isacharunderscore}{\kern0pt}eq{\isacharunderscore}{\kern0pt}Suc\ pos{\isadigit{2}}\ state{\isacharunderscore}{\kern0pt}def\ zero{\isacharunderscore}{\kern0pt}less{\isacharunderscore}{\kern0pt}one{\isacharunderscore}{\kern0pt}class{\isachardot}{\kern0pt}zero{\isacharunderscore}{\kern0pt}less{\isacharunderscore}{\kern0pt}one{\isacharparenright}{\kern0pt}\isanewline
\ \ \ \ \ \ \ \ \ \ \ \ \ \ \isacommand{moreover}\isamarkupfalse%
\ \isacommand{have}\isamarkupfalse%
\ {\isachardoublequoteopen}r{\isadigit{1}}\ {\isachardollar}{\kern0pt}{\isachardollar}{\kern0pt}\ {\isacharparenleft}{\kern0pt}{\isadigit{1}}{\isacharcomma}{\kern0pt}{\isadigit{0}}{\isacharparenright}{\kern0pt}\ {\isacharequal}{\kern0pt}\ r{\isadigit{2}}\ {\isachardollar}{\kern0pt}{\isachardollar}{\kern0pt}\ {\isacharparenleft}{\kern0pt}{\isadigit{1}}{\isacharcomma}{\kern0pt}{\isadigit{0}}{\isacharparenright}{\kern0pt}{\isachardoublequoteclose}\ \isanewline
\ \ \ \ \ \ \ \ \ \ \ \ \ \ \ \ \isacommand{using}\isamarkupfalse%
\ r{\isadigit{1}}{\isacharunderscore}{\kern0pt}def\ r{\isadigit{2}}{\isacharunderscore}{\kern0pt}def\ mat{\isacharunderscore}{\kern0pt}of{\isacharunderscore}{\kern0pt}cols{\isacharunderscore}{\kern0pt}list{\isacharunderscore}{\kern0pt}def\ ket{\isacharunderscore}{\kern0pt}vec{\isacharunderscore}{\kern0pt}def\ \isacommand{by}\isamarkupfalse%
\ simp\isanewline
\ \ \ \ \ \ \ \ \ \ \ \ \ \ \isacommand{ultimately}\isamarkupfalse%
\ \isacommand{show}\isamarkupfalse%
\ {\isachardoublequoteopen}r{\isadigit{1}}\ {\isachardollar}{\kern0pt}{\isachardollar}{\kern0pt}\ {\isacharparenleft}{\kern0pt}i{\isacharcomma}{\kern0pt}j{\isacharparenright}{\kern0pt}\ {\isacharequal}{\kern0pt}\ r{\isadigit{2}}\ {\isachardollar}{\kern0pt}{\isachardollar}{\kern0pt}\ {\isacharparenleft}{\kern0pt}i{\isacharcomma}{\kern0pt}j{\isacharparenright}{\kern0pt}{\isachardoublequoteclose}\ \isacommand{using}\isamarkupfalse%
\ a{\isadigit{4}}\ \isacommand{by}\isamarkupfalse%
\ auto\isanewline
\ \ \ \ \ \ \ \ \ \ \ \ \isacommand{next}\isamarkupfalse%
\isanewline
\ \ \ \ \ \ \ \ \ \ \ \ \ \ \isacommand{define}\isamarkupfalse%
\ r{\isadigit{1}}\ r{\isadigit{2}}\ \isakeyword{where}\ {\isachardoublequoteopen}r{\isadigit{1}}\ {\isacharequal}{\kern0pt}\ mat{\isacharunderscore}{\kern0pt}of{\isacharunderscore}{\kern0pt}cols{\isacharunderscore}{\kern0pt}list\ {\isadigit{2}}\ {\isacharbrackleft}{\kern0pt}{\isacharbrackleft}{\kern0pt}{\isadigit{1}}{\isacharcomma}{\kern0pt}{\isacharminus}{\kern0pt}{\isadigit{1}}{\isacharbrackright}{\kern0pt}{\isacharbrackright}{\kern0pt}{\isachardoublequoteclose}\ \isakeyword{and}\ {\isachardoublequoteopen}r{\isadigit{2}}\ {\isacharequal}{\kern0pt}\ {\isacharbar}{\kern0pt}zero{\isasymrangle}\ {\isacharminus}{\kern0pt}\ {\isacharbar}{\kern0pt}one{\isasymrangle}{\isachardoublequoteclose}\isanewline
\ \ \ \ \ \ \ \ \ \ \ \ \ \ \isacommand{thus}\isamarkupfalse%
\ {\isachardoublequoteopen}dim{\isacharunderscore}{\kern0pt}row\ r{\isadigit{1}}\ {\isacharequal}{\kern0pt}\ dim{\isacharunderscore}{\kern0pt}row\ r{\isadigit{2}}{\isachardoublequoteclose}\ \isacommand{using}\isamarkupfalse%
\ mat{\isacharunderscore}{\kern0pt}of{\isacharunderscore}{\kern0pt}cols{\isacharunderscore}{\kern0pt}list{\isacharunderscore}{\kern0pt}def\ ket{\isacharunderscore}{\kern0pt}vec{\isacharunderscore}{\kern0pt}def\ \isacommand{by}\isamarkupfalse%
\ simp\isanewline
\ \ \ \ \ \ \ \ \ \ \ \ \isacommand{next}\isamarkupfalse%
\isanewline
\ \ \ \ \ \ \ \ \ \ \ \ \ \ \isacommand{define}\isamarkupfalse%
\ r{\isadigit{1}}\ r{\isadigit{2}}\ \isakeyword{where}\ {\isachardoublequoteopen}r{\isadigit{1}}\ {\isacharequal}{\kern0pt}\ mat{\isacharunderscore}{\kern0pt}of{\isacharunderscore}{\kern0pt}cols{\isacharunderscore}{\kern0pt}list\ {\isadigit{2}}\ {\isacharbrackleft}{\kern0pt}{\isacharbrackleft}{\kern0pt}{\isadigit{1}}{\isacharcomma}{\kern0pt}{\isacharminus}{\kern0pt}{\isadigit{1}}{\isacharbrackright}{\kern0pt}{\isacharbrackright}{\kern0pt}{\isachardoublequoteclose}\ \isakeyword{and}\ {\isachardoublequoteopen}r{\isadigit{2}}\ {\isacharequal}{\kern0pt}\ {\isacharbar}{\kern0pt}zero{\isasymrangle}\ {\isacharminus}{\kern0pt}\ {\isacharbar}{\kern0pt}one{\isasymrangle}{\isachardoublequoteclose}\isanewline
\ \ \ \ \ \ \ \ \ \ \ \ \ \ \isacommand{thus}\isamarkupfalse%
\ {\isachardoublequoteopen}dim{\isacharunderscore}{\kern0pt}col\ r{\isadigit{1}}\ {\isacharequal}{\kern0pt}\ dim{\isacharunderscore}{\kern0pt}col\ r{\isadigit{2}}{\isachardoublequoteclose}\ \isacommand{using}\isamarkupfalse%
\ mat{\isacharunderscore}{\kern0pt}of{\isacharunderscore}{\kern0pt}cols{\isacharunderscore}{\kern0pt}list{\isacharunderscore}{\kern0pt}def\ ket{\isacharunderscore}{\kern0pt}vec{\isacharunderscore}{\kern0pt}def\ \isacommand{by}\isamarkupfalse%
\ simp\isanewline
\ \ \ \ \ \ \ \ \ \ \ \ \isacommand{qed}\isamarkupfalse%
\isanewline
\ \ \ \ \ \ \ \ \ \ \ \ \isacommand{thus}\isamarkupfalse%
\ {\isacharquery}{\kern0pt}thesis\ \isacommand{by}\isamarkupfalse%
\ simp\isanewline
\ \ \ \ \ \ \ \ \ \ \isacommand{qed}\isamarkupfalse%
\isanewline
\ \ \ \ \ \ \ \ \ \ \isacommand{also}\isamarkupfalse%
\ \isacommand{have}\isamarkupfalse%
\ {\isachardoublequoteopen}{\isasymdots}\ {\isacharequal}{\kern0pt}\ {\isadigit{1}}{\isacharslash}{\kern0pt}sqrt\ {\isadigit{2}}\ {\isasymcdot}\isactrlsub m\ {\isacharparenleft}{\kern0pt}kron\ {\isacharparenleft}{\kern0pt}{\isasymlambda}l{\isachardot}{\kern0pt}\ {\isacharbar}{\kern0pt}zero{\isasymrangle}\ {\isacharminus}{\kern0pt}\ {\isacharbar}{\kern0pt}one{\isasymrangle}{\isacharparenright}{\kern0pt}\ {\isacharbrackleft}{\kern0pt}{\isadigit{1}}{\isacharbrackright}{\kern0pt}{\isacharparenright}{\kern0pt}{\isachardoublequoteclose}\isanewline
\ \ \ \ \ \ \ \ \ \ \ \ \isacommand{using}\isamarkupfalse%
\ kron{\isachardot}{\kern0pt}simps\ \isacommand{by}\isamarkupfalse%
\ auto\isanewline
\ \ \ \ \ \ \ \ \ \ \isacommand{also}\isamarkupfalse%
\ \isacommand{have}\isamarkupfalse%
\ {\isachardoublequoteopen}{\isasymdots}\ {\isacharequal}{\kern0pt}\ {\isadigit{1}}{\isacharslash}{\kern0pt}sqrt\ {\isadigit{2}}\ {\isasymcdot}\isactrlsub m\ {\isacharparenleft}{\kern0pt}kron\ {\isacharparenleft}{\kern0pt}{\isasymlambda}l{\isachardot}{\kern0pt}\ {\isacharbar}{\kern0pt}zero{\isasymrangle}\ {\isacharplus}{\kern0pt}\ exp\ {\isacharparenleft}{\kern0pt}{\isadigit{2}}{\isacharasterisk}{\kern0pt}{\isasymi}{\isacharasterisk}{\kern0pt}pi{\isacharasterisk}{\kern0pt}{\isadigit{1}}{\isacharslash}{\kern0pt}{\isacharparenleft}{\kern0pt}{\isadigit{2}}{\isacharcircum}{\kern0pt}l{\isacharparenright}{\kern0pt}{\isacharparenright}{\kern0pt}\ {\isasymcdot}\isactrlsub m\ {\isacharbar}{\kern0pt}one{\isasymrangle}{\isacharparenright}{\kern0pt}\ {\isacharbrackleft}{\kern0pt}{\isadigit{1}}{\isacharbrackright}{\kern0pt}{\isacharparenright}{\kern0pt}{\isachardoublequoteclose}\isanewline
\ \ \ \ \ \ \ \ \ \ \isacommand{proof}\isamarkupfalse%
\ {\isacharminus}{\kern0pt}\isanewline
\ \ \ \ \ \ \ \ \ \ \ \ \isacommand{have}\isamarkupfalse%
\ {\isachardoublequoteopen}exp\ {\isacharparenleft}{\kern0pt}{\isadigit{2}}{\isacharasterisk}{\kern0pt}{\isasymi}{\isacharasterisk}{\kern0pt}pi{\isacharasterisk}{\kern0pt}{\isadigit{1}}{\isacharslash}{\kern0pt}{\isacharparenleft}{\kern0pt}{\isadigit{2}}{\isacharcircum}{\kern0pt}{\isadigit{1}}{\isacharparenright}{\kern0pt}{\isacharparenright}{\kern0pt}\ {\isacharequal}{\kern0pt}\ {\isacharminus}{\kern0pt}{\isadigit{1}}{\isachardoublequoteclose}\ \isacommand{using}\isamarkupfalse%
\ exp{\isacharunderscore}{\kern0pt}pi{\isacharunderscore}{\kern0pt}i\ \isacommand{by}\isamarkupfalse%
\ auto\isanewline
\ \ \ \ \ \ \ \ \ \ \ \ \isacommand{hence}\isamarkupfalse%
\ {\isachardoublequoteopen}{\isacharbar}{\kern0pt}zero{\isasymrangle}\ {\isacharplus}{\kern0pt}\ exp\ {\isacharparenleft}{\kern0pt}{\isadigit{2}}{\isacharasterisk}{\kern0pt}{\isasymi}{\isacharasterisk}{\kern0pt}pi{\isacharasterisk}{\kern0pt}{\isadigit{1}}{\isacharslash}{\kern0pt}{\isacharparenleft}{\kern0pt}{\isadigit{2}}{\isacharcircum}{\kern0pt}{\isadigit{1}}{\isacharparenright}{\kern0pt}{\isacharparenright}{\kern0pt}\ {\isasymcdot}\isactrlsub m\ {\isacharbar}{\kern0pt}one{\isasymrangle}\ {\isacharequal}{\kern0pt}\ {\isacharbar}{\kern0pt}zero{\isasymrangle}\ {\isacharplus}{\kern0pt}\ {\isacharparenleft}{\kern0pt}{\isacharminus}{\kern0pt}{\isadigit{1}}{\isacharparenright}{\kern0pt}\ {\isasymcdot}\isactrlsub m\ {\isacharbar}{\kern0pt}one{\isasymrangle}{\isachardoublequoteclose}\ \isacommand{by}\isamarkupfalse%
\ simp\isanewline
\ \ \ \ \ \ \ \ \ \ \ \ \isacommand{also}\isamarkupfalse%
\ \isacommand{have}\isamarkupfalse%
\ {\isachardoublequoteopen}{\isasymdots}\ {\isacharequal}{\kern0pt}\ {\isacharbar}{\kern0pt}zero{\isasymrangle}\ {\isacharminus}{\kern0pt}\ {\isacharbar}{\kern0pt}one{\isasymrangle}{\isachardoublequoteclose}\ \isacommand{by}\isamarkupfalse%
\ auto\isanewline
\ \ \ \ \ \ \ \ \ \ \ \ \isacommand{thus}\isamarkupfalse%
\ {\isacharquery}{\kern0pt}thesis\ \isacommand{by}\isamarkupfalse%
\ auto\isanewline
\ \ \ \ \ \ \ \ \ \ \isacommand{qed}\isamarkupfalse%
\isanewline
\ \ \ \ \ \ \ \ \ \ \isacommand{also}\isamarkupfalse%
\ \isacommand{have}\isamarkupfalse%
\ {\isachardoublequoteopen}{\isasymdots}\ {\isacharequal}{\kern0pt}\ reverse{\isacharunderscore}{\kern0pt}QFT{\isacharunderscore}{\kern0pt}product{\isacharunderscore}{\kern0pt}representation\ {\isadigit{1}}\ {\isacharparenleft}{\kern0pt}Suc\ {\isadigit{0}}{\isacharparenright}{\kern0pt}{\isachardoublequoteclose}\ \isanewline
\ \ \ \ \ \ \ \ \ \ \ \ \isacommand{using}\isamarkupfalse%
\ reverse{\isacharunderscore}{\kern0pt}QFT{\isacharunderscore}{\kern0pt}product{\isacharunderscore}{\kern0pt}representation{\isacharunderscore}{\kern0pt}def\ \isacommand{by}\isamarkupfalse%
\ auto\isanewline
\ \ \ \ \ \ \ \ \ \ \isacommand{finally}\isamarkupfalse%
\ \isacommand{show}\isamarkupfalse%
\ {\isachardoublequoteopen}H\ {\isacharasterisk}{\kern0pt}\ {\isacharbar}{\kern0pt}unit{\isacharunderscore}{\kern0pt}vec\ {\isacharparenleft}{\kern0pt}{\isadigit{2}}\ {\isacharcircum}{\kern0pt}\ Suc\ {\isadigit{0}}{\isacharparenright}{\kern0pt}\ j{\isasymrangle}\ {\isacharequal}{\kern0pt}\ reverse{\isacharunderscore}{\kern0pt}QFT{\isacharunderscore}{\kern0pt}product{\isacharunderscore}{\kern0pt}representation\ j\ {\isacharparenleft}{\kern0pt}Suc\ {\isadigit{0}}{\isacharparenright}{\kern0pt}{\isachardoublequoteclose}\isanewline
\ \ \ \ \ \ \ \ \ \ \ \ \isacommand{using}\isamarkupfalse%
\ j{\isadigit{1}}\ \isacommand{by}\isamarkupfalse%
\ simp\isanewline
\ \ \ \ \ \ \ \ \isacommand{qed}\isamarkupfalse%
\isanewline
\ \ \ \ \ \ \ \ \isacommand{finally}\isamarkupfalse%
\ \isacommand{show}\isamarkupfalse%
\ {\isacharquery}{\kern0pt}thesis\ \isacommand{by}\isamarkupfalse%
\ this\isanewline
\ \ \ \ \ \ \ \ \isacommand{qed}\isamarkupfalse%
\isanewline
\ \ \ \ \ \ \isacommand{qed}\isamarkupfalse%
\isanewline
\ \ \ \ \isacommand{qed}\isamarkupfalse%
\isanewline
\isacommand{next}\isamarkupfalse%
\isanewline
\ \ \isacommand{case}\isamarkupfalse%
\ {\isadigit{3}}\isanewline
\ \ \isacommand{fix}\isamarkupfalse%
\ n{\isacharprime}{\kern0pt}{\isacharcolon}{\kern0pt}{\isacharcolon}{\kern0pt}nat\isanewline
\ \ \isacommand{define}\isamarkupfalse%
\ n\ \isakeyword{where}\ {\isachardoublequoteopen}n\ {\isacharequal}{\kern0pt}\ Suc\ n{\isacharprime}{\kern0pt}{\isachardoublequoteclose}\isanewline
\ \ \isacommand{assume}\isamarkupfalse%
\ HI{\isacharcolon}{\kern0pt}{\isachardoublequoteopen}{\isasymforall}j{\isacharless}{\kern0pt}{\isadigit{2}}\ {\isacharcircum}{\kern0pt}\ n{\isachardot}{\kern0pt}\ QFT\ n\ {\isacharasterisk}{\kern0pt}\ {\isacharbar}{\kern0pt}state{\isacharunderscore}{\kern0pt}basis\ n\ j{\isasymrangle}\ {\isacharequal}{\kern0pt}\ reverse{\isacharunderscore}{\kern0pt}QFT{\isacharunderscore}{\kern0pt}product{\isacharunderscore}{\kern0pt}representation\ j\ n{\isachardoublequoteclose}\isanewline
\ \ \isacommand{show}\isamarkupfalse%
\ {\isachardoublequoteopen}{\isasymforall}j{\isacharless}{\kern0pt}{\isadigit{2}}{\isacharcircum}{\kern0pt}Suc\ n{\isachardot}{\kern0pt}\isanewline
\ \ \ \ \ \ \ \ \ \ \ QFT\ {\isacharparenleft}{\kern0pt}Suc\ n{\isacharparenright}{\kern0pt}\ {\isacharasterisk}{\kern0pt}\ {\isacharbar}{\kern0pt}state{\isacharunderscore}{\kern0pt}basis\ {\isacharparenleft}{\kern0pt}Suc\ n{\isacharparenright}{\kern0pt}\ j{\isasymrangle}\ {\isacharequal}{\kern0pt}\ reverse{\isacharunderscore}{\kern0pt}QFT{\isacharunderscore}{\kern0pt}product{\isacharunderscore}{\kern0pt}representation\ j\ {\isacharparenleft}{\kern0pt}Suc\ n{\isacharparenright}{\kern0pt}{\isachardoublequoteclose}\isanewline
\ \ \isacommand{proof}\isamarkupfalse%
\ {\isacharparenleft}{\kern0pt}rule\ allI{\isacharparenright}{\kern0pt}\isanewline
\ \ \ \ \isacommand{fix}\isamarkupfalse%
\ j{\isacharcolon}{\kern0pt}{\isacharcolon}{\kern0pt}nat\isanewline
\ \ \ \ \isacommand{show}\isamarkupfalse%
\ {\isachardoublequoteopen}j\ {\isacharless}{\kern0pt}\ {\isadigit{2}}\ {\isacharcircum}{\kern0pt}\ Suc\ n\ {\isasymlongrightarrow}\ QFT\ {\isacharparenleft}{\kern0pt}Suc\ n{\isacharparenright}{\kern0pt}\ {\isacharasterisk}{\kern0pt}\ {\isacharbar}{\kern0pt}state{\isacharunderscore}{\kern0pt}basis\ {\isacharparenleft}{\kern0pt}Suc\ n{\isacharparenright}{\kern0pt}\ j{\isasymrangle}\ {\isacharequal}{\kern0pt}\isanewline
\ \ \ \ \ \ \ \ \ \ \ \ \ \ \ \ \ \ \ \ \ \ \ \ \ \ \ \ reverse{\isacharunderscore}{\kern0pt}QFT{\isacharunderscore}{\kern0pt}product{\isacharunderscore}{\kern0pt}representation\ j\ {\isacharparenleft}{\kern0pt}Suc\ n{\isacharparenright}{\kern0pt}{\isachardoublequoteclose}\isanewline
\ \ \ \ \isacommand{proof}\isamarkupfalse%
\ \isanewline
\ \ \ \ \ \ \isacommand{assume}\isamarkupfalse%
\ aj{\isacharcolon}{\kern0pt}{\isachardoublequoteopen}j\ {\isacharless}{\kern0pt}\ {\isadigit{2}}\ {\isacharcircum}{\kern0pt}\ Suc\ n{\isachardoublequoteclose}\isanewline
\ \ \ \ \ \ \isacommand{show}\isamarkupfalse%
\ {\isachardoublequoteopen}QFT\ {\isacharparenleft}{\kern0pt}Suc\ n{\isacharparenright}{\kern0pt}\ {\isacharasterisk}{\kern0pt}\isanewline
\ \ \ \ \ \ \ \ \ {\isacharbar}{\kern0pt}state{\isacharunderscore}{\kern0pt}basis\ {\isacharparenleft}{\kern0pt}Suc\ n{\isacharparenright}{\kern0pt}\ j{\isasymrangle}\ {\isacharequal}{\kern0pt}\isanewline
\ \ \ \ \ \ \ \ \ reverse{\isacharunderscore}{\kern0pt}QFT{\isacharunderscore}{\kern0pt}product{\isacharunderscore}{\kern0pt}representation\ j\isanewline
\ \ \ \ \ \ \ \ \ \ {\isacharparenleft}{\kern0pt}Suc\ n{\isacharparenright}{\kern0pt}{\isachardoublequoteclose}\isanewline
\ \ \ \ \ \ \isacommand{proof}\isamarkupfalse%
\ {\isacharminus}{\kern0pt}\isanewline
\ \ \ \ \ \ \ \ \isacommand{define}\isamarkupfalse%
\ jd\ jm\ \isakeyword{where}\ {\isachardoublequoteopen}jd\ {\isacharequal}{\kern0pt}\ j\ div\ {\isadigit{2}}{\isacharcircum}{\kern0pt}n{\isachardoublequoteclose}\ \isakeyword{and}\ {\isachardoublequoteopen}jm\ {\isacharequal}{\kern0pt}\ j\ mod\ {\isadigit{2}}{\isacharcircum}{\kern0pt}n{\isachardoublequoteclose}\isanewline
\ \ \ \ \ \ \ \ \isacommand{hence}\isamarkupfalse%
\ {\isachardoublequoteopen}jm\ {\isacharless}{\kern0pt}\ {\isadigit{2}}{\isacharcircum}{\kern0pt}n{\isachardoublequoteclose}\ \isacommand{by}\isamarkupfalse%
\ auto\isanewline
\ \ \ \ \ \ \ \ \isacommand{hence}\isamarkupfalse%
\ HI{\isacharunderscore}{\kern0pt}jm{\isacharcolon}{\kern0pt}{\isachardoublequoteopen}QFT\ n\ {\isacharasterisk}{\kern0pt}\ {\isacharbar}{\kern0pt}state{\isacharunderscore}{\kern0pt}basis\ n\ jm{\isasymrangle}\ {\isacharequal}{\kern0pt}\ reverse{\isacharunderscore}{\kern0pt}QFT{\isacharunderscore}{\kern0pt}product{\isacharunderscore}{\kern0pt}representation\ jm\ n{\isachardoublequoteclose}\ \isanewline
\ \ \ \ \ \ \ \ \ \ \isacommand{using}\isamarkupfalse%
\ HI\ \isacommand{by}\isamarkupfalse%
\ auto\isanewline
\ \ \ \ \ \ \ \ \isacommand{have}\isamarkupfalse%
\ {\isachardoublequoteopen}{\isacharparenleft}{\kern0pt}QFT\ {\isacharparenleft}{\kern0pt}Suc\ n{\isacharparenright}{\kern0pt}{\isacharparenright}{\kern0pt}\ {\isacharasterisk}{\kern0pt}\ {\isacharbar}{\kern0pt}state{\isacharunderscore}{\kern0pt}basis\ {\isacharparenleft}{\kern0pt}Suc\ n{\isacharparenright}{\kern0pt}\ j{\isasymrangle}\ {\isacharequal}{\kern0pt}\ \isanewline
\ \ \ \ \ \ \ \ {\isacharparenleft}{\kern0pt}{\isacharparenleft}{\kern0pt}{\isacharparenleft}{\kern0pt}{\isadigit{1}}\isactrlsub m\ {\isadigit{2}}{\isacharparenright}{\kern0pt}\ {\isasymOtimes}\ {\isacharparenleft}{\kern0pt}QFT\ n{\isacharparenright}{\kern0pt}{\isacharparenright}{\kern0pt}\ {\isacharasterisk}{\kern0pt}\ {\isacharparenleft}{\kern0pt}controlled{\isacharunderscore}{\kern0pt}rotations\ {\isacharparenleft}{\kern0pt}Suc\ n{\isacharparenright}{\kern0pt}{\isacharparenright}{\kern0pt}\ {\isacharasterisk}{\kern0pt}\ {\isacharparenleft}{\kern0pt}H\ {\isasymOtimes}\ {\isacharparenleft}{\kern0pt}{\isacharparenleft}{\kern0pt}{\isadigit{1}}\isactrlsub m\ {\isacharparenleft}{\kern0pt}{\isadigit{2}}{\isacharcircum}{\kern0pt}n{\isacharparenright}{\kern0pt}{\isacharparenright}{\kern0pt}{\isacharparenright}{\kern0pt}{\isacharparenright}{\kern0pt}{\isacharparenright}{\kern0pt}\ {\isacharasterisk}{\kern0pt}\ \isanewline
\ \ \ \ \ \ \ \ {\isacharbar}{\kern0pt}state{\isacharunderscore}{\kern0pt}basis\ {\isacharparenleft}{\kern0pt}Suc\ n{\isacharparenright}{\kern0pt}\ j{\isasymrangle}{\isachardoublequoteclose}\isanewline
\ \ \ \ \ \ \ \ \ \ \isacommand{using}\isamarkupfalse%
\ QFT{\isachardot}{\kern0pt}simps{\isacharparenleft}{\kern0pt}{\isadigit{3}}{\isacharparenright}{\kern0pt}\ n{\isacharunderscore}{\kern0pt}def\ \isacommand{by}\isamarkupfalse%
\ simp\isanewline
\ \ \ \ \ \ \ \ \isacommand{also}\isamarkupfalse%
\ \isacommand{have}\isamarkupfalse%
\ {\isachardoublequoteopen}{\isasymdots}\ {\isacharequal}{\kern0pt}\ {\isacharparenleft}{\kern0pt}{\isacharparenleft}{\kern0pt}{\isacharparenleft}{\kern0pt}{\isadigit{1}}\isactrlsub m\ {\isadigit{2}}{\isacharparenright}{\kern0pt}\ {\isasymOtimes}\ {\isacharparenleft}{\kern0pt}QFT\ n{\isacharparenright}{\kern0pt}{\isacharparenright}{\kern0pt}\ {\isacharasterisk}{\kern0pt}\ {\isacharparenleft}{\kern0pt}controlled{\isacharunderscore}{\kern0pt}rotations\ {\isacharparenleft}{\kern0pt}Suc\ n{\isacharparenright}{\kern0pt}{\isacharparenright}{\kern0pt}{\isacharparenright}{\kern0pt}\ {\isacharasterisk}{\kern0pt}\ \isanewline
\ \ \ \ \ \ \ \ \ \ \ \ \ \ \ \ \ \ \ \ \ \ \ \ {\isacharparenleft}{\kern0pt}{\isacharparenleft}{\kern0pt}{\isacharparenleft}{\kern0pt}H\ {\isasymOtimes}\ {\isacharparenleft}{\kern0pt}{\isacharparenleft}{\kern0pt}{\isadigit{1}}\isactrlsub m\ {\isacharparenleft}{\kern0pt}{\isadigit{2}}{\isacharcircum}{\kern0pt}n{\isacharparenright}{\kern0pt}{\isacharparenright}{\kern0pt}{\isacharparenright}{\kern0pt}{\isacharparenright}{\kern0pt}{\isacharparenright}{\kern0pt}\ {\isacharasterisk}{\kern0pt}\ {\isacharbar}{\kern0pt}state{\isacharunderscore}{\kern0pt}basis\ {\isacharparenleft}{\kern0pt}Suc\ n{\isacharparenright}{\kern0pt}\ j{\isasymrangle}{\isacharparenright}{\kern0pt}{\isachardoublequoteclose}\isanewline
\ \ \ \ \ \ \ \ \isacommand{proof}\isamarkupfalse%
\ {\isacharparenleft}{\kern0pt}rule\ assoc{\isacharunderscore}{\kern0pt}mult{\isacharunderscore}{\kern0pt}mat{\isacharparenright}{\kern0pt}\isanewline
\ \ \ \ \ \ \ \ \ \ \isacommand{show}\isamarkupfalse%
\ {\isachardoublequoteopen}{\isacharparenleft}{\kern0pt}{\isadigit{1}}\isactrlsub m\ {\isadigit{2}}\ {\isasymOtimes}\ QFT\ n{\isacharparenright}{\kern0pt}\ {\isacharasterisk}{\kern0pt}\ controlled{\isacharunderscore}{\kern0pt}rotations\ {\isacharparenleft}{\kern0pt}Suc\ n{\isacharparenright}{\kern0pt}\ {\isasymin}\ carrier{\isacharunderscore}{\kern0pt}mat\ {\isacharparenleft}{\kern0pt}{\isadigit{2}}{\isacharcircum}{\kern0pt}{\isacharparenleft}{\kern0pt}Suc\ n{\isacharparenright}{\kern0pt}{\isacharparenright}{\kern0pt}\ {\isacharparenleft}{\kern0pt}{\isadigit{2}}{\isacharcircum}{\kern0pt}{\isacharparenleft}{\kern0pt}Suc\ n{\isacharparenright}{\kern0pt}{\isacharparenright}{\kern0pt}{\isachardoublequoteclose}\isanewline
\ \ \ \ \ \ \ \ \ \ \isacommand{proof}\isamarkupfalse%
\ {\isacharparenleft}{\kern0pt}rule\ mult{\isacharunderscore}{\kern0pt}carrier{\isacharunderscore}{\kern0pt}mat{\isacharparenright}{\kern0pt}\isanewline
\ \ \ \ \ \ \ \ \ \ \ \ \isacommand{show}\isamarkupfalse%
\ {\isachardoublequoteopen}{\isadigit{1}}\isactrlsub m\ {\isadigit{2}}\ {\isasymOtimes}\ QFT\ n\ {\isasymin}\ carrier{\isacharunderscore}{\kern0pt}mat\ {\isacharparenleft}{\kern0pt}{\isadigit{2}}\ {\isacharcircum}{\kern0pt}\ Suc\ n{\isacharparenright}{\kern0pt}\ {\isacharparenleft}{\kern0pt}{\isadigit{2}}\ {\isacharcircum}{\kern0pt}\ Suc\ n{\isacharparenright}{\kern0pt}{\isachardoublequoteclose}\ \isacommand{by}\isamarkupfalse%
\ simp\isanewline
\ \ \ \ \ \ \ \ \ \ \ \ \isacommand{show}\isamarkupfalse%
\ {\isachardoublequoteopen}controlled{\isacharunderscore}{\kern0pt}rotations\ {\isacharparenleft}{\kern0pt}Suc\ n{\isacharparenright}{\kern0pt}\ {\isasymin}\ carrier{\isacharunderscore}{\kern0pt}mat\ {\isacharparenleft}{\kern0pt}{\isadigit{2}}\ {\isacharcircum}{\kern0pt}\ Suc\ n{\isacharparenright}{\kern0pt}\ {\isacharparenleft}{\kern0pt}{\isadigit{2}}\ {\isacharcircum}{\kern0pt}\ Suc\ n{\isacharparenright}{\kern0pt}{\isachardoublequoteclose}\isanewline
\ \ \ \ \ \ \ \ \ \ \ \ \ \ \isacommand{using}\isamarkupfalse%
\ controlled{\isacharunderscore}{\kern0pt}rotations{\isacharunderscore}{\kern0pt}carrier{\isacharunderscore}{\kern0pt}mat\ \isacommand{by}\isamarkupfalse%
\ blast\isanewline
\ \ \ \ \ \ \ \ \ \ \isacommand{qed}\isamarkupfalse%
\isanewline
\ \ \ \ \ \ \ \ \isacommand{next}\isamarkupfalse%
\isanewline
\ \ \ \ \ \ \ \ \ \ \isacommand{show}\isamarkupfalse%
\ {\isachardoublequoteopen}H\ {\isasymOtimes}\ {\isadigit{1}}\isactrlsub m\ {\isacharparenleft}{\kern0pt}{\isadigit{2}}\ {\isacharcircum}{\kern0pt}\ n{\isacharparenright}{\kern0pt}\ {\isasymin}\ carrier{\isacharunderscore}{\kern0pt}mat\ {\isacharparenleft}{\kern0pt}{\isadigit{2}}\ {\isacharcircum}{\kern0pt}\ Suc\ n{\isacharparenright}{\kern0pt}\ {\isacharparenleft}{\kern0pt}{\isadigit{2}}\ {\isacharcircum}{\kern0pt}\ Suc\ n{\isacharparenright}{\kern0pt}{\isachardoublequoteclose}\isanewline
\ \ \ \ \ \ \ \ \ \ \ \ \isacommand{using}\isamarkupfalse%
\ tensor{\isacharunderscore}{\kern0pt}carrier{\isacharunderscore}{\kern0pt}mat\ \isanewline
\ \ \ \ \ \ \ \ \ \ \ \ \isacommand{by}\isamarkupfalse%
\ {\isacharparenleft}{\kern0pt}metis\ QFT{\isachardot}{\kern0pt}simps{\isacharparenleft}{\kern0pt}{\isadigit{2}}{\isacharparenright}{\kern0pt}\ QFT{\isacharunderscore}{\kern0pt}carrier{\isacharunderscore}{\kern0pt}mat\ one{\isacharunderscore}{\kern0pt}carrier{\isacharunderscore}{\kern0pt}mat\ power{\isacharunderscore}{\kern0pt}Suc\ power{\isacharunderscore}{\kern0pt}Suc{\isadigit{0}}{\isacharunderscore}{\kern0pt}right{\isacharparenright}{\kern0pt}\isanewline
\ \ \ \ \ \ \ \ \isacommand{next}\isamarkupfalse%
\isanewline
\ \ \ \ \ \ \ \ \ \ \isacommand{show}\isamarkupfalse%
\ {\isachardoublequoteopen}{\isacharbar}{\kern0pt}state{\isacharunderscore}{\kern0pt}basis\ {\isacharparenleft}{\kern0pt}Suc\ n{\isacharparenright}{\kern0pt}\ j{\isasymrangle}\ {\isasymin}\ carrier{\isacharunderscore}{\kern0pt}mat\ {\isacharparenleft}{\kern0pt}{\isadigit{2}}\ {\isacharcircum}{\kern0pt}\ Suc\ n{\isacharparenright}{\kern0pt}\ {\isadigit{1}}{\isachardoublequoteclose}\isanewline
\ \ \ \ \ \ \ \ \ \ \ \ \isacommand{using}\isamarkupfalse%
\ state{\isacharunderscore}{\kern0pt}basis{\isacharunderscore}{\kern0pt}carrier{\isacharunderscore}{\kern0pt}mat\ \isacommand{by}\isamarkupfalse%
\ blast\isanewline
\ \ \ \ \ \ \ \ \isacommand{qed}\isamarkupfalse%
\isanewline
\ \ \ \ \ \ \ \ \isacommand{also}\isamarkupfalse%
\ \isacommand{have}\isamarkupfalse%
\ {\isachardoublequoteopen}{\isasymdots}\ {\isacharequal}{\kern0pt}\ {\isacharparenleft}{\kern0pt}{\isacharparenleft}{\kern0pt}{\isacharparenleft}{\kern0pt}{\isadigit{1}}\isactrlsub m\ {\isadigit{2}}{\isacharparenright}{\kern0pt}\ {\isasymOtimes}\ {\isacharparenleft}{\kern0pt}QFT\ n{\isacharparenright}{\kern0pt}{\isacharparenright}{\kern0pt}\ {\isacharasterisk}{\kern0pt}\ {\isacharparenleft}{\kern0pt}controlled{\isacharunderscore}{\kern0pt}rotations\ {\isacharparenleft}{\kern0pt}Suc\ n{\isacharparenright}{\kern0pt}{\isacharparenright}{\kern0pt}{\isacharparenright}{\kern0pt}\ {\isacharasterisk}{\kern0pt}\isanewline
\ \ \ \ \ \ \ \ \ \ \ \ \ \ \ \ \ \ \ \ \ \ \ \ {\isacharparenleft}{\kern0pt}{\isacharparenleft}{\kern0pt}{\isadigit{1}}{\isacharslash}{\kern0pt}sqrt\ {\isadigit{2}}\ {\isasymcdot}\isactrlsub m\ {\isacharparenleft}{\kern0pt}\ {\isacharbar}{\kern0pt}zero{\isasymrangle}\ {\isacharplus}{\kern0pt}\ exp{\isacharparenleft}{\kern0pt}{\isadigit{2}}{\isacharasterisk}{\kern0pt}{\isasymi}{\isacharasterisk}{\kern0pt}pi{\isacharasterisk}{\kern0pt}jd{\isacharslash}{\kern0pt}{\isadigit{2}}{\isacharparenright}{\kern0pt}\ {\isasymcdot}\isactrlsub m\ {\isacharbar}{\kern0pt}one{\isasymrangle}{\isacharparenright}{\kern0pt}{\isacharparenright}{\kern0pt}\ {\isasymOtimes}\ {\isacharbar}{\kern0pt}state{\isacharunderscore}{\kern0pt}basis\ n\ jm{\isasymrangle}{\isacharparenright}{\kern0pt}{\isachardoublequoteclose}\isanewline
\ \ \ \ \ \ \ \ \ \ \isacommand{using}\isamarkupfalse%
\ aj\ H{\isacharunderscore}{\kern0pt}on{\isacharunderscore}{\kern0pt}first{\isacharunderscore}{\kern0pt}qubit\ jd{\isacharunderscore}{\kern0pt}def\ jm{\isacharunderscore}{\kern0pt}def\ \isacommand{by}\isamarkupfalse%
\ simp\isanewline
\ \ \ \ \ \ \ \ \isacommand{also}\isamarkupfalse%
\ \isacommand{have}\isamarkupfalse%
\ {\isachardoublequoteopen}{\isasymdots}\ {\isacharequal}{\kern0pt}\ {\isacharparenleft}{\kern0pt}{\isacharparenleft}{\kern0pt}{\isadigit{1}}\isactrlsub m\ {\isadigit{2}}{\isacharparenright}{\kern0pt}\ {\isasymOtimes}\ {\isacharparenleft}{\kern0pt}QFT\ n{\isacharparenright}{\kern0pt}{\isacharparenright}{\kern0pt}\ {\isacharasterisk}{\kern0pt}\ {\isacharparenleft}{\kern0pt}controlled{\isacharunderscore}{\kern0pt}rotations\ {\isacharparenleft}{\kern0pt}Suc\ n{\isacharparenright}{\kern0pt}\ {\isacharasterisk}{\kern0pt}\isanewline
\ \ \ \ \ \ \ \ \ \ \ \ \ \ \ \ \ \ \ \ \ \ \ \ {\isacharparenleft}{\kern0pt}{\isacharparenleft}{\kern0pt}{\isacharparenleft}{\kern0pt}{\isadigit{1}}{\isacharslash}{\kern0pt}sqrt\ {\isadigit{2}}\ {\isasymcdot}\isactrlsub m\ {\isacharparenleft}{\kern0pt}\ {\isacharbar}{\kern0pt}zero{\isasymrangle}\ {\isacharplus}{\kern0pt}\ exp{\isacharparenleft}{\kern0pt}{\isadigit{2}}{\isacharasterisk}{\kern0pt}{\isasymi}{\isacharasterisk}{\kern0pt}pi{\isacharasterisk}{\kern0pt}jd{\isacharslash}{\kern0pt}{\isadigit{2}}{\isacharparenright}{\kern0pt}\ {\isasymcdot}\isactrlsub m\ {\isacharbar}{\kern0pt}one{\isasymrangle}{\isacharparenright}{\kern0pt}{\isacharparenright}{\kern0pt}\ {\isasymOtimes}\ {\isacharbar}{\kern0pt}state{\isacharunderscore}{\kern0pt}basis\ n\ jm{\isasymrangle}{\isacharparenright}{\kern0pt}{\isacharparenright}{\kern0pt}{\isacharparenright}{\kern0pt}{\isachardoublequoteclose}\isanewline
\ \ \ \ \ \ \ \ \ \ \isacommand{using}\isamarkupfalse%
\ assoc{\isacharunderscore}{\kern0pt}mult{\isacharunderscore}{\kern0pt}mat\ tensor{\isacharunderscore}{\kern0pt}carrier{\isacharunderscore}{\kern0pt}mat\ QFT{\isacharunderscore}{\kern0pt}carrier{\isacharunderscore}{\kern0pt}mat\ one{\isacharunderscore}{\kern0pt}carrier{\isacharunderscore}{\kern0pt}mat\ \isanewline
\ \ \ \ \ \ \ \ \ \ \ \ state{\isacharunderscore}{\kern0pt}basis{\isacharunderscore}{\kern0pt}carrier{\isacharunderscore}{\kern0pt}mat\isanewline
\ \ \ \ \ \ \ \ \ \ \isacommand{by}\isamarkupfalse%
\ {\isacharparenleft}{\kern0pt}smt\ {\isacharparenleft}{\kern0pt}verit{\isacharcomma}{\kern0pt}\ ccfv{\isacharunderscore}{\kern0pt}threshold{\isacharparenright}{\kern0pt}\ H{\isacharunderscore}{\kern0pt}on{\isacharunderscore}{\kern0pt}first{\isacharunderscore}{\kern0pt}qubit\ QFT{\isachardot}{\kern0pt}simps{\isacharparenleft}{\kern0pt}{\isadigit{2}}{\isacharparenright}{\kern0pt}\ aj\ \isanewline
\ \ \ \ \ \ \ \ \ \ \ \ \ \ controlled{\isacharunderscore}{\kern0pt}rotations{\isacharunderscore}{\kern0pt}carrier{\isacharunderscore}{\kern0pt}mat\ jd{\isacharunderscore}{\kern0pt}def\ jm{\isacharunderscore}{\kern0pt}def\ mult{\isacharunderscore}{\kern0pt}carrier{\isacharunderscore}{\kern0pt}mat\ power{\isacharunderscore}{\kern0pt}Suc\ \isanewline
\ \ \ \ \ \ \ \ \ \ \ \ \ \ power{\isacharunderscore}{\kern0pt}Suc{\isadigit{0}}{\isacharunderscore}{\kern0pt}right{\isacharparenright}{\kern0pt}\isanewline
\ \ \ \ \ \ \ \ \isacommand{also}\isamarkupfalse%
\ \isacommand{have}\isamarkupfalse%
\ {\isachardoublequoteopen}{\isasymdots}\ {\isacharequal}{\kern0pt}\ {\isacharparenleft}{\kern0pt}{\isacharparenleft}{\kern0pt}{\isadigit{1}}\isactrlsub m\ {\isadigit{2}}{\isacharparenright}{\kern0pt}\ {\isasymOtimes}\ {\isacharparenleft}{\kern0pt}QFT\ n{\isacharparenright}{\kern0pt}{\isacharparenright}{\kern0pt}\ {\isacharasterisk}{\kern0pt}\ \isanewline
\ \ \ \ \ \ \ \ \ \ \ \ \ \ \ \ \ \ \ \ \ \ \ \ {\isacharparenleft}{\kern0pt}{\isadigit{1}}{\isacharslash}{\kern0pt}sqrt\ {\isadigit{2}}\ {\isasymcdot}\isactrlsub m\ {\isacharparenleft}{\kern0pt}{\isacharparenleft}{\kern0pt}\ {\isacharbar}{\kern0pt}zero{\isasymrangle}\ {\isacharplus}{\kern0pt}\ exp{\isacharparenleft}{\kern0pt}{\isadigit{2}}{\isacharasterisk}{\kern0pt}{\isasymi}{\isacharasterisk}{\kern0pt}pi{\isacharasterisk}{\kern0pt}j{\isacharslash}{\kern0pt}{\isacharparenleft}{\kern0pt}{\isadigit{2}}{\isacharcircum}{\kern0pt}{\isacharparenleft}{\kern0pt}Suc\ n{\isacharparenright}{\kern0pt}{\isacharparenright}{\kern0pt}{\isacharparenright}{\kern0pt}\ {\isasymcdot}\isactrlsub m\ {\isacharbar}{\kern0pt}one{\isasymrangle}{\isacharparenright}{\kern0pt}{\isacharparenright}{\kern0pt}\ {\isasymOtimes}\ \isanewline
\ \ \ \ \ \ \ \ \ \ \ \ \ \ \ \ \ \ \ \ \ \ \ \ {\isacharbar}{\kern0pt}state{\isacharunderscore}{\kern0pt}basis\ n\ jm{\isasymrangle}{\isacharparenright}{\kern0pt}{\isachardoublequoteclose}\isanewline
\ \ \ \ \ \ \ \ \ \ \isacommand{using}\isamarkupfalse%
\ controlled{\isacharunderscore}{\kern0pt}rotations{\isacharunderscore}{\kern0pt}on{\isacharunderscore}{\kern0pt}first{\isacharunderscore}{\kern0pt}qubit\ aj\ jd{\isacharunderscore}{\kern0pt}def\ jm{\isacharunderscore}{\kern0pt}def\ \isacommand{by}\isamarkupfalse%
\ simp\isanewline
\ \ \ \ \ \ \ \ \isacommand{also}\isamarkupfalse%
\ \isacommand{have}\isamarkupfalse%
\ {\isachardoublequoteopen}{\isasymdots}\ {\isacharequal}{\kern0pt}\ {\isacharparenleft}{\kern0pt}{\isacharparenleft}{\kern0pt}{\isadigit{1}}\isactrlsub m\ {\isadigit{2}}{\isacharparenright}{\kern0pt}\ {\isacharasterisk}{\kern0pt}\ {\isacharparenleft}{\kern0pt}{\isadigit{1}}{\isacharslash}{\kern0pt}sqrt\ {\isadigit{2}}\ {\isasymcdot}\isactrlsub m\ {\isacharparenleft}{\kern0pt}{\isacharparenleft}{\kern0pt}\ {\isacharbar}{\kern0pt}zero{\isasymrangle}\ {\isacharplus}{\kern0pt}\ exp{\isacharparenleft}{\kern0pt}{\isadigit{2}}{\isacharasterisk}{\kern0pt}{\isasymi}{\isacharasterisk}{\kern0pt}pi{\isacharasterisk}{\kern0pt}j{\isacharslash}{\kern0pt}{\isacharparenleft}{\kern0pt}{\isadigit{2}}{\isacharcircum}{\kern0pt}{\isacharparenleft}{\kern0pt}Suc\ n{\isacharparenright}{\kern0pt}{\isacharparenright}{\kern0pt}{\isacharparenright}{\kern0pt}\ {\isasymcdot}\isactrlsub m\ {\isacharbar}{\kern0pt}one{\isasymrangle}{\isacharparenright}{\kern0pt}{\isacharparenright}{\kern0pt}{\isacharparenright}{\kern0pt}{\isacharparenright}{\kern0pt}\ {\isasymOtimes}\isanewline
\ \ \ \ \ \ \ \ \ \ \ \ \ \ \ \ \ \ \ \ \ \ \ \ {\isacharparenleft}{\kern0pt}{\isacharparenleft}{\kern0pt}QFT\ n{\isacharparenright}{\kern0pt}\ {\isacharasterisk}{\kern0pt}\ {\isacharbar}{\kern0pt}state{\isacharunderscore}{\kern0pt}basis\ n\ jm{\isasymrangle}{\isacharparenright}{\kern0pt}{\isachardoublequoteclose}\isanewline
\ \ \ \ \ \ \ \ \isacommand{proof}\isamarkupfalse%
\ {\isacharminus}{\kern0pt}\isanewline
\ \ \ \ \ \ \ \ \ \ \isacommand{have}\isamarkupfalse%
\ {\isachardoublequoteopen}dim{\isacharunderscore}{\kern0pt}col\ {\isacharparenleft}{\kern0pt}{\isadigit{1}}\isactrlsub m\ {\isadigit{2}}{\isacharparenright}{\kern0pt}\ {\isacharequal}{\kern0pt}\ dim{\isacharunderscore}{\kern0pt}row\ {\isacharparenleft}{\kern0pt}{\isadigit{1}}{\isacharslash}{\kern0pt}sqrt\ {\isadigit{2}}\ {\isasymcdot}\isactrlsub m\ {\isacharparenleft}{\kern0pt}{\isacharparenleft}{\kern0pt}\ {\isacharbar}{\kern0pt}zero{\isasymrangle}\ {\isacharplus}{\kern0pt}\ exp{\isacharparenleft}{\kern0pt}{\isadigit{2}}{\isacharasterisk}{\kern0pt}{\isasymi}{\isacharasterisk}{\kern0pt}pi{\isacharasterisk}{\kern0pt}j{\isacharslash}{\kern0pt}{\isacharparenleft}{\kern0pt}{\isadigit{2}}{\isacharcircum}{\kern0pt}{\isacharparenleft}{\kern0pt}Suc\ n{\isacharparenright}{\kern0pt}{\isacharparenright}{\kern0pt}{\isacharparenright}{\kern0pt}\ {\isasymcdot}\isactrlsub m\ {\isacharbar}{\kern0pt}one{\isasymrangle}{\isacharparenright}{\kern0pt}{\isacharparenright}{\kern0pt}{\isacharparenright}{\kern0pt}{\isachardoublequoteclose}\isanewline
\ \ \ \ \ \ \ \ \ \ \isacommand{proof}\isamarkupfalse%
\ {\isacharminus}{\kern0pt}\isanewline
\ \ \ \ \ \ \ \ \ \ \ \ \isacommand{have}\isamarkupfalse%
\ {\isachardoublequoteopen}dim{\isacharunderscore}{\kern0pt}col\ {\isacharparenleft}{\kern0pt}{\isadigit{1}}\isactrlsub m\ {\isadigit{2}}{\isacharparenright}{\kern0pt}\ {\isacharequal}{\kern0pt}\ {\isadigit{2}}{\isachardoublequoteclose}\ \isacommand{by}\isamarkupfalse%
\ simp\isanewline
\ \ \ \ \ \ \ \ \ \ \ \ \isacommand{moreover}\isamarkupfalse%
\ \isacommand{have}\isamarkupfalse%
\ {\isachardoublequoteopen}dim{\isacharunderscore}{\kern0pt}row\ {\isacharparenleft}{\kern0pt}{\isadigit{1}}{\isacharslash}{\kern0pt}sqrt\ {\isadigit{2}}\ {\isasymcdot}\isactrlsub m\ {\isacharparenleft}{\kern0pt}{\isacharparenleft}{\kern0pt}\ {\isacharbar}{\kern0pt}zero{\isasymrangle}\ {\isacharplus}{\kern0pt}\ exp{\isacharparenleft}{\kern0pt}{\isadigit{2}}{\isacharasterisk}{\kern0pt}{\isasymi}{\isacharasterisk}{\kern0pt}pi{\isacharasterisk}{\kern0pt}j{\isacharslash}{\kern0pt}{\isacharparenleft}{\kern0pt}{\isadigit{2}}{\isacharcircum}{\kern0pt}{\isacharparenleft}{\kern0pt}Suc\ n{\isacharparenright}{\kern0pt}{\isacharparenright}{\kern0pt}{\isacharparenright}{\kern0pt}\ {\isasymcdot}\isactrlsub m\ {\isacharbar}{\kern0pt}one{\isasymrangle}{\isacharparenright}{\kern0pt}{\isacharparenright}{\kern0pt}{\isacharparenright}{\kern0pt}\ {\isacharequal}{\kern0pt}\ {\isadigit{2}}{\isachardoublequoteclose}\isanewline
\ \ \ \ \ \ \ \ \ \ \ \ \ \ \isacommand{using}\isamarkupfalse%
\ smult{\isacharunderscore}{\kern0pt}carrier{\isacharunderscore}{\kern0pt}mat\ mat{\isacharunderscore}{\kern0pt}of{\isacharunderscore}{\kern0pt}cols{\isacharunderscore}{\kern0pt}list{\isacharunderscore}{\kern0pt}def\ add{\isacharunderscore}{\kern0pt}carrier{\isacharunderscore}{\kern0pt}mat\ ket{\isacharunderscore}{\kern0pt}vec{\isacharunderscore}{\kern0pt}def\ \isacommand{by}\isamarkupfalse%
\ simp\isanewline
\ \ \ \ \ \ \ \ \ \ \ \ \isacommand{ultimately}\isamarkupfalse%
\ \isacommand{show}\isamarkupfalse%
\ {\isacharquery}{\kern0pt}thesis\ \isacommand{by}\isamarkupfalse%
\ simp\isanewline
\ \ \ \ \ \ \ \ \ \ \isacommand{qed}\isamarkupfalse%
\isanewline
\ \ \ \ \ \ \ \ \ \ \isacommand{moreover}\isamarkupfalse%
\ \isacommand{have}\isamarkupfalse%
\ {\isachardoublequoteopen}dim{\isacharunderscore}{\kern0pt}col\ {\isacharparenleft}{\kern0pt}QFT\ n{\isacharparenright}{\kern0pt}\ {\isacharequal}{\kern0pt}\ dim{\isacharunderscore}{\kern0pt}row\ {\isacharbar}{\kern0pt}state{\isacharunderscore}{\kern0pt}basis\ n\ jm{\isasymrangle}{\isachardoublequoteclose}\isanewline
\ \ \ \ \ \ \ \ \ \ \ \ \isacommand{using}\isamarkupfalse%
\ state{\isacharunderscore}{\kern0pt}basis{\isacharunderscore}{\kern0pt}carrier{\isacharunderscore}{\kern0pt}mat\ QFT{\isacharunderscore}{\kern0pt}carrier{\isacharunderscore}{\kern0pt}mat\isanewline
\ \ \ \ \ \ \ \ \ \ \ \ \isacommand{by}\isamarkupfalse%
\ {\isacharparenleft}{\kern0pt}metis\ carrier{\isacharunderscore}{\kern0pt}matD{\isacharparenleft}{\kern0pt}{\isadigit{1}}{\isacharparenright}{\kern0pt}\ carrier{\isacharunderscore}{\kern0pt}matD{\isacharparenleft}{\kern0pt}{\isadigit{2}}{\isacharparenright}{\kern0pt}{\isacharparenright}{\kern0pt}\isanewline
\ \ \ \ \ \ \ \ \ \ \isacommand{moreover}\isamarkupfalse%
\ \isacommand{have}\isamarkupfalse%
\ {\isachardoublequoteopen}dim{\isacharunderscore}{\kern0pt}col\ {\isacharparenleft}{\kern0pt}{\isadigit{1}}\isactrlsub m\ {\isadigit{2}}{\isacharparenright}{\kern0pt}\ {\isachargreater}{\kern0pt}\ {\isadigit{0}}{\isachardoublequoteclose}\ \isacommand{by}\isamarkupfalse%
\ simp\isanewline
\ \ \ \ \ \ \ \ \ \ \isacommand{moreover}\isamarkupfalse%
\ \isacommand{have}\isamarkupfalse%
\ {\isachardoublequoteopen}dim{\isacharunderscore}{\kern0pt}col\ {\isacharparenleft}{\kern0pt}{\isadigit{1}}{\isacharslash}{\kern0pt}sqrt\ {\isadigit{2}}\ {\isasymcdot}\isactrlsub m\ {\isacharparenleft}{\kern0pt}{\isacharparenleft}{\kern0pt}\ {\isacharbar}{\kern0pt}zero{\isasymrangle}\ {\isacharplus}{\kern0pt}\ exp{\isacharparenleft}{\kern0pt}{\isadigit{2}}{\isacharasterisk}{\kern0pt}{\isasymi}{\isacharasterisk}{\kern0pt}pi{\isacharasterisk}{\kern0pt}j{\isacharslash}{\kern0pt}{\isacharparenleft}{\kern0pt}{\isadigit{2}}{\isacharcircum}{\kern0pt}{\isacharparenleft}{\kern0pt}Suc\ n{\isacharparenright}{\kern0pt}{\isacharparenright}{\kern0pt}{\isacharparenright}{\kern0pt}\ {\isasymcdot}\isactrlsub m\ {\isacharbar}{\kern0pt}one{\isasymrangle}{\isacharparenright}{\kern0pt}{\isacharparenright}{\kern0pt}{\isacharparenright}{\kern0pt}\ {\isachargreater}{\kern0pt}\ {\isadigit{0}}{\isachardoublequoteclose}\isanewline
\ \ \ \ \ \ \ \ \ \ \ \ \isacommand{using}\isamarkupfalse%
\ smult{\isacharunderscore}{\kern0pt}carrier{\isacharunderscore}{\kern0pt}mat\ mat{\isacharunderscore}{\kern0pt}of{\isacharunderscore}{\kern0pt}cols{\isacharunderscore}{\kern0pt}list{\isacharunderscore}{\kern0pt}def\ add{\isacharunderscore}{\kern0pt}carrier{\isacharunderscore}{\kern0pt}mat\ ket{\isacharunderscore}{\kern0pt}vec{\isacharunderscore}{\kern0pt}def\ \isacommand{by}\isamarkupfalse%
\ simp\isanewline
\ \ \ \ \ \ \ \ \ \ \isacommand{moreover}\isamarkupfalse%
\ \isacommand{have}\isamarkupfalse%
\ {\isachardoublequoteopen}dim{\isacharunderscore}{\kern0pt}col\ {\isacharparenleft}{\kern0pt}QFT\ n{\isacharparenright}{\kern0pt}\ {\isachargreater}{\kern0pt}\ {\isadigit{0}}{\isachardoublequoteclose}\ \isacommand{using}\isamarkupfalse%
\ QFT{\isacharunderscore}{\kern0pt}carrier{\isacharunderscore}{\kern0pt}mat\isanewline
\ \ \ \ \ \ \ \ \ \ \ \ \isacommand{by}\isamarkupfalse%
\ {\isacharparenleft}{\kern0pt}metis\ carrier{\isacharunderscore}{\kern0pt}matD{\isacharparenleft}{\kern0pt}{\isadigit{2}}{\isacharparenright}{\kern0pt}\ pos{\isadigit{2}}\ zero{\isacharunderscore}{\kern0pt}less{\isacharunderscore}{\kern0pt}power{\isacharparenright}{\kern0pt}\isanewline
\ \ \ \ \ \ \ \ \ \ \isacommand{moreover}\isamarkupfalse%
\ \isacommand{have}\isamarkupfalse%
\ {\isachardoublequoteopen}dim{\isacharunderscore}{\kern0pt}col\ {\isacharbar}{\kern0pt}state{\isacharunderscore}{\kern0pt}basis\ n\ jm{\isasymrangle}\ {\isachargreater}{\kern0pt}\ {\isadigit{0}}{\isachardoublequoteclose}\ \isacommand{using}\isamarkupfalse%
\ state{\isacharunderscore}{\kern0pt}basis{\isacharunderscore}{\kern0pt}carrier{\isacharunderscore}{\kern0pt}mat\isanewline
\ \ \ \ \ \ \ \ \ \ \ \ \isacommand{by}\isamarkupfalse%
\ {\isacharparenleft}{\kern0pt}simp\ add{\isacharcolon}{\kern0pt}\ ket{\isacharunderscore}{\kern0pt}vec{\isacharunderscore}{\kern0pt}def{\isacharparenright}{\kern0pt}\isanewline
\ \ \ \ \ \ \ \ \ \ \isacommand{ultimately}\isamarkupfalse%
\ \isacommand{show}\isamarkupfalse%
\ {\isachardoublequoteopen}{\isacharparenleft}{\kern0pt}{\isacharparenleft}{\kern0pt}{\isadigit{1}}\isactrlsub m\ {\isadigit{2}}{\isacharparenright}{\kern0pt}\ {\isasymOtimes}\ {\isacharparenleft}{\kern0pt}QFT\ n{\isacharparenright}{\kern0pt}{\isacharparenright}{\kern0pt}\ {\isacharasterisk}{\kern0pt}\ \isanewline
\ \ \ \ \ \ \ \ \ \ \ \ \ \ \ \ {\isacharparenleft}{\kern0pt}{\isadigit{1}}{\isacharslash}{\kern0pt}sqrt\ {\isadigit{2}}\ {\isasymcdot}\isactrlsub m\ {\isacharparenleft}{\kern0pt}{\isacharparenleft}{\kern0pt}\ {\isacharbar}{\kern0pt}zero{\isasymrangle}\ {\isacharplus}{\kern0pt}\ exp{\isacharparenleft}{\kern0pt}{\isadigit{2}}{\isacharasterisk}{\kern0pt}{\isasymi}{\isacharasterisk}{\kern0pt}pi{\isacharasterisk}{\kern0pt}j{\isacharslash}{\kern0pt}{\isacharparenleft}{\kern0pt}{\isadigit{2}}{\isacharcircum}{\kern0pt}{\isacharparenleft}{\kern0pt}Suc\ n{\isacharparenright}{\kern0pt}{\isacharparenright}{\kern0pt}{\isacharparenright}{\kern0pt}\ {\isasymcdot}\isactrlsub m\ {\isacharbar}{\kern0pt}one{\isasymrangle}{\isacharparenright}{\kern0pt}{\isacharparenright}{\kern0pt}\ {\isasymOtimes}\ {\isacharbar}{\kern0pt}state{\isacharunderscore}{\kern0pt}basis\ n\ jm{\isasymrangle}{\isacharparenright}{\kern0pt}\ {\isacharequal}{\kern0pt}\isanewline
\ \ \ \ \ \ \ \ \ \ \ \ \ \ \ \ {\isacharparenleft}{\kern0pt}{\isacharparenleft}{\kern0pt}{\isadigit{1}}\isactrlsub m\ {\isadigit{2}}{\isacharparenright}{\kern0pt}\ {\isacharasterisk}{\kern0pt}\ {\isacharparenleft}{\kern0pt}{\isadigit{1}}{\isacharslash}{\kern0pt}sqrt\ {\isadigit{2}}\ {\isasymcdot}\isactrlsub m\ {\isacharparenleft}{\kern0pt}{\isacharparenleft}{\kern0pt}\ {\isacharbar}{\kern0pt}zero{\isasymrangle}\ {\isacharplus}{\kern0pt}\ exp{\isacharparenleft}{\kern0pt}{\isadigit{2}}{\isacharasterisk}{\kern0pt}{\isasymi}{\isacharasterisk}{\kern0pt}pi{\isacharasterisk}{\kern0pt}j{\isacharslash}{\kern0pt}{\isacharparenleft}{\kern0pt}{\isadigit{2}}{\isacharcircum}{\kern0pt}{\isacharparenleft}{\kern0pt}Suc\ n{\isacharparenright}{\kern0pt}{\isacharparenright}{\kern0pt}{\isacharparenright}{\kern0pt}\ {\isasymcdot}\isactrlsub m\ {\isacharbar}{\kern0pt}one{\isasymrangle}{\isacharparenright}{\kern0pt}{\isacharparenright}{\kern0pt}{\isacharparenright}{\kern0pt}{\isacharparenright}{\kern0pt}\ {\isasymOtimes}\isanewline
\ \ \ \ \ \ \ \ \ \ \ \ \ \ \ \ {\isacharparenleft}{\kern0pt}{\isacharparenleft}{\kern0pt}QFT\ n{\isacharparenright}{\kern0pt}\ {\isacharasterisk}{\kern0pt}\ {\isacharbar}{\kern0pt}state{\isacharunderscore}{\kern0pt}basis\ n\ jm{\isasymrangle}{\isacharparenright}{\kern0pt}{\isachardoublequoteclose}\ \isanewline
\ \ \ \ \ \ \ \ \ \ \ \ \isacommand{using}\isamarkupfalse%
\ mult{\isacharunderscore}{\kern0pt}distr{\isacharunderscore}{\kern0pt}tensor\ \isacommand{by}\isamarkupfalse%
\ {\isacharparenleft}{\kern0pt}smt\ {\isacharparenleft}{\kern0pt}verit{\isacharcomma}{\kern0pt}\ best{\isacharparenright}{\kern0pt}{\isacharparenright}{\kern0pt}\isanewline
\ \ \ \ \ \ \ \ \isacommand{qed}\isamarkupfalse%
\isanewline
\ \ \ \ \ \ \ \ \isacommand{also}\isamarkupfalse%
\ \isacommand{have}\isamarkupfalse%
\ {\isachardoublequoteopen}{\isasymdots}\ {\isacharequal}{\kern0pt}\ {\isacharparenleft}{\kern0pt}{\isadigit{1}}{\isacharslash}{\kern0pt}sqrt\ {\isadigit{2}}\ {\isasymcdot}\isactrlsub m\ {\isacharparenleft}{\kern0pt}{\isacharparenleft}{\kern0pt}\ {\isacharbar}{\kern0pt}zero{\isasymrangle}\ {\isacharplus}{\kern0pt}\ exp{\isacharparenleft}{\kern0pt}{\isadigit{2}}{\isacharasterisk}{\kern0pt}{\isasymi}{\isacharasterisk}{\kern0pt}pi{\isacharasterisk}{\kern0pt}j{\isacharslash}{\kern0pt}{\isacharparenleft}{\kern0pt}{\isadigit{2}}{\isacharcircum}{\kern0pt}{\isacharparenleft}{\kern0pt}Suc\ n{\isacharparenright}{\kern0pt}{\isacharparenright}{\kern0pt}{\isacharparenright}{\kern0pt}\ {\isasymcdot}\isactrlsub m\ {\isacharbar}{\kern0pt}one{\isasymrangle}{\isacharparenright}{\kern0pt}{\isacharparenright}{\kern0pt}{\isacharparenright}{\kern0pt}\ {\isasymOtimes}\isanewline
\ \ \ \ \ \ \ \ \ \ \ \ \ \ \ \ \ \ \ \ \ \ \ \ reverse{\isacharunderscore}{\kern0pt}QFT{\isacharunderscore}{\kern0pt}product{\isacharunderscore}{\kern0pt}representation\ jm\ n{\isachardoublequoteclose}\isanewline
\ \ \ \ \ \ \ \ \ \ \isacommand{using}\isamarkupfalse%
\ ket{\isacharunderscore}{\kern0pt}one{\isacharunderscore}{\kern0pt}is{\isacharunderscore}{\kern0pt}state\ state{\isachardot}{\kern0pt}dim{\isacharunderscore}{\kern0pt}row\ HI{\isacharunderscore}{\kern0pt}jm\ \isacommand{by}\isamarkupfalse%
\ auto\isanewline
\ \ \ \ \ \ \ \ \isacommand{also}\isamarkupfalse%
\ \isacommand{have}\isamarkupfalse%
\ {\isachardoublequoteopen}{\isasymdots}\ {\isacharequal}{\kern0pt}\ reverse{\isacharunderscore}{\kern0pt}QFT{\isacharunderscore}{\kern0pt}product{\isacharunderscore}{\kern0pt}representation\ j\ {\isacharparenleft}{\kern0pt}Suc\ n{\isacharparenright}{\kern0pt}{\isachardoublequoteclose}\isanewline
\ \ \ \ \ \ \ \ \isacommand{proof}\isamarkupfalse%
\ {\isacharminus}{\kern0pt}\isanewline
\ \ \ \ \ \ \ \ \ \ \isacommand{have}\isamarkupfalse%
\ {\isachardoublequoteopen}{\isacharparenleft}{\kern0pt}{\isadigit{1}}{\isacharslash}{\kern0pt}sqrt\ {\isadigit{2}}\ {\isasymcdot}\isactrlsub m\ {\isacharparenleft}{\kern0pt}{\isacharparenleft}{\kern0pt}\ {\isacharbar}{\kern0pt}zero{\isasymrangle}\ {\isacharplus}{\kern0pt}\ exp{\isacharparenleft}{\kern0pt}{\isadigit{2}}{\isacharasterisk}{\kern0pt}{\isasymi}{\isacharasterisk}{\kern0pt}pi{\isacharasterisk}{\kern0pt}j{\isacharslash}{\kern0pt}{\isacharparenleft}{\kern0pt}{\isadigit{2}}{\isacharcircum}{\kern0pt}{\isacharparenleft}{\kern0pt}Suc\ n{\isacharparenright}{\kern0pt}{\isacharparenright}{\kern0pt}{\isacharparenright}{\kern0pt}\ {\isasymcdot}\isactrlsub m\ {\isacharbar}{\kern0pt}one{\isasymrangle}{\isacharparenright}{\kern0pt}{\isacharparenright}{\kern0pt}{\isacharparenright}{\kern0pt}\ {\isasymOtimes}\isanewline
\ \ \ \ \ \ \ \ \ \ \ \ \ \ \ \ reverse{\isacharunderscore}{\kern0pt}QFT{\isacharunderscore}{\kern0pt}product{\isacharunderscore}{\kern0pt}representation\ jm\ n\ {\isacharequal}{\kern0pt}\isanewline
\ \ \ \ \ \ \ \ \ \ \ \ \ \ \ \ {\isacharparenleft}{\kern0pt}{\isadigit{1}}{\isacharslash}{\kern0pt}sqrt\ {\isadigit{2}}\ {\isasymcdot}\isactrlsub m\ {\isacharparenleft}{\kern0pt}{\isacharparenleft}{\kern0pt}\ {\isacharbar}{\kern0pt}zero{\isasymrangle}\ {\isacharplus}{\kern0pt}\ exp{\isacharparenleft}{\kern0pt}{\isadigit{2}}{\isacharasterisk}{\kern0pt}{\isasymi}{\isacharasterisk}{\kern0pt}pi{\isacharasterisk}{\kern0pt}j{\isacharslash}{\kern0pt}{\isacharparenleft}{\kern0pt}{\isadigit{2}}{\isacharcircum}{\kern0pt}{\isacharparenleft}{\kern0pt}Suc\ n{\isacharparenright}{\kern0pt}{\isacharparenright}{\kern0pt}{\isacharparenright}{\kern0pt}\ {\isasymcdot}\isactrlsub m\ {\isacharbar}{\kern0pt}one{\isasymrangle}{\isacharparenright}{\kern0pt}{\isacharparenright}{\kern0pt}{\isacharparenright}{\kern0pt}\ {\isasymOtimes}\isanewline
\ \ \ \ \ \ \ \ \ \ \ \ \ \ \ \ {\isacharparenleft}{\kern0pt}{\isadigit{1}}{\isacharslash}{\kern0pt}sqrt\ {\isacharparenleft}{\kern0pt}{\isadigit{2}}{\isacharcircum}{\kern0pt}n{\isacharparenright}{\kern0pt}\ {\isasymcdot}\isactrlsub m\ {\isacharparenleft}{\kern0pt}kron\ {\isacharparenleft}{\kern0pt}{\isasymlambda}{\isacharparenleft}{\kern0pt}l{\isacharcolon}{\kern0pt}{\isacharcolon}{\kern0pt}nat{\isacharparenright}{\kern0pt}{\isachardot}{\kern0pt}\ {\isacharbar}{\kern0pt}zero{\isasymrangle}\ {\isacharplus}{\kern0pt}\ exp\ {\isacharparenleft}{\kern0pt}{\isadigit{2}}{\isacharasterisk}{\kern0pt}{\isasymi}{\isacharasterisk}{\kern0pt}pi{\isacharasterisk}{\kern0pt}jm{\isacharslash}{\kern0pt}{\isacharparenleft}{\kern0pt}{\isadigit{2}}{\isacharcircum}{\kern0pt}l{\isacharparenright}{\kern0pt}{\isacharparenright}{\kern0pt}\ {\isasymcdot}\isactrlsub m\ {\isacharbar}{\kern0pt}one{\isasymrangle}{\isacharparenright}{\kern0pt}\ \isanewline
\ \ \ \ \ \ \ \ \ \ \ \ \ \ \ \ \ \ \ \ \ \ \ \ \ \ \ \ \ \ \ \ \ {\isacharparenleft}{\kern0pt}map\ nat\ {\isacharparenleft}{\kern0pt}rev\ {\isacharbrackleft}{\kern0pt}{\isadigit{1}}{\isachardot}{\kern0pt}{\isachardot}{\kern0pt}n{\isacharbrackright}{\kern0pt}{\isacharparenright}{\kern0pt}{\isacharparenright}{\kern0pt}{\isacharparenright}{\kern0pt}{\isacharparenright}{\kern0pt}{\isachardoublequoteclose}\isanewline
\ \ \ \ \ \ \ \ \ \ \ \ \isacommand{using}\isamarkupfalse%
\ reverse{\isacharunderscore}{\kern0pt}QFT{\isacharunderscore}{\kern0pt}product{\isacharunderscore}{\kern0pt}representation{\isacharunderscore}{\kern0pt}def\ \isacommand{by}\isamarkupfalse%
\ simp\isanewline
\ \ \ \ \ \ \ \ \ \ \isacommand{also}\isamarkupfalse%
\ \isacommand{have}\isamarkupfalse%
\ {\isachardoublequoteopen}{\isasymdots}\ {\isacharequal}{\kern0pt}\ {\isacharparenleft}{\kern0pt}{\isadigit{1}}{\isacharslash}{\kern0pt}sqrt\ {\isadigit{2}}\ {\isasymcdot}\isactrlsub m\ {\isacharparenleft}{\kern0pt}{\isacharparenleft}{\kern0pt}\ {\isacharbar}{\kern0pt}zero{\isasymrangle}\ {\isacharplus}{\kern0pt}\ exp{\isacharparenleft}{\kern0pt}{\isadigit{2}}{\isacharasterisk}{\kern0pt}{\isasymi}{\isacharasterisk}{\kern0pt}pi{\isacharasterisk}{\kern0pt}j{\isacharslash}{\kern0pt}{\isacharparenleft}{\kern0pt}{\isadigit{2}}{\isacharcircum}{\kern0pt}{\isacharparenleft}{\kern0pt}Suc\ n{\isacharparenright}{\kern0pt}{\isacharparenright}{\kern0pt}{\isacharparenright}{\kern0pt}\ {\isasymcdot}\isactrlsub m\ {\isacharbar}{\kern0pt}one{\isasymrangle}{\isacharparenright}{\kern0pt}{\isacharparenright}{\kern0pt}{\isacharparenright}{\kern0pt}\ {\isasymOtimes}\isanewline
\ \ \ \ \ \ \ \ \ \ \ \ \ \ \ \ \ \ \ \ \ \ \ \ \ \ {\isacharparenleft}{\kern0pt}{\isadigit{1}}{\isacharslash}{\kern0pt}sqrt\ {\isacharparenleft}{\kern0pt}{\isadigit{2}}{\isacharcircum}{\kern0pt}n{\isacharparenright}{\kern0pt}\ {\isasymcdot}\isactrlsub m\ {\isacharparenleft}{\kern0pt}kron\ {\isacharparenleft}{\kern0pt}{\isasymlambda}{\isacharparenleft}{\kern0pt}l{\isacharcolon}{\kern0pt}{\isacharcolon}{\kern0pt}nat{\isacharparenright}{\kern0pt}{\isachardot}{\kern0pt}\ {\isacharbar}{\kern0pt}zero{\isasymrangle}\ {\isacharplus}{\kern0pt}\ exp\ {\isacharparenleft}{\kern0pt}{\isadigit{2}}{\isacharasterisk}{\kern0pt}{\isasymi}{\isacharasterisk}{\kern0pt}pi{\isacharasterisk}{\kern0pt}j{\isacharslash}{\kern0pt}{\isacharparenleft}{\kern0pt}{\isadigit{2}}{\isacharcircum}{\kern0pt}l{\isacharparenright}{\kern0pt}{\isacharparenright}{\kern0pt}\ {\isasymcdot}\isactrlsub m\ {\isacharbar}{\kern0pt}one{\isasymrangle}{\isacharparenright}{\kern0pt}\ \isanewline
\ \ \ \ \ \ \ \ \ \ \ \ \ \ \ \ \ \ \ \ \ \ \ \ \ \ {\isacharparenleft}{\kern0pt}map\ nat\ {\isacharparenleft}{\kern0pt}rev\ {\isacharbrackleft}{\kern0pt}{\isadigit{1}}{\isachardot}{\kern0pt}{\isachardot}{\kern0pt}n{\isacharbrackright}{\kern0pt}{\isacharparenright}{\kern0pt}{\isacharparenright}{\kern0pt}{\isacharparenright}{\kern0pt}{\isacharparenright}{\kern0pt}{\isachardoublequoteclose}\isanewline
\ \ \ \ \ \ \ \ \ \ \ \ \isacommand{using}\isamarkupfalse%
\ kron{\isacharunderscore}{\kern0pt}j\ jm{\isacharunderscore}{\kern0pt}def\ \isacommand{by}\isamarkupfalse%
\ simp\isanewline
\ \ \ \ \ \ \ \ \ \ \isacommand{also}\isamarkupfalse%
\ \isacommand{have}\isamarkupfalse%
\ {\isachardoublequoteopen}{\isasymdots}\ {\isacharequal}{\kern0pt}\ {\isacharparenleft}{\kern0pt}{\isacharparenleft}{\kern0pt}{\isadigit{1}}{\isacharslash}{\kern0pt}sqrt\ {\isadigit{2}}{\isacharparenright}{\kern0pt}{\isacharasterisk}{\kern0pt}{\isacharparenleft}{\kern0pt}{\isadigit{1}}{\isacharslash}{\kern0pt}sqrt\ {\isacharparenleft}{\kern0pt}{\isadigit{2}}{\isacharcircum}{\kern0pt}n{\isacharparenright}{\kern0pt}{\isacharparenright}{\kern0pt}{\isacharparenright}{\kern0pt}\ {\isasymcdot}\isactrlsub m\ \isanewline
\ \ \ \ \ \ \ \ \ \ \ \ \ \ \ \ \ \ \ \ \ \ \ \ \ \ {\isacharparenleft}{\kern0pt}{\isacharparenleft}{\kern0pt}{\isacharparenleft}{\kern0pt}\ {\isacharbar}{\kern0pt}zero{\isasymrangle}\ {\isacharplus}{\kern0pt}\ exp{\isacharparenleft}{\kern0pt}{\isadigit{2}}{\isacharasterisk}{\kern0pt}{\isasymi}{\isacharasterisk}{\kern0pt}pi{\isacharasterisk}{\kern0pt}j{\isacharslash}{\kern0pt}{\isacharparenleft}{\kern0pt}{\isadigit{2}}{\isacharcircum}{\kern0pt}{\isacharparenleft}{\kern0pt}Suc\ n{\isacharparenright}{\kern0pt}{\isacharparenright}{\kern0pt}{\isacharparenright}{\kern0pt}\ {\isasymcdot}\isactrlsub m\ {\isacharbar}{\kern0pt}one{\isasymrangle}{\isacharparenright}{\kern0pt}{\isacharparenright}{\kern0pt}\ {\isasymOtimes}\isanewline
\ \ \ \ \ \ \ \ \ \ \ \ \ \ \ \ \ \ \ \ \ \ \ \ \ \ {\isacharparenleft}{\kern0pt}kron\ {\isacharparenleft}{\kern0pt}{\isasymlambda}{\isacharparenleft}{\kern0pt}l{\isacharcolon}{\kern0pt}{\isacharcolon}{\kern0pt}nat{\isacharparenright}{\kern0pt}{\isachardot}{\kern0pt}\ {\isacharbar}{\kern0pt}zero{\isasymrangle}\ {\isacharplus}{\kern0pt}\ exp\ {\isacharparenleft}{\kern0pt}{\isadigit{2}}{\isacharasterisk}{\kern0pt}{\isasymi}{\isacharasterisk}{\kern0pt}pi{\isacharasterisk}{\kern0pt}j{\isacharslash}{\kern0pt}{\isacharparenleft}{\kern0pt}{\isadigit{2}}{\isacharcircum}{\kern0pt}l{\isacharparenright}{\kern0pt}{\isacharparenright}{\kern0pt}\ {\isasymcdot}\isactrlsub m\ {\isacharbar}{\kern0pt}one{\isasymrangle}{\isacharparenright}{\kern0pt}\ \isanewline
\ \ \ \ \ \ \ \ \ \ \ \ \ \ \ \ \ \ \ \ \ \ \ \ \ \ {\isacharparenleft}{\kern0pt}map\ nat\ {\isacharparenleft}{\kern0pt}rev\ {\isacharbrackleft}{\kern0pt}{\isadigit{1}}{\isachardot}{\kern0pt}{\isachardot}{\kern0pt}n{\isacharbrackright}{\kern0pt}{\isacharparenright}{\kern0pt}{\isacharparenright}{\kern0pt}{\isacharparenright}{\kern0pt}{\isacharparenright}{\kern0pt}{\isachardoublequoteclose}\isanewline
\ \ \ \ \ \ \ \ \ \ \isacommand{proof}\isamarkupfalse%
\ {\isacharminus}{\kern0pt}\isanewline
\ \ \ \ \ \ \ \ \ \ \ \ \isacommand{have}\isamarkupfalse%
\ {\isachardoublequoteopen}dim{\isacharunderscore}{\kern0pt}col\ {\isacharparenleft}{\kern0pt}\ {\isacharbar}{\kern0pt}zero{\isasymrangle}\ {\isacharplus}{\kern0pt}\ exp{\isacharparenleft}{\kern0pt}{\isadigit{2}}{\isacharasterisk}{\kern0pt}{\isasymi}{\isacharasterisk}{\kern0pt}pi{\isacharasterisk}{\kern0pt}j{\isacharslash}{\kern0pt}{\isacharparenleft}{\kern0pt}{\isadigit{2}}{\isacharcircum}{\kern0pt}{\isacharparenleft}{\kern0pt}Suc\ n{\isacharparenright}{\kern0pt}{\isacharparenright}{\kern0pt}{\isacharparenright}{\kern0pt}\ {\isasymcdot}\isactrlsub m\ {\isacharbar}{\kern0pt}one{\isasymrangle}{\isacharparenright}{\kern0pt}\ {\isachargreater}{\kern0pt}\ {\isadigit{0}}{\isachardoublequoteclose}\isanewline
\ \ \ \ \ \ \ \ \ \ \ \ \ \ \isacommand{by}\isamarkupfalse%
\ {\isacharparenleft}{\kern0pt}simp\ add{\isacharcolon}{\kern0pt}\ ket{\isacharunderscore}{\kern0pt}vec{\isacharunderscore}{\kern0pt}def{\isacharparenright}{\kern0pt}\isanewline
\ \ \ \ \ \ \ \ \ \ \ \ \isacommand{moreover}\isamarkupfalse%
\ \isacommand{have}\isamarkupfalse%
\ {\isachardoublequoteopen}dim{\isacharunderscore}{\kern0pt}col\ {\isacharparenleft}{\kern0pt}kron\ {\isacharparenleft}{\kern0pt}{\isasymlambda}{\isacharparenleft}{\kern0pt}l{\isacharcolon}{\kern0pt}{\isacharcolon}{\kern0pt}nat{\isacharparenright}{\kern0pt}{\isachardot}{\kern0pt}\ {\isacharbar}{\kern0pt}zero{\isasymrangle}\ {\isacharplus}{\kern0pt}\ exp\ {\isacharparenleft}{\kern0pt}{\isadigit{2}}{\isacharasterisk}{\kern0pt}{\isasymi}{\isacharasterisk}{\kern0pt}pi{\isacharasterisk}{\kern0pt}j{\isacharslash}{\kern0pt}{\isacharparenleft}{\kern0pt}{\isadigit{2}}{\isacharcircum}{\kern0pt}l{\isacharparenright}{\kern0pt}{\isacharparenright}{\kern0pt}\ {\isasymcdot}\isactrlsub m\ {\isacharbar}{\kern0pt}one{\isasymrangle}{\isacharparenright}{\kern0pt}\ \isanewline
\ \ \ \ \ \ \ \ \ \ \ \ \ \ \ \ \ \ \ \ \ \ \ \ \ \ {\isacharparenleft}{\kern0pt}map\ nat\ {\isacharparenleft}{\kern0pt}rev\ {\isacharbrackleft}{\kern0pt}{\isadigit{1}}{\isachardot}{\kern0pt}{\isachardot}{\kern0pt}n{\isacharbrackright}{\kern0pt}{\isacharparenright}{\kern0pt}{\isacharparenright}{\kern0pt}{\isacharparenright}{\kern0pt}\ {\isachargreater}{\kern0pt}\ {\isadigit{0}}{\isachardoublequoteclose}\isanewline
\ \ \ \ \ \ \ \ \ \ \ \ \ \ \isacommand{using}\isamarkupfalse%
\ kron{\isacharunderscore}{\kern0pt}carrier{\isacharunderscore}{\kern0pt}mat\ ket{\isacharunderscore}{\kern0pt}vec{\isacharunderscore}{\kern0pt}def\isanewline
\ \ \ \ \ \ \ \ \ \ \ \ \ \ \isacommand{by}\isamarkupfalse%
\ {\isacharparenleft}{\kern0pt}metis\ {\isacharparenleft}{\kern0pt}no{\isacharunderscore}{\kern0pt}types{\isacharcomma}{\kern0pt}\ lifting{\isacharparenright}{\kern0pt}\ calculation\ carrier{\isacharunderscore}{\kern0pt}matD{\isacharparenleft}{\kern0pt}{\isadigit{2}}{\isacharparenright}{\kern0pt}\ dim{\isacharunderscore}{\kern0pt}col{\isacharunderscore}{\kern0pt}mat{\isacharparenleft}{\kern0pt}{\isadigit{1}}{\isacharparenright}{\kern0pt}\ \isanewline
\ \ \ \ \ \ \ \ \ \ \ \ \ \ \ \ \ \ dim{\isacharunderscore}{\kern0pt}row{\isacharunderscore}{\kern0pt}mat{\isacharparenleft}{\kern0pt}{\isadigit{1}}{\isacharparenright}{\kern0pt}\ index{\isacharunderscore}{\kern0pt}add{\isacharunderscore}{\kern0pt}mat{\isacharparenleft}{\kern0pt}{\isadigit{2}}{\isacharparenright}{\kern0pt}\ index{\isacharunderscore}{\kern0pt}add{\isacharunderscore}{\kern0pt}mat{\isacharparenleft}{\kern0pt}{\isadigit{3}}{\isacharparenright}{\kern0pt}\ index{\isacharunderscore}{\kern0pt}smult{\isacharunderscore}{\kern0pt}mat{\isacharparenleft}{\kern0pt}{\isadigit{2}}{\isacharparenright}{\kern0pt}\ \isanewline
\ \ \ \ \ \ \ \ \ \ \ \ \ \ \ \ \ \ index{\isacharunderscore}{\kern0pt}smult{\isacharunderscore}{\kern0pt}mat{\isacharparenleft}{\kern0pt}{\isadigit{3}}{\isacharparenright}{\kern0pt}\ index{\isacharunderscore}{\kern0pt}unit{\isacharunderscore}{\kern0pt}vec{\isacharparenleft}{\kern0pt}{\isadigit{3}}{\isacharparenright}{\kern0pt}{\isacharparenright}{\kern0pt}\isanewline
\ \ \ \ \ \ \ \ \ \ \ \ \isacommand{ultimately}\isamarkupfalse%
\ \isacommand{show}\isamarkupfalse%
\ {\isacharquery}{\kern0pt}thesis\ \isacommand{by}\isamarkupfalse%
\ simp\isanewline
\ \ \ \ \ \ \ \ \ \ \isacommand{qed}\isamarkupfalse%
\isanewline
\ \ \ \ \ \ \ \ \ \ \isacommand{also}\isamarkupfalse%
\ \isacommand{have}\isamarkupfalse%
\ {\isachardoublequoteopen}{\isasymdots}\ {\isacharequal}{\kern0pt}\ {\isacharparenleft}{\kern0pt}{\isadigit{1}}{\isacharslash}{\kern0pt}sqrt\ {\isacharparenleft}{\kern0pt}{\isadigit{2}}{\isacharcircum}{\kern0pt}{\isacharparenleft}{\kern0pt}Suc\ n{\isacharparenright}{\kern0pt}{\isacharparenright}{\kern0pt}{\isacharparenright}{\kern0pt}\ {\isasymcdot}\isactrlsub m\ \isanewline
\ \ \ \ \ \ \ \ \ \ \ \ \ \ \ \ \ \ \ \ \ \ \ \ \ \ {\isacharparenleft}{\kern0pt}{\isacharparenleft}{\kern0pt}{\isacharparenleft}{\kern0pt}\ {\isacharbar}{\kern0pt}zero{\isasymrangle}\ {\isacharplus}{\kern0pt}\ exp{\isacharparenleft}{\kern0pt}{\isadigit{2}}{\isacharasterisk}{\kern0pt}{\isasymi}{\isacharasterisk}{\kern0pt}pi{\isacharasterisk}{\kern0pt}j{\isacharslash}{\kern0pt}{\isacharparenleft}{\kern0pt}{\isadigit{2}}{\isacharcircum}{\kern0pt}{\isacharparenleft}{\kern0pt}Suc\ n{\isacharparenright}{\kern0pt}{\isacharparenright}{\kern0pt}{\isacharparenright}{\kern0pt}\ {\isasymcdot}\isactrlsub m\ {\isacharbar}{\kern0pt}one{\isasymrangle}{\isacharparenright}{\kern0pt}{\isacharparenright}{\kern0pt}\ {\isasymOtimes}\isanewline
\ \ \ \ \ \ \ \ \ \ \ \ \ \ \ \ \ \ \ \ \ \ \ \ \ \ {\isacharparenleft}{\kern0pt}kron\ {\isacharparenleft}{\kern0pt}{\isasymlambda}{\isacharparenleft}{\kern0pt}l{\isacharcolon}{\kern0pt}{\isacharcolon}{\kern0pt}nat{\isacharparenright}{\kern0pt}{\isachardot}{\kern0pt}\ {\isacharbar}{\kern0pt}zero{\isasymrangle}\ {\isacharplus}{\kern0pt}\ exp\ {\isacharparenleft}{\kern0pt}{\isadigit{2}}{\isacharasterisk}{\kern0pt}{\isasymi}{\isacharasterisk}{\kern0pt}pi{\isacharasterisk}{\kern0pt}j{\isacharslash}{\kern0pt}{\isacharparenleft}{\kern0pt}{\isadigit{2}}{\isacharcircum}{\kern0pt}l{\isacharparenright}{\kern0pt}{\isacharparenright}{\kern0pt}\ {\isasymcdot}\isactrlsub m\ {\isacharbar}{\kern0pt}one{\isasymrangle}{\isacharparenright}{\kern0pt}\ \isanewline
\ \ \ \ \ \ \ \ \ \ \ \ \ \ \ \ \ \ \ \ \ \ \ \ \ \ {\isacharparenleft}{\kern0pt}map\ nat\ {\isacharparenleft}{\kern0pt}rev\ {\isacharbrackleft}{\kern0pt}{\isadigit{1}}{\isachardot}{\kern0pt}{\isachardot}{\kern0pt}n{\isacharbrackright}{\kern0pt}{\isacharparenright}{\kern0pt}{\isacharparenright}{\kern0pt}{\isacharparenright}{\kern0pt}{\isacharparenright}{\kern0pt}{\isachardoublequoteclose}\isanewline
\ \ \ \ \ \ \ \ \ \ \ \ \isacommand{by}\isamarkupfalse%
\ {\isacharparenleft}{\kern0pt}simp\ add{\isacharcolon}{\kern0pt}\ real{\isacharunderscore}{\kern0pt}sqrt{\isacharunderscore}{\kern0pt}mult{\isacharparenright}{\kern0pt}\isanewline
\ \ \ \ \ \ \ \ \ \ \isacommand{also}\isamarkupfalse%
\ \isacommand{have}\isamarkupfalse%
\ {\isachardoublequoteopen}{\isasymdots}\ {\isacharequal}{\kern0pt}\ {\isacharparenleft}{\kern0pt}{\isadigit{1}}{\isacharslash}{\kern0pt}sqrt\ {\isacharparenleft}{\kern0pt}{\isadigit{2}}{\isacharcircum}{\kern0pt}{\isacharparenleft}{\kern0pt}Suc\ n{\isacharparenright}{\kern0pt}{\isacharparenright}{\kern0pt}{\isacharparenright}{\kern0pt}\ {\isasymcdot}\isactrlsub m\ \isanewline
\ \ \ \ \ \ \ \ \ \ \ \ \ \ \ \ \ \ \ \ \ \ \ \ \ \ {\isacharparenleft}{\kern0pt}kron\ {\isacharparenleft}{\kern0pt}{\isasymlambda}{\isacharparenleft}{\kern0pt}l{\isacharcolon}{\kern0pt}{\isacharcolon}{\kern0pt}nat{\isacharparenright}{\kern0pt}{\isachardot}{\kern0pt}\ {\isacharbar}{\kern0pt}zero{\isasymrangle}\ {\isacharplus}{\kern0pt}\ exp\ {\isacharparenleft}{\kern0pt}{\isadigit{2}}{\isacharasterisk}{\kern0pt}{\isasymi}{\isacharasterisk}{\kern0pt}pi{\isacharasterisk}{\kern0pt}j{\isacharslash}{\kern0pt}{\isacharparenleft}{\kern0pt}{\isadigit{2}}{\isacharcircum}{\kern0pt}l{\isacharparenright}{\kern0pt}{\isacharparenright}{\kern0pt}\ {\isasymcdot}\isactrlsub m\ {\isacharbar}{\kern0pt}one{\isasymrangle}{\isacharparenright}{\kern0pt}\ \isanewline
\ \ \ \ \ \ \ \ \ \ \ \ \ \ \ \ \ \ \ \ \ \ \ \ \ \ {\isacharparenleft}{\kern0pt}map\ nat\ {\isacharparenleft}{\kern0pt}rev\ {\isacharbrackleft}{\kern0pt}{\isadigit{1}}{\isachardot}{\kern0pt}{\isachardot}{\kern0pt}{\isacharparenleft}{\kern0pt}Suc\ n{\isacharparenright}{\kern0pt}{\isacharbrackright}{\kern0pt}{\isacharparenright}{\kern0pt}{\isacharparenright}{\kern0pt}{\isacharparenright}{\kern0pt}{\isachardoublequoteclose}\isanewline
\ \ \ \ \ \ \ \ \ \ \isacommand{proof}\isamarkupfalse%
\ {\isacharminus}{\kern0pt}\isanewline
\ \ \ \ \ \ \ \ \ \ \ \ \isacommand{define}\isamarkupfalse%
\ f\ \isakeyword{where}\ {\isachardoublequoteopen}f\ {\isacharequal}{\kern0pt}\ {\isacharparenleft}{\kern0pt}{\isasymlambda}{\isacharparenleft}{\kern0pt}l{\isacharcolon}{\kern0pt}{\isacharcolon}{\kern0pt}nat{\isacharparenright}{\kern0pt}{\isachardot}{\kern0pt}\ {\isacharbar}{\kern0pt}zero{\isasymrangle}\ {\isacharplus}{\kern0pt}\ exp\ {\isacharparenleft}{\kern0pt}{\isadigit{2}}{\isacharasterisk}{\kern0pt}{\isasymi}{\isacharasterisk}{\kern0pt}pi{\isacharasterisk}{\kern0pt}j{\isacharslash}{\kern0pt}{\isacharparenleft}{\kern0pt}{\isadigit{2}}{\isacharcircum}{\kern0pt}l{\isacharparenright}{\kern0pt}{\isacharparenright}{\kern0pt}\ {\isasymcdot}\isactrlsub m\ {\isacharbar}{\kern0pt}one{\isasymrangle}{\isacharparenright}{\kern0pt}{\isachardoublequoteclose}\isanewline
\ \ \ \ \ \ \ \ \ \ \ \ \isacommand{hence}\isamarkupfalse%
\ {\isachardoublequoteopen}{\isacharbar}{\kern0pt}zero{\isasymrangle}\ {\isacharplus}{\kern0pt}\ exp{\isacharparenleft}{\kern0pt}{\isadigit{2}}{\isacharasterisk}{\kern0pt}{\isasymi}{\isacharasterisk}{\kern0pt}pi{\isacharasterisk}{\kern0pt}j{\isacharslash}{\kern0pt}{\isacharparenleft}{\kern0pt}{\isadigit{2}}{\isacharcircum}{\kern0pt}{\isacharparenleft}{\kern0pt}Suc\ n{\isacharparenright}{\kern0pt}{\isacharparenright}{\kern0pt}{\isacharparenright}{\kern0pt}\ {\isasymcdot}\isactrlsub m\ {\isacharbar}{\kern0pt}one{\isasymrangle}\ {\isacharequal}{\kern0pt}\ f\ {\isacharparenleft}{\kern0pt}Suc\ n{\isacharparenright}{\kern0pt}{\isachardoublequoteclose}\ \isacommand{by}\isamarkupfalse%
\ simp\isanewline
\ \ \ \ \ \ \ \ \ \ \ \ \isacommand{hence}\isamarkupfalse%
\ {\isachardoublequoteopen}{\isacharparenleft}{\kern0pt}{\isacharparenleft}{\kern0pt}{\isacharparenleft}{\kern0pt}\ {\isacharbar}{\kern0pt}zero{\isasymrangle}\ {\isacharplus}{\kern0pt}\ exp{\isacharparenleft}{\kern0pt}{\isadigit{2}}{\isacharasterisk}{\kern0pt}{\isasymi}{\isacharasterisk}{\kern0pt}pi{\isacharasterisk}{\kern0pt}j{\isacharslash}{\kern0pt}{\isacharparenleft}{\kern0pt}{\isadigit{2}}{\isacharcircum}{\kern0pt}{\isacharparenleft}{\kern0pt}Suc\ n{\isacharparenright}{\kern0pt}{\isacharparenright}{\kern0pt}{\isacharparenright}{\kern0pt}\ {\isasymcdot}\isactrlsub m\ {\isacharbar}{\kern0pt}one{\isasymrangle}{\isacharparenright}{\kern0pt}{\isacharparenright}{\kern0pt}\ {\isasymOtimes}\isanewline
\ \ \ \ \ \ \ \ \ \ \ \ \ \ \ \ \ \ \ {\isacharparenleft}{\kern0pt}kron\ {\isacharparenleft}{\kern0pt}{\isasymlambda}{\isacharparenleft}{\kern0pt}l{\isacharcolon}{\kern0pt}{\isacharcolon}{\kern0pt}nat{\isacharparenright}{\kern0pt}{\isachardot}{\kern0pt}\ {\isacharbar}{\kern0pt}zero{\isasymrangle}\ {\isacharplus}{\kern0pt}\ exp\ {\isacharparenleft}{\kern0pt}{\isadigit{2}}{\isacharasterisk}{\kern0pt}{\isasymi}{\isacharasterisk}{\kern0pt}pi{\isacharasterisk}{\kern0pt}j{\isacharslash}{\kern0pt}{\isacharparenleft}{\kern0pt}{\isadigit{2}}{\isacharcircum}{\kern0pt}l{\isacharparenright}{\kern0pt}{\isacharparenright}{\kern0pt}\ {\isasymcdot}\isactrlsub m\ {\isacharbar}{\kern0pt}one{\isasymrangle}{\isacharparenright}{\kern0pt}\ \isanewline
\ \ \ \ \ \ \ \ \ \ \ \ \ \ \ \ \ \ \ {\isacharparenleft}{\kern0pt}map\ nat\ {\isacharparenleft}{\kern0pt}rev\ {\isacharbrackleft}{\kern0pt}{\isadigit{1}}{\isachardot}{\kern0pt}{\isachardot}{\kern0pt}n{\isacharbrackright}{\kern0pt}{\isacharparenright}{\kern0pt}{\isacharparenright}{\kern0pt}{\isacharparenright}{\kern0pt}{\isacharparenright}{\kern0pt}\ {\isacharequal}{\kern0pt}\ \isanewline
\ \ \ \ \ \ \ \ \ \ \ \ \ \ \ \ \ \ \ {\isacharparenleft}{\kern0pt}f\ {\isacharparenleft}{\kern0pt}Suc\ n{\isacharparenright}{\kern0pt}{\isacharparenright}{\kern0pt}\ {\isasymOtimes}\ {\isacharparenleft}{\kern0pt}kron\ f\ {\isacharparenleft}{\kern0pt}map\ nat\ {\isacharparenleft}{\kern0pt}rev\ {\isacharbrackleft}{\kern0pt}{\isadigit{1}}{\isachardot}{\kern0pt}{\isachardot}{\kern0pt}n{\isacharbrackright}{\kern0pt}{\isacharparenright}{\kern0pt}{\isacharparenright}{\kern0pt}{\isacharparenright}{\kern0pt}{\isachardoublequoteclose}\isanewline
\ \ \ \ \ \ \ \ \ \ \ \ \ \ \isacommand{using}\isamarkupfalse%
\ f{\isacharunderscore}{\kern0pt}def\ \isacommand{by}\isamarkupfalse%
\ simp\isanewline
\ \ \ \ \ \ \ \ \ \ \ \ \isacommand{also}\isamarkupfalse%
\ \isacommand{have}\isamarkupfalse%
\ {\isachardoublequoteopen}{\isasymdots}\ {\isacharequal}{\kern0pt}\ kron\ f\ {\isacharparenleft}{\kern0pt}{\isacharparenleft}{\kern0pt}Suc\ n{\isacharparenright}{\kern0pt}{\isacharhash}{\kern0pt}{\isacharparenleft}{\kern0pt}map\ nat\ {\isacharparenleft}{\kern0pt}rev\ {\isacharbrackleft}{\kern0pt}{\isadigit{1}}{\isachardot}{\kern0pt}{\isachardot}{\kern0pt}n{\isacharbrackright}{\kern0pt}{\isacharparenright}{\kern0pt}{\isacharparenright}{\kern0pt}{\isacharparenright}{\kern0pt}{\isachardoublequoteclose}\isanewline
\ \ \ \ \ \ \ \ \ \ \ \ \ \ \isacommand{using}\isamarkupfalse%
\ kron{\isachardot}{\kern0pt}simps{\isacharparenleft}{\kern0pt}{\isadigit{2}}{\isacharparenright}{\kern0pt}\ \isacommand{by}\isamarkupfalse%
\ simp\isanewline
\ \ \ \ \ \ \ \ \ \ \ \ \isacommand{also}\isamarkupfalse%
\ \isacommand{have}\isamarkupfalse%
\ {\isachardoublequoteopen}{\isasymdots}\ {\isacharequal}{\kern0pt}\ kron\ f\ {\isacharparenleft}{\kern0pt}map\ nat\ {\isacharparenleft}{\kern0pt}rev\ {\isacharbrackleft}{\kern0pt}{\isadigit{1}}{\isachardot}{\kern0pt}{\isachardot}{\kern0pt}{\isacharparenleft}{\kern0pt}Suc\ n{\isacharparenright}{\kern0pt}{\isacharbrackright}{\kern0pt}{\isacharparenright}{\kern0pt}{\isacharparenright}{\kern0pt}{\isachardoublequoteclose}\isanewline
\ \ \ \ \ \ \ \ \ \ \ \ \ \ \isacommand{using}\isamarkupfalse%
\ map{\isacharunderscore}{\kern0pt}def\ rev{\isacharunderscore}{\kern0pt}append\isanewline
\ \ \ \ \ \ \ \ \ \ \ \ \ \ \isacommand{by}\isamarkupfalse%
\ {\isacharparenleft}{\kern0pt}smt\ {\isacharparenleft}{\kern0pt}z{\isadigit{3}}{\isacharparenright}{\kern0pt}\ append{\isacharunderscore}{\kern0pt}Cons\ append{\isacharunderscore}{\kern0pt}self{\isacharunderscore}{\kern0pt}conv{\isadigit{2}}\ list{\isachardot}{\kern0pt}simps{\isacharparenleft}{\kern0pt}{\isadigit{9}}{\isacharparenright}{\kern0pt}\ nat{\isacharunderscore}{\kern0pt}int\ negative{\isacharunderscore}{\kern0pt}zless\ \isanewline
\ \ \ \ \ \ \ \ \ \ \ \ \ \ \ \ \ \ of{\isacharunderscore}{\kern0pt}nat{\isacharunderscore}{\kern0pt}Suc\ rev{\isacharunderscore}{\kern0pt}eq{\isacharunderscore}{\kern0pt}Cons{\isacharunderscore}{\kern0pt}iff\ rev{\isacharunderscore}{\kern0pt}is{\isacharunderscore}{\kern0pt}Nil{\isacharunderscore}{\kern0pt}conv\ upto{\isacharunderscore}{\kern0pt}rec{\isadigit{2}}{\isacharparenright}{\kern0pt}\isanewline
\ \ \ \ \ \ \ \ \ \ \ \ \isacommand{finally}\isamarkupfalse%
\ \isacommand{have}\isamarkupfalse%
\ {\isachardoublequoteopen}{\isacharparenleft}{\kern0pt}{\isacharparenleft}{\kern0pt}{\isacharparenleft}{\kern0pt}\ {\isacharbar}{\kern0pt}zero{\isasymrangle}\ {\isacharplus}{\kern0pt}\ exp{\isacharparenleft}{\kern0pt}{\isadigit{2}}{\isacharasterisk}{\kern0pt}{\isasymi}{\isacharasterisk}{\kern0pt}pi{\isacharasterisk}{\kern0pt}j{\isacharslash}{\kern0pt}{\isacharparenleft}{\kern0pt}{\isadigit{2}}{\isacharcircum}{\kern0pt}{\isacharparenleft}{\kern0pt}Suc\ n{\isacharparenright}{\kern0pt}{\isacharparenright}{\kern0pt}{\isacharparenright}{\kern0pt}\ {\isasymcdot}\isactrlsub m\ {\isacharbar}{\kern0pt}one{\isasymrangle}{\isacharparenright}{\kern0pt}{\isacharparenright}{\kern0pt}\ {\isasymOtimes}\isanewline
\ \ \ \ \ \ \ \ \ \ \ \ \ \ \ \ \ \ \ \ \ \ \ \ \ \ {\isacharparenleft}{\kern0pt}kron\ {\isacharparenleft}{\kern0pt}{\isasymlambda}{\isacharparenleft}{\kern0pt}l{\isacharcolon}{\kern0pt}{\isacharcolon}{\kern0pt}nat{\isacharparenright}{\kern0pt}{\isachardot}{\kern0pt}\ {\isacharbar}{\kern0pt}zero{\isasymrangle}\ {\isacharplus}{\kern0pt}\ exp\ {\isacharparenleft}{\kern0pt}{\isadigit{2}}{\isacharasterisk}{\kern0pt}{\isasymi}{\isacharasterisk}{\kern0pt}pi{\isacharasterisk}{\kern0pt}j{\isacharslash}{\kern0pt}{\isacharparenleft}{\kern0pt}{\isadigit{2}}{\isacharcircum}{\kern0pt}l{\isacharparenright}{\kern0pt}{\isacharparenright}{\kern0pt}\ {\isasymcdot}\isactrlsub m\ {\isacharbar}{\kern0pt}one{\isasymrangle}{\isacharparenright}{\kern0pt}\ \isanewline
\ \ \ \ \ \ \ \ \ \ \ \ \ \ \ \ \ \ \ \ \ \ \ \ \ \ {\isacharparenleft}{\kern0pt}map\ nat\ {\isacharparenleft}{\kern0pt}rev\ {\isacharbrackleft}{\kern0pt}{\isadigit{1}}{\isachardot}{\kern0pt}{\isachardot}{\kern0pt}n{\isacharbrackright}{\kern0pt}{\isacharparenright}{\kern0pt}{\isacharparenright}{\kern0pt}{\isacharparenright}{\kern0pt}{\isacharparenright}{\kern0pt}\ {\isacharequal}{\kern0pt}\isanewline
\ \ \ \ \ \ \ \ \ \ \ \ \ \ \ \ \ \ \ \ \ \ \ \ \ \ {\isacharparenleft}{\kern0pt}kron\ {\isacharparenleft}{\kern0pt}{\isasymlambda}{\isacharparenleft}{\kern0pt}l{\isacharcolon}{\kern0pt}{\isacharcolon}{\kern0pt}nat{\isacharparenright}{\kern0pt}{\isachardot}{\kern0pt}\ {\isacharbar}{\kern0pt}zero{\isasymrangle}\ {\isacharplus}{\kern0pt}\ exp\ {\isacharparenleft}{\kern0pt}{\isadigit{2}}{\isacharasterisk}{\kern0pt}{\isasymi}{\isacharasterisk}{\kern0pt}pi{\isacharasterisk}{\kern0pt}j{\isacharslash}{\kern0pt}{\isacharparenleft}{\kern0pt}{\isadigit{2}}{\isacharcircum}{\kern0pt}l{\isacharparenright}{\kern0pt}{\isacharparenright}{\kern0pt}\ {\isasymcdot}\isactrlsub m\ {\isacharbar}{\kern0pt}one{\isasymrangle}{\isacharparenright}{\kern0pt}\ \isanewline
\ \ \ \ \ \ \ \ \ \ \ \ \ \ \ \ \ \ \ \ \ \ \ \ \ \ {\isacharparenleft}{\kern0pt}map\ nat\ {\isacharparenleft}{\kern0pt}rev\ {\isacharbrackleft}{\kern0pt}{\isadigit{1}}{\isachardot}{\kern0pt}{\isachardot}{\kern0pt}{\isacharparenleft}{\kern0pt}Suc\ n{\isacharparenright}{\kern0pt}{\isacharbrackright}{\kern0pt}{\isacharparenright}{\kern0pt}{\isacharparenright}{\kern0pt}{\isacharparenright}{\kern0pt}{\isachardoublequoteclose}\isanewline
\ \ \ \ \ \ \ \ \ \ \ \ \ \ \isacommand{using}\isamarkupfalse%
\ f{\isacharunderscore}{\kern0pt}def\ \isacommand{by}\isamarkupfalse%
\ simp\isanewline
\ \ \ \ \ \ \ \ \ \ \ \ \isacommand{thus}\isamarkupfalse%
\ {\isacharquery}{\kern0pt}thesis\ \isacommand{by}\isamarkupfalse%
\ simp\isanewline
\ \ \ \ \ \ \ \ \ \ \isacommand{qed}\isamarkupfalse%
\isanewline
\ \ \ \ \ \ \ \ \ \ \isacommand{also}\isamarkupfalse%
\ \isacommand{have}\isamarkupfalse%
\ {\isachardoublequoteopen}{\isasymdots}\ {\isacharequal}{\kern0pt}\ reverse{\isacharunderscore}{\kern0pt}QFT{\isacharunderscore}{\kern0pt}product{\isacharunderscore}{\kern0pt}representation\ j\ {\isacharparenleft}{\kern0pt}Suc\ n{\isacharparenright}{\kern0pt}{\isachardoublequoteclose}\isanewline
\ \ \ \ \ \ \ \ \ \ \ \ \isacommand{using}\isamarkupfalse%
\ reverse{\isacharunderscore}{\kern0pt}QFT{\isacharunderscore}{\kern0pt}product{\isacharunderscore}{\kern0pt}representation{\isacharunderscore}{\kern0pt}def\ \isacommand{by}\isamarkupfalse%
\ simp\isanewline
\ \ \ \ \ \ \ \ \ \ \isacommand{finally}\isamarkupfalse%
\ \isacommand{show}\isamarkupfalse%
\ {\isacharquery}{\kern0pt}thesis\ \isacommand{by}\isamarkupfalse%
\ this\isanewline
\ \ \ \ \ \ \ \ \isacommand{qed}\isamarkupfalse%
\isanewline
\ \ \ \ \ \ \ \ \isacommand{finally}\isamarkupfalse%
\ \isacommand{show}\isamarkupfalse%
\ {\isacharquery}{\kern0pt}thesis\ \isacommand{by}\isamarkupfalse%
\ this\isanewline
\ \ \ \ \ \ \isacommand{qed}\isamarkupfalse%
\isanewline
\ \ \ \ \isacommand{qed}\isamarkupfalse%
\isanewline
\ \ \isacommand{qed}\isamarkupfalse%
\isanewline
\isacommand{qed}\isamarkupfalse%
%
\endisatagproof
{\isafoldproof}%
%
\isadelimproof
%
\endisadelimproof
%
\isadelimdocument
%
\endisadelimdocument
%
\isatagdocument
%
\isamarkupsubsection{QFT with qubits reordering correctness%
}
\isamarkuptrue%
%
\endisatagdocument
{\isafolddocument}%
%
\isadelimdocument
%
\endisadelimdocument
\isacommand{lemma}\isamarkupfalse%
\ SWAP{\isacharunderscore}{\kern0pt}down{\isacharunderscore}{\kern0pt}kron{\isacharcolon}{\kern0pt}\isanewline
\ \ \ \ \isakeyword{assumes}\ {\isachardoublequoteopen}{\isasymforall}m{\isachardot}{\kern0pt}\ dim{\isacharunderscore}{\kern0pt}row\ {\isacharparenleft}{\kern0pt}f\ m{\isacharparenright}{\kern0pt}\ {\isacharequal}{\kern0pt}\ {\isadigit{2}}\ {\isasymand}\ dim{\isacharunderscore}{\kern0pt}col\ {\isacharparenleft}{\kern0pt}f\ m{\isacharparenright}{\kern0pt}\ {\isacharequal}{\kern0pt}\ {\isadigit{1}}{\isachardoublequoteclose}\ \isanewline
\ \ \isakeyword{shows}\ {\isachardoublequoteopen}SWAP{\isacharunderscore}{\kern0pt}down\ {\isacharparenleft}{\kern0pt}length\ {\isacharparenleft}{\kern0pt}x{\isacharhash}{\kern0pt}xs{\isacharparenright}{\kern0pt}{\isacharparenright}{\kern0pt}\ {\isacharasterisk}{\kern0pt}\ kron\ f\ {\isacharparenleft}{\kern0pt}x{\isacharhash}{\kern0pt}xs{\isacharparenright}{\kern0pt}\ {\isacharequal}{\kern0pt}\ kron\ f\ xs\ {\isasymOtimes}\ f\ x{\isachardoublequoteclose}\isanewline
%
\isadelimproof
%
\endisadelimproof
%
\isatagproof
\isacommand{proof}\isamarkupfalse%
\ {\isacharparenleft}{\kern0pt}induct\ xs\ rule{\isacharcolon}{\kern0pt}\ rev{\isacharunderscore}{\kern0pt}induct{\isacharparenright}{\kern0pt}\isanewline
\ \ \isacommand{case}\isamarkupfalse%
\ Nil\isanewline
\ \ \isacommand{have}\isamarkupfalse%
\ {\isachardoublequoteopen}SWAP{\isacharunderscore}{\kern0pt}down\ {\isacharparenleft}{\kern0pt}length\ {\isacharbrackleft}{\kern0pt}x{\isacharbrackright}{\kern0pt}{\isacharparenright}{\kern0pt}\ {\isacharasterisk}{\kern0pt}\ kron\ f\ {\isacharbrackleft}{\kern0pt}x{\isacharbrackright}{\kern0pt}\ {\isacharequal}{\kern0pt}\ {\isacharparenleft}{\kern0pt}{\isadigit{1}}\isactrlsub m\ {\isadigit{2}}{\isacharparenright}{\kern0pt}\ {\isacharasterisk}{\kern0pt}\ f\ x{\isachardoublequoteclose}\ \isacommand{using}\isamarkupfalse%
\ SWAP{\isacharunderscore}{\kern0pt}down{\isachardot}{\kern0pt}simps{\isacharparenleft}{\kern0pt}{\isadigit{2}}{\isacharparenright}{\kern0pt}\ kron{\isachardot}{\kern0pt}simps{\isacharparenleft}{\kern0pt}{\isadigit{2}}{\isacharparenright}{\kern0pt}\isanewline
\ \ \ \ \isacommand{by}\isamarkupfalse%
\ {\isacharparenleft}{\kern0pt}metis\ carrier{\isacharunderscore}{\kern0pt}matI\ kron{\isachardot}{\kern0pt}simps{\isacharparenleft}{\kern0pt}{\isadigit{1}}{\isacharparenright}{\kern0pt}\ length{\isacharunderscore}{\kern0pt}{\isadigit{0}}{\isacharunderscore}{\kern0pt}conv\ length{\isacharunderscore}{\kern0pt}Cons\ right{\isacharunderscore}{\kern0pt}tensor{\isacharunderscore}{\kern0pt}id{\isacharparenright}{\kern0pt}\isanewline
\ \ \isacommand{also}\isamarkupfalse%
\ \isacommand{have}\isamarkupfalse%
\ {\isachardoublequoteopen}{\isasymdots}\ {\isacharequal}{\kern0pt}\ f\ x{\isachardoublequoteclose}\ \isacommand{using}\isamarkupfalse%
\ left{\isacharunderscore}{\kern0pt}mult{\isacharunderscore}{\kern0pt}one{\isacharunderscore}{\kern0pt}mat{\isacharprime}{\kern0pt}\ assms\ \isacommand{by}\isamarkupfalse%
\ auto\isanewline
\ \ \isacommand{also}\isamarkupfalse%
\ \isacommand{have}\isamarkupfalse%
\ {\isachardoublequoteopen}{\isasymdots}\ {\isacharequal}{\kern0pt}\ {\isacharparenleft}{\kern0pt}{\isadigit{1}}\isactrlsub m\ {\isadigit{1}}{\isacharparenright}{\kern0pt}\ {\isasymOtimes}\ f\ x{\isachardoublequoteclose}\ \isacommand{using}\isamarkupfalse%
\ left{\isacharunderscore}{\kern0pt}tensor{\isacharunderscore}{\kern0pt}id\ \isacommand{by}\isamarkupfalse%
\ auto\isanewline
\ \ \isacommand{also}\isamarkupfalse%
\ \isacommand{have}\isamarkupfalse%
\ {\isachardoublequoteopen}{\isasymdots}\ {\isacharequal}{\kern0pt}\ kron\ f\ {\isacharbrackleft}{\kern0pt}{\isacharbrackright}{\kern0pt}\ {\isasymOtimes}\ f\ x{\isachardoublequoteclose}\ \isacommand{using}\isamarkupfalse%
\ kron{\isachardot}{\kern0pt}simps\ \isacommand{by}\isamarkupfalse%
\ auto\isanewline
\ \ \isacommand{finally}\isamarkupfalse%
\ \isacommand{show}\isamarkupfalse%
\ {\isacharquery}{\kern0pt}case\ \isacommand{by}\isamarkupfalse%
\ this\isanewline
\isacommand{next}\isamarkupfalse%
\isanewline
\ \ \isacommand{case}\isamarkupfalse%
\ {\isacharparenleft}{\kern0pt}snoc\ a\ xs{\isacharparenright}{\kern0pt}\isanewline
\ \ \isacommand{assume}\isamarkupfalse%
\ HI{\isacharcolon}{\kern0pt}{\isachardoublequoteopen}SWAP{\isacharunderscore}{\kern0pt}down\ {\isacharparenleft}{\kern0pt}length\ {\isacharparenleft}{\kern0pt}x{\isacharhash}{\kern0pt}xs{\isacharparenright}{\kern0pt}{\isacharparenright}{\kern0pt}\ {\isacharasterisk}{\kern0pt}\ kron\ f\ {\isacharparenleft}{\kern0pt}x{\isacharhash}{\kern0pt}xs{\isacharparenright}{\kern0pt}\ {\isacharequal}{\kern0pt}\ kron\ f\ xs\ {\isasymOtimes}\ f\ x{\isachardoublequoteclose}\isanewline
\ \ \isacommand{define}\isamarkupfalse%
\ n{\isacharcolon}{\kern0pt}{\isacharcolon}{\kern0pt}nat\ \isakeyword{where}\ {\isachardoublequoteopen}n\ {\isacharequal}{\kern0pt}\ length\ xs{\isachardoublequoteclose}\isanewline
\ \ \isacommand{show}\isamarkupfalse%
\ {\isacharquery}{\kern0pt}case\isanewline
\ \ \isacommand{proof}\isamarkupfalse%
\ {\isacharparenleft}{\kern0pt}cases{\isacharparenright}{\kern0pt}\isanewline
\ \ \ \ \isacommand{assume}\isamarkupfalse%
\ Nil{\isacharcolon}{\kern0pt}{\isachardoublequoteopen}xs\ {\isacharequal}{\kern0pt}\ {\isacharbrackleft}{\kern0pt}{\isacharbrackright}{\kern0pt}{\isachardoublequoteclose}\isanewline
\ \ \ \ \isacommand{hence}\isamarkupfalse%
\ {\isachardoublequoteopen}n\ {\isacharequal}{\kern0pt}\ {\isadigit{0}}{\isachardoublequoteclose}\ \isacommand{using}\isamarkupfalse%
\ n{\isacharunderscore}{\kern0pt}def\ \isacommand{by}\isamarkupfalse%
\ auto\isanewline
\ \ \ \ \isacommand{have}\isamarkupfalse%
\ {\isachardoublequoteopen}SWAP{\isacharunderscore}{\kern0pt}down\ {\isacharparenleft}{\kern0pt}length\ {\isacharparenleft}{\kern0pt}x{\isacharhash}{\kern0pt}xs{\isacharat}{\kern0pt}{\isacharbrackleft}{\kern0pt}a{\isacharbrackright}{\kern0pt}{\isacharparenright}{\kern0pt}{\isacharparenright}{\kern0pt}\ {\isacharasterisk}{\kern0pt}\ kron\ f\ {\isacharparenleft}{\kern0pt}x{\isacharhash}{\kern0pt}xs{\isacharat}{\kern0pt}{\isacharbrackleft}{\kern0pt}a{\isacharbrackright}{\kern0pt}{\isacharparenright}{\kern0pt}\ {\isacharequal}{\kern0pt}\isanewline
\ \ \ \ \ \ \ \ \ \ SWAP{\isacharunderscore}{\kern0pt}down\ {\isacharparenleft}{\kern0pt}Suc\ {\isacharparenleft}{\kern0pt}Suc\ {\isadigit{0}}{\isacharparenright}{\kern0pt}{\isacharparenright}{\kern0pt}\ {\isacharasterisk}{\kern0pt}\ kron\ f\ {\isacharparenleft}{\kern0pt}x{\isacharhash}{\kern0pt}{\isacharbrackleft}{\kern0pt}a{\isacharbrackright}{\kern0pt}{\isacharparenright}{\kern0pt}{\isachardoublequoteclose}\isanewline
\ \ \ \ \ \ \isacommand{using}\isamarkupfalse%
\ n{\isacharunderscore}{\kern0pt}def\ Nil\ \isacommand{by}\isamarkupfalse%
\ simp\isanewline
\ \ \ \ \isacommand{also}\isamarkupfalse%
\ \isacommand{have}\isamarkupfalse%
\ {\isachardoublequoteopen}{\isasymdots}\ {\isacharequal}{\kern0pt}\ SWAP\ {\isacharasterisk}{\kern0pt}\ kron\ f\ {\isacharparenleft}{\kern0pt}x{\isacharhash}{\kern0pt}{\isacharbrackleft}{\kern0pt}a{\isacharbrackright}{\kern0pt}{\isacharparenright}{\kern0pt}{\isachardoublequoteclose}\ \isacommand{using}\isamarkupfalse%
\ SWAP{\isacharunderscore}{\kern0pt}down{\isachardot}{\kern0pt}simps{\isacharparenleft}{\kern0pt}{\isadigit{3}}{\isacharparenright}{\kern0pt}\ \isacommand{by}\isamarkupfalse%
\ simp\isanewline
\ \ \ \ \isacommand{also}\isamarkupfalse%
\ \isacommand{have}\isamarkupfalse%
\ {\isachardoublequoteopen}{\isasymdots}\ {\isacharequal}{\kern0pt}\ SWAP\ {\isacharasterisk}{\kern0pt}\ {\isacharparenleft}{\kern0pt}{\isacharparenleft}{\kern0pt}f\ x{\isacharparenright}{\kern0pt}\ {\isasymOtimes}\ {\isacharparenleft}{\kern0pt}f\ a{\isacharparenright}{\kern0pt}{\isacharparenright}{\kern0pt}{\isachardoublequoteclose}\ \isacommand{using}\isamarkupfalse%
\ kron{\isachardot}{\kern0pt}simps{\isacharparenleft}{\kern0pt}{\isadigit{2}}{\isacharparenright}{\kern0pt}\isanewline
\ \ \ \ \ \ \isacommand{by}\isamarkupfalse%
\ {\isacharparenleft}{\kern0pt}metis\ carrier{\isacharunderscore}{\kern0pt}matI\ kron{\isachardot}{\kern0pt}simps{\isacharparenleft}{\kern0pt}{\isadigit{1}}{\isacharparenright}{\kern0pt}\ right{\isacharunderscore}{\kern0pt}tensor{\isacharunderscore}{\kern0pt}id{\isacharparenright}{\kern0pt}\isanewline
\ \ \ \ \isacommand{also}\isamarkupfalse%
\ \isacommand{have}\isamarkupfalse%
\ {\isachardoublequoteopen}{\isasymdots}\ {\isacharequal}{\kern0pt}\ {\isacharparenleft}{\kern0pt}f\ a{\isacharparenright}{\kern0pt}\ {\isasymOtimes}\ {\isacharparenleft}{\kern0pt}f\ x{\isacharparenright}{\kern0pt}{\isachardoublequoteclose}\ \isacommand{using}\isamarkupfalse%
\ SWAP{\isacharunderscore}{\kern0pt}tensor\ assms\ \isacommand{by}\isamarkupfalse%
\ auto\isanewline
\ \ \ \ \isacommand{also}\isamarkupfalse%
\ \isacommand{have}\isamarkupfalse%
\ {\isachardoublequoteopen}{\isasymdots}\ {\isacharequal}{\kern0pt}\ kron\ f\ {\isacharparenleft}{\kern0pt}xs{\isacharat}{\kern0pt}{\isacharbrackleft}{\kern0pt}a{\isacharbrackright}{\kern0pt}{\isacharparenright}{\kern0pt}\ {\isasymOtimes}\ {\isacharparenleft}{\kern0pt}f\ x{\isacharparenright}{\kern0pt}{\isachardoublequoteclose}\ \isacommand{using}\isamarkupfalse%
\ kron{\isachardot}{\kern0pt}simps\ Nil\isanewline
\ \ \ \ \ \ \isacommand{by}\isamarkupfalse%
\ {\isacharparenleft}{\kern0pt}metis\ carrier{\isacharunderscore}{\kern0pt}mat{\isacharunderscore}{\kern0pt}triv\ kron{\isacharunderscore}{\kern0pt}cons{\isacharunderscore}{\kern0pt}right\ left{\isacharunderscore}{\kern0pt}tensor{\isacharunderscore}{\kern0pt}id{\isacharparenright}{\kern0pt}\isanewline
\ \ \ \ \isacommand{finally}\isamarkupfalse%
\ \isacommand{show}\isamarkupfalse%
\ {\isacharquery}{\kern0pt}case\ \isacommand{by}\isamarkupfalse%
\ this\isanewline
\ \ \isacommand{next}\isamarkupfalse%
\isanewline
\ \ \ \ \isacommand{assume}\isamarkupfalse%
\ NNil{\isacharcolon}{\kern0pt}{\isachardoublequoteopen}xs\ {\isasymnoteq}\ {\isacharbrackleft}{\kern0pt}{\isacharbrackright}{\kern0pt}{\isachardoublequoteclose}\isanewline
\ \ \ \ \isacommand{hence}\isamarkupfalse%
\ {\isachardoublequoteopen}n\ {\isachargreater}{\kern0pt}\ {\isadigit{0}}{\isachardoublequoteclose}\ \isacommand{using}\isamarkupfalse%
\ n{\isacharunderscore}{\kern0pt}def\ \isacommand{by}\isamarkupfalse%
\ auto\isanewline
\ \ \ \ \isacommand{hence}\isamarkupfalse%
\ e{\isacharcolon}{\kern0pt}{\isachardoublequoteopen}{\isasymexists}m{\isachardot}{\kern0pt}\ n\ {\isacharequal}{\kern0pt}\ Suc\ m{\isachardoublequoteclose}\ \isacommand{by}\isamarkupfalse%
\ {\isacharparenleft}{\kern0pt}simp\ add{\isacharcolon}{\kern0pt}\ gr{\isadigit{0}}{\isacharunderscore}{\kern0pt}implies{\isacharunderscore}{\kern0pt}Suc{\isacharparenright}{\kern0pt}\isanewline
\ \ \ \ \isacommand{have}\isamarkupfalse%
\ {\isachardoublequoteopen}SWAP{\isacharunderscore}{\kern0pt}down\ {\isacharparenleft}{\kern0pt}length\ {\isacharparenleft}{\kern0pt}x{\isacharhash}{\kern0pt}xs{\isacharat}{\kern0pt}{\isacharbrackleft}{\kern0pt}a{\isacharbrackright}{\kern0pt}{\isacharparenright}{\kern0pt}{\isacharparenright}{\kern0pt}\ {\isacharasterisk}{\kern0pt}\ kron\ f\ {\isacharparenleft}{\kern0pt}x{\isacharhash}{\kern0pt}xs{\isacharat}{\kern0pt}{\isacharbrackleft}{\kern0pt}a{\isacharbrackright}{\kern0pt}{\isacharparenright}{\kern0pt}\ {\isacharequal}{\kern0pt}\isanewline
\ \ \ \ \ \ \ \ \ \ SWAP{\isacharunderscore}{\kern0pt}down\ {\isacharparenleft}{\kern0pt}Suc\ {\isacharparenleft}{\kern0pt}Suc\ n{\isacharparenright}{\kern0pt}{\isacharparenright}{\kern0pt}\ {\isacharasterisk}{\kern0pt}\ kron\ f\ {\isacharparenleft}{\kern0pt}x{\isacharhash}{\kern0pt}xs{\isacharat}{\kern0pt}{\isacharbrackleft}{\kern0pt}a{\isacharbrackright}{\kern0pt}{\isacharparenright}{\kern0pt}{\isachardoublequoteclose}\isanewline
\ \ \ \ \ \ \isacommand{using}\isamarkupfalse%
\ n{\isacharunderscore}{\kern0pt}def\ \isacommand{by}\isamarkupfalse%
\ auto\isanewline
\ \ \ \ \isacommand{also}\isamarkupfalse%
\ \isacommand{have}\isamarkupfalse%
\ {\isachardoublequoteopen}{\isasymdots}\ {\isacharequal}{\kern0pt}\ {\isacharparenleft}{\kern0pt}{\isacharparenleft}{\kern0pt}{\isadigit{1}}\isactrlsub m\ {\isacharparenleft}{\kern0pt}{\isadigit{2}}{\isacharcircum}{\kern0pt}n{\isacharparenright}{\kern0pt}{\isacharparenright}{\kern0pt}\ {\isasymOtimes}\ SWAP{\isacharparenright}{\kern0pt}\ {\isacharasterisk}{\kern0pt}\ {\isacharparenleft}{\kern0pt}{\isacharparenleft}{\kern0pt}SWAP{\isacharunderscore}{\kern0pt}down\ {\isacharparenleft}{\kern0pt}Suc\ n{\isacharparenright}{\kern0pt}{\isacharparenright}{\kern0pt}\ {\isasymOtimes}\ {\isacharparenleft}{\kern0pt}{\isadigit{1}}\isactrlsub m\ {\isadigit{2}}{\isacharparenright}{\kern0pt}{\isacharparenright}{\kern0pt}\ {\isacharasterisk}{\kern0pt}\ kron\ f\ {\isacharparenleft}{\kern0pt}x{\isacharhash}{\kern0pt}xs{\isacharat}{\kern0pt}{\isacharbrackleft}{\kern0pt}a{\isacharbrackright}{\kern0pt}{\isacharparenright}{\kern0pt}{\isachardoublequoteclose}\isanewline
\ \ \ \ \ \ \isacommand{using}\isamarkupfalse%
\ SWAP{\isacharunderscore}{\kern0pt}down{\isachardot}{\kern0pt}simps\ e\ \isacommand{by}\isamarkupfalse%
\ auto\isanewline
\ \ \ \ \isacommand{also}\isamarkupfalse%
\ \isacommand{have}\isamarkupfalse%
\ {\isachardoublequoteopen}{\isasymdots}\ {\isacharequal}{\kern0pt}\ {\isacharparenleft}{\kern0pt}{\isacharparenleft}{\kern0pt}{\isadigit{1}}\isactrlsub m\ {\isacharparenleft}{\kern0pt}{\isadigit{2}}{\isacharcircum}{\kern0pt}n{\isacharparenright}{\kern0pt}{\isacharparenright}{\kern0pt}\ {\isasymOtimes}\ SWAP{\isacharparenright}{\kern0pt}\ {\isacharasterisk}{\kern0pt}\ {\isacharparenleft}{\kern0pt}{\isacharparenleft}{\kern0pt}{\isacharparenleft}{\kern0pt}SWAP{\isacharunderscore}{\kern0pt}down\ {\isacharparenleft}{\kern0pt}Suc\ n{\isacharparenright}{\kern0pt}{\isacharparenright}{\kern0pt}\ {\isasymOtimes}\ {\isacharparenleft}{\kern0pt}{\isadigit{1}}\isactrlsub m\ {\isadigit{2}}{\isacharparenright}{\kern0pt}{\isacharparenright}{\kern0pt}\ {\isacharasterisk}{\kern0pt}\ kron\ f\ {\isacharparenleft}{\kern0pt}x{\isacharhash}{\kern0pt}xs{\isacharat}{\kern0pt}{\isacharbrackleft}{\kern0pt}a{\isacharbrackright}{\kern0pt}{\isacharparenright}{\kern0pt}{\isacharparenright}{\kern0pt}{\isachardoublequoteclose}\isanewline
\ \ \ \ \isacommand{proof}\isamarkupfalse%
\ {\isacharparenleft}{\kern0pt}rule\ assoc{\isacharunderscore}{\kern0pt}mult{\isacharunderscore}{\kern0pt}mat{\isacharparenright}{\kern0pt}\isanewline
\ \ \ \ \ \ \isacommand{show}\isamarkupfalse%
\ {\isachardoublequoteopen}{\isacharparenleft}{\kern0pt}{\isacharparenleft}{\kern0pt}{\isadigit{1}}\isactrlsub m\ {\isacharparenleft}{\kern0pt}{\isadigit{2}}{\isacharcircum}{\kern0pt}n{\isacharparenright}{\kern0pt}{\isacharparenright}{\kern0pt}\ {\isasymOtimes}\ SWAP{\isacharparenright}{\kern0pt}\ {\isasymin}\ carrier{\isacharunderscore}{\kern0pt}mat\ {\isacharparenleft}{\kern0pt}{\isadigit{2}}{\isacharcircum}{\kern0pt}{\isacharparenleft}{\kern0pt}Suc\ {\isacharparenleft}{\kern0pt}Suc\ n{\isacharparenright}{\kern0pt}{\isacharparenright}{\kern0pt}{\isacharparenright}{\kern0pt}\ {\isacharparenleft}{\kern0pt}{\isadigit{2}}{\isacharcircum}{\kern0pt}{\isacharparenleft}{\kern0pt}Suc\ {\isacharparenleft}{\kern0pt}Suc\ n{\isacharparenright}{\kern0pt}{\isacharparenright}{\kern0pt}{\isacharparenright}{\kern0pt}{\isachardoublequoteclose}\isanewline
\ \ \ \ \ \ \isacommand{proof}\isamarkupfalse%
\ {\isacharminus}{\kern0pt}\isanewline
\ \ \ \ \ \ \ \ \isacommand{have}\isamarkupfalse%
\ {\isachardoublequoteopen}{\isacharparenleft}{\kern0pt}{\isadigit{1}}\isactrlsub m\ {\isacharparenleft}{\kern0pt}{\isadigit{2}}{\isacharcircum}{\kern0pt}n{\isacharparenright}{\kern0pt}{\isacharparenright}{\kern0pt}\ {\isasymin}\ carrier{\isacharunderscore}{\kern0pt}mat\ {\isacharparenleft}{\kern0pt}{\isadigit{2}}{\isacharcircum}{\kern0pt}n{\isacharparenright}{\kern0pt}\ {\isacharparenleft}{\kern0pt}{\isadigit{2}}{\isacharcircum}{\kern0pt}n{\isacharparenright}{\kern0pt}{\isachardoublequoteclose}\ \isacommand{by}\isamarkupfalse%
\ simp\isanewline
\ \ \ \ \ \ \ \ \isacommand{moreover}\isamarkupfalse%
\ \isacommand{have}\isamarkupfalse%
\ {\isachardoublequoteopen}SWAP\ {\isasymin}\ carrier{\isacharunderscore}{\kern0pt}mat\ {\isadigit{4}}\ {\isadigit{4}}{\isachardoublequoteclose}\ \isacommand{using}\isamarkupfalse%
\ SWAP{\isacharunderscore}{\kern0pt}carrier{\isacharunderscore}{\kern0pt}mat\ \isacommand{by}\isamarkupfalse%
\ simp\isanewline
\ \ \ \ \ \ \ \ \isacommand{ultimately}\isamarkupfalse%
\ \isacommand{show}\isamarkupfalse%
\ {\isacharquery}{\kern0pt}thesis\ \isacommand{using}\isamarkupfalse%
\ tensor{\isacharunderscore}{\kern0pt}carrier{\isacharunderscore}{\kern0pt}mat\isanewline
\ \ \ \ \ \ \ \ \ \ \isacommand{by}\isamarkupfalse%
\ {\isacharparenleft}{\kern0pt}smt\ {\isacharparenleft}{\kern0pt}verit{\isacharcomma}{\kern0pt}\ ccfv{\isacharunderscore}{\kern0pt}threshold{\isacharparenright}{\kern0pt}\ mult{\isacharunderscore}{\kern0pt}numeral{\isacharunderscore}{\kern0pt}left{\isacharunderscore}{\kern0pt}semiring{\isacharunderscore}{\kern0pt}numeral\ num{\isacharunderscore}{\kern0pt}double\ \isanewline
\ \ \ \ \ \ \ \ \ \ \ \ \ \ numeral{\isacharunderscore}{\kern0pt}times{\isacharunderscore}{\kern0pt}numeral\ power{\isacharunderscore}{\kern0pt}Suc\ power{\isacharunderscore}{\kern0pt}commuting{\isacharunderscore}{\kern0pt}commutes{\isacharparenright}{\kern0pt}\isanewline
\ \ \ \ \ \ \isacommand{qed}\isamarkupfalse%
\isanewline
\ \ \ \ \isacommand{next}\isamarkupfalse%
\isanewline
\ \ \ \ \ \ \isacommand{show}\isamarkupfalse%
\ {\isachardoublequoteopen}SWAP{\isacharunderscore}{\kern0pt}down\ {\isacharparenleft}{\kern0pt}Suc\ n{\isacharparenright}{\kern0pt}\ {\isasymOtimes}\ {\isadigit{1}}\isactrlsub m\ {\isadigit{2}}\ {\isasymin}\ carrier{\isacharunderscore}{\kern0pt}mat\ {\isacharparenleft}{\kern0pt}{\isadigit{2}}\ {\isacharcircum}{\kern0pt}\ Suc\ {\isacharparenleft}{\kern0pt}Suc\ n{\isacharparenright}{\kern0pt}{\isacharparenright}{\kern0pt}\ {\isacharparenleft}{\kern0pt}{\isadigit{2}}\ {\isacharcircum}{\kern0pt}\ Suc\ {\isacharparenleft}{\kern0pt}Suc\ n{\isacharparenright}{\kern0pt}{\isacharparenright}{\kern0pt}{\isachardoublequoteclose}\isanewline
\ \ \ \ \ \ \isacommand{proof}\isamarkupfalse%
\ {\isacharminus}{\kern0pt}\isanewline
\ \ \ \ \ \ \ \ \isacommand{have}\isamarkupfalse%
\ {\isachardoublequoteopen}SWAP{\isacharunderscore}{\kern0pt}down\ {\isacharparenleft}{\kern0pt}Suc\ n{\isacharparenright}{\kern0pt}\ {\isasymin}\ carrier{\isacharunderscore}{\kern0pt}mat\ {\isacharparenleft}{\kern0pt}{\isadigit{2}}{\isacharcircum}{\kern0pt}{\isacharparenleft}{\kern0pt}Suc\ n{\isacharparenright}{\kern0pt}{\isacharparenright}{\kern0pt}\ {\isacharparenleft}{\kern0pt}{\isadigit{2}}{\isacharcircum}{\kern0pt}{\isacharparenleft}{\kern0pt}Suc\ n{\isacharparenright}{\kern0pt}{\isacharparenright}{\kern0pt}{\isachardoublequoteclose}\ \isacommand{using}\isamarkupfalse%
\ SWAP{\isacharunderscore}{\kern0pt}down{\isacharunderscore}{\kern0pt}carrier{\isacharunderscore}{\kern0pt}mat\isanewline
\ \ \ \ \ \ \ \ \ \ \isacommand{by}\isamarkupfalse%
\ blast\isanewline
\ \ \ \ \ \ \ \ \isacommand{moreover}\isamarkupfalse%
\ \isacommand{have}\isamarkupfalse%
\ {\isachardoublequoteopen}{\isadigit{1}}\isactrlsub m\ {\isadigit{2}}\ {\isasymin}\ carrier{\isacharunderscore}{\kern0pt}mat\ {\isadigit{2}}\ {\isadigit{2}}{\isachardoublequoteclose}\ \isacommand{by}\isamarkupfalse%
\ simp\isanewline
\ \ \ \ \ \ \ \ \isacommand{ultimately}\isamarkupfalse%
\ \isacommand{show}\isamarkupfalse%
\ {\isacharquery}{\kern0pt}thesis\ \isacommand{using}\isamarkupfalse%
\ tensor{\isacharunderscore}{\kern0pt}carrier{\isacharunderscore}{\kern0pt}mat\ \isacommand{by}\isamarkupfalse%
\ auto\isanewline
\ \ \ \ \ \ \isacommand{qed}\isamarkupfalse%
\isanewline
\ \ \ \ \isacommand{next}\isamarkupfalse%
\isanewline
\ \ \ \ \ \ \isacommand{show}\isamarkupfalse%
\ {\isachardoublequoteopen}kron\ f\ {\isacharparenleft}{\kern0pt}x\ {\isacharhash}{\kern0pt}\ xs\ {\isacharat}{\kern0pt}\ {\isacharbrackleft}{\kern0pt}a{\isacharbrackright}{\kern0pt}{\isacharparenright}{\kern0pt}\ {\isasymin}\ carrier{\isacharunderscore}{\kern0pt}mat\ {\isacharparenleft}{\kern0pt}{\isadigit{2}}\ {\isacharcircum}{\kern0pt}\ Suc\ {\isacharparenleft}{\kern0pt}Suc\ n{\isacharparenright}{\kern0pt}{\isacharparenright}{\kern0pt}\ {\isadigit{1}}{\isachardoublequoteclose}\ \isacommand{using}\isamarkupfalse%
\ kron{\isacharunderscore}{\kern0pt}carrier{\isacharunderscore}{\kern0pt}mat\isanewline
\ \ \ \ \ \ \ \ \isacommand{by}\isamarkupfalse%
\ {\isacharparenleft}{\kern0pt}metis\ assms\ length{\isacharunderscore}{\kern0pt}Cons\ length{\isacharunderscore}{\kern0pt}append{\isacharunderscore}{\kern0pt}singleton\ n{\isacharunderscore}{\kern0pt}def{\isacharparenright}{\kern0pt}\isanewline
\ \ \ \ \isacommand{qed}\isamarkupfalse%
\isanewline
\ \ \ \ \isacommand{also}\isamarkupfalse%
\ \isacommand{have}\isamarkupfalse%
\ {\isachardoublequoteopen}{\isasymdots}\ {\isacharequal}{\kern0pt}\ {\isacharparenleft}{\kern0pt}{\isacharparenleft}{\kern0pt}{\isadigit{1}}\isactrlsub m\ {\isacharparenleft}{\kern0pt}{\isadigit{2}}{\isacharcircum}{\kern0pt}n{\isacharparenright}{\kern0pt}{\isacharparenright}{\kern0pt}\ {\isasymOtimes}\ SWAP{\isacharparenright}{\kern0pt}\ {\isacharasterisk}{\kern0pt}\ {\isacharparenleft}{\kern0pt}{\isacharparenleft}{\kern0pt}{\isacharparenleft}{\kern0pt}SWAP{\isacharunderscore}{\kern0pt}down\ {\isacharparenleft}{\kern0pt}Suc\ n{\isacharparenright}{\kern0pt}{\isacharparenright}{\kern0pt}\ {\isasymOtimes}\ {\isacharparenleft}{\kern0pt}{\isadigit{1}}\isactrlsub m\ {\isadigit{2}}{\isacharparenright}{\kern0pt}{\isacharparenright}{\kern0pt}\ {\isacharasterisk}{\kern0pt}\ \isanewline
\ \ \ \ \ \ \ \ \ \ \ \ \ \ \ \ \ \ \ \ {\isacharparenleft}{\kern0pt}kron\ f\ {\isacharparenleft}{\kern0pt}x{\isacharhash}{\kern0pt}xs{\isacharparenright}{\kern0pt}\ {\isasymOtimes}\ f\ a{\isacharparenright}{\kern0pt}{\isacharparenright}{\kern0pt}{\isachardoublequoteclose}\isanewline
\ \ \ \ \ \ \isacommand{using}\isamarkupfalse%
\ kron{\isachardot}{\kern0pt}simps\ \isacommand{by}\isamarkupfalse%
\ {\isacharparenleft}{\kern0pt}metis\ append{\isacharunderscore}{\kern0pt}Cons\ kron{\isacharunderscore}{\kern0pt}cons{\isacharunderscore}{\kern0pt}right{\isacharparenright}{\kern0pt}\isanewline
\ \ \ \ \isacommand{also}\isamarkupfalse%
\ \isacommand{have}\isamarkupfalse%
\ {\isachardoublequoteopen}{\isasymdots}\ {\isacharequal}{\kern0pt}\ {\isacharparenleft}{\kern0pt}{\isacharparenleft}{\kern0pt}{\isadigit{1}}\isactrlsub m\ {\isacharparenleft}{\kern0pt}{\isadigit{2}}{\isacharcircum}{\kern0pt}n{\isacharparenright}{\kern0pt}{\isacharparenright}{\kern0pt}\ {\isasymOtimes}\ SWAP{\isacharparenright}{\kern0pt}\ {\isacharasterisk}{\kern0pt}\ {\isacharparenleft}{\kern0pt}{\isacharparenleft}{\kern0pt}{\isacharparenleft}{\kern0pt}SWAP{\isacharunderscore}{\kern0pt}down\ {\isacharparenleft}{\kern0pt}Suc\ n{\isacharparenright}{\kern0pt}{\isacharparenright}{\kern0pt}{\isacharasterisk}{\kern0pt}{\isacharparenleft}{\kern0pt}kron\ f\ {\isacharparenleft}{\kern0pt}x{\isacharhash}{\kern0pt}xs{\isacharparenright}{\kern0pt}{\isacharparenright}{\kern0pt}{\isacharparenright}{\kern0pt}\ {\isasymOtimes}\isanewline
\ \ \ \ \ \ \ \ \ \ \ \ \ \ \ \ \ \ \ \ \ \ \ \ \ \ \ \ \ \ \ \ \ \ \ \ \ \ \ \ \ \ \ \ {\isacharparenleft}{\kern0pt}{\isadigit{1}}\isactrlsub m\ {\isadigit{2}}{\isacharparenright}{\kern0pt}\ {\isacharasterisk}{\kern0pt}\ {\isacharparenleft}{\kern0pt}f\ a{\isacharparenright}{\kern0pt}{\isacharparenright}{\kern0pt}{\isachardoublequoteclose}\isanewline
\ \ \ \ \isacommand{proof}\isamarkupfalse%
\ {\isacharminus}{\kern0pt}\isanewline
\ \ \ \ \ \ \isacommand{have}\isamarkupfalse%
\ c{\isadigit{1}}{\isacharcolon}{\kern0pt}{\isachardoublequoteopen}dim{\isacharunderscore}{\kern0pt}col\ {\isacharparenleft}{\kern0pt}SWAP{\isacharunderscore}{\kern0pt}down\ {\isacharparenleft}{\kern0pt}Suc\ n{\isacharparenright}{\kern0pt}{\isacharparenright}{\kern0pt}\ {\isacharequal}{\kern0pt}\ {\isadigit{2}}{\isacharcircum}{\kern0pt}{\isacharparenleft}{\kern0pt}Suc\ n{\isacharparenright}{\kern0pt}{\isachardoublequoteclose}\ \isacommand{using}\isamarkupfalse%
\ SWAP{\isacharunderscore}{\kern0pt}down{\isacharunderscore}{\kern0pt}carrier{\isacharunderscore}{\kern0pt}mat\ \isacommand{by}\isamarkupfalse%
\ blast\isanewline
\ \ \ \ \ \ \isacommand{hence}\isamarkupfalse%
\ a{\isadigit{3}}{\isacharcolon}{\kern0pt}\ {\isachardoublequoteopen}dim{\isacharunderscore}{\kern0pt}col\ {\isacharparenleft}{\kern0pt}SWAP{\isacharunderscore}{\kern0pt}down\ {\isacharparenleft}{\kern0pt}Suc\ n{\isacharparenright}{\kern0pt}{\isacharparenright}{\kern0pt}\ {\isachargreater}{\kern0pt}\ {\isadigit{0}}{\isachardoublequoteclose}\ \isacommand{by}\isamarkupfalse%
\ simp\isanewline
\ \ \ \ \ \ \isacommand{have}\isamarkupfalse%
\ r{\isadigit{2}}{\isacharcolon}{\kern0pt}{\isachardoublequoteopen}dim{\isacharunderscore}{\kern0pt}row\ {\isacharparenleft}{\kern0pt}kron\ f\ {\isacharparenleft}{\kern0pt}x{\isacharhash}{\kern0pt}xs{\isacharparenright}{\kern0pt}{\isacharparenright}{\kern0pt}\ {\isacharequal}{\kern0pt}\ {\isadigit{2}}{\isacharcircum}{\kern0pt}{\isacharparenleft}{\kern0pt}Suc\ n{\isacharparenright}{\kern0pt}{\isachardoublequoteclose}\ \isacommand{using}\isamarkupfalse%
\ kron{\isacharunderscore}{\kern0pt}carrier{\isacharunderscore}{\kern0pt}mat\ assms\ n{\isacharunderscore}{\kern0pt}def\ \isacommand{by}\isamarkupfalse%
\ auto\isanewline
\ \ \ \ \ \ \isacommand{hence}\isamarkupfalse%
\ a{\isadigit{4}}{\isacharcolon}{\kern0pt}{\isachardoublequoteopen}dim{\isacharunderscore}{\kern0pt}row\ {\isacharparenleft}{\kern0pt}kron\ f\ {\isacharparenleft}{\kern0pt}x{\isacharhash}{\kern0pt}xs{\isacharparenright}{\kern0pt}{\isacharparenright}{\kern0pt}\ {\isachargreater}{\kern0pt}\ {\isadigit{0}}{\isachardoublequoteclose}\ \isacommand{by}\isamarkupfalse%
\ simp\isanewline
\ \ \ \ \ \ \isacommand{with}\isamarkupfalse%
\ c{\isadigit{1}}\ r{\isadigit{2}}\ \isacommand{have}\isamarkupfalse%
\ a{\isadigit{1}}{\isacharcolon}{\kern0pt}{\isachardoublequoteopen}dim{\isacharunderscore}{\kern0pt}col\ {\isacharparenleft}{\kern0pt}SWAP{\isacharunderscore}{\kern0pt}down\ {\isacharparenleft}{\kern0pt}Suc\ n{\isacharparenright}{\kern0pt}{\isacharparenright}{\kern0pt}\ {\isacharequal}{\kern0pt}\ dim{\isacharunderscore}{\kern0pt}row\ {\isacharparenleft}{\kern0pt}kron\ f\ {\isacharparenleft}{\kern0pt}x{\isacharhash}{\kern0pt}xs{\isacharparenright}{\kern0pt}{\isacharparenright}{\kern0pt}{\isachardoublequoteclose}\ \isacommand{by}\isamarkupfalse%
\ simp\isanewline
\ \ \ \ \ \ \isacommand{have}\isamarkupfalse%
\ c{\isadigit{3}}{\isacharcolon}{\kern0pt}{\isachardoublequoteopen}dim{\isacharunderscore}{\kern0pt}col\ {\isacharparenleft}{\kern0pt}{\isadigit{1}}\isactrlsub m\ {\isadigit{2}}{\isacharparenright}{\kern0pt}\ {\isacharequal}{\kern0pt}\ {\isadigit{2}}{\isachardoublequoteclose}\ \isacommand{by}\isamarkupfalse%
\ simp\isanewline
\ \ \ \ \ \ \isacommand{hence}\isamarkupfalse%
\ a{\isadigit{5}}{\isacharcolon}{\kern0pt}{\isachardoublequoteopen}dim{\isacharunderscore}{\kern0pt}col\ {\isacharparenleft}{\kern0pt}{\isadigit{1}}\isactrlsub m\ {\isadigit{2}}{\isacharparenright}{\kern0pt}\ {\isachargreater}{\kern0pt}\ {\isadigit{0}}{\isachardoublequoteclose}\ \isacommand{by}\isamarkupfalse%
\ simp\isanewline
\ \ \ \ \ \ \isacommand{have}\isamarkupfalse%
\ r{\isadigit{4}}{\isacharcolon}{\kern0pt}{\isachardoublequoteopen}dim{\isacharunderscore}{\kern0pt}row\ {\isacharparenleft}{\kern0pt}f\ a{\isacharparenright}{\kern0pt}\ {\isacharequal}{\kern0pt}\ {\isadigit{2}}{\isachardoublequoteclose}\ \isacommand{using}\isamarkupfalse%
\ assms\ \isacommand{by}\isamarkupfalse%
\ simp\isanewline
\ \ \ \ \ \ \isacommand{hence}\isamarkupfalse%
\ a{\isadigit{6}}{\isacharcolon}{\kern0pt}{\isachardoublequoteopen}dim{\isacharunderscore}{\kern0pt}row\ {\isacharparenleft}{\kern0pt}f\ a{\isacharparenright}{\kern0pt}\ {\isachargreater}{\kern0pt}\ {\isadigit{0}}{\isachardoublequoteclose}\ \isacommand{by}\isamarkupfalse%
\ simp\isanewline
\ \ \ \ \ \ \isacommand{with}\isamarkupfalse%
\ c{\isadigit{3}}\ r{\isadigit{4}}\ \isacommand{have}\isamarkupfalse%
\ a{\isadigit{2}}{\isacharcolon}{\kern0pt}{\isachardoublequoteopen}dim{\isacharunderscore}{\kern0pt}col\ {\isacharparenleft}{\kern0pt}{\isadigit{1}}\isactrlsub m\ {\isadigit{2}}{\isacharparenright}{\kern0pt}\ {\isacharequal}{\kern0pt}\ dim{\isacharunderscore}{\kern0pt}row\ {\isacharparenleft}{\kern0pt}f\ a{\isacharparenright}{\kern0pt}{\isachardoublequoteclose}\ \isacommand{by}\isamarkupfalse%
\ simp\isanewline
\ \ \ \ \ \ \isacommand{have}\isamarkupfalse%
\ {\isachardoublequoteopen}{\isacharparenleft}{\kern0pt}{\isacharparenleft}{\kern0pt}{\isacharparenleft}{\kern0pt}SWAP{\isacharunderscore}{\kern0pt}down\ {\isacharparenleft}{\kern0pt}Suc\ n{\isacharparenright}{\kern0pt}{\isacharparenright}{\kern0pt}\ {\isasymOtimes}\ {\isacharparenleft}{\kern0pt}{\isadigit{1}}\isactrlsub m\ {\isadigit{2}}{\isacharparenright}{\kern0pt}{\isacharparenright}{\kern0pt}\ {\isacharasterisk}{\kern0pt}\ {\isacharparenleft}{\kern0pt}kron\ f\ {\isacharparenleft}{\kern0pt}x{\isacharhash}{\kern0pt}xs{\isacharparenright}{\kern0pt}\ {\isasymOtimes}\ f\ a{\isacharparenright}{\kern0pt}{\isacharparenright}{\kern0pt}\ {\isacharequal}{\kern0pt}\ \isanewline
\ \ \ \ \ \ \ \ \ \ \ \ {\isacharparenleft}{\kern0pt}{\isacharparenleft}{\kern0pt}{\isacharparenleft}{\kern0pt}SWAP{\isacharunderscore}{\kern0pt}down\ {\isacharparenleft}{\kern0pt}Suc\ n{\isacharparenright}{\kern0pt}{\isacharparenright}{\kern0pt}{\isacharasterisk}{\kern0pt}{\isacharparenleft}{\kern0pt}kron\ f\ {\isacharparenleft}{\kern0pt}x{\isacharhash}{\kern0pt}xs{\isacharparenright}{\kern0pt}{\isacharparenright}{\kern0pt}{\isacharparenright}{\kern0pt}\ {\isasymOtimes}\ {\isacharparenleft}{\kern0pt}{\isadigit{1}}\isactrlsub m\ {\isadigit{2}}{\isacharparenright}{\kern0pt}\ {\isacharasterisk}{\kern0pt}\ {\isacharparenleft}{\kern0pt}f\ a{\isacharparenright}{\kern0pt}{\isacharparenright}{\kern0pt}{\isachardoublequoteclose}\isanewline
\ \ \ \ \ \ \ \ \isacommand{using}\isamarkupfalse%
\ a{\isadigit{1}}\ a{\isadigit{2}}\ a{\isadigit{3}}\ a{\isadigit{4}}\ a{\isadigit{5}}\ a{\isadigit{6}}\isanewline
\ \ \ \ \ \ \ \ \isacommand{by}\isamarkupfalse%
\ {\isacharparenleft}{\kern0pt}metis\ assms\ carrier{\isacharunderscore}{\kern0pt}matD{\isacharparenleft}{\kern0pt}{\isadigit{2}}{\isacharparenright}{\kern0pt}\ gr{\isadigit{0}}I\ kron{\isacharunderscore}{\kern0pt}carrier{\isacharunderscore}{\kern0pt}mat\ mult{\isacharunderscore}{\kern0pt}distr{\isacharunderscore}{\kern0pt}tensor\ zero{\isacharunderscore}{\kern0pt}neq{\isacharunderscore}{\kern0pt}one{\isacharparenright}{\kern0pt}\isanewline
\ \ \ \ \ \ \isacommand{thus}\isamarkupfalse%
\ {\isacharquery}{\kern0pt}thesis\ \isacommand{by}\isamarkupfalse%
\ simp\isanewline
\ \ \ \ \isacommand{qed}\isamarkupfalse%
\isanewline
\ \ \ \ \isacommand{also}\isamarkupfalse%
\ \isacommand{have}\isamarkupfalse%
\ {\isachardoublequoteopen}{\isasymdots}\ {\isacharequal}{\kern0pt}\ {\isacharparenleft}{\kern0pt}{\isacharparenleft}{\kern0pt}{\isadigit{1}}\isactrlsub m\ {\isacharparenleft}{\kern0pt}{\isadigit{2}}{\isacharcircum}{\kern0pt}n{\isacharparenright}{\kern0pt}{\isacharparenright}{\kern0pt}\ {\isasymOtimes}\ SWAP{\isacharparenright}{\kern0pt}\ {\isacharasterisk}{\kern0pt}\ {\isacharparenleft}{\kern0pt}kron\ f\ xs\ {\isasymOtimes}\ f\ x\ {\isasymOtimes}\ f\ a{\isacharparenright}{\kern0pt}{\isachardoublequoteclose}\isanewline
\ \ \ \ \ \ \isacommand{using}\isamarkupfalse%
\ HI\ \isacommand{by}\isamarkupfalse%
\ {\isacharparenleft}{\kern0pt}simp\ add{\isacharcolon}{\kern0pt}\ assms\ n{\isacharunderscore}{\kern0pt}def{\isacharparenright}{\kern0pt}\isanewline
\ \ \ \ \isacommand{also}\isamarkupfalse%
\ \isacommand{have}\isamarkupfalse%
\ {\isachardoublequoteopen}{\isasymdots}\ {\isacharequal}{\kern0pt}\ {\isacharparenleft}{\kern0pt}{\isacharparenleft}{\kern0pt}{\isadigit{1}}\isactrlsub m\ {\isacharparenleft}{\kern0pt}{\isadigit{2}}{\isacharcircum}{\kern0pt}n{\isacharparenright}{\kern0pt}{\isacharparenright}{\kern0pt}\ {\isasymOtimes}\ SWAP{\isacharparenright}{\kern0pt}\ {\isacharasterisk}{\kern0pt}\ {\isacharparenleft}{\kern0pt}kron\ f\ xs\ {\isasymOtimes}\ {\isacharparenleft}{\kern0pt}f\ x\ {\isasymOtimes}\ f\ a{\isacharparenright}{\kern0pt}{\isacharparenright}{\kern0pt}{\isachardoublequoteclose}\isanewline
\ \ \ \ \ \ \isacommand{using}\isamarkupfalse%
\ tensor{\isacharunderscore}{\kern0pt}mat{\isacharunderscore}{\kern0pt}is{\isacharunderscore}{\kern0pt}assoc\ \isacommand{by}\isamarkupfalse%
\ auto\ \isanewline
\ \ \ \ \isacommand{also}\isamarkupfalse%
\ \isacommand{have}\isamarkupfalse%
\ {\isachardoublequoteopen}{\isasymdots}\ {\isacharequal}{\kern0pt}\ {\isacharparenleft}{\kern0pt}{\isacharparenleft}{\kern0pt}{\isadigit{1}}\isactrlsub m\ {\isacharparenleft}{\kern0pt}{\isadigit{2}}{\isacharcircum}{\kern0pt}n{\isacharparenright}{\kern0pt}{\isacharparenright}{\kern0pt}\ {\isacharasterisk}{\kern0pt}\ {\isacharparenleft}{\kern0pt}kron\ f\ xs{\isacharparenright}{\kern0pt}{\isacharparenright}{\kern0pt}\ {\isasymOtimes}\ {\isacharparenleft}{\kern0pt}SWAP\ {\isacharasterisk}{\kern0pt}\ {\isacharparenleft}{\kern0pt}f\ x\ {\isasymOtimes}\ f\ a{\isacharparenright}{\kern0pt}{\isacharparenright}{\kern0pt}{\isachardoublequoteclose}\isanewline
\ \ \ \ \ \ \isacommand{using}\isamarkupfalse%
\ mult{\isacharunderscore}{\kern0pt}distr{\isacharunderscore}{\kern0pt}tensor\ \isanewline
\ \ \ \ \ \ \isacommand{by}\isamarkupfalse%
\ {\isacharparenleft}{\kern0pt}smt\ {\isacharparenleft}{\kern0pt}verit{\isacharcomma}{\kern0pt}\ del{\isacharunderscore}{\kern0pt}insts{\isacharparenright}{\kern0pt}\ SWAP{\isacharunderscore}{\kern0pt}ncols\ assms\ carrier{\isacharunderscore}{\kern0pt}matD{\isacharparenleft}{\kern0pt}{\isadigit{2}}{\isacharparenright}{\kern0pt}\ dim{\isacharunderscore}{\kern0pt}col{\isacharunderscore}{\kern0pt}tensor{\isacharunderscore}{\kern0pt}mat\ \isanewline
\ \ \ \ \ \ \ \ \ \ dim{\isacharunderscore}{\kern0pt}row{\isacharunderscore}{\kern0pt}tensor{\isacharunderscore}{\kern0pt}mat\ index{\isacharunderscore}{\kern0pt}mult{\isacharunderscore}{\kern0pt}mat{\isacharparenleft}{\kern0pt}{\isadigit{2}}{\isacharparenright}{\kern0pt}\ index{\isacharunderscore}{\kern0pt}one{\isacharunderscore}{\kern0pt}mat{\isacharparenleft}{\kern0pt}{\isadigit{2}}{\isacharparenright}{\kern0pt}\ index{\isacharunderscore}{\kern0pt}one{\isacharunderscore}{\kern0pt}mat{\isacharparenleft}{\kern0pt}{\isadigit{3}}{\isacharparenright}{\kern0pt}\ kron{\isacharunderscore}{\kern0pt}carrier{\isacharunderscore}{\kern0pt}mat\isanewline
\ \ \ \ \ \ \ \ \ \ left{\isacharunderscore}{\kern0pt}mult{\isacharunderscore}{\kern0pt}one{\isacharunderscore}{\kern0pt}mat\ n{\isacharunderscore}{\kern0pt}def\ numeral{\isacharunderscore}{\kern0pt}One\ numeral{\isacharunderscore}{\kern0pt}times{\isacharunderscore}{\kern0pt}numeral\ semiring{\isacharunderscore}{\kern0pt}norm{\isacharparenleft}{\kern0pt}{\isadigit{1}}{\isadigit{1}}{\isacharparenright}{\kern0pt}\ \isanewline
\ \ \ \ \ \ \ \ \ \ semiring{\isacharunderscore}{\kern0pt}norm{\isacharparenleft}{\kern0pt}{\isadigit{1}}{\isadigit{3}}{\isacharparenright}{\kern0pt}\ zero{\isacharunderscore}{\kern0pt}less{\isacharunderscore}{\kern0pt}numeral\ zero{\isacharunderscore}{\kern0pt}less{\isacharunderscore}{\kern0pt}power{\isacharparenright}{\kern0pt}\isanewline
\ \ \ \ \isacommand{also}\isamarkupfalse%
\ \isacommand{have}\isamarkupfalse%
\ {\isachardoublequoteopen}{\isasymdots}\ {\isacharequal}{\kern0pt}\ kron\ f\ xs\ {\isasymOtimes}\ f\ a\ {\isasymOtimes}\ f\ x{\isachardoublequoteclose}\ \isacommand{using}\isamarkupfalse%
\ SWAP{\isacharunderscore}{\kern0pt}tensor\isanewline
\ \ \ \ \ \ \isacommand{by}\isamarkupfalse%
\ {\isacharparenleft}{\kern0pt}metis\ assms\ carrier{\isacharunderscore}{\kern0pt}matI\ kron{\isacharunderscore}{\kern0pt}carrier{\isacharunderscore}{\kern0pt}mat\ left{\isacharunderscore}{\kern0pt}mult{\isacharunderscore}{\kern0pt}one{\isacharunderscore}{\kern0pt}mat\ n{\isacharunderscore}{\kern0pt}def\ tensor{\isacharunderscore}{\kern0pt}mat{\isacharunderscore}{\kern0pt}is{\isacharunderscore}{\kern0pt}assoc{\isacharparenright}{\kern0pt}\isanewline
\ \ \ \ \isacommand{also}\isamarkupfalse%
\ \isacommand{have}\isamarkupfalse%
\ {\isachardoublequoteopen}{\isasymdots}\ {\isacharequal}{\kern0pt}\ kron\ f\ {\isacharparenleft}{\kern0pt}xs{\isacharat}{\kern0pt}{\isacharbrackleft}{\kern0pt}a{\isacharbrackright}{\kern0pt}{\isacharparenright}{\kern0pt}\ {\isasymOtimes}\ f\ x{\isachardoublequoteclose}\ \isacommand{using}\isamarkupfalse%
\ kron{\isachardot}{\kern0pt}simps\ kron{\isacharunderscore}{\kern0pt}cons{\isacharunderscore}{\kern0pt}right\ \isacommand{by}\isamarkupfalse%
\ presburger\ \isanewline
\ \ \ \ \isacommand{finally}\isamarkupfalse%
\ \isacommand{show}\isamarkupfalse%
\ {\isacharquery}{\kern0pt}thesis\ \isacommand{by}\isamarkupfalse%
\ this\isanewline
\ \ \isacommand{qed}\isamarkupfalse%
\isanewline
\isacommand{qed}\isamarkupfalse%
%
\endisatagproof
{\isafoldproof}%
%
\isadelimproof
\isanewline
%
\endisadelimproof
\isanewline
\isanewline
\isacommand{lemma}\isamarkupfalse%
\ SWAP{\isacharunderscore}{\kern0pt}down{\isacharunderscore}{\kern0pt}kron{\isacharunderscore}{\kern0pt}map{\isacharunderscore}{\kern0pt}rev{\isacharcolon}{\kern0pt}\isanewline
\ \ \isakeyword{assumes}\ {\isachardoublequoteopen}{\isasymforall}m{\isachardot}{\kern0pt}\ dim{\isacharunderscore}{\kern0pt}row\ {\isacharparenleft}{\kern0pt}f\ m{\isacharparenright}{\kern0pt}\ {\isacharequal}{\kern0pt}\ {\isadigit{2}}\ {\isasymand}\ dim{\isacharunderscore}{\kern0pt}col\ {\isacharparenleft}{\kern0pt}f\ m{\isacharparenright}{\kern0pt}\ {\isacharequal}{\kern0pt}\ {\isadigit{1}}{\isachardoublequoteclose}\isanewline
\ \ \isakeyword{shows}\ {\isachardoublequoteopen}{\isacharparenleft}{\kern0pt}SWAP{\isacharunderscore}{\kern0pt}down\ {\isacharparenleft}{\kern0pt}Suc\ k{\isacharparenright}{\kern0pt}{\isacharparenright}{\kern0pt}\ {\isacharasterisk}{\kern0pt}\ \isanewline
\ \ \ \ \ \ \ \ kron\ f\ {\isacharparenleft}{\kern0pt}map\ nat\ {\isacharparenleft}{\kern0pt}rev\ {\isacharbrackleft}{\kern0pt}{\isadigit{1}}{\isachardot}{\kern0pt}{\isachardot}{\kern0pt}int\ {\isacharparenleft}{\kern0pt}Suc\ k{\isacharparenright}{\kern0pt}{\isacharbrackright}{\kern0pt}{\isacharparenright}{\kern0pt}{\isacharparenright}{\kern0pt}\ {\isacharequal}{\kern0pt}\ \isanewline
\ \ \ \ \ \ \ \ \ {\isacharparenleft}{\kern0pt}kron\ f\ {\isacharparenleft}{\kern0pt}map\ nat\ {\isacharparenleft}{\kern0pt}rev\ {\isacharbrackleft}{\kern0pt}{\isadigit{1}}{\isachardot}{\kern0pt}{\isachardot}{\kern0pt}int\ k{\isacharbrackright}{\kern0pt}{\isacharparenright}{\kern0pt}{\isacharparenright}{\kern0pt}\ {\isasymOtimes}\ {\isacharparenleft}{\kern0pt}f\ {\isacharparenleft}{\kern0pt}Suc\ k{\isacharparenright}{\kern0pt}{\isacharparenright}{\kern0pt}{\isacharparenright}{\kern0pt}{\isachardoublequoteclose}\isanewline
%
\isadelimproof
%
\endisadelimproof
%
\isatagproof
\isacommand{proof}\isamarkupfalse%
\ {\isacharminus}{\kern0pt}\isanewline
\ \ \isacommand{have}\isamarkupfalse%
\ {\isachardoublequoteopen}rev\ {\isacharbrackleft}{\kern0pt}{\isadigit{1}}{\isachardot}{\kern0pt}{\isachardot}{\kern0pt}int\ {\isacharparenleft}{\kern0pt}Suc\ k{\isacharparenright}{\kern0pt}{\isacharbrackright}{\kern0pt}\ {\isacharequal}{\kern0pt}\ int\ {\isacharparenleft}{\kern0pt}Suc\ k{\isacharparenright}{\kern0pt}\ {\isacharhash}{\kern0pt}\ rev\ {\isacharbrackleft}{\kern0pt}{\isadigit{1}}{\isachardot}{\kern0pt}{\isachardot}{\kern0pt}int\ k{\isacharbrackright}{\kern0pt}{\isachardoublequoteclose}\ \isacommand{using}\isamarkupfalse%
\ rev{\isacharunderscore}{\kern0pt}append\ upto{\isacharunderscore}{\kern0pt}rec{\isadigit{2}}\ \isacommand{by}\isamarkupfalse%
\ simp\isanewline
\ \ \isacommand{hence}\isamarkupfalse%
\ {\isadigit{1}}{\isacharcolon}{\kern0pt}{\isachardoublequoteopen}map\ nat\ {\isacharparenleft}{\kern0pt}rev\ {\isacharbrackleft}{\kern0pt}{\isadigit{1}}{\isachardot}{\kern0pt}{\isachardot}{\kern0pt}int\ {\isacharparenleft}{\kern0pt}Suc\ k{\isacharparenright}{\kern0pt}{\isacharbrackright}{\kern0pt}{\isacharparenright}{\kern0pt}\ {\isacharequal}{\kern0pt}\ Suc\ k\ {\isacharhash}{\kern0pt}\ {\isacharparenleft}{\kern0pt}map\ nat\ {\isacharparenleft}{\kern0pt}rev\ {\isacharbrackleft}{\kern0pt}{\isadigit{1}}{\isachardot}{\kern0pt}{\isachardot}{\kern0pt}\ int\ k{\isacharbrackright}{\kern0pt}{\isacharparenright}{\kern0pt}{\isacharparenright}{\kern0pt}{\isachardoublequoteclose}\isanewline
\ \ \ \ \isacommand{using}\isamarkupfalse%
\ list{\isachardot}{\kern0pt}map{\isacharparenleft}{\kern0pt}{\isadigit{2}}{\isacharparenright}{\kern0pt}\ \isacommand{by}\isamarkupfalse%
\ simp\isanewline
\ \ \isacommand{define}\isamarkupfalse%
\ x\ xs\ \isakeyword{where}\ {\isachardoublequoteopen}x\ {\isacharequal}{\kern0pt}\ Suc\ k{\isachardoublequoteclose}\ \isakeyword{and}\ {\isachardoublequoteopen}xs\ {\isacharequal}{\kern0pt}\ {\isacharparenleft}{\kern0pt}map\ nat\ {\isacharparenleft}{\kern0pt}rev\ {\isacharbrackleft}{\kern0pt}{\isadigit{1}}{\isachardot}{\kern0pt}{\isachardot}{\kern0pt}\ int\ k{\isacharbrackright}{\kern0pt}{\isacharparenright}{\kern0pt}{\isacharparenright}{\kern0pt}{\isachardoublequoteclose}\isanewline
\ \ \isacommand{have}\isamarkupfalse%
\ {\isachardoublequoteopen}length\ xs\ {\isacharequal}{\kern0pt}\ k{\isachardoublequoteclose}\ \isacommand{using}\isamarkupfalse%
\ xs{\isacharunderscore}{\kern0pt}def\ \isacommand{by}\isamarkupfalse%
\ simp\isanewline
\ \ \isacommand{hence}\isamarkupfalse%
\ {\isadigit{2}}{\isacharcolon}{\kern0pt}{\isachardoublequoteopen}length\ {\isacharparenleft}{\kern0pt}x{\isacharhash}{\kern0pt}xs{\isacharparenright}{\kern0pt}\ {\isacharequal}{\kern0pt}\ Suc\ k{\isachardoublequoteclose}\ \isacommand{by}\isamarkupfalse%
\ simp\isanewline
\ \ \isacommand{with}\isamarkupfalse%
\ {\isadigit{1}}\ {\isadigit{2}}\ x{\isacharunderscore}{\kern0pt}def\ xs{\isacharunderscore}{\kern0pt}def\ \isacommand{have}\isamarkupfalse%
\ {\isachardoublequoteopen}{\isacharparenleft}{\kern0pt}SWAP{\isacharunderscore}{\kern0pt}down\ {\isacharparenleft}{\kern0pt}Suc\ k{\isacharparenright}{\kern0pt}{\isacharparenright}{\kern0pt}\ {\isacharasterisk}{\kern0pt}\ kron\ f\ {\isacharparenleft}{\kern0pt}map\ nat\ {\isacharparenleft}{\kern0pt}rev\ {\isacharbrackleft}{\kern0pt}{\isadigit{1}}{\isachardot}{\kern0pt}{\isachardot}{\kern0pt}int\ {\isacharparenleft}{\kern0pt}Suc\ k{\isacharparenright}{\kern0pt}{\isacharbrackright}{\kern0pt}{\isacharparenright}{\kern0pt}{\isacharparenright}{\kern0pt}\ {\isacharequal}{\kern0pt}\isanewline
\ \ \ \ \ \ \ \ \ \ \ \ \ \ \ \ \ \ \ \ \ \ \ \ \ \ \ \ \ \ {\isacharparenleft}{\kern0pt}SWAP{\isacharunderscore}{\kern0pt}down\ {\isacharparenleft}{\kern0pt}length\ {\isacharparenleft}{\kern0pt}x{\isacharhash}{\kern0pt}xs{\isacharparenright}{\kern0pt}{\isacharparenright}{\kern0pt}{\isacharparenright}{\kern0pt}\ {\isacharasterisk}{\kern0pt}\ kron\ f\ {\isacharparenleft}{\kern0pt}x{\isacharhash}{\kern0pt}xs{\isacharparenright}{\kern0pt}{\isachardoublequoteclose}\ \isacommand{by}\isamarkupfalse%
\ auto\isanewline
\ \ \isacommand{also}\isamarkupfalse%
\ \isacommand{have}\isamarkupfalse%
\ {\isachardoublequoteopen}{\isasymdots}\ {\isacharequal}{\kern0pt}\ kron\ f\ xs\ {\isasymOtimes}\ f\ x{\isachardoublequoteclose}\ \isacommand{using}\isamarkupfalse%
\ SWAP{\isacharunderscore}{\kern0pt}down{\isacharunderscore}{\kern0pt}kron\ x{\isacharunderscore}{\kern0pt}def\ xs{\isacharunderscore}{\kern0pt}def\ assms\ \isacommand{by}\isamarkupfalse%
\ blast\isanewline
\ \ \isacommand{finally}\isamarkupfalse%
\ \isacommand{show}\isamarkupfalse%
\ {\isacharquery}{\kern0pt}thesis\ \isacommand{using}\isamarkupfalse%
\ x{\isacharunderscore}{\kern0pt}def\ xs{\isacharunderscore}{\kern0pt}def\ \isacommand{by}\isamarkupfalse%
\ simp\isanewline
\isacommand{qed}\isamarkupfalse%
%
\endisatagproof
{\isafoldproof}%
%
\isadelimproof
\isanewline
%
\endisadelimproof
\isanewline
\isanewline
\isacommand{lemma}\isamarkupfalse%
\ reverse{\isacharunderscore}{\kern0pt}qubits{\isacharunderscore}{\kern0pt}kron{\isacharcolon}{\kern0pt}\isanewline
\ \ \isakeyword{assumes}\ {\isachardoublequoteopen}{\isasymforall}m{\isachardot}{\kern0pt}\ dim{\isacharunderscore}{\kern0pt}row\ {\isacharparenleft}{\kern0pt}f\ m{\isacharparenright}{\kern0pt}\ {\isacharequal}{\kern0pt}\ {\isadigit{2}}\ {\isasymand}\ dim{\isacharunderscore}{\kern0pt}col\ {\isacharparenleft}{\kern0pt}f\ m{\isacharparenright}{\kern0pt}\ {\isacharequal}{\kern0pt}\ {\isadigit{1}}{\isachardoublequoteclose}\isanewline
\ \ \isakeyword{shows}\ {\isachardoublequoteopen}{\isacharparenleft}{\kern0pt}reverse{\isacharunderscore}{\kern0pt}qubits\ n{\isacharparenright}{\kern0pt}\ {\isacharasterisk}{\kern0pt}\ {\isacharparenleft}{\kern0pt}kron\ f\ {\isacharparenleft}{\kern0pt}map\ nat\ {\isacharparenleft}{\kern0pt}rev\ {\isacharbrackleft}{\kern0pt}{\isadigit{1}}{\isachardot}{\kern0pt}{\isachardot}{\kern0pt}n{\isacharbrackright}{\kern0pt}{\isacharparenright}{\kern0pt}{\isacharparenright}{\kern0pt}{\isacharparenright}{\kern0pt}\ {\isacharequal}{\kern0pt}\ kron\ f\ {\isacharparenleft}{\kern0pt}map\ nat\ {\isacharbrackleft}{\kern0pt}{\isadigit{1}}{\isachardot}{\kern0pt}{\isachardot}{\kern0pt}n{\isacharbrackright}{\kern0pt}{\isacharparenright}{\kern0pt}{\isachardoublequoteclose}\isanewline
%
\isadelimproof
%
\endisadelimproof
%
\isatagproof
\isacommand{proof}\isamarkupfalse%
\ {\isacharparenleft}{\kern0pt}induct\ n\ rule{\isacharcolon}{\kern0pt}\ reverse{\isacharunderscore}{\kern0pt}qubits{\isachardot}{\kern0pt}induct{\isacharparenright}{\kern0pt}\isanewline
\ \ \isacommand{case}\isamarkupfalse%
\ {\isadigit{1}}\isanewline
\ \ \isacommand{then}\isamarkupfalse%
\ \isacommand{show}\isamarkupfalse%
\ {\isacharquery}{\kern0pt}case\ \isacommand{by}\isamarkupfalse%
\ auto\isanewline
\isacommand{next}\isamarkupfalse%
\isanewline
\ \ \isacommand{case}\isamarkupfalse%
\ {\isadigit{2}}\isanewline
\ \ \isacommand{then}\isamarkupfalse%
\ \isacommand{show}\isamarkupfalse%
\ {\isacharquery}{\kern0pt}case\isanewline
\ \ \isacommand{proof}\isamarkupfalse%
\ {\isacharminus}{\kern0pt}\isanewline
\ \ \ \ \isacommand{have}\isamarkupfalse%
\ {\isadigit{1}}{\isacharcolon}{\kern0pt}{\isachardoublequoteopen}rev\ {\isacharbrackleft}{\kern0pt}{\isadigit{1}}{\isacharbrackright}{\kern0pt}\ {\isacharequal}{\kern0pt}\ {\isacharbrackleft}{\kern0pt}{\isadigit{1}}{\isacharbrackright}{\kern0pt}{\isachardoublequoteclose}\ \isacommand{using}\isamarkupfalse%
\ rev{\isacharunderscore}{\kern0pt}def\ \isacommand{by}\isamarkupfalse%
\ auto\isanewline
\ \ \ \ \isacommand{have}\isamarkupfalse%
\ {\isadigit{2}}{\isacharcolon}{\kern0pt}{\isachardoublequoteopen}reverse{\isacharunderscore}{\kern0pt}qubits\ {\isacharparenleft}{\kern0pt}Suc\ {\isadigit{0}}{\isacharparenright}{\kern0pt}\ {\isacharequal}{\kern0pt}\ {\isadigit{1}}\isactrlsub m\ {\isadigit{2}}{\isachardoublequoteclose}\ \isacommand{by}\isamarkupfalse%
\ simp\isanewline
\ \ \ \ \isacommand{have}\isamarkupfalse%
\ {\isadigit{3}}{\isacharcolon}{\kern0pt}{\isachardoublequoteopen}{\isacharparenleft}{\kern0pt}f\ {\isadigit{1}}{\isacharparenright}{\kern0pt}\ {\isasymin}\ carrier{\isacharunderscore}{\kern0pt}mat\ {\isadigit{2}}\ {\isadigit{1}}{\isachardoublequoteclose}\ \isacommand{using}\isamarkupfalse%
\ assms\ carrier{\isacharunderscore}{\kern0pt}mat{\isacharunderscore}{\kern0pt}def\ \isacommand{by}\isamarkupfalse%
\ auto\isanewline
\ \ \ \ \isacommand{have}\isamarkupfalse%
\ {\isadigit{4}}{\isacharcolon}{\kern0pt}{\isachardoublequoteopen}kron\ f\ {\isacharbrackleft}{\kern0pt}{\isadigit{1}}{\isacharbrackright}{\kern0pt}\ {\isacharequal}{\kern0pt}\ {\isacharparenleft}{\kern0pt}f\ {\isadigit{1}}{\isacharparenright}{\kern0pt}{\isachardoublequoteclose}\ \isacommand{using}\isamarkupfalse%
\ kron{\isachardot}{\kern0pt}simps{\isacharparenleft}{\kern0pt}{\isadigit{2}}{\isacharparenright}{\kern0pt}\ \isacommand{by}\isamarkupfalse%
\ auto\isanewline
\ \ \ \ \isacommand{show}\isamarkupfalse%
\ {\isacharquery}{\kern0pt}case\ \isacommand{using}\isamarkupfalse%
\ {\isadigit{1}}\ {\isadigit{2}}\ {\isadigit{3}}\ {\isadigit{4}}\ \isacommand{by}\isamarkupfalse%
\ auto\isanewline
\ \ \isacommand{qed}\isamarkupfalse%
\isanewline
\isacommand{next}\isamarkupfalse%
\isanewline
\ \ \isacommand{case}\isamarkupfalse%
\ {\isadigit{3}}\isanewline
\ \ \isacommand{have}\isamarkupfalse%
\ {\isachardoublequoteopen}reverse{\isacharunderscore}{\kern0pt}qubits\ {\isacharparenleft}{\kern0pt}Suc\ {\isacharparenleft}{\kern0pt}Suc\ {\isadigit{0}}{\isacharparenright}{\kern0pt}{\isacharparenright}{\kern0pt}\ {\isacharasterisk}{\kern0pt}\ kron\ f\ {\isacharparenleft}{\kern0pt}map\ nat\ {\isacharparenleft}{\kern0pt}rev\ {\isacharbrackleft}{\kern0pt}{\isadigit{1}}{\isachardot}{\kern0pt}{\isachardot}{\kern0pt}int\ {\isacharparenleft}{\kern0pt}Suc\ {\isacharparenleft}{\kern0pt}Suc\ {\isadigit{0}}{\isacharparenright}{\kern0pt}{\isacharparenright}{\kern0pt}{\isacharbrackright}{\kern0pt}{\isacharparenright}{\kern0pt}{\isacharparenright}{\kern0pt}\ {\isacharequal}{\kern0pt}\ \isanewline
\ \ \ \ \ \ \ \ SWAP\ {\isacharasterisk}{\kern0pt}\ kron\ f\ {\isacharbrackleft}{\kern0pt}{\isadigit{2}}{\isacharcomma}{\kern0pt}{\isadigit{1}}{\isacharbrackright}{\kern0pt}{\isachardoublequoteclose}\isanewline
\ \ \ \ \isacommand{using}\isamarkupfalse%
\ reverse{\isacharunderscore}{\kern0pt}qubits{\isachardot}{\kern0pt}simps{\isacharparenleft}{\kern0pt}{\isadigit{3}}{\isacharparenright}{\kern0pt}\ upto{\isacharunderscore}{\kern0pt}rec{\isadigit{1}}\ \isacommand{by}\isamarkupfalse%
\ auto\isanewline
\ \ \isacommand{also}\isamarkupfalse%
\ \isacommand{have}\isamarkupfalse%
\ {\isachardoublequoteopen}{\isasymdots}\ {\isacharequal}{\kern0pt}\ SWAP\ {\isacharasterisk}{\kern0pt}\ {\isacharparenleft}{\kern0pt}{\isacharparenleft}{\kern0pt}f\ {\isadigit{2}}{\isacharparenright}{\kern0pt}\ {\isasymOtimes}\ {\isacharparenleft}{\kern0pt}f\ {\isadigit{1}}{\isacharparenright}{\kern0pt}{\isacharparenright}{\kern0pt}{\isachardoublequoteclose}\isanewline
\ \ \ \ \isacommand{using}\isamarkupfalse%
\ right{\isacharunderscore}{\kern0pt}tensor{\isacharunderscore}{\kern0pt}id\ \isacommand{by}\isamarkupfalse%
\ {\isacharparenleft}{\kern0pt}metis\ carrier{\isacharunderscore}{\kern0pt}mat{\isacharunderscore}{\kern0pt}triv\ kron{\isachardot}{\kern0pt}simps{\isacharparenleft}{\kern0pt}{\isadigit{1}}{\isacharparenright}{\kern0pt}\ kron{\isachardot}{\kern0pt}simps{\isacharparenleft}{\kern0pt}{\isadigit{2}}{\isacharparenright}{\kern0pt}{\isacharparenright}{\kern0pt}\isanewline
\ \ \isacommand{also}\isamarkupfalse%
\ \isacommand{have}\isamarkupfalse%
\ {\isachardoublequoteopen}{\isasymdots}\ {\isacharequal}{\kern0pt}\ {\isacharparenleft}{\kern0pt}f\ {\isadigit{1}}{\isacharparenright}{\kern0pt}\ {\isasymOtimes}\ {\isacharparenleft}{\kern0pt}f\ {\isadigit{2}}{\isacharparenright}{\kern0pt}{\isachardoublequoteclose}\ \isacommand{using}\isamarkupfalse%
\ SWAP{\isacharunderscore}{\kern0pt}tensor\ assms\ \isacommand{by}\isamarkupfalse%
\ auto\isanewline
\ \ \isacommand{also}\isamarkupfalse%
\ \isacommand{have}\isamarkupfalse%
\ {\isachardoublequoteopen}{\isasymdots}\ {\isacharequal}{\kern0pt}\ kron\ f\ {\isacharbrackleft}{\kern0pt}{\isadigit{1}}{\isacharcomma}{\kern0pt}{\isadigit{2}}{\isacharbrackright}{\kern0pt}{\isachardoublequoteclose}\ \isacommand{using}\isamarkupfalse%
\ upto{\isacharunderscore}{\kern0pt}rec{\isadigit{1}}\ assms\ \isacommand{by}\isamarkupfalse%
\ auto\isanewline
\ \ \isacommand{also}\isamarkupfalse%
\ \isacommand{have}\isamarkupfalse%
\ {\isachardoublequoteopen}{\isasymdots}\ {\isacharequal}{\kern0pt}\ kron\ f\ {\isacharparenleft}{\kern0pt}map\ nat\ {\isacharbrackleft}{\kern0pt}{\isadigit{1}}{\isachardot}{\kern0pt}{\isachardot}{\kern0pt}int\ {\isacharparenleft}{\kern0pt}Suc\ {\isacharparenleft}{\kern0pt}Suc\ {\isadigit{0}}{\isacharparenright}{\kern0pt}{\isacharparenright}{\kern0pt}{\isacharbrackright}{\kern0pt}{\isacharparenright}{\kern0pt}{\isachardoublequoteclose}\ \isacommand{using}\isamarkupfalse%
\ right{\isacharunderscore}{\kern0pt}tensor{\isacharunderscore}{\kern0pt}id\ assms\ \isanewline
\ \ \ \ \isacommand{by}\isamarkupfalse%
\ {\isacharparenleft}{\kern0pt}auto\ simp\ add{\isacharcolon}{\kern0pt}\ upto{\isacharunderscore}{\kern0pt}rec{\isadigit{1}}{\isacharparenright}{\kern0pt}\isanewline
\ \ \isacommand{finally}\isamarkupfalse%
\ \isacommand{show}\isamarkupfalse%
\ {\isachardoublequoteopen}reverse{\isacharunderscore}{\kern0pt}qubits\ {\isacharparenleft}{\kern0pt}Suc\ {\isacharparenleft}{\kern0pt}Suc\ {\isadigit{0}}{\isacharparenright}{\kern0pt}{\isacharparenright}{\kern0pt}\ {\isacharasterisk}{\kern0pt}\ kron\ f\ {\isacharparenleft}{\kern0pt}map\ nat\ {\isacharparenleft}{\kern0pt}rev\ {\isacharbrackleft}{\kern0pt}{\isadigit{1}}{\isachardot}{\kern0pt}{\isachardot}{\kern0pt}int\ {\isacharparenleft}{\kern0pt}Suc\ {\isacharparenleft}{\kern0pt}Suc\ {\isadigit{0}}{\isacharparenright}{\kern0pt}{\isacharparenright}{\kern0pt}{\isacharbrackright}{\kern0pt}{\isacharparenright}{\kern0pt}{\isacharparenright}{\kern0pt}\ {\isacharequal}{\kern0pt}\isanewline
\ \ \ \ \ \ \ \ \ \ \ \ \ \ \ \ kron\ f\ {\isacharparenleft}{\kern0pt}map\ nat\ {\isacharbrackleft}{\kern0pt}{\isadigit{1}}{\isachardot}{\kern0pt}{\isachardot}{\kern0pt}int\ {\isacharparenleft}{\kern0pt}Suc\ {\isacharparenleft}{\kern0pt}Suc\ {\isadigit{0}}{\isacharparenright}{\kern0pt}{\isacharparenright}{\kern0pt}{\isacharbrackright}{\kern0pt}{\isacharparenright}{\kern0pt}{\isachardoublequoteclose}\ \isacommand{by}\isamarkupfalse%
\ this\isanewline
\isacommand{next}\isamarkupfalse%
\ \isanewline
\ \ \isacommand{case}\isamarkupfalse%
\ {\isadigit{4}}\isanewline
\ \ \isacommand{fix}\isamarkupfalse%
\ n{\isacharcolon}{\kern0pt}{\isacharcolon}{\kern0pt}nat\isanewline
\ \ \isacommand{define}\isamarkupfalse%
\ k{\isacharcolon}{\kern0pt}{\isacharcolon}{\kern0pt}nat\ \isakeyword{where}\ {\isachardoublequoteopen}k\ {\isacharequal}{\kern0pt}\ Suc\ {\isacharparenleft}{\kern0pt}Suc\ n{\isacharparenright}{\kern0pt}{\isachardoublequoteclose}\isanewline
\ \ \isacommand{assume}\isamarkupfalse%
\ HI{\isacharcolon}{\kern0pt}{\isachardoublequoteopen}reverse{\isacharunderscore}{\kern0pt}qubits\ {\isacharparenleft}{\kern0pt}Suc\ {\isacharparenleft}{\kern0pt}Suc\ n{\isacharparenright}{\kern0pt}{\isacharparenright}{\kern0pt}\ {\isacharasterisk}{\kern0pt}\ kron\ f\ {\isacharparenleft}{\kern0pt}map\ nat\ {\isacharparenleft}{\kern0pt}rev\ {\isacharbrackleft}{\kern0pt}{\isadigit{1}}{\isachardot}{\kern0pt}{\isachardot}{\kern0pt}int\ {\isacharparenleft}{\kern0pt}Suc\ {\isacharparenleft}{\kern0pt}Suc\ n{\isacharparenright}{\kern0pt}{\isacharparenright}{\kern0pt}{\isacharbrackright}{\kern0pt}{\isacharparenright}{\kern0pt}{\isacharparenright}{\kern0pt}\ {\isacharequal}{\kern0pt}\isanewline
\ \ \ \ \ \ \ \ \ \ \ \ \ kron\ f\ {\isacharparenleft}{\kern0pt}map\ nat\ {\isacharbrackleft}{\kern0pt}{\isadigit{1}}{\isachardot}{\kern0pt}{\isachardot}{\kern0pt}int\ {\isacharparenleft}{\kern0pt}Suc\ {\isacharparenleft}{\kern0pt}Suc\ n{\isacharparenright}{\kern0pt}{\isacharparenright}{\kern0pt}{\isacharbrackright}{\kern0pt}{\isacharparenright}{\kern0pt}{\isachardoublequoteclose}\isanewline
\ \ \isacommand{have}\isamarkupfalse%
\ sk{\isacharcolon}{\kern0pt}{\isachardoublequoteopen}{\isacharparenleft}{\kern0pt}SWAP{\isacharunderscore}{\kern0pt}down\ {\isacharparenleft}{\kern0pt}Suc\ k{\isacharparenright}{\kern0pt}{\isacharparenright}{\kern0pt}\ {\isacharasterisk}{\kern0pt}\ kron\ f\ {\isacharparenleft}{\kern0pt}map\ nat\ {\isacharparenleft}{\kern0pt}rev\ {\isacharbrackleft}{\kern0pt}{\isadigit{1}}{\isachardot}{\kern0pt}{\isachardot}{\kern0pt}int\ {\isacharparenleft}{\kern0pt}Suc\ k{\isacharparenright}{\kern0pt}{\isacharbrackright}{\kern0pt}{\isacharparenright}{\kern0pt}{\isacharparenright}{\kern0pt}\ {\isacharequal}{\kern0pt}\ \isanewline
\ \ \ \ \ \ \ \ {\isacharparenleft}{\kern0pt}kron\ f\ {\isacharparenleft}{\kern0pt}map\ nat\ {\isacharparenleft}{\kern0pt}rev\ {\isacharbrackleft}{\kern0pt}{\isadigit{1}}{\isachardot}{\kern0pt}{\isachardot}{\kern0pt}int\ k{\isacharbrackright}{\kern0pt}{\isacharparenright}{\kern0pt}{\isacharparenright}{\kern0pt}\ {\isasymOtimes}\ {\isacharparenleft}{\kern0pt}f\ {\isacharparenleft}{\kern0pt}Suc\ k{\isacharparenright}{\kern0pt}{\isacharparenright}{\kern0pt}{\isacharparenright}{\kern0pt}{\isachardoublequoteclose}\ \isanewline
\ \ \ \ \isacommand{using}\isamarkupfalse%
\ SWAP{\isacharunderscore}{\kern0pt}down{\isacharunderscore}{\kern0pt}kron{\isacharunderscore}{\kern0pt}map{\isacharunderscore}{\kern0pt}rev\ assms\ \isacommand{by}\isamarkupfalse%
\ this\isanewline
\ \ \isacommand{have}\isamarkupfalse%
\ {\isachardoublequoteopen}reverse{\isacharunderscore}{\kern0pt}qubits\ {\isacharparenleft}{\kern0pt}Suc\ k{\isacharparenright}{\kern0pt}\ {\isacharasterisk}{\kern0pt}\ kron\ f\ {\isacharparenleft}{\kern0pt}map\ nat\ {\isacharparenleft}{\kern0pt}rev\ {\isacharbrackleft}{\kern0pt}{\isadigit{1}}{\isachardot}{\kern0pt}{\isachardot}{\kern0pt}int\ {\isacharparenleft}{\kern0pt}Suc\ k{\isacharparenright}{\kern0pt}{\isacharbrackright}{\kern0pt}{\isacharparenright}{\kern0pt}{\isacharparenright}{\kern0pt}\ {\isacharequal}{\kern0pt}\isanewline
\ \ \ \ \ \ \ \ {\isacharparenleft}{\kern0pt}{\isacharparenleft}{\kern0pt}reverse{\isacharunderscore}{\kern0pt}qubits\ k{\isacharparenright}{\kern0pt}\ {\isasymOtimes}\ {\isacharparenleft}{\kern0pt}{\isadigit{1}}\isactrlsub m\ {\isadigit{2}}{\isacharparenright}{\kern0pt}{\isacharparenright}{\kern0pt}\ {\isacharasterisk}{\kern0pt}\ {\isacharparenleft}{\kern0pt}SWAP{\isacharunderscore}{\kern0pt}down\ {\isacharparenleft}{\kern0pt}Suc\ k{\isacharparenright}{\kern0pt}{\isacharparenright}{\kern0pt}\ {\isacharasterisk}{\kern0pt}\ \isanewline
\ \ \ \ \ \ \ \ kron\ f\ {\isacharparenleft}{\kern0pt}map\ nat\ {\isacharparenleft}{\kern0pt}rev\ {\isacharbrackleft}{\kern0pt}{\isadigit{1}}{\isachardot}{\kern0pt}{\isachardot}{\kern0pt}int\ {\isacharparenleft}{\kern0pt}Suc\ k{\isacharparenright}{\kern0pt}{\isacharbrackright}{\kern0pt}{\isacharparenright}{\kern0pt}{\isacharparenright}{\kern0pt}{\isachardoublequoteclose}\isanewline
\ \ \ \ \isacommand{using}\isamarkupfalse%
\ reverse{\isacharunderscore}{\kern0pt}qubits{\isachardot}{\kern0pt}simps{\isacharparenleft}{\kern0pt}{\isadigit{4}}{\isacharparenright}{\kern0pt}\ k{\isacharunderscore}{\kern0pt}def\ \isacommand{by}\isamarkupfalse%
\ simp\isanewline
\ \ \isacommand{also}\isamarkupfalse%
\ \isacommand{have}\isamarkupfalse%
\ {\isachardoublequoteopen}{\isasymdots}\ {\isacharequal}{\kern0pt}\ {\isacharparenleft}{\kern0pt}{\isacharparenleft}{\kern0pt}reverse{\isacharunderscore}{\kern0pt}qubits\ k{\isacharparenright}{\kern0pt}\ {\isasymOtimes}\ {\isacharparenleft}{\kern0pt}{\isadigit{1}}\isactrlsub m\ {\isadigit{2}}{\isacharparenright}{\kern0pt}{\isacharparenright}{\kern0pt}\ {\isacharasterisk}{\kern0pt}\ {\isacharparenleft}{\kern0pt}{\isacharparenleft}{\kern0pt}SWAP{\isacharunderscore}{\kern0pt}down\ {\isacharparenleft}{\kern0pt}Suc\ k{\isacharparenright}{\kern0pt}{\isacharparenright}{\kern0pt}\ {\isacharasterisk}{\kern0pt}\ \isanewline
\ \ \ \ \ \ \ \ \ \ \ \ kron\ f\ {\isacharparenleft}{\kern0pt}map\ nat\ {\isacharparenleft}{\kern0pt}rev\ {\isacharbrackleft}{\kern0pt}{\isadigit{1}}{\isachardot}{\kern0pt}{\isachardot}{\kern0pt}int\ {\isacharparenleft}{\kern0pt}Suc\ k{\isacharparenright}{\kern0pt}{\isacharbrackright}{\kern0pt}{\isacharparenright}{\kern0pt}{\isacharparenright}{\kern0pt}{\isacharparenright}{\kern0pt}{\isachardoublequoteclose}\isanewline
\ \ \isacommand{proof}\isamarkupfalse%
\ {\isacharparenleft}{\kern0pt}rule\ assoc{\isacharunderscore}{\kern0pt}mult{\isacharunderscore}{\kern0pt}mat{\isacharparenright}{\kern0pt}\isanewline
\ \ \ \ \isacommand{show}\isamarkupfalse%
\ {\isachardoublequoteopen}{\isacharparenleft}{\kern0pt}reverse{\isacharunderscore}{\kern0pt}qubits\ k{\isacharparenright}{\kern0pt}\ {\isasymOtimes}\ {\isacharparenleft}{\kern0pt}{\isadigit{1}}\isactrlsub m\ {\isadigit{2}}{\isacharparenright}{\kern0pt}\ {\isasymin}\ carrier{\isacharunderscore}{\kern0pt}mat\ {\isacharparenleft}{\kern0pt}{\isadigit{2}}{\isacharcircum}{\kern0pt}{\isacharparenleft}{\kern0pt}k{\isacharplus}{\kern0pt}{\isadigit{1}}{\isacharparenright}{\kern0pt}{\isacharparenright}{\kern0pt}\ {\isacharparenleft}{\kern0pt}{\isadigit{2}}{\isacharcircum}{\kern0pt}{\isacharparenleft}{\kern0pt}k{\isacharplus}{\kern0pt}{\isadigit{1}}{\isacharparenright}{\kern0pt}{\isacharparenright}{\kern0pt}{\isachardoublequoteclose}\isanewline
\ \ \ \ \isacommand{proof}\isamarkupfalse%
\ {\isacharminus}{\kern0pt}\isanewline
\ \ \ \ \ \ \isacommand{have}\isamarkupfalse%
\ {\isachardoublequoteopen}reverse{\isacharunderscore}{\kern0pt}qubits\ k\ {\isasymin}\ carrier{\isacharunderscore}{\kern0pt}mat\ {\isacharparenleft}{\kern0pt}{\isadigit{2}}{\isacharcircum}{\kern0pt}k{\isacharparenright}{\kern0pt}\ {\isacharparenleft}{\kern0pt}{\isadigit{2}}{\isacharcircum}{\kern0pt}k{\isacharparenright}{\kern0pt}{\isachardoublequoteclose}\ \isacommand{by}\isamarkupfalse%
\ simp\isanewline
\ \ \ \ \ \ \isacommand{moreover}\isamarkupfalse%
\ \isacommand{have}\isamarkupfalse%
\ {\isachardoublequoteopen}{\isadigit{1}}\isactrlsub m\ {\isadigit{2}}\ {\isasymin}\ carrier{\isacharunderscore}{\kern0pt}mat\ {\isadigit{2}}\ {\isadigit{2}}{\isachardoublequoteclose}\ \isacommand{by}\isamarkupfalse%
\ simp\isanewline
\ \ \ \ \ \ \isacommand{ultimately}\isamarkupfalse%
\ \isacommand{show}\isamarkupfalse%
\ {\isacharquery}{\kern0pt}thesis\ \isacommand{using}\isamarkupfalse%
\ tensor{\isacharunderscore}{\kern0pt}carrier{\isacharunderscore}{\kern0pt}mat\ \isacommand{by}\isamarkupfalse%
\ {\isacharparenleft}{\kern0pt}smt\ {\isacharparenleft}{\kern0pt}verit{\isacharparenright}{\kern0pt}\ power{\isacharunderscore}{\kern0pt}add\ power{\isacharunderscore}{\kern0pt}one{\isacharunderscore}{\kern0pt}right{\isacharparenright}{\kern0pt}\isanewline
\ \ \ \ \isacommand{qed}\isamarkupfalse%
\isanewline
\ \ \isacommand{next}\isamarkupfalse%
\isanewline
\ \ \ \ \isacommand{show}\isamarkupfalse%
\ {\isachardoublequoteopen}{\isacharparenleft}{\kern0pt}SWAP{\isacharunderscore}{\kern0pt}down\ {\isacharparenleft}{\kern0pt}Suc\ k{\isacharparenright}{\kern0pt}{\isacharparenright}{\kern0pt}\ {\isasymin}\ carrier{\isacharunderscore}{\kern0pt}mat\ {\isacharparenleft}{\kern0pt}{\isadigit{2}}{\isacharcircum}{\kern0pt}{\isacharparenleft}{\kern0pt}k{\isacharplus}{\kern0pt}{\isadigit{1}}{\isacharparenright}{\kern0pt}{\isacharparenright}{\kern0pt}\ {\isacharparenleft}{\kern0pt}{\isadigit{2}}{\isacharcircum}{\kern0pt}{\isacharparenleft}{\kern0pt}k{\isacharplus}{\kern0pt}{\isadigit{1}}{\isacharparenright}{\kern0pt}{\isacharparenright}{\kern0pt}{\isachardoublequoteclose}\isanewline
\ \ \ \ \ \ \isacommand{using}\isamarkupfalse%
\ Suc{\isacharunderscore}{\kern0pt}eq{\isacharunderscore}{\kern0pt}plus{\isadigit{1}}\ SWAP{\isacharunderscore}{\kern0pt}down{\isacharunderscore}{\kern0pt}carrier{\isacharunderscore}{\kern0pt}mat\ \isacommand{by}\isamarkupfalse%
\ presburger\isanewline
\ \ \isacommand{next}\isamarkupfalse%
\isanewline
\ \ \ \ \isacommand{show}\isamarkupfalse%
\ {\isachardoublequoteopen}kron\ f\ {\isacharparenleft}{\kern0pt}map\ nat\ {\isacharparenleft}{\kern0pt}rev\ {\isacharbrackleft}{\kern0pt}{\isadigit{1}}{\isachardot}{\kern0pt}{\isachardot}{\kern0pt}int\ {\isacharparenleft}{\kern0pt}Suc\ k{\isacharparenright}{\kern0pt}{\isacharbrackright}{\kern0pt}{\isacharparenright}{\kern0pt}{\isacharparenright}{\kern0pt}\ {\isasymin}\ carrier{\isacharunderscore}{\kern0pt}mat\ {\isacharparenleft}{\kern0pt}{\isadigit{2}}\ {\isacharcircum}{\kern0pt}\ {\isacharparenleft}{\kern0pt}k\ {\isacharplus}{\kern0pt}\ {\isadigit{1}}{\isacharparenright}{\kern0pt}{\isacharparenright}{\kern0pt}\ {\isadigit{1}}{\isachardoublequoteclose}\isanewline
\ \ \ \ \isacommand{proof}\isamarkupfalse%
\ {\isacharminus}{\kern0pt}\isanewline
\ \ \ \ \ \ \isacommand{define}\isamarkupfalse%
\ xs\ \isakeyword{where}\ {\isachardoublequoteopen}xs\ {\isacharequal}{\kern0pt}\ {\isacharparenleft}{\kern0pt}map\ nat\ {\isacharparenleft}{\kern0pt}rev\ {\isacharbrackleft}{\kern0pt}{\isadigit{1}}{\isachardot}{\kern0pt}{\isachardot}{\kern0pt}int\ {\isacharparenleft}{\kern0pt}Suc\ k{\isacharparenright}{\kern0pt}{\isacharbrackright}{\kern0pt}{\isacharparenright}{\kern0pt}{\isacharparenright}{\kern0pt}{\isachardoublequoteclose}\isanewline
\ \ \ \ \ \ \isacommand{then}\isamarkupfalse%
\ \isacommand{have}\isamarkupfalse%
\ k{\isadigit{1}}{\isacharcolon}{\kern0pt}{\isachardoublequoteopen}length\ xs\ {\isacharequal}{\kern0pt}\ k\ {\isacharplus}{\kern0pt}\ {\isadigit{1}}{\isachardoublequoteclose}\ \isacommand{by}\isamarkupfalse%
\ auto\isanewline
\ \ \ \ \ \ \isacommand{then}\isamarkupfalse%
\ \isacommand{have}\isamarkupfalse%
\ {\isachardoublequoteopen}kron\ f\ xs\ {\isasymin}\ carrier{\isacharunderscore}{\kern0pt}mat\ {\isacharparenleft}{\kern0pt}{\isadigit{2}}\ {\isacharcircum}{\kern0pt}\ {\isacharparenleft}{\kern0pt}k\ {\isacharplus}{\kern0pt}\ {\isadigit{1}}{\isacharparenright}{\kern0pt}{\isacharparenright}{\kern0pt}\ {\isadigit{1}}{\isachardoublequoteclose}\isanewline
\ \ \ \ \ \ \ \ \isacommand{using}\isamarkupfalse%
\ kron{\isacharunderscore}{\kern0pt}carrier{\isacharunderscore}{\kern0pt}mat\ assms\ k{\isadigit{1}}\ \isacommand{by}\isamarkupfalse%
\ metis\isanewline
\ \ \ \ \ \ \isacommand{thus}\isamarkupfalse%
\ {\isacharquery}{\kern0pt}thesis\ \isacommand{using}\isamarkupfalse%
\ xs{\isacharunderscore}{\kern0pt}def\ \isacommand{by}\isamarkupfalse%
\ simp\isanewline
\ \ \ \ \isacommand{qed}\isamarkupfalse%
\isanewline
\ \ \isacommand{qed}\isamarkupfalse%
\isanewline
\ \ \isacommand{also}\isamarkupfalse%
\ \isacommand{have}\isamarkupfalse%
\ {\isachardoublequoteopen}{\isasymdots}\ {\isacharequal}{\kern0pt}\ {\isacharparenleft}{\kern0pt}{\isacharparenleft}{\kern0pt}reverse{\isacharunderscore}{\kern0pt}qubits\ k{\isacharparenright}{\kern0pt}\ {\isasymOtimes}\ {\isacharparenleft}{\kern0pt}{\isadigit{1}}\isactrlsub m\ {\isadigit{2}}{\isacharparenright}{\kern0pt}{\isacharparenright}{\kern0pt}\ {\isacharasterisk}{\kern0pt}\ {\isacharparenleft}{\kern0pt}kron\ f\ {\isacharparenleft}{\kern0pt}map\ nat\ {\isacharparenleft}{\kern0pt}rev\ {\isacharbrackleft}{\kern0pt}{\isadigit{1}}{\isachardot}{\kern0pt}{\isachardot}{\kern0pt}int\ k{\isacharbrackright}{\kern0pt}{\isacharparenright}{\kern0pt}{\isacharparenright}{\kern0pt}\ {\isasymOtimes}\ {\isacharparenleft}{\kern0pt}f\ {\isacharparenleft}{\kern0pt}Suc\ k{\isacharparenright}{\kern0pt}{\isacharparenright}{\kern0pt}{\isacharparenright}{\kern0pt}{\isachardoublequoteclose}\isanewline
\ \ \ \ \isacommand{using}\isamarkupfalse%
\ sk\ \isacommand{by}\isamarkupfalse%
\ simp\isanewline
\ \ \isacommand{also}\isamarkupfalse%
\ \isacommand{have}\isamarkupfalse%
\ {\isachardoublequoteopen}{\isasymdots}\ {\isacharequal}{\kern0pt}\ {\isacharparenleft}{\kern0pt}{\isacharparenleft}{\kern0pt}reverse{\isacharunderscore}{\kern0pt}qubits\ k{\isacharparenright}{\kern0pt}\ {\isacharasterisk}{\kern0pt}\ {\isacharparenleft}{\kern0pt}kron\ f\ {\isacharparenleft}{\kern0pt}map\ nat\ {\isacharparenleft}{\kern0pt}rev\ {\isacharbrackleft}{\kern0pt}{\isadigit{1}}{\isachardot}{\kern0pt}{\isachardot}{\kern0pt}int\ k{\isacharbrackright}{\kern0pt}{\isacharparenright}{\kern0pt}{\isacharparenright}{\kern0pt}{\isacharparenright}{\kern0pt}{\isacharparenright}{\kern0pt}\ {\isasymOtimes}\ {\isacharparenleft}{\kern0pt}{\isacharparenleft}{\kern0pt}{\isadigit{1}}\isactrlsub m\ {\isadigit{2}}{\isacharparenright}{\kern0pt}\ {\isacharasterisk}{\kern0pt}\ {\isacharparenleft}{\kern0pt}f\ {\isacharparenleft}{\kern0pt}Suc\ k{\isacharparenright}{\kern0pt}{\isacharparenright}{\kern0pt}{\isacharparenright}{\kern0pt}{\isachardoublequoteclose}\isanewline
\ \ \isacommand{proof}\isamarkupfalse%
\ {\isacharminus}{\kern0pt}\isanewline
\ \ \ \ \isacommand{have}\isamarkupfalse%
\ c{\isadigit{1}}{\isacharcolon}{\kern0pt}{\isachardoublequoteopen}dim{\isacharunderscore}{\kern0pt}col\ {\isacharparenleft}{\kern0pt}reverse{\isacharunderscore}{\kern0pt}qubits\ k{\isacharparenright}{\kern0pt}\ {\isacharequal}{\kern0pt}\ {\isadigit{2}}{\isacharcircum}{\kern0pt}k{\isachardoublequoteclose}\ \isacommand{using}\isamarkupfalse%
\ reverse{\isacharunderscore}{\kern0pt}qubits{\isacharunderscore}{\kern0pt}carrier{\isacharunderscore}{\kern0pt}mat\ \isacommand{by}\isamarkupfalse%
\ blast\isanewline
\ \ \ \ \isacommand{have}\isamarkupfalse%
\ r{\isadigit{2}}{\isacharcolon}{\kern0pt}{\isachardoublequoteopen}dim{\isacharunderscore}{\kern0pt}row\ {\isacharparenleft}{\kern0pt}kron\ f\ {\isacharparenleft}{\kern0pt}map\ nat\ {\isacharparenleft}{\kern0pt}rev\ {\isacharbrackleft}{\kern0pt}{\isadigit{1}}{\isachardot}{\kern0pt}{\isachardot}{\kern0pt}int\ k{\isacharbrackright}{\kern0pt}{\isacharparenright}{\kern0pt}{\isacharparenright}{\kern0pt}{\isacharparenright}{\kern0pt}\ {\isacharequal}{\kern0pt}\ {\isadigit{2}}{\isacharcircum}{\kern0pt}k{\isachardoublequoteclose}\ \isanewline
\ \ \ \ \ \ \isacommand{using}\isamarkupfalse%
\ kron{\isacharunderscore}{\kern0pt}carrier{\isacharunderscore}{\kern0pt}mat\ \isacommand{by}\isamarkupfalse%
\ {\isacharparenleft}{\kern0pt}metis\ HI\ assms\ carrier{\isacharunderscore}{\kern0pt}matD{\isacharparenleft}{\kern0pt}{\isadigit{1}}{\isacharparenright}{\kern0pt}\ index{\isacharunderscore}{\kern0pt}mult{\isacharunderscore}{\kern0pt}mat{\isacharparenleft}{\kern0pt}{\isadigit{2}}{\isacharparenright}{\kern0pt}\ k{\isacharunderscore}{\kern0pt}def\ length{\isacharunderscore}{\kern0pt}map\ \isanewline
\ \ \ \ \ \ \ \ \ \ length{\isacharunderscore}{\kern0pt}rev\ reverse{\isacharunderscore}{\kern0pt}qubits{\isacharunderscore}{\kern0pt}carrier{\isacharunderscore}{\kern0pt}mat{\isacharparenright}{\kern0pt}\isanewline
\ \ \ \ \isacommand{with}\isamarkupfalse%
\ c{\isadigit{1}}\ r{\isadigit{2}}\ \isacommand{have}\isamarkupfalse%
\ a{\isadigit{1}}{\isacharcolon}{\kern0pt}{\isachardoublequoteopen}dim{\isacharunderscore}{\kern0pt}col\ {\isacharparenleft}{\kern0pt}reverse{\isacharunderscore}{\kern0pt}qubits\ k{\isacharparenright}{\kern0pt}\ {\isacharequal}{\kern0pt}\ dim{\isacharunderscore}{\kern0pt}row\ {\isacharparenleft}{\kern0pt}kron\ f\ {\isacharparenleft}{\kern0pt}map\ nat\ {\isacharparenleft}{\kern0pt}rev\ {\isacharbrackleft}{\kern0pt}{\isadigit{1}}{\isachardot}{\kern0pt}{\isachardot}{\kern0pt}int\ k{\isacharbrackright}{\kern0pt}{\isacharparenright}{\kern0pt}{\isacharparenright}{\kern0pt}{\isacharparenright}{\kern0pt}{\isachardoublequoteclose}\isanewline
\ \ \ \ \ \ \isacommand{by}\isamarkupfalse%
\ auto\isanewline
\ \ \ \ \isacommand{have}\isamarkupfalse%
\ c{\isadigit{3}}{\isacharcolon}{\kern0pt}{\isachardoublequoteopen}dim{\isacharunderscore}{\kern0pt}col\ {\isacharparenleft}{\kern0pt}{\isadigit{1}}\isactrlsub m\ {\isadigit{2}}{\isacharparenright}{\kern0pt}\ {\isacharequal}{\kern0pt}\ {\isadigit{2}}{\isachardoublequoteclose}\ \isacommand{by}\isamarkupfalse%
\ simp\isanewline
\ \ \ \ \isacommand{have}\isamarkupfalse%
\ r{\isadigit{4}}{\isacharcolon}{\kern0pt}{\isachardoublequoteopen}dim{\isacharunderscore}{\kern0pt}row\ {\isacharparenleft}{\kern0pt}f\ {\isacharparenleft}{\kern0pt}Suc\ k{\isacharparenright}{\kern0pt}{\isacharparenright}{\kern0pt}\ {\isacharequal}{\kern0pt}\ {\isadigit{2}}{\isachardoublequoteclose}\ \isacommand{using}\isamarkupfalse%
\ assms\ \isacommand{by}\isamarkupfalse%
\ simp\isanewline
\ \ \ \ \isacommand{with}\isamarkupfalse%
\ c{\isadigit{3}}\ r{\isadigit{4}}\ \isacommand{have}\isamarkupfalse%
\ a{\isadigit{2}}{\isacharcolon}{\kern0pt}{\isachardoublequoteopen}dim{\isacharunderscore}{\kern0pt}col\ {\isacharparenleft}{\kern0pt}{\isadigit{1}}\isactrlsub m\ {\isadigit{2}}{\isacharparenright}{\kern0pt}\ {\isacharequal}{\kern0pt}\ dim{\isacharunderscore}{\kern0pt}row\ {\isacharparenleft}{\kern0pt}f\ {\isacharparenleft}{\kern0pt}Suc\ k{\isacharparenright}{\kern0pt}{\isacharparenright}{\kern0pt}{\isachardoublequoteclose}\ \isacommand{by}\isamarkupfalse%
\ simp\isanewline
\ \ \ \ \isacommand{have}\isamarkupfalse%
\ a{\isadigit{3}}{\isacharcolon}{\kern0pt}{\isachardoublequoteopen}dim{\isacharunderscore}{\kern0pt}col\ {\isacharparenleft}{\kern0pt}reverse{\isacharunderscore}{\kern0pt}qubits\ k{\isacharparenright}{\kern0pt}\ {\isachargreater}{\kern0pt}\ {\isadigit{0}}{\isachardoublequoteclose}\ \isacommand{using}\isamarkupfalse%
\ c{\isadigit{1}}\ \isacommand{by}\isamarkupfalse%
\ auto\isanewline
\ \ \ \ \isacommand{have}\isamarkupfalse%
\ a{\isadigit{4}}{\isacharcolon}{\kern0pt}{\isachardoublequoteopen}dim{\isacharunderscore}{\kern0pt}row\ {\isacharparenleft}{\kern0pt}kron\ f\ {\isacharparenleft}{\kern0pt}map\ nat\ {\isacharparenleft}{\kern0pt}rev\ {\isacharbrackleft}{\kern0pt}{\isadigit{1}}{\isachardot}{\kern0pt}{\isachardot}{\kern0pt}int\ k{\isacharbrackright}{\kern0pt}{\isacharparenright}{\kern0pt}{\isacharparenright}{\kern0pt}{\isacharparenright}{\kern0pt}\ {\isachargreater}{\kern0pt}\ {\isadigit{0}}{\isachardoublequoteclose}\ \isacommand{using}\isamarkupfalse%
\ r{\isadigit{2}}\ \isacommand{by}\isamarkupfalse%
\ auto\isanewline
\ \ \ \ \isacommand{have}\isamarkupfalse%
\ a{\isadigit{5}}{\isacharcolon}{\kern0pt}{\isachardoublequoteopen}dim{\isacharunderscore}{\kern0pt}col\ {\isacharparenleft}{\kern0pt}{\isadigit{1}}\isactrlsub m\ {\isadigit{2}}{\isacharparenright}{\kern0pt}\ {\isachargreater}{\kern0pt}\ {\isadigit{0}}{\isachardoublequoteclose}\ \isacommand{using}\isamarkupfalse%
\ c{\isadigit{3}}\ \isacommand{by}\isamarkupfalse%
\ auto\isanewline
\ \ \ \ \isacommand{have}\isamarkupfalse%
\ a{\isadigit{6}}{\isacharcolon}{\kern0pt}{\isachardoublequoteopen}dim{\isacharunderscore}{\kern0pt}row\ {\isacharparenleft}{\kern0pt}f\ {\isacharparenleft}{\kern0pt}Suc\ k{\isacharparenright}{\kern0pt}{\isacharparenright}{\kern0pt}\ {\isachargreater}{\kern0pt}\ {\isadigit{0}}{\isachardoublequoteclose}\ \isacommand{using}\isamarkupfalse%
\ r{\isadigit{4}}\ \isacommand{by}\isamarkupfalse%
\ auto\isanewline
\ \ \ \ \isacommand{show}\isamarkupfalse%
\ {\isacharquery}{\kern0pt}thesis\ \isacommand{using}\isamarkupfalse%
\ a{\isadigit{1}}\ a{\isadigit{2}}\ a{\isadigit{3}}\ a{\isadigit{4}}\ a{\isadigit{5}}\ a{\isadigit{6}}\ mult{\isacharunderscore}{\kern0pt}distr{\isacharunderscore}{\kern0pt}tensor\isanewline
\ \ \ \ \ \ \isacommand{by}\isamarkupfalse%
\ {\isacharparenleft}{\kern0pt}metis\ assms\ carrier{\isacharunderscore}{\kern0pt}matD{\isacharparenleft}{\kern0pt}{\isadigit{2}}{\isacharparenright}{\kern0pt}\ kron{\isacharunderscore}{\kern0pt}carrier{\isacharunderscore}{\kern0pt}mat\ zero{\isacharunderscore}{\kern0pt}less{\isacharunderscore}{\kern0pt}one{\isacharunderscore}{\kern0pt}class{\isachardot}{\kern0pt}zero{\isacharunderscore}{\kern0pt}less{\isacharunderscore}{\kern0pt}one{\isacharparenright}{\kern0pt}\isanewline
\ \ \isacommand{qed}\isamarkupfalse%
\isanewline
\ \ \isacommand{also}\isamarkupfalse%
\ \isacommand{have}\isamarkupfalse%
\ {\isachardoublequoteopen}{\isasymdots}\ {\isacharequal}{\kern0pt}\ kron\ f\ {\isacharparenleft}{\kern0pt}map\ nat\ {\isacharbrackleft}{\kern0pt}{\isadigit{1}}{\isachardot}{\kern0pt}{\isachardot}{\kern0pt}int\ k{\isacharbrackright}{\kern0pt}{\isacharparenright}{\kern0pt}\ {\isasymOtimes}\ {\isacharparenleft}{\kern0pt}f\ {\isacharparenleft}{\kern0pt}Suc\ k{\isacharparenright}{\kern0pt}{\isacharparenright}{\kern0pt}{\isachardoublequoteclose}\isanewline
\ \ \ \ \isacommand{using}\isamarkupfalse%
\ HI\ k{\isacharunderscore}{\kern0pt}def\ assms\ \isacommand{by}\isamarkupfalse%
\ auto\isanewline
\ \ \isacommand{also}\isamarkupfalse%
\ \isacommand{have}\isamarkupfalse%
\ {\isachardoublequoteopen}{\isasymdots}\ {\isacharequal}{\kern0pt}\ kron\ f\ {\isacharparenleft}{\kern0pt}map\ nat\ {\isacharbrackleft}{\kern0pt}{\isadigit{1}}{\isachardot}{\kern0pt}{\isachardot}{\kern0pt}int\ {\isacharparenleft}{\kern0pt}Suc\ k{\isacharparenright}{\kern0pt}{\isacharbrackright}{\kern0pt}{\isacharparenright}{\kern0pt}{\isachardoublequoteclose}\ \isacommand{using}\isamarkupfalse%
\ kron{\isacharunderscore}{\kern0pt}cons{\isacharunderscore}{\kern0pt}right\isanewline
\ \ \ \ \isacommand{by}\isamarkupfalse%
\ {\isacharparenleft}{\kern0pt}smt\ {\isacharparenleft}{\kern0pt}verit{\isacharcomma}{\kern0pt}\ ccfv{\isacharunderscore}{\kern0pt}threshold{\isacharparenright}{\kern0pt}\ list{\isachardot}{\kern0pt}simps{\isacharparenleft}{\kern0pt}{\isadigit{8}}{\isacharparenright}{\kern0pt}\ list{\isachardot}{\kern0pt}simps{\isacharparenleft}{\kern0pt}{\isadigit{9}}{\isacharparenright}{\kern0pt}\ map{\isacharunderscore}{\kern0pt}append\ nat{\isacharunderscore}{\kern0pt}int\ negative{\isacharunderscore}{\kern0pt}zless\ \isanewline
\ \ \ \ \ \ \ \ of{\isacharunderscore}{\kern0pt}nat{\isacharunderscore}{\kern0pt}Suc\ upto{\isacharunderscore}{\kern0pt}rec{\isadigit{2}}{\isacharparenright}{\kern0pt}\isanewline
\ \ \isacommand{finally}\isamarkupfalse%
\ \isacommand{show}\isamarkupfalse%
\ {\isachardoublequoteopen}reverse{\isacharunderscore}{\kern0pt}qubits\ {\isacharparenleft}{\kern0pt}Suc\ {\isacharparenleft}{\kern0pt}Suc\ {\isacharparenleft}{\kern0pt}Suc\ n{\isacharparenright}{\kern0pt}{\isacharparenright}{\kern0pt}{\isacharparenright}{\kern0pt}\ {\isacharasterisk}{\kern0pt}\ \isanewline
\ \ \ \ \ \ \ \ \ \ \ \ \ \ \ \ kron\ f\ {\isacharparenleft}{\kern0pt}map\ nat\ {\isacharparenleft}{\kern0pt}rev\ {\isacharbrackleft}{\kern0pt}{\isadigit{1}}{\isachardot}{\kern0pt}{\isachardot}{\kern0pt}int\ {\isacharparenleft}{\kern0pt}Suc\ {\isacharparenleft}{\kern0pt}Suc\ {\isacharparenleft}{\kern0pt}Suc\ n{\isacharparenright}{\kern0pt}{\isacharparenright}{\kern0pt}{\isacharparenright}{\kern0pt}{\isacharbrackright}{\kern0pt}{\isacharparenright}{\kern0pt}{\isacharparenright}{\kern0pt}\ {\isacharequal}{\kern0pt}\isanewline
\ \ \ \ \ \ \ \ \ \ \ \ \ \ \ \ kron\ f\ {\isacharparenleft}{\kern0pt}map\ nat\ {\isacharbrackleft}{\kern0pt}{\isadigit{1}}{\isachardot}{\kern0pt}{\isachardot}{\kern0pt}int\ {\isacharparenleft}{\kern0pt}Suc\ {\isacharparenleft}{\kern0pt}Suc\ {\isacharparenleft}{\kern0pt}Suc\ n{\isacharparenright}{\kern0pt}{\isacharparenright}{\kern0pt}{\isacharparenright}{\kern0pt}{\isacharbrackright}{\kern0pt}{\isacharparenright}{\kern0pt}{\isachardoublequoteclose}\ \isacommand{using}\isamarkupfalse%
\ k{\isacharunderscore}{\kern0pt}def\ \isacommand{by}\isamarkupfalse%
\ simp\isanewline
\isacommand{qed}\isamarkupfalse%
%
\endisatagproof
{\isafoldproof}%
%
\isadelimproof
\isanewline
%
\endisadelimproof
\isanewline
\ \ \ \ \isanewline
\isacommand{lemma}\isamarkupfalse%
\ prod{\isacharunderscore}{\kern0pt}rep{\isacharunderscore}{\kern0pt}fun{\isacharcolon}{\kern0pt}\isanewline
\ \ \isakeyword{assumes}\ {\isachardoublequoteopen}f\ {\isacharequal}{\kern0pt}\ {\isacharparenleft}{\kern0pt}{\isasymlambda}{\isacharparenleft}{\kern0pt}l{\isacharcolon}{\kern0pt}{\isacharcolon}{\kern0pt}nat{\isacharparenright}{\kern0pt}{\isachardot}{\kern0pt}\ {\isacharbar}{\kern0pt}zero{\isasymrangle}\ {\isacharplus}{\kern0pt}\ exp\ {\isacharparenleft}{\kern0pt}{\isadigit{2}}{\isacharasterisk}{\kern0pt}{\isasymi}{\isacharasterisk}{\kern0pt}pi{\isacharasterisk}{\kern0pt}j{\isacharslash}{\kern0pt}{\isacharparenleft}{\kern0pt}{\isadigit{2}}{\isacharcircum}{\kern0pt}l{\isacharparenright}{\kern0pt}{\isacharparenright}{\kern0pt}\ {\isasymcdot}\isactrlsub m\ {\isacharbar}{\kern0pt}one{\isasymrangle}{\isacharparenright}{\kern0pt}{\isachardoublequoteclose}\isanewline
\ \ \isakeyword{shows}\ {\isachardoublequoteopen}{\isasymforall}m{\isachardot}{\kern0pt}\ dim{\isacharunderscore}{\kern0pt}row\ {\isacharparenleft}{\kern0pt}f\ m{\isacharparenright}{\kern0pt}\ {\isacharequal}{\kern0pt}\ {\isadigit{2}}\ {\isasymand}\ dim{\isacharunderscore}{\kern0pt}col\ {\isacharparenleft}{\kern0pt}f\ m{\isacharparenright}{\kern0pt}\ {\isacharequal}{\kern0pt}\ {\isadigit{1}}{\isachardoublequoteclose}\isanewline
%
\isadelimproof
\ \ %
\endisadelimproof
%
\isatagproof
\isacommand{apply}\isamarkupfalse%
\ {\isacharparenleft}{\kern0pt}rule\ allI{\isacharparenright}{\kern0pt}\isanewline
\ \ \isacommand{apply}\isamarkupfalse%
\ {\isacharparenleft}{\kern0pt}rule\ conjI{\isacharparenright}{\kern0pt}\isanewline
\ \ \ \isacommand{apply}\isamarkupfalse%
\ {\isacharparenleft}{\kern0pt}simp\ add{\isacharcolon}{\kern0pt}\ assms\ ket{\isacharunderscore}{\kern0pt}vec{\isacharunderscore}{\kern0pt}def\ cpx{\isacharunderscore}{\kern0pt}vec{\isacharunderscore}{\kern0pt}length{\isacharunderscore}{\kern0pt}def{\isacharparenright}{\kern0pt}{\isacharplus}{\kern0pt}\isanewline
\ \ \isacommand{done}\isamarkupfalse%
%
\endisatagproof
{\isafoldproof}%
%
\isadelimproof
\isanewline
%
\endisadelimproof
\isanewline
\isacommand{lemma}\isamarkupfalse%
\ rev{\isacharunderscore}{\kern0pt}upto{\isacharcolon}{\kern0pt}\isanewline
\ \ \isakeyword{assumes}\ {\isachardoublequoteopen}n{\isadigit{1}}\ {\isasymle}\ n{\isadigit{2}}{\isachardoublequoteclose}\isanewline
\ \ \isakeyword{shows}\ {\isachardoublequoteopen}rev\ {\isacharbrackleft}{\kern0pt}n{\isadigit{1}}{\isachardot}{\kern0pt}{\isachardot}{\kern0pt}n{\isadigit{2}}{\isacharbrackright}{\kern0pt}\ {\isacharequal}{\kern0pt}\ n{\isadigit{2}}\ {\isacharhash}{\kern0pt}\ rev\ {\isacharbrackleft}{\kern0pt}n{\isadigit{1}}{\isachardot}{\kern0pt}{\isachardot}{\kern0pt}{\isacharparenleft}{\kern0pt}n{\isadigit{2}}{\isacharminus}{\kern0pt}{\isadigit{1}}{\isacharparenright}{\kern0pt}{\isacharbrackright}{\kern0pt}{\isachardoublequoteclose}\isanewline
%
\isadelimproof
\ \ %
\endisadelimproof
%
\isatagproof
\isacommand{apply}\isamarkupfalse%
\ {\isacharparenleft}{\kern0pt}simp{\isacharparenright}{\kern0pt}\isanewline
\ \ \isacommand{apply}\isamarkupfalse%
\ {\isacharparenleft}{\kern0pt}rule\ upto{\isacharunderscore}{\kern0pt}rec{\isadigit{2}}{\isacharparenright}{\kern0pt}\isanewline
\ \ \isacommand{apply}\isamarkupfalse%
\ {\isacharparenleft}{\kern0pt}simp\ add{\isacharcolon}{\kern0pt}assms{\isacharparenright}{\kern0pt}\isanewline
\ \ \isacommand{done}\isamarkupfalse%
%
\endisatagproof
{\isafoldproof}%
%
\isadelimproof
\isanewline
%
\endisadelimproof
\isanewline
\isacommand{lemma}\isamarkupfalse%
\ dim{\isacharunderscore}{\kern0pt}row{\isacharunderscore}{\kern0pt}kron{\isacharcolon}{\kern0pt}\isanewline
\ \ \isakeyword{shows}\ {\isachardoublequoteopen}dim{\isacharunderscore}{\kern0pt}row\ {\isacharparenleft}{\kern0pt}kron\ f\ xs{\isacharparenright}{\kern0pt}\ {\isacharequal}{\kern0pt}\ {\isacharparenleft}{\kern0pt}{\isasymProd}x{\isasymleftarrow}xs{\isachardot}{\kern0pt}\ dim{\isacharunderscore}{\kern0pt}row\ {\isacharparenleft}{\kern0pt}f\ x{\isacharparenright}{\kern0pt}{\isacharparenright}{\kern0pt}{\isachardoublequoteclose}\isanewline
%
\isadelimproof
%
\endisadelimproof
%
\isatagproof
\isacommand{proof}\isamarkupfalse%
\ {\isacharparenleft}{\kern0pt}induct\ xs{\isacharparenright}{\kern0pt}\isanewline
\ \ \isacommand{case}\isamarkupfalse%
\ Nil\isanewline
\ \ \isacommand{show}\isamarkupfalse%
\ {\isacharquery}{\kern0pt}case\ \isacommand{using}\isamarkupfalse%
\ kron{\isachardot}{\kern0pt}simps{\isacharparenleft}{\kern0pt}{\isadigit{1}}{\isacharparenright}{\kern0pt}\ prod{\isacharunderscore}{\kern0pt}list{\isacharunderscore}{\kern0pt}def\ \isacommand{by}\isamarkupfalse%
\ auto\isanewline
\isacommand{next}\isamarkupfalse%
\isanewline
\ \ \isacommand{case}\isamarkupfalse%
\ {\isacharparenleft}{\kern0pt}Cons\ a\ xs{\isacharparenright}{\kern0pt}\isanewline
\ \ \isacommand{assume}\isamarkupfalse%
\ HI{\isacharcolon}{\kern0pt}{\isachardoublequoteopen}dim{\isacharunderscore}{\kern0pt}row\ {\isacharparenleft}{\kern0pt}kron\ f\ xs{\isacharparenright}{\kern0pt}\ {\isacharequal}{\kern0pt}\ {\isacharparenleft}{\kern0pt}{\isasymProd}x{\isasymleftarrow}xs{\isachardot}{\kern0pt}\ dim{\isacharunderscore}{\kern0pt}row\ {\isacharparenleft}{\kern0pt}f\ x{\isacharparenright}{\kern0pt}{\isacharparenright}{\kern0pt}{\isachardoublequoteclose}\isanewline
\ \ \isacommand{have}\isamarkupfalse%
\ {\isachardoublequoteopen}dim{\isacharunderscore}{\kern0pt}row\ {\isacharparenleft}{\kern0pt}kron\ f\ {\isacharparenleft}{\kern0pt}a{\isacharhash}{\kern0pt}xs{\isacharparenright}{\kern0pt}{\isacharparenright}{\kern0pt}\ {\isacharequal}{\kern0pt}\ dim{\isacharunderscore}{\kern0pt}row\ {\isacharparenleft}{\kern0pt}{\isacharparenleft}{\kern0pt}f\ a{\isacharparenright}{\kern0pt}\ {\isasymOtimes}\ {\isacharparenleft}{\kern0pt}kron\ f\ xs{\isacharparenright}{\kern0pt}{\isacharparenright}{\kern0pt}{\isachardoublequoteclose}\ \isacommand{using}\isamarkupfalse%
\ kron{\isachardot}{\kern0pt}simps{\isacharparenleft}{\kern0pt}{\isadigit{2}}{\isacharparenright}{\kern0pt}\ \isacommand{by}\isamarkupfalse%
\ auto\isanewline
\ \ \isacommand{hence}\isamarkupfalse%
\ {\isachardoublequoteopen}{\isasymdots}\ {\isacharequal}{\kern0pt}\ {\isacharparenleft}{\kern0pt}dim{\isacharunderscore}{\kern0pt}row\ {\isacharparenleft}{\kern0pt}f\ a{\isacharparenright}{\kern0pt}{\isacharparenright}{\kern0pt}\ {\isacharasterisk}{\kern0pt}\ {\isacharparenleft}{\kern0pt}dim{\isacharunderscore}{\kern0pt}row\ {\isacharparenleft}{\kern0pt}kron\ f\ xs{\isacharparenright}{\kern0pt}{\isacharparenright}{\kern0pt}{\isachardoublequoteclose}\ \isacommand{by}\isamarkupfalse%
\ auto\isanewline
\ \ \isacommand{hence}\isamarkupfalse%
\ {\isachardoublequoteopen}{\isasymdots}\ {\isacharequal}{\kern0pt}\ {\isacharparenleft}{\kern0pt}dim{\isacharunderscore}{\kern0pt}row\ {\isacharparenleft}{\kern0pt}f\ a{\isacharparenright}{\kern0pt}{\isacharparenright}{\kern0pt}\ {\isacharasterisk}{\kern0pt}\ {\isacharparenleft}{\kern0pt}{\isasymProd}x{\isasymleftarrow}xs{\isachardot}{\kern0pt}\ dim{\isacharunderscore}{\kern0pt}row\ {\isacharparenleft}{\kern0pt}f\ x{\isacharparenright}{\kern0pt}{\isacharparenright}{\kern0pt}{\isachardoublequoteclose}\ \isacommand{using}\isamarkupfalse%
\ HI\ \isacommand{by}\isamarkupfalse%
\ auto\isanewline
\ \ \isacommand{hence}\isamarkupfalse%
\ {\isachardoublequoteopen}{\isasymdots}\ {\isacharequal}{\kern0pt}\ {\isacharparenleft}{\kern0pt}{\isasymProd}x{\isasymleftarrow}a\ {\isacharhash}{\kern0pt}\ xs{\isachardot}{\kern0pt}\ dim{\isacharunderscore}{\kern0pt}row\ {\isacharparenleft}{\kern0pt}f\ x{\isacharparenright}{\kern0pt}{\isacharparenright}{\kern0pt}{\isachardoublequoteclose}\ \isacommand{by}\isamarkupfalse%
\ auto\isanewline
\ \ \isacommand{thus}\isamarkupfalse%
\ {\isacharquery}{\kern0pt}case\ \isacommand{using}\isamarkupfalse%
\ HI\ \isacommand{by}\isamarkupfalse%
\ auto\isanewline
\isacommand{qed}\isamarkupfalse%
%
\endisatagproof
{\isafoldproof}%
%
\isadelimproof
\isanewline
%
\endisadelimproof
\isanewline
\isacommand{lemma}\isamarkupfalse%
\ dim{\isacharunderscore}{\kern0pt}col{\isacharunderscore}{\kern0pt}kron{\isacharcolon}{\kern0pt}\isanewline
\ \ \isakeyword{shows}\ {\isachardoublequoteopen}dim{\isacharunderscore}{\kern0pt}col\ {\isacharparenleft}{\kern0pt}kron\ f\ xs{\isacharparenright}{\kern0pt}\ {\isacharequal}{\kern0pt}\ {\isacharparenleft}{\kern0pt}{\isasymProd}x{\isasymleftarrow}xs{\isachardot}{\kern0pt}\ dim{\isacharunderscore}{\kern0pt}col\ {\isacharparenleft}{\kern0pt}f\ x{\isacharparenright}{\kern0pt}{\isacharparenright}{\kern0pt}{\isachardoublequoteclose}\isanewline
%
\isadelimproof
%
\endisadelimproof
%
\isatagproof
\isacommand{proof}\isamarkupfalse%
\ {\isacharparenleft}{\kern0pt}induct\ xs{\isacharparenright}{\kern0pt}\isanewline
\ \ \isacommand{case}\isamarkupfalse%
\ Nil\isanewline
\ \ \isacommand{show}\isamarkupfalse%
\ {\isacharquery}{\kern0pt}case\ \isacommand{using}\isamarkupfalse%
\ kron{\isachardot}{\kern0pt}simps{\isacharparenleft}{\kern0pt}{\isadigit{1}}{\isacharparenright}{\kern0pt}\ prod{\isacharunderscore}{\kern0pt}list{\isacharunderscore}{\kern0pt}def\ \isacommand{by}\isamarkupfalse%
\ auto\isanewline
\isacommand{next}\isamarkupfalse%
\isanewline
\ \ \isacommand{case}\isamarkupfalse%
\ {\isacharparenleft}{\kern0pt}Cons\ a\ xs{\isacharparenright}{\kern0pt}\isanewline
\ \ \isacommand{assume}\isamarkupfalse%
\ HI{\isacharcolon}{\kern0pt}{\isachardoublequoteopen}dim{\isacharunderscore}{\kern0pt}col\ {\isacharparenleft}{\kern0pt}kron\ f\ xs{\isacharparenright}{\kern0pt}\ {\isacharequal}{\kern0pt}\ {\isacharparenleft}{\kern0pt}{\isasymProd}x{\isasymleftarrow}xs{\isachardot}{\kern0pt}\ dim{\isacharunderscore}{\kern0pt}col\ {\isacharparenleft}{\kern0pt}f\ x{\isacharparenright}{\kern0pt}{\isacharparenright}{\kern0pt}{\isachardoublequoteclose}\isanewline
\ \ \isacommand{have}\isamarkupfalse%
\ {\isachardoublequoteopen}dim{\isacharunderscore}{\kern0pt}col\ {\isacharparenleft}{\kern0pt}kron\ f\ {\isacharparenleft}{\kern0pt}a{\isacharhash}{\kern0pt}xs{\isacharparenright}{\kern0pt}{\isacharparenright}{\kern0pt}\ {\isacharequal}{\kern0pt}\ dim{\isacharunderscore}{\kern0pt}col\ {\isacharparenleft}{\kern0pt}{\isacharparenleft}{\kern0pt}f\ a{\isacharparenright}{\kern0pt}\ {\isasymOtimes}\ {\isacharparenleft}{\kern0pt}kron\ f\ xs{\isacharparenright}{\kern0pt}{\isacharparenright}{\kern0pt}{\isachardoublequoteclose}\ \isacommand{using}\isamarkupfalse%
\ kron{\isachardot}{\kern0pt}simps{\isacharparenleft}{\kern0pt}{\isadigit{2}}{\isacharparenright}{\kern0pt}\ \isacommand{by}\isamarkupfalse%
\ auto\isanewline
\ \ \isacommand{hence}\isamarkupfalse%
\ {\isachardoublequoteopen}{\isasymdots}\ {\isacharequal}{\kern0pt}\ {\isacharparenleft}{\kern0pt}dim{\isacharunderscore}{\kern0pt}col\ {\isacharparenleft}{\kern0pt}f\ a{\isacharparenright}{\kern0pt}{\isacharparenright}{\kern0pt}\ {\isacharasterisk}{\kern0pt}\ {\isacharparenleft}{\kern0pt}dim{\isacharunderscore}{\kern0pt}col\ {\isacharparenleft}{\kern0pt}kron\ f\ xs{\isacharparenright}{\kern0pt}{\isacharparenright}{\kern0pt}{\isachardoublequoteclose}\ \isacommand{by}\isamarkupfalse%
\ auto\isanewline
\ \ \isacommand{hence}\isamarkupfalse%
\ {\isachardoublequoteopen}{\isasymdots}\ {\isacharequal}{\kern0pt}\ {\isacharparenleft}{\kern0pt}dim{\isacharunderscore}{\kern0pt}col\ {\isacharparenleft}{\kern0pt}f\ a{\isacharparenright}{\kern0pt}{\isacharparenright}{\kern0pt}\ {\isacharasterisk}{\kern0pt}\ {\isacharparenleft}{\kern0pt}{\isasymProd}x{\isasymleftarrow}xs{\isachardot}{\kern0pt}\ dim{\isacharunderscore}{\kern0pt}col\ {\isacharparenleft}{\kern0pt}f\ x{\isacharparenright}{\kern0pt}{\isacharparenright}{\kern0pt}{\isachardoublequoteclose}\ \isacommand{using}\isamarkupfalse%
\ HI\ \isacommand{by}\isamarkupfalse%
\ auto\isanewline
\ \ \isacommand{hence}\isamarkupfalse%
\ {\isachardoublequoteopen}{\isasymdots}\ {\isacharequal}{\kern0pt}\ {\isacharparenleft}{\kern0pt}{\isasymProd}x{\isasymleftarrow}a\ {\isacharhash}{\kern0pt}\ xs{\isachardot}{\kern0pt}\ dim{\isacharunderscore}{\kern0pt}col\ {\isacharparenleft}{\kern0pt}f\ x{\isacharparenright}{\kern0pt}{\isacharparenright}{\kern0pt}{\isachardoublequoteclose}\ \isacommand{by}\isamarkupfalse%
\ auto\isanewline
\ \ \isacommand{thus}\isamarkupfalse%
\ {\isacharquery}{\kern0pt}case\ \isacommand{using}\isamarkupfalse%
\ HI\ \isacommand{by}\isamarkupfalse%
\ auto\isanewline
\isacommand{qed}\isamarkupfalse%
%
\endisatagproof
{\isafoldproof}%
%
\isadelimproof
\isanewline
%
\endisadelimproof
\isanewline
\isacommand{lemma}\isamarkupfalse%
\ prod{\isacharunderscore}{\kern0pt}{\isadigit{2}}{\isacharunderscore}{\kern0pt}n{\isacharcolon}{\kern0pt}\isanewline
\ \ {\isachardoublequoteopen}{\isacharparenleft}{\kern0pt}{\isasymProd}x{\isasymleftarrow}map\ nat\ {\isacharparenleft}{\kern0pt}rev\ {\isacharbrackleft}{\kern0pt}{\isadigit{1}}{\isachardot}{\kern0pt}{\isachardot}{\kern0pt}int\ n{\isacharbrackright}{\kern0pt}{\isacharparenright}{\kern0pt}{\isachardot}{\kern0pt}\ {\isadigit{2}}{\isacharparenright}{\kern0pt}\ {\isacharequal}{\kern0pt}\ {\isadigit{2}}\ {\isacharcircum}{\kern0pt}\ n{\isachardoublequoteclose}\isanewline
%
\isadelimproof
\ \ %
\endisadelimproof
%
\isatagproof
\isacommand{apply}\isamarkupfalse%
\ {\isacharparenleft}{\kern0pt}induct\ n{\isacharparenright}{\kern0pt}\isanewline
\ \ \ \isacommand{apply}\isamarkupfalse%
\ {\isacharparenleft}{\kern0pt}simp\ add{\isacharcolon}{\kern0pt}\ rev{\isacharunderscore}{\kern0pt}upto{\isacharparenright}{\kern0pt}{\isacharplus}{\kern0pt}\isanewline
\ \ \isacommand{done}\isamarkupfalse%
%
\endisatagproof
{\isafoldproof}%
%
\isadelimproof
\isanewline
%
\endisadelimproof
\isanewline
\isacommand{lemma}\isamarkupfalse%
\ prod{\isacharunderscore}{\kern0pt}{\isadigit{2}}{\isacharunderscore}{\kern0pt}n{\isacharunderscore}{\kern0pt}b{\isacharcolon}{\kern0pt}\isanewline
\ \ {\isachardoublequoteopen}{\isacharparenleft}{\kern0pt}{\isasymProd}x{\isasymleftarrow}map\ nat\ {\isacharbrackleft}{\kern0pt}{\isadigit{1}}{\isachardot}{\kern0pt}{\isachardot}{\kern0pt}int\ n{\isacharbrackright}{\kern0pt}{\isachardot}{\kern0pt}\ {\isadigit{2}}{\isacharparenright}{\kern0pt}\ {\isacharequal}{\kern0pt}\ {\isadigit{2}}\ {\isacharcircum}{\kern0pt}\ n{\isachardoublequoteclose}\isanewline
%
\isadelimproof
\ \ %
\endisadelimproof
%
\isatagproof
\isacommand{apply}\isamarkupfalse%
\ {\isacharparenleft}{\kern0pt}induct\ n{\isacharparenright}{\kern0pt}\isanewline
\ \ \ \isacommand{apply}\isamarkupfalse%
\ simp\isanewline
\ \ \isacommand{apply}\isamarkupfalse%
\ {\isacharparenleft}{\kern0pt}simp\ add{\isacharcolon}{\kern0pt}\ upto{\isacharunderscore}{\kern0pt}rec{\isadigit{2}}\ power{\isacharunderscore}{\kern0pt}commutes{\isacharparenright}{\kern0pt}\isanewline
\ \ \isacommand{done}\isamarkupfalse%
%
\endisatagproof
{\isafoldproof}%
%
\isadelimproof
\isanewline
%
\endisadelimproof
\isanewline
\isacommand{lemma}\isamarkupfalse%
\ prod{\isacharunderscore}{\kern0pt}{\isadigit{1}}{\isacharunderscore}{\kern0pt}n{\isacharcolon}{\kern0pt}\isanewline
\ \ {\isachardoublequoteopen}{\isacharparenleft}{\kern0pt}{\isasymProd}x{\isasymleftarrow}map\ nat\ {\isacharparenleft}{\kern0pt}rev\ {\isacharbrackleft}{\kern0pt}{\isadigit{1}}{\isachardot}{\kern0pt}{\isachardot}{\kern0pt}int\ n{\isacharbrackright}{\kern0pt}{\isacharparenright}{\kern0pt}{\isachardot}{\kern0pt}\ {\isadigit{1}}{\isacharparenright}{\kern0pt}\ {\isacharequal}{\kern0pt}\ {\isadigit{1}}{\isachardoublequoteclose}\isanewline
%
\isadelimproof
\ \ %
\endisadelimproof
%
\isatagproof
\isacommand{apply}\isamarkupfalse%
\ {\isacharparenleft}{\kern0pt}induct\ n{\isacharparenright}{\kern0pt}\isanewline
\ \ \ \isacommand{apply}\isamarkupfalse%
\ {\isacharparenleft}{\kern0pt}simp\ add{\isacharcolon}{\kern0pt}\ rev{\isacharunderscore}{\kern0pt}upto{\isacharparenright}{\kern0pt}{\isacharplus}{\kern0pt}\isanewline
\ \ \isacommand{done}\isamarkupfalse%
%
\endisatagproof
{\isafoldproof}%
%
\isadelimproof
\isanewline
%
\endisadelimproof
\isanewline
\isacommand{lemma}\isamarkupfalse%
\ prod{\isacharunderscore}{\kern0pt}{\isadigit{1}}{\isacharunderscore}{\kern0pt}n{\isacharunderscore}{\kern0pt}b{\isacharcolon}{\kern0pt}\isanewline
\ \ {\isachardoublequoteopen}{\isacharparenleft}{\kern0pt}{\isasymProd}x{\isasymleftarrow}map\ nat\ {\isacharbrackleft}{\kern0pt}{\isadigit{1}}{\isachardot}{\kern0pt}{\isachardot}{\kern0pt}int\ n{\isacharbrackright}{\kern0pt}{\isachardot}{\kern0pt}\ Suc\ {\isadigit{0}}{\isacharparenright}{\kern0pt}\ {\isacharequal}{\kern0pt}\ Suc\ {\isadigit{0}}{\isachardoublequoteclose}\isanewline
%
\isadelimproof
\ \ %
\endisadelimproof
%
\isatagproof
\isacommand{apply}\isamarkupfalse%
\ {\isacharparenleft}{\kern0pt}induct\ n{\isacharparenright}{\kern0pt}\isanewline
\ \ \ \isacommand{apply}\isamarkupfalse%
\ simp\isanewline
\ \ \isacommand{apply}\isamarkupfalse%
\ {\isacharparenleft}{\kern0pt}simp\ add{\isacharcolon}{\kern0pt}\ upto{\isacharunderscore}{\kern0pt}rec{\isadigit{2}}{\isacharparenright}{\kern0pt}\isanewline
\ \ \isacommand{done}\isamarkupfalse%
%
\endisatagproof
{\isafoldproof}%
%
\isadelimproof
\isanewline
%
\endisadelimproof
\isanewline
\isacommand{lemma}\isamarkupfalse%
\ reverse{\isacharunderscore}{\kern0pt}qubits{\isacharunderscore}{\kern0pt}product{\isacharunderscore}{\kern0pt}representation{\isacharcolon}{\kern0pt}\isanewline
\ \ {\isachardoublequoteopen}reverse{\isacharunderscore}{\kern0pt}qubits\ n\ {\isacharasterisk}{\kern0pt}\ reverse{\isacharunderscore}{\kern0pt}QFT{\isacharunderscore}{\kern0pt}product{\isacharunderscore}{\kern0pt}representation\ j\ n\ {\isacharequal}{\kern0pt}\ QFT{\isacharunderscore}{\kern0pt}product{\isacharunderscore}{\kern0pt}representation\ j\ n{\isachardoublequoteclose}\isanewline
%
\isadelimproof
%
\endisadelimproof
%
\isatagproof
\isacommand{proof}\isamarkupfalse%
\ {\isacharminus}{\kern0pt}\isanewline
\ \ \isacommand{have}\isamarkupfalse%
\ {\isachardoublequoteopen}{\isacharparenleft}{\kern0pt}reverse{\isacharunderscore}{\kern0pt}qubits\ n{\isacharparenright}{\kern0pt}\ {\isacharasterisk}{\kern0pt}\ reverse{\isacharunderscore}{\kern0pt}QFT{\isacharunderscore}{\kern0pt}product{\isacharunderscore}{\kern0pt}representation\ j\ n\ {\isacharequal}{\kern0pt}\ {\isacharparenleft}{\kern0pt}reverse{\isacharunderscore}{\kern0pt}qubits\ n{\isacharparenright}{\kern0pt}\ {\isacharasterisk}{\kern0pt}\isanewline
\ \ \ \ \ \ \ {\isacharparenleft}{\kern0pt}{\isacharparenleft}{\kern0pt}{\isadigit{1}}{\isacharslash}{\kern0pt}sqrt{\isacharparenleft}{\kern0pt}{\isadigit{2}}{\isacharcircum}{\kern0pt}n{\isacharparenright}{\kern0pt}{\isacharparenright}{\kern0pt}\ {\isasymcdot}\isactrlsub m\ kron\ {\isacharparenleft}{\kern0pt}{\isasymlambda}l{\isachardot}{\kern0pt}\ {\isacharbar}{\kern0pt}zero{\isasymrangle}\ {\isacharplus}{\kern0pt}\ exp\ {\isacharparenleft}{\kern0pt}{\isadigit{2}}{\isacharasterisk}{\kern0pt}{\isasymi}{\isacharasterisk}{\kern0pt}pi{\isacharasterisk}{\kern0pt}j{\isacharslash}{\kern0pt}{\isadigit{2}}{\isacharcircum}{\kern0pt}l{\isacharparenright}{\kern0pt}\ {\isasymcdot}\isactrlsub m\ {\isacharbar}{\kern0pt}one{\isasymrangle}{\isacharparenright}{\kern0pt}\ {\isacharparenleft}{\kern0pt}map\ nat\ {\isacharparenleft}{\kern0pt}rev\ {\isacharbrackleft}{\kern0pt}{\isadigit{1}}{\isachardot}{\kern0pt}{\isachardot}{\kern0pt}int\ n{\isacharbrackright}{\kern0pt}{\isacharparenright}{\kern0pt}{\isacharparenright}{\kern0pt}{\isacharparenright}{\kern0pt}{\isachardoublequoteclose}\isanewline
\ \ \ \ \isacommand{using}\isamarkupfalse%
\ reverse{\isacharunderscore}{\kern0pt}QFT{\isacharunderscore}{\kern0pt}product{\isacharunderscore}{\kern0pt}representation{\isacharunderscore}{\kern0pt}def\ \isacommand{by}\isamarkupfalse%
\ simp\isanewline
\ \ \isacommand{also}\isamarkupfalse%
\ \isacommand{have}\isamarkupfalse%
\ {\isachardoublequoteopen}{\isasymdots}\ {\isacharequal}{\kern0pt}\ {\isacharparenleft}{\kern0pt}{\isadigit{1}}{\isacharslash}{\kern0pt}sqrt{\isacharparenleft}{\kern0pt}{\isadigit{2}}{\isacharcircum}{\kern0pt}n{\isacharparenright}{\kern0pt}{\isacharparenright}{\kern0pt}\ {\isasymcdot}\isactrlsub m\ {\isacharparenleft}{\kern0pt}{\isacharparenleft}{\kern0pt}reverse{\isacharunderscore}{\kern0pt}qubits\ n{\isacharparenright}{\kern0pt}\ {\isacharasterisk}{\kern0pt}\isanewline
\ \ \ \ \ \ \ \ \ \ \ \ \ \ \ \ kron\ {\isacharparenleft}{\kern0pt}{\isasymlambda}l{\isachardot}{\kern0pt}\ {\isacharbar}{\kern0pt}zero{\isasymrangle}\ {\isacharplus}{\kern0pt}\ exp\ {\isacharparenleft}{\kern0pt}{\isadigit{2}}{\isacharasterisk}{\kern0pt}{\isasymi}{\isacharasterisk}{\kern0pt}pi{\isacharasterisk}{\kern0pt}j{\isacharslash}{\kern0pt}{\isadigit{2}}{\isacharcircum}{\kern0pt}l{\isacharparenright}{\kern0pt}\ {\isasymcdot}\isactrlsub m\ {\isacharbar}{\kern0pt}one{\isasymrangle}{\isacharparenright}{\kern0pt}\ {\isacharparenleft}{\kern0pt}map\ nat\ {\isacharparenleft}{\kern0pt}rev\ {\isacharbrackleft}{\kern0pt}{\isadigit{1}}{\isachardot}{\kern0pt}{\isachardot}{\kern0pt}int\ n{\isacharbrackright}{\kern0pt}{\isacharparenright}{\kern0pt}{\isacharparenright}{\kern0pt}{\isacharparenright}{\kern0pt}{\isachardoublequoteclose}\isanewline
\ \ \isacommand{proof}\isamarkupfalse%
\ {\isacharparenleft}{\kern0pt}rule\ mult{\isacharunderscore}{\kern0pt}smult{\isacharunderscore}{\kern0pt}distrib{\isacharparenright}{\kern0pt}\isanewline
\ \ \ \ \isacommand{show}\isamarkupfalse%
\ {\isachardoublequoteopen}reverse{\isacharunderscore}{\kern0pt}qubits\ n\ {\isasymin}\ carrier{\isacharunderscore}{\kern0pt}mat\ {\isacharparenleft}{\kern0pt}{\isadigit{2}}{\isacharcircum}{\kern0pt}n{\isacharparenright}{\kern0pt}\ {\isacharparenleft}{\kern0pt}{\isadigit{2}}{\isacharcircum}{\kern0pt}n{\isacharparenright}{\kern0pt}{\isachardoublequoteclose}\ \isacommand{by}\isamarkupfalse%
\ simp\isanewline
\ \ \isacommand{next}\isamarkupfalse%
\isanewline
\ \ \ \ \isacommand{show}\isamarkupfalse%
\ {\isachardoublequoteopen}kron\ {\isacharparenleft}{\kern0pt}{\isasymlambda}l{\isachardot}{\kern0pt}\ {\isacharbar}{\kern0pt}zero{\isasymrangle}\ {\isacharplus}{\kern0pt}\ exp\ {\isacharparenleft}{\kern0pt}{\isadigit{2}}{\isacharasterisk}{\kern0pt}{\isasymi}{\isacharasterisk}{\kern0pt}pi{\isacharasterisk}{\kern0pt}j{\isacharslash}{\kern0pt}{\isadigit{2}}{\isacharcircum}{\kern0pt}l{\isacharparenright}{\kern0pt}\ {\isasymcdot}\isactrlsub m\ {\isacharbar}{\kern0pt}one{\isasymrangle}{\isacharparenright}{\kern0pt}\ {\isacharparenleft}{\kern0pt}map\ nat\ {\isacharparenleft}{\kern0pt}rev\ {\isacharbrackleft}{\kern0pt}{\isadigit{1}}{\isachardot}{\kern0pt}{\isachardot}{\kern0pt}int\ n{\isacharbrackright}{\kern0pt}{\isacharparenright}{\kern0pt}{\isacharparenright}{\kern0pt}\ \isanewline
\ \ \ \ \ \ \ \ \ \ {\isasymin}\ carrier{\isacharunderscore}{\kern0pt}mat\ {\isacharparenleft}{\kern0pt}{\isadigit{2}}{\isacharcircum}{\kern0pt}n{\isacharparenright}{\kern0pt}\ {\isadigit{1}}{\isachardoublequoteclose}\isanewline
\ \ \ \ \isacommand{proof}\isamarkupfalse%
\isanewline
\ \ \ \ \ \ \isacommand{show}\isamarkupfalse%
\ {\isachardoublequoteopen}dim{\isacharunderscore}{\kern0pt}row\ {\isacharparenleft}{\kern0pt}kron\ {\isacharparenleft}{\kern0pt}{\isasymlambda}{\isacharparenleft}{\kern0pt}l{\isacharcolon}{\kern0pt}{\isacharcolon}{\kern0pt}nat{\isacharparenright}{\kern0pt}{\isachardot}{\kern0pt}\ {\isacharbar}{\kern0pt}zero{\isasymrangle}\ {\isacharplus}{\kern0pt}\ exp\ {\isacharparenleft}{\kern0pt}{\isadigit{2}}{\isacharasterisk}{\kern0pt}{\isasymi}{\isacharasterisk}{\kern0pt}pi{\isacharasterisk}{\kern0pt}j{\isacharslash}{\kern0pt}{\isacharparenleft}{\kern0pt}{\isadigit{2}}{\isacharcircum}{\kern0pt}l{\isacharparenright}{\kern0pt}{\isacharparenright}{\kern0pt}\ {\isasymcdot}\isactrlsub m\ {\isacharbar}{\kern0pt}one{\isasymrangle}{\isacharparenright}{\kern0pt}\ {\isacharparenleft}{\kern0pt}map\ nat\ {\isacharparenleft}{\kern0pt}rev\ {\isacharbrackleft}{\kern0pt}{\isadigit{1}}{\isachardot}{\kern0pt}{\isachardot}{\kern0pt}n{\isacharbrackright}{\kern0pt}{\isacharparenright}{\kern0pt}{\isacharparenright}{\kern0pt}{\isacharparenright}{\kern0pt}\isanewline
\ \ \ \ \ \ \ \ \ \ {\isacharequal}{\kern0pt}\ {\isadigit{2}}\ {\isacharcircum}{\kern0pt}\ n{\isachardoublequoteclose}\isanewline
\ \ \ \ \ \ \isacommand{proof}\isamarkupfalse%
\ {\isacharminus}{\kern0pt}\isanewline
\ \ \ \ \ \ \ \ \isacommand{have}\isamarkupfalse%
\ a{\isadigit{1}}{\isacharcolon}{\kern0pt}{\isachardoublequoteopen}dim{\isacharunderscore}{\kern0pt}row\ {\isacharparenleft}{\kern0pt}kron\ {\isacharparenleft}{\kern0pt}{\isasymlambda}l{\isachardot}{\kern0pt}\ {\isacharbar}{\kern0pt}zero{\isasymrangle}\ {\isacharplus}{\kern0pt}\ exp\ {\isacharparenleft}{\kern0pt}{\isadigit{2}}\ {\isacharasterisk}{\kern0pt}\ {\isasymi}\ {\isacharasterisk}{\kern0pt}\ complex{\isacharunderscore}{\kern0pt}of{\isacharunderscore}{\kern0pt}real\ pi\ {\isacharasterisk}{\kern0pt}\ complex{\isacharunderscore}{\kern0pt}of{\isacharunderscore}{\kern0pt}nat\ j\ {\isacharslash}{\kern0pt}\ {\isadigit{2}}\ {\isacharcircum}{\kern0pt}\ l{\isacharparenright}{\kern0pt}\ {\isasymcdot}\isactrlsub m\ {\isacharbar}{\kern0pt}one{\isasymrangle}{\isacharparenright}{\kern0pt}\ {\isacharparenleft}{\kern0pt}map\ nat\ {\isacharparenleft}{\kern0pt}rev\ {\isacharbrackleft}{\kern0pt}{\isadigit{1}}{\isachardot}{\kern0pt}{\isachardot}{\kern0pt}int\ n{\isacharbrackright}{\kern0pt}{\isacharparenright}{\kern0pt}{\isacharparenright}{\kern0pt}{\isacharparenright}{\kern0pt}\isanewline
\ \ \ \ \ \ \ \ \ \ \ \ {\isacharequal}{\kern0pt}\ {\isacharparenleft}{\kern0pt}{\isasymProd}x{\isasymleftarrow}{\isacharparenleft}{\kern0pt}map\ nat\ {\isacharparenleft}{\kern0pt}rev\ {\isacharbrackleft}{\kern0pt}{\isadigit{1}}{\isachardot}{\kern0pt}{\isachardot}{\kern0pt}int\ n{\isacharbrackright}{\kern0pt}{\isacharparenright}{\kern0pt}{\isacharparenright}{\kern0pt}{\isachardot}{\kern0pt}\ {\isacharparenleft}{\kern0pt}dim{\isacharunderscore}{\kern0pt}row\ {\isacharparenleft}{\kern0pt}{\isacharparenleft}{\kern0pt}{\isasymlambda}l{\isachardot}{\kern0pt}\ {\isacharbar}{\kern0pt}zero{\isasymrangle}\ {\isacharplus}{\kern0pt}\ exp\ {\isacharparenleft}{\kern0pt}{\isadigit{2}}\ {\isacharasterisk}{\kern0pt}\ {\isasymi}\ {\isacharasterisk}{\kern0pt}\ complex{\isacharunderscore}{\kern0pt}of{\isacharunderscore}{\kern0pt}real\ pi\ {\isacharasterisk}{\kern0pt}\ complex{\isacharunderscore}{\kern0pt}of{\isacharunderscore}{\kern0pt}nat\ j\ {\isacharslash}{\kern0pt}\ {\isadigit{2}}\ {\isacharcircum}{\kern0pt}\ l{\isacharparenright}{\kern0pt}\ {\isasymcdot}\isactrlsub m\ {\isacharbar}{\kern0pt}one{\isasymrangle}{\isacharparenright}{\kern0pt}\ x{\isacharparenright}{\kern0pt}{\isacharparenright}{\kern0pt}{\isacharparenright}{\kern0pt}{\isachardoublequoteclose}\isanewline
\ \ \ \ \ \ \ \ \ \ \isacommand{using}\isamarkupfalse%
\ dim{\isacharunderscore}{\kern0pt}row{\isacharunderscore}{\kern0pt}kron\ \isacommand{by}\isamarkupfalse%
\ simp\isanewline
\ \ \ \ \ \ \ \ \isacommand{hence}\isamarkupfalse%
\ b{\isadigit{1}}{\isacharcolon}{\kern0pt}{\isachardoublequoteopen}{\isasymdots}\ {\isacharequal}{\kern0pt}\ {\isacharparenleft}{\kern0pt}{\isasymProd}x{\isasymleftarrow}{\isacharparenleft}{\kern0pt}map\ nat\ {\isacharparenleft}{\kern0pt}rev\ {\isacharbrackleft}{\kern0pt}{\isadigit{1}}{\isachardot}{\kern0pt}{\isachardot}{\kern0pt}int\ n{\isacharbrackright}{\kern0pt}{\isacharparenright}{\kern0pt}{\isacharparenright}{\kern0pt}{\isachardot}{\kern0pt}\ {\isadigit{2}}{\isacharparenright}{\kern0pt}{\isachardoublequoteclose}\ \isacommand{using}\isamarkupfalse%
\ prod{\isacharunderscore}{\kern0pt}rep{\isacharunderscore}{\kern0pt}fun\ \isacommand{by}\isamarkupfalse%
\ auto\isanewline
\ \ \ \ \ \ \ \ \isacommand{hence}\isamarkupfalse%
\ {\isachardoublequoteopen}{\isasymdots}\ {\isacharequal}{\kern0pt}\ {\isadigit{2}}\ {\isacharcircum}{\kern0pt}\ n{\isachardoublequoteclose}\ \isacommand{using}\isamarkupfalse%
\ prod{\isacharunderscore}{\kern0pt}{\isadigit{2}}{\isacharunderscore}{\kern0pt}n\ \isacommand{by}\isamarkupfalse%
\ simp\isanewline
\ \ \ \ \ \ \ \ \isacommand{thus}\isamarkupfalse%
\ {\isacharquery}{\kern0pt}thesis\ \isacommand{using}\isamarkupfalse%
\ a{\isadigit{1}}\ b{\isadigit{1}}\ \isacommand{by}\isamarkupfalse%
\ auto\isanewline
\ \ \ \ \ \ \isacommand{qed}\isamarkupfalse%
\isanewline
\ \ \ \ \isacommand{next}\isamarkupfalse%
\isanewline
\ \ \ \ \ \ \isacommand{show}\isamarkupfalse%
\ {\isachardoublequoteopen}dim{\isacharunderscore}{\kern0pt}col\ {\isacharparenleft}{\kern0pt}kron\ {\isacharparenleft}{\kern0pt}{\isasymlambda}{\isacharparenleft}{\kern0pt}l{\isacharcolon}{\kern0pt}{\isacharcolon}{\kern0pt}nat{\isacharparenright}{\kern0pt}{\isachardot}{\kern0pt}\ {\isacharbar}{\kern0pt}zero{\isasymrangle}\ {\isacharplus}{\kern0pt}\ exp\ {\isacharparenleft}{\kern0pt}{\isadigit{2}}{\isacharasterisk}{\kern0pt}{\isasymi}{\isacharasterisk}{\kern0pt}pi{\isacharasterisk}{\kern0pt}j{\isacharslash}{\kern0pt}{\isacharparenleft}{\kern0pt}{\isadigit{2}}{\isacharcircum}{\kern0pt}l{\isacharparenright}{\kern0pt}{\isacharparenright}{\kern0pt}\ {\isasymcdot}\isactrlsub m\ {\isacharbar}{\kern0pt}one{\isasymrangle}{\isacharparenright}{\kern0pt}\ {\isacharparenleft}{\kern0pt}map\ nat\ {\isacharparenleft}{\kern0pt}rev\ {\isacharbrackleft}{\kern0pt}{\isadigit{1}}{\isachardot}{\kern0pt}{\isachardot}{\kern0pt}n{\isacharbrackright}{\kern0pt}{\isacharparenright}{\kern0pt}{\isacharparenright}{\kern0pt}{\isacharparenright}{\kern0pt}\isanewline
\ \ \ \ \ \ \ \ \ \ \ \ {\isacharequal}{\kern0pt}\ {\isadigit{1}}{\isachardoublequoteclose}\isanewline
\ \ \ \ \ \ \isacommand{proof}\isamarkupfalse%
\ {\isacharminus}{\kern0pt}\isanewline
\ \ \ \ \ \ \ \ \isacommand{have}\isamarkupfalse%
\ a{\isadigit{2}}{\isacharcolon}{\kern0pt}{\isachardoublequoteopen}dim{\isacharunderscore}{\kern0pt}col\ {\isacharparenleft}{\kern0pt}kron\ {\isacharparenleft}{\kern0pt}{\isasymlambda}l{\isachardot}{\kern0pt}\ {\isacharbar}{\kern0pt}zero{\isasymrangle}\ {\isacharplus}{\kern0pt}\ exp\ {\isacharparenleft}{\kern0pt}{\isadigit{2}}\ {\isacharasterisk}{\kern0pt}\ {\isasymi}\ {\isacharasterisk}{\kern0pt}\ complex{\isacharunderscore}{\kern0pt}of{\isacharunderscore}{\kern0pt}real\ pi\ {\isacharasterisk}{\kern0pt}\ complex{\isacharunderscore}{\kern0pt}of{\isacharunderscore}{\kern0pt}nat\ j\ {\isacharslash}{\kern0pt}\ {\isadigit{2}}\ {\isacharcircum}{\kern0pt}\ l{\isacharparenright}{\kern0pt}\ {\isasymcdot}\isactrlsub m\ {\isacharbar}{\kern0pt}one{\isasymrangle}{\isacharparenright}{\kern0pt}\ {\isacharparenleft}{\kern0pt}map\ nat\ {\isacharparenleft}{\kern0pt}rev\ {\isacharbrackleft}{\kern0pt}{\isadigit{1}}{\isachardot}{\kern0pt}{\isachardot}{\kern0pt}int\ n{\isacharbrackright}{\kern0pt}{\isacharparenright}{\kern0pt}{\isacharparenright}{\kern0pt}{\isacharparenright}{\kern0pt}\isanewline
\ \ \ \ \ \ \ \ \ \ \ \ {\isacharequal}{\kern0pt}\ {\isacharparenleft}{\kern0pt}{\isasymProd}x{\isasymleftarrow}{\isacharparenleft}{\kern0pt}map\ nat\ {\isacharparenleft}{\kern0pt}rev\ {\isacharbrackleft}{\kern0pt}{\isadigit{1}}{\isachardot}{\kern0pt}{\isachardot}{\kern0pt}int\ n{\isacharbrackright}{\kern0pt}{\isacharparenright}{\kern0pt}{\isacharparenright}{\kern0pt}{\isachardot}{\kern0pt}\ {\isacharparenleft}{\kern0pt}dim{\isacharunderscore}{\kern0pt}col\ {\isacharparenleft}{\kern0pt}{\isacharparenleft}{\kern0pt}{\isasymlambda}l{\isachardot}{\kern0pt}\ {\isacharbar}{\kern0pt}zero{\isasymrangle}\ {\isacharplus}{\kern0pt}\ exp\ {\isacharparenleft}{\kern0pt}{\isadigit{2}}\ {\isacharasterisk}{\kern0pt}\ {\isasymi}\ {\isacharasterisk}{\kern0pt}\ complex{\isacharunderscore}{\kern0pt}of{\isacharunderscore}{\kern0pt}real\ pi\ {\isacharasterisk}{\kern0pt}\ complex{\isacharunderscore}{\kern0pt}of{\isacharunderscore}{\kern0pt}nat\ j\ {\isacharslash}{\kern0pt}\ {\isadigit{2}}\ {\isacharcircum}{\kern0pt}\ l{\isacharparenright}{\kern0pt}\ {\isasymcdot}\isactrlsub m\ {\isacharbar}{\kern0pt}one{\isasymrangle}{\isacharparenright}{\kern0pt}\ x{\isacharparenright}{\kern0pt}{\isacharparenright}{\kern0pt}{\isacharparenright}{\kern0pt}{\isachardoublequoteclose}\isanewline
\ \ \ \ \ \ \ \ \ \ \isacommand{using}\isamarkupfalse%
\ dim{\isacharunderscore}{\kern0pt}col{\isacharunderscore}{\kern0pt}kron\ \isacommand{by}\isamarkupfalse%
\ simp\isanewline
\ \ \ \ \ \ \ \ \isacommand{also}\isamarkupfalse%
\ \isacommand{have}\isamarkupfalse%
\ {\isachardoublequoteopen}{\isasymdots}\ {\isacharequal}{\kern0pt}\ {\isacharparenleft}{\kern0pt}{\isasymProd}x{\isasymleftarrow}{\isacharparenleft}{\kern0pt}map\ nat\ {\isacharparenleft}{\kern0pt}rev\ {\isacharbrackleft}{\kern0pt}{\isadigit{1}}{\isachardot}{\kern0pt}{\isachardot}{\kern0pt}int\ n{\isacharbrackright}{\kern0pt}{\isacharparenright}{\kern0pt}{\isacharparenright}{\kern0pt}{\isachardot}{\kern0pt}\ {\isadigit{1}}{\isacharparenright}{\kern0pt}{\isachardoublequoteclose}\ \isacommand{using}\isamarkupfalse%
\ prod{\isacharunderscore}{\kern0pt}rep{\isacharunderscore}{\kern0pt}fun\ \isacommand{by}\isamarkupfalse%
\ auto\isanewline
\ \ \ \ \ \ \ \ \isacommand{also}\isamarkupfalse%
\ \isacommand{have}\isamarkupfalse%
\ {\isachardoublequoteopen}{\isasymdots}\ {\isacharequal}{\kern0pt}\ {\isadigit{1}}{\isachardoublequoteclose}\ \isacommand{using}\isamarkupfalse%
\ prod{\isacharunderscore}{\kern0pt}{\isadigit{1}}{\isacharunderscore}{\kern0pt}n\ \isacommand{by}\isamarkupfalse%
\ metis\isanewline
\ \ \ \ \ \ \ \ \isacommand{finally}\isamarkupfalse%
\ \isacommand{show}\isamarkupfalse%
\ {\isacharquery}{\kern0pt}thesis\ \isacommand{using}\isamarkupfalse%
\ a{\isadigit{2}}\ \isacommand{by}\isamarkupfalse%
\ auto\isanewline
\ \ \ \ \ \ \isacommand{qed}\isamarkupfalse%
\isanewline
\ \ \ \ \isacommand{qed}\isamarkupfalse%
\isanewline
\ \ \isacommand{qed}\isamarkupfalse%
\isanewline
\ \ \isacommand{also}\isamarkupfalse%
\ \isacommand{have}\isamarkupfalse%
\ {\isachardoublequoteopen}{\isasymdots}\ {\isacharequal}{\kern0pt}\ {\isacharparenleft}{\kern0pt}{\isadigit{1}}\ {\isacharslash}{\kern0pt}\ sqrt\ {\isacharparenleft}{\kern0pt}{\isadigit{2}}{\isacharcircum}{\kern0pt}n{\isacharparenright}{\kern0pt}{\isacharparenright}{\kern0pt}\ {\isasymcdot}\isactrlsub m\ kron\ {\isacharparenleft}{\kern0pt}{\isasymlambda}l{\isachardot}{\kern0pt}\ {\isacharbar}{\kern0pt}zero{\isasymrangle}\ {\isacharplus}{\kern0pt}\ exp\ {\isacharparenleft}{\kern0pt}{\isadigit{2}}{\isacharasterisk}{\kern0pt}{\isasymi}{\isacharasterisk}{\kern0pt}pi{\isacharasterisk}{\kern0pt}j{\isacharslash}{\kern0pt}{\isadigit{2}}{\isacharcircum}{\kern0pt}l{\isacharparenright}{\kern0pt}\ {\isasymcdot}\isactrlsub m\ {\isacharbar}{\kern0pt}one{\isasymrangle}{\isacharparenright}{\kern0pt}\ {\isacharparenleft}{\kern0pt}map\ nat\ {\isacharbrackleft}{\kern0pt}{\isadigit{1}}{\isachardot}{\kern0pt}{\isachardot}{\kern0pt}int\ n{\isacharbrackright}{\kern0pt}{\isacharparenright}{\kern0pt}{\isachardoublequoteclose}\isanewline
\ \ \ \ \isacommand{using}\isamarkupfalse%
\ reverse{\isacharunderscore}{\kern0pt}qubits{\isacharunderscore}{\kern0pt}kron\ prod{\isacharunderscore}{\kern0pt}rep{\isacharunderscore}{\kern0pt}fun\ \isacommand{by}\isamarkupfalse%
\ presburger\isanewline
\ \ \isacommand{also}\isamarkupfalse%
\ \isacommand{have}\isamarkupfalse%
\ {\isachardoublequoteopen}{\isasymdots}\ {\isacharequal}{\kern0pt}\ QFT{\isacharunderscore}{\kern0pt}product{\isacharunderscore}{\kern0pt}representation\ j\ n{\isachardoublequoteclose}\ \isacommand{using}\isamarkupfalse%
\ QFT{\isacharunderscore}{\kern0pt}product{\isacharunderscore}{\kern0pt}representation{\isacharunderscore}{\kern0pt}def\ \isacommand{by}\isamarkupfalse%
\ simp\isanewline
\ \ \isacommand{finally}\isamarkupfalse%
\ \isacommand{show}\isamarkupfalse%
\ {\isacharquery}{\kern0pt}thesis\ \isacommand{by}\isamarkupfalse%
\ this\isanewline
\isacommand{qed}\isamarkupfalse%
%
\endisatagproof
{\isafoldproof}%
%
\isadelimproof
%
\endisadelimproof
%
\begin{isamarkuptext}%
Finally, we proof the correctness of the algorithm%
\end{isamarkuptext}\isamarkuptrue%
\isacommand{theorem}\isamarkupfalse%
\ ordered{\isacharunderscore}{\kern0pt}QFT{\isacharunderscore}{\kern0pt}is{\isacharunderscore}{\kern0pt}correct{\isacharcolon}{\kern0pt}\isanewline
\ \ \isakeyword{assumes}\ {\isachardoublequoteopen}j\ {\isacharless}{\kern0pt}\ {\isadigit{2}}{\isacharcircum}{\kern0pt}n{\isachardoublequoteclose}\isanewline
\ \ \isakeyword{shows}\ {\isachardoublequoteopen}{\isacharparenleft}{\kern0pt}ordered{\isacharunderscore}{\kern0pt}QFT\ n{\isacharparenright}{\kern0pt}\ {\isacharasterisk}{\kern0pt}\ {\isacharbar}{\kern0pt}state{\isacharunderscore}{\kern0pt}basis\ n\ j{\isasymrangle}\ {\isacharequal}{\kern0pt}\ QFT{\isacharunderscore}{\kern0pt}product{\isacharunderscore}{\kern0pt}representation\ j\ n{\isachardoublequoteclose}\isanewline
%
\isadelimproof
%
\endisadelimproof
%
\isatagproof
\isacommand{proof}\isamarkupfalse%
\ {\isacharminus}{\kern0pt}\isanewline
\ \ \isacommand{have}\isamarkupfalse%
\ {\isachardoublequoteopen}{\isacharparenleft}{\kern0pt}ordered{\isacharunderscore}{\kern0pt}QFT\ n{\isacharparenright}{\kern0pt}\ {\isacharasterisk}{\kern0pt}\ {\isacharbar}{\kern0pt}state{\isacharunderscore}{\kern0pt}basis\ n\ j{\isasymrangle}\ {\isacharequal}{\kern0pt}\ {\isacharparenleft}{\kern0pt}reverse{\isacharunderscore}{\kern0pt}qubits\ n{\isacharparenright}{\kern0pt}\ {\isacharasterisk}{\kern0pt}\ {\isacharparenleft}{\kern0pt}QFT\ n{\isacharparenright}{\kern0pt}\ {\isacharasterisk}{\kern0pt}\ {\isacharbar}{\kern0pt}state{\isacharunderscore}{\kern0pt}basis\ n\ j{\isasymrangle}{\isachardoublequoteclose}\isanewline
\ \ \ \ \isacommand{using}\isamarkupfalse%
\ ordered{\isacharunderscore}{\kern0pt}QFT{\isacharunderscore}{\kern0pt}def\ \isacommand{by}\isamarkupfalse%
\ simp\isanewline
\ \ \isacommand{also}\isamarkupfalse%
\ \isacommand{have}\isamarkupfalse%
\ {\isachardoublequoteopen}{\isasymdots}\ {\isacharequal}{\kern0pt}\ {\isacharparenleft}{\kern0pt}reverse{\isacharunderscore}{\kern0pt}qubits\ n{\isacharparenright}{\kern0pt}\ {\isacharasterisk}{\kern0pt}\ {\isacharparenleft}{\kern0pt}{\isacharparenleft}{\kern0pt}QFT\ n{\isacharparenright}{\kern0pt}\ {\isacharasterisk}{\kern0pt}\ {\isacharbar}{\kern0pt}state{\isacharunderscore}{\kern0pt}basis\ n\ j{\isasymrangle}{\isacharparenright}{\kern0pt}{\isachardoublequoteclose}\isanewline
\ \ \isacommand{proof}\isamarkupfalse%
\ {\isacharparenleft}{\kern0pt}rule\ assoc{\isacharunderscore}{\kern0pt}mult{\isacharunderscore}{\kern0pt}mat{\isacharparenright}{\kern0pt}\isanewline
\ \ \ \ \isacommand{show}\isamarkupfalse%
\ {\isachardoublequoteopen}reverse{\isacharunderscore}{\kern0pt}qubits\ n\ {\isasymin}\ carrier{\isacharunderscore}{\kern0pt}mat\ {\isacharparenleft}{\kern0pt}{\isadigit{2}}{\isacharcircum}{\kern0pt}n{\isacharparenright}{\kern0pt}\ {\isacharparenleft}{\kern0pt}{\isadigit{2}}{\isacharcircum}{\kern0pt}n{\isacharparenright}{\kern0pt}{\isachardoublequoteclose}\ \isacommand{by}\isamarkupfalse%
\ simp\isanewline
\ \ \isacommand{next}\isamarkupfalse%
\isanewline
\ \ \ \ \isacommand{show}\isamarkupfalse%
\ {\isachardoublequoteopen}QFT\ n\ {\isasymin}\ carrier{\isacharunderscore}{\kern0pt}mat\ {\isacharparenleft}{\kern0pt}{\isadigit{2}}{\isacharcircum}{\kern0pt}n{\isacharparenright}{\kern0pt}\ {\isacharparenleft}{\kern0pt}{\isadigit{2}}{\isacharcircum}{\kern0pt}n{\isacharparenright}{\kern0pt}{\isachardoublequoteclose}\ \isacommand{by}\isamarkupfalse%
\ simp\isanewline
\ \ \isacommand{next}\isamarkupfalse%
\isanewline
\ \ \ \ \isacommand{show}\isamarkupfalse%
\ {\isachardoublequoteopen}{\isacharbar}{\kern0pt}state{\isacharunderscore}{\kern0pt}basis\ n\ j{\isasymrangle}\ {\isasymin}\ carrier{\isacharunderscore}{\kern0pt}mat\ {\isacharparenleft}{\kern0pt}{\isadigit{2}}\ {\isacharcircum}{\kern0pt}\ n{\isacharparenright}{\kern0pt}\ {\isadigit{1}}{\isachardoublequoteclose}\ \isacommand{using}\isamarkupfalse%
\ state{\isacharunderscore}{\kern0pt}basis{\isacharunderscore}{\kern0pt}carrier{\isacharunderscore}{\kern0pt}mat\ \isacommand{by}\isamarkupfalse%
\ simp\isanewline
\ \ \isacommand{qed}\isamarkupfalse%
\isanewline
\ \ \isacommand{also}\isamarkupfalse%
\ \isacommand{have}\isamarkupfalse%
\ {\isachardoublequoteopen}{\isasymdots}\ {\isacharequal}{\kern0pt}\ {\isacharparenleft}{\kern0pt}reverse{\isacharunderscore}{\kern0pt}qubits\ n{\isacharparenright}{\kern0pt}\ {\isacharasterisk}{\kern0pt}\ reverse{\isacharunderscore}{\kern0pt}QFT{\isacharunderscore}{\kern0pt}product{\isacharunderscore}{\kern0pt}representation\ j\ n{\isachardoublequoteclose}\isanewline
\ \ \ \ \isacommand{using}\isamarkupfalse%
\ QFT{\isacharunderscore}{\kern0pt}is{\isacharunderscore}{\kern0pt}correct\ assms\ \isacommand{by}\isamarkupfalse%
\ simp\isanewline
\ \ \isacommand{also}\isamarkupfalse%
\ \isacommand{have}\isamarkupfalse%
\ {\isachardoublequoteopen}{\isasymdots}\ {\isacharequal}{\kern0pt}\ QFT{\isacharunderscore}{\kern0pt}product{\isacharunderscore}{\kern0pt}representation\ j\ n{\isachardoublequoteclose}\isanewline
\ \ \ \ \isacommand{using}\isamarkupfalse%
\ reverse{\isacharunderscore}{\kern0pt}qubits{\isacharunderscore}{\kern0pt}product{\isacharunderscore}{\kern0pt}representation\ \isacommand{by}\isamarkupfalse%
\ simp\isanewline
\ \ \isacommand{finally}\isamarkupfalse%
\ \isacommand{show}\isamarkupfalse%
\ {\isacharquery}{\kern0pt}thesis\ \isacommand{by}\isamarkupfalse%
\ this\isanewline
\isacommand{qed}\isamarkupfalse%
%
\endisatagproof
{\isafoldproof}%
%
\isadelimproof
%
\endisadelimproof
%
\isadelimdocument
%
\endisadelimdocument
%
\isatagdocument
%
\isamarkupsection{Unitarity%
}
\isamarkuptrue%
%
\endisatagdocument
{\isafolddocument}%
%
\isadelimdocument
%
\endisadelimdocument
%
\begin{isamarkuptext}%
Although unitarity is not required to proof QFT's correctness, in this section we will prove
it, i.e., QFT and ordered\_QFT functions create quantum gates and QFT product representation is a 
quantum state.%
\end{isamarkuptext}\isamarkuptrue%
\isacommand{lemma}\isamarkupfalse%
\ state{\isacharunderscore}{\kern0pt}basis{\isacharunderscore}{\kern0pt}is{\isacharunderscore}{\kern0pt}state{\isacharcolon}{\kern0pt}\isanewline
\ \ \isakeyword{assumes}\ {\isachardoublequoteopen}j\ {\isacharless}{\kern0pt}\ n{\isachardoublequoteclose}\isanewline
\ \ \isakeyword{shows}\ {\isachardoublequoteopen}state\ n\ {\isacharbar}{\kern0pt}state{\isacharunderscore}{\kern0pt}basis\ n\ j{\isasymrangle}{\isachardoublequoteclose}\isanewline
%
\isadelimproof
%
\endisadelimproof
%
\isatagproof
\isacommand{proof}\isamarkupfalse%
\isanewline
\ \ \isacommand{show}\isamarkupfalse%
\ {\isachardoublequoteopen}dim{\isacharunderscore}{\kern0pt}col\ {\isacharbar}{\kern0pt}state{\isacharunderscore}{\kern0pt}basis\ n\ j{\isasymrangle}\ {\isacharequal}{\kern0pt}\ {\isadigit{1}}{\isachardoublequoteclose}\ \isacommand{by}\isamarkupfalse%
\ {\isacharparenleft}{\kern0pt}simp\ add{\isacharcolon}{\kern0pt}\ ket{\isacharunderscore}{\kern0pt}vec{\isacharunderscore}{\kern0pt}def{\isacharparenright}{\kern0pt}\isanewline
\ \ \isacommand{show}\isamarkupfalse%
\ {\isachardoublequoteopen}dim{\isacharunderscore}{\kern0pt}row\ {\isacharbar}{\kern0pt}state{\isacharunderscore}{\kern0pt}basis\ n\ j{\isasymrangle}\ {\isacharequal}{\kern0pt}\ {\isadigit{2}}{\isacharcircum}{\kern0pt}n{\isachardoublequoteclose}\ \isacommand{by}\isamarkupfalse%
\ {\isacharparenleft}{\kern0pt}simp\ add{\isacharcolon}{\kern0pt}\ ket{\isacharunderscore}{\kern0pt}vec{\isacharunderscore}{\kern0pt}def\ state{\isacharunderscore}{\kern0pt}basis{\isacharunderscore}{\kern0pt}def{\isacharparenright}{\kern0pt}\isanewline
\ \ \isacommand{show}\isamarkupfalse%
\ {\isachardoublequoteopen}{\isasymparallel}Matrix{\isachardot}{\kern0pt}col\ {\isacharbar}{\kern0pt}state{\isacharunderscore}{\kern0pt}basis\ n\ j{\isasymrangle}\ {\isadigit{0}}{\isasymparallel}\ {\isacharequal}{\kern0pt}\ {\isadigit{1}}{\isachardoublequoteclose}\isanewline
\ \ \ \ \isacommand{by}\isamarkupfalse%
\ {\isacharparenleft}{\kern0pt}metis\ assms\ ket{\isacharunderscore}{\kern0pt}vec{\isacharunderscore}{\kern0pt}col\ less{\isacharunderscore}{\kern0pt}exp\ order{\isacharunderscore}{\kern0pt}less{\isacharunderscore}{\kern0pt}trans\ state{\isacharunderscore}{\kern0pt}basis{\isacharunderscore}{\kern0pt}def\ unit{\isacharunderscore}{\kern0pt}cpx{\isacharunderscore}{\kern0pt}vec{\isacharunderscore}{\kern0pt}length{\isacharparenright}{\kern0pt}\isanewline
\isacommand{qed}\isamarkupfalse%
%
\endisatagproof
{\isafoldproof}%
%
\isadelimproof
\isanewline
%
\endisadelimproof
\isanewline
\isacommand{lemma}\isamarkupfalse%
\ R{\isacharunderscore}{\kern0pt}dagger{\isacharunderscore}{\kern0pt}mat{\isacharcolon}{\kern0pt}\isanewline
\ \ \isakeyword{shows}\ {\isachardoublequoteopen}{\isacharparenleft}{\kern0pt}R\ k{\isacharparenright}{\kern0pt}\isactrlsup {\isasymdagger}\ {\isacharequal}{\kern0pt}\ Matrix{\isachardot}{\kern0pt}mat\ {\isadigit{2}}\ {\isadigit{2}}\ {\isacharparenleft}{\kern0pt}{\isasymlambda}{\isacharparenleft}{\kern0pt}i{\isacharcomma}{\kern0pt}j{\isacharparenright}{\kern0pt}{\isachardot}{\kern0pt}\ if\ i{\isasymnoteq}j\ then\ {\isadigit{0}}\ else\ {\isacharparenleft}{\kern0pt}if\ i{\isacharequal}{\kern0pt}{\isadigit{0}}\ then\ {\isadigit{1}}\ else\ exp{\isacharparenleft}{\kern0pt}{\isacharminus}{\kern0pt}{\isadigit{2}}{\isacharasterisk}{\kern0pt}pi{\isacharasterisk}{\kern0pt}{\isasymi}{\isacharslash}{\kern0pt}{\isadigit{2}}{\isacharcircum}{\kern0pt}k{\isacharparenright}{\kern0pt}{\isacharparenright}{\kern0pt}{\isacharparenright}{\kern0pt}{\isachardoublequoteclose}\isanewline
%
\isadelimproof
%
\endisadelimproof
%
\isatagproof
\isacommand{proof}\isamarkupfalse%
\isanewline
\ \ \isacommand{define}\isamarkupfalse%
\ m\ \isakeyword{where}\ {\isachardoublequoteopen}m\ {\isacharequal}{\kern0pt}\ Matrix{\isachardot}{\kern0pt}mat\ {\isadigit{2}}\ {\isadigit{2}}\ \isanewline
\ \ {\isacharparenleft}{\kern0pt}{\isasymlambda}{\isacharparenleft}{\kern0pt}i{\isacharcomma}{\kern0pt}j{\isacharparenright}{\kern0pt}{\isachardot}{\kern0pt}\ if\ i{\isasymnoteq}j\ then\ {\isadigit{0}}\ else\ {\isacharparenleft}{\kern0pt}if\ i{\isacharequal}{\kern0pt}{\isadigit{0}}\ then\ {\isadigit{1}}\ else\ exp{\isacharparenleft}{\kern0pt}{\isacharminus}{\kern0pt}{\isadigit{2}}{\isacharasterisk}{\kern0pt}pi{\isacharasterisk}{\kern0pt}{\isasymi}{\isacharslash}{\kern0pt}{\isadigit{2}}{\isacharcircum}{\kern0pt}k{\isacharparenright}{\kern0pt}{\isacharparenright}{\kern0pt}{\isacharparenright}{\kern0pt}{\isachardoublequoteclose}\isanewline
\ \ \isacommand{thus}\isamarkupfalse%
\ {\isachardoublequoteopen}{\isasymAnd}i\ j{\isachardot}{\kern0pt}\ i\ {\isacharless}{\kern0pt}\ dim{\isacharunderscore}{\kern0pt}row\ m\ {\isasymLongrightarrow}\ j\ {\isacharless}{\kern0pt}\ dim{\isacharunderscore}{\kern0pt}col\ m\ {\isasymLongrightarrow}\ R\ k\isactrlsup {\isasymdagger}\ {\isachardollar}{\kern0pt}{\isachardollar}{\kern0pt}\ {\isacharparenleft}{\kern0pt}i{\isacharcomma}{\kern0pt}\ j{\isacharparenright}{\kern0pt}\ {\isacharequal}{\kern0pt}\ m\ {\isachardollar}{\kern0pt}{\isachardollar}{\kern0pt}\ {\isacharparenleft}{\kern0pt}i{\isacharcomma}{\kern0pt}\ j{\isacharparenright}{\kern0pt}{\isachardoublequoteclose}\isanewline
\ \ \isacommand{proof}\isamarkupfalse%
\ {\isacharminus}{\kern0pt}\isanewline
\ \ \ \ \isacommand{fix}\isamarkupfalse%
\ i\ j\isanewline
\ \ \ \ \isacommand{assume}\isamarkupfalse%
\ {\isachardoublequoteopen}i\ {\isacharless}{\kern0pt}\ dim{\isacharunderscore}{\kern0pt}row\ m{\isachardoublequoteclose}\isanewline
\ \ \ \ \isacommand{hence}\isamarkupfalse%
\ i{\isadigit{2}}{\isacharcolon}{\kern0pt}{\isachardoublequoteopen}i\ {\isacharless}{\kern0pt}\ {\isadigit{2}}{\isachardoublequoteclose}\ \isacommand{using}\isamarkupfalse%
\ m{\isacharunderscore}{\kern0pt}def\ \isacommand{by}\isamarkupfalse%
\ auto\isanewline
\ \ \ \ \isacommand{assume}\isamarkupfalse%
\ {\isachardoublequoteopen}j\ {\isacharless}{\kern0pt}\ dim{\isacharunderscore}{\kern0pt}col\ m{\isachardoublequoteclose}\isanewline
\ \ \ \ \isacommand{hence}\isamarkupfalse%
\ j{\isadigit{2}}{\isacharcolon}{\kern0pt}{\isachardoublequoteopen}j\ {\isacharless}{\kern0pt}\ {\isadigit{2}}{\isachardoublequoteclose}\ \isacommand{using}\isamarkupfalse%
\ m{\isacharunderscore}{\kern0pt}def\ \isacommand{by}\isamarkupfalse%
\ auto\isanewline
\ \ \ \ \isacommand{show}\isamarkupfalse%
\ {\isachardoublequoteopen}R\ k\isactrlsup {\isasymdagger}\ {\isachardollar}{\kern0pt}{\isachardollar}{\kern0pt}\ {\isacharparenleft}{\kern0pt}i{\isacharcomma}{\kern0pt}\ j{\isacharparenright}{\kern0pt}\ {\isacharequal}{\kern0pt}\ m\ {\isachardollar}{\kern0pt}{\isachardollar}{\kern0pt}\ {\isacharparenleft}{\kern0pt}i{\isacharcomma}{\kern0pt}\ j{\isacharparenright}{\kern0pt}{\isachardoublequoteclose}\isanewline
\ \ \ \ \isacommand{proof}\isamarkupfalse%
\ {\isacharparenleft}{\kern0pt}rule\ disjE{\isacharparenright}{\kern0pt}\isanewline
\ \ \ \ \ \ \isacommand{show}\isamarkupfalse%
\ {\isachardoublequoteopen}i\ {\isacharequal}{\kern0pt}\ {\isadigit{0}}\ {\isasymor}\ i\ {\isacharequal}{\kern0pt}\ {\isadigit{1}}{\isachardoublequoteclose}\ \isacommand{using}\isamarkupfalse%
\ i{\isadigit{2}}\ \isacommand{by}\isamarkupfalse%
\ auto\isanewline
\ \ \ \ \isacommand{next}\isamarkupfalse%
\isanewline
\ \ \ \ \ \ \isacommand{assume}\isamarkupfalse%
\ i{\isadigit{0}}{\isacharcolon}{\kern0pt}{\isachardoublequoteopen}i\ {\isacharequal}{\kern0pt}\ {\isadigit{0}}{\isachardoublequoteclose}\isanewline
\ \ \ \ \ \ \isacommand{show}\isamarkupfalse%
\ {\isachardoublequoteopen}R\ k\isactrlsup {\isasymdagger}\ {\isachardollar}{\kern0pt}{\isachardollar}{\kern0pt}\ {\isacharparenleft}{\kern0pt}i{\isacharcomma}{\kern0pt}\ j{\isacharparenright}{\kern0pt}\ {\isacharequal}{\kern0pt}\ m\ {\isachardollar}{\kern0pt}{\isachardollar}{\kern0pt}\ {\isacharparenleft}{\kern0pt}i{\isacharcomma}{\kern0pt}\ j{\isacharparenright}{\kern0pt}{\isachardoublequoteclose}\isanewline
\ \ \ \ \ \ \isacommand{proof}\isamarkupfalse%
\ {\isacharparenleft}{\kern0pt}rule\ disjE{\isacharparenright}{\kern0pt}\isanewline
\ \ \ \ \ \ \ \ \isacommand{show}\isamarkupfalse%
\ {\isachardoublequoteopen}j\ {\isacharequal}{\kern0pt}\ {\isadigit{0}}\ {\isasymor}\ j\ {\isacharequal}{\kern0pt}\ {\isadigit{1}}{\isachardoublequoteclose}\ \isacommand{using}\isamarkupfalse%
\ j{\isadigit{2}}\ \isacommand{by}\isamarkupfalse%
\ auto\isanewline
\ \ \ \ \ \ \isacommand{next}\isamarkupfalse%
\ \isanewline
\ \ \ \ \ \ \ \ \isacommand{assume}\isamarkupfalse%
\ j{\isadigit{0}}{\isacharcolon}{\kern0pt}{\isachardoublequoteopen}j\ {\isacharequal}{\kern0pt}\ {\isadigit{0}}{\isachardoublequoteclose}\isanewline
\ \ \ \ \ \ \ \ \isacommand{show}\isamarkupfalse%
\ {\isachardoublequoteopen}R\ k\isactrlsup {\isasymdagger}\ {\isachardollar}{\kern0pt}{\isachardollar}{\kern0pt}\ {\isacharparenleft}{\kern0pt}i{\isacharcomma}{\kern0pt}\ j{\isacharparenright}{\kern0pt}\ {\isacharequal}{\kern0pt}\ m\ {\isachardollar}{\kern0pt}{\isachardollar}{\kern0pt}\ {\isacharparenleft}{\kern0pt}i{\isacharcomma}{\kern0pt}\ j{\isacharparenright}{\kern0pt}{\isachardoublequoteclose}\ \isanewline
\ \ \ \ \ \ \ \ \isacommand{proof}\isamarkupfalse%
\ {\isacharminus}{\kern0pt}\isanewline
\ \ \ \ \ \ \ \ \ \ \isacommand{have}\isamarkupfalse%
\ {\isachardoublequoteopen}R\ k\isactrlsup {\isasymdagger}\ {\isachardollar}{\kern0pt}{\isachardollar}{\kern0pt}\ {\isacharparenleft}{\kern0pt}{\isadigit{0}}{\isacharcomma}{\kern0pt}{\isadigit{0}}{\isacharparenright}{\kern0pt}\ {\isacharequal}{\kern0pt}\ cnj\ {\isacharparenleft}{\kern0pt}R\ k\ {\isachardollar}{\kern0pt}{\isachardollar}{\kern0pt}\ {\isacharparenleft}{\kern0pt}{\isadigit{0}}{\isacharcomma}{\kern0pt}{\isadigit{0}}{\isacharparenright}{\kern0pt}{\isacharparenright}{\kern0pt}{\isachardoublequoteclose}\ \isanewline
\ \ \ \ \ \ \ \ \ \ \ \ \isacommand{using}\isamarkupfalse%
\ dagger{\isacharunderscore}{\kern0pt}def\isanewline
\ \ \ \ \ \ \ \ \ \ \ \ \isacommand{by}\isamarkupfalse%
\ {\isacharparenleft}{\kern0pt}metis\ {\isacharparenleft}{\kern0pt}no{\isacharunderscore}{\kern0pt}types{\isacharcomma}{\kern0pt}\ lifting{\isacharparenright}{\kern0pt}\ One{\isacharunderscore}{\kern0pt}nat{\isacharunderscore}{\kern0pt}def\ R{\isacharunderscore}{\kern0pt}def\ Suc{\isacharunderscore}{\kern0pt}{\isadigit{1}}\ Suc{\isacharunderscore}{\kern0pt}eq{\isacharunderscore}{\kern0pt}plus{\isadigit{1}}\ \isanewline
\ \ \ \ \ \ \ \ \ \ \ \ \ \ \ \ Tensor{\isachardot}{\kern0pt}mat{\isacharunderscore}{\kern0pt}of{\isacharunderscore}{\kern0pt}cols{\isacharunderscore}{\kern0pt}list{\isacharunderscore}{\kern0pt}def\ dim{\isacharunderscore}{\kern0pt}col{\isacharunderscore}{\kern0pt}mat{\isacharparenleft}{\kern0pt}{\isadigit{1}}{\isacharparenright}{\kern0pt}\ dim{\isacharunderscore}{\kern0pt}row{\isacharunderscore}{\kern0pt}mat{\isacharparenleft}{\kern0pt}{\isadigit{1}}{\isacharparenright}{\kern0pt}\ index{\isacharunderscore}{\kern0pt}mat{\isacharparenleft}{\kern0pt}{\isadigit{1}}{\isacharparenright}{\kern0pt}\ list{\isachardot}{\kern0pt}size{\isacharparenleft}{\kern0pt}{\isadigit{3}}{\isacharparenright}{\kern0pt}\isanewline
\ \ \ \ \ \ \ \ \ \ \ \ \ \ \ \ list{\isachardot}{\kern0pt}size{\isacharparenleft}{\kern0pt}{\isadigit{4}}{\isacharparenright}{\kern0pt}\ old{\isachardot}{\kern0pt}prod{\isachardot}{\kern0pt}case\ power{\isacharunderscore}{\kern0pt}eq{\isacharunderscore}{\kern0pt}{\isadigit{0}}{\isacharunderscore}{\kern0pt}iff\ power{\isacharunderscore}{\kern0pt}zero{\isacharunderscore}{\kern0pt}numeral{\isacharparenright}{\kern0pt}\isanewline
\ \ \ \ \ \ \ \ \ \ \isacommand{also}\isamarkupfalse%
\ \isacommand{have}\isamarkupfalse%
\ {\isachardoublequoteopen}{\isasymdots}\ {\isacharequal}{\kern0pt}\ {\isadigit{1}}{\isachardoublequoteclose}\ \isanewline
\ \ \ \ \ \ \ \ \ \ \ \ \isacommand{using}\isamarkupfalse%
\ R{\isacharunderscore}{\kern0pt}def\ mat{\isacharunderscore}{\kern0pt}of{\isacharunderscore}{\kern0pt}cols{\isacharunderscore}{\kern0pt}list{\isacharunderscore}{\kern0pt}def\isanewline
\ \ \ \ \ \ \ \ \ \ \ \ \isacommand{by}\isamarkupfalse%
\ {\isacharparenleft}{\kern0pt}metis\ One{\isacharunderscore}{\kern0pt}nat{\isacharunderscore}{\kern0pt}def\ Suc{\isacharunderscore}{\kern0pt}{\isadigit{1}}\ Suc{\isacharunderscore}{\kern0pt}eq{\isacharunderscore}{\kern0pt}plus{\isadigit{1}}\ complex{\isacharunderscore}{\kern0pt}cnj{\isacharunderscore}{\kern0pt}one{\isacharunderscore}{\kern0pt}iff\ index{\isacharunderscore}{\kern0pt}mat{\isacharunderscore}{\kern0pt}of{\isacharunderscore}{\kern0pt}cols{\isacharunderscore}{\kern0pt}list\ \isanewline
\ \ \ \ \ \ \ \ \ \ \ \ \ \ \ \ list{\isachardot}{\kern0pt}size{\isacharparenleft}{\kern0pt}{\isadigit{3}}{\isacharparenright}{\kern0pt}\ list{\isachardot}{\kern0pt}size{\isacharparenleft}{\kern0pt}{\isadigit{4}}{\isacharparenright}{\kern0pt}\ nth{\isacharunderscore}{\kern0pt}Cons{\isacharunderscore}{\kern0pt}{\isadigit{0}}\ pos{\isadigit{2}}{\isacharparenright}{\kern0pt}\isanewline
\ \ \ \ \ \ \ \ \ \ \isacommand{also}\isamarkupfalse%
\ \isacommand{have}\isamarkupfalse%
\ {\isachardoublequoteopen}{\isasymdots}\ {\isacharequal}{\kern0pt}\ m\ {\isachardollar}{\kern0pt}{\isachardollar}{\kern0pt}\ {\isacharparenleft}{\kern0pt}{\isadigit{0}}{\isacharcomma}{\kern0pt}{\isadigit{0}}{\isacharparenright}{\kern0pt}{\isachardoublequoteclose}\ \isacommand{using}\isamarkupfalse%
\ m{\isacharunderscore}{\kern0pt}def\ \isacommand{by}\isamarkupfalse%
\ simp\isanewline
\ \ \ \ \ \ \ \ \ \ \isacommand{finally}\isamarkupfalse%
\ \isacommand{show}\isamarkupfalse%
\ {\isacharquery}{\kern0pt}thesis\ \isacommand{using}\isamarkupfalse%
\ i{\isadigit{0}}\ j{\isadigit{0}}\ \isacommand{by}\isamarkupfalse%
\ auto\isanewline
\ \ \ \ \ \ \ \ \isacommand{qed}\isamarkupfalse%
\isanewline
\ \ \ \ \ \ \isacommand{next}\isamarkupfalse%
\isanewline
\ \ \ \ \ \ \ \ \isacommand{assume}\isamarkupfalse%
\ j{\isadigit{1}}{\isacharcolon}{\kern0pt}{\isachardoublequoteopen}j\ {\isacharequal}{\kern0pt}\ {\isadigit{1}}{\isachardoublequoteclose}\isanewline
\ \ \ \ \ \ \ \ \isacommand{show}\isamarkupfalse%
\ {\isachardoublequoteopen}R\ k\isactrlsup {\isasymdagger}\ {\isachardollar}{\kern0pt}{\isachardollar}{\kern0pt}\ {\isacharparenleft}{\kern0pt}i{\isacharcomma}{\kern0pt}\ j{\isacharparenright}{\kern0pt}\ {\isacharequal}{\kern0pt}\ m\ {\isachardollar}{\kern0pt}{\isachardollar}{\kern0pt}\ {\isacharparenleft}{\kern0pt}i{\isacharcomma}{\kern0pt}\ j{\isacharparenright}{\kern0pt}{\isachardoublequoteclose}\isanewline
\ \ \ \ \ \ \ \ \isacommand{proof}\isamarkupfalse%
\ {\isacharminus}{\kern0pt}\isanewline
\ \ \ \ \ \ \ \ \ \ \isacommand{have}\isamarkupfalse%
\ {\isachardoublequoteopen}R\ k\isactrlsup {\isasymdagger}\ {\isachardollar}{\kern0pt}{\isachardollar}{\kern0pt}\ {\isacharparenleft}{\kern0pt}{\isadigit{0}}{\isacharcomma}{\kern0pt}{\isadigit{1}}{\isacharparenright}{\kern0pt}\ {\isacharequal}{\kern0pt}\ cnj\ {\isacharparenleft}{\kern0pt}R\ k\ {\isachardollar}{\kern0pt}{\isachardollar}{\kern0pt}\ {\isacharparenleft}{\kern0pt}{\isadigit{1}}{\isacharcomma}{\kern0pt}{\isadigit{0}}{\isacharparenright}{\kern0pt}{\isacharparenright}{\kern0pt}{\isachardoublequoteclose}\isanewline
\ \ \ \ \ \ \ \ \ \ \ \ \isacommand{using}\isamarkupfalse%
\ dagger{\isacharunderscore}{\kern0pt}def\ \isanewline
\ \ \ \ \ \ \ \ \ \ \ \ \isacommand{by}\isamarkupfalse%
\ {\isacharparenleft}{\kern0pt}metis\ {\isacharparenleft}{\kern0pt}no{\isacharunderscore}{\kern0pt}types{\isacharcomma}{\kern0pt}\ lifting{\isacharparenright}{\kern0pt}\ One{\isacharunderscore}{\kern0pt}nat{\isacharunderscore}{\kern0pt}def\ R{\isacharunderscore}{\kern0pt}def\ Suc{\isacharunderscore}{\kern0pt}{\isadigit{1}}\ Suc{\isacharunderscore}{\kern0pt}eq{\isacharunderscore}{\kern0pt}plus{\isadigit{1}}\ \isanewline
\ \ \ \ \ \ \ \ \ \ \ \ \ \ \ \ Tensor{\isachardot}{\kern0pt}mat{\isacharunderscore}{\kern0pt}of{\isacharunderscore}{\kern0pt}cols{\isacharunderscore}{\kern0pt}list{\isacharunderscore}{\kern0pt}def\ {\isacartoucheopen}j\ {\isacharless}{\kern0pt}\ dim{\isacharunderscore}{\kern0pt}col\ m{\isacartoucheclose}\ dim{\isacharunderscore}{\kern0pt}col{\isacharunderscore}{\kern0pt}mat{\isacharparenleft}{\kern0pt}{\isadigit{1}}{\isacharparenright}{\kern0pt}\ dim{\isacharunderscore}{\kern0pt}row{\isacharunderscore}{\kern0pt}mat{\isacharparenleft}{\kern0pt}{\isadigit{1}}{\isacharparenright}{\kern0pt}\ \isanewline
\ \ \ \ \ \ \ \ \ \ \ \ \ \ \ \ index{\isacharunderscore}{\kern0pt}mat{\isacharparenleft}{\kern0pt}{\isadigit{1}}{\isacharparenright}{\kern0pt}\ j{\isadigit{1}}\ list{\isachardot}{\kern0pt}size{\isacharparenleft}{\kern0pt}{\isadigit{3}}{\isacharparenright}{\kern0pt}\ list{\isachardot}{\kern0pt}size{\isacharparenleft}{\kern0pt}{\isadigit{4}}{\isacharparenright}{\kern0pt}\ m{\isacharunderscore}{\kern0pt}def\ old{\isachardot}{\kern0pt}prod{\isachardot}{\kern0pt}case\ pos{\isadigit{2}}{\isacharparenright}{\kern0pt}\isanewline
\ \ \ \ \ \ \ \ \ \ \isacommand{also}\isamarkupfalse%
\ \isacommand{have}\isamarkupfalse%
\ {\isachardoublequoteopen}{\isasymdots}\ {\isacharequal}{\kern0pt}\ {\isadigit{0}}{\isachardoublequoteclose}\isanewline
\ \ \ \ \ \ \ \ \ \ \ \ \isacommand{using}\isamarkupfalse%
\ R{\isacharunderscore}{\kern0pt}def\ mat{\isacharunderscore}{\kern0pt}of{\isacharunderscore}{\kern0pt}cols{\isacharunderscore}{\kern0pt}list{\isacharunderscore}{\kern0pt}def\isanewline
\ \ \ \ \ \ \ \ \ \ \ \ \isacommand{by}\isamarkupfalse%
\ {\isacharparenleft}{\kern0pt}metis\ {\isacharparenleft}{\kern0pt}no{\isacharunderscore}{\kern0pt}types{\isacharcomma}{\kern0pt}\ lifting{\isacharparenright}{\kern0pt}\ One{\isacharunderscore}{\kern0pt}nat{\isacharunderscore}{\kern0pt}def\ Suc{\isacharunderscore}{\kern0pt}{\isadigit{1}}\ Suc{\isacharunderscore}{\kern0pt}eq{\isacharunderscore}{\kern0pt}plus{\isadigit{1}}\ {\isacartoucheopen}j\ {\isacharless}{\kern0pt}\ dim{\isacharunderscore}{\kern0pt}col\ m{\isacartoucheclose}\ \isanewline
\ \ \ \ \ \ \ \ \ \ \ \ \ \ \ \ complex{\isacharunderscore}{\kern0pt}cnj{\isacharunderscore}{\kern0pt}zero{\isacharunderscore}{\kern0pt}iff\ dim{\isacharunderscore}{\kern0pt}col{\isacharunderscore}{\kern0pt}mat{\isacharparenleft}{\kern0pt}{\isadigit{1}}{\isacharparenright}{\kern0pt}\ index{\isacharunderscore}{\kern0pt}mat{\isacharunderscore}{\kern0pt}of{\isacharunderscore}{\kern0pt}cols{\isacharunderscore}{\kern0pt}list\ j{\isadigit{1}}\ list{\isachardot}{\kern0pt}size{\isacharparenleft}{\kern0pt}{\isadigit{3}}{\isacharparenright}{\kern0pt}\ \isanewline
\ \ \ \ \ \ \ \ \ \ \ \ \ \ \ \ list{\isachardot}{\kern0pt}size{\isacharparenleft}{\kern0pt}{\isadigit{4}}{\isacharparenright}{\kern0pt}\ m{\isacharunderscore}{\kern0pt}def\ nth{\isacharunderscore}{\kern0pt}Cons{\isacharunderscore}{\kern0pt}{\isadigit{0}}\ nth{\isacharunderscore}{\kern0pt}Cons{\isacharunderscore}{\kern0pt}Suc\ pos{\isadigit{2}}{\isacharparenright}{\kern0pt}\isanewline
\ \ \ \ \ \ \ \ \ \ \isacommand{also}\isamarkupfalse%
\ \isacommand{have}\isamarkupfalse%
\ {\isachardoublequoteopen}{\isasymdots}\ {\isacharequal}{\kern0pt}\ m\ {\isachardollar}{\kern0pt}{\isachardollar}{\kern0pt}\ {\isacharparenleft}{\kern0pt}{\isadigit{0}}{\isacharcomma}{\kern0pt}{\isadigit{1}}{\isacharparenright}{\kern0pt}{\isachardoublequoteclose}\ \isacommand{using}\isamarkupfalse%
\ m{\isacharunderscore}{\kern0pt}def\ \isacommand{by}\isamarkupfalse%
\ auto\isanewline
\ \ \ \ \ \ \ \ \ \ \isacommand{finally}\isamarkupfalse%
\ \isacommand{show}\isamarkupfalse%
\ {\isacharquery}{\kern0pt}thesis\ \isacommand{using}\isamarkupfalse%
\ i{\isadigit{0}}\ j{\isadigit{1}}\ \isacommand{by}\isamarkupfalse%
\ auto\isanewline
\ \ \ \ \ \ \ \ \isacommand{qed}\isamarkupfalse%
\isanewline
\ \ \ \ \ \ \isacommand{qed}\isamarkupfalse%
\isanewline
\ \ \ \ \isacommand{next}\isamarkupfalse%
\isanewline
\ \ \ \ \ \ \isacommand{assume}\isamarkupfalse%
\ i{\isadigit{1}}{\isacharcolon}{\kern0pt}{\isachardoublequoteopen}i\ {\isacharequal}{\kern0pt}\ {\isadigit{1}}{\isachardoublequoteclose}\isanewline
\ \ \ \ \ \ \isacommand{show}\isamarkupfalse%
\ {\isachardoublequoteopen}R\ k\isactrlsup {\isasymdagger}\ {\isachardollar}{\kern0pt}{\isachardollar}{\kern0pt}\ {\isacharparenleft}{\kern0pt}i{\isacharcomma}{\kern0pt}\ j{\isacharparenright}{\kern0pt}\ {\isacharequal}{\kern0pt}\ m\ {\isachardollar}{\kern0pt}{\isachardollar}{\kern0pt}\ {\isacharparenleft}{\kern0pt}i{\isacharcomma}{\kern0pt}\ j{\isacharparenright}{\kern0pt}{\isachardoublequoteclose}\isanewline
\ \ \ \ \ \ \isacommand{proof}\isamarkupfalse%
\ {\isacharparenleft}{\kern0pt}rule\ disjE{\isacharparenright}{\kern0pt}\isanewline
\ \ \ \ \ \ \ \ \isacommand{show}\isamarkupfalse%
\ {\isachardoublequoteopen}j\ {\isacharequal}{\kern0pt}\ {\isadigit{0}}\ {\isasymor}\ j\ {\isacharequal}{\kern0pt}\ {\isadigit{1}}{\isachardoublequoteclose}\ \isacommand{using}\isamarkupfalse%
\ j{\isadigit{2}}\ \isacommand{by}\isamarkupfalse%
\ auto\isanewline
\ \ \ \ \ \ \isacommand{next}\isamarkupfalse%
\ \isanewline
\ \ \ \ \ \ \ \ \isacommand{assume}\isamarkupfalse%
\ j{\isadigit{0}}{\isacharcolon}{\kern0pt}{\isachardoublequoteopen}j\ {\isacharequal}{\kern0pt}\ {\isadigit{0}}{\isachardoublequoteclose}\isanewline
\ \ \ \ \ \ \ \ \isacommand{show}\isamarkupfalse%
\ {\isachardoublequoteopen}R\ k\isactrlsup {\isasymdagger}\ {\isachardollar}{\kern0pt}{\isachardollar}{\kern0pt}\ {\isacharparenleft}{\kern0pt}i{\isacharcomma}{\kern0pt}\ j{\isacharparenright}{\kern0pt}\ {\isacharequal}{\kern0pt}\ m\ {\isachardollar}{\kern0pt}{\isachardollar}{\kern0pt}\ {\isacharparenleft}{\kern0pt}i{\isacharcomma}{\kern0pt}\ j{\isacharparenright}{\kern0pt}{\isachardoublequoteclose}\isanewline
\ \ \ \ \ \ \ \ \isacommand{proof}\isamarkupfalse%
\ {\isacharminus}{\kern0pt}\isanewline
\ \ \ \ \ \ \ \ \ \ \isacommand{have}\isamarkupfalse%
\ {\isachardoublequoteopen}R\ k\isactrlsup {\isasymdagger}\ {\isachardollar}{\kern0pt}{\isachardollar}{\kern0pt}\ {\isacharparenleft}{\kern0pt}{\isadigit{1}}{\isacharcomma}{\kern0pt}{\isadigit{0}}{\isacharparenright}{\kern0pt}\ {\isacharequal}{\kern0pt}\ cnj\ {\isacharparenleft}{\kern0pt}R\ k\ {\isachardollar}{\kern0pt}{\isachardollar}{\kern0pt}\ {\isacharparenleft}{\kern0pt}{\isadigit{0}}{\isacharcomma}{\kern0pt}{\isadigit{1}}{\isacharparenright}{\kern0pt}{\isacharparenright}{\kern0pt}{\isachardoublequoteclose}\isanewline
\ \ \ \ \ \ \ \ \ \ \ \ \isacommand{using}\isamarkupfalse%
\ dagger{\isacharunderscore}{\kern0pt}def\isanewline
\ \ \ \ \ \ \ \ \ \ \ \ \isacommand{by}\isamarkupfalse%
\ {\isacharparenleft}{\kern0pt}metis\ {\isacharparenleft}{\kern0pt}no{\isacharunderscore}{\kern0pt}types{\isacharcomma}{\kern0pt}\ lifting{\isacharparenright}{\kern0pt}\ One{\isacharunderscore}{\kern0pt}nat{\isacharunderscore}{\kern0pt}def\ R{\isacharunderscore}{\kern0pt}def\ Suc{\isacharunderscore}{\kern0pt}{\isadigit{1}}\ Suc{\isacharunderscore}{\kern0pt}eq{\isacharunderscore}{\kern0pt}plus{\isadigit{1}}\ \isanewline
\ \ \ \ \ \ \ \ \ \ \ \ \ \ \ \ Tensor{\isachardot}{\kern0pt}mat{\isacharunderscore}{\kern0pt}of{\isacharunderscore}{\kern0pt}cols{\isacharunderscore}{\kern0pt}list{\isacharunderscore}{\kern0pt}def\ dim{\isacharunderscore}{\kern0pt}col{\isacharunderscore}{\kern0pt}mat{\isacharparenleft}{\kern0pt}{\isadigit{1}}{\isacharparenright}{\kern0pt}\ dim{\isacharunderscore}{\kern0pt}row{\isacharunderscore}{\kern0pt}mat{\isacharparenleft}{\kern0pt}{\isadigit{1}}{\isacharparenright}{\kern0pt}\ index{\isacharunderscore}{\kern0pt}mat{\isacharparenleft}{\kern0pt}{\isadigit{1}}{\isacharparenright}{\kern0pt}\ \isanewline
\ \ \ \ \ \ \ \ \ \ \ \ \ \ \ \ less{\isacharunderscore}{\kern0pt}Suc{\isacharunderscore}{\kern0pt}numeral\ list{\isachardot}{\kern0pt}size{\isacharparenleft}{\kern0pt}{\isadigit{3}}{\isacharparenright}{\kern0pt}\ list{\isachardot}{\kern0pt}size{\isacharparenleft}{\kern0pt}{\isadigit{4}}{\isacharparenright}{\kern0pt}\ old{\isachardot}{\kern0pt}prod{\isachardot}{\kern0pt}case\ power{\isacharunderscore}{\kern0pt}eq{\isacharunderscore}{\kern0pt}{\isadigit{0}}{\isacharunderscore}{\kern0pt}iff\ \isanewline
\ \ \ \ \ \ \ \ \ \ \ \ \ \ \ \ power{\isacharunderscore}{\kern0pt}zero{\isacharunderscore}{\kern0pt}numeral\ pred{\isacharunderscore}{\kern0pt}numeral{\isacharunderscore}{\kern0pt}simps{\isacharparenleft}{\kern0pt}{\isadigit{2}}{\isacharparenright}{\kern0pt}{\isacharparenright}{\kern0pt}\isanewline
\ \ \ \ \ \ \ \ \ \ \isacommand{also}\isamarkupfalse%
\ \isacommand{have}\isamarkupfalse%
\ {\isachardoublequoteopen}{\isasymdots}\ {\isacharequal}{\kern0pt}\ {\isadigit{0}}{\isachardoublequoteclose}\isanewline
\ \ \ \ \ \ \ \ \ \ \ \ \isacommand{using}\isamarkupfalse%
\ R{\isacharunderscore}{\kern0pt}def\ mat{\isacharunderscore}{\kern0pt}of{\isacharunderscore}{\kern0pt}cols{\isacharunderscore}{\kern0pt}list{\isacharunderscore}{\kern0pt}def\isanewline
\ \ \ \ \ \ \ \ \ \ \ \ \isacommand{by}\isamarkupfalse%
\ {\isacharparenleft}{\kern0pt}metis\ One{\isacharunderscore}{\kern0pt}nat{\isacharunderscore}{\kern0pt}def\ Suc{\isacharunderscore}{\kern0pt}eq{\isacharunderscore}{\kern0pt}plus{\isadigit{1}}\ complex{\isacharunderscore}{\kern0pt}cnj{\isacharunderscore}{\kern0pt}zero{\isacharunderscore}{\kern0pt}iff\ index{\isacharunderscore}{\kern0pt}mat{\isacharunderscore}{\kern0pt}of{\isacharunderscore}{\kern0pt}cols{\isacharunderscore}{\kern0pt}list\isanewline
\ \ \ \ \ \ \ \ \ \ \ \ \ \ \ \ less{\isacharunderscore}{\kern0pt}Suc{\isacharunderscore}{\kern0pt}eq{\isacharunderscore}{\kern0pt}{\isadigit{0}}{\isacharunderscore}{\kern0pt}disj\ list{\isachardot}{\kern0pt}size{\isacharparenleft}{\kern0pt}{\isadigit{4}}{\isacharparenright}{\kern0pt}\ nth{\isacharunderscore}{\kern0pt}Cons{\isacharunderscore}{\kern0pt}{\isadigit{0}}\ nth{\isacharunderscore}{\kern0pt}Cons{\isacharunderscore}{\kern0pt}Suc\ pos{\isadigit{2}}{\isacharparenright}{\kern0pt}\isanewline
\ \ \ \ \ \ \ \ \ \ \isacommand{also}\isamarkupfalse%
\ \isacommand{have}\isamarkupfalse%
\ {\isachardoublequoteopen}{\isasymdots}\ {\isacharequal}{\kern0pt}\ m\ {\isachardollar}{\kern0pt}{\isachardollar}{\kern0pt}\ {\isacharparenleft}{\kern0pt}{\isadigit{1}}{\isacharcomma}{\kern0pt}{\isadigit{0}}{\isacharparenright}{\kern0pt}{\isachardoublequoteclose}\ \isacommand{using}\isamarkupfalse%
\ m{\isacharunderscore}{\kern0pt}def\ \isacommand{by}\isamarkupfalse%
\ simp\isanewline
\ \ \ \ \ \ \ \ \ \ \isacommand{finally}\isamarkupfalse%
\ \isacommand{show}\isamarkupfalse%
\ {\isacharquery}{\kern0pt}thesis\ \isacommand{using}\isamarkupfalse%
\ i{\isadigit{1}}\ j{\isadigit{0}}\ \isacommand{by}\isamarkupfalse%
\ simp\isanewline
\ \ \ \ \ \ \ \ \isacommand{qed}\isamarkupfalse%
\isanewline
\ \ \ \ \ \ \isacommand{next}\isamarkupfalse%
\isanewline
\ \ \ \ \ \ \ \ \isacommand{assume}\isamarkupfalse%
\ j{\isadigit{1}}{\isacharcolon}{\kern0pt}{\isachardoublequoteopen}j\ {\isacharequal}{\kern0pt}\ {\isadigit{1}}{\isachardoublequoteclose}\isanewline
\ \ \ \ \ \ \ \ \isacommand{show}\isamarkupfalse%
\ {\isachardoublequoteopen}R\ k\isactrlsup {\isasymdagger}\ {\isachardollar}{\kern0pt}{\isachardollar}{\kern0pt}\ {\isacharparenleft}{\kern0pt}i{\isacharcomma}{\kern0pt}\ j{\isacharparenright}{\kern0pt}\ {\isacharequal}{\kern0pt}\ m\ {\isachardollar}{\kern0pt}{\isachardollar}{\kern0pt}\ {\isacharparenleft}{\kern0pt}i{\isacharcomma}{\kern0pt}\ j{\isacharparenright}{\kern0pt}{\isachardoublequoteclose}\ \isanewline
\ \ \ \ \ \ \ \ \isacommand{proof}\isamarkupfalse%
\ {\isacharminus}{\kern0pt}\isanewline
\ \ \ \ \ \ \ \ \ \ \isacommand{have}\isamarkupfalse%
\ {\isachardoublequoteopen}R\ k\isactrlsup {\isasymdagger}\ {\isachardollar}{\kern0pt}{\isachardollar}{\kern0pt}\ {\isacharparenleft}{\kern0pt}{\isadigit{1}}{\isacharcomma}{\kern0pt}{\isadigit{1}}{\isacharparenright}{\kern0pt}\ {\isacharequal}{\kern0pt}\ cnj\ {\isacharparenleft}{\kern0pt}R\ k\ {\isachardollar}{\kern0pt}{\isachardollar}{\kern0pt}\ {\isacharparenleft}{\kern0pt}{\isadigit{1}}{\isacharcomma}{\kern0pt}{\isadigit{1}}{\isacharparenright}{\kern0pt}{\isacharparenright}{\kern0pt}{\isachardoublequoteclose}\isanewline
\ \ \ \ \ \ \ \ \ \ \ \ \isacommand{using}\isamarkupfalse%
\ dagger{\isacharunderscore}{\kern0pt}def\isanewline
\ \ \ \ \ \ \ \ \ \ \ \ \isacommand{by}\isamarkupfalse%
\ {\isacharparenleft}{\kern0pt}metis\ {\isacharparenleft}{\kern0pt}no{\isacharunderscore}{\kern0pt}types{\isacharcomma}{\kern0pt}\ lifting{\isacharparenright}{\kern0pt}\ One{\isacharunderscore}{\kern0pt}nat{\isacharunderscore}{\kern0pt}def\ R{\isacharunderscore}{\kern0pt}def\ Suc{\isacharunderscore}{\kern0pt}{\isadigit{1}}\ Suc{\isacharunderscore}{\kern0pt}eq{\isacharunderscore}{\kern0pt}plus{\isadigit{1}}\ \isanewline
\ \ \ \ \ \ \ \ \ \ \ \ \ \ \ \ Tensor{\isachardot}{\kern0pt}mat{\isacharunderscore}{\kern0pt}of{\isacharunderscore}{\kern0pt}cols{\isacharunderscore}{\kern0pt}list{\isacharunderscore}{\kern0pt}def\ dim{\isacharunderscore}{\kern0pt}col{\isacharunderscore}{\kern0pt}mat{\isacharparenleft}{\kern0pt}{\isadigit{1}}{\isacharparenright}{\kern0pt}\ dim{\isacharunderscore}{\kern0pt}row{\isacharunderscore}{\kern0pt}mat{\isacharparenleft}{\kern0pt}{\isadigit{1}}{\isacharparenright}{\kern0pt}\ index{\isacharunderscore}{\kern0pt}mat{\isacharparenleft}{\kern0pt}{\isadigit{1}}{\isacharparenright}{\kern0pt}\ \isanewline
\ \ \ \ \ \ \ \ \ \ \ \ \ \ \ \ less{\isacharunderscore}{\kern0pt}Suc{\isacharunderscore}{\kern0pt}numeral\ list{\isachardot}{\kern0pt}size{\isacharparenleft}{\kern0pt}{\isadigit{3}}{\isacharparenright}{\kern0pt}\ list{\isachardot}{\kern0pt}size{\isacharparenleft}{\kern0pt}{\isadigit{4}}{\isacharparenright}{\kern0pt}\ old{\isachardot}{\kern0pt}prod{\isachardot}{\kern0pt}case\ power{\isacharunderscore}{\kern0pt}eq{\isacharunderscore}{\kern0pt}{\isadigit{0}}{\isacharunderscore}{\kern0pt}iff\ \isanewline
\ \ \ \ \ \ \ \ \ \ \ \ \ \ \ \ power{\isacharunderscore}{\kern0pt}zero{\isacharunderscore}{\kern0pt}numeral\ pred{\isacharunderscore}{\kern0pt}numeral{\isacharunderscore}{\kern0pt}simps{\isacharparenleft}{\kern0pt}{\isadigit{2}}{\isacharparenright}{\kern0pt}{\isacharparenright}{\kern0pt}\isanewline
\ \ \ \ \ \ \ \ \ \ \isacommand{also}\isamarkupfalse%
\ \isacommand{have}\isamarkupfalse%
\ {\isachardoublequoteopen}{\isasymdots}\ {\isacharequal}{\kern0pt}\ cnj\ {\isacharparenleft}{\kern0pt}exp{\isacharparenleft}{\kern0pt}{\isadigit{2}}{\isacharasterisk}{\kern0pt}pi{\isacharasterisk}{\kern0pt}{\isasymi}{\isacharslash}{\kern0pt}{\isadigit{2}}{\isacharcircum}{\kern0pt}k{\isacharparenright}{\kern0pt}{\isacharparenright}{\kern0pt}{\isachardoublequoteclose}\isanewline
\ \ \ \ \ \ \ \ \ \ \ \ \isacommand{using}\isamarkupfalse%
\ R{\isacharunderscore}{\kern0pt}def\ mat{\isacharunderscore}{\kern0pt}of{\isacharunderscore}{\kern0pt}cols{\isacharunderscore}{\kern0pt}list{\isacharunderscore}{\kern0pt}def\ \isanewline
\ \ \ \ \ \ \ \ \ \ \ \ \isacommand{by}\isamarkupfalse%
\ {\isacharparenleft}{\kern0pt}metis\ One{\isacharunderscore}{\kern0pt}nat{\isacharunderscore}{\kern0pt}def\ Suc{\isacharunderscore}{\kern0pt}{\isadigit{1}}\ Suc{\isacharunderscore}{\kern0pt}eq{\isacharunderscore}{\kern0pt}plus{\isadigit{1}}\ index{\isacharunderscore}{\kern0pt}mat{\isacharunderscore}{\kern0pt}of{\isacharunderscore}{\kern0pt}cols{\isacharunderscore}{\kern0pt}list\ lessI\ list{\isachardot}{\kern0pt}size{\isacharparenleft}{\kern0pt}{\isadigit{3}}{\isacharparenright}{\kern0pt}\ \isanewline
\ \ \ \ \ \ \ \ \ \ \ \ \ \ \ \ list{\isachardot}{\kern0pt}size{\isacharparenleft}{\kern0pt}{\isadigit{4}}{\isacharparenright}{\kern0pt}\ nth{\isacharunderscore}{\kern0pt}Cons{\isacharunderscore}{\kern0pt}{\isadigit{0}}\ nth{\isacharunderscore}{\kern0pt}Cons{\isacharunderscore}{\kern0pt}Suc{\isacharparenright}{\kern0pt}\isanewline
\ \ \ \ \ \ \ \ \ \ \isacommand{also}\isamarkupfalse%
\ \isacommand{have}\isamarkupfalse%
\ {\isachardoublequoteopen}{\isasymdots}\ {\isacharequal}{\kern0pt}\ exp\ {\isacharparenleft}{\kern0pt}{\isacharminus}{\kern0pt}{\isadigit{2}}{\isacharasterisk}{\kern0pt}pi{\isacharasterisk}{\kern0pt}{\isasymi}{\isacharslash}{\kern0pt}{\isadigit{2}}{\isacharcircum}{\kern0pt}k{\isacharparenright}{\kern0pt}{\isachardoublequoteclose}\isanewline
\ \ \ \ \ \ \ \ \ \ \ \ \isacommand{by}\isamarkupfalse%
\ {\isacharparenleft}{\kern0pt}smt\ {\isacharparenleft}{\kern0pt}verit{\isacharcomma}{\kern0pt}\ ccfv{\isacharunderscore}{\kern0pt}threshold{\isacharparenright}{\kern0pt}\ exp{\isacharunderscore}{\kern0pt}of{\isacharunderscore}{\kern0pt}real{\isacharunderscore}{\kern0pt}cnj\ mult{\isachardot}{\kern0pt}commute\ mult{\isachardot}{\kern0pt}left{\isacharunderscore}{\kern0pt}commute\ \isanewline
\ \ \ \ \ \ \ \ \ \ \ \ \ \ \ \ mult{\isacharunderscore}{\kern0pt}{\isadigit{1}}s{\isacharunderscore}{\kern0pt}ring{\isacharunderscore}{\kern0pt}{\isadigit{1}}{\isacharparenleft}{\kern0pt}{\isadigit{1}}{\isacharparenright}{\kern0pt}\ of{\isacharunderscore}{\kern0pt}real{\isacharunderscore}{\kern0pt}divide\ of{\isacharunderscore}{\kern0pt}real{\isacharunderscore}{\kern0pt}minus\ of{\isacharunderscore}{\kern0pt}real{\isacharunderscore}{\kern0pt}numeral\ of{\isacharunderscore}{\kern0pt}real{\isacharunderscore}{\kern0pt}power\ \isanewline
\ \ \ \ \ \ \ \ \ \ \ \ \ \ \ \ times{\isacharunderscore}{\kern0pt}divide{\isacharunderscore}{\kern0pt}eq{\isacharunderscore}{\kern0pt}right{\isacharparenright}{\kern0pt}\isanewline
\ \ \ \ \ \ \ \ \ \ \isacommand{also}\isamarkupfalse%
\ \isacommand{have}\isamarkupfalse%
\ {\isachardoublequoteopen}{\isasymdots}\ {\isacharequal}{\kern0pt}\ m\ {\isachardollar}{\kern0pt}{\isachardollar}{\kern0pt}\ {\isacharparenleft}{\kern0pt}{\isadigit{1}}{\isacharcomma}{\kern0pt}{\isadigit{1}}{\isacharparenright}{\kern0pt}{\isachardoublequoteclose}\ \isacommand{using}\isamarkupfalse%
\ m{\isacharunderscore}{\kern0pt}def\ \isacommand{by}\isamarkupfalse%
\ simp\isanewline
\ \ \ \ \ \ \ \ \ \ \isacommand{finally}\isamarkupfalse%
\ \isacommand{have}\isamarkupfalse%
\ {\isachardoublequoteopen}R\ k\isactrlsup {\isasymdagger}\ {\isachardollar}{\kern0pt}{\isachardollar}{\kern0pt}\ {\isacharparenleft}{\kern0pt}i{\isacharcomma}{\kern0pt}\ j{\isacharparenright}{\kern0pt}\ {\isacharequal}{\kern0pt}\ m\ {\isachardollar}{\kern0pt}{\isachardollar}{\kern0pt}\ {\isacharparenleft}{\kern0pt}i{\isacharcomma}{\kern0pt}\ j{\isacharparenright}{\kern0pt}{\isachardoublequoteclose}\ \isacommand{using}\isamarkupfalse%
\ i{\isadigit{1}}\ j{\isadigit{1}}\ \isacommand{by}\isamarkupfalse%
\ simp\isanewline
\ \ \ \ \ \ \ \ \ \ \isacommand{thus}\isamarkupfalse%
\ {\isacharquery}{\kern0pt}thesis\ \isacommand{by}\isamarkupfalse%
\ this\isanewline
\ \ \ \ \ \ \ \ \isacommand{qed}\isamarkupfalse%
\isanewline
\ \ \ \ \ \ \isacommand{qed}\isamarkupfalse%
\isanewline
\ \ \ \ \isacommand{qed}\isamarkupfalse%
\isanewline
\ \ \isacommand{qed}\isamarkupfalse%
\isanewline
\isacommand{next}\isamarkupfalse%
\isanewline
\ \ \isacommand{define}\isamarkupfalse%
\ m\ \isakeyword{where}\ {\isachardoublequoteopen}m\ {\isacharequal}{\kern0pt}\ Matrix{\isachardot}{\kern0pt}mat\ {\isadigit{2}}\ {\isadigit{2}}\ \isanewline
\ \ {\isacharparenleft}{\kern0pt}{\isasymlambda}{\isacharparenleft}{\kern0pt}i{\isacharcomma}{\kern0pt}j{\isacharparenright}{\kern0pt}{\isachardot}{\kern0pt}\ if\ i{\isasymnoteq}j\ then\ {\isadigit{0}}\ else\ {\isacharparenleft}{\kern0pt}if\ i{\isacharequal}{\kern0pt}{\isadigit{0}}\ then\ {\isadigit{1}}\ else\ exp{\isacharparenleft}{\kern0pt}{\isacharminus}{\kern0pt}{\isadigit{2}}{\isacharasterisk}{\kern0pt}pi{\isacharasterisk}{\kern0pt}{\isasymi}{\isacharslash}{\kern0pt}{\isadigit{2}}{\isacharcircum}{\kern0pt}k{\isacharparenright}{\kern0pt}{\isacharparenright}{\kern0pt}{\isacharparenright}{\kern0pt}{\isachardoublequoteclose}\isanewline
\ \ \isacommand{thus}\isamarkupfalse%
\ {\isachardoublequoteopen}dim{\isacharunderscore}{\kern0pt}row\ R\ k\isactrlsup {\isasymdagger}\ {\isacharequal}{\kern0pt}\ dim{\isacharunderscore}{\kern0pt}row\ m{\isachardoublequoteclose}\ \isanewline
\ \ \ \ \isacommand{by}\isamarkupfalse%
\ {\isacharparenleft}{\kern0pt}metis\ {\isacharparenleft}{\kern0pt}no{\isacharunderscore}{\kern0pt}types{\isacharcomma}{\kern0pt}\ lifting{\isacharparenright}{\kern0pt}\ One{\isacharunderscore}{\kern0pt}nat{\isacharunderscore}{\kern0pt}def\ R{\isacharunderscore}{\kern0pt}def\ Suc{\isacharunderscore}{\kern0pt}{\isadigit{1}}\ Suc{\isacharunderscore}{\kern0pt}eq{\isacharunderscore}{\kern0pt}plus{\isadigit{1}}\ Tensor{\isachardot}{\kern0pt}mat{\isacharunderscore}{\kern0pt}of{\isacharunderscore}{\kern0pt}cols{\isacharunderscore}{\kern0pt}list{\isacharunderscore}{\kern0pt}def\isanewline
\ \ \ \ \ \ \ \ dim{\isacharunderscore}{\kern0pt}col{\isacharunderscore}{\kern0pt}mat{\isacharparenleft}{\kern0pt}{\isadigit{1}}{\isacharparenright}{\kern0pt}\ dim{\isacharunderscore}{\kern0pt}row{\isacharunderscore}{\kern0pt}mat{\isacharparenleft}{\kern0pt}{\isadigit{1}}{\isacharparenright}{\kern0pt}\ dim{\isacharunderscore}{\kern0pt}row{\isacharunderscore}{\kern0pt}of{\isacharunderscore}{\kern0pt}dagger\ list{\isachardot}{\kern0pt}size{\isacharparenleft}{\kern0pt}{\isadigit{3}}{\isacharparenright}{\kern0pt}\ list{\isachardot}{\kern0pt}size{\isacharparenleft}{\kern0pt}{\isadigit{4}}{\isacharparenright}{\kern0pt}{\isacharparenright}{\kern0pt}\isanewline
\isacommand{next}\isamarkupfalse%
\isanewline
\ \ \isacommand{define}\isamarkupfalse%
\ m\ \isakeyword{where}\ {\isachardoublequoteopen}m\ {\isacharequal}{\kern0pt}\ Matrix{\isachardot}{\kern0pt}mat\ {\isadigit{2}}\ {\isadigit{2}}\ \isanewline
\ \ {\isacharparenleft}{\kern0pt}{\isasymlambda}{\isacharparenleft}{\kern0pt}i{\isacharcomma}{\kern0pt}j{\isacharparenright}{\kern0pt}{\isachardot}{\kern0pt}\ if\ i{\isasymnoteq}j\ then\ {\isadigit{0}}\ else\ {\isacharparenleft}{\kern0pt}if\ i{\isacharequal}{\kern0pt}{\isadigit{0}}\ then\ {\isadigit{1}}\ else\ exp{\isacharparenleft}{\kern0pt}{\isacharminus}{\kern0pt}{\isadigit{2}}{\isacharasterisk}{\kern0pt}pi{\isacharasterisk}{\kern0pt}{\isasymi}{\isacharslash}{\kern0pt}{\isadigit{2}}{\isacharcircum}{\kern0pt}k{\isacharparenright}{\kern0pt}{\isacharparenright}{\kern0pt}{\isacharparenright}{\kern0pt}{\isachardoublequoteclose}\isanewline
\ \ \isacommand{thus}\isamarkupfalse%
\ {\isachardoublequoteopen}dim{\isacharunderscore}{\kern0pt}col\ R\ k\isactrlsup {\isasymdagger}\ {\isacharequal}{\kern0pt}\ dim{\isacharunderscore}{\kern0pt}col\ m{\isachardoublequoteclose}\ \isanewline
\ \ \ \ \isacommand{by}\isamarkupfalse%
\ {\isacharparenleft}{\kern0pt}simp\ add{\isacharcolon}{\kern0pt}\ R{\isacharunderscore}{\kern0pt}def\ Tensor{\isachardot}{\kern0pt}mat{\isacharunderscore}{\kern0pt}of{\isacharunderscore}{\kern0pt}cols{\isacharunderscore}{\kern0pt}list{\isacharunderscore}{\kern0pt}def{\isacharparenright}{\kern0pt}\isanewline
\isacommand{qed}\isamarkupfalse%
%
\endisatagproof
{\isafoldproof}%
%
\isadelimproof
\isanewline
%
\endisadelimproof
\isanewline
\isacommand{lemma}\isamarkupfalse%
\ R{\isacharunderscore}{\kern0pt}is{\isacharunderscore}{\kern0pt}gate{\isacharcolon}{\kern0pt}\isanewline
\ \ \isakeyword{shows}\ {\isachardoublequoteopen}gate\ {\isadigit{1}}\ {\isacharparenleft}{\kern0pt}R\ n{\isacharparenright}{\kern0pt}{\isachardoublequoteclose}\isanewline
%
\isadelimproof
%
\endisadelimproof
%
\isatagproof
\isacommand{proof}\isamarkupfalse%
\isanewline
\ \ \isacommand{show}\isamarkupfalse%
\ {\isachardoublequoteopen}dim{\isacharunderscore}{\kern0pt}row\ {\isacharparenleft}{\kern0pt}R\ n{\isacharparenright}{\kern0pt}\ {\isacharequal}{\kern0pt}\ {\isadigit{2}}{\isacharcircum}{\kern0pt}{\isadigit{1}}{\isachardoublequoteclose}\ \isacommand{using}\isamarkupfalse%
\ R{\isacharunderscore}{\kern0pt}def\ \isacommand{by}\isamarkupfalse%
\ {\isacharparenleft}{\kern0pt}simp\ add{\isacharcolon}{\kern0pt}\ Tensor{\isachardot}{\kern0pt}mat{\isacharunderscore}{\kern0pt}of{\isacharunderscore}{\kern0pt}cols{\isacharunderscore}{\kern0pt}list{\isacharunderscore}{\kern0pt}def{\isacharparenright}{\kern0pt}\isanewline
\ \ \isacommand{show}\isamarkupfalse%
\ {\isachardoublequoteopen}square{\isacharunderscore}{\kern0pt}mat\ {\isacharparenleft}{\kern0pt}R\ n{\isacharparenright}{\kern0pt}{\isachardoublequoteclose}\ \isacommand{using}\isamarkupfalse%
\ R{\isacharunderscore}{\kern0pt}def\ \isacommand{by}\isamarkupfalse%
\ {\isacharparenleft}{\kern0pt}simp\ add{\isacharcolon}{\kern0pt}\ Tensor{\isachardot}{\kern0pt}mat{\isacharunderscore}{\kern0pt}of{\isacharunderscore}{\kern0pt}cols{\isacharunderscore}{\kern0pt}list{\isacharunderscore}{\kern0pt}def{\isacharparenright}{\kern0pt}\isanewline
\ \ \isacommand{show}\isamarkupfalse%
\ {\isachardoublequoteopen}unitary\ {\isacharparenleft}{\kern0pt}R\ n{\isacharparenright}{\kern0pt}{\isachardoublequoteclose}\isanewline
\ \ \isacommand{proof}\isamarkupfalse%
\ {\isacharminus}{\kern0pt}\isanewline
\ \ \ \ \isacommand{have}\isamarkupfalse%
\ {\isachardoublequoteopen}{\isacharparenleft}{\kern0pt}{\isacharparenleft}{\kern0pt}R\ n{\isacharparenright}{\kern0pt}\isactrlsup {\isasymdagger}{\isacharparenright}{\kern0pt}\ {\isacharasterisk}{\kern0pt}\ {\isacharparenleft}{\kern0pt}R\ n{\isacharparenright}{\kern0pt}\ {\isacharequal}{\kern0pt}\ {\isadigit{1}}\isactrlsub m\ {\isadigit{2}}\ {\isasymand}\ {\isacharparenleft}{\kern0pt}R\ n{\isacharparenright}{\kern0pt}\ {\isacharasterisk}{\kern0pt}\ {\isacharparenleft}{\kern0pt}{\isacharparenleft}{\kern0pt}R\ n{\isacharparenright}{\kern0pt}\isactrlsup {\isasymdagger}{\isacharparenright}{\kern0pt}\ {\isacharequal}{\kern0pt}\ {\isadigit{1}}\isactrlsub m\ {\isadigit{2}}{\isachardoublequoteclose}\isanewline
\ \ \ \ \isacommand{proof}\isamarkupfalse%
\isanewline
\ \ \ \ \ \ \isacommand{show}\isamarkupfalse%
\ {\isachardoublequoteopen}R\ n\isactrlsup {\isasymdagger}\ {\isacharasterisk}{\kern0pt}\ R\ n\ {\isacharequal}{\kern0pt}\ {\isadigit{1}}\isactrlsub m\ {\isadigit{2}}{\isachardoublequoteclose}\isanewline
\ \ \ \ \ \ \isacommand{proof}\isamarkupfalse%
\isanewline
\ \ \ \ \ \ \ \ \isacommand{show}\isamarkupfalse%
\ {\isachardoublequoteopen}{\isasymAnd}i\ j{\isachardot}{\kern0pt}\ i\ {\isacharless}{\kern0pt}\ dim{\isacharunderscore}{\kern0pt}row\ {\isacharparenleft}{\kern0pt}{\isadigit{1}}\isactrlsub m\ {\isadigit{2}}{\isacharparenright}{\kern0pt}\ {\isasymLongrightarrow}\ j\ {\isacharless}{\kern0pt}\ dim{\isacharunderscore}{\kern0pt}col\ {\isacharparenleft}{\kern0pt}{\isadigit{1}}\isactrlsub m\ {\isadigit{2}}{\isacharparenright}{\kern0pt}\ {\isasymLongrightarrow}\ \isanewline
\ \ \ \ \ \ \ \ \ \ \ \ \ \ {\isacharparenleft}{\kern0pt}R\ n\isactrlsup {\isasymdagger}\ {\isacharasterisk}{\kern0pt}\ R\ n{\isacharparenright}{\kern0pt}\ {\isachardollar}{\kern0pt}{\isachardollar}{\kern0pt}\ {\isacharparenleft}{\kern0pt}i{\isacharcomma}{\kern0pt}\ j{\isacharparenright}{\kern0pt}\ {\isacharequal}{\kern0pt}\ {\isadigit{1}}\isactrlsub m\ {\isadigit{2}}\ {\isachardollar}{\kern0pt}{\isachardollar}{\kern0pt}\ {\isacharparenleft}{\kern0pt}i{\isacharcomma}{\kern0pt}\ j{\isacharparenright}{\kern0pt}{\isachardoublequoteclose}\isanewline
\ \ \ \ \ \ \ \ \isacommand{proof}\isamarkupfalse%
\ {\isacharminus}{\kern0pt}\isanewline
\ \ \ \ \ \ \ \ \ \ \isacommand{fix}\isamarkupfalse%
\ i\ j\isanewline
\ \ \ \ \ \ \ \ \ \ \isacommand{assume}\isamarkupfalse%
\ {\isachardoublequoteopen}i\ {\isacharless}{\kern0pt}\ dim{\isacharunderscore}{\kern0pt}row\ {\isacharparenleft}{\kern0pt}{\isadigit{1}}\isactrlsub m\ {\isadigit{2}}{\isacharparenright}{\kern0pt}{\isachardoublequoteclose}\isanewline
\ \ \ \ \ \ \ \ \ \ \isacommand{hence}\isamarkupfalse%
\ i{\isadigit{2}}{\isacharcolon}{\kern0pt}{\isachardoublequoteopen}i\ {\isacharless}{\kern0pt}\ {\isadigit{2}}{\isachardoublequoteclose}\ \isacommand{by}\isamarkupfalse%
\ auto\isanewline
\ \ \ \ \ \ \ \ \ \ \isacommand{assume}\isamarkupfalse%
\ {\isachardoublequoteopen}j\ {\isacharless}{\kern0pt}\ dim{\isacharunderscore}{\kern0pt}col\ {\isacharparenleft}{\kern0pt}{\isadigit{1}}\isactrlsub m\ {\isadigit{2}}{\isacharparenright}{\kern0pt}{\isachardoublequoteclose}\isanewline
\ \ \ \ \ \ \ \ \ \ \isacommand{hence}\isamarkupfalse%
\ j{\isadigit{2}}{\isacharcolon}{\kern0pt}{\isachardoublequoteopen}j\ {\isacharless}{\kern0pt}\ {\isadigit{2}}{\isachardoublequoteclose}\ \isacommand{by}\isamarkupfalse%
\ auto\isanewline
\ \ \ \ \ \ \ \ \ \ \isacommand{show}\isamarkupfalse%
\ {\isachardoublequoteopen}{\isacharparenleft}{\kern0pt}R\ n\isactrlsup {\isasymdagger}\ {\isacharasterisk}{\kern0pt}\ R\ n{\isacharparenright}{\kern0pt}\ {\isachardollar}{\kern0pt}{\isachardollar}{\kern0pt}\ {\isacharparenleft}{\kern0pt}i{\isacharcomma}{\kern0pt}\ j{\isacharparenright}{\kern0pt}\ {\isacharequal}{\kern0pt}\ {\isadigit{1}}\isactrlsub m\ {\isadigit{2}}\ {\isachardollar}{\kern0pt}{\isachardollar}{\kern0pt}\ {\isacharparenleft}{\kern0pt}i{\isacharcomma}{\kern0pt}\ j{\isacharparenright}{\kern0pt}{\isachardoublequoteclose}\isanewline
\ \ \ \ \ \ \ \ \ \ \isacommand{proof}\isamarkupfalse%
\ {\isacharparenleft}{\kern0pt}rule\ disjE{\isacharparenright}{\kern0pt}\isanewline
\ \ \ \ \ \ \ \ \ \ \ \ \isacommand{show}\isamarkupfalse%
\ {\isachardoublequoteopen}i\ {\isacharequal}{\kern0pt}\ {\isadigit{0}}\ {\isasymor}\ i\ {\isacharequal}{\kern0pt}\ {\isadigit{1}}{\isachardoublequoteclose}\ \isacommand{using}\isamarkupfalse%
\ i{\isadigit{2}}\ \isacommand{by}\isamarkupfalse%
\ auto\isanewline
\ \ \ \ \ \ \ \ \ \ \isacommand{next}\isamarkupfalse%
\isanewline
\ \ \ \ \ \ \ \ \ \ \ \ \isacommand{assume}\isamarkupfalse%
\ i{\isadigit{0}}{\isacharcolon}{\kern0pt}{\isachardoublequoteopen}i\ {\isacharequal}{\kern0pt}\ {\isadigit{0}}{\isachardoublequoteclose}\isanewline
\ \ \ \ \ \ \ \ \ \ \ \ \isacommand{show}\isamarkupfalse%
\ {\isachardoublequoteopen}{\isacharparenleft}{\kern0pt}R\ n\isactrlsup {\isasymdagger}\ {\isacharasterisk}{\kern0pt}\ R\ n{\isacharparenright}{\kern0pt}\ {\isachardollar}{\kern0pt}{\isachardollar}{\kern0pt}\ {\isacharparenleft}{\kern0pt}i{\isacharcomma}{\kern0pt}\ j{\isacharparenright}{\kern0pt}\ {\isacharequal}{\kern0pt}\ {\isadigit{1}}\isactrlsub m\ {\isadigit{2}}\ {\isachardollar}{\kern0pt}{\isachardollar}{\kern0pt}\ {\isacharparenleft}{\kern0pt}i{\isacharcomma}{\kern0pt}\ j{\isacharparenright}{\kern0pt}{\isachardoublequoteclose}\isanewline
\ \ \ \ \ \ \ \ \ \ \ \ \isacommand{proof}\isamarkupfalse%
\ {\isacharparenleft}{\kern0pt}rule\ disjE{\isacharparenright}{\kern0pt}\isanewline
\ \ \ \ \ \ \ \ \ \ \ \ \ \ \isacommand{show}\isamarkupfalse%
\ {\isachardoublequoteopen}j\ {\isacharequal}{\kern0pt}\ {\isadigit{0}}\ {\isasymor}\ j\ {\isacharequal}{\kern0pt}\ {\isadigit{1}}{\isachardoublequoteclose}\ \isacommand{using}\isamarkupfalse%
\ j{\isadigit{2}}\ \isacommand{by}\isamarkupfalse%
\ auto\isanewline
\ \ \ \ \ \ \ \ \ \ \ \ \isacommand{next}\isamarkupfalse%
\isanewline
\ \ \ \ \ \ \ \ \ \ \ \ \ \ \isacommand{assume}\isamarkupfalse%
\ j{\isadigit{0}}{\isacharcolon}{\kern0pt}{\isachardoublequoteopen}j\ {\isacharequal}{\kern0pt}\ {\isadigit{0}}{\isachardoublequoteclose}\isanewline
\ \ \ \ \ \ \ \ \ \ \ \ \ \ \isacommand{show}\isamarkupfalse%
\ {\isachardoublequoteopen}{\isacharparenleft}{\kern0pt}R\ n\isactrlsup {\isasymdagger}\ {\isacharasterisk}{\kern0pt}\ R\ n{\isacharparenright}{\kern0pt}\ {\isachardollar}{\kern0pt}{\isachardollar}{\kern0pt}\ {\isacharparenleft}{\kern0pt}i{\isacharcomma}{\kern0pt}\ j{\isacharparenright}{\kern0pt}\ {\isacharequal}{\kern0pt}\ {\isadigit{1}}\isactrlsub m\ {\isadigit{2}}\ {\isachardollar}{\kern0pt}{\isachardollar}{\kern0pt}\ {\isacharparenleft}{\kern0pt}i{\isacharcomma}{\kern0pt}\ j{\isacharparenright}{\kern0pt}{\isachardoublequoteclose}\isanewline
\ \ \ \ \ \ \ \ \ \ \ \ \ \ \isacommand{proof}\isamarkupfalse%
\ {\isacharminus}{\kern0pt}\isanewline
\ \ \ \ \ \ \ \ \ \ \ \ \ \ \ \ \isacommand{have}\isamarkupfalse%
\ {\isachardoublequoteopen}{\isacharparenleft}{\kern0pt}R\ n\isactrlsup {\isasymdagger}\ {\isacharasterisk}{\kern0pt}\ R\ n{\isacharparenright}{\kern0pt}\ {\isachardollar}{\kern0pt}{\isachardollar}{\kern0pt}\ {\isacharparenleft}{\kern0pt}{\isadigit{0}}{\isacharcomma}{\kern0pt}{\isadigit{0}}{\isacharparenright}{\kern0pt}\ {\isacharequal}{\kern0pt}\ {\isacharparenleft}{\kern0pt}{\isacharparenleft}{\kern0pt}R\ n{\isacharparenright}{\kern0pt}\isactrlsup {\isasymdagger}\ {\isachardollar}{\kern0pt}{\isachardollar}{\kern0pt}\ {\isacharparenleft}{\kern0pt}{\isadigit{0}}{\isacharcomma}{\kern0pt}{\isadigit{0}}{\isacharparenright}{\kern0pt}{\isacharparenright}{\kern0pt}\ {\isacharasterisk}{\kern0pt}\ {\isacharparenleft}{\kern0pt}{\isacharparenleft}{\kern0pt}R\ n{\isacharparenright}{\kern0pt}\ {\isachardollar}{\kern0pt}{\isachardollar}{\kern0pt}\ {\isacharparenleft}{\kern0pt}{\isadigit{0}}{\isacharcomma}{\kern0pt}{\isadigit{0}}{\isacharparenright}{\kern0pt}{\isacharparenright}{\kern0pt}\ {\isacharplus}{\kern0pt}\isanewline
\ \ \ \ \ \ \ \ \ \ \ \ \ \ \ \ \ \ \ \ \ \ {\isacharparenleft}{\kern0pt}{\isacharparenleft}{\kern0pt}R\ n{\isacharparenright}{\kern0pt}\isactrlsup {\isasymdagger}\ {\isachardollar}{\kern0pt}{\isachardollar}{\kern0pt}\ {\isacharparenleft}{\kern0pt}{\isadigit{0}}{\isacharcomma}{\kern0pt}{\isadigit{1}}{\isacharparenright}{\kern0pt}{\isacharparenright}{\kern0pt}\ {\isacharasterisk}{\kern0pt}\ {\isacharparenleft}{\kern0pt}{\isacharparenleft}{\kern0pt}R\ n{\isacharparenright}{\kern0pt}\ {\isachardollar}{\kern0pt}{\isachardollar}{\kern0pt}\ {\isacharparenleft}{\kern0pt}{\isadigit{1}}{\isacharcomma}{\kern0pt}{\isadigit{0}}{\isacharparenright}{\kern0pt}{\isacharparenright}{\kern0pt}{\isachardoublequoteclose}\isanewline
\ \ \ \ \ \ \ \ \ \ \ \ \ \ \ \ \ \ \isacommand{using}\isamarkupfalse%
\ {\isacartoucheopen}dim{\isacharunderscore}{\kern0pt}row\ {\isacharparenleft}{\kern0pt}R\ n{\isacharparenright}{\kern0pt}\ {\isacharequal}{\kern0pt}\ {\isadigit{2}}\ {\isacharcircum}{\kern0pt}\ {\isadigit{1}}{\isacartoucheclose}\ {\isacartoucheopen}square{\isacharunderscore}{\kern0pt}mat\ {\isacharparenleft}{\kern0pt}R\ n{\isacharparenright}{\kern0pt}{\isacartoucheclose}\ sumof{\isadigit{2}}\ \isacommand{by}\isamarkupfalse%
\ fastforce\isanewline
\ \ \ \ \ \ \ \ \ \ \ \ \ \ \ \ \isacommand{also}\isamarkupfalse%
\ \isacommand{have}\isamarkupfalse%
\ {\isachardoublequoteopen}{\isasymdots}\ {\isacharequal}{\kern0pt}\ {\isadigit{1}}{\isachardoublequoteclose}\ \isacommand{using}\isamarkupfalse%
\ R{\isacharunderscore}{\kern0pt}dagger{\isacharunderscore}{\kern0pt}mat\ R{\isacharunderscore}{\kern0pt}def\ index{\isacharunderscore}{\kern0pt}mat{\isacharunderscore}{\kern0pt}of{\isacharunderscore}{\kern0pt}cols{\isacharunderscore}{\kern0pt}list\isanewline
\ \ \ \ \ \ \ \ \ \ \ \ \ \ \ \ \ \ \isacommand{by}\isamarkupfalse%
\ {\isacharparenleft}{\kern0pt}smt\ {\isacharparenleft}{\kern0pt}verit{\isacharcomma}{\kern0pt}\ del{\isacharunderscore}{\kern0pt}insts{\isacharparenright}{\kern0pt}\ Suc{\isacharunderscore}{\kern0pt}{\isadigit{1}}\ Suc{\isacharunderscore}{\kern0pt}eq{\isacharunderscore}{\kern0pt}plus{\isadigit{1}}\ add{\isachardot}{\kern0pt}commute\ add{\isacharunderscore}{\kern0pt}{\isadigit{0}}\ index{\isacharunderscore}{\kern0pt}mat{\isacharparenleft}{\kern0pt}{\isadigit{1}}{\isacharparenright}{\kern0pt}\ \isanewline
\ \ \ \ \ \ \ \ \ \ \ \ \ \ \ \ \ \ \ \ \ \ lessI\ list{\isachardot}{\kern0pt}size{\isacharparenleft}{\kern0pt}{\isadigit{3}}{\isacharparenright}{\kern0pt}\ list{\isachardot}{\kern0pt}size{\isacharparenleft}{\kern0pt}{\isadigit{4}}{\isacharparenright}{\kern0pt}\ mult{\isacharunderscore}{\kern0pt}{\isadigit{1}}\ mult{\isacharunderscore}{\kern0pt}zero{\isacharunderscore}{\kern0pt}left\ nth{\isacharunderscore}{\kern0pt}Cons{\isacharunderscore}{\kern0pt}{\isadigit{0}}\ \isanewline
\ \ \ \ \ \ \ \ \ \ \ \ \ \ \ \ \ \ \ \ \ \ nth{\isacharunderscore}{\kern0pt}Cons{\isacharunderscore}{\kern0pt}Suc\ old{\isachardot}{\kern0pt}prod{\isachardot}{\kern0pt}case\ pos{\isadigit{2}}{\isacharparenright}{\kern0pt}\isanewline
\ \ \ \ \ \ \ \ \ \ \ \ \ \ \ \ \isacommand{also}\isamarkupfalse%
\ \isacommand{have}\isamarkupfalse%
\ {\isachardoublequoteopen}{\isasymdots}\ {\isacharequal}{\kern0pt}\ {\isadigit{1}}\isactrlsub m\ {\isadigit{2}}\ {\isachardollar}{\kern0pt}{\isachardollar}{\kern0pt}\ {\isacharparenleft}{\kern0pt}{\isadigit{0}}{\isacharcomma}{\kern0pt}{\isadigit{0}}{\isacharparenright}{\kern0pt}{\isachardoublequoteclose}\ \isacommand{by}\isamarkupfalse%
\ simp\isanewline
\ \ \ \ \ \ \ \ \ \ \ \ \ \ \ \ \isacommand{finally}\isamarkupfalse%
\ \isacommand{show}\isamarkupfalse%
\ {\isacharquery}{\kern0pt}thesis\ \isacommand{using}\isamarkupfalse%
\ i{\isadigit{0}}\ j{\isadigit{0}}\ \isacommand{by}\isamarkupfalse%
\ simp\isanewline
\ \ \ \ \ \ \ \ \ \ \ \ \ \ \isacommand{qed}\isamarkupfalse%
\ \isanewline
\ \ \ \ \ \ \ \ \ \ \ \ \isacommand{next}\isamarkupfalse%
\isanewline
\ \ \ \ \ \ \ \ \ \ \ \ \ \ \isacommand{assume}\isamarkupfalse%
\ j{\isadigit{1}}{\isacharcolon}{\kern0pt}{\isachardoublequoteopen}j\ {\isacharequal}{\kern0pt}\ {\isadigit{1}}{\isachardoublequoteclose}\isanewline
\ \ \ \ \ \ \ \ \ \ \ \ \ \ \isacommand{show}\isamarkupfalse%
\ {\isachardoublequoteopen}{\isacharparenleft}{\kern0pt}R\ n\isactrlsup {\isasymdagger}\ {\isacharasterisk}{\kern0pt}\ R\ n{\isacharparenright}{\kern0pt}\ {\isachardollar}{\kern0pt}{\isachardollar}{\kern0pt}\ {\isacharparenleft}{\kern0pt}i{\isacharcomma}{\kern0pt}\ j{\isacharparenright}{\kern0pt}\ {\isacharequal}{\kern0pt}\ {\isadigit{1}}\isactrlsub m\ {\isadigit{2}}\ {\isachardollar}{\kern0pt}{\isachardollar}{\kern0pt}\ {\isacharparenleft}{\kern0pt}i{\isacharcomma}{\kern0pt}\ j{\isacharparenright}{\kern0pt}{\isachardoublequoteclose}\ \isanewline
\ \ \ \ \ \ \ \ \ \ \ \ \ \ \isacommand{proof}\isamarkupfalse%
\ {\isacharminus}{\kern0pt}\isanewline
\ \ \ \ \ \ \ \ \ \ \ \ \ \ \ \ \isacommand{have}\isamarkupfalse%
\ {\isachardoublequoteopen}{\isacharparenleft}{\kern0pt}R\ n\isactrlsup {\isasymdagger}\ {\isacharasterisk}{\kern0pt}\ R\ n{\isacharparenright}{\kern0pt}\ {\isachardollar}{\kern0pt}{\isachardollar}{\kern0pt}\ {\isacharparenleft}{\kern0pt}{\isadigit{0}}{\isacharcomma}{\kern0pt}{\isadigit{1}}{\isacharparenright}{\kern0pt}\ {\isacharequal}{\kern0pt}\ {\isacharparenleft}{\kern0pt}{\isacharparenleft}{\kern0pt}R\ n{\isacharparenright}{\kern0pt}\isactrlsup {\isasymdagger}\ {\isachardollar}{\kern0pt}{\isachardollar}{\kern0pt}\ {\isacharparenleft}{\kern0pt}{\isadigit{0}}{\isacharcomma}{\kern0pt}{\isadigit{0}}{\isacharparenright}{\kern0pt}{\isacharparenright}{\kern0pt}\ {\isacharasterisk}{\kern0pt}\ {\isacharparenleft}{\kern0pt}{\isacharparenleft}{\kern0pt}R\ n{\isacharparenright}{\kern0pt}\ {\isachardollar}{\kern0pt}{\isachardollar}{\kern0pt}\ {\isacharparenleft}{\kern0pt}{\isadigit{0}}{\isacharcomma}{\kern0pt}{\isadigit{1}}{\isacharparenright}{\kern0pt}{\isacharparenright}{\kern0pt}\ {\isacharplus}{\kern0pt}\isanewline
\ \ \ \ \ \ \ \ \ \ \ \ \ \ \ \ \ \ \ \ \ \ {\isacharparenleft}{\kern0pt}{\isacharparenleft}{\kern0pt}R\ n{\isacharparenright}{\kern0pt}\isactrlsup {\isasymdagger}\ {\isachardollar}{\kern0pt}{\isachardollar}{\kern0pt}\ {\isacharparenleft}{\kern0pt}{\isadigit{0}}{\isacharcomma}{\kern0pt}{\isadigit{1}}{\isacharparenright}{\kern0pt}{\isacharparenright}{\kern0pt}\ {\isacharasterisk}{\kern0pt}\ {\isacharparenleft}{\kern0pt}{\isacharparenleft}{\kern0pt}R\ n{\isacharparenright}{\kern0pt}\ {\isachardollar}{\kern0pt}{\isachardollar}{\kern0pt}\ {\isacharparenleft}{\kern0pt}{\isadigit{1}}{\isacharcomma}{\kern0pt}{\isadigit{1}}{\isacharparenright}{\kern0pt}{\isacharparenright}{\kern0pt}{\isachardoublequoteclose}\isanewline
\ \ \ \ \ \ \ \ \ \ \ \ \ \ \ \ \ \ \isacommand{using}\isamarkupfalse%
\ {\isacartoucheopen}dim{\isacharunderscore}{\kern0pt}row\ {\isacharparenleft}{\kern0pt}R\ n{\isacharparenright}{\kern0pt}\ {\isacharequal}{\kern0pt}\ {\isadigit{2}}\ {\isacharcircum}{\kern0pt}\ {\isadigit{1}}{\isacartoucheclose}\ {\isacartoucheopen}square{\isacharunderscore}{\kern0pt}mat\ {\isacharparenleft}{\kern0pt}R\ n{\isacharparenright}{\kern0pt}{\isacartoucheclose}\ sumof{\isadigit{2}}\ \isacommand{by}\isamarkupfalse%
\ fastforce\isanewline
\ \ \ \ \ \ \ \ \ \ \ \ \ \ \ \ \isacommand{also}\isamarkupfalse%
\ \isacommand{have}\isamarkupfalse%
\ {\isachardoublequoteopen}{\isasymdots}\ {\isacharequal}{\kern0pt}\ {\isadigit{0}}{\isachardoublequoteclose}\ \isacommand{using}\isamarkupfalse%
\ R{\isacharunderscore}{\kern0pt}dagger{\isacharunderscore}{\kern0pt}mat\ R{\isacharunderscore}{\kern0pt}def\ index{\isacharunderscore}{\kern0pt}mat{\isacharunderscore}{\kern0pt}of{\isacharunderscore}{\kern0pt}cols{\isacharunderscore}{\kern0pt}list\isanewline
\ \ \ \ \ \ \ \ \ \ \ \ \ \ \ \ \ \ \isacommand{by}\isamarkupfalse%
\ {\isacharparenleft}{\kern0pt}smt\ {\isacharparenleft}{\kern0pt}verit{\isacharparenright}{\kern0pt}\ Suc{\isacharunderscore}{\kern0pt}{\isadigit{1}}\ Suc{\isacharunderscore}{\kern0pt}eq{\isacharunderscore}{\kern0pt}plus{\isadigit{1}}\ add{\isacharunderscore}{\kern0pt}cancel{\isacharunderscore}{\kern0pt}left{\isacharunderscore}{\kern0pt}left\ index{\isacharunderscore}{\kern0pt}mat{\isacharparenleft}{\kern0pt}{\isadigit{1}}{\isacharparenright}{\kern0pt}\ lessI\isanewline
\ \ \ \ \ \ \ \ \ \ \ \ \ \ \ \ \ \ \ \ \ \ list{\isachardot}{\kern0pt}size{\isacharparenleft}{\kern0pt}{\isadigit{3}}{\isacharparenright}{\kern0pt}\ list{\isachardot}{\kern0pt}size{\isacharparenleft}{\kern0pt}{\isadigit{4}}{\isacharparenright}{\kern0pt}\ mult{\isacharunderscore}{\kern0pt}eq{\isacharunderscore}{\kern0pt}{\isadigit{0}}{\isacharunderscore}{\kern0pt}iff\ nth{\isacharunderscore}{\kern0pt}Cons{\isacharunderscore}{\kern0pt}{\isadigit{0}}\ nth{\isacharunderscore}{\kern0pt}Cons{\isacharunderscore}{\kern0pt}Suc\ old{\isachardot}{\kern0pt}prod{\isachardot}{\kern0pt}case\ \isanewline
\ \ \ \ \ \ \ \ \ \ \ \ \ \ \ \ \ \ \ \ \ \ pos{\isadigit{2}}{\isacharparenright}{\kern0pt}\isanewline
\ \ \ \ \ \ \ \ \ \ \ \ \ \ \ \ \isacommand{also}\isamarkupfalse%
\ \isacommand{have}\isamarkupfalse%
\ {\isachardoublequoteopen}{\isasymdots}\ {\isacharequal}{\kern0pt}\ {\isadigit{1}}\isactrlsub m\ {\isadigit{2}}\ {\isachardollar}{\kern0pt}{\isachardollar}{\kern0pt}\ {\isacharparenleft}{\kern0pt}{\isadigit{0}}{\isacharcomma}{\kern0pt}{\isadigit{1}}{\isacharparenright}{\kern0pt}{\isachardoublequoteclose}\ \isacommand{by}\isamarkupfalse%
\ simp\isanewline
\ \ \ \ \ \ \ \ \ \ \ \ \ \ \ \ \isacommand{finally}\isamarkupfalse%
\ \isacommand{show}\isamarkupfalse%
\ {\isacharquery}{\kern0pt}thesis\ \isacommand{using}\isamarkupfalse%
\ i{\isadigit{0}}\ j{\isadigit{1}}\ \isacommand{by}\isamarkupfalse%
\ simp\isanewline
\ \ \ \ \ \ \ \ \ \ \ \ \ \ \isacommand{qed}\isamarkupfalse%
\isanewline
\ \ \ \ \ \ \ \ \ \ \ \ \isacommand{qed}\isamarkupfalse%
\isanewline
\ \ \ \ \ \ \ \ \ \ \isacommand{next}\isamarkupfalse%
\isanewline
\ \ \ \ \ \ \ \ \ \ \ \ \isacommand{assume}\isamarkupfalse%
\ i{\isadigit{1}}{\isacharcolon}{\kern0pt}{\isachardoublequoteopen}i\ {\isacharequal}{\kern0pt}\ {\isadigit{1}}{\isachardoublequoteclose}\isanewline
\ \ \ \ \ \ \ \ \ \ \ \ \isacommand{show}\isamarkupfalse%
\ {\isachardoublequoteopen}{\isacharparenleft}{\kern0pt}{\isacharparenleft}{\kern0pt}R\ n\isactrlsup {\isasymdagger}{\isacharparenright}{\kern0pt}\ {\isacharasterisk}{\kern0pt}\ R\ n{\isacharparenright}{\kern0pt}\ {\isachardollar}{\kern0pt}{\isachardollar}{\kern0pt}\ {\isacharparenleft}{\kern0pt}i{\isacharcomma}{\kern0pt}\ j{\isacharparenright}{\kern0pt}\ {\isacharequal}{\kern0pt}\ {\isadigit{1}}\isactrlsub m\ {\isadigit{2}}\ {\isachardollar}{\kern0pt}{\isachardollar}{\kern0pt}\ {\isacharparenleft}{\kern0pt}i{\isacharcomma}{\kern0pt}\ j{\isacharparenright}{\kern0pt}{\isachardoublequoteclose}\isanewline
\ \ \ \ \ \ \ \ \ \ \ \ \isacommand{proof}\isamarkupfalse%
\ {\isacharparenleft}{\kern0pt}rule\ disjE{\isacharparenright}{\kern0pt}\isanewline
\ \ \ \ \ \ \ \ \ \ \ \ \ \ \isacommand{show}\isamarkupfalse%
\ {\isachardoublequoteopen}j\ {\isacharequal}{\kern0pt}\ {\isadigit{0}}\ {\isasymor}\ j\ {\isacharequal}{\kern0pt}\ {\isadigit{1}}{\isachardoublequoteclose}\ \isacommand{using}\isamarkupfalse%
\ j{\isadigit{2}}\ \isacommand{by}\isamarkupfalse%
\ auto\isanewline
\ \ \ \ \ \ \ \ \ \ \ \ \isacommand{next}\isamarkupfalse%
\isanewline
\ \ \ \ \ \ \ \ \ \ \ \ \ \ \isacommand{assume}\isamarkupfalse%
\ j{\isadigit{0}}{\isacharcolon}{\kern0pt}{\isachardoublequoteopen}j\ {\isacharequal}{\kern0pt}\ {\isadigit{0}}{\isachardoublequoteclose}\isanewline
\ \ \ \ \ \ \ \ \ \ \ \ \ \ \isacommand{show}\isamarkupfalse%
\ {\isachardoublequoteopen}{\isacharparenleft}{\kern0pt}R\ n\isactrlsup {\isasymdagger}\ {\isacharasterisk}{\kern0pt}\ R\ n{\isacharparenright}{\kern0pt}\ {\isachardollar}{\kern0pt}{\isachardollar}{\kern0pt}\ {\isacharparenleft}{\kern0pt}i{\isacharcomma}{\kern0pt}\ j{\isacharparenright}{\kern0pt}\ {\isacharequal}{\kern0pt}\ {\isadigit{1}}\isactrlsub m\ {\isadigit{2}}\ {\isachardollar}{\kern0pt}{\isachardollar}{\kern0pt}\ {\isacharparenleft}{\kern0pt}i{\isacharcomma}{\kern0pt}\ j{\isacharparenright}{\kern0pt}{\isachardoublequoteclose}\isanewline
\ \ \ \ \ \ \ \ \ \ \ \ \ \ \isacommand{proof}\isamarkupfalse%
\ {\isacharminus}{\kern0pt}\isanewline
\ \ \ \ \ \ \ \ \ \ \ \ \ \ \ \ \isacommand{have}\isamarkupfalse%
\ {\isachardoublequoteopen}{\isacharparenleft}{\kern0pt}R\ n\isactrlsup {\isasymdagger}\ {\isacharasterisk}{\kern0pt}\ R\ n{\isacharparenright}{\kern0pt}\ {\isachardollar}{\kern0pt}{\isachardollar}{\kern0pt}\ {\isacharparenleft}{\kern0pt}{\isadigit{1}}{\isacharcomma}{\kern0pt}{\isadigit{0}}{\isacharparenright}{\kern0pt}\ {\isacharequal}{\kern0pt}\ {\isacharparenleft}{\kern0pt}{\isacharparenleft}{\kern0pt}R\ n{\isacharparenright}{\kern0pt}\isactrlsup {\isasymdagger}\ {\isachardollar}{\kern0pt}{\isachardollar}{\kern0pt}\ {\isacharparenleft}{\kern0pt}{\isadigit{1}}{\isacharcomma}{\kern0pt}{\isadigit{0}}{\isacharparenright}{\kern0pt}{\isacharparenright}{\kern0pt}\ {\isacharasterisk}{\kern0pt}\ {\isacharparenleft}{\kern0pt}{\isacharparenleft}{\kern0pt}R\ n{\isacharparenright}{\kern0pt}\ {\isachardollar}{\kern0pt}{\isachardollar}{\kern0pt}\ {\isacharparenleft}{\kern0pt}{\isadigit{0}}{\isacharcomma}{\kern0pt}{\isadigit{0}}{\isacharparenright}{\kern0pt}{\isacharparenright}{\kern0pt}\ {\isacharplus}{\kern0pt}\isanewline
\ \ \ \ \ \ \ \ \ \ \ \ \ \ \ \ \ \ \ \ \ \ {\isacharparenleft}{\kern0pt}{\isacharparenleft}{\kern0pt}R\ n{\isacharparenright}{\kern0pt}\isactrlsup {\isasymdagger}\ {\isachardollar}{\kern0pt}{\isachardollar}{\kern0pt}\ {\isacharparenleft}{\kern0pt}{\isadigit{1}}{\isacharcomma}{\kern0pt}{\isadigit{1}}{\isacharparenright}{\kern0pt}{\isacharparenright}{\kern0pt}\ {\isacharasterisk}{\kern0pt}\ {\isacharparenleft}{\kern0pt}{\isacharparenleft}{\kern0pt}R\ n{\isacharparenright}{\kern0pt}\ {\isachardollar}{\kern0pt}{\isachardollar}{\kern0pt}\ {\isacharparenleft}{\kern0pt}{\isadigit{1}}{\isacharcomma}{\kern0pt}{\isadigit{0}}{\isacharparenright}{\kern0pt}{\isacharparenright}{\kern0pt}{\isachardoublequoteclose}\isanewline
\ \ \ \ \ \ \ \ \ \ \ \ \ \ \ \ \ \ \isacommand{using}\isamarkupfalse%
\ {\isacartoucheopen}dim{\isacharunderscore}{\kern0pt}row\ {\isacharparenleft}{\kern0pt}R\ n{\isacharparenright}{\kern0pt}\ {\isacharequal}{\kern0pt}\ {\isadigit{2}}\ {\isacharcircum}{\kern0pt}\ {\isadigit{1}}{\isacartoucheclose}\ {\isacartoucheopen}square{\isacharunderscore}{\kern0pt}mat\ {\isacharparenleft}{\kern0pt}R\ n{\isacharparenright}{\kern0pt}{\isacartoucheclose}\ sumof{\isadigit{2}}\ \isacommand{by}\isamarkupfalse%
\ fastforce\isanewline
\ \ \ \ \ \ \ \ \ \ \ \ \ \ \ \ \isacommand{also}\isamarkupfalse%
\ \isacommand{have}\isamarkupfalse%
\ {\isachardoublequoteopen}{\isasymdots}\ {\isacharequal}{\kern0pt}\ {\isadigit{0}}{\isachardoublequoteclose}\ \isacommand{using}\isamarkupfalse%
\ R{\isacharunderscore}{\kern0pt}dagger{\isacharunderscore}{\kern0pt}mat\ R{\isacharunderscore}{\kern0pt}def\ index{\isacharunderscore}{\kern0pt}mat{\isacharunderscore}{\kern0pt}of{\isacharunderscore}{\kern0pt}cols{\isacharunderscore}{\kern0pt}list\isanewline
\ \ \ \ \ \ \ \ \ \ \ \ \ \ \ \ \ \ \isacommand{by}\isamarkupfalse%
\ {\isacharparenleft}{\kern0pt}smt\ {\isacharparenleft}{\kern0pt}verit{\isacharparenright}{\kern0pt}\ Suc{\isacharunderscore}{\kern0pt}{\isadigit{1}}\ Suc{\isacharunderscore}{\kern0pt}eq{\isacharunderscore}{\kern0pt}plus{\isadigit{1}}\ add{\isacharunderscore}{\kern0pt}cancel{\isacharunderscore}{\kern0pt}right{\isacharunderscore}{\kern0pt}right\ index{\isacharunderscore}{\kern0pt}mat{\isacharparenleft}{\kern0pt}{\isadigit{1}}{\isacharparenright}{\kern0pt}\ lessI\ \isanewline
\ \ \ \ \ \ \ \ \ \ \ \ \ \ \ \ \ \ \ \ \ \ list{\isachardot}{\kern0pt}size{\isacharparenleft}{\kern0pt}{\isadigit{3}}{\isacharparenright}{\kern0pt}\ list{\isachardot}{\kern0pt}size{\isacharparenleft}{\kern0pt}{\isadigit{4}}{\isacharparenright}{\kern0pt}\ mult{\isacharunderscore}{\kern0pt}eq{\isacharunderscore}{\kern0pt}{\isadigit{0}}{\isacharunderscore}{\kern0pt}iff\ nth{\isacharunderscore}{\kern0pt}Cons{\isacharunderscore}{\kern0pt}{\isadigit{0}}\ nth{\isacharunderscore}{\kern0pt}Cons{\isacharunderscore}{\kern0pt}Suc\ old{\isachardot}{\kern0pt}prod{\isachardot}{\kern0pt}case\isanewline
\ \ \ \ \ \ \ \ \ \ \ \ \ \ \ \ \ \ \ \ \ \ plus{\isacharunderscore}{\kern0pt}{\isadigit{1}}{\isacharunderscore}{\kern0pt}eq{\isacharunderscore}{\kern0pt}Suc\ pos{\isadigit{2}}{\isacharparenright}{\kern0pt}\isanewline
\ \ \ \ \ \ \ \ \ \ \ \ \ \ \ \ \isacommand{also}\isamarkupfalse%
\ \isacommand{have}\isamarkupfalse%
\ {\isachardoublequoteopen}{\isasymdots}\ {\isacharequal}{\kern0pt}\ {\isadigit{1}}\isactrlsub m\ {\isadigit{2}}\ {\isachardollar}{\kern0pt}{\isachardollar}{\kern0pt}\ {\isacharparenleft}{\kern0pt}{\isadigit{1}}{\isacharcomma}{\kern0pt}{\isadigit{0}}{\isacharparenright}{\kern0pt}{\isachardoublequoteclose}\ \isacommand{by}\isamarkupfalse%
\ simp\isanewline
\ \ \ \ \ \ \ \ \ \ \ \ \ \ \ \ \isacommand{finally}\isamarkupfalse%
\ \isacommand{show}\isamarkupfalse%
\ {\isacharquery}{\kern0pt}thesis\ \isacommand{using}\isamarkupfalse%
\ i{\isadigit{1}}\ j{\isadigit{0}}\ \isacommand{by}\isamarkupfalse%
\ simp\isanewline
\ \ \ \ \ \ \ \ \ \ \ \ \ \ \isacommand{qed}\isamarkupfalse%
\isanewline
\ \ \ \ \ \ \ \ \ \ \ \ \isacommand{next}\isamarkupfalse%
\isanewline
\ \ \ \ \ \ \ \ \ \ \ \ \ \ \isacommand{assume}\isamarkupfalse%
\ j{\isadigit{1}}{\isacharcolon}{\kern0pt}{\isachardoublequoteopen}j\ {\isacharequal}{\kern0pt}\ {\isadigit{1}}{\isachardoublequoteclose}\isanewline
\ \ \ \ \ \ \ \ \ \ \ \ \ \ \isacommand{show}\isamarkupfalse%
\ {\isachardoublequoteopen}{\isacharparenleft}{\kern0pt}R\ n\isactrlsup {\isasymdagger}\ {\isacharasterisk}{\kern0pt}\ R\ n{\isacharparenright}{\kern0pt}\ {\isachardollar}{\kern0pt}{\isachardollar}{\kern0pt}\ {\isacharparenleft}{\kern0pt}i{\isacharcomma}{\kern0pt}\ j{\isacharparenright}{\kern0pt}\ {\isacharequal}{\kern0pt}\ {\isadigit{1}}\isactrlsub m\ {\isadigit{2}}\ {\isachardollar}{\kern0pt}{\isachardollar}{\kern0pt}\ {\isacharparenleft}{\kern0pt}i{\isacharcomma}{\kern0pt}\ j{\isacharparenright}{\kern0pt}{\isachardoublequoteclose}\isanewline
\ \ \ \ \ \ \ \ \ \ \ \ \ \ \isacommand{proof}\isamarkupfalse%
\ {\isacharminus}{\kern0pt}\isanewline
\ \ \ \ \ \ \ \ \ \ \ \ \ \ \ \ \isacommand{have}\isamarkupfalse%
\ {\isachardoublequoteopen}{\isacharparenleft}{\kern0pt}R\ n\isactrlsup {\isasymdagger}\ {\isacharasterisk}{\kern0pt}\ R\ n{\isacharparenright}{\kern0pt}\ {\isachardollar}{\kern0pt}{\isachardollar}{\kern0pt}\ {\isacharparenleft}{\kern0pt}{\isadigit{1}}{\isacharcomma}{\kern0pt}{\isadigit{1}}{\isacharparenright}{\kern0pt}\ {\isacharequal}{\kern0pt}\ {\isacharparenleft}{\kern0pt}{\isacharparenleft}{\kern0pt}R\ n{\isacharparenright}{\kern0pt}\isactrlsup {\isasymdagger}\ {\isachardollar}{\kern0pt}{\isachardollar}{\kern0pt}\ {\isacharparenleft}{\kern0pt}{\isadigit{1}}{\isacharcomma}{\kern0pt}{\isadigit{0}}{\isacharparenright}{\kern0pt}{\isacharparenright}{\kern0pt}\ {\isacharasterisk}{\kern0pt}\ {\isacharparenleft}{\kern0pt}{\isacharparenleft}{\kern0pt}R\ n{\isacharparenright}{\kern0pt}\ {\isachardollar}{\kern0pt}{\isachardollar}{\kern0pt}\ {\isacharparenleft}{\kern0pt}{\isadigit{0}}{\isacharcomma}{\kern0pt}{\isadigit{1}}{\isacharparenright}{\kern0pt}{\isacharparenright}{\kern0pt}\ {\isacharplus}{\kern0pt}\isanewline
\ \ \ \ \ \ \ \ \ \ \ \ \ \ \ \ \ \ \ \ \ \ {\isacharparenleft}{\kern0pt}{\isacharparenleft}{\kern0pt}R\ n{\isacharparenright}{\kern0pt}\isactrlsup {\isasymdagger}\ {\isachardollar}{\kern0pt}{\isachardollar}{\kern0pt}\ {\isacharparenleft}{\kern0pt}{\isadigit{1}}{\isacharcomma}{\kern0pt}{\isadigit{1}}{\isacharparenright}{\kern0pt}{\isacharparenright}{\kern0pt}\ {\isacharasterisk}{\kern0pt}\ {\isacharparenleft}{\kern0pt}{\isacharparenleft}{\kern0pt}R\ n{\isacharparenright}{\kern0pt}\ {\isachardollar}{\kern0pt}{\isachardollar}{\kern0pt}\ {\isacharparenleft}{\kern0pt}{\isadigit{1}}{\isacharcomma}{\kern0pt}{\isadigit{1}}{\isacharparenright}{\kern0pt}{\isacharparenright}{\kern0pt}{\isachardoublequoteclose}\isanewline
\ \ \ \ \ \ \ \ \ \ \ \ \ \ \ \ \ \ \isacommand{using}\isamarkupfalse%
\ {\isacartoucheopen}dim{\isacharunderscore}{\kern0pt}row\ {\isacharparenleft}{\kern0pt}R\ n{\isacharparenright}{\kern0pt}\ {\isacharequal}{\kern0pt}\ {\isadigit{2}}\ {\isacharcircum}{\kern0pt}\ {\isadigit{1}}{\isacartoucheclose}\ {\isacartoucheopen}square{\isacharunderscore}{\kern0pt}mat\ {\isacharparenleft}{\kern0pt}R\ n{\isacharparenright}{\kern0pt}{\isacartoucheclose}\ sumof{\isadigit{2}}\ \isacommand{by}\isamarkupfalse%
\ fastforce\isanewline
\ \ \ \ \ \ \ \ \ \ \ \ \ \ \ \ \isacommand{also}\isamarkupfalse%
\ \isacommand{have}\isamarkupfalse%
\ {\isachardoublequoteopen}{\isasymdots}\ {\isacharequal}{\kern0pt}\ exp{\isacharparenleft}{\kern0pt}{\isacharminus}{\kern0pt}{\isadigit{2}}{\isacharasterisk}{\kern0pt}pi{\isacharasterisk}{\kern0pt}{\isasymi}{\isacharslash}{\kern0pt}{\isadigit{2}}{\isacharcircum}{\kern0pt}n{\isacharparenright}{\kern0pt}\ {\isacharasterisk}{\kern0pt}\ exp{\isacharparenleft}{\kern0pt}{\isadigit{2}}{\isacharasterisk}{\kern0pt}pi{\isacharasterisk}{\kern0pt}{\isasymi}{\isacharslash}{\kern0pt}{\isadigit{2}}{\isacharcircum}{\kern0pt}n{\isacharparenright}{\kern0pt}{\isachardoublequoteclose}\isanewline
\ \ \ \ \ \ \ \ \ \ \ \ \ \ \ \ \ \ \isacommand{using}\isamarkupfalse%
\ R{\isacharunderscore}{\kern0pt}dagger{\isacharunderscore}{\kern0pt}mat\ R{\isacharunderscore}{\kern0pt}def\ index{\isacharunderscore}{\kern0pt}mat{\isacharunderscore}{\kern0pt}of{\isacharunderscore}{\kern0pt}cols{\isacharunderscore}{\kern0pt}list\ \isacommand{by}\isamarkupfalse%
\ auto\isanewline
\ \ \ \ \ \ \ \ \ \ \ \ \ \ \ \ \isacommand{also}\isamarkupfalse%
\ \isacommand{have}\isamarkupfalse%
\ {\isachardoublequoteopen}{\isasymdots}\ {\isacharequal}{\kern0pt}\ {\isadigit{1}}{\isachardoublequoteclose}\isanewline
\ \ \ \ \ \ \ \ \ \ \ \ \ \ \ \ \ \ \isacommand{by}\isamarkupfalse%
\ {\isacharparenleft}{\kern0pt}metis\ {\isacharparenleft}{\kern0pt}no{\isacharunderscore}{\kern0pt}types{\isacharcomma}{\kern0pt}\ lifting{\isacharparenright}{\kern0pt}\ exp{\isacharunderscore}{\kern0pt}minus{\isacharunderscore}{\kern0pt}inverse\ minus{\isacharunderscore}{\kern0pt}divide{\isacharunderscore}{\kern0pt}divide\ \isanewline
\ \ \ \ \ \ \ \ \ \ \ \ \ \ \ \ \ \ \ \ \ \ minus{\isacharunderscore}{\kern0pt}divide{\isacharunderscore}{\kern0pt}right\ mult{\isacharunderscore}{\kern0pt}minus{\isacharunderscore}{\kern0pt}left\ of{\isacharunderscore}{\kern0pt}real{\isacharunderscore}{\kern0pt}minus{\isacharparenright}{\kern0pt}\isanewline
\ \ \ \ \ \ \ \ \ \ \ \ \ \ \ \ \isacommand{also}\isamarkupfalse%
\ \isacommand{have}\isamarkupfalse%
\ {\isachardoublequoteopen}{\isasymdots}\ {\isacharequal}{\kern0pt}\ {\isadigit{1}}\isactrlsub m\ {\isadigit{2}}\ {\isachardollar}{\kern0pt}{\isachardollar}{\kern0pt}\ {\isacharparenleft}{\kern0pt}{\isadigit{1}}{\isacharcomma}{\kern0pt}{\isadigit{1}}{\isacharparenright}{\kern0pt}{\isachardoublequoteclose}\ \isacommand{by}\isamarkupfalse%
\ simp\isanewline
\ \ \ \ \ \ \ \ \ \ \ \ \ \ \ \ \isacommand{finally}\isamarkupfalse%
\ \isacommand{show}\isamarkupfalse%
\ {\isacharquery}{\kern0pt}thesis\ \isacommand{using}\isamarkupfalse%
\ i{\isadigit{1}}\ j{\isadigit{1}}\ \isacommand{by}\isamarkupfalse%
\ simp\isanewline
\ \ \ \ \ \ \ \ \ \ \ \ \ \ \isacommand{qed}\isamarkupfalse%
\isanewline
\ \ \ \ \ \ \ \ \ \ \ \ \isacommand{qed}\isamarkupfalse%
\isanewline
\ \ \ \ \ \ \ \ \ \ \isacommand{qed}\isamarkupfalse%
\isanewline
\ \ \ \ \ \ \ \ \isacommand{qed}\isamarkupfalse%
\isanewline
\ \ \ \ \ \ \isacommand{next}\isamarkupfalse%
\isanewline
\ \ \ \ \ \ \ \ \isacommand{show}\isamarkupfalse%
\ {\isachardoublequoteopen}dim{\isacharunderscore}{\kern0pt}row\ {\isacharparenleft}{\kern0pt}R\ n\isactrlsup {\isasymdagger}\ {\isacharasterisk}{\kern0pt}\ R\ n{\isacharparenright}{\kern0pt}\ {\isacharequal}{\kern0pt}\ dim{\isacharunderscore}{\kern0pt}row\ {\isacharparenleft}{\kern0pt}{\isadigit{1}}\isactrlsub m\ {\isadigit{2}}{\isacharparenright}{\kern0pt}{\isachardoublequoteclose}\isanewline
\ \ \ \ \ \ \ \ \ \ \isacommand{using}\isamarkupfalse%
\ {\isacartoucheopen}dim{\isacharunderscore}{\kern0pt}row\ {\isacharparenleft}{\kern0pt}R\ n{\isacharparenright}{\kern0pt}\ {\isacharequal}{\kern0pt}\ {\isadigit{2}}\ {\isacharcircum}{\kern0pt}\ {\isadigit{1}}{\isacartoucheclose}\ {\isacartoucheopen}square{\isacharunderscore}{\kern0pt}mat\ {\isacharparenleft}{\kern0pt}R\ n{\isacharparenright}{\kern0pt}{\isacartoucheclose}\ \isacommand{by}\isamarkupfalse%
\ auto\isanewline
\ \ \ \ \ \ \isacommand{next}\isamarkupfalse%
\isanewline
\ \ \ \ \ \ \ \ \isacommand{show}\isamarkupfalse%
\ {\isachardoublequoteopen}dim{\isacharunderscore}{\kern0pt}col\ {\isacharparenleft}{\kern0pt}R\ n\isactrlsup {\isasymdagger}\ {\isacharasterisk}{\kern0pt}\ R\ n{\isacharparenright}{\kern0pt}\ {\isacharequal}{\kern0pt}\ dim{\isacharunderscore}{\kern0pt}col\ {\isacharparenleft}{\kern0pt}{\isadigit{1}}\isactrlsub m\ {\isadigit{2}}{\isacharparenright}{\kern0pt}{\isachardoublequoteclose}\isanewline
\ \ \ \ \ \ \ \ \ \ \isacommand{using}\isamarkupfalse%
\ {\isacartoucheopen}dim{\isacharunderscore}{\kern0pt}row\ {\isacharparenleft}{\kern0pt}R\ n{\isacharparenright}{\kern0pt}\ {\isacharequal}{\kern0pt}\ {\isadigit{2}}\ {\isacharcircum}{\kern0pt}\ {\isadigit{1}}{\isacartoucheclose}\ {\isacartoucheopen}square{\isacharunderscore}{\kern0pt}mat\ {\isacharparenleft}{\kern0pt}R\ n{\isacharparenright}{\kern0pt}{\isacartoucheclose}\ \isacommand{by}\isamarkupfalse%
\ auto\isanewline
\ \ \ \ \ \ \isacommand{qed}\isamarkupfalse%
\isanewline
\ \ \ \ \isacommand{next}\isamarkupfalse%
\isanewline
\ \ \ \ \ \ \isacommand{show}\isamarkupfalse%
\ {\isachardoublequoteopen}R\ n\ {\isacharasterisk}{\kern0pt}\ {\isacharparenleft}{\kern0pt}{\isacharparenleft}{\kern0pt}R\ n{\isacharparenright}{\kern0pt}\isactrlsup {\isasymdagger}{\isacharparenright}{\kern0pt}\ {\isacharequal}{\kern0pt}\ {\isadigit{1}}\isactrlsub m\ {\isadigit{2}}{\isachardoublequoteclose}\isanewline
\ \ \ \ \ \ \isacommand{proof}\isamarkupfalse%
\isanewline
\ \ \ \ \ \ \ \ \isacommand{show}\isamarkupfalse%
\ {\isachardoublequoteopen}{\isasymAnd}i\ j{\isachardot}{\kern0pt}\ i\ {\isacharless}{\kern0pt}\ dim{\isacharunderscore}{\kern0pt}row\ {\isacharparenleft}{\kern0pt}{\isadigit{1}}\isactrlsub m\ {\isadigit{2}}{\isacharparenright}{\kern0pt}\ {\isasymLongrightarrow}\ j\ {\isacharless}{\kern0pt}\ dim{\isacharunderscore}{\kern0pt}col\ {\isacharparenleft}{\kern0pt}{\isadigit{1}}\isactrlsub m\ {\isadigit{2}}{\isacharparenright}{\kern0pt}\ {\isasymLongrightarrow}\isanewline
\ \ \ \ \ \ \ \ \ \ \ \ \ \ {\isacharparenleft}{\kern0pt}R\ n\ {\isacharasterisk}{\kern0pt}\ {\isacharparenleft}{\kern0pt}R\ n\isactrlsup {\isasymdagger}{\isacharparenright}{\kern0pt}{\isacharparenright}{\kern0pt}\ {\isachardollar}{\kern0pt}{\isachardollar}{\kern0pt}\ {\isacharparenleft}{\kern0pt}i{\isacharcomma}{\kern0pt}\ j{\isacharparenright}{\kern0pt}\ {\isacharequal}{\kern0pt}\ {\isadigit{1}}\isactrlsub m\ {\isadigit{2}}\ {\isachardollar}{\kern0pt}{\isachardollar}{\kern0pt}\ {\isacharparenleft}{\kern0pt}i{\isacharcomma}{\kern0pt}\ j{\isacharparenright}{\kern0pt}{\isachardoublequoteclose}\isanewline
\ \ \ \ \ \ \ \ \isacommand{proof}\isamarkupfalse%
\ {\isacharminus}{\kern0pt}\isanewline
\ \ \ \ \ \ \ \ \ \ \isacommand{fix}\isamarkupfalse%
\ i\ j\isanewline
\ \ \ \ \ \ \ \ \ \ \isacommand{assume}\isamarkupfalse%
\ {\isachardoublequoteopen}i\ {\isacharless}{\kern0pt}\ dim{\isacharunderscore}{\kern0pt}row\ {\isacharparenleft}{\kern0pt}{\isadigit{1}}\isactrlsub m\ {\isadigit{2}}{\isacharparenright}{\kern0pt}{\isachardoublequoteclose}\isanewline
\ \ \ \ \ \ \ \ \ \ \isacommand{hence}\isamarkupfalse%
\ i{\isadigit{2}}{\isacharcolon}{\kern0pt}{\isachardoublequoteopen}i\ {\isacharless}{\kern0pt}\ {\isadigit{2}}{\isachardoublequoteclose}\ \isacommand{by}\isamarkupfalse%
\ auto\isanewline
\ \ \ \ \ \ \ \ \ \ \isacommand{assume}\isamarkupfalse%
\ {\isachardoublequoteopen}j\ {\isacharless}{\kern0pt}\ dim{\isacharunderscore}{\kern0pt}col\ {\isacharparenleft}{\kern0pt}{\isadigit{1}}\isactrlsub m\ {\isadigit{2}}{\isacharparenright}{\kern0pt}{\isachardoublequoteclose}\isanewline
\ \ \ \ \ \ \ \ \ \ \isacommand{hence}\isamarkupfalse%
\ j{\isadigit{2}}{\isacharcolon}{\kern0pt}{\isachardoublequoteopen}j\ {\isacharless}{\kern0pt}\ {\isadigit{2}}{\isachardoublequoteclose}\ \isacommand{by}\isamarkupfalse%
\ auto\isanewline
\ \ \ \ \ \ \ \ \ \ \isacommand{show}\isamarkupfalse%
\ {\isachardoublequoteopen}{\isacharparenleft}{\kern0pt}R\ n\ {\isacharasterisk}{\kern0pt}\ {\isacharparenleft}{\kern0pt}R\ n\isactrlsup {\isasymdagger}{\isacharparenright}{\kern0pt}{\isacharparenright}{\kern0pt}\ {\isachardollar}{\kern0pt}{\isachardollar}{\kern0pt}\ {\isacharparenleft}{\kern0pt}i{\isacharcomma}{\kern0pt}\ j{\isacharparenright}{\kern0pt}\ {\isacharequal}{\kern0pt}\ {\isadigit{1}}\isactrlsub m\ {\isadigit{2}}\ {\isachardollar}{\kern0pt}{\isachardollar}{\kern0pt}\ {\isacharparenleft}{\kern0pt}i{\isacharcomma}{\kern0pt}\ j{\isacharparenright}{\kern0pt}{\isachardoublequoteclose}\isanewline
\ \ \ \ \ \ \ \ \ \ \isacommand{proof}\isamarkupfalse%
\ {\isacharparenleft}{\kern0pt}rule\ disjE{\isacharparenright}{\kern0pt}\isanewline
\ \ \ \ \ \ \ \ \ \ \ \ \isacommand{show}\isamarkupfalse%
\ {\isachardoublequoteopen}i\ {\isacharequal}{\kern0pt}\ {\isadigit{0}}\ {\isasymor}\ i\ {\isacharequal}{\kern0pt}\ {\isadigit{1}}{\isachardoublequoteclose}\ \isacommand{using}\isamarkupfalse%
\ i{\isadigit{2}}\ \isacommand{by}\isamarkupfalse%
\ auto\isanewline
\ \ \ \ \ \ \ \ \ \ \isacommand{next}\isamarkupfalse%
\isanewline
\ \ \ \ \ \ \ \ \ \ \ \ \isacommand{assume}\isamarkupfalse%
\ i{\isadigit{0}}{\isacharcolon}{\kern0pt}{\isachardoublequoteopen}i\ {\isacharequal}{\kern0pt}\ {\isadigit{0}}{\isachardoublequoteclose}\isanewline
\ \ \ \ \ \ \ \ \ \ \ \ \isacommand{show}\isamarkupfalse%
\ {\isachardoublequoteopen}{\isacharparenleft}{\kern0pt}R\ n\ {\isacharasterisk}{\kern0pt}\ {\isacharparenleft}{\kern0pt}R\ n\isactrlsup {\isasymdagger}{\isacharparenright}{\kern0pt}{\isacharparenright}{\kern0pt}\ {\isachardollar}{\kern0pt}{\isachardollar}{\kern0pt}\ {\isacharparenleft}{\kern0pt}i{\isacharcomma}{\kern0pt}\ j{\isacharparenright}{\kern0pt}\ {\isacharequal}{\kern0pt}\ {\isadigit{1}}\isactrlsub m\ {\isadigit{2}}\ {\isachardollar}{\kern0pt}{\isachardollar}{\kern0pt}\ {\isacharparenleft}{\kern0pt}i{\isacharcomma}{\kern0pt}\ j{\isacharparenright}{\kern0pt}{\isachardoublequoteclose}\isanewline
\ \ \ \ \ \ \ \ \ \ \ \ \isacommand{proof}\isamarkupfalse%
\ {\isacharparenleft}{\kern0pt}rule\ disjE{\isacharparenright}{\kern0pt}\isanewline
\ \ \ \ \ \ \ \ \ \ \ \ \ \ \isacommand{show}\isamarkupfalse%
\ {\isachardoublequoteopen}j\ {\isacharequal}{\kern0pt}\ {\isadigit{0}}\ {\isasymor}\ j\ {\isacharequal}{\kern0pt}\ {\isadigit{1}}{\isachardoublequoteclose}\ \isacommand{using}\isamarkupfalse%
\ j{\isadigit{2}}\ \isacommand{by}\isamarkupfalse%
\ auto\isanewline
\ \ \ \ \ \ \ \ \ \ \ \ \isacommand{next}\isamarkupfalse%
\isanewline
\ \ \ \ \ \ \ \ \ \ \ \ \ \ \isacommand{assume}\isamarkupfalse%
\ j{\isadigit{0}}{\isacharcolon}{\kern0pt}{\isachardoublequoteopen}j\ {\isacharequal}{\kern0pt}\ {\isadigit{0}}{\isachardoublequoteclose}\isanewline
\ \ \ \ \ \ \ \ \ \ \ \ \ \ \isacommand{show}\isamarkupfalse%
\ {\isachardoublequoteopen}{\isacharparenleft}{\kern0pt}R\ n\ {\isacharasterisk}{\kern0pt}\ {\isacharparenleft}{\kern0pt}R\ n\isactrlsup {\isasymdagger}{\isacharparenright}{\kern0pt}{\isacharparenright}{\kern0pt}\ {\isachardollar}{\kern0pt}{\isachardollar}{\kern0pt}\ {\isacharparenleft}{\kern0pt}i{\isacharcomma}{\kern0pt}\ j{\isacharparenright}{\kern0pt}\ {\isacharequal}{\kern0pt}\ {\isadigit{1}}\isactrlsub m\ {\isadigit{2}}\ {\isachardollar}{\kern0pt}{\isachardollar}{\kern0pt}\ {\isacharparenleft}{\kern0pt}i{\isacharcomma}{\kern0pt}\ j{\isacharparenright}{\kern0pt}{\isachardoublequoteclose}\isanewline
\ \ \ \ \ \ \ \ \ \ \ \ \ \ \isacommand{proof}\isamarkupfalse%
\ {\isacharminus}{\kern0pt}\isanewline
\ \ \ \ \ \ \ \ \ \ \ \ \ \ \ \ \isacommand{have}\isamarkupfalse%
\ {\isachardoublequoteopen}{\isacharparenleft}{\kern0pt}R\ n\ {\isacharasterisk}{\kern0pt}\ {\isacharparenleft}{\kern0pt}R\ n\isactrlsup {\isasymdagger}{\isacharparenright}{\kern0pt}{\isacharparenright}{\kern0pt}\ {\isachardollar}{\kern0pt}{\isachardollar}{\kern0pt}\ {\isacharparenleft}{\kern0pt}{\isadigit{0}}{\isacharcomma}{\kern0pt}{\isadigit{0}}{\isacharparenright}{\kern0pt}\ {\isacharequal}{\kern0pt}\ {\isacharparenleft}{\kern0pt}{\isacharparenleft}{\kern0pt}R\ n{\isacharparenright}{\kern0pt}\ {\isachardollar}{\kern0pt}{\isachardollar}{\kern0pt}\ {\isacharparenleft}{\kern0pt}{\isadigit{0}}{\isacharcomma}{\kern0pt}{\isadigit{0}}{\isacharparenright}{\kern0pt}{\isacharparenright}{\kern0pt}\ {\isacharasterisk}{\kern0pt}\ {\isacharparenleft}{\kern0pt}{\isacharparenleft}{\kern0pt}R\ n{\isacharparenright}{\kern0pt}\isactrlsup {\isasymdagger}\ {\isachardollar}{\kern0pt}{\isachardollar}{\kern0pt}\ {\isacharparenleft}{\kern0pt}{\isadigit{0}}{\isacharcomma}{\kern0pt}{\isadigit{0}}{\isacharparenright}{\kern0pt}{\isacharparenright}{\kern0pt}\ {\isacharplus}{\kern0pt}\isanewline
\ \ \ \ \ \ \ \ \ \ \ \ \ \ \ \ \ \ \ \ \ \ {\isacharparenleft}{\kern0pt}{\isacharparenleft}{\kern0pt}R\ n{\isacharparenright}{\kern0pt}\ {\isachardollar}{\kern0pt}{\isachardollar}{\kern0pt}\ {\isacharparenleft}{\kern0pt}{\isadigit{0}}{\isacharcomma}{\kern0pt}{\isadigit{1}}{\isacharparenright}{\kern0pt}{\isacharparenright}{\kern0pt}\ {\isacharasterisk}{\kern0pt}\ {\isacharparenleft}{\kern0pt}{\isacharparenleft}{\kern0pt}R\ n{\isacharparenright}{\kern0pt}\isactrlsup {\isasymdagger}\ {\isachardollar}{\kern0pt}{\isachardollar}{\kern0pt}\ {\isacharparenleft}{\kern0pt}{\isadigit{1}}{\isacharcomma}{\kern0pt}{\isadigit{0}}{\isacharparenright}{\kern0pt}{\isacharparenright}{\kern0pt}{\isachardoublequoteclose}\isanewline
\ \ \ \ \ \ \ \ \ \ \ \ \ \ \ \ \ \ \isacommand{using}\isamarkupfalse%
\ {\isacartoucheopen}dim{\isacharunderscore}{\kern0pt}row\ {\isacharparenleft}{\kern0pt}R\ n{\isacharparenright}{\kern0pt}\ {\isacharequal}{\kern0pt}\ {\isadigit{2}}\ {\isacharcircum}{\kern0pt}\ {\isadigit{1}}{\isacartoucheclose}\ {\isacartoucheopen}square{\isacharunderscore}{\kern0pt}mat\ {\isacharparenleft}{\kern0pt}R\ n{\isacharparenright}{\kern0pt}{\isacartoucheclose}\ sumof{\isadigit{2}}\ \isacommand{by}\isamarkupfalse%
\ fastforce\isanewline
\ \ \ \ \ \ \ \ \ \ \ \ \ \ \ \ \isacommand{also}\isamarkupfalse%
\ \isacommand{have}\isamarkupfalse%
\ {\isachardoublequoteopen}{\isasymdots}\ {\isacharequal}{\kern0pt}\ {\isadigit{1}}{\isachardoublequoteclose}\ \isacommand{using}\isamarkupfalse%
\ R{\isacharunderscore}{\kern0pt}dagger{\isacharunderscore}{\kern0pt}mat\ R{\isacharunderscore}{\kern0pt}def\ index{\isacharunderscore}{\kern0pt}mat{\isacharunderscore}{\kern0pt}of{\isacharunderscore}{\kern0pt}cols{\isacharunderscore}{\kern0pt}list\ \isacommand{by}\isamarkupfalse%
\ simp\isanewline
\ \ \ \ \ \ \ \ \ \ \ \ \ \ \ \ \isacommand{also}\isamarkupfalse%
\ \isacommand{have}\isamarkupfalse%
\ {\isachardoublequoteopen}{\isasymdots}\ {\isacharequal}{\kern0pt}\ {\isadigit{1}}\isactrlsub m\ {\isadigit{2}}\ {\isachardollar}{\kern0pt}{\isachardollar}{\kern0pt}\ {\isacharparenleft}{\kern0pt}{\isadigit{0}}{\isacharcomma}{\kern0pt}{\isadigit{0}}{\isacharparenright}{\kern0pt}{\isachardoublequoteclose}\ \isacommand{by}\isamarkupfalse%
\ simp\isanewline
\ \ \ \ \ \ \ \ \ \ \ \ \ \ \ \ \isacommand{finally}\isamarkupfalse%
\ \isacommand{show}\isamarkupfalse%
\ {\isacharquery}{\kern0pt}thesis\ \isacommand{using}\isamarkupfalse%
\ i{\isadigit{0}}\ j{\isadigit{0}}\ \isacommand{by}\isamarkupfalse%
\ simp\isanewline
\ \ \ \ \ \ \ \ \ \ \ \ \ \ \isacommand{qed}\isamarkupfalse%
\isanewline
\ \ \ \ \ \ \ \ \ \ \ \ \isacommand{next}\isamarkupfalse%
\isanewline
\ \ \ \ \ \ \ \ \ \ \ \ \ \ \isacommand{assume}\isamarkupfalse%
\ j{\isadigit{1}}{\isacharcolon}{\kern0pt}{\isachardoublequoteopen}j\ {\isacharequal}{\kern0pt}\ {\isadigit{1}}{\isachardoublequoteclose}\isanewline
\ \ \ \ \ \ \ \ \ \ \ \ \ \ \isacommand{show}\isamarkupfalse%
\ {\isachardoublequoteopen}{\isacharparenleft}{\kern0pt}R\ n\ {\isacharasterisk}{\kern0pt}\ {\isacharparenleft}{\kern0pt}R\ n\isactrlsup {\isasymdagger}{\isacharparenright}{\kern0pt}{\isacharparenright}{\kern0pt}\ {\isachardollar}{\kern0pt}{\isachardollar}{\kern0pt}\ {\isacharparenleft}{\kern0pt}i{\isacharcomma}{\kern0pt}\ j{\isacharparenright}{\kern0pt}\ {\isacharequal}{\kern0pt}\ {\isadigit{1}}\isactrlsub m\ {\isadigit{2}}\ {\isachardollar}{\kern0pt}{\isachardollar}{\kern0pt}\ {\isacharparenleft}{\kern0pt}i{\isacharcomma}{\kern0pt}\ j{\isacharparenright}{\kern0pt}{\isachardoublequoteclose}\isanewline
\ \ \ \ \ \ \ \ \ \ \ \ \ \ \isacommand{proof}\isamarkupfalse%
\ {\isacharminus}{\kern0pt}\isanewline
\ \ \ \ \ \ \ \ \ \ \ \ \ \ \ \ \isacommand{have}\isamarkupfalse%
\ {\isachardoublequoteopen}{\isacharparenleft}{\kern0pt}R\ n\ {\isacharasterisk}{\kern0pt}\ {\isacharparenleft}{\kern0pt}R\ n\isactrlsup {\isasymdagger}{\isacharparenright}{\kern0pt}{\isacharparenright}{\kern0pt}\ {\isachardollar}{\kern0pt}{\isachardollar}{\kern0pt}\ {\isacharparenleft}{\kern0pt}{\isadigit{0}}{\isacharcomma}{\kern0pt}{\isadigit{1}}{\isacharparenright}{\kern0pt}\ {\isacharequal}{\kern0pt}\ {\isacharparenleft}{\kern0pt}{\isacharparenleft}{\kern0pt}R\ n{\isacharparenright}{\kern0pt}\ {\isachardollar}{\kern0pt}{\isachardollar}{\kern0pt}\ {\isacharparenleft}{\kern0pt}{\isadigit{0}}{\isacharcomma}{\kern0pt}{\isadigit{0}}{\isacharparenright}{\kern0pt}{\isacharparenright}{\kern0pt}\ {\isacharasterisk}{\kern0pt}\ {\isacharparenleft}{\kern0pt}{\isacharparenleft}{\kern0pt}R\ n{\isacharparenright}{\kern0pt}\isactrlsup {\isasymdagger}\ {\isachardollar}{\kern0pt}{\isachardollar}{\kern0pt}\ {\isacharparenleft}{\kern0pt}{\isadigit{0}}{\isacharcomma}{\kern0pt}{\isadigit{1}}{\isacharparenright}{\kern0pt}{\isacharparenright}{\kern0pt}\ {\isacharplus}{\kern0pt}\isanewline
\ \ \ \ \ \ \ \ \ \ \ \ \ \ \ \ \ \ \ \ \ \ {\isacharparenleft}{\kern0pt}{\isacharparenleft}{\kern0pt}R\ n{\isacharparenright}{\kern0pt}\ {\isachardollar}{\kern0pt}{\isachardollar}{\kern0pt}\ {\isacharparenleft}{\kern0pt}{\isadigit{0}}{\isacharcomma}{\kern0pt}{\isadigit{1}}{\isacharparenright}{\kern0pt}{\isacharparenright}{\kern0pt}\ {\isacharasterisk}{\kern0pt}\ {\isacharparenleft}{\kern0pt}{\isacharparenleft}{\kern0pt}R\ n{\isacharparenright}{\kern0pt}\isactrlsup {\isasymdagger}\ {\isachardollar}{\kern0pt}{\isachardollar}{\kern0pt}\ {\isacharparenleft}{\kern0pt}{\isadigit{1}}{\isacharcomma}{\kern0pt}{\isadigit{1}}{\isacharparenright}{\kern0pt}{\isacharparenright}{\kern0pt}{\isachardoublequoteclose}\isanewline
\ \ \ \ \ \ \ \ \ \ \ \ \ \ \ \ \ \ \isacommand{using}\isamarkupfalse%
\ {\isacartoucheopen}dim{\isacharunderscore}{\kern0pt}row\ {\isacharparenleft}{\kern0pt}R\ n{\isacharparenright}{\kern0pt}\ {\isacharequal}{\kern0pt}\ {\isadigit{2}}\ {\isacharcircum}{\kern0pt}\ {\isadigit{1}}{\isacartoucheclose}\ {\isacartoucheopen}square{\isacharunderscore}{\kern0pt}mat\ {\isacharparenleft}{\kern0pt}R\ n{\isacharparenright}{\kern0pt}{\isacartoucheclose}\ sumof{\isadigit{2}}\ \isacommand{by}\isamarkupfalse%
\ fastforce\isanewline
\ \ \ \ \ \ \ \ \ \ \ \ \ \ \ \ \isacommand{also}\isamarkupfalse%
\ \isacommand{have}\isamarkupfalse%
\ {\isachardoublequoteopen}{\isasymdots}\ {\isacharequal}{\kern0pt}\ {\isadigit{0}}{\isachardoublequoteclose}\ \isacommand{using}\isamarkupfalse%
\ R{\isacharunderscore}{\kern0pt}dagger{\isacharunderscore}{\kern0pt}mat\ R{\isacharunderscore}{\kern0pt}def\ index{\isacharunderscore}{\kern0pt}mat{\isacharunderscore}{\kern0pt}of{\isacharunderscore}{\kern0pt}cols{\isacharunderscore}{\kern0pt}list\ \isacommand{by}\isamarkupfalse%
\ simp\isanewline
\ \ \ \ \ \ \ \ \ \ \ \ \ \ \ \ \isacommand{also}\isamarkupfalse%
\ \isacommand{have}\isamarkupfalse%
\ {\isachardoublequoteopen}{\isasymdots}\ {\isacharequal}{\kern0pt}\ {\isadigit{1}}\isactrlsub m\ {\isadigit{2}}\ {\isachardollar}{\kern0pt}{\isachardollar}{\kern0pt}\ {\isacharparenleft}{\kern0pt}{\isadigit{0}}{\isacharcomma}{\kern0pt}{\isadigit{1}}{\isacharparenright}{\kern0pt}{\isachardoublequoteclose}\ \isacommand{by}\isamarkupfalse%
\ simp\isanewline
\ \ \ \ \ \ \ \ \ \ \ \ \ \ \ \ \isacommand{finally}\isamarkupfalse%
\ \isacommand{show}\isamarkupfalse%
\ {\isacharquery}{\kern0pt}thesis\ \isacommand{using}\isamarkupfalse%
\ i{\isadigit{0}}\ j{\isadigit{1}}\ \isacommand{by}\isamarkupfalse%
\ simp\isanewline
\ \ \ \ \ \ \ \ \ \ \ \ \ \ \isacommand{qed}\isamarkupfalse%
\isanewline
\ \ \ \ \ \ \ \ \ \ \ \ \isacommand{qed}\isamarkupfalse%
\isanewline
\ \ \ \ \ \ \ \ \ \ \isacommand{next}\isamarkupfalse%
\isanewline
\ \ \ \ \ \ \ \ \ \ \ \ \isacommand{assume}\isamarkupfalse%
\ i{\isadigit{1}}{\isacharcolon}{\kern0pt}{\isachardoublequoteopen}i\ {\isacharequal}{\kern0pt}\ {\isadigit{1}}{\isachardoublequoteclose}\isanewline
\ \ \ \ \ \ \ \ \ \ \ \ \isacommand{show}\isamarkupfalse%
\ {\isachardoublequoteopen}{\isacharparenleft}{\kern0pt}R\ n\ {\isacharasterisk}{\kern0pt}\ {\isacharparenleft}{\kern0pt}R\ n\isactrlsup {\isasymdagger}{\isacharparenright}{\kern0pt}{\isacharparenright}{\kern0pt}\ {\isachardollar}{\kern0pt}{\isachardollar}{\kern0pt}\ {\isacharparenleft}{\kern0pt}i{\isacharcomma}{\kern0pt}\ j{\isacharparenright}{\kern0pt}\ {\isacharequal}{\kern0pt}\ {\isadigit{1}}\isactrlsub m\ {\isadigit{2}}\ {\isachardollar}{\kern0pt}{\isachardollar}{\kern0pt}\ {\isacharparenleft}{\kern0pt}i{\isacharcomma}{\kern0pt}\ j{\isacharparenright}{\kern0pt}{\isachardoublequoteclose}\isanewline
\ \ \ \ \ \ \ \ \ \ \ \ \isacommand{proof}\isamarkupfalse%
\ {\isacharparenleft}{\kern0pt}rule\ disjE{\isacharparenright}{\kern0pt}\isanewline
\ \ \ \ \ \ \ \ \ \ \ \ \ \ \isacommand{show}\isamarkupfalse%
\ {\isachardoublequoteopen}j\ {\isacharequal}{\kern0pt}\ {\isadigit{0}}\ {\isasymor}\ j\ {\isacharequal}{\kern0pt}\ {\isadigit{1}}{\isachardoublequoteclose}\ \isacommand{using}\isamarkupfalse%
\ j{\isadigit{2}}\ \isacommand{by}\isamarkupfalse%
\ auto\isanewline
\ \ \ \ \ \ \ \ \ \ \ \ \isacommand{next}\isamarkupfalse%
\isanewline
\ \ \ \ \ \ \ \ \ \ \ \ \ \ \isacommand{assume}\isamarkupfalse%
\ j{\isadigit{0}}{\isacharcolon}{\kern0pt}{\isachardoublequoteopen}j\ {\isacharequal}{\kern0pt}\ {\isadigit{0}}{\isachardoublequoteclose}\isanewline
\ \ \ \ \ \ \ \ \ \ \ \ \ \ \isacommand{show}\isamarkupfalse%
\ {\isachardoublequoteopen}{\isacharparenleft}{\kern0pt}R\ n\ {\isacharasterisk}{\kern0pt}\ {\isacharparenleft}{\kern0pt}R\ n\isactrlsup {\isasymdagger}{\isacharparenright}{\kern0pt}{\isacharparenright}{\kern0pt}\ {\isachardollar}{\kern0pt}{\isachardollar}{\kern0pt}\ {\isacharparenleft}{\kern0pt}i{\isacharcomma}{\kern0pt}\ j{\isacharparenright}{\kern0pt}\ {\isacharequal}{\kern0pt}\ {\isadigit{1}}\isactrlsub m\ {\isadigit{2}}\ {\isachardollar}{\kern0pt}{\isachardollar}{\kern0pt}\ {\isacharparenleft}{\kern0pt}i{\isacharcomma}{\kern0pt}\ j{\isacharparenright}{\kern0pt}{\isachardoublequoteclose}\isanewline
\ \ \ \ \ \ \ \ \ \ \ \ \ \ \isacommand{proof}\isamarkupfalse%
\ {\isacharminus}{\kern0pt}\isanewline
\ \ \ \ \ \ \ \ \ \ \ \ \ \ \ \ \isacommand{have}\isamarkupfalse%
\ {\isachardoublequoteopen}{\isacharparenleft}{\kern0pt}R\ n\ {\isacharasterisk}{\kern0pt}\ {\isacharparenleft}{\kern0pt}R\ n\isactrlsup {\isasymdagger}{\isacharparenright}{\kern0pt}{\isacharparenright}{\kern0pt}\ {\isachardollar}{\kern0pt}{\isachardollar}{\kern0pt}\ {\isacharparenleft}{\kern0pt}{\isadigit{1}}{\isacharcomma}{\kern0pt}{\isadigit{0}}{\isacharparenright}{\kern0pt}\ {\isacharequal}{\kern0pt}\ {\isacharparenleft}{\kern0pt}{\isacharparenleft}{\kern0pt}R\ n{\isacharparenright}{\kern0pt}\ {\isachardollar}{\kern0pt}{\isachardollar}{\kern0pt}\ {\isacharparenleft}{\kern0pt}{\isadigit{1}}{\isacharcomma}{\kern0pt}{\isadigit{0}}{\isacharparenright}{\kern0pt}{\isacharparenright}{\kern0pt}\ {\isacharasterisk}{\kern0pt}\ {\isacharparenleft}{\kern0pt}{\isacharparenleft}{\kern0pt}R\ n{\isacharparenright}{\kern0pt}\isactrlsup {\isasymdagger}\ {\isachardollar}{\kern0pt}{\isachardollar}{\kern0pt}\ {\isacharparenleft}{\kern0pt}{\isadigit{0}}{\isacharcomma}{\kern0pt}{\isadigit{0}}{\isacharparenright}{\kern0pt}{\isacharparenright}{\kern0pt}\ {\isacharplus}{\kern0pt}\isanewline
\ \ \ \ \ \ \ \ \ \ \ \ \ \ \ \ \ \ \ \ \ \ {\isacharparenleft}{\kern0pt}{\isacharparenleft}{\kern0pt}R\ n{\isacharparenright}{\kern0pt}\ {\isachardollar}{\kern0pt}{\isachardollar}{\kern0pt}\ {\isacharparenleft}{\kern0pt}{\isadigit{1}}{\isacharcomma}{\kern0pt}{\isadigit{1}}{\isacharparenright}{\kern0pt}{\isacharparenright}{\kern0pt}\ {\isacharasterisk}{\kern0pt}\ {\isacharparenleft}{\kern0pt}{\isacharparenleft}{\kern0pt}R\ n{\isacharparenright}{\kern0pt}\isactrlsup {\isasymdagger}\ {\isachardollar}{\kern0pt}{\isachardollar}{\kern0pt}\ {\isacharparenleft}{\kern0pt}{\isadigit{1}}{\isacharcomma}{\kern0pt}{\isadigit{0}}{\isacharparenright}{\kern0pt}{\isacharparenright}{\kern0pt}{\isachardoublequoteclose}\isanewline
\ \ \ \ \ \ \ \ \ \ \ \ \ \ \ \ \ \ \isacommand{using}\isamarkupfalse%
\ {\isacartoucheopen}dim{\isacharunderscore}{\kern0pt}row\ {\isacharparenleft}{\kern0pt}R\ n{\isacharparenright}{\kern0pt}\ {\isacharequal}{\kern0pt}\ {\isadigit{2}}\ {\isacharcircum}{\kern0pt}\ {\isadigit{1}}{\isacartoucheclose}\ {\isacartoucheopen}square{\isacharunderscore}{\kern0pt}mat\ {\isacharparenleft}{\kern0pt}R\ n{\isacharparenright}{\kern0pt}{\isacartoucheclose}\ sumof{\isadigit{2}}\ \isacommand{by}\isamarkupfalse%
\ fastforce\isanewline
\ \ \ \ \ \ \ \ \ \ \ \ \ \ \ \ \isacommand{also}\isamarkupfalse%
\ \isacommand{have}\isamarkupfalse%
\ {\isachardoublequoteopen}{\isasymdots}\ {\isacharequal}{\kern0pt}\ {\isadigit{1}}\isactrlsub m\ {\isadigit{2}}\ {\isachardollar}{\kern0pt}{\isachardollar}{\kern0pt}\ {\isacharparenleft}{\kern0pt}{\isadigit{1}}{\isacharcomma}{\kern0pt}{\isadigit{0}}{\isacharparenright}{\kern0pt}{\isachardoublequoteclose}\ \isanewline
\ \ \ \ \ \ \ \ \ \ \ \ \ \ \ \ \ \ \isacommand{using}\isamarkupfalse%
\ R{\isacharunderscore}{\kern0pt}dagger{\isacharunderscore}{\kern0pt}mat\ R{\isacharunderscore}{\kern0pt}def\ index{\isacharunderscore}{\kern0pt}mat{\isacharunderscore}{\kern0pt}of{\isacharunderscore}{\kern0pt}cols{\isacharunderscore}{\kern0pt}list\ \isacommand{by}\isamarkupfalse%
\ simp\isanewline
\ \ \ \ \ \ \ \ \ \ \ \ \ \ \ \ \isacommand{finally}\isamarkupfalse%
\ \isacommand{show}\isamarkupfalse%
\ {\isacharquery}{\kern0pt}thesis\ \isacommand{using}\isamarkupfalse%
\ i{\isadigit{1}}\ j{\isadigit{0}}\ \isacommand{by}\isamarkupfalse%
\ simp\isanewline
\ \ \ \ \ \ \ \ \ \ \ \ \ \ \isacommand{qed}\isamarkupfalse%
\isanewline
\ \ \ \ \ \ \ \ \ \ \ \ \isacommand{next}\isamarkupfalse%
\isanewline
\ \ \ \ \ \ \ \ \ \ \ \ \ \ \isacommand{assume}\isamarkupfalse%
\ j{\isadigit{1}}{\isacharcolon}{\kern0pt}{\isachardoublequoteopen}j\ {\isacharequal}{\kern0pt}\ {\isadigit{1}}{\isachardoublequoteclose}\isanewline
\ \ \ \ \ \ \ \ \ \ \ \ \ \ \isacommand{show}\isamarkupfalse%
\ {\isachardoublequoteopen}{\isacharparenleft}{\kern0pt}R\ n\ {\isacharasterisk}{\kern0pt}\ {\isacharparenleft}{\kern0pt}R\ n\isactrlsup {\isasymdagger}{\isacharparenright}{\kern0pt}{\isacharparenright}{\kern0pt}\ {\isachardollar}{\kern0pt}{\isachardollar}{\kern0pt}\ {\isacharparenleft}{\kern0pt}i{\isacharcomma}{\kern0pt}\ j{\isacharparenright}{\kern0pt}\ {\isacharequal}{\kern0pt}\ {\isadigit{1}}\isactrlsub m\ {\isadigit{2}}\ {\isachardollar}{\kern0pt}{\isachardollar}{\kern0pt}\ {\isacharparenleft}{\kern0pt}i{\isacharcomma}{\kern0pt}\ j{\isacharparenright}{\kern0pt}{\isachardoublequoteclose}\ \isanewline
\ \ \ \ \ \ \ \ \ \ \ \ \ \ \isacommand{proof}\isamarkupfalse%
\ {\isacharminus}{\kern0pt}\isanewline
\ \ \ \ \ \ \ \ \ \ \ \ \ \ \ \ \isacommand{have}\isamarkupfalse%
\ {\isachardoublequoteopen}{\isacharparenleft}{\kern0pt}R\ n\ {\isacharasterisk}{\kern0pt}\ {\isacharparenleft}{\kern0pt}R\ n\isactrlsup {\isasymdagger}{\isacharparenright}{\kern0pt}{\isacharparenright}{\kern0pt}\ {\isachardollar}{\kern0pt}{\isachardollar}{\kern0pt}\ {\isacharparenleft}{\kern0pt}{\isadigit{1}}{\isacharcomma}{\kern0pt}{\isadigit{1}}{\isacharparenright}{\kern0pt}\ {\isacharequal}{\kern0pt}\ {\isacharparenleft}{\kern0pt}{\isacharparenleft}{\kern0pt}R\ n{\isacharparenright}{\kern0pt}\ {\isachardollar}{\kern0pt}{\isachardollar}{\kern0pt}\ {\isacharparenleft}{\kern0pt}{\isadigit{1}}{\isacharcomma}{\kern0pt}{\isadigit{0}}{\isacharparenright}{\kern0pt}{\isacharparenright}{\kern0pt}\ {\isacharasterisk}{\kern0pt}\ {\isacharparenleft}{\kern0pt}{\isacharparenleft}{\kern0pt}R\ n{\isacharparenright}{\kern0pt}\isactrlsup {\isasymdagger}\ {\isachardollar}{\kern0pt}{\isachardollar}{\kern0pt}\ {\isacharparenleft}{\kern0pt}{\isadigit{0}}{\isacharcomma}{\kern0pt}{\isadigit{1}}{\isacharparenright}{\kern0pt}{\isacharparenright}{\kern0pt}\ {\isacharplus}{\kern0pt}\isanewline
\ \ \ \ \ \ \ \ \ \ \ \ \ \ \ \ \ \ \ \ \ \ {\isacharparenleft}{\kern0pt}{\isacharparenleft}{\kern0pt}R\ n{\isacharparenright}{\kern0pt}\ {\isachardollar}{\kern0pt}{\isachardollar}{\kern0pt}\ {\isacharparenleft}{\kern0pt}{\isadigit{1}}{\isacharcomma}{\kern0pt}{\isadigit{1}}{\isacharparenright}{\kern0pt}{\isacharparenright}{\kern0pt}\ {\isacharasterisk}{\kern0pt}\ {\isacharparenleft}{\kern0pt}{\isacharparenleft}{\kern0pt}R\ n{\isacharparenright}{\kern0pt}\isactrlsup {\isasymdagger}\ {\isachardollar}{\kern0pt}{\isachardollar}{\kern0pt}\ {\isacharparenleft}{\kern0pt}{\isadigit{1}}{\isacharcomma}{\kern0pt}{\isadigit{1}}{\isacharparenright}{\kern0pt}{\isacharparenright}{\kern0pt}{\isachardoublequoteclose}\isanewline
\ \ \ \ \ \ \ \ \ \ \ \ \ \ \ \ \ \ \isacommand{using}\isamarkupfalse%
\ {\isacartoucheopen}dim{\isacharunderscore}{\kern0pt}row\ {\isacharparenleft}{\kern0pt}R\ n{\isacharparenright}{\kern0pt}\ {\isacharequal}{\kern0pt}\ {\isadigit{2}}\ {\isacharcircum}{\kern0pt}\ {\isadigit{1}}{\isacartoucheclose}\ {\isacartoucheopen}square{\isacharunderscore}{\kern0pt}mat\ {\isacharparenleft}{\kern0pt}R\ n{\isacharparenright}{\kern0pt}{\isacartoucheclose}\ sumof{\isadigit{2}}\ \isacommand{by}\isamarkupfalse%
\ fastforce\isanewline
\ \ \ \ \ \ \ \ \ \ \ \ \ \ \ \ \isacommand{also}\isamarkupfalse%
\ \isacommand{have}\isamarkupfalse%
\ {\isachardoublequoteopen}{\isasymdots}\ {\isacharequal}{\kern0pt}\ exp{\isacharparenleft}{\kern0pt}{\isadigit{2}}{\isacharasterisk}{\kern0pt}pi{\isacharasterisk}{\kern0pt}{\isasymi}{\isacharslash}{\kern0pt}{\isadigit{2}}{\isacharcircum}{\kern0pt}n{\isacharparenright}{\kern0pt}\ {\isacharasterisk}{\kern0pt}\ exp{\isacharparenleft}{\kern0pt}{\isacharminus}{\kern0pt}{\isadigit{2}}{\isacharasterisk}{\kern0pt}pi{\isacharasterisk}{\kern0pt}{\isasymi}{\isacharslash}{\kern0pt}{\isadigit{2}}{\isacharcircum}{\kern0pt}n{\isacharparenright}{\kern0pt}{\isachardoublequoteclose}\isanewline
\ \ \ \ \ \ \ \ \ \ \ \ \ \ \ \ \ \ \isacommand{using}\isamarkupfalse%
\ R{\isacharunderscore}{\kern0pt}dagger{\isacharunderscore}{\kern0pt}mat\ R{\isacharunderscore}{\kern0pt}def\ index{\isacharunderscore}{\kern0pt}mat{\isacharunderscore}{\kern0pt}of{\isacharunderscore}{\kern0pt}cols{\isacharunderscore}{\kern0pt}list\ \isacommand{by}\isamarkupfalse%
\ simp\isanewline
\ \ \ \ \ \ \ \ \ \ \ \ \ \ \ \ \isacommand{also}\isamarkupfalse%
\ \isacommand{have}\isamarkupfalse%
\ {\isachardoublequoteopen}{\isasymdots}\ {\isacharequal}{\kern0pt}\ {\isadigit{1}}{\isachardoublequoteclose}\isanewline
\ \ \ \ \ \ \ \ \ \ \ \ \ \ \ \ \ \ \isacommand{by}\isamarkupfalse%
\ {\isacharparenleft}{\kern0pt}simp\ add{\isacharcolon}{\kern0pt}\ exp{\isacharunderscore}{\kern0pt}minus{\isacharunderscore}{\kern0pt}inverse{\isacharparenright}{\kern0pt}\isanewline
\ \ \ \ \ \ \ \ \ \ \ \ \ \ \ \ \isacommand{also}\isamarkupfalse%
\ \isacommand{have}\isamarkupfalse%
\ {\isachardoublequoteopen}{\isasymdots}\ {\isacharequal}{\kern0pt}\ {\isadigit{1}}\isactrlsub m\ {\isadigit{2}}\ {\isachardollar}{\kern0pt}{\isachardollar}{\kern0pt}\ {\isacharparenleft}{\kern0pt}{\isadigit{1}}{\isacharcomma}{\kern0pt}{\isadigit{1}}{\isacharparenright}{\kern0pt}{\isachardoublequoteclose}\ \isacommand{by}\isamarkupfalse%
\ simp\isanewline
\ \ \ \ \ \ \ \ \ \ \ \ \ \ \ \ \isacommand{finally}\isamarkupfalse%
\ \isacommand{show}\isamarkupfalse%
\ {\isacharquery}{\kern0pt}thesis\ \isacommand{using}\isamarkupfalse%
\ i{\isadigit{1}}\ j{\isadigit{1}}\ \isacommand{by}\isamarkupfalse%
\ simp\isanewline
\ \ \ \ \ \ \ \ \ \ \ \ \ \ \isacommand{qed}\isamarkupfalse%
\isanewline
\ \ \ \ \ \ \ \ \ \ \ \ \isacommand{qed}\isamarkupfalse%
\isanewline
\ \ \ \ \ \ \ \ \ \ \isacommand{qed}\isamarkupfalse%
\isanewline
\ \ \ \ \ \ \ \ \isacommand{qed}\isamarkupfalse%
\isanewline
\ \ \ \ \ \ \isacommand{next}\isamarkupfalse%
\isanewline
\ \ \ \ \ \ \ \ \isacommand{show}\isamarkupfalse%
\ {\isachardoublequoteopen}dim{\isacharunderscore}{\kern0pt}row\ {\isacharparenleft}{\kern0pt}R\ n\ {\isacharasterisk}{\kern0pt}\ {\isacharparenleft}{\kern0pt}R\ n\isactrlsup {\isasymdagger}{\isacharparenright}{\kern0pt}{\isacharparenright}{\kern0pt}\ {\isacharequal}{\kern0pt}\ dim{\isacharunderscore}{\kern0pt}row\ {\isacharparenleft}{\kern0pt}{\isadigit{1}}\isactrlsub m\ {\isadigit{2}}{\isacharparenright}{\kern0pt}{\isachardoublequoteclose}\isanewline
\ \ \ \ \ \ \ \ \ \ \isacommand{by}\isamarkupfalse%
\ {\isacharparenleft}{\kern0pt}simp\ add{\isacharcolon}{\kern0pt}\ {\isacartoucheopen}dim{\isacharunderscore}{\kern0pt}row\ {\isacharparenleft}{\kern0pt}R\ n{\isacharparenright}{\kern0pt}\ {\isacharequal}{\kern0pt}\ {\isadigit{2}}\ {\isacharcircum}{\kern0pt}\ {\isadigit{1}}{\isacartoucheclose}{\isacharparenright}{\kern0pt}\isanewline
\ \ \ \ \ \ \isacommand{next}\isamarkupfalse%
\isanewline
\ \ \ \ \ \ \ \ \isacommand{show}\isamarkupfalse%
\ {\isachardoublequoteopen}dim{\isacharunderscore}{\kern0pt}col\ {\isacharparenleft}{\kern0pt}R\ n\ {\isacharasterisk}{\kern0pt}\ {\isacharparenleft}{\kern0pt}R\ n\isactrlsup {\isasymdagger}{\isacharparenright}{\kern0pt}{\isacharparenright}{\kern0pt}\ {\isacharequal}{\kern0pt}\ dim{\isacharunderscore}{\kern0pt}col\ {\isacharparenleft}{\kern0pt}{\isadigit{1}}\isactrlsub m\ {\isadigit{2}}{\isacharparenright}{\kern0pt}{\isachardoublequoteclose}\isanewline
\ \ \ \ \ \ \ \ \ \ \isacommand{by}\isamarkupfalse%
\ {\isacharparenleft}{\kern0pt}simp\ add{\isacharcolon}{\kern0pt}\ {\isacartoucheopen}dim{\isacharunderscore}{\kern0pt}row\ {\isacharparenleft}{\kern0pt}R\ n{\isacharparenright}{\kern0pt}\ {\isacharequal}{\kern0pt}\ {\isadigit{2}}\ {\isacharcircum}{\kern0pt}\ {\isadigit{1}}{\isacartoucheclose}{\isacharparenright}{\kern0pt}\isanewline
\ \ \ \ \ \ \isacommand{qed}\isamarkupfalse%
\isanewline
\ \ \ \ \isacommand{qed}\isamarkupfalse%
\isanewline
\ \ \ \ \isacommand{thus}\isamarkupfalse%
\ {\isacharquery}{\kern0pt}thesis\ \isacommand{using}\isamarkupfalse%
\ unitary{\isacharunderscore}{\kern0pt}def\ R{\isacharunderscore}{\kern0pt}def\ mat{\isacharunderscore}{\kern0pt}of{\isacharunderscore}{\kern0pt}cols{\isacharunderscore}{\kern0pt}list{\isacharunderscore}{\kern0pt}def\ \isacommand{by}\isamarkupfalse%
\ auto\isanewline
\ \ \isacommand{qed}\isamarkupfalse%
\isanewline
\isacommand{qed}\isamarkupfalse%
%
\endisatagproof
{\isafoldproof}%
%
\isadelimproof
\isanewline
%
\endisadelimproof
\isanewline
\isacommand{lemma}\isamarkupfalse%
\ SWAP{\isacharunderscore}{\kern0pt}dagger{\isacharunderscore}{\kern0pt}mat{\isacharcolon}{\kern0pt}\isanewline
\ \ \isakeyword{shows}\ {\isachardoublequoteopen}SWAP\isactrlsup {\isasymdagger}\ {\isacharequal}{\kern0pt}\ SWAP{\isachardoublequoteclose}\isanewline
%
\isadelimproof
%
\endisadelimproof
%
\isatagproof
\isacommand{proof}\isamarkupfalse%
\ {\isacharminus}{\kern0pt}\isanewline
\ \ \isacommand{have}\isamarkupfalse%
\ {\isachardoublequoteopen}SWAP\isactrlsup {\isasymdagger}\ {\isacharequal}{\kern0pt}\ Matrix{\isachardot}{\kern0pt}mat\ {\isadigit{4}}\ {\isadigit{4}}\ {\isacharparenleft}{\kern0pt}{\isasymlambda}{\isacharparenleft}{\kern0pt}i{\isacharcomma}{\kern0pt}j{\isacharparenright}{\kern0pt}{\isachardot}{\kern0pt}\ cnj\ {\isacharparenleft}{\kern0pt}SWAP\ {\isachardollar}{\kern0pt}{\isachardollar}{\kern0pt}\ {\isacharparenleft}{\kern0pt}j{\isacharcomma}{\kern0pt}i{\isacharparenright}{\kern0pt}{\isacharparenright}{\kern0pt}{\isacharparenright}{\kern0pt}{\isachardoublequoteclose}\ \isanewline
\ \ \ \ \isacommand{using}\isamarkupfalse%
\ dagger{\isacharunderscore}{\kern0pt}def\ SWAP{\isacharunderscore}{\kern0pt}carrier{\isacharunderscore}{\kern0pt}mat\ \isanewline
\ \ \ \ \isacommand{by}\isamarkupfalse%
\ {\isacharparenleft}{\kern0pt}metis\ SWAP{\isacharunderscore}{\kern0pt}ncols\ carrier{\isacharunderscore}{\kern0pt}matD{\isacharparenleft}{\kern0pt}{\isadigit{1}}{\isacharparenright}{\kern0pt}{\isacharparenright}{\kern0pt}\isanewline
\ \ \isacommand{also}\isamarkupfalse%
\ \isacommand{have}\isamarkupfalse%
\ {\isachardoublequoteopen}{\isasymdots}\ {\isacharequal}{\kern0pt}\ Matrix{\isachardot}{\kern0pt}mat\ {\isadigit{4}}\ {\isadigit{4}}\ {\isacharparenleft}{\kern0pt}{\isasymlambda}{\isacharparenleft}{\kern0pt}i{\isacharcomma}{\kern0pt}j{\isacharparenright}{\kern0pt}{\isachardot}{\kern0pt}\ cnj\ {\isacharparenleft}{\kern0pt}SWAP\ {\isachardollar}{\kern0pt}{\isachardollar}{\kern0pt}\ {\isacharparenleft}{\kern0pt}i{\isacharcomma}{\kern0pt}j{\isacharparenright}{\kern0pt}{\isacharparenright}{\kern0pt}{\isacharparenright}{\kern0pt}{\isachardoublequoteclose}\ \isanewline
\ \ \ \ \isacommand{using}\isamarkupfalse%
\ SWAP{\isacharunderscore}{\kern0pt}def\ SWAP{\isacharunderscore}{\kern0pt}index\isanewline
\ \ \isacommand{proof}\isamarkupfalse%
\ {\isacharminus}{\kern0pt}\isanewline
\ \ \ \ \isacommand{obtain}\isamarkupfalse%
\ nn\ {\isacharcolon}{\kern0pt}{\isacharcolon}{\kern0pt}\ {\isachardoublequoteopen}{\isacharparenleft}{\kern0pt}nat\ {\isasymtimes}\ nat\ {\isasymRightarrow}\ complex{\isacharparenright}{\kern0pt}\ {\isasymRightarrow}\ {\isacharparenleft}{\kern0pt}nat\ {\isasymtimes}\ nat\ {\isasymRightarrow}\ complex{\isacharparenright}{\kern0pt}\ {\isasymRightarrow}\ nat\ {\isasymRightarrow}\ nat\ {\isasymRightarrow}\ nat{\isachardoublequoteclose}\ \isakeyword{and}\ nna\ {\isacharcolon}{\kern0pt}{\isacharcolon}{\kern0pt}\ {\isachardoublequoteopen}{\isacharparenleft}{\kern0pt}nat\ {\isasymtimes}\ nat\ {\isasymRightarrow}\ complex{\isacharparenright}{\kern0pt}\ {\isasymRightarrow}\ {\isacharparenleft}{\kern0pt}nat\ {\isasymtimes}\ nat\ {\isasymRightarrow}\ complex{\isacharparenright}{\kern0pt}\ {\isasymRightarrow}\ nat\ {\isasymRightarrow}\ nat\ {\isasymRightarrow}\ nat{\isachardoublequoteclose}\ \isakeyword{where}\isanewline
\ \ \ \ \ \ {\isachardoublequoteopen}{\isasymforall}x{\isadigit{0}}\ x{\isadigit{1}}\ x{\isadigit{3}}\ x{\isadigit{5}}{\isachardot}{\kern0pt}\ {\isacharparenleft}{\kern0pt}{\isasymexists}v{\isadigit{6}}\ v{\isadigit{7}}{\isachardot}{\kern0pt}\ {\isacharparenleft}{\kern0pt}v{\isadigit{6}}\ {\isacharless}{\kern0pt}\ x{\isadigit{5}}\ {\isasymand}\ v{\isadigit{7}}\ {\isacharless}{\kern0pt}\ x{\isadigit{3}}{\isacharparenright}{\kern0pt}\ {\isasymand}\ x{\isadigit{1}}\ {\isacharparenleft}{\kern0pt}v{\isadigit{6}}{\isacharcomma}{\kern0pt}\ v{\isadigit{7}}{\isacharparenright}{\kern0pt}\ {\isasymnoteq}\ x{\isadigit{0}}\ {\isacharparenleft}{\kern0pt}v{\isadigit{6}}{\isacharcomma}{\kern0pt}\ v{\isadigit{7}}{\isacharparenright}{\kern0pt}{\isacharparenright}{\kern0pt}\ {\isacharequal}{\kern0pt}\ {\isacharparenleft}{\kern0pt}{\isacharparenleft}{\kern0pt}nn\ x{\isadigit{0}}\ x{\isadigit{1}}\ x{\isadigit{3}}\ x{\isadigit{5}}\ {\isacharless}{\kern0pt}\ x{\isadigit{5}}\ {\isasymand}\ nna\ x{\isadigit{0}}\ x{\isadigit{1}}\ x{\isadigit{3}}\ x{\isadigit{5}}\ {\isacharless}{\kern0pt}\ x{\isadigit{3}}{\isacharparenright}{\kern0pt}\ {\isasymand}\ x{\isadigit{1}}\ {\isacharparenleft}{\kern0pt}nn\ x{\isadigit{0}}\ x{\isadigit{1}}\ x{\isadigit{3}}\ x{\isadigit{5}}{\isacharcomma}{\kern0pt}\ nna\ x{\isadigit{0}}\ x{\isadigit{1}}\ x{\isadigit{3}}\ x{\isadigit{5}}{\isacharparenright}{\kern0pt}\ {\isasymnoteq}\ x{\isadigit{0}}\ {\isacharparenleft}{\kern0pt}nn\ x{\isadigit{0}}\ x{\isadigit{1}}\ x{\isadigit{3}}\ x{\isadigit{5}}{\isacharcomma}{\kern0pt}\ nna\ x{\isadigit{0}}\ x{\isadigit{1}}\ x{\isadigit{3}}\ x{\isadigit{5}}{\isacharparenright}{\kern0pt}{\isacharparenright}{\kern0pt}{\isachardoublequoteclose}\isanewline
\ \ \ \ \ \ \isacommand{by}\isamarkupfalse%
\ moura\isanewline
\ \ \ \ \isacommand{then}\isamarkupfalse%
\ \isacommand{have}\isamarkupfalse%
\ {\isachardoublequoteopen}{\isasymforall}n\ na\ nb\ nc\ f\ fa{\isachardot}{\kern0pt}\ {\isacharparenleft}{\kern0pt}n\ {\isasymnoteq}\ na\ {\isasymor}\ nb\ {\isasymnoteq}\ nc\ {\isasymor}\ {\isacharparenleft}{\kern0pt}nn\ fa\ f\ nb\ n\ {\isacharless}{\kern0pt}\ n\ {\isasymand}\ nna\ fa\ f\ nb\ n\ {\isacharless}{\kern0pt}\ nb{\isacharparenright}{\kern0pt}\ {\isasymand}\ f\ {\isacharparenleft}{\kern0pt}nn\ fa\ f\ nb\ n{\isacharcomma}{\kern0pt}\ nna\ fa\ f\ nb\ n{\isacharparenright}{\kern0pt}\ {\isasymnoteq}\ fa\ {\isacharparenleft}{\kern0pt}nn\ fa\ f\ nb\ n{\isacharcomma}{\kern0pt}\ nna\ fa\ f\ nb\ n{\isacharparenright}{\kern0pt}{\isacharparenright}{\kern0pt}\ {\isasymor}\ Matrix{\isachardot}{\kern0pt}mat\ n\ nb\ f\ {\isacharequal}{\kern0pt}\ Matrix{\isachardot}{\kern0pt}mat\ na\ nc\ fa{\isachardoublequoteclose}\isanewline
\ \ \ \ \ \ \isacommand{by}\isamarkupfalse%
\ {\isacharparenleft}{\kern0pt}meson\ cong{\isacharunderscore}{\kern0pt}mat{\isacharparenright}{\kern0pt}\isanewline
\ \ \ \ \isacommand{moreover}\isamarkupfalse%
\isanewline
\ \ \ \ \isacommand{{\isacharbraceleft}{\kern0pt}}\isamarkupfalse%
\ \isacommand{assume}\isamarkupfalse%
\ {\isachardoublequoteopen}nn\ {\isacharparenleft}{\kern0pt}{\isasymlambda}{\isacharparenleft}{\kern0pt}na{\isacharcomma}{\kern0pt}\ n{\isacharparenright}{\kern0pt}{\isachardot}{\kern0pt}\ cnj\ {\isacharparenleft}{\kern0pt}SWAP\ {\isachardollar}{\kern0pt}{\isachardollar}{\kern0pt}\ {\isacharparenleft}{\kern0pt}n{\isacharcomma}{\kern0pt}\ na{\isacharparenright}{\kern0pt}{\isacharparenright}{\kern0pt}{\isacharparenright}{\kern0pt}\ {\isacharparenleft}{\kern0pt}{\isasymlambda}{\isacharparenleft}{\kern0pt}na{\isacharcomma}{\kern0pt}\ n{\isacharparenright}{\kern0pt}{\isachardot}{\kern0pt}\ cnj\ {\isacharparenleft}{\kern0pt}SWAP\ {\isachardollar}{\kern0pt}{\isachardollar}{\kern0pt}\ {\isacharparenleft}{\kern0pt}na{\isacharcomma}{\kern0pt}\ n{\isacharparenright}{\kern0pt}{\isacharparenright}{\kern0pt}{\isacharparenright}{\kern0pt}\ {\isadigit{4}}\ {\isadigit{4}}\ {\isasymnoteq}\ {\isadigit{3}}\ {\isasymor}\ nna\ {\isacharparenleft}{\kern0pt}{\isasymlambda}{\isacharparenleft}{\kern0pt}na{\isacharcomma}{\kern0pt}\ n{\isacharparenright}{\kern0pt}{\isachardot}{\kern0pt}\ cnj\ {\isacharparenleft}{\kern0pt}SWAP\ {\isachardollar}{\kern0pt}{\isachardollar}{\kern0pt}\ {\isacharparenleft}{\kern0pt}n{\isacharcomma}{\kern0pt}\ na{\isacharparenright}{\kern0pt}{\isacharparenright}{\kern0pt}{\isacharparenright}{\kern0pt}\ {\isacharparenleft}{\kern0pt}{\isasymlambda}{\isacharparenleft}{\kern0pt}na{\isacharcomma}{\kern0pt}\ n{\isacharparenright}{\kern0pt}{\isachardot}{\kern0pt}\ cnj\ {\isacharparenleft}{\kern0pt}SWAP\ {\isachardollar}{\kern0pt}{\isachardollar}{\kern0pt}\ {\isacharparenleft}{\kern0pt}na{\isacharcomma}{\kern0pt}\ n{\isacharparenright}{\kern0pt}{\isacharparenright}{\kern0pt}{\isacharparenright}{\kern0pt}\ {\isadigit{4}}\ {\isadigit{4}}\ {\isasymnoteq}\ {\isadigit{3}}{\isachardoublequoteclose}\isanewline
\ \ \ \ \ \ \isacommand{then}\isamarkupfalse%
\ \isacommand{have}\isamarkupfalse%
\ {\isachardoublequoteopen}{\isacharparenleft}{\kern0pt}if\ nn\ {\isacharparenleft}{\kern0pt}{\isasymlambda}{\isacharparenleft}{\kern0pt}na{\isacharcomma}{\kern0pt}\ n{\isacharparenright}{\kern0pt}{\isachardot}{\kern0pt}\ cnj\ {\isacharparenleft}{\kern0pt}SWAP\ {\isachardollar}{\kern0pt}{\isachardollar}{\kern0pt}\ {\isacharparenleft}{\kern0pt}n{\isacharcomma}{\kern0pt}\ na{\isacharparenright}{\kern0pt}{\isacharparenright}{\kern0pt}{\isacharparenright}{\kern0pt}\ {\isacharparenleft}{\kern0pt}{\isasymlambda}{\isacharparenleft}{\kern0pt}na{\isacharcomma}{\kern0pt}\ n{\isacharparenright}{\kern0pt}{\isachardot}{\kern0pt}\ cnj\ {\isacharparenleft}{\kern0pt}SWAP\ {\isachardollar}{\kern0pt}{\isachardollar}{\kern0pt}\ {\isacharparenleft}{\kern0pt}na{\isacharcomma}{\kern0pt}\ n{\isacharparenright}{\kern0pt}{\isacharparenright}{\kern0pt}{\isacharparenright}{\kern0pt}\ {\isadigit{4}}\ {\isadigit{4}}\ {\isasymnoteq}\ {\isadigit{2}}\ {\isasymor}\ nna\ {\isacharparenleft}{\kern0pt}{\isasymlambda}{\isacharparenleft}{\kern0pt}na{\isacharcomma}{\kern0pt}\ n{\isacharparenright}{\kern0pt}{\isachardot}{\kern0pt}\ cnj\ {\isacharparenleft}{\kern0pt}SWAP\ {\isachardollar}{\kern0pt}{\isachardollar}{\kern0pt}\ {\isacharparenleft}{\kern0pt}n{\isacharcomma}{\kern0pt}\ na{\isacharparenright}{\kern0pt}{\isacharparenright}{\kern0pt}{\isacharparenright}{\kern0pt}\ {\isacharparenleft}{\kern0pt}{\isasymlambda}{\isacharparenleft}{\kern0pt}na{\isacharcomma}{\kern0pt}\ n{\isacharparenright}{\kern0pt}{\isachardot}{\kern0pt}\ cnj\ {\isacharparenleft}{\kern0pt}SWAP\ {\isachardollar}{\kern0pt}{\isachardollar}{\kern0pt}\ {\isacharparenleft}{\kern0pt}na{\isacharcomma}{\kern0pt}\ n{\isacharparenright}{\kern0pt}{\isacharparenright}{\kern0pt}{\isacharparenright}{\kern0pt}\ {\isadigit{4}}\ {\isadigit{4}}\ {\isasymnoteq}\ {\isadigit{1}}\ then\ if\ nn\ {\isacharparenleft}{\kern0pt}{\isasymlambda}{\isacharparenleft}{\kern0pt}na{\isacharcomma}{\kern0pt}\ n{\isacharparenright}{\kern0pt}{\isachardot}{\kern0pt}\ cnj\ {\isacharparenleft}{\kern0pt}SWAP\ {\isachardollar}{\kern0pt}{\isachardollar}{\kern0pt}\ {\isacharparenleft}{\kern0pt}n{\isacharcomma}{\kern0pt}\ na{\isacharparenright}{\kern0pt}{\isacharparenright}{\kern0pt}{\isacharparenright}{\kern0pt}\ {\isacharparenleft}{\kern0pt}{\isasymlambda}{\isacharparenleft}{\kern0pt}na{\isacharcomma}{\kern0pt}\ n{\isacharparenright}{\kern0pt}{\isachardot}{\kern0pt}\ cnj\ {\isacharparenleft}{\kern0pt}SWAP\ {\isachardollar}{\kern0pt}{\isachardollar}{\kern0pt}\ {\isacharparenleft}{\kern0pt}na{\isacharcomma}{\kern0pt}\ n{\isacharparenright}{\kern0pt}{\isacharparenright}{\kern0pt}{\isacharparenright}{\kern0pt}\ {\isadigit{4}}\ {\isadigit{4}}\ {\isasymnoteq}\ {\isadigit{3}}\ {\isasymor}\ nna\ {\isacharparenleft}{\kern0pt}{\isasymlambda}{\isacharparenleft}{\kern0pt}na{\isacharcomma}{\kern0pt}\ n{\isacharparenright}{\kern0pt}{\isachardot}{\kern0pt}\ cnj\ {\isacharparenleft}{\kern0pt}SWAP\ {\isachardollar}{\kern0pt}{\isachardollar}{\kern0pt}\ {\isacharparenleft}{\kern0pt}n{\isacharcomma}{\kern0pt}\ na{\isacharparenright}{\kern0pt}{\isacharparenright}{\kern0pt}{\isacharparenright}{\kern0pt}\ {\isacharparenleft}{\kern0pt}{\isasymlambda}{\isacharparenleft}{\kern0pt}na{\isacharcomma}{\kern0pt}\ n{\isacharparenright}{\kern0pt}{\isachardot}{\kern0pt}\ cnj\ {\isacharparenleft}{\kern0pt}SWAP\ {\isachardollar}{\kern0pt}{\isachardollar}{\kern0pt}\ {\isacharparenleft}{\kern0pt}na{\isacharcomma}{\kern0pt}\ n{\isacharparenright}{\kern0pt}{\isacharparenright}{\kern0pt}{\isacharparenright}{\kern0pt}\ {\isadigit{4}}\ {\isadigit{4}}\ {\isasymnoteq}\ {\isadigit{3}}\ then\ {\isacharparenleft}{\kern0pt}if\ nn\ {\isacharparenleft}{\kern0pt}{\isasymlambda}{\isacharparenleft}{\kern0pt}na{\isacharcomma}{\kern0pt}\ n{\isacharparenright}{\kern0pt}{\isachardot}{\kern0pt}\ cnj\ {\isacharparenleft}{\kern0pt}SWAP\ {\isachardollar}{\kern0pt}{\isachardollar}{\kern0pt}\ {\isacharparenleft}{\kern0pt}n{\isacharcomma}{\kern0pt}\ na{\isacharparenright}{\kern0pt}{\isacharparenright}{\kern0pt}{\isacharparenright}{\kern0pt}\ {\isacharparenleft}{\kern0pt}{\isasymlambda}{\isacharparenleft}{\kern0pt}na{\isacharcomma}{\kern0pt}\ n{\isacharparenright}{\kern0pt}{\isachardot}{\kern0pt}\ cnj\ {\isacharparenleft}{\kern0pt}SWAP\ {\isachardollar}{\kern0pt}{\isachardollar}{\kern0pt}\ {\isacharparenleft}{\kern0pt}na{\isacharcomma}{\kern0pt}\ n{\isacharparenright}{\kern0pt}{\isacharparenright}{\kern0pt}{\isacharparenright}{\kern0pt}\ {\isadigit{4}}\ {\isadigit{4}}\ {\isacharequal}{\kern0pt}\ {\isadigit{0}}\ {\isasymand}\ nna\ {\isacharparenleft}{\kern0pt}{\isasymlambda}{\isacharparenleft}{\kern0pt}na{\isacharcomma}{\kern0pt}\ n{\isacharparenright}{\kern0pt}{\isachardot}{\kern0pt}\ cnj\ {\isacharparenleft}{\kern0pt}SWAP\ {\isachardollar}{\kern0pt}{\isachardollar}{\kern0pt}\ {\isacharparenleft}{\kern0pt}n{\isacharcomma}{\kern0pt}\ na{\isacharparenright}{\kern0pt}{\isacharparenright}{\kern0pt}{\isacharparenright}{\kern0pt}\ {\isacharparenleft}{\kern0pt}{\isasymlambda}{\isacharparenleft}{\kern0pt}na{\isacharcomma}{\kern0pt}\ n{\isacharparenright}{\kern0pt}{\isachardot}{\kern0pt}\ cnj\ {\isacharparenleft}{\kern0pt}SWAP\ {\isachardollar}{\kern0pt}{\isachardollar}{\kern0pt}\ {\isacharparenleft}{\kern0pt}na{\isacharcomma}{\kern0pt}\ n{\isacharparenright}{\kern0pt}{\isacharparenright}{\kern0pt}{\isacharparenright}{\kern0pt}\ {\isadigit{4}}\ {\isadigit{4}}\ {\isacharequal}{\kern0pt}\ {\isadigit{0}}\ then\ {\isadigit{1}}{\isacharcolon}{\kern0pt}{\isacharcolon}{\kern0pt}complex\ else\ if\ nn\ {\isacharparenleft}{\kern0pt}{\isasymlambda}{\isacharparenleft}{\kern0pt}na{\isacharcomma}{\kern0pt}\ n{\isacharparenright}{\kern0pt}{\isachardot}{\kern0pt}\ cnj\ {\isacharparenleft}{\kern0pt}SWAP\ {\isachardollar}{\kern0pt}{\isachardollar}{\kern0pt}\ {\isacharparenleft}{\kern0pt}n{\isacharcomma}{\kern0pt}\ na{\isacharparenright}{\kern0pt}{\isacharparenright}{\kern0pt}{\isacharparenright}{\kern0pt}\ {\isacharparenleft}{\kern0pt}{\isasymlambda}{\isacharparenleft}{\kern0pt}na{\isacharcomma}{\kern0pt}\ n{\isacharparenright}{\kern0pt}{\isachardot}{\kern0pt}\ cnj\ {\isacharparenleft}{\kern0pt}SWAP\ {\isachardollar}{\kern0pt}{\isachardollar}{\kern0pt}\ {\isacharparenleft}{\kern0pt}na{\isacharcomma}{\kern0pt}\ n{\isacharparenright}{\kern0pt}{\isacharparenright}{\kern0pt}{\isacharparenright}{\kern0pt}\ {\isadigit{4}}\ {\isadigit{4}}\ {\isacharequal}{\kern0pt}\ {\isadigit{1}}\ {\isasymand}\ nna\ {\isacharparenleft}{\kern0pt}{\isasymlambda}{\isacharparenleft}{\kern0pt}na{\isacharcomma}{\kern0pt}\ n{\isacharparenright}{\kern0pt}{\isachardot}{\kern0pt}\ cnj\ {\isacharparenleft}{\kern0pt}SWAP\ {\isachardollar}{\kern0pt}{\isachardollar}{\kern0pt}\ {\isacharparenleft}{\kern0pt}n{\isacharcomma}{\kern0pt}\ na{\isacharparenright}{\kern0pt}{\isacharparenright}{\kern0pt}{\isacharparenright}{\kern0pt}\ {\isacharparenleft}{\kern0pt}{\isasymlambda}{\isacharparenleft}{\kern0pt}na{\isacharcomma}{\kern0pt}\ n{\isacharparenright}{\kern0pt}{\isachardot}{\kern0pt}\ cnj\ {\isacharparenleft}{\kern0pt}SWAP\ {\isachardollar}{\kern0pt}{\isachardollar}{\kern0pt}\ {\isacharparenleft}{\kern0pt}na{\isacharcomma}{\kern0pt}\ n{\isacharparenright}{\kern0pt}{\isacharparenright}{\kern0pt}{\isacharparenright}{\kern0pt}\ {\isadigit{4}}\ {\isadigit{4}}\ {\isacharequal}{\kern0pt}\ {\isadigit{2}}\ then\ {\isadigit{1}}\ else\ if\ nn\ {\isacharparenleft}{\kern0pt}{\isasymlambda}{\isacharparenleft}{\kern0pt}na{\isacharcomma}{\kern0pt}\ n{\isacharparenright}{\kern0pt}{\isachardot}{\kern0pt}\ cnj\ {\isacharparenleft}{\kern0pt}SWAP\ {\isachardollar}{\kern0pt}{\isachardollar}{\kern0pt}\ {\isacharparenleft}{\kern0pt}n{\isacharcomma}{\kern0pt}\ na{\isacharparenright}{\kern0pt}{\isacharparenright}{\kern0pt}{\isacharparenright}{\kern0pt}\ {\isacharparenleft}{\kern0pt}{\isasymlambda}{\isacharparenleft}{\kern0pt}na{\isacharcomma}{\kern0pt}\ n{\isacharparenright}{\kern0pt}{\isachardot}{\kern0pt}\ cnj\ {\isacharparenleft}{\kern0pt}SWAP\ {\isachardollar}{\kern0pt}{\isachardollar}{\kern0pt}\ {\isacharparenleft}{\kern0pt}na{\isacharcomma}{\kern0pt}\ n{\isacharparenright}{\kern0pt}{\isacharparenright}{\kern0pt}{\isacharparenright}{\kern0pt}\ {\isadigit{4}}\ {\isadigit{4}}\ {\isacharequal}{\kern0pt}\ {\isadigit{2}}\ {\isasymand}\ nna\ {\isacharparenleft}{\kern0pt}{\isasymlambda}{\isacharparenleft}{\kern0pt}na{\isacharcomma}{\kern0pt}\ n{\isacharparenright}{\kern0pt}{\isachardot}{\kern0pt}\ cnj\ {\isacharparenleft}{\kern0pt}SWAP\ {\isachardollar}{\kern0pt}{\isachardollar}{\kern0pt}\ {\isacharparenleft}{\kern0pt}n{\isacharcomma}{\kern0pt}\ na{\isacharparenright}{\kern0pt}{\isacharparenright}{\kern0pt}{\isacharparenright}{\kern0pt}\ {\isacharparenleft}{\kern0pt}{\isasymlambda}{\isacharparenleft}{\kern0pt}na{\isacharcomma}{\kern0pt}\ n{\isacharparenright}{\kern0pt}{\isachardot}{\kern0pt}\ cnj\ {\isacharparenleft}{\kern0pt}SWAP\ {\isachardollar}{\kern0pt}{\isachardollar}{\kern0pt}\ {\isacharparenleft}{\kern0pt}na{\isacharcomma}{\kern0pt}\ n{\isacharparenright}{\kern0pt}{\isacharparenright}{\kern0pt}{\isacharparenright}{\kern0pt}\ {\isadigit{4}}\ {\isadigit{4}}\ {\isacharequal}{\kern0pt}\ {\isadigit{1}}\ then\ {\isadigit{1}}\ else\ if\ nn\ {\isacharparenleft}{\kern0pt}{\isasymlambda}{\isacharparenleft}{\kern0pt}na{\isacharcomma}{\kern0pt}\ n{\isacharparenright}{\kern0pt}{\isachardot}{\kern0pt}\ cnj\ {\isacharparenleft}{\kern0pt}SWAP\ {\isachardollar}{\kern0pt}{\isachardollar}{\kern0pt}\ {\isacharparenleft}{\kern0pt}n{\isacharcomma}{\kern0pt}\ na{\isacharparenright}{\kern0pt}{\isacharparenright}{\kern0pt}{\isacharparenright}{\kern0pt}\ {\isacharparenleft}{\kern0pt}{\isasymlambda}{\isacharparenleft}{\kern0pt}na{\isacharcomma}{\kern0pt}\ n{\isacharparenright}{\kern0pt}{\isachardot}{\kern0pt}\ cnj\ {\isacharparenleft}{\kern0pt}SWAP\ {\isachardollar}{\kern0pt}{\isachardollar}{\kern0pt}\ {\isacharparenleft}{\kern0pt}na{\isacharcomma}{\kern0pt}\ n{\isacharparenright}{\kern0pt}{\isacharparenright}{\kern0pt}{\isacharparenright}{\kern0pt}\ {\isadigit{4}}\ {\isadigit{4}}\ {\isacharequal}{\kern0pt}\ {\isadigit{3}}\ {\isasymand}\ nna\ {\isacharparenleft}{\kern0pt}{\isasymlambda}{\isacharparenleft}{\kern0pt}na{\isacharcomma}{\kern0pt}\ n{\isacharparenright}{\kern0pt}{\isachardot}{\kern0pt}\ cnj\ {\isacharparenleft}{\kern0pt}SWAP\ {\isachardollar}{\kern0pt}{\isachardollar}{\kern0pt}\ {\isacharparenleft}{\kern0pt}n{\isacharcomma}{\kern0pt}\ na{\isacharparenright}{\kern0pt}{\isacharparenright}{\kern0pt}{\isacharparenright}{\kern0pt}\ {\isacharparenleft}{\kern0pt}{\isasymlambda}{\isacharparenleft}{\kern0pt}na{\isacharcomma}{\kern0pt}\ n{\isacharparenright}{\kern0pt}{\isachardot}{\kern0pt}\ cnj\ {\isacharparenleft}{\kern0pt}SWAP\ {\isachardollar}{\kern0pt}{\isachardollar}{\kern0pt}\ {\isacharparenleft}{\kern0pt}na{\isacharcomma}{\kern0pt}\ n{\isacharparenright}{\kern0pt}{\isacharparenright}{\kern0pt}{\isacharparenright}{\kern0pt}\ {\isadigit{4}}\ {\isadigit{4}}\ {\isacharequal}{\kern0pt}\ {\isadigit{3}}\ then\ {\isadigit{1}}\ else\ {\isadigit{0}}{\isacharparenright}{\kern0pt}\ {\isacharequal}{\kern0pt}\ {\isadigit{0}}\ else\ {\isacharparenleft}{\kern0pt}if\ nn\ {\isacharparenleft}{\kern0pt}{\isasymlambda}{\isacharparenleft}{\kern0pt}na{\isacharcomma}{\kern0pt}\ n{\isacharparenright}{\kern0pt}{\isachardot}{\kern0pt}\ cnj\ {\isacharparenleft}{\kern0pt}SWAP\ {\isachardollar}{\kern0pt}{\isachardollar}{\kern0pt}\ {\isacharparenleft}{\kern0pt}n{\isacharcomma}{\kern0pt}\ na{\isacharparenright}{\kern0pt}{\isacharparenright}{\kern0pt}{\isacharparenright}{\kern0pt}\ {\isacharparenleft}{\kern0pt}{\isasymlambda}{\isacharparenleft}{\kern0pt}na{\isacharcomma}{\kern0pt}\ n{\isacharparenright}{\kern0pt}{\isachardot}{\kern0pt}\ cnj\ {\isacharparenleft}{\kern0pt}SWAP\ {\isachardollar}{\kern0pt}{\isachardollar}{\kern0pt}\ {\isacharparenleft}{\kern0pt}na{\isacharcomma}{\kern0pt}\ n{\isacharparenright}{\kern0pt}{\isacharparenright}{\kern0pt}{\isacharparenright}{\kern0pt}\ {\isadigit{4}}\ {\isadigit{4}}\ {\isacharequal}{\kern0pt}\ {\isadigit{0}}\ {\isasymand}\ nna\ {\isacharparenleft}{\kern0pt}{\isasymlambda}{\isacharparenleft}{\kern0pt}na{\isacharcomma}{\kern0pt}\ n{\isacharparenright}{\kern0pt}{\isachardot}{\kern0pt}\ cnj\ {\isacharparenleft}{\kern0pt}SWAP\ {\isachardollar}{\kern0pt}{\isachardollar}{\kern0pt}\ {\isacharparenleft}{\kern0pt}n{\isacharcomma}{\kern0pt}\ na{\isacharparenright}{\kern0pt}{\isacharparenright}{\kern0pt}{\isacharparenright}{\kern0pt}\ {\isacharparenleft}{\kern0pt}{\isasymlambda}{\isacharparenleft}{\kern0pt}na{\isacharcomma}{\kern0pt}\ n{\isacharparenright}{\kern0pt}{\isachardot}{\kern0pt}\ cnj\ {\isacharparenleft}{\kern0pt}SWAP\ {\isachardollar}{\kern0pt}{\isachardollar}{\kern0pt}\ {\isacharparenleft}{\kern0pt}na{\isacharcomma}{\kern0pt}\ n{\isacharparenright}{\kern0pt}{\isacharparenright}{\kern0pt}{\isacharparenright}{\kern0pt}\ {\isadigit{4}}\ {\isadigit{4}}\ {\isacharequal}{\kern0pt}\ {\isadigit{0}}\ then\ {\isadigit{1}}{\isacharcolon}{\kern0pt}{\isacharcolon}{\kern0pt}complex\ else\ if\ nn\ {\isacharparenleft}{\kern0pt}{\isasymlambda}{\isacharparenleft}{\kern0pt}na{\isacharcomma}{\kern0pt}\ n{\isacharparenright}{\kern0pt}{\isachardot}{\kern0pt}\ cnj\ {\isacharparenleft}{\kern0pt}SWAP\ {\isachardollar}{\kern0pt}{\isachardollar}{\kern0pt}\ {\isacharparenleft}{\kern0pt}n{\isacharcomma}{\kern0pt}\ na{\isacharparenright}{\kern0pt}{\isacharparenright}{\kern0pt}{\isacharparenright}{\kern0pt}\ {\isacharparenleft}{\kern0pt}{\isasymlambda}{\isacharparenleft}{\kern0pt}na{\isacharcomma}{\kern0pt}\ n{\isacharparenright}{\kern0pt}{\isachardot}{\kern0pt}\ cnj\ {\isacharparenleft}{\kern0pt}SWAP\ {\isachardollar}{\kern0pt}{\isachardollar}{\kern0pt}\ {\isacharparenleft}{\kern0pt}na{\isacharcomma}{\kern0pt}\ n{\isacharparenright}{\kern0pt}{\isacharparenright}{\kern0pt}{\isacharparenright}{\kern0pt}\ {\isadigit{4}}\ {\isadigit{4}}\ {\isacharequal}{\kern0pt}\ {\isadigit{1}}\ {\isasymand}\ nna\ {\isacharparenleft}{\kern0pt}{\isasymlambda}{\isacharparenleft}{\kern0pt}na{\isacharcomma}{\kern0pt}\ n{\isacharparenright}{\kern0pt}{\isachardot}{\kern0pt}\ cnj\ {\isacharparenleft}{\kern0pt}SWAP\ {\isachardollar}{\kern0pt}{\isachardollar}{\kern0pt}\ {\isacharparenleft}{\kern0pt}n{\isacharcomma}{\kern0pt}\ na{\isacharparenright}{\kern0pt}{\isacharparenright}{\kern0pt}{\isacharparenright}{\kern0pt}\ {\isacharparenleft}{\kern0pt}{\isasymlambda}{\isacharparenleft}{\kern0pt}na{\isacharcomma}{\kern0pt}\ n{\isacharparenright}{\kern0pt}{\isachardot}{\kern0pt}\ cnj\ {\isacharparenleft}{\kern0pt}SWAP\ {\isachardollar}{\kern0pt}{\isachardollar}{\kern0pt}\ {\isacharparenleft}{\kern0pt}na{\isacharcomma}{\kern0pt}\ n{\isacharparenright}{\kern0pt}{\isacharparenright}{\kern0pt}{\isacharparenright}{\kern0pt}\ {\isadigit{4}}\ {\isadigit{4}}\ {\isacharequal}{\kern0pt}\ {\isadigit{2}}\ then\ {\isadigit{1}}\ else\ if\ nn\ {\isacharparenleft}{\kern0pt}{\isasymlambda}{\isacharparenleft}{\kern0pt}na{\isacharcomma}{\kern0pt}\ n{\isacharparenright}{\kern0pt}{\isachardot}{\kern0pt}\ cnj\ {\isacharparenleft}{\kern0pt}SWAP\ {\isachardollar}{\kern0pt}{\isachardollar}{\kern0pt}\ {\isacharparenleft}{\kern0pt}n{\isacharcomma}{\kern0pt}\ na{\isacharparenright}{\kern0pt}{\isacharparenright}{\kern0pt}{\isacharparenright}{\kern0pt}\ {\isacharparenleft}{\kern0pt}{\isasymlambda}{\isacharparenleft}{\kern0pt}na{\isacharcomma}{\kern0pt}\ n{\isacharparenright}{\kern0pt}{\isachardot}{\kern0pt}\ cnj\ {\isacharparenleft}{\kern0pt}SWAP\ {\isachardollar}{\kern0pt}{\isachardollar}{\kern0pt}\ {\isacharparenleft}{\kern0pt}na{\isacharcomma}{\kern0pt}\ n{\isacharparenright}{\kern0pt}{\isacharparenright}{\kern0pt}{\isacharparenright}{\kern0pt}\ {\isadigit{4}}\ {\isadigit{4}}\ {\isacharequal}{\kern0pt}\ {\isadigit{2}}\ {\isasymand}\ nna\ {\isacharparenleft}{\kern0pt}{\isasymlambda}{\isacharparenleft}{\kern0pt}na{\isacharcomma}{\kern0pt}\ n{\isacharparenright}{\kern0pt}{\isachardot}{\kern0pt}\ cnj\ {\isacharparenleft}{\kern0pt}SWAP\ {\isachardollar}{\kern0pt}{\isachardollar}{\kern0pt}\ {\isacharparenleft}{\kern0pt}n{\isacharcomma}{\kern0pt}\ na{\isacharparenright}{\kern0pt}{\isacharparenright}{\kern0pt}{\isacharparenright}{\kern0pt}\ {\isacharparenleft}{\kern0pt}{\isasymlambda}{\isacharparenleft}{\kern0pt}na{\isacharcomma}{\kern0pt}\ n{\isacharparenright}{\kern0pt}{\isachardot}{\kern0pt}\ cnj\ {\isacharparenleft}{\kern0pt}SWAP\ {\isachardollar}{\kern0pt}{\isachardollar}{\kern0pt}\ {\isacharparenleft}{\kern0pt}na{\isacharcomma}{\kern0pt}\ n{\isacharparenright}{\kern0pt}{\isacharparenright}{\kern0pt}{\isacharparenright}{\kern0pt}\ {\isadigit{4}}\ {\isadigit{4}}\ {\isacharequal}{\kern0pt}\ {\isadigit{1}}\ then\ {\isadigit{1}}\ else\ if\ nn\ {\isacharparenleft}{\kern0pt}{\isasymlambda}{\isacharparenleft}{\kern0pt}na{\isacharcomma}{\kern0pt}\ n{\isacharparenright}{\kern0pt}{\isachardot}{\kern0pt}\ cnj\ {\isacharparenleft}{\kern0pt}SWAP\ {\isachardollar}{\kern0pt}{\isachardollar}{\kern0pt}\ {\isacharparenleft}{\kern0pt}n{\isacharcomma}{\kern0pt}\ na{\isacharparenright}{\kern0pt}{\isacharparenright}{\kern0pt}{\isacharparenright}{\kern0pt}\ {\isacharparenleft}{\kern0pt}{\isasymlambda}{\isacharparenleft}{\kern0pt}na{\isacharcomma}{\kern0pt}\ n{\isacharparenright}{\kern0pt}{\isachardot}{\kern0pt}\ cnj\ {\isacharparenleft}{\kern0pt}SWAP\ {\isachardollar}{\kern0pt}{\isachardollar}{\kern0pt}\ {\isacharparenleft}{\kern0pt}na{\isacharcomma}{\kern0pt}\ n{\isacharparenright}{\kern0pt}{\isacharparenright}{\kern0pt}{\isacharparenright}{\kern0pt}\ {\isadigit{4}}\ {\isadigit{4}}\ {\isacharequal}{\kern0pt}\ {\isadigit{3}}\ {\isasymand}\ nna\ {\isacharparenleft}{\kern0pt}{\isasymlambda}{\isacharparenleft}{\kern0pt}na{\isacharcomma}{\kern0pt}\ n{\isacharparenright}{\kern0pt}{\isachardot}{\kern0pt}\ cnj\ {\isacharparenleft}{\kern0pt}SWAP\ {\isachardollar}{\kern0pt}{\isachardollar}{\kern0pt}\ {\isacharparenleft}{\kern0pt}n{\isacharcomma}{\kern0pt}\ na{\isacharparenright}{\kern0pt}{\isacharparenright}{\kern0pt}{\isacharparenright}{\kern0pt}\ {\isacharparenleft}{\kern0pt}{\isasymlambda}{\isacharparenleft}{\kern0pt}na{\isacharcomma}{\kern0pt}\ n{\isacharparenright}{\kern0pt}{\isachardot}{\kern0pt}\ cnj\ {\isacharparenleft}{\kern0pt}SWAP\ {\isachardollar}{\kern0pt}{\isachardollar}{\kern0pt}\ {\isacharparenleft}{\kern0pt}na{\isacharcomma}{\kern0pt}\ n{\isacharparenright}{\kern0pt}{\isacharparenright}{\kern0pt}{\isacharparenright}{\kern0pt}\ {\isadigit{4}}\ {\isadigit{4}}\ {\isacharequal}{\kern0pt}\ {\isadigit{3}}\ then\ {\isadigit{1}}\ else\ {\isadigit{0}}{\isacharparenright}{\kern0pt}\ {\isacharequal}{\kern0pt}\ {\isadigit{1}}\ else\ {\isacharparenleft}{\kern0pt}if\ nn\ {\isacharparenleft}{\kern0pt}{\isasymlambda}{\isacharparenleft}{\kern0pt}na{\isacharcomma}{\kern0pt}\ n{\isacharparenright}{\kern0pt}{\isachardot}{\kern0pt}\ cnj\ {\isacharparenleft}{\kern0pt}SWAP\ {\isachardollar}{\kern0pt}{\isachardollar}{\kern0pt}\ {\isacharparenleft}{\kern0pt}n{\isacharcomma}{\kern0pt}\ na{\isacharparenright}{\kern0pt}{\isacharparenright}{\kern0pt}{\isacharparenright}{\kern0pt}\ {\isacharparenleft}{\kern0pt}{\isasymlambda}{\isacharparenleft}{\kern0pt}na{\isacharcomma}{\kern0pt}\ n{\isacharparenright}{\kern0pt}{\isachardot}{\kern0pt}\ cnj\ {\isacharparenleft}{\kern0pt}SWAP\ {\isachardollar}{\kern0pt}{\isachardollar}{\kern0pt}\ {\isacharparenleft}{\kern0pt}na{\isacharcomma}{\kern0pt}\ n{\isacharparenright}{\kern0pt}{\isacharparenright}{\kern0pt}{\isacharparenright}{\kern0pt}\ {\isadigit{4}}\ {\isadigit{4}}\ {\isacharequal}{\kern0pt}\ {\isadigit{0}}\ {\isasymand}\ nna\ {\isacharparenleft}{\kern0pt}{\isasymlambda}{\isacharparenleft}{\kern0pt}na{\isacharcomma}{\kern0pt}\ n{\isacharparenright}{\kern0pt}{\isachardot}{\kern0pt}\ cnj\ {\isacharparenleft}{\kern0pt}SWAP\ {\isachardollar}{\kern0pt}{\isachardollar}{\kern0pt}\ {\isacharparenleft}{\kern0pt}n{\isacharcomma}{\kern0pt}\ na{\isacharparenright}{\kern0pt}{\isacharparenright}{\kern0pt}{\isacharparenright}{\kern0pt}\ {\isacharparenleft}{\kern0pt}{\isasymlambda}{\isacharparenleft}{\kern0pt}na{\isacharcomma}{\kern0pt}\ n{\isacharparenright}{\kern0pt}{\isachardot}{\kern0pt}\ cnj\ {\isacharparenleft}{\kern0pt}SWAP\ {\isachardollar}{\kern0pt}{\isachardollar}{\kern0pt}\ {\isacharparenleft}{\kern0pt}na{\isacharcomma}{\kern0pt}\ n{\isacharparenright}{\kern0pt}{\isacharparenright}{\kern0pt}{\isacharparenright}{\kern0pt}\ {\isadigit{4}}\ {\isadigit{4}}\ {\isacharequal}{\kern0pt}\ {\isadigit{0}}\ then\ {\isadigit{1}}{\isacharcolon}{\kern0pt}{\isacharcolon}{\kern0pt}complex\ else\ if\ nn\ {\isacharparenleft}{\kern0pt}{\isasymlambda}{\isacharparenleft}{\kern0pt}na{\isacharcomma}{\kern0pt}\ n{\isacharparenright}{\kern0pt}{\isachardot}{\kern0pt}\ cnj\ {\isacharparenleft}{\kern0pt}SWAP\ {\isachardollar}{\kern0pt}{\isachardollar}{\kern0pt}\ {\isacharparenleft}{\kern0pt}n{\isacharcomma}{\kern0pt}\ na{\isacharparenright}{\kern0pt}{\isacharparenright}{\kern0pt}{\isacharparenright}{\kern0pt}\ {\isacharparenleft}{\kern0pt}{\isasymlambda}{\isacharparenleft}{\kern0pt}na{\isacharcomma}{\kern0pt}\ n{\isacharparenright}{\kern0pt}{\isachardot}{\kern0pt}\ cnj\ {\isacharparenleft}{\kern0pt}SWAP\ {\isachardollar}{\kern0pt}{\isachardollar}{\kern0pt}\ {\isacharparenleft}{\kern0pt}na{\isacharcomma}{\kern0pt}\ n{\isacharparenright}{\kern0pt}{\isacharparenright}{\kern0pt}{\isacharparenright}{\kern0pt}\ {\isadigit{4}}\ {\isadigit{4}}\ {\isacharequal}{\kern0pt}\ {\isadigit{1}}\ {\isasymand}\ nna\ {\isacharparenleft}{\kern0pt}{\isasymlambda}{\isacharparenleft}{\kern0pt}na{\isacharcomma}{\kern0pt}\ n{\isacharparenright}{\kern0pt}{\isachardot}{\kern0pt}\ cnj\ {\isacharparenleft}{\kern0pt}SWAP\ {\isachardollar}{\kern0pt}{\isachardollar}{\kern0pt}\ {\isacharparenleft}{\kern0pt}n{\isacharcomma}{\kern0pt}\ na{\isacharparenright}{\kern0pt}{\isacharparenright}{\kern0pt}{\isacharparenright}{\kern0pt}\ {\isacharparenleft}{\kern0pt}{\isasymlambda}{\isacharparenleft}{\kern0pt}na{\isacharcomma}{\kern0pt}\ n{\isacharparenright}{\kern0pt}{\isachardot}{\kern0pt}\ cnj\ {\isacharparenleft}{\kern0pt}SWAP\ {\isachardollar}{\kern0pt}{\isachardollar}{\kern0pt}\ {\isacharparenleft}{\kern0pt}na{\isacharcomma}{\kern0pt}\ n{\isacharparenright}{\kern0pt}{\isacharparenright}{\kern0pt}{\isacharparenright}{\kern0pt}\ {\isadigit{4}}\ {\isadigit{4}}\ {\isacharequal}{\kern0pt}\ {\isadigit{2}}\ then\ {\isadigit{1}}\ else\ if\ nn\ {\isacharparenleft}{\kern0pt}{\isasymlambda}{\isacharparenleft}{\kern0pt}na{\isacharcomma}{\kern0pt}\ n{\isacharparenright}{\kern0pt}{\isachardot}{\kern0pt}\ cnj\ {\isacharparenleft}{\kern0pt}SWAP\ {\isachardollar}{\kern0pt}{\isachardollar}{\kern0pt}\ {\isacharparenleft}{\kern0pt}n{\isacharcomma}{\kern0pt}\ na{\isacharparenright}{\kern0pt}{\isacharparenright}{\kern0pt}{\isacharparenright}{\kern0pt}\ {\isacharparenleft}{\kern0pt}{\isasymlambda}{\isacharparenleft}{\kern0pt}na{\isacharcomma}{\kern0pt}\ n{\isacharparenright}{\kern0pt}{\isachardot}{\kern0pt}\ cnj\ {\isacharparenleft}{\kern0pt}SWAP\ {\isachardollar}{\kern0pt}{\isachardollar}{\kern0pt}\ {\isacharparenleft}{\kern0pt}na{\isacharcomma}{\kern0pt}\ n{\isacharparenright}{\kern0pt}{\isacharparenright}{\kern0pt}{\isacharparenright}{\kern0pt}\ {\isadigit{4}}\ {\isadigit{4}}\ {\isacharequal}{\kern0pt}\ {\isadigit{2}}\ {\isasymand}\ nna\ {\isacharparenleft}{\kern0pt}{\isasymlambda}{\isacharparenleft}{\kern0pt}na{\isacharcomma}{\kern0pt}\ n{\isacharparenright}{\kern0pt}{\isachardot}{\kern0pt}\ cnj\ {\isacharparenleft}{\kern0pt}SWAP\ {\isachardollar}{\kern0pt}{\isachardollar}{\kern0pt}\ {\isacharparenleft}{\kern0pt}n{\isacharcomma}{\kern0pt}\ na{\isacharparenright}{\kern0pt}{\isacharparenright}{\kern0pt}{\isacharparenright}{\kern0pt}\ {\isacharparenleft}{\kern0pt}{\isasymlambda}{\isacharparenleft}{\kern0pt}na{\isacharcomma}{\kern0pt}\ n{\isacharparenright}{\kern0pt}{\isachardot}{\kern0pt}\ cnj\ {\isacharparenleft}{\kern0pt}SWAP\ {\isachardollar}{\kern0pt}{\isachardollar}{\kern0pt}\ {\isacharparenleft}{\kern0pt}na{\isacharcomma}{\kern0pt}\ n{\isacharparenright}{\kern0pt}{\isacharparenright}{\kern0pt}{\isacharparenright}{\kern0pt}\ {\isadigit{4}}\ {\isadigit{4}}\ {\isacharequal}{\kern0pt}\ {\isadigit{1}}\ then\ {\isadigit{1}}\ else\ if\ nn\ {\isacharparenleft}{\kern0pt}{\isasymlambda}{\isacharparenleft}{\kern0pt}na{\isacharcomma}{\kern0pt}\ n{\isacharparenright}{\kern0pt}{\isachardot}{\kern0pt}\ cnj\ {\isacharparenleft}{\kern0pt}SWAP\ {\isachardollar}{\kern0pt}{\isachardollar}{\kern0pt}\ {\isacharparenleft}{\kern0pt}n{\isacharcomma}{\kern0pt}\ na{\isacharparenright}{\kern0pt}{\isacharparenright}{\kern0pt}{\isacharparenright}{\kern0pt}\ {\isacharparenleft}{\kern0pt}{\isasymlambda}{\isacharparenleft}{\kern0pt}na{\isacharcomma}{\kern0pt}\ n{\isacharparenright}{\kern0pt}{\isachardot}{\kern0pt}\ cnj\ {\isacharparenleft}{\kern0pt}SWAP\ {\isachardollar}{\kern0pt}{\isachardollar}{\kern0pt}\ {\isacharparenleft}{\kern0pt}na{\isacharcomma}{\kern0pt}\ n{\isacharparenright}{\kern0pt}{\isacharparenright}{\kern0pt}{\isacharparenright}{\kern0pt}\ {\isadigit{4}}\ {\isadigit{4}}\ {\isacharequal}{\kern0pt}\ {\isadigit{3}}\ {\isasymand}\ nna\ {\isacharparenleft}{\kern0pt}{\isasymlambda}{\isacharparenleft}{\kern0pt}na{\isacharcomma}{\kern0pt}\ n{\isacharparenright}{\kern0pt}{\isachardot}{\kern0pt}\ cnj\ {\isacharparenleft}{\kern0pt}SWAP\ {\isachardollar}{\kern0pt}{\isachardollar}{\kern0pt}\ {\isacharparenleft}{\kern0pt}n{\isacharcomma}{\kern0pt}\ na{\isacharparenright}{\kern0pt}{\isacharparenright}{\kern0pt}{\isacharparenright}{\kern0pt}\ {\isacharparenleft}{\kern0pt}{\isasymlambda}{\isacharparenleft}{\kern0pt}na{\isacharcomma}{\kern0pt}\ n{\isacharparenright}{\kern0pt}{\isachardot}{\kern0pt}\ cnj\ {\isacharparenleft}{\kern0pt}SWAP\ {\isachardollar}{\kern0pt}{\isachardollar}{\kern0pt}\ {\isacharparenleft}{\kern0pt}na{\isacharcomma}{\kern0pt}\ n{\isacharparenright}{\kern0pt}{\isacharparenright}{\kern0pt}{\isacharparenright}{\kern0pt}\ {\isadigit{4}}\ {\isadigit{4}}\ {\isacharequal}{\kern0pt}\ {\isadigit{3}}\ then\ {\isadigit{1}}\ else\ {\isadigit{0}}{\isacharparenright}{\kern0pt}\ {\isacharequal}{\kern0pt}\ {\isadigit{1}}{\isacharparenright}{\kern0pt}\ {\isasymlongrightarrow}\ nn\ {\isacharparenleft}{\kern0pt}{\isasymlambda}{\isacharparenleft}{\kern0pt}na{\isacharcomma}{\kern0pt}\ n{\isacharparenright}{\kern0pt}{\isachardot}{\kern0pt}\ cnj\ {\isacharparenleft}{\kern0pt}SWAP\ {\isachardollar}{\kern0pt}{\isachardollar}{\kern0pt}\ {\isacharparenleft}{\kern0pt}n{\isacharcomma}{\kern0pt}\ na{\isacharparenright}{\kern0pt}{\isacharparenright}{\kern0pt}{\isacharparenright}{\kern0pt}\ {\isacharparenleft}{\kern0pt}{\isasymlambda}{\isacharparenleft}{\kern0pt}na{\isacharcomma}{\kern0pt}\ n{\isacharparenright}{\kern0pt}{\isachardot}{\kern0pt}\ cnj\ {\isacharparenleft}{\kern0pt}SWAP\ {\isachardollar}{\kern0pt}{\isachardollar}{\kern0pt}\ {\isacharparenleft}{\kern0pt}na{\isacharcomma}{\kern0pt}\ n{\isacharparenright}{\kern0pt}{\isacharparenright}{\kern0pt}{\isacharparenright}{\kern0pt}\ {\isadigit{4}}\ {\isadigit{4}}\ {\isacharequal}{\kern0pt}\ {\isadigit{2}}\ {\isasymand}\ nna\ {\isacharparenleft}{\kern0pt}{\isasymlambda}{\isacharparenleft}{\kern0pt}na{\isacharcomma}{\kern0pt}\ n{\isacharparenright}{\kern0pt}{\isachardot}{\kern0pt}\ cnj\ {\isacharparenleft}{\kern0pt}SWAP\ {\isachardollar}{\kern0pt}{\isachardollar}{\kern0pt}\ {\isacharparenleft}{\kern0pt}n{\isacharcomma}{\kern0pt}\ na{\isacharparenright}{\kern0pt}{\isacharparenright}{\kern0pt}{\isacharparenright}{\kern0pt}\ {\isacharparenleft}{\kern0pt}{\isasymlambda}{\isacharparenleft}{\kern0pt}na{\isacharcomma}{\kern0pt}\ n{\isacharparenright}{\kern0pt}{\isachardot}{\kern0pt}\ cnj\ {\isacharparenleft}{\kern0pt}SWAP\ {\isachardollar}{\kern0pt}{\isachardollar}{\kern0pt}\ {\isacharparenleft}{\kern0pt}na{\isacharcomma}{\kern0pt}\ n{\isacharparenright}{\kern0pt}{\isacharparenright}{\kern0pt}{\isacharparenright}{\kern0pt}\ {\isadigit{4}}\ {\isadigit{4}}\ {\isacharequal}{\kern0pt}\ {\isadigit{1}}\ {\isasymor}\ {\isacharparenleft}{\kern0pt}if\ nn\ {\isacharparenleft}{\kern0pt}{\isasymlambda}{\isacharparenleft}{\kern0pt}na{\isacharcomma}{\kern0pt}\ n{\isacharparenright}{\kern0pt}{\isachardot}{\kern0pt}\ cnj\ {\isacharparenleft}{\kern0pt}SWAP\ {\isachardollar}{\kern0pt}{\isachardollar}{\kern0pt}\ {\isacharparenleft}{\kern0pt}n{\isacharcomma}{\kern0pt}\ na{\isacharparenright}{\kern0pt}{\isacharparenright}{\kern0pt}{\isacharparenright}{\kern0pt}\ {\isacharparenleft}{\kern0pt}{\isasymlambda}{\isacharparenleft}{\kern0pt}na{\isacharcomma}{\kern0pt}\ n{\isacharparenright}{\kern0pt}{\isachardot}{\kern0pt}\ cnj\ {\isacharparenleft}{\kern0pt}SWAP\ {\isachardollar}{\kern0pt}{\isachardollar}{\kern0pt}\ {\isacharparenleft}{\kern0pt}na{\isacharcomma}{\kern0pt}\ n{\isacharparenright}{\kern0pt}{\isacharparenright}{\kern0pt}{\isacharparenright}{\kern0pt}\ {\isadigit{4}}\ {\isadigit{4}}\ {\isacharequal}{\kern0pt}\ {\isadigit{0}}\ {\isasymand}\ nna\ {\isacharparenleft}{\kern0pt}{\isasymlambda}{\isacharparenleft}{\kern0pt}na{\isacharcomma}{\kern0pt}\ n{\isacharparenright}{\kern0pt}{\isachardot}{\kern0pt}\ cnj\ {\isacharparenleft}{\kern0pt}SWAP\ {\isachardollar}{\kern0pt}{\isachardollar}{\kern0pt}\ {\isacharparenleft}{\kern0pt}n{\isacharcomma}{\kern0pt}\ na{\isacharparenright}{\kern0pt}{\isacharparenright}{\kern0pt}{\isacharparenright}{\kern0pt}\ {\isacharparenleft}{\kern0pt}{\isasymlambda}{\isacharparenleft}{\kern0pt}na{\isacharcomma}{\kern0pt}\ n{\isacharparenright}{\kern0pt}{\isachardot}{\kern0pt}\ cnj\ {\isacharparenleft}{\kern0pt}SWAP\ {\isachardollar}{\kern0pt}{\isachardollar}{\kern0pt}\ {\isacharparenleft}{\kern0pt}na{\isacharcomma}{\kern0pt}\ n{\isacharparenright}{\kern0pt}{\isacharparenright}{\kern0pt}{\isacharparenright}{\kern0pt}\ {\isadigit{4}}\ {\isadigit{4}}\ {\isacharequal}{\kern0pt}\ {\isadigit{0}}\ then\ {\isadigit{1}}{\isacharcolon}{\kern0pt}{\isacharcolon}{\kern0pt}complex\ else\ if\ nn\ {\isacharparenleft}{\kern0pt}{\isasymlambda}{\isacharparenleft}{\kern0pt}na{\isacharcomma}{\kern0pt}\ n{\isacharparenright}{\kern0pt}{\isachardot}{\kern0pt}\ cnj\ {\isacharparenleft}{\kern0pt}SWAP\ {\isachardollar}{\kern0pt}{\isachardollar}{\kern0pt}\ {\isacharparenleft}{\kern0pt}n{\isacharcomma}{\kern0pt}\ na{\isacharparenright}{\kern0pt}{\isacharparenright}{\kern0pt}{\isacharparenright}{\kern0pt}\ {\isacharparenleft}{\kern0pt}{\isasymlambda}{\isacharparenleft}{\kern0pt}na{\isacharcomma}{\kern0pt}\ n{\isacharparenright}{\kern0pt}{\isachardot}{\kern0pt}\ cnj\ {\isacharparenleft}{\kern0pt}SWAP\ {\isachardollar}{\kern0pt}{\isachardollar}{\kern0pt}\ {\isacharparenleft}{\kern0pt}na{\isacharcomma}{\kern0pt}\ n{\isacharparenright}{\kern0pt}{\isacharparenright}{\kern0pt}{\isacharparenright}{\kern0pt}\ {\isadigit{4}}\ {\isadigit{4}}\ {\isacharequal}{\kern0pt}\ {\isadigit{1}}\ {\isasymand}\ nna\ {\isacharparenleft}{\kern0pt}{\isasymlambda}{\isacharparenleft}{\kern0pt}na{\isacharcomma}{\kern0pt}\ n{\isacharparenright}{\kern0pt}{\isachardot}{\kern0pt}\ cnj\ {\isacharparenleft}{\kern0pt}SWAP\ {\isachardollar}{\kern0pt}{\isachardollar}{\kern0pt}\ {\isacharparenleft}{\kern0pt}n{\isacharcomma}{\kern0pt}\ na{\isacharparenright}{\kern0pt}{\isacharparenright}{\kern0pt}{\isacharparenright}{\kern0pt}\ {\isacharparenleft}{\kern0pt}{\isasymlambda}{\isacharparenleft}{\kern0pt}na{\isacharcomma}{\kern0pt}\ n{\isacharparenright}{\kern0pt}{\isachardot}{\kern0pt}\ cnj\ {\isacharparenleft}{\kern0pt}SWAP\ {\isachardollar}{\kern0pt}{\isachardollar}{\kern0pt}\ {\isacharparenleft}{\kern0pt}na{\isacharcomma}{\kern0pt}\ n{\isacharparenright}{\kern0pt}{\isacharparenright}{\kern0pt}{\isacharparenright}{\kern0pt}\ {\isadigit{4}}\ {\isadigit{4}}\ {\isacharequal}{\kern0pt}\ {\isadigit{2}}\ then\ {\isadigit{1}}\ else\ if\ nn\ {\isacharparenleft}{\kern0pt}{\isasymlambda}{\isacharparenleft}{\kern0pt}na{\isacharcomma}{\kern0pt}\ n{\isacharparenright}{\kern0pt}{\isachardot}{\kern0pt}\ cnj\ {\isacharparenleft}{\kern0pt}SWAP\ {\isachardollar}{\kern0pt}{\isachardollar}{\kern0pt}\ {\isacharparenleft}{\kern0pt}n{\isacharcomma}{\kern0pt}\ na{\isacharparenright}{\kern0pt}{\isacharparenright}{\kern0pt}{\isacharparenright}{\kern0pt}\ {\isacharparenleft}{\kern0pt}{\isasymlambda}{\isacharparenleft}{\kern0pt}na{\isacharcomma}{\kern0pt}\ n{\isacharparenright}{\kern0pt}{\isachardot}{\kern0pt}\ cnj\ {\isacharparenleft}{\kern0pt}SWAP\ {\isachardollar}{\kern0pt}{\isachardollar}{\kern0pt}\ {\isacharparenleft}{\kern0pt}na{\isacharcomma}{\kern0pt}\ n{\isacharparenright}{\kern0pt}{\isacharparenright}{\kern0pt}{\isacharparenright}{\kern0pt}\ {\isadigit{4}}\ {\isadigit{4}}\ {\isacharequal}{\kern0pt}\ {\isadigit{2}}\ {\isasymand}\ nna\ {\isacharparenleft}{\kern0pt}{\isasymlambda}{\isacharparenleft}{\kern0pt}na{\isacharcomma}{\kern0pt}\ n{\isacharparenright}{\kern0pt}{\isachardot}{\kern0pt}\ cnj\ {\isacharparenleft}{\kern0pt}SWAP\ {\isachardollar}{\kern0pt}{\isachardollar}{\kern0pt}\ {\isacharparenleft}{\kern0pt}n{\isacharcomma}{\kern0pt}\ na{\isacharparenright}{\kern0pt}{\isacharparenright}{\kern0pt}{\isacharparenright}{\kern0pt}\ {\isacharparenleft}{\kern0pt}{\isasymlambda}{\isacharparenleft}{\kern0pt}na{\isacharcomma}{\kern0pt}\ n{\isacharparenright}{\kern0pt}{\isachardot}{\kern0pt}\ cnj\ {\isacharparenleft}{\kern0pt}SWAP\ {\isachardollar}{\kern0pt}{\isachardollar}{\kern0pt}\ {\isacharparenleft}{\kern0pt}na{\isacharcomma}{\kern0pt}\ n{\isacharparenright}{\kern0pt}{\isacharparenright}{\kern0pt}{\isacharparenright}{\kern0pt}\ {\isadigit{4}}\ {\isadigit{4}}\ {\isacharequal}{\kern0pt}\ {\isadigit{1}}\ then\ {\isadigit{1}}\ else\ if\ nn\ {\isacharparenleft}{\kern0pt}{\isasymlambda}{\isacharparenleft}{\kern0pt}na{\isacharcomma}{\kern0pt}\ n{\isacharparenright}{\kern0pt}{\isachardot}{\kern0pt}\ cnj\ {\isacharparenleft}{\kern0pt}SWAP\ {\isachardollar}{\kern0pt}{\isachardollar}{\kern0pt}\ {\isacharparenleft}{\kern0pt}n{\isacharcomma}{\kern0pt}\ na{\isacharparenright}{\kern0pt}{\isacharparenright}{\kern0pt}{\isacharparenright}{\kern0pt}\ {\isacharparenleft}{\kern0pt}{\isasymlambda}{\isacharparenleft}{\kern0pt}na{\isacharcomma}{\kern0pt}\ n{\isacharparenright}{\kern0pt}{\isachardot}{\kern0pt}\ cnj\ {\isacharparenleft}{\kern0pt}SWAP\ {\isachardollar}{\kern0pt}{\isachardollar}{\kern0pt}\ {\isacharparenleft}{\kern0pt}na{\isacharcomma}{\kern0pt}\ n{\isacharparenright}{\kern0pt}{\isacharparenright}{\kern0pt}{\isacharparenright}{\kern0pt}\ {\isadigit{4}}\ {\isadigit{4}}\ {\isacharequal}{\kern0pt}\ {\isadigit{3}}\ {\isasymand}\ nna\ {\isacharparenleft}{\kern0pt}{\isasymlambda}{\isacharparenleft}{\kern0pt}na{\isacharcomma}{\kern0pt}\ n{\isacharparenright}{\kern0pt}{\isachardot}{\kern0pt}\ cnj\ {\isacharparenleft}{\kern0pt}SWAP\ {\isachardollar}{\kern0pt}{\isachardollar}{\kern0pt}\ {\isacharparenleft}{\kern0pt}n{\isacharcomma}{\kern0pt}\ na{\isacharparenright}{\kern0pt}{\isacharparenright}{\kern0pt}{\isacharparenright}{\kern0pt}\ {\isacharparenleft}{\kern0pt}{\isasymlambda}{\isacharparenleft}{\kern0pt}na{\isacharcomma}{\kern0pt}\ n{\isacharparenright}{\kern0pt}{\isachardot}{\kern0pt}\ cnj\ {\isacharparenleft}{\kern0pt}SWAP\ {\isachardollar}{\kern0pt}{\isachardollar}{\kern0pt}\ {\isacharparenleft}{\kern0pt}na{\isacharcomma}{\kern0pt}\ n{\isacharparenright}{\kern0pt}{\isacharparenright}{\kern0pt}{\isacharparenright}{\kern0pt}\ {\isadigit{4}}\ {\isadigit{4}}\ {\isacharequal}{\kern0pt}\ {\isadigit{3}}\ then\ {\isadigit{1}}\ else\ {\isadigit{0}}{\isacharparenright}{\kern0pt}\ {\isacharequal}{\kern0pt}\ {\isadigit{0}}{\isachardoublequoteclose}\isanewline
\ \ \ \ \ \ \ \ \isacommand{by}\isamarkupfalse%
\ presburger\ \isacommand{{\isacharbraceright}{\kern0pt}}\isamarkupfalse%
\isanewline
\ \ \ \ \isacommand{moreover}\isamarkupfalse%
\isanewline
\ \ \ \ \isacommand{{\isacharbraceleft}{\kern0pt}}\isamarkupfalse%
\ \isacommand{assume}\isamarkupfalse%
\ {\isachardoublequoteopen}nna\ {\isacharparenleft}{\kern0pt}{\isasymlambda}{\isacharparenleft}{\kern0pt}na{\isacharcomma}{\kern0pt}\ n{\isacharparenright}{\kern0pt}{\isachardot}{\kern0pt}\ cnj\ {\isacharparenleft}{\kern0pt}SWAP\ {\isachardollar}{\kern0pt}{\isachardollar}{\kern0pt}\ {\isacharparenleft}{\kern0pt}n{\isacharcomma}{\kern0pt}\ na{\isacharparenright}{\kern0pt}{\isacharparenright}{\kern0pt}{\isacharparenright}{\kern0pt}\ {\isacharparenleft}{\kern0pt}{\isasymlambda}{\isacharparenleft}{\kern0pt}na{\isacharcomma}{\kern0pt}\ n{\isacharparenright}{\kern0pt}{\isachardot}{\kern0pt}\ cnj\ {\isacharparenleft}{\kern0pt}SWAP\ {\isachardollar}{\kern0pt}{\isachardollar}{\kern0pt}\ {\isacharparenleft}{\kern0pt}na{\isacharcomma}{\kern0pt}\ n{\isacharparenright}{\kern0pt}{\isacharparenright}{\kern0pt}{\isacharparenright}{\kern0pt}\ {\isadigit{4}}\ {\isadigit{4}}\ {\isacharequal}{\kern0pt}\ {\isadigit{3}}\ {\isasymand}\ nn\ {\isacharparenleft}{\kern0pt}{\isasymlambda}{\isacharparenleft}{\kern0pt}na{\isacharcomma}{\kern0pt}\ n{\isacharparenright}{\kern0pt}{\isachardot}{\kern0pt}\ cnj\ {\isacharparenleft}{\kern0pt}SWAP\ {\isachardollar}{\kern0pt}{\isachardollar}{\kern0pt}\ {\isacharparenleft}{\kern0pt}n{\isacharcomma}{\kern0pt}\ na{\isacharparenright}{\kern0pt}{\isacharparenright}{\kern0pt}{\isacharparenright}{\kern0pt}\ {\isacharparenleft}{\kern0pt}{\isasymlambda}{\isacharparenleft}{\kern0pt}na{\isacharcomma}{\kern0pt}\ n{\isacharparenright}{\kern0pt}{\isachardot}{\kern0pt}\ cnj\ {\isacharparenleft}{\kern0pt}SWAP\ {\isachardollar}{\kern0pt}{\isachardollar}{\kern0pt}\ {\isacharparenleft}{\kern0pt}na{\isacharcomma}{\kern0pt}\ n{\isacharparenright}{\kern0pt}{\isacharparenright}{\kern0pt}{\isacharparenright}{\kern0pt}\ {\isadigit{4}}\ {\isadigit{4}}\ {\isacharequal}{\kern0pt}\ {\isadigit{3}}{\isachardoublequoteclose}\isanewline
\ \ \ \ \ \ \isacommand{then}\isamarkupfalse%
\ \isacommand{have}\isamarkupfalse%
\ {\isachardoublequoteopen}{\isacharparenleft}{\kern0pt}if\ nna\ {\isacharparenleft}{\kern0pt}{\isasymlambda}{\isacharparenleft}{\kern0pt}na{\isacharcomma}{\kern0pt}\ n{\isacharparenright}{\kern0pt}{\isachardot}{\kern0pt}\ cnj\ {\isacharparenleft}{\kern0pt}SWAP\ {\isachardollar}{\kern0pt}{\isachardollar}{\kern0pt}\ {\isacharparenleft}{\kern0pt}n{\isacharcomma}{\kern0pt}\ na{\isacharparenright}{\kern0pt}{\isacharparenright}{\kern0pt}{\isacharparenright}{\kern0pt}\ {\isacharparenleft}{\kern0pt}{\isasymlambda}{\isacharparenleft}{\kern0pt}na{\isacharcomma}{\kern0pt}\ n{\isacharparenright}{\kern0pt}{\isachardot}{\kern0pt}\ cnj\ {\isacharparenleft}{\kern0pt}SWAP\ {\isachardollar}{\kern0pt}{\isachardollar}{\kern0pt}\ {\isacharparenleft}{\kern0pt}na{\isacharcomma}{\kern0pt}\ n{\isacharparenright}{\kern0pt}{\isacharparenright}{\kern0pt}{\isacharparenright}{\kern0pt}\ {\isadigit{4}}\ {\isadigit{4}}\ {\isasymnoteq}\ {\isadigit{2}}\ {\isasymor}\ nn\ {\isacharparenleft}{\kern0pt}{\isasymlambda}{\isacharparenleft}{\kern0pt}na{\isacharcomma}{\kern0pt}\ n{\isacharparenright}{\kern0pt}{\isachardot}{\kern0pt}\ cnj\ {\isacharparenleft}{\kern0pt}SWAP\ {\isachardollar}{\kern0pt}{\isachardollar}{\kern0pt}\ {\isacharparenleft}{\kern0pt}n{\isacharcomma}{\kern0pt}\ na{\isacharparenright}{\kern0pt}{\isacharparenright}{\kern0pt}{\isacharparenright}{\kern0pt}\ {\isacharparenleft}{\kern0pt}{\isasymlambda}{\isacharparenleft}{\kern0pt}na{\isacharcomma}{\kern0pt}\ n{\isacharparenright}{\kern0pt}{\isachardot}{\kern0pt}\ cnj\ {\isacharparenleft}{\kern0pt}SWAP\ {\isachardollar}{\kern0pt}{\isachardollar}{\kern0pt}\ {\isacharparenleft}{\kern0pt}na{\isacharcomma}{\kern0pt}\ n{\isacharparenright}{\kern0pt}{\isacharparenright}{\kern0pt}{\isacharparenright}{\kern0pt}\ {\isadigit{4}}\ {\isadigit{4}}\ {\isasymnoteq}\ {\isadigit{1}}\ then\ if\ nna\ {\isacharparenleft}{\kern0pt}{\isasymlambda}{\isacharparenleft}{\kern0pt}na{\isacharcomma}{\kern0pt}\ n{\isacharparenright}{\kern0pt}{\isachardot}{\kern0pt}\ cnj\ {\isacharparenleft}{\kern0pt}SWAP\ {\isachardollar}{\kern0pt}{\isachardollar}{\kern0pt}\ {\isacharparenleft}{\kern0pt}n{\isacharcomma}{\kern0pt}\ na{\isacharparenright}{\kern0pt}{\isacharparenright}{\kern0pt}{\isacharparenright}{\kern0pt}\ {\isacharparenleft}{\kern0pt}{\isasymlambda}{\isacharparenleft}{\kern0pt}na{\isacharcomma}{\kern0pt}\ n{\isacharparenright}{\kern0pt}{\isachardot}{\kern0pt}\ cnj\ {\isacharparenleft}{\kern0pt}SWAP\ {\isachardollar}{\kern0pt}{\isachardollar}{\kern0pt}\ {\isacharparenleft}{\kern0pt}na{\isacharcomma}{\kern0pt}\ n{\isacharparenright}{\kern0pt}{\isacharparenright}{\kern0pt}{\isacharparenright}{\kern0pt}\ {\isadigit{4}}\ {\isadigit{4}}\ {\isasymnoteq}\ {\isadigit{3}}\ {\isasymor}\ nn\ {\isacharparenleft}{\kern0pt}{\isasymlambda}{\isacharparenleft}{\kern0pt}na{\isacharcomma}{\kern0pt}\ n{\isacharparenright}{\kern0pt}{\isachardot}{\kern0pt}\ cnj\ {\isacharparenleft}{\kern0pt}SWAP\ {\isachardollar}{\kern0pt}{\isachardollar}{\kern0pt}\ {\isacharparenleft}{\kern0pt}n{\isacharcomma}{\kern0pt}\ na{\isacharparenright}{\kern0pt}{\isacharparenright}{\kern0pt}{\isacharparenright}{\kern0pt}\ {\isacharparenleft}{\kern0pt}{\isasymlambda}{\isacharparenleft}{\kern0pt}na{\isacharcomma}{\kern0pt}\ n{\isacharparenright}{\kern0pt}{\isachardot}{\kern0pt}\ cnj\ {\isacharparenleft}{\kern0pt}SWAP\ {\isachardollar}{\kern0pt}{\isachardollar}{\kern0pt}\ {\isacharparenleft}{\kern0pt}na{\isacharcomma}{\kern0pt}\ n{\isacharparenright}{\kern0pt}{\isacharparenright}{\kern0pt}{\isacharparenright}{\kern0pt}\ {\isadigit{4}}\ {\isadigit{4}}\ {\isasymnoteq}\ {\isadigit{3}}\ then\ {\isacharparenleft}{\kern0pt}if\ nna\ {\isacharparenleft}{\kern0pt}{\isasymlambda}{\isacharparenleft}{\kern0pt}na{\isacharcomma}{\kern0pt}\ n{\isacharparenright}{\kern0pt}{\isachardot}{\kern0pt}\ cnj\ {\isacharparenleft}{\kern0pt}SWAP\ {\isachardollar}{\kern0pt}{\isachardollar}{\kern0pt}\ {\isacharparenleft}{\kern0pt}n{\isacharcomma}{\kern0pt}\ na{\isacharparenright}{\kern0pt}{\isacharparenright}{\kern0pt}{\isacharparenright}{\kern0pt}\ {\isacharparenleft}{\kern0pt}{\isasymlambda}{\isacharparenleft}{\kern0pt}na{\isacharcomma}{\kern0pt}\ n{\isacharparenright}{\kern0pt}{\isachardot}{\kern0pt}\ cnj\ {\isacharparenleft}{\kern0pt}SWAP\ {\isachardollar}{\kern0pt}{\isachardollar}{\kern0pt}\ {\isacharparenleft}{\kern0pt}na{\isacharcomma}{\kern0pt}\ n{\isacharparenright}{\kern0pt}{\isacharparenright}{\kern0pt}{\isacharparenright}{\kern0pt}\ {\isadigit{4}}\ {\isadigit{4}}\ {\isacharequal}{\kern0pt}\ {\isadigit{0}}\ {\isasymand}\ nn\ {\isacharparenleft}{\kern0pt}{\isasymlambda}{\isacharparenleft}{\kern0pt}na{\isacharcomma}{\kern0pt}\ n{\isacharparenright}{\kern0pt}{\isachardot}{\kern0pt}\ cnj\ {\isacharparenleft}{\kern0pt}SWAP\ {\isachardollar}{\kern0pt}{\isachardollar}{\kern0pt}\ {\isacharparenleft}{\kern0pt}n{\isacharcomma}{\kern0pt}\ na{\isacharparenright}{\kern0pt}{\isacharparenright}{\kern0pt}{\isacharparenright}{\kern0pt}\ {\isacharparenleft}{\kern0pt}{\isasymlambda}{\isacharparenleft}{\kern0pt}na{\isacharcomma}{\kern0pt}\ n{\isacharparenright}{\kern0pt}{\isachardot}{\kern0pt}\ cnj\ {\isacharparenleft}{\kern0pt}SWAP\ {\isachardollar}{\kern0pt}{\isachardollar}{\kern0pt}\ {\isacharparenleft}{\kern0pt}na{\isacharcomma}{\kern0pt}\ n{\isacharparenright}{\kern0pt}{\isacharparenright}{\kern0pt}{\isacharparenright}{\kern0pt}\ {\isadigit{4}}\ {\isadigit{4}}\ {\isacharequal}{\kern0pt}\ {\isadigit{0}}\ then\ {\isadigit{1}}{\isacharcolon}{\kern0pt}{\isacharcolon}{\kern0pt}complex\ else\ if\ nna\ {\isacharparenleft}{\kern0pt}{\isasymlambda}{\isacharparenleft}{\kern0pt}na{\isacharcomma}{\kern0pt}\ n{\isacharparenright}{\kern0pt}{\isachardot}{\kern0pt}\ cnj\ {\isacharparenleft}{\kern0pt}SWAP\ {\isachardollar}{\kern0pt}{\isachardollar}{\kern0pt}\ {\isacharparenleft}{\kern0pt}n{\isacharcomma}{\kern0pt}\ na{\isacharparenright}{\kern0pt}{\isacharparenright}{\kern0pt}{\isacharparenright}{\kern0pt}\ {\isacharparenleft}{\kern0pt}{\isasymlambda}{\isacharparenleft}{\kern0pt}na{\isacharcomma}{\kern0pt}\ n{\isacharparenright}{\kern0pt}{\isachardot}{\kern0pt}\ cnj\ {\isacharparenleft}{\kern0pt}SWAP\ {\isachardollar}{\kern0pt}{\isachardollar}{\kern0pt}\ {\isacharparenleft}{\kern0pt}na{\isacharcomma}{\kern0pt}\ n{\isacharparenright}{\kern0pt}{\isacharparenright}{\kern0pt}{\isacharparenright}{\kern0pt}\ {\isadigit{4}}\ {\isadigit{4}}\ {\isacharequal}{\kern0pt}\ {\isadigit{1}}\ {\isasymand}\ nn\ {\isacharparenleft}{\kern0pt}{\isasymlambda}{\isacharparenleft}{\kern0pt}na{\isacharcomma}{\kern0pt}\ n{\isacharparenright}{\kern0pt}{\isachardot}{\kern0pt}\ cnj\ {\isacharparenleft}{\kern0pt}SWAP\ {\isachardollar}{\kern0pt}{\isachardollar}{\kern0pt}\ {\isacharparenleft}{\kern0pt}n{\isacharcomma}{\kern0pt}\ na{\isacharparenright}{\kern0pt}{\isacharparenright}{\kern0pt}{\isacharparenright}{\kern0pt}\ {\isacharparenleft}{\kern0pt}{\isasymlambda}{\isacharparenleft}{\kern0pt}na{\isacharcomma}{\kern0pt}\ n{\isacharparenright}{\kern0pt}{\isachardot}{\kern0pt}\ cnj\ {\isacharparenleft}{\kern0pt}SWAP\ {\isachardollar}{\kern0pt}{\isachardollar}{\kern0pt}\ {\isacharparenleft}{\kern0pt}na{\isacharcomma}{\kern0pt}\ n{\isacharparenright}{\kern0pt}{\isacharparenright}{\kern0pt}{\isacharparenright}{\kern0pt}\ {\isadigit{4}}\ {\isadigit{4}}\ {\isacharequal}{\kern0pt}\ {\isadigit{2}}\ then\ {\isadigit{1}}\ else\ if\ nna\ {\isacharparenleft}{\kern0pt}{\isasymlambda}{\isacharparenleft}{\kern0pt}na{\isacharcomma}{\kern0pt}\ n{\isacharparenright}{\kern0pt}{\isachardot}{\kern0pt}\ cnj\ {\isacharparenleft}{\kern0pt}SWAP\ {\isachardollar}{\kern0pt}{\isachardollar}{\kern0pt}\ {\isacharparenleft}{\kern0pt}n{\isacharcomma}{\kern0pt}\ na{\isacharparenright}{\kern0pt}{\isacharparenright}{\kern0pt}{\isacharparenright}{\kern0pt}\ {\isacharparenleft}{\kern0pt}{\isasymlambda}{\isacharparenleft}{\kern0pt}na{\isacharcomma}{\kern0pt}\ n{\isacharparenright}{\kern0pt}{\isachardot}{\kern0pt}\ cnj\ {\isacharparenleft}{\kern0pt}SWAP\ {\isachardollar}{\kern0pt}{\isachardollar}{\kern0pt}\ {\isacharparenleft}{\kern0pt}na{\isacharcomma}{\kern0pt}\ n{\isacharparenright}{\kern0pt}{\isacharparenright}{\kern0pt}{\isacharparenright}{\kern0pt}\ {\isadigit{4}}\ {\isadigit{4}}\ {\isacharequal}{\kern0pt}\ {\isadigit{2}}\ {\isasymand}\ nn\ {\isacharparenleft}{\kern0pt}{\isasymlambda}{\isacharparenleft}{\kern0pt}na{\isacharcomma}{\kern0pt}\ n{\isacharparenright}{\kern0pt}{\isachardot}{\kern0pt}\ cnj\ {\isacharparenleft}{\kern0pt}SWAP\ {\isachardollar}{\kern0pt}{\isachardollar}{\kern0pt}\ {\isacharparenleft}{\kern0pt}n{\isacharcomma}{\kern0pt}\ na{\isacharparenright}{\kern0pt}{\isacharparenright}{\kern0pt}{\isacharparenright}{\kern0pt}\ {\isacharparenleft}{\kern0pt}{\isasymlambda}{\isacharparenleft}{\kern0pt}na{\isacharcomma}{\kern0pt}\ n{\isacharparenright}{\kern0pt}{\isachardot}{\kern0pt}\ cnj\ {\isacharparenleft}{\kern0pt}SWAP\ {\isachardollar}{\kern0pt}{\isachardollar}{\kern0pt}\ {\isacharparenleft}{\kern0pt}na{\isacharcomma}{\kern0pt}\ n{\isacharparenright}{\kern0pt}{\isacharparenright}{\kern0pt}{\isacharparenright}{\kern0pt}\ {\isadigit{4}}\ {\isadigit{4}}\ {\isacharequal}{\kern0pt}\ {\isadigit{1}}\ then\ {\isadigit{1}}\ else\ if\ nna\ {\isacharparenleft}{\kern0pt}{\isasymlambda}{\isacharparenleft}{\kern0pt}na{\isacharcomma}{\kern0pt}\ n{\isacharparenright}{\kern0pt}{\isachardot}{\kern0pt}\ cnj\ {\isacharparenleft}{\kern0pt}SWAP\ {\isachardollar}{\kern0pt}{\isachardollar}{\kern0pt}\ {\isacharparenleft}{\kern0pt}n{\isacharcomma}{\kern0pt}\ na{\isacharparenright}{\kern0pt}{\isacharparenright}{\kern0pt}{\isacharparenright}{\kern0pt}\ {\isacharparenleft}{\kern0pt}{\isasymlambda}{\isacharparenleft}{\kern0pt}na{\isacharcomma}{\kern0pt}\ n{\isacharparenright}{\kern0pt}{\isachardot}{\kern0pt}\ cnj\ {\isacharparenleft}{\kern0pt}SWAP\ {\isachardollar}{\kern0pt}{\isachardollar}{\kern0pt}\ {\isacharparenleft}{\kern0pt}na{\isacharcomma}{\kern0pt}\ n{\isacharparenright}{\kern0pt}{\isacharparenright}{\kern0pt}{\isacharparenright}{\kern0pt}\ {\isadigit{4}}\ {\isadigit{4}}\ {\isacharequal}{\kern0pt}\ {\isadigit{3}}\ {\isasymand}\ nn\ {\isacharparenleft}{\kern0pt}{\isasymlambda}{\isacharparenleft}{\kern0pt}na{\isacharcomma}{\kern0pt}\ n{\isacharparenright}{\kern0pt}{\isachardot}{\kern0pt}\ cnj\ {\isacharparenleft}{\kern0pt}SWAP\ {\isachardollar}{\kern0pt}{\isachardollar}{\kern0pt}\ {\isacharparenleft}{\kern0pt}n{\isacharcomma}{\kern0pt}\ na{\isacharparenright}{\kern0pt}{\isacharparenright}{\kern0pt}{\isacharparenright}{\kern0pt}\ {\isacharparenleft}{\kern0pt}{\isasymlambda}{\isacharparenleft}{\kern0pt}na{\isacharcomma}{\kern0pt}\ n{\isacharparenright}{\kern0pt}{\isachardot}{\kern0pt}\ cnj\ {\isacharparenleft}{\kern0pt}SWAP\ {\isachardollar}{\kern0pt}{\isachardollar}{\kern0pt}\ {\isacharparenleft}{\kern0pt}na{\isacharcomma}{\kern0pt}\ n{\isacharparenright}{\kern0pt}{\isacharparenright}{\kern0pt}{\isacharparenright}{\kern0pt}\ {\isadigit{4}}\ {\isadigit{4}}\ {\isacharequal}{\kern0pt}\ {\isadigit{3}}\ then\ {\isadigit{1}}\ else\ {\isadigit{0}}{\isacharparenright}{\kern0pt}\ {\isacharequal}{\kern0pt}\ {\isadigit{0}}\ else\ {\isacharparenleft}{\kern0pt}if\ nna\ {\isacharparenleft}{\kern0pt}{\isasymlambda}{\isacharparenleft}{\kern0pt}na{\isacharcomma}{\kern0pt}\ n{\isacharparenright}{\kern0pt}{\isachardot}{\kern0pt}\ cnj\ {\isacharparenleft}{\kern0pt}SWAP\ {\isachardollar}{\kern0pt}{\isachardollar}{\kern0pt}\ {\isacharparenleft}{\kern0pt}n{\isacharcomma}{\kern0pt}\ na{\isacharparenright}{\kern0pt}{\isacharparenright}{\kern0pt}{\isacharparenright}{\kern0pt}\ {\isacharparenleft}{\kern0pt}{\isasymlambda}{\isacharparenleft}{\kern0pt}na{\isacharcomma}{\kern0pt}\ n{\isacharparenright}{\kern0pt}{\isachardot}{\kern0pt}\ cnj\ {\isacharparenleft}{\kern0pt}SWAP\ {\isachardollar}{\kern0pt}{\isachardollar}{\kern0pt}\ {\isacharparenleft}{\kern0pt}na{\isacharcomma}{\kern0pt}\ n{\isacharparenright}{\kern0pt}{\isacharparenright}{\kern0pt}{\isacharparenright}{\kern0pt}\ {\isadigit{4}}\ {\isadigit{4}}\ {\isacharequal}{\kern0pt}\ {\isadigit{0}}\ {\isasymand}\ nn\ {\isacharparenleft}{\kern0pt}{\isasymlambda}{\isacharparenleft}{\kern0pt}na{\isacharcomma}{\kern0pt}\ n{\isacharparenright}{\kern0pt}{\isachardot}{\kern0pt}\ cnj\ {\isacharparenleft}{\kern0pt}SWAP\ {\isachardollar}{\kern0pt}{\isachardollar}{\kern0pt}\ {\isacharparenleft}{\kern0pt}n{\isacharcomma}{\kern0pt}\ na{\isacharparenright}{\kern0pt}{\isacharparenright}{\kern0pt}{\isacharparenright}{\kern0pt}\ {\isacharparenleft}{\kern0pt}{\isasymlambda}{\isacharparenleft}{\kern0pt}na{\isacharcomma}{\kern0pt}\ n{\isacharparenright}{\kern0pt}{\isachardot}{\kern0pt}\ cnj\ {\isacharparenleft}{\kern0pt}SWAP\ {\isachardollar}{\kern0pt}{\isachardollar}{\kern0pt}\ {\isacharparenleft}{\kern0pt}na{\isacharcomma}{\kern0pt}\ n{\isacharparenright}{\kern0pt}{\isacharparenright}{\kern0pt}{\isacharparenright}{\kern0pt}\ {\isadigit{4}}\ {\isadigit{4}}\ {\isacharequal}{\kern0pt}\ {\isadigit{0}}\ then\ {\isadigit{1}}{\isacharcolon}{\kern0pt}{\isacharcolon}{\kern0pt}complex\ else\ if\ nna\ {\isacharparenleft}{\kern0pt}{\isasymlambda}{\isacharparenleft}{\kern0pt}na{\isacharcomma}{\kern0pt}\ n{\isacharparenright}{\kern0pt}{\isachardot}{\kern0pt}\ cnj\ {\isacharparenleft}{\kern0pt}SWAP\ {\isachardollar}{\kern0pt}{\isachardollar}{\kern0pt}\ {\isacharparenleft}{\kern0pt}n{\isacharcomma}{\kern0pt}\ na{\isacharparenright}{\kern0pt}{\isacharparenright}{\kern0pt}{\isacharparenright}{\kern0pt}\ {\isacharparenleft}{\kern0pt}{\isasymlambda}{\isacharparenleft}{\kern0pt}na{\isacharcomma}{\kern0pt}\ n{\isacharparenright}{\kern0pt}{\isachardot}{\kern0pt}\ cnj\ {\isacharparenleft}{\kern0pt}SWAP\ {\isachardollar}{\kern0pt}{\isachardollar}{\kern0pt}\ {\isacharparenleft}{\kern0pt}na{\isacharcomma}{\kern0pt}\ n{\isacharparenright}{\kern0pt}{\isacharparenright}{\kern0pt}{\isacharparenright}{\kern0pt}\ {\isadigit{4}}\ {\isadigit{4}}\ {\isacharequal}{\kern0pt}\ {\isadigit{1}}\ {\isasymand}\ nn\ {\isacharparenleft}{\kern0pt}{\isasymlambda}{\isacharparenleft}{\kern0pt}na{\isacharcomma}{\kern0pt}\ n{\isacharparenright}{\kern0pt}{\isachardot}{\kern0pt}\ cnj\ {\isacharparenleft}{\kern0pt}SWAP\ {\isachardollar}{\kern0pt}{\isachardollar}{\kern0pt}\ {\isacharparenleft}{\kern0pt}n{\isacharcomma}{\kern0pt}\ na{\isacharparenright}{\kern0pt}{\isacharparenright}{\kern0pt}{\isacharparenright}{\kern0pt}\ {\isacharparenleft}{\kern0pt}{\isasymlambda}{\isacharparenleft}{\kern0pt}na{\isacharcomma}{\kern0pt}\ n{\isacharparenright}{\kern0pt}{\isachardot}{\kern0pt}\ cnj\ {\isacharparenleft}{\kern0pt}SWAP\ {\isachardollar}{\kern0pt}{\isachardollar}{\kern0pt}\ {\isacharparenleft}{\kern0pt}na{\isacharcomma}{\kern0pt}\ n{\isacharparenright}{\kern0pt}{\isacharparenright}{\kern0pt}{\isacharparenright}{\kern0pt}\ {\isadigit{4}}\ {\isadigit{4}}\ {\isacharequal}{\kern0pt}\ {\isadigit{2}}\ then\ {\isadigit{1}}\ else\ if\ nna\ {\isacharparenleft}{\kern0pt}{\isasymlambda}{\isacharparenleft}{\kern0pt}na{\isacharcomma}{\kern0pt}\ n{\isacharparenright}{\kern0pt}{\isachardot}{\kern0pt}\ cnj\ {\isacharparenleft}{\kern0pt}SWAP\ {\isachardollar}{\kern0pt}{\isachardollar}{\kern0pt}\ {\isacharparenleft}{\kern0pt}n{\isacharcomma}{\kern0pt}\ na{\isacharparenright}{\kern0pt}{\isacharparenright}{\kern0pt}{\isacharparenright}{\kern0pt}\ {\isacharparenleft}{\kern0pt}{\isasymlambda}{\isacharparenleft}{\kern0pt}na{\isacharcomma}{\kern0pt}\ n{\isacharparenright}{\kern0pt}{\isachardot}{\kern0pt}\ cnj\ {\isacharparenleft}{\kern0pt}SWAP\ {\isachardollar}{\kern0pt}{\isachardollar}{\kern0pt}\ {\isacharparenleft}{\kern0pt}na{\isacharcomma}{\kern0pt}\ n{\isacharparenright}{\kern0pt}{\isacharparenright}{\kern0pt}{\isacharparenright}{\kern0pt}\ {\isadigit{4}}\ {\isadigit{4}}\ {\isacharequal}{\kern0pt}\ {\isadigit{2}}\ {\isasymand}\ nn\ {\isacharparenleft}{\kern0pt}{\isasymlambda}{\isacharparenleft}{\kern0pt}na{\isacharcomma}{\kern0pt}\ n{\isacharparenright}{\kern0pt}{\isachardot}{\kern0pt}\ cnj\ {\isacharparenleft}{\kern0pt}SWAP\ {\isachardollar}{\kern0pt}{\isachardollar}{\kern0pt}\ {\isacharparenleft}{\kern0pt}n{\isacharcomma}{\kern0pt}\ na{\isacharparenright}{\kern0pt}{\isacharparenright}{\kern0pt}{\isacharparenright}{\kern0pt}\ {\isacharparenleft}{\kern0pt}{\isasymlambda}{\isacharparenleft}{\kern0pt}na{\isacharcomma}{\kern0pt}\ n{\isacharparenright}{\kern0pt}{\isachardot}{\kern0pt}\ cnj\ {\isacharparenleft}{\kern0pt}SWAP\ {\isachardollar}{\kern0pt}{\isachardollar}{\kern0pt}\ {\isacharparenleft}{\kern0pt}na{\isacharcomma}{\kern0pt}\ n{\isacharparenright}{\kern0pt}{\isacharparenright}{\kern0pt}{\isacharparenright}{\kern0pt}\ {\isadigit{4}}\ {\isadigit{4}}\ {\isacharequal}{\kern0pt}\ {\isadigit{1}}\ then\ {\isadigit{1}}\ else\ if\ nna\ {\isacharparenleft}{\kern0pt}{\isasymlambda}{\isacharparenleft}{\kern0pt}na{\isacharcomma}{\kern0pt}\ n{\isacharparenright}{\kern0pt}{\isachardot}{\kern0pt}\ cnj\ {\isacharparenleft}{\kern0pt}SWAP\ {\isachardollar}{\kern0pt}{\isachardollar}{\kern0pt}\ {\isacharparenleft}{\kern0pt}n{\isacharcomma}{\kern0pt}\ na{\isacharparenright}{\kern0pt}{\isacharparenright}{\kern0pt}{\isacharparenright}{\kern0pt}\ {\isacharparenleft}{\kern0pt}{\isasymlambda}{\isacharparenleft}{\kern0pt}na{\isacharcomma}{\kern0pt}\ n{\isacharparenright}{\kern0pt}{\isachardot}{\kern0pt}\ cnj\ {\isacharparenleft}{\kern0pt}SWAP\ {\isachardollar}{\kern0pt}{\isachardollar}{\kern0pt}\ {\isacharparenleft}{\kern0pt}na{\isacharcomma}{\kern0pt}\ n{\isacharparenright}{\kern0pt}{\isacharparenright}{\kern0pt}{\isacharparenright}{\kern0pt}\ {\isadigit{4}}\ {\isadigit{4}}\ {\isacharequal}{\kern0pt}\ {\isadigit{3}}\ {\isasymand}\ nn\ {\isacharparenleft}{\kern0pt}{\isasymlambda}{\isacharparenleft}{\kern0pt}na{\isacharcomma}{\kern0pt}\ n{\isacharparenright}{\kern0pt}{\isachardot}{\kern0pt}\ cnj\ {\isacharparenleft}{\kern0pt}SWAP\ {\isachardollar}{\kern0pt}{\isachardollar}{\kern0pt}\ {\isacharparenleft}{\kern0pt}n{\isacharcomma}{\kern0pt}\ na{\isacharparenright}{\kern0pt}{\isacharparenright}{\kern0pt}{\isacharparenright}{\kern0pt}\ {\isacharparenleft}{\kern0pt}{\isasymlambda}{\isacharparenleft}{\kern0pt}na{\isacharcomma}{\kern0pt}\ n{\isacharparenright}{\kern0pt}{\isachardot}{\kern0pt}\ cnj\ {\isacharparenleft}{\kern0pt}SWAP\ {\isachardollar}{\kern0pt}{\isachardollar}{\kern0pt}\ {\isacharparenleft}{\kern0pt}na{\isacharcomma}{\kern0pt}\ n{\isacharparenright}{\kern0pt}{\isacharparenright}{\kern0pt}{\isacharparenright}{\kern0pt}\ {\isadigit{4}}\ {\isadigit{4}}\ {\isacharequal}{\kern0pt}\ {\isadigit{3}}\ then\ {\isadigit{1}}\ else\ {\isadigit{0}}{\isacharparenright}{\kern0pt}\ {\isacharequal}{\kern0pt}\ {\isadigit{1}}\ else\ {\isacharparenleft}{\kern0pt}if\ nna\ {\isacharparenleft}{\kern0pt}{\isasymlambda}{\isacharparenleft}{\kern0pt}na{\isacharcomma}{\kern0pt}\ n{\isacharparenright}{\kern0pt}{\isachardot}{\kern0pt}\ cnj\ {\isacharparenleft}{\kern0pt}SWAP\ {\isachardollar}{\kern0pt}{\isachardollar}{\kern0pt}\ {\isacharparenleft}{\kern0pt}n{\isacharcomma}{\kern0pt}\ na{\isacharparenright}{\kern0pt}{\isacharparenright}{\kern0pt}{\isacharparenright}{\kern0pt}\ {\isacharparenleft}{\kern0pt}{\isasymlambda}{\isacharparenleft}{\kern0pt}na{\isacharcomma}{\kern0pt}\ n{\isacharparenright}{\kern0pt}{\isachardot}{\kern0pt}\ cnj\ {\isacharparenleft}{\kern0pt}SWAP\ {\isachardollar}{\kern0pt}{\isachardollar}{\kern0pt}\ {\isacharparenleft}{\kern0pt}na{\isacharcomma}{\kern0pt}\ n{\isacharparenright}{\kern0pt}{\isacharparenright}{\kern0pt}{\isacharparenright}{\kern0pt}\ {\isadigit{4}}\ {\isadigit{4}}\ {\isacharequal}{\kern0pt}\ {\isadigit{0}}\ {\isasymand}\ nn\ {\isacharparenleft}{\kern0pt}{\isasymlambda}{\isacharparenleft}{\kern0pt}na{\isacharcomma}{\kern0pt}\ n{\isacharparenright}{\kern0pt}{\isachardot}{\kern0pt}\ cnj\ {\isacharparenleft}{\kern0pt}SWAP\ {\isachardollar}{\kern0pt}{\isachardollar}{\kern0pt}\ {\isacharparenleft}{\kern0pt}n{\isacharcomma}{\kern0pt}\ na{\isacharparenright}{\kern0pt}{\isacharparenright}{\kern0pt}{\isacharparenright}{\kern0pt}\ {\isacharparenleft}{\kern0pt}{\isasymlambda}{\isacharparenleft}{\kern0pt}na{\isacharcomma}{\kern0pt}\ n{\isacharparenright}{\kern0pt}{\isachardot}{\kern0pt}\ cnj\ {\isacharparenleft}{\kern0pt}SWAP\ {\isachardollar}{\kern0pt}{\isachardollar}{\kern0pt}\ {\isacharparenleft}{\kern0pt}na{\isacharcomma}{\kern0pt}\ n{\isacharparenright}{\kern0pt}{\isacharparenright}{\kern0pt}{\isacharparenright}{\kern0pt}\ {\isadigit{4}}\ {\isadigit{4}}\ {\isacharequal}{\kern0pt}\ {\isadigit{0}}\ then\ {\isadigit{1}}{\isacharcolon}{\kern0pt}{\isacharcolon}{\kern0pt}complex\ else\ if\ nna\ {\isacharparenleft}{\kern0pt}{\isasymlambda}{\isacharparenleft}{\kern0pt}na{\isacharcomma}{\kern0pt}\ n{\isacharparenright}{\kern0pt}{\isachardot}{\kern0pt}\ cnj\ {\isacharparenleft}{\kern0pt}SWAP\ {\isachardollar}{\kern0pt}{\isachardollar}{\kern0pt}\ {\isacharparenleft}{\kern0pt}n{\isacharcomma}{\kern0pt}\ na{\isacharparenright}{\kern0pt}{\isacharparenright}{\kern0pt}{\isacharparenright}{\kern0pt}\ {\isacharparenleft}{\kern0pt}{\isasymlambda}{\isacharparenleft}{\kern0pt}na{\isacharcomma}{\kern0pt}\ n{\isacharparenright}{\kern0pt}{\isachardot}{\kern0pt}\ cnj\ {\isacharparenleft}{\kern0pt}SWAP\ {\isachardollar}{\kern0pt}{\isachardollar}{\kern0pt}\ {\isacharparenleft}{\kern0pt}na{\isacharcomma}{\kern0pt}\ n{\isacharparenright}{\kern0pt}{\isacharparenright}{\kern0pt}{\isacharparenright}{\kern0pt}\ {\isadigit{4}}\ {\isadigit{4}}\ {\isacharequal}{\kern0pt}\ {\isadigit{1}}\ {\isasymand}\ nn\ {\isacharparenleft}{\kern0pt}{\isasymlambda}{\isacharparenleft}{\kern0pt}na{\isacharcomma}{\kern0pt}\ n{\isacharparenright}{\kern0pt}{\isachardot}{\kern0pt}\ cnj\ {\isacharparenleft}{\kern0pt}SWAP\ {\isachardollar}{\kern0pt}{\isachardollar}{\kern0pt}\ {\isacharparenleft}{\kern0pt}n{\isacharcomma}{\kern0pt}\ na{\isacharparenright}{\kern0pt}{\isacharparenright}{\kern0pt}{\isacharparenright}{\kern0pt}\ {\isacharparenleft}{\kern0pt}{\isasymlambda}{\isacharparenleft}{\kern0pt}na{\isacharcomma}{\kern0pt}\ n{\isacharparenright}{\kern0pt}{\isachardot}{\kern0pt}\ cnj\ {\isacharparenleft}{\kern0pt}SWAP\ {\isachardollar}{\kern0pt}{\isachardollar}{\kern0pt}\ {\isacharparenleft}{\kern0pt}na{\isacharcomma}{\kern0pt}\ n{\isacharparenright}{\kern0pt}{\isacharparenright}{\kern0pt}{\isacharparenright}{\kern0pt}\ {\isadigit{4}}\ {\isadigit{4}}\ {\isacharequal}{\kern0pt}\ {\isadigit{2}}\ then\ {\isadigit{1}}\ else\ if\ nna\ {\isacharparenleft}{\kern0pt}{\isasymlambda}{\isacharparenleft}{\kern0pt}na{\isacharcomma}{\kern0pt}\ n{\isacharparenright}{\kern0pt}{\isachardot}{\kern0pt}\ cnj\ {\isacharparenleft}{\kern0pt}SWAP\ {\isachardollar}{\kern0pt}{\isachardollar}{\kern0pt}\ {\isacharparenleft}{\kern0pt}n{\isacharcomma}{\kern0pt}\ na{\isacharparenright}{\kern0pt}{\isacharparenright}{\kern0pt}{\isacharparenright}{\kern0pt}\ {\isacharparenleft}{\kern0pt}{\isasymlambda}{\isacharparenleft}{\kern0pt}na{\isacharcomma}{\kern0pt}\ n{\isacharparenright}{\kern0pt}{\isachardot}{\kern0pt}\ cnj\ {\isacharparenleft}{\kern0pt}SWAP\ {\isachardollar}{\kern0pt}{\isachardollar}{\kern0pt}\ {\isacharparenleft}{\kern0pt}na{\isacharcomma}{\kern0pt}\ n{\isacharparenright}{\kern0pt}{\isacharparenright}{\kern0pt}{\isacharparenright}{\kern0pt}\ {\isadigit{4}}\ {\isadigit{4}}\ {\isacharequal}{\kern0pt}\ {\isadigit{2}}\ {\isasymand}\ nn\ {\isacharparenleft}{\kern0pt}{\isasymlambda}{\isacharparenleft}{\kern0pt}na{\isacharcomma}{\kern0pt}\ n{\isacharparenright}{\kern0pt}{\isachardot}{\kern0pt}\ cnj\ {\isacharparenleft}{\kern0pt}SWAP\ {\isachardollar}{\kern0pt}{\isachardollar}{\kern0pt}\ {\isacharparenleft}{\kern0pt}n{\isacharcomma}{\kern0pt}\ na{\isacharparenright}{\kern0pt}{\isacharparenright}{\kern0pt}{\isacharparenright}{\kern0pt}\ {\isacharparenleft}{\kern0pt}{\isasymlambda}{\isacharparenleft}{\kern0pt}na{\isacharcomma}{\kern0pt}\ n{\isacharparenright}{\kern0pt}{\isachardot}{\kern0pt}\ cnj\ {\isacharparenleft}{\kern0pt}SWAP\ {\isachardollar}{\kern0pt}{\isachardollar}{\kern0pt}\ {\isacharparenleft}{\kern0pt}na{\isacharcomma}{\kern0pt}\ n{\isacharparenright}{\kern0pt}{\isacharparenright}{\kern0pt}{\isacharparenright}{\kern0pt}\ {\isadigit{4}}\ {\isadigit{4}}\ {\isacharequal}{\kern0pt}\ {\isadigit{1}}\ then\ {\isadigit{1}}\ else\ if\ nna\ {\isacharparenleft}{\kern0pt}{\isasymlambda}{\isacharparenleft}{\kern0pt}na{\isacharcomma}{\kern0pt}\ n{\isacharparenright}{\kern0pt}{\isachardot}{\kern0pt}\ cnj\ {\isacharparenleft}{\kern0pt}SWAP\ {\isachardollar}{\kern0pt}{\isachardollar}{\kern0pt}\ {\isacharparenleft}{\kern0pt}n{\isacharcomma}{\kern0pt}\ na{\isacharparenright}{\kern0pt}{\isacharparenright}{\kern0pt}{\isacharparenright}{\kern0pt}\ {\isacharparenleft}{\kern0pt}{\isasymlambda}{\isacharparenleft}{\kern0pt}na{\isacharcomma}{\kern0pt}\ n{\isacharparenright}{\kern0pt}{\isachardot}{\kern0pt}\ cnj\ {\isacharparenleft}{\kern0pt}SWAP\ {\isachardollar}{\kern0pt}{\isachardollar}{\kern0pt}\ {\isacharparenleft}{\kern0pt}na{\isacharcomma}{\kern0pt}\ n{\isacharparenright}{\kern0pt}{\isacharparenright}{\kern0pt}{\isacharparenright}{\kern0pt}\ {\isadigit{4}}\ {\isadigit{4}}\ {\isacharequal}{\kern0pt}\ {\isadigit{3}}\ {\isasymand}\ nn\ {\isacharparenleft}{\kern0pt}{\isasymlambda}{\isacharparenleft}{\kern0pt}na{\isacharcomma}{\kern0pt}\ n{\isacharparenright}{\kern0pt}{\isachardot}{\kern0pt}\ cnj\ {\isacharparenleft}{\kern0pt}SWAP\ {\isachardollar}{\kern0pt}{\isachardollar}{\kern0pt}\ {\isacharparenleft}{\kern0pt}n{\isacharcomma}{\kern0pt}\ na{\isacharparenright}{\kern0pt}{\isacharparenright}{\kern0pt}{\isacharparenright}{\kern0pt}\ {\isacharparenleft}{\kern0pt}{\isasymlambda}{\isacharparenleft}{\kern0pt}na{\isacharcomma}{\kern0pt}\ n{\isacharparenright}{\kern0pt}{\isachardot}{\kern0pt}\ cnj\ {\isacharparenleft}{\kern0pt}SWAP\ {\isachardollar}{\kern0pt}{\isachardollar}{\kern0pt}\ {\isacharparenleft}{\kern0pt}na{\isacharcomma}{\kern0pt}\ n{\isacharparenright}{\kern0pt}{\isacharparenright}{\kern0pt}{\isacharparenright}{\kern0pt}\ {\isadigit{4}}\ {\isadigit{4}}\ {\isacharequal}{\kern0pt}\ {\isadigit{3}}\ then\ {\isadigit{1}}\ else\ {\isadigit{0}}{\isacharparenright}{\kern0pt}\ {\isacharequal}{\kern0pt}\ {\isadigit{1}}{\isacharparenright}{\kern0pt}\ {\isasymlongrightarrow}\ {\isacharparenleft}{\kern0pt}if\ nna\ {\isacharparenleft}{\kern0pt}{\isasymlambda}{\isacharparenleft}{\kern0pt}na{\isacharcomma}{\kern0pt}\ n{\isacharparenright}{\kern0pt}{\isachardot}{\kern0pt}\ cnj\ {\isacharparenleft}{\kern0pt}SWAP\ {\isachardollar}{\kern0pt}{\isachardollar}{\kern0pt}\ {\isacharparenleft}{\kern0pt}n{\isacharcomma}{\kern0pt}\ na{\isacharparenright}{\kern0pt}{\isacharparenright}{\kern0pt}{\isacharparenright}{\kern0pt}\ {\isacharparenleft}{\kern0pt}{\isasymlambda}{\isacharparenleft}{\kern0pt}na{\isacharcomma}{\kern0pt}\ n{\isacharparenright}{\kern0pt}{\isachardot}{\kern0pt}\ cnj\ {\isacharparenleft}{\kern0pt}SWAP\ {\isachardollar}{\kern0pt}{\isachardollar}{\kern0pt}\ {\isacharparenleft}{\kern0pt}na{\isacharcomma}{\kern0pt}\ n{\isacharparenright}{\kern0pt}{\isacharparenright}{\kern0pt}{\isacharparenright}{\kern0pt}\ {\isadigit{4}}\ {\isadigit{4}}\ {\isacharequal}{\kern0pt}\ {\isadigit{0}}\ {\isasymand}\ nn\ {\isacharparenleft}{\kern0pt}{\isasymlambda}{\isacharparenleft}{\kern0pt}na{\isacharcomma}{\kern0pt}\ n{\isacharparenright}{\kern0pt}{\isachardot}{\kern0pt}\ cnj\ {\isacharparenleft}{\kern0pt}SWAP\ {\isachardollar}{\kern0pt}{\isachardollar}{\kern0pt}\ {\isacharparenleft}{\kern0pt}n{\isacharcomma}{\kern0pt}\ na{\isacharparenright}{\kern0pt}{\isacharparenright}{\kern0pt}{\isacharparenright}{\kern0pt}\ {\isacharparenleft}{\kern0pt}{\isasymlambda}{\isacharparenleft}{\kern0pt}na{\isacharcomma}{\kern0pt}\ n{\isacharparenright}{\kern0pt}{\isachardot}{\kern0pt}\ cnj\ {\isacharparenleft}{\kern0pt}SWAP\ {\isachardollar}{\kern0pt}{\isachardollar}{\kern0pt}\ {\isacharparenleft}{\kern0pt}na{\isacharcomma}{\kern0pt}\ n{\isacharparenright}{\kern0pt}{\isacharparenright}{\kern0pt}{\isacharparenright}{\kern0pt}\ {\isadigit{4}}\ {\isadigit{4}}\ {\isacharequal}{\kern0pt}\ {\isadigit{0}}\ then\ {\isadigit{1}}{\isacharcolon}{\kern0pt}{\isacharcolon}{\kern0pt}complex\ else\ if\ nna\ {\isacharparenleft}{\kern0pt}{\isasymlambda}{\isacharparenleft}{\kern0pt}na{\isacharcomma}{\kern0pt}\ n{\isacharparenright}{\kern0pt}{\isachardot}{\kern0pt}\ cnj\ {\isacharparenleft}{\kern0pt}SWAP\ {\isachardollar}{\kern0pt}{\isachardollar}{\kern0pt}\ {\isacharparenleft}{\kern0pt}n{\isacharcomma}{\kern0pt}\ na{\isacharparenright}{\kern0pt}{\isacharparenright}{\kern0pt}{\isacharparenright}{\kern0pt}\ {\isacharparenleft}{\kern0pt}{\isasymlambda}{\isacharparenleft}{\kern0pt}na{\isacharcomma}{\kern0pt}\ n{\isacharparenright}{\kern0pt}{\isachardot}{\kern0pt}\ cnj\ {\isacharparenleft}{\kern0pt}SWAP\ {\isachardollar}{\kern0pt}{\isachardollar}{\kern0pt}\ {\isacharparenleft}{\kern0pt}na{\isacharcomma}{\kern0pt}\ n{\isacharparenright}{\kern0pt}{\isacharparenright}{\kern0pt}{\isacharparenright}{\kern0pt}\ {\isadigit{4}}\ {\isadigit{4}}\ {\isacharequal}{\kern0pt}\ {\isadigit{1}}\ {\isasymand}\ nn\ {\isacharparenleft}{\kern0pt}{\isasymlambda}{\isacharparenleft}{\kern0pt}na{\isacharcomma}{\kern0pt}\ n{\isacharparenright}{\kern0pt}{\isachardot}{\kern0pt}\ cnj\ {\isacharparenleft}{\kern0pt}SWAP\ {\isachardollar}{\kern0pt}{\isachardollar}{\kern0pt}\ {\isacharparenleft}{\kern0pt}n{\isacharcomma}{\kern0pt}\ na{\isacharparenright}{\kern0pt}{\isacharparenright}{\kern0pt}{\isacharparenright}{\kern0pt}\ {\isacharparenleft}{\kern0pt}{\isasymlambda}{\isacharparenleft}{\kern0pt}na{\isacharcomma}{\kern0pt}\ n{\isacharparenright}{\kern0pt}{\isachardot}{\kern0pt}\ cnj\ {\isacharparenleft}{\kern0pt}SWAP\ {\isachardollar}{\kern0pt}{\isachardollar}{\kern0pt}\ {\isacharparenleft}{\kern0pt}na{\isacharcomma}{\kern0pt}\ n{\isacharparenright}{\kern0pt}{\isacharparenright}{\kern0pt}{\isacharparenright}{\kern0pt}\ {\isadigit{4}}\ {\isadigit{4}}\ {\isacharequal}{\kern0pt}\ {\isadigit{2}}\ then\ {\isadigit{1}}\ else\ if\ nna\ {\isacharparenleft}{\kern0pt}{\isasymlambda}{\isacharparenleft}{\kern0pt}na{\isacharcomma}{\kern0pt}\ n{\isacharparenright}{\kern0pt}{\isachardot}{\kern0pt}\ cnj\ {\isacharparenleft}{\kern0pt}SWAP\ {\isachardollar}{\kern0pt}{\isachardollar}{\kern0pt}\ {\isacharparenleft}{\kern0pt}n{\isacharcomma}{\kern0pt}\ na{\isacharparenright}{\kern0pt}{\isacharparenright}{\kern0pt}{\isacharparenright}{\kern0pt}\ {\isacharparenleft}{\kern0pt}{\isasymlambda}{\isacharparenleft}{\kern0pt}na{\isacharcomma}{\kern0pt}\ n{\isacharparenright}{\kern0pt}{\isachardot}{\kern0pt}\ cnj\ {\isacharparenleft}{\kern0pt}SWAP\ {\isachardollar}{\kern0pt}{\isachardollar}{\kern0pt}\ {\isacharparenleft}{\kern0pt}na{\isacharcomma}{\kern0pt}\ n{\isacharparenright}{\kern0pt}{\isacharparenright}{\kern0pt}{\isacharparenright}{\kern0pt}\ {\isadigit{4}}\ {\isadigit{4}}\ {\isacharequal}{\kern0pt}\ {\isadigit{2}}\ {\isasymand}\ nn\ {\isacharparenleft}{\kern0pt}{\isasymlambda}{\isacharparenleft}{\kern0pt}na{\isacharcomma}{\kern0pt}\ n{\isacharparenright}{\kern0pt}{\isachardot}{\kern0pt}\ cnj\ {\isacharparenleft}{\kern0pt}SWAP\ {\isachardollar}{\kern0pt}{\isachardollar}{\kern0pt}\ {\isacharparenleft}{\kern0pt}n{\isacharcomma}{\kern0pt}\ na{\isacharparenright}{\kern0pt}{\isacharparenright}{\kern0pt}{\isacharparenright}{\kern0pt}\ {\isacharparenleft}{\kern0pt}{\isasymlambda}{\isacharparenleft}{\kern0pt}na{\isacharcomma}{\kern0pt}\ n{\isacharparenright}{\kern0pt}{\isachardot}{\kern0pt}\ cnj\ {\isacharparenleft}{\kern0pt}SWAP\ {\isachardollar}{\kern0pt}{\isachardollar}{\kern0pt}\ {\isacharparenleft}{\kern0pt}na{\isacharcomma}{\kern0pt}\ n{\isacharparenright}{\kern0pt}{\isacharparenright}{\kern0pt}{\isacharparenright}{\kern0pt}\ {\isadigit{4}}\ {\isadigit{4}}\ {\isacharequal}{\kern0pt}\ {\isadigit{1}}\ then\ {\isadigit{1}}\ else\ if\ nna\ {\isacharparenleft}{\kern0pt}{\isasymlambda}{\isacharparenleft}{\kern0pt}na{\isacharcomma}{\kern0pt}\ n{\isacharparenright}{\kern0pt}{\isachardot}{\kern0pt}\ cnj\ {\isacharparenleft}{\kern0pt}SWAP\ {\isachardollar}{\kern0pt}{\isachardollar}{\kern0pt}\ {\isacharparenleft}{\kern0pt}n{\isacharcomma}{\kern0pt}\ na{\isacharparenright}{\kern0pt}{\isacharparenright}{\kern0pt}{\isacharparenright}{\kern0pt}\ {\isacharparenleft}{\kern0pt}{\isasymlambda}{\isacharparenleft}{\kern0pt}na{\isacharcomma}{\kern0pt}\ n{\isacharparenright}{\kern0pt}{\isachardot}{\kern0pt}\ cnj\ {\isacharparenleft}{\kern0pt}SWAP\ {\isachardollar}{\kern0pt}{\isachardollar}{\kern0pt}\ {\isacharparenleft}{\kern0pt}na{\isacharcomma}{\kern0pt}\ n{\isacharparenright}{\kern0pt}{\isacharparenright}{\kern0pt}{\isacharparenright}{\kern0pt}\ {\isadigit{4}}\ {\isadigit{4}}\ {\isacharequal}{\kern0pt}\ {\isadigit{3}}\ {\isasymand}\ nn\ {\isacharparenleft}{\kern0pt}{\isasymlambda}{\isacharparenleft}{\kern0pt}na{\isacharcomma}{\kern0pt}\ n{\isacharparenright}{\kern0pt}{\isachardot}{\kern0pt}\ cnj\ {\isacharparenleft}{\kern0pt}SWAP\ {\isachardollar}{\kern0pt}{\isachardollar}{\kern0pt}\ {\isacharparenleft}{\kern0pt}n{\isacharcomma}{\kern0pt}\ na{\isacharparenright}{\kern0pt}{\isacharparenright}{\kern0pt}{\isacharparenright}{\kern0pt}\ {\isacharparenleft}{\kern0pt}{\isasymlambda}{\isacharparenleft}{\kern0pt}na{\isacharcomma}{\kern0pt}\ n{\isacharparenright}{\kern0pt}{\isachardot}{\kern0pt}\ cnj\ {\isacharparenleft}{\kern0pt}SWAP\ {\isachardollar}{\kern0pt}{\isachardollar}{\kern0pt}\ {\isacharparenleft}{\kern0pt}na{\isacharcomma}{\kern0pt}\ n{\isacharparenright}{\kern0pt}{\isacharparenright}{\kern0pt}{\isacharparenright}{\kern0pt}\ {\isadigit{4}}\ {\isadigit{4}}\ {\isacharequal}{\kern0pt}\ {\isadigit{3}}\ then\ {\isadigit{1}}\ else\ {\isadigit{0}}{\isacharparenright}{\kern0pt}\ {\isacharequal}{\kern0pt}\ {\isadigit{1}}{\isachardoublequoteclose}\isanewline
\ \ \ \ \ \ \ \ \isacommand{by}\isamarkupfalse%
\ presburger\ \isacommand{{\isacharbraceright}{\kern0pt}}\isamarkupfalse%
\isanewline
\ \ \ \ \isacommand{moreover}\isamarkupfalse%
\isanewline
\ \ \ \ \isacommand{{\isacharbraceleft}{\kern0pt}}\isamarkupfalse%
\ \isacommand{assume}\isamarkupfalse%
\ {\isachardoublequoteopen}{\isacharparenleft}{\kern0pt}if\ nn\ {\isacharparenleft}{\kern0pt}{\isasymlambda}{\isacharparenleft}{\kern0pt}na{\isacharcomma}{\kern0pt}\ n{\isacharparenright}{\kern0pt}{\isachardot}{\kern0pt}\ cnj\ {\isacharparenleft}{\kern0pt}SWAP\ {\isachardollar}{\kern0pt}{\isachardollar}{\kern0pt}\ {\isacharparenleft}{\kern0pt}n{\isacharcomma}{\kern0pt}\ na{\isacharparenright}{\kern0pt}{\isacharparenright}{\kern0pt}{\isacharparenright}{\kern0pt}\ {\isacharparenleft}{\kern0pt}{\isasymlambda}{\isacharparenleft}{\kern0pt}na{\isacharcomma}{\kern0pt}\ n{\isacharparenright}{\kern0pt}{\isachardot}{\kern0pt}\ cnj\ {\isacharparenleft}{\kern0pt}SWAP\ {\isachardollar}{\kern0pt}{\isachardollar}{\kern0pt}\ {\isacharparenleft}{\kern0pt}na{\isacharcomma}{\kern0pt}\ n{\isacharparenright}{\kern0pt}{\isacharparenright}{\kern0pt}{\isacharparenright}{\kern0pt}\ {\isadigit{4}}\ {\isadigit{4}}\ {\isacharequal}{\kern0pt}\ {\isadigit{0}}\ {\isasymand}\ nna\ {\isacharparenleft}{\kern0pt}{\isasymlambda}{\isacharparenleft}{\kern0pt}na{\isacharcomma}{\kern0pt}\ n{\isacharparenright}{\kern0pt}{\isachardot}{\kern0pt}\ cnj\ {\isacharparenleft}{\kern0pt}SWAP\ {\isachardollar}{\kern0pt}{\isachardollar}{\kern0pt}\ {\isacharparenleft}{\kern0pt}n{\isacharcomma}{\kern0pt}\ na{\isacharparenright}{\kern0pt}{\isacharparenright}{\kern0pt}{\isacharparenright}{\kern0pt}\ {\isacharparenleft}{\kern0pt}{\isasymlambda}{\isacharparenleft}{\kern0pt}na{\isacharcomma}{\kern0pt}\ n{\isacharparenright}{\kern0pt}{\isachardot}{\kern0pt}\ cnj\ {\isacharparenleft}{\kern0pt}SWAP\ {\isachardollar}{\kern0pt}{\isachardollar}{\kern0pt}\ {\isacharparenleft}{\kern0pt}na{\isacharcomma}{\kern0pt}\ n{\isacharparenright}{\kern0pt}{\isacharparenright}{\kern0pt}{\isacharparenright}{\kern0pt}\ {\isadigit{4}}\ {\isadigit{4}}\ {\isacharequal}{\kern0pt}\ {\isadigit{0}}\ then\ {\isadigit{1}}{\isacharcolon}{\kern0pt}{\isacharcolon}{\kern0pt}complex\ else\ if\ nn\ {\isacharparenleft}{\kern0pt}{\isasymlambda}{\isacharparenleft}{\kern0pt}na{\isacharcomma}{\kern0pt}\ n{\isacharparenright}{\kern0pt}{\isachardot}{\kern0pt}\ cnj\ {\isacharparenleft}{\kern0pt}SWAP\ {\isachardollar}{\kern0pt}{\isachardollar}{\kern0pt}\ {\isacharparenleft}{\kern0pt}n{\isacharcomma}{\kern0pt}\ na{\isacharparenright}{\kern0pt}{\isacharparenright}{\kern0pt}{\isacharparenright}{\kern0pt}\ {\isacharparenleft}{\kern0pt}{\isasymlambda}{\isacharparenleft}{\kern0pt}na{\isacharcomma}{\kern0pt}\ n{\isacharparenright}{\kern0pt}{\isachardot}{\kern0pt}\ cnj\ {\isacharparenleft}{\kern0pt}SWAP\ {\isachardollar}{\kern0pt}{\isachardollar}{\kern0pt}\ {\isacharparenleft}{\kern0pt}na{\isacharcomma}{\kern0pt}\ n{\isacharparenright}{\kern0pt}{\isacharparenright}{\kern0pt}{\isacharparenright}{\kern0pt}\ {\isadigit{4}}\ {\isadigit{4}}\ {\isacharequal}{\kern0pt}\ {\isadigit{1}}\ {\isasymand}\ nna\ {\isacharparenleft}{\kern0pt}{\isasymlambda}{\isacharparenleft}{\kern0pt}na{\isacharcomma}{\kern0pt}\ n{\isacharparenright}{\kern0pt}{\isachardot}{\kern0pt}\ cnj\ {\isacharparenleft}{\kern0pt}SWAP\ {\isachardollar}{\kern0pt}{\isachardollar}{\kern0pt}\ {\isacharparenleft}{\kern0pt}n{\isacharcomma}{\kern0pt}\ na{\isacharparenright}{\kern0pt}{\isacharparenright}{\kern0pt}{\isacharparenright}{\kern0pt}\ {\isacharparenleft}{\kern0pt}{\isasymlambda}{\isacharparenleft}{\kern0pt}na{\isacharcomma}{\kern0pt}\ n{\isacharparenright}{\kern0pt}{\isachardot}{\kern0pt}\ cnj\ {\isacharparenleft}{\kern0pt}SWAP\ {\isachardollar}{\kern0pt}{\isachardollar}{\kern0pt}\ {\isacharparenleft}{\kern0pt}na{\isacharcomma}{\kern0pt}\ n{\isacharparenright}{\kern0pt}{\isacharparenright}{\kern0pt}{\isacharparenright}{\kern0pt}\ {\isadigit{4}}\ {\isadigit{4}}\ {\isacharequal}{\kern0pt}\ {\isadigit{2}}\ then\ {\isadigit{1}}\ else\ if\ nn\ {\isacharparenleft}{\kern0pt}{\isasymlambda}{\isacharparenleft}{\kern0pt}na{\isacharcomma}{\kern0pt}\ n{\isacharparenright}{\kern0pt}{\isachardot}{\kern0pt}\ cnj\ {\isacharparenleft}{\kern0pt}SWAP\ {\isachardollar}{\kern0pt}{\isachardollar}{\kern0pt}\ {\isacharparenleft}{\kern0pt}n{\isacharcomma}{\kern0pt}\ na{\isacharparenright}{\kern0pt}{\isacharparenright}{\kern0pt}{\isacharparenright}{\kern0pt}\ {\isacharparenleft}{\kern0pt}{\isasymlambda}{\isacharparenleft}{\kern0pt}na{\isacharcomma}{\kern0pt}\ n{\isacharparenright}{\kern0pt}{\isachardot}{\kern0pt}\ cnj\ {\isacharparenleft}{\kern0pt}SWAP\ {\isachardollar}{\kern0pt}{\isachardollar}{\kern0pt}\ {\isacharparenleft}{\kern0pt}na{\isacharcomma}{\kern0pt}\ n{\isacharparenright}{\kern0pt}{\isacharparenright}{\kern0pt}{\isacharparenright}{\kern0pt}\ {\isadigit{4}}\ {\isadigit{4}}\ {\isacharequal}{\kern0pt}\ {\isadigit{2}}\ {\isasymand}\ nna\ {\isacharparenleft}{\kern0pt}{\isasymlambda}{\isacharparenleft}{\kern0pt}na{\isacharcomma}{\kern0pt}\ n{\isacharparenright}{\kern0pt}{\isachardot}{\kern0pt}\ cnj\ {\isacharparenleft}{\kern0pt}SWAP\ {\isachardollar}{\kern0pt}{\isachardollar}{\kern0pt}\ {\isacharparenleft}{\kern0pt}n{\isacharcomma}{\kern0pt}\ na{\isacharparenright}{\kern0pt}{\isacharparenright}{\kern0pt}{\isacharparenright}{\kern0pt}\ {\isacharparenleft}{\kern0pt}{\isasymlambda}{\isacharparenleft}{\kern0pt}na{\isacharcomma}{\kern0pt}\ n{\isacharparenright}{\kern0pt}{\isachardot}{\kern0pt}\ cnj\ {\isacharparenleft}{\kern0pt}SWAP\ {\isachardollar}{\kern0pt}{\isachardollar}{\kern0pt}\ {\isacharparenleft}{\kern0pt}na{\isacharcomma}{\kern0pt}\ n{\isacharparenright}{\kern0pt}{\isacharparenright}{\kern0pt}{\isacharparenright}{\kern0pt}\ {\isadigit{4}}\ {\isadigit{4}}\ {\isacharequal}{\kern0pt}\ {\isadigit{1}}\ then\ {\isadigit{1}}\ else\ if\ nn\ {\isacharparenleft}{\kern0pt}{\isasymlambda}{\isacharparenleft}{\kern0pt}na{\isacharcomma}{\kern0pt}\ n{\isacharparenright}{\kern0pt}{\isachardot}{\kern0pt}\ cnj\ {\isacharparenleft}{\kern0pt}SWAP\ {\isachardollar}{\kern0pt}{\isachardollar}{\kern0pt}\ {\isacharparenleft}{\kern0pt}n{\isacharcomma}{\kern0pt}\ na{\isacharparenright}{\kern0pt}{\isacharparenright}{\kern0pt}{\isacharparenright}{\kern0pt}\ {\isacharparenleft}{\kern0pt}{\isasymlambda}{\isacharparenleft}{\kern0pt}na{\isacharcomma}{\kern0pt}\ n{\isacharparenright}{\kern0pt}{\isachardot}{\kern0pt}\ cnj\ {\isacharparenleft}{\kern0pt}SWAP\ {\isachardollar}{\kern0pt}{\isachardollar}{\kern0pt}\ {\isacharparenleft}{\kern0pt}na{\isacharcomma}{\kern0pt}\ n{\isacharparenright}{\kern0pt}{\isacharparenright}{\kern0pt}{\isacharparenright}{\kern0pt}\ {\isadigit{4}}\ {\isadigit{4}}\ {\isacharequal}{\kern0pt}\ {\isadigit{3}}\ {\isasymand}\ nna\ {\isacharparenleft}{\kern0pt}{\isasymlambda}{\isacharparenleft}{\kern0pt}na{\isacharcomma}{\kern0pt}\ n{\isacharparenright}{\kern0pt}{\isachardot}{\kern0pt}\ cnj\ {\isacharparenleft}{\kern0pt}SWAP\ {\isachardollar}{\kern0pt}{\isachardollar}{\kern0pt}\ {\isacharparenleft}{\kern0pt}n{\isacharcomma}{\kern0pt}\ na{\isacharparenright}{\kern0pt}{\isacharparenright}{\kern0pt}{\isacharparenright}{\kern0pt}\ {\isacharparenleft}{\kern0pt}{\isasymlambda}{\isacharparenleft}{\kern0pt}na{\isacharcomma}{\kern0pt}\ n{\isacharparenright}{\kern0pt}{\isachardot}{\kern0pt}\ cnj\ {\isacharparenleft}{\kern0pt}SWAP\ {\isachardollar}{\kern0pt}{\isachardollar}{\kern0pt}\ {\isacharparenleft}{\kern0pt}na{\isacharcomma}{\kern0pt}\ n{\isacharparenright}{\kern0pt}{\isacharparenright}{\kern0pt}{\isacharparenright}{\kern0pt}\ {\isadigit{4}}\ {\isadigit{4}}\ {\isacharequal}{\kern0pt}\ {\isadigit{3}}\ then\ {\isadigit{1}}\ else\ {\isadigit{0}}{\isacharparenright}{\kern0pt}\ {\isacharequal}{\kern0pt}\ {\isadigit{0}}\ {\isasymand}\ {\isacharparenleft}{\kern0pt}if\ nna\ {\isacharparenleft}{\kern0pt}{\isasymlambda}{\isacharparenleft}{\kern0pt}na{\isacharcomma}{\kern0pt}\ n{\isacharparenright}{\kern0pt}{\isachardot}{\kern0pt}\ cnj\ {\isacharparenleft}{\kern0pt}SWAP\ {\isachardollar}{\kern0pt}{\isachardollar}{\kern0pt}\ {\isacharparenleft}{\kern0pt}n{\isacharcomma}{\kern0pt}\ na{\isacharparenright}{\kern0pt}{\isacharparenright}{\kern0pt}{\isacharparenright}{\kern0pt}\ {\isacharparenleft}{\kern0pt}{\isasymlambda}{\isacharparenleft}{\kern0pt}na{\isacharcomma}{\kern0pt}\ n{\isacharparenright}{\kern0pt}{\isachardot}{\kern0pt}\ cnj\ {\isacharparenleft}{\kern0pt}SWAP\ {\isachardollar}{\kern0pt}{\isachardollar}{\kern0pt}\ {\isacharparenleft}{\kern0pt}na{\isacharcomma}{\kern0pt}\ n{\isacharparenright}{\kern0pt}{\isacharparenright}{\kern0pt}{\isacharparenright}{\kern0pt}\ {\isadigit{4}}\ {\isadigit{4}}\ {\isacharequal}{\kern0pt}\ {\isadigit{0}}\ {\isasymand}\ nn\ {\isacharparenleft}{\kern0pt}{\isasymlambda}{\isacharparenleft}{\kern0pt}na{\isacharcomma}{\kern0pt}\ n{\isacharparenright}{\kern0pt}{\isachardot}{\kern0pt}\ cnj\ {\isacharparenleft}{\kern0pt}SWAP\ {\isachardollar}{\kern0pt}{\isachardollar}{\kern0pt}\ {\isacharparenleft}{\kern0pt}n{\isacharcomma}{\kern0pt}\ na{\isacharparenright}{\kern0pt}{\isacharparenright}{\kern0pt}{\isacharparenright}{\kern0pt}\ {\isacharparenleft}{\kern0pt}{\isasymlambda}{\isacharparenleft}{\kern0pt}na{\isacharcomma}{\kern0pt}\ n{\isacharparenright}{\kern0pt}{\isachardot}{\kern0pt}\ cnj\ {\isacharparenleft}{\kern0pt}SWAP\ {\isachardollar}{\kern0pt}{\isachardollar}{\kern0pt}\ {\isacharparenleft}{\kern0pt}na{\isacharcomma}{\kern0pt}\ n{\isacharparenright}{\kern0pt}{\isacharparenright}{\kern0pt}{\isacharparenright}{\kern0pt}\ {\isadigit{4}}\ {\isadigit{4}}\ {\isacharequal}{\kern0pt}\ {\isadigit{0}}\ then\ {\isadigit{1}}{\isacharcolon}{\kern0pt}{\isacharcolon}{\kern0pt}complex\ else\ if\ nna\ {\isacharparenleft}{\kern0pt}{\isasymlambda}{\isacharparenleft}{\kern0pt}na{\isacharcomma}{\kern0pt}\ n{\isacharparenright}{\kern0pt}{\isachardot}{\kern0pt}\ cnj\ {\isacharparenleft}{\kern0pt}SWAP\ {\isachardollar}{\kern0pt}{\isachardollar}{\kern0pt}\ {\isacharparenleft}{\kern0pt}n{\isacharcomma}{\kern0pt}\ na{\isacharparenright}{\kern0pt}{\isacharparenright}{\kern0pt}{\isacharparenright}{\kern0pt}\ {\isacharparenleft}{\kern0pt}{\isasymlambda}{\isacharparenleft}{\kern0pt}na{\isacharcomma}{\kern0pt}\ n{\isacharparenright}{\kern0pt}{\isachardot}{\kern0pt}\ cnj\ {\isacharparenleft}{\kern0pt}SWAP\ {\isachardollar}{\kern0pt}{\isachardollar}{\kern0pt}\ {\isacharparenleft}{\kern0pt}na{\isacharcomma}{\kern0pt}\ n{\isacharparenright}{\kern0pt}{\isacharparenright}{\kern0pt}{\isacharparenright}{\kern0pt}\ {\isadigit{4}}\ {\isadigit{4}}\ {\isacharequal}{\kern0pt}\ {\isadigit{1}}\ {\isasymand}\ nn\ {\isacharparenleft}{\kern0pt}{\isasymlambda}{\isacharparenleft}{\kern0pt}na{\isacharcomma}{\kern0pt}\ n{\isacharparenright}{\kern0pt}{\isachardot}{\kern0pt}\ cnj\ {\isacharparenleft}{\kern0pt}SWAP\ {\isachardollar}{\kern0pt}{\isachardollar}{\kern0pt}\ {\isacharparenleft}{\kern0pt}n{\isacharcomma}{\kern0pt}\ na{\isacharparenright}{\kern0pt}{\isacharparenright}{\kern0pt}{\isacharparenright}{\kern0pt}\ {\isacharparenleft}{\kern0pt}{\isasymlambda}{\isacharparenleft}{\kern0pt}na{\isacharcomma}{\kern0pt}\ n{\isacharparenright}{\kern0pt}{\isachardot}{\kern0pt}\ cnj\ {\isacharparenleft}{\kern0pt}SWAP\ {\isachardollar}{\kern0pt}{\isachardollar}{\kern0pt}\ {\isacharparenleft}{\kern0pt}na{\isacharcomma}{\kern0pt}\ n{\isacharparenright}{\kern0pt}{\isacharparenright}{\kern0pt}{\isacharparenright}{\kern0pt}\ {\isadigit{4}}\ {\isadigit{4}}\ {\isacharequal}{\kern0pt}\ {\isadigit{2}}\ then\ {\isadigit{1}}\ else\ if\ nna\ {\isacharparenleft}{\kern0pt}{\isasymlambda}{\isacharparenleft}{\kern0pt}na{\isacharcomma}{\kern0pt}\ n{\isacharparenright}{\kern0pt}{\isachardot}{\kern0pt}\ cnj\ {\isacharparenleft}{\kern0pt}SWAP\ {\isachardollar}{\kern0pt}{\isachardollar}{\kern0pt}\ {\isacharparenleft}{\kern0pt}n{\isacharcomma}{\kern0pt}\ na{\isacharparenright}{\kern0pt}{\isacharparenright}{\kern0pt}{\isacharparenright}{\kern0pt}\ {\isacharparenleft}{\kern0pt}{\isasymlambda}{\isacharparenleft}{\kern0pt}na{\isacharcomma}{\kern0pt}\ n{\isacharparenright}{\kern0pt}{\isachardot}{\kern0pt}\ cnj\ {\isacharparenleft}{\kern0pt}SWAP\ {\isachardollar}{\kern0pt}{\isachardollar}{\kern0pt}\ {\isacharparenleft}{\kern0pt}na{\isacharcomma}{\kern0pt}\ n{\isacharparenright}{\kern0pt}{\isacharparenright}{\kern0pt}{\isacharparenright}{\kern0pt}\ {\isadigit{4}}\ {\isadigit{4}}\ {\isacharequal}{\kern0pt}\ {\isadigit{2}}\ {\isasymand}\ nn\ {\isacharparenleft}{\kern0pt}{\isasymlambda}{\isacharparenleft}{\kern0pt}na{\isacharcomma}{\kern0pt}\ n{\isacharparenright}{\kern0pt}{\isachardot}{\kern0pt}\ cnj\ {\isacharparenleft}{\kern0pt}SWAP\ {\isachardollar}{\kern0pt}{\isachardollar}{\kern0pt}\ {\isacharparenleft}{\kern0pt}n{\isacharcomma}{\kern0pt}\ na{\isacharparenright}{\kern0pt}{\isacharparenright}{\kern0pt}{\isacharparenright}{\kern0pt}\ {\isacharparenleft}{\kern0pt}{\isasymlambda}{\isacharparenleft}{\kern0pt}na{\isacharcomma}{\kern0pt}\ n{\isacharparenright}{\kern0pt}{\isachardot}{\kern0pt}\ cnj\ {\isacharparenleft}{\kern0pt}SWAP\ {\isachardollar}{\kern0pt}{\isachardollar}{\kern0pt}\ {\isacharparenleft}{\kern0pt}na{\isacharcomma}{\kern0pt}\ n{\isacharparenright}{\kern0pt}{\isacharparenright}{\kern0pt}{\isacharparenright}{\kern0pt}\ {\isadigit{4}}\ {\isadigit{4}}\ {\isacharequal}{\kern0pt}\ {\isadigit{1}}\ then\ {\isadigit{1}}\ else\ if\ nna\ {\isacharparenleft}{\kern0pt}{\isasymlambda}{\isacharparenleft}{\kern0pt}na{\isacharcomma}{\kern0pt}\ n{\isacharparenright}{\kern0pt}{\isachardot}{\kern0pt}\ cnj\ {\isacharparenleft}{\kern0pt}SWAP\ {\isachardollar}{\kern0pt}{\isachardollar}{\kern0pt}\ {\isacharparenleft}{\kern0pt}n{\isacharcomma}{\kern0pt}\ na{\isacharparenright}{\kern0pt}{\isacharparenright}{\kern0pt}{\isacharparenright}{\kern0pt}\ {\isacharparenleft}{\kern0pt}{\isasymlambda}{\isacharparenleft}{\kern0pt}na{\isacharcomma}{\kern0pt}\ n{\isacharparenright}{\kern0pt}{\isachardot}{\kern0pt}\ cnj\ {\isacharparenleft}{\kern0pt}SWAP\ {\isachardollar}{\kern0pt}{\isachardollar}{\kern0pt}\ {\isacharparenleft}{\kern0pt}na{\isacharcomma}{\kern0pt}\ n{\isacharparenright}{\kern0pt}{\isacharparenright}{\kern0pt}{\isacharparenright}{\kern0pt}\ {\isadigit{4}}\ {\isadigit{4}}\ {\isacharequal}{\kern0pt}\ {\isadigit{3}}\ {\isasymand}\ nn\ {\isacharparenleft}{\kern0pt}{\isasymlambda}{\isacharparenleft}{\kern0pt}na{\isacharcomma}{\kern0pt}\ n{\isacharparenright}{\kern0pt}{\isachardot}{\kern0pt}\ cnj\ {\isacharparenleft}{\kern0pt}SWAP\ {\isachardollar}{\kern0pt}{\isachardollar}{\kern0pt}\ {\isacharparenleft}{\kern0pt}n{\isacharcomma}{\kern0pt}\ na{\isacharparenright}{\kern0pt}{\isacharparenright}{\kern0pt}{\isacharparenright}{\kern0pt}\ {\isacharparenleft}{\kern0pt}{\isasymlambda}{\isacharparenleft}{\kern0pt}na{\isacharcomma}{\kern0pt}\ n{\isacharparenright}{\kern0pt}{\isachardot}{\kern0pt}\ cnj\ {\isacharparenleft}{\kern0pt}SWAP\ {\isachardollar}{\kern0pt}{\isachardollar}{\kern0pt}\ {\isacharparenleft}{\kern0pt}na{\isacharcomma}{\kern0pt}\ n{\isacharparenright}{\kern0pt}{\isacharparenright}{\kern0pt}{\isacharparenright}{\kern0pt}\ {\isadigit{4}}\ {\isadigit{4}}\ {\isacharequal}{\kern0pt}\ {\isadigit{3}}\ then\ {\isadigit{1}}\ else\ {\isadigit{0}}{\isacharparenright}{\kern0pt}\ {\isacharequal}{\kern0pt}\ {\isadigit{0}}{\isachardoublequoteclose}\isanewline
\ \ \ \ \ \ \isacommand{moreover}\isamarkupfalse%
\isanewline
\ \ \ \ \ \ \isacommand{{\isacharbraceleft}{\kern0pt}}\isamarkupfalse%
\ \isacommand{assume}\isamarkupfalse%
\ {\isachardoublequoteopen}{\isacharparenleft}{\kern0pt}{\isacharparenleft}{\kern0pt}if\ nn\ {\isacharparenleft}{\kern0pt}{\isasymlambda}{\isacharparenleft}{\kern0pt}n{\isacharcomma}{\kern0pt}\ na{\isacharparenright}{\kern0pt}{\isachardot}{\kern0pt}\ cnj\ {\isacharparenleft}{\kern0pt}SWAP\ {\isachardollar}{\kern0pt}{\isachardollar}{\kern0pt}\ {\isacharparenleft}{\kern0pt}na{\isacharcomma}{\kern0pt}\ n{\isacharparenright}{\kern0pt}{\isacharparenright}{\kern0pt}{\isacharparenright}{\kern0pt}\ {\isacharparenleft}{\kern0pt}{\isasymlambda}{\isacharparenleft}{\kern0pt}n{\isacharcomma}{\kern0pt}\ na{\isacharparenright}{\kern0pt}{\isachardot}{\kern0pt}\ cnj\ {\isacharparenleft}{\kern0pt}SWAP\ {\isachardollar}{\kern0pt}{\isachardollar}{\kern0pt}\ {\isacharparenleft}{\kern0pt}n{\isacharcomma}{\kern0pt}\ na{\isacharparenright}{\kern0pt}{\isacharparenright}{\kern0pt}{\isacharparenright}{\kern0pt}\ {\isadigit{4}}\ {\isadigit{4}}\ {\isacharequal}{\kern0pt}\ {\isadigit{0}}\ {\isasymand}\ nna\ {\isacharparenleft}{\kern0pt}{\isasymlambda}{\isacharparenleft}{\kern0pt}n{\isacharcomma}{\kern0pt}\ na{\isacharparenright}{\kern0pt}{\isachardot}{\kern0pt}\ cnj\ {\isacharparenleft}{\kern0pt}SWAP\ {\isachardollar}{\kern0pt}{\isachardollar}{\kern0pt}\ {\isacharparenleft}{\kern0pt}na{\isacharcomma}{\kern0pt}\ n{\isacharparenright}{\kern0pt}{\isacharparenright}{\kern0pt}{\isacharparenright}{\kern0pt}\ {\isacharparenleft}{\kern0pt}{\isasymlambda}{\isacharparenleft}{\kern0pt}n{\isacharcomma}{\kern0pt}\ na{\isacharparenright}{\kern0pt}{\isachardot}{\kern0pt}\ cnj\ {\isacharparenleft}{\kern0pt}SWAP\ {\isachardollar}{\kern0pt}{\isachardollar}{\kern0pt}\ {\isacharparenleft}{\kern0pt}n{\isacharcomma}{\kern0pt}\ na{\isacharparenright}{\kern0pt}{\isacharparenright}{\kern0pt}{\isacharparenright}{\kern0pt}\ {\isadigit{4}}\ {\isadigit{4}}\ {\isacharequal}{\kern0pt}\ {\isadigit{0}}\ then\ {\isadigit{1}}{\isacharcolon}{\kern0pt}{\isacharcolon}{\kern0pt}complex\ else\ if\ nn\ {\isacharparenleft}{\kern0pt}{\isasymlambda}{\isacharparenleft}{\kern0pt}n{\isacharcomma}{\kern0pt}\ na{\isacharparenright}{\kern0pt}{\isachardot}{\kern0pt}\ cnj\ {\isacharparenleft}{\kern0pt}SWAP\ {\isachardollar}{\kern0pt}{\isachardollar}{\kern0pt}\ {\isacharparenleft}{\kern0pt}na{\isacharcomma}{\kern0pt}\ n{\isacharparenright}{\kern0pt}{\isacharparenright}{\kern0pt}{\isacharparenright}{\kern0pt}\ {\isacharparenleft}{\kern0pt}{\isasymlambda}{\isacharparenleft}{\kern0pt}n{\isacharcomma}{\kern0pt}\ na{\isacharparenright}{\kern0pt}{\isachardot}{\kern0pt}\ cnj\ {\isacharparenleft}{\kern0pt}SWAP\ {\isachardollar}{\kern0pt}{\isachardollar}{\kern0pt}\ {\isacharparenleft}{\kern0pt}n{\isacharcomma}{\kern0pt}\ na{\isacharparenright}{\kern0pt}{\isacharparenright}{\kern0pt}{\isacharparenright}{\kern0pt}\ {\isadigit{4}}\ {\isadigit{4}}\ {\isacharequal}{\kern0pt}\ {\isadigit{1}}\ {\isasymand}\ nna\ {\isacharparenleft}{\kern0pt}{\isasymlambda}{\isacharparenleft}{\kern0pt}n{\isacharcomma}{\kern0pt}\ na{\isacharparenright}{\kern0pt}{\isachardot}{\kern0pt}\ cnj\ {\isacharparenleft}{\kern0pt}SWAP\ {\isachardollar}{\kern0pt}{\isachardollar}{\kern0pt}\ {\isacharparenleft}{\kern0pt}na{\isacharcomma}{\kern0pt}\ n{\isacharparenright}{\kern0pt}{\isacharparenright}{\kern0pt}{\isacharparenright}{\kern0pt}\ {\isacharparenleft}{\kern0pt}{\isasymlambda}{\isacharparenleft}{\kern0pt}n{\isacharcomma}{\kern0pt}\ na{\isacharparenright}{\kern0pt}{\isachardot}{\kern0pt}\ cnj\ {\isacharparenleft}{\kern0pt}SWAP\ {\isachardollar}{\kern0pt}{\isachardollar}{\kern0pt}\ {\isacharparenleft}{\kern0pt}n{\isacharcomma}{\kern0pt}\ na{\isacharparenright}{\kern0pt}{\isacharparenright}{\kern0pt}{\isacharparenright}{\kern0pt}\ {\isadigit{4}}\ {\isadigit{4}}\ {\isacharequal}{\kern0pt}\ {\isadigit{2}}\ then\ {\isadigit{1}}\ else\ if\ nn\ {\isacharparenleft}{\kern0pt}{\isasymlambda}{\isacharparenleft}{\kern0pt}n{\isacharcomma}{\kern0pt}\ na{\isacharparenright}{\kern0pt}{\isachardot}{\kern0pt}\ cnj\ {\isacharparenleft}{\kern0pt}SWAP\ {\isachardollar}{\kern0pt}{\isachardollar}{\kern0pt}\ {\isacharparenleft}{\kern0pt}na{\isacharcomma}{\kern0pt}\ n{\isacharparenright}{\kern0pt}{\isacharparenright}{\kern0pt}{\isacharparenright}{\kern0pt}\ {\isacharparenleft}{\kern0pt}{\isasymlambda}{\isacharparenleft}{\kern0pt}n{\isacharcomma}{\kern0pt}\ na{\isacharparenright}{\kern0pt}{\isachardot}{\kern0pt}\ cnj\ {\isacharparenleft}{\kern0pt}SWAP\ {\isachardollar}{\kern0pt}{\isachardollar}{\kern0pt}\ {\isacharparenleft}{\kern0pt}n{\isacharcomma}{\kern0pt}\ na{\isacharparenright}{\kern0pt}{\isacharparenright}{\kern0pt}{\isacharparenright}{\kern0pt}\ {\isadigit{4}}\ {\isadigit{4}}\ {\isacharequal}{\kern0pt}\ {\isadigit{2}}\ {\isasymand}\ nna\ {\isacharparenleft}{\kern0pt}{\isasymlambda}{\isacharparenleft}{\kern0pt}n{\isacharcomma}{\kern0pt}\ na{\isacharparenright}{\kern0pt}{\isachardot}{\kern0pt}\ cnj\ {\isacharparenleft}{\kern0pt}SWAP\ {\isachardollar}{\kern0pt}{\isachardollar}{\kern0pt}\ {\isacharparenleft}{\kern0pt}na{\isacharcomma}{\kern0pt}\ n{\isacharparenright}{\kern0pt}{\isacharparenright}{\kern0pt}{\isacharparenright}{\kern0pt}\ {\isacharparenleft}{\kern0pt}{\isasymlambda}{\isacharparenleft}{\kern0pt}n{\isacharcomma}{\kern0pt}\ na{\isacharparenright}{\kern0pt}{\isachardot}{\kern0pt}\ cnj\ {\isacharparenleft}{\kern0pt}SWAP\ {\isachardollar}{\kern0pt}{\isachardollar}{\kern0pt}\ {\isacharparenleft}{\kern0pt}n{\isacharcomma}{\kern0pt}\ na{\isacharparenright}{\kern0pt}{\isacharparenright}{\kern0pt}{\isacharparenright}{\kern0pt}\ {\isadigit{4}}\ {\isadigit{4}}\ {\isacharequal}{\kern0pt}\ {\isadigit{1}}\ then\ {\isadigit{1}}\ else\ if\ nn\ {\isacharparenleft}{\kern0pt}{\isasymlambda}{\isacharparenleft}{\kern0pt}n{\isacharcomma}{\kern0pt}\ na{\isacharparenright}{\kern0pt}{\isachardot}{\kern0pt}\ cnj\ {\isacharparenleft}{\kern0pt}SWAP\ {\isachardollar}{\kern0pt}{\isachardollar}{\kern0pt}\ {\isacharparenleft}{\kern0pt}na{\isacharcomma}{\kern0pt}\ n{\isacharparenright}{\kern0pt}{\isacharparenright}{\kern0pt}{\isacharparenright}{\kern0pt}\ {\isacharparenleft}{\kern0pt}{\isasymlambda}{\isacharparenleft}{\kern0pt}n{\isacharcomma}{\kern0pt}\ na{\isacharparenright}{\kern0pt}{\isachardot}{\kern0pt}\ cnj\ {\isacharparenleft}{\kern0pt}SWAP\ {\isachardollar}{\kern0pt}{\isachardollar}{\kern0pt}\ {\isacharparenleft}{\kern0pt}n{\isacharcomma}{\kern0pt}\ na{\isacharparenright}{\kern0pt}{\isacharparenright}{\kern0pt}{\isacharparenright}{\kern0pt}\ {\isadigit{4}}\ {\isadigit{4}}\ {\isacharequal}{\kern0pt}\ {\isadigit{3}}\ {\isasymand}\ nna\ {\isacharparenleft}{\kern0pt}{\isasymlambda}{\isacharparenleft}{\kern0pt}n{\isacharcomma}{\kern0pt}\ na{\isacharparenright}{\kern0pt}{\isachardot}{\kern0pt}\ cnj\ {\isacharparenleft}{\kern0pt}SWAP\ {\isachardollar}{\kern0pt}{\isachardollar}{\kern0pt}\ {\isacharparenleft}{\kern0pt}na{\isacharcomma}{\kern0pt}\ n{\isacharparenright}{\kern0pt}{\isacharparenright}{\kern0pt}{\isacharparenright}{\kern0pt}\ {\isacharparenleft}{\kern0pt}{\isasymlambda}{\isacharparenleft}{\kern0pt}n{\isacharcomma}{\kern0pt}\ na{\isacharparenright}{\kern0pt}{\isachardot}{\kern0pt}\ cnj\ {\isacharparenleft}{\kern0pt}SWAP\ {\isachardollar}{\kern0pt}{\isachardollar}{\kern0pt}\ {\isacharparenleft}{\kern0pt}n{\isacharcomma}{\kern0pt}\ na{\isacharparenright}{\kern0pt}{\isacharparenright}{\kern0pt}{\isacharparenright}{\kern0pt}\ {\isadigit{4}}\ {\isadigit{4}}\ {\isacharequal}{\kern0pt}\ {\isadigit{3}}\ then\ {\isadigit{1}}\ else\ {\isadigit{0}}{\isacharparenright}{\kern0pt}\ {\isacharequal}{\kern0pt}\ {\isadigit{0}}\ {\isasymand}\ {\isacharparenleft}{\kern0pt}if\ nna\ {\isacharparenleft}{\kern0pt}{\isasymlambda}{\isacharparenleft}{\kern0pt}n{\isacharcomma}{\kern0pt}\ na{\isacharparenright}{\kern0pt}{\isachardot}{\kern0pt}\ cnj\ {\isacharparenleft}{\kern0pt}SWAP\ {\isachardollar}{\kern0pt}{\isachardollar}{\kern0pt}\ {\isacharparenleft}{\kern0pt}na{\isacharcomma}{\kern0pt}\ n{\isacharparenright}{\kern0pt}{\isacharparenright}{\kern0pt}{\isacharparenright}{\kern0pt}\ {\isacharparenleft}{\kern0pt}{\isasymlambda}{\isacharparenleft}{\kern0pt}n{\isacharcomma}{\kern0pt}\ na{\isacharparenright}{\kern0pt}{\isachardot}{\kern0pt}\ cnj\ {\isacharparenleft}{\kern0pt}SWAP\ {\isachardollar}{\kern0pt}{\isachardollar}{\kern0pt}\ {\isacharparenleft}{\kern0pt}n{\isacharcomma}{\kern0pt}\ na{\isacharparenright}{\kern0pt}{\isacharparenright}{\kern0pt}{\isacharparenright}{\kern0pt}\ {\isadigit{4}}\ {\isadigit{4}}\ {\isacharequal}{\kern0pt}\ {\isadigit{0}}\ {\isasymand}\ nn\ {\isacharparenleft}{\kern0pt}{\isasymlambda}{\isacharparenleft}{\kern0pt}n{\isacharcomma}{\kern0pt}\ na{\isacharparenright}{\kern0pt}{\isachardot}{\kern0pt}\ cnj\ {\isacharparenleft}{\kern0pt}SWAP\ {\isachardollar}{\kern0pt}{\isachardollar}{\kern0pt}\ {\isacharparenleft}{\kern0pt}na{\isacharcomma}{\kern0pt}\ n{\isacharparenright}{\kern0pt}{\isacharparenright}{\kern0pt}{\isacharparenright}{\kern0pt}\ {\isacharparenleft}{\kern0pt}{\isasymlambda}{\isacharparenleft}{\kern0pt}n{\isacharcomma}{\kern0pt}\ na{\isacharparenright}{\kern0pt}{\isachardot}{\kern0pt}\ cnj\ {\isacharparenleft}{\kern0pt}SWAP\ {\isachardollar}{\kern0pt}{\isachardollar}{\kern0pt}\ {\isacharparenleft}{\kern0pt}n{\isacharcomma}{\kern0pt}\ na{\isacharparenright}{\kern0pt}{\isacharparenright}{\kern0pt}{\isacharparenright}{\kern0pt}\ {\isadigit{4}}\ {\isadigit{4}}\ {\isacharequal}{\kern0pt}\ {\isadigit{0}}\ then\ {\isadigit{1}}{\isacharcolon}{\kern0pt}{\isacharcolon}{\kern0pt}complex\ else\ if\ nna\ {\isacharparenleft}{\kern0pt}{\isasymlambda}{\isacharparenleft}{\kern0pt}n{\isacharcomma}{\kern0pt}\ na{\isacharparenright}{\kern0pt}{\isachardot}{\kern0pt}\ cnj\ {\isacharparenleft}{\kern0pt}SWAP\ {\isachardollar}{\kern0pt}{\isachardollar}{\kern0pt}\ {\isacharparenleft}{\kern0pt}na{\isacharcomma}{\kern0pt}\ n{\isacharparenright}{\kern0pt}{\isacharparenright}{\kern0pt}{\isacharparenright}{\kern0pt}\ {\isacharparenleft}{\kern0pt}{\isasymlambda}{\isacharparenleft}{\kern0pt}n{\isacharcomma}{\kern0pt}\ na{\isacharparenright}{\kern0pt}{\isachardot}{\kern0pt}\ cnj\ {\isacharparenleft}{\kern0pt}SWAP\ {\isachardollar}{\kern0pt}{\isachardollar}{\kern0pt}\ {\isacharparenleft}{\kern0pt}n{\isacharcomma}{\kern0pt}\ na{\isacharparenright}{\kern0pt}{\isacharparenright}{\kern0pt}{\isacharparenright}{\kern0pt}\ {\isadigit{4}}\ {\isadigit{4}}\ {\isacharequal}{\kern0pt}\ {\isadigit{1}}\ {\isasymand}\ nn\ {\isacharparenleft}{\kern0pt}{\isasymlambda}{\isacharparenleft}{\kern0pt}n{\isacharcomma}{\kern0pt}\ na{\isacharparenright}{\kern0pt}{\isachardot}{\kern0pt}\ cnj\ {\isacharparenleft}{\kern0pt}SWAP\ {\isachardollar}{\kern0pt}{\isachardollar}{\kern0pt}\ {\isacharparenleft}{\kern0pt}na{\isacharcomma}{\kern0pt}\ n{\isacharparenright}{\kern0pt}{\isacharparenright}{\kern0pt}{\isacharparenright}{\kern0pt}\ {\isacharparenleft}{\kern0pt}{\isasymlambda}{\isacharparenleft}{\kern0pt}n{\isacharcomma}{\kern0pt}\ na{\isacharparenright}{\kern0pt}{\isachardot}{\kern0pt}\ cnj\ {\isacharparenleft}{\kern0pt}SWAP\ {\isachardollar}{\kern0pt}{\isachardollar}{\kern0pt}\ {\isacharparenleft}{\kern0pt}n{\isacharcomma}{\kern0pt}\ na{\isacharparenright}{\kern0pt}{\isacharparenright}{\kern0pt}{\isacharparenright}{\kern0pt}\ {\isadigit{4}}\ {\isadigit{4}}\ {\isacharequal}{\kern0pt}\ {\isadigit{2}}\ then\ {\isadigit{1}}\ else\ if\ nna\ {\isacharparenleft}{\kern0pt}{\isasymlambda}{\isacharparenleft}{\kern0pt}n{\isacharcomma}{\kern0pt}\ na{\isacharparenright}{\kern0pt}{\isachardot}{\kern0pt}\ cnj\ {\isacharparenleft}{\kern0pt}SWAP\ {\isachardollar}{\kern0pt}{\isachardollar}{\kern0pt}\ {\isacharparenleft}{\kern0pt}na{\isacharcomma}{\kern0pt}\ n{\isacharparenright}{\kern0pt}{\isacharparenright}{\kern0pt}{\isacharparenright}{\kern0pt}\ {\isacharparenleft}{\kern0pt}{\isasymlambda}{\isacharparenleft}{\kern0pt}n{\isacharcomma}{\kern0pt}\ na{\isacharparenright}{\kern0pt}{\isachardot}{\kern0pt}\ cnj\ {\isacharparenleft}{\kern0pt}SWAP\ {\isachardollar}{\kern0pt}{\isachardollar}{\kern0pt}\ {\isacharparenleft}{\kern0pt}n{\isacharcomma}{\kern0pt}\ na{\isacharparenright}{\kern0pt}{\isacharparenright}{\kern0pt}{\isacharparenright}{\kern0pt}\ {\isadigit{4}}\ {\isadigit{4}}\ {\isacharequal}{\kern0pt}\ {\isadigit{2}}\ {\isasymand}\ nn\ {\isacharparenleft}{\kern0pt}{\isasymlambda}{\isacharparenleft}{\kern0pt}n{\isacharcomma}{\kern0pt}\ na{\isacharparenright}{\kern0pt}{\isachardot}{\kern0pt}\ cnj\ {\isacharparenleft}{\kern0pt}SWAP\ {\isachardollar}{\kern0pt}{\isachardollar}{\kern0pt}\ {\isacharparenleft}{\kern0pt}na{\isacharcomma}{\kern0pt}\ n{\isacharparenright}{\kern0pt}{\isacharparenright}{\kern0pt}{\isacharparenright}{\kern0pt}\ {\isacharparenleft}{\kern0pt}{\isasymlambda}{\isacharparenleft}{\kern0pt}n{\isacharcomma}{\kern0pt}\ na{\isacharparenright}{\kern0pt}{\isachardot}{\kern0pt}\ cnj\ {\isacharparenleft}{\kern0pt}SWAP\ {\isachardollar}{\kern0pt}{\isachardollar}{\kern0pt}\ {\isacharparenleft}{\kern0pt}n{\isacharcomma}{\kern0pt}\ na{\isacharparenright}{\kern0pt}{\isacharparenright}{\kern0pt}{\isacharparenright}{\kern0pt}\ {\isadigit{4}}\ {\isadigit{4}}\ {\isacharequal}{\kern0pt}\ {\isadigit{1}}\ then\ {\isadigit{1}}\ else\ if\ nna\ {\isacharparenleft}{\kern0pt}{\isasymlambda}{\isacharparenleft}{\kern0pt}n{\isacharcomma}{\kern0pt}\ na{\isacharparenright}{\kern0pt}{\isachardot}{\kern0pt}\ cnj\ {\isacharparenleft}{\kern0pt}SWAP\ {\isachardollar}{\kern0pt}{\isachardollar}{\kern0pt}\ {\isacharparenleft}{\kern0pt}na{\isacharcomma}{\kern0pt}\ n{\isacharparenright}{\kern0pt}{\isacharparenright}{\kern0pt}{\isacharparenright}{\kern0pt}\ {\isacharparenleft}{\kern0pt}{\isasymlambda}{\isacharparenleft}{\kern0pt}n{\isacharcomma}{\kern0pt}\ na{\isacharparenright}{\kern0pt}{\isachardot}{\kern0pt}\ cnj\ {\isacharparenleft}{\kern0pt}SWAP\ {\isachardollar}{\kern0pt}{\isachardollar}{\kern0pt}\ {\isacharparenleft}{\kern0pt}n{\isacharcomma}{\kern0pt}\ na{\isacharparenright}{\kern0pt}{\isacharparenright}{\kern0pt}{\isacharparenright}{\kern0pt}\ {\isadigit{4}}\ {\isadigit{4}}\ {\isacharequal}{\kern0pt}\ {\isadigit{3}}\ {\isasymand}\ nn\ {\isacharparenleft}{\kern0pt}{\isasymlambda}{\isacharparenleft}{\kern0pt}n{\isacharcomma}{\kern0pt}\ na{\isacharparenright}{\kern0pt}{\isachardot}{\kern0pt}\ cnj\ {\isacharparenleft}{\kern0pt}SWAP\ {\isachardollar}{\kern0pt}{\isachardollar}{\kern0pt}\ {\isacharparenleft}{\kern0pt}na{\isacharcomma}{\kern0pt}\ n{\isacharparenright}{\kern0pt}{\isacharparenright}{\kern0pt}{\isacharparenright}{\kern0pt}\ {\isacharparenleft}{\kern0pt}{\isasymlambda}{\isacharparenleft}{\kern0pt}n{\isacharcomma}{\kern0pt}\ na{\isacharparenright}{\kern0pt}{\isachardot}{\kern0pt}\ cnj\ {\isacharparenleft}{\kern0pt}SWAP\ {\isachardollar}{\kern0pt}{\isachardollar}{\kern0pt}\ {\isacharparenleft}{\kern0pt}n{\isacharcomma}{\kern0pt}\ na{\isacharparenright}{\kern0pt}{\isacharparenright}{\kern0pt}{\isacharparenright}{\kern0pt}\ {\isadigit{4}}\ {\isadigit{4}}\ {\isacharequal}{\kern0pt}\ {\isadigit{3}}\ then\ {\isadigit{1}}\ else\ {\isadigit{0}}{\isacharparenright}{\kern0pt}\ {\isacharequal}{\kern0pt}\ {\isadigit{0}}{\isacharparenright}{\kern0pt}\ {\isasymand}\ {\isacharparenleft}{\kern0pt}case\ {\isacharparenleft}{\kern0pt}nn\ {\isacharparenleft}{\kern0pt}{\isasymlambda}{\isacharparenleft}{\kern0pt}n{\isacharcomma}{\kern0pt}\ na{\isacharparenright}{\kern0pt}{\isachardot}{\kern0pt}\ cnj\ {\isacharparenleft}{\kern0pt}SWAP\ {\isachardollar}{\kern0pt}{\isachardollar}{\kern0pt}\ {\isacharparenleft}{\kern0pt}na{\isacharcomma}{\kern0pt}\ n{\isacharparenright}{\kern0pt}{\isacharparenright}{\kern0pt}{\isacharparenright}{\kern0pt}\ {\isacharparenleft}{\kern0pt}{\isasymlambda}{\isacharparenleft}{\kern0pt}n{\isacharcomma}{\kern0pt}\ na{\isacharparenright}{\kern0pt}{\isachardot}{\kern0pt}\ cnj\ {\isacharparenleft}{\kern0pt}SWAP\ {\isachardollar}{\kern0pt}{\isachardollar}{\kern0pt}\ {\isacharparenleft}{\kern0pt}n{\isacharcomma}{\kern0pt}\ na{\isacharparenright}{\kern0pt}{\isacharparenright}{\kern0pt}{\isacharparenright}{\kern0pt}\ {\isadigit{4}}\ {\isadigit{4}}{\isacharcomma}{\kern0pt}\ nna\ {\isacharparenleft}{\kern0pt}{\isasymlambda}{\isacharparenleft}{\kern0pt}n{\isacharcomma}{\kern0pt}\ na{\isacharparenright}{\kern0pt}{\isachardot}{\kern0pt}\ cnj\ {\isacharparenleft}{\kern0pt}SWAP\ {\isachardollar}{\kern0pt}{\isachardollar}{\kern0pt}\ {\isacharparenleft}{\kern0pt}na{\isacharcomma}{\kern0pt}\ n{\isacharparenright}{\kern0pt}{\isacharparenright}{\kern0pt}{\isacharparenright}{\kern0pt}\ {\isacharparenleft}{\kern0pt}{\isasymlambda}{\isacharparenleft}{\kern0pt}n{\isacharcomma}{\kern0pt}\ na{\isacharparenright}{\kern0pt}{\isachardot}{\kern0pt}\ cnj\ {\isacharparenleft}{\kern0pt}SWAP\ {\isachardollar}{\kern0pt}{\isachardollar}{\kern0pt}\ {\isacharparenleft}{\kern0pt}n{\isacharcomma}{\kern0pt}\ na{\isacharparenright}{\kern0pt}{\isacharparenright}{\kern0pt}{\isacharparenright}{\kern0pt}\ {\isadigit{4}}\ {\isadigit{4}}{\isacharparenright}{\kern0pt}\ of\ {\isacharparenleft}{\kern0pt}n{\isacharcomma}{\kern0pt}\ na{\isacharparenright}{\kern0pt}\ {\isasymRightarrow}\ cnj\ {\isacharparenleft}{\kern0pt}SWAP\ {\isachardollar}{\kern0pt}{\isachardollar}{\kern0pt}\ {\isacharparenleft}{\kern0pt}n{\isacharcomma}{\kern0pt}\ na{\isacharparenright}{\kern0pt}{\isacharparenright}{\kern0pt}{\isacharparenright}{\kern0pt}\ {\isasymnoteq}\ {\isacharparenleft}{\kern0pt}case\ {\isacharparenleft}{\kern0pt}nn\ {\isacharparenleft}{\kern0pt}{\isasymlambda}{\isacharparenleft}{\kern0pt}n{\isacharcomma}{\kern0pt}\ na{\isacharparenright}{\kern0pt}{\isachardot}{\kern0pt}\ cnj\ {\isacharparenleft}{\kern0pt}SWAP\ {\isachardollar}{\kern0pt}{\isachardollar}{\kern0pt}\ {\isacharparenleft}{\kern0pt}na{\isacharcomma}{\kern0pt}\ n{\isacharparenright}{\kern0pt}{\isacharparenright}{\kern0pt}{\isacharparenright}{\kern0pt}\ {\isacharparenleft}{\kern0pt}{\isasymlambda}{\isacharparenleft}{\kern0pt}n{\isacharcomma}{\kern0pt}\ na{\isacharparenright}{\kern0pt}{\isachardot}{\kern0pt}\ cnj\ {\isacharparenleft}{\kern0pt}SWAP\ {\isachardollar}{\kern0pt}{\isachardollar}{\kern0pt}\ {\isacharparenleft}{\kern0pt}n{\isacharcomma}{\kern0pt}\ na{\isacharparenright}{\kern0pt}{\isacharparenright}{\kern0pt}{\isacharparenright}{\kern0pt}\ {\isadigit{4}}\ {\isadigit{4}}{\isacharcomma}{\kern0pt}\ nna\ {\isacharparenleft}{\kern0pt}{\isasymlambda}{\isacharparenleft}{\kern0pt}n{\isacharcomma}{\kern0pt}\ na{\isacharparenright}{\kern0pt}{\isachardot}{\kern0pt}\ cnj\ {\isacharparenleft}{\kern0pt}SWAP\ {\isachardollar}{\kern0pt}{\isachardollar}{\kern0pt}\ {\isacharparenleft}{\kern0pt}na{\isacharcomma}{\kern0pt}\ n{\isacharparenright}{\kern0pt}{\isacharparenright}{\kern0pt}{\isacharparenright}{\kern0pt}\ {\isacharparenleft}{\kern0pt}{\isasymlambda}{\isacharparenleft}{\kern0pt}n{\isacharcomma}{\kern0pt}\ na{\isacharparenright}{\kern0pt}{\isachardot}{\kern0pt}\ cnj\ {\isacharparenleft}{\kern0pt}SWAP\ {\isachardollar}{\kern0pt}{\isachardollar}{\kern0pt}\ {\isacharparenleft}{\kern0pt}n{\isacharcomma}{\kern0pt}\ na{\isacharparenright}{\kern0pt}{\isacharparenright}{\kern0pt}{\isacharparenright}{\kern0pt}\ {\isadigit{4}}\ {\isadigit{4}}{\isacharparenright}{\kern0pt}\ of\ {\isacharparenleft}{\kern0pt}n{\isacharcomma}{\kern0pt}\ na{\isacharparenright}{\kern0pt}\ {\isasymRightarrow}\ cnj\ {\isacharparenleft}{\kern0pt}SWAP\ {\isachardollar}{\kern0pt}{\isachardollar}{\kern0pt}\ {\isacharparenleft}{\kern0pt}na{\isacharcomma}{\kern0pt}\ n{\isacharparenright}{\kern0pt}{\isacharparenright}{\kern0pt}{\isacharparenright}{\kern0pt}{\isachardoublequoteclose}\isanewline
\ \ \ \ \ \ \ \ \isacommand{then}\isamarkupfalse%
\ \isacommand{have}\isamarkupfalse%
\ {\isachardoublequoteopen}Matrix{\isachardot}{\kern0pt}mat\ {\isadigit{4}}\ {\isadigit{4}}\ {\isacharparenleft}{\kern0pt}{\isasymlambda}{\isacharparenleft}{\kern0pt}n{\isacharcomma}{\kern0pt}\ na{\isacharparenright}{\kern0pt}{\isachardot}{\kern0pt}\ if\ n\ {\isacharequal}{\kern0pt}\ {\isadigit{0}}\ {\isasymand}\ na\ {\isacharequal}{\kern0pt}\ {\isadigit{0}}\ then\ {\isadigit{1}}{\isacharcolon}{\kern0pt}{\isacharcolon}{\kern0pt}complex\ else\ if\ n\ {\isacharequal}{\kern0pt}\ {\isadigit{1}}\ {\isasymand}\ na\ {\isacharequal}{\kern0pt}\ {\isadigit{2}}\ then\ {\isadigit{1}}\ else\ if\ n\ {\isacharequal}{\kern0pt}\ {\isadigit{2}}\ {\isasymand}\ na\ {\isacharequal}{\kern0pt}\ {\isadigit{1}}\ then\ {\isadigit{1}}\ else\ if\ n\ {\isacharequal}{\kern0pt}\ {\isadigit{3}}\ {\isasymand}\ na\ {\isacharequal}{\kern0pt}\ {\isadigit{3}}\ then\ {\isadigit{1}}\ else\ {\isadigit{0}}{\isacharparenright}{\kern0pt}\ {\isachardollar}{\kern0pt}{\isachardollar}{\kern0pt}\ {\isacharparenleft}{\kern0pt}nn\ {\isacharparenleft}{\kern0pt}{\isasymlambda}{\isacharparenleft}{\kern0pt}n{\isacharcomma}{\kern0pt}\ na{\isacharparenright}{\kern0pt}{\isachardot}{\kern0pt}\ cnj\ {\isacharparenleft}{\kern0pt}SWAP\ {\isachardollar}{\kern0pt}{\isachardollar}{\kern0pt}\ {\isacharparenleft}{\kern0pt}na{\isacharcomma}{\kern0pt}\ n{\isacharparenright}{\kern0pt}{\isacharparenright}{\kern0pt}{\isacharparenright}{\kern0pt}\ {\isacharparenleft}{\kern0pt}{\isasymlambda}{\isacharparenleft}{\kern0pt}n{\isacharcomma}{\kern0pt}\ na{\isacharparenright}{\kern0pt}{\isachardot}{\kern0pt}\ cnj\ {\isacharparenleft}{\kern0pt}SWAP\ {\isachardollar}{\kern0pt}{\isachardollar}{\kern0pt}\ {\isacharparenleft}{\kern0pt}n{\isacharcomma}{\kern0pt}\ na{\isacharparenright}{\kern0pt}{\isacharparenright}{\kern0pt}{\isacharparenright}{\kern0pt}\ {\isadigit{4}}\ {\isadigit{4}}{\isacharcomma}{\kern0pt}\ nna\ {\isacharparenleft}{\kern0pt}{\isasymlambda}{\isacharparenleft}{\kern0pt}n{\isacharcomma}{\kern0pt}\ na{\isacharparenright}{\kern0pt}{\isachardot}{\kern0pt}\ cnj\ {\isacharparenleft}{\kern0pt}SWAP\ {\isachardollar}{\kern0pt}{\isachardollar}{\kern0pt}\ {\isacharparenleft}{\kern0pt}na{\isacharcomma}{\kern0pt}\ n{\isacharparenright}{\kern0pt}{\isacharparenright}{\kern0pt}{\isacharparenright}{\kern0pt}\ {\isacharparenleft}{\kern0pt}{\isasymlambda}{\isacharparenleft}{\kern0pt}n{\isacharcomma}{\kern0pt}\ na{\isacharparenright}{\kern0pt}{\isachardot}{\kern0pt}\ cnj\ {\isacharparenleft}{\kern0pt}SWAP\ {\isachardollar}{\kern0pt}{\isachardollar}{\kern0pt}\ {\isacharparenleft}{\kern0pt}n{\isacharcomma}{\kern0pt}\ na{\isacharparenright}{\kern0pt}{\isacharparenright}{\kern0pt}{\isacharparenright}{\kern0pt}\ {\isadigit{4}}\ {\isadigit{4}}{\isacharparenright}{\kern0pt}\ {\isacharequal}{\kern0pt}\ {\isacharparenleft}{\kern0pt}case\ {\isacharparenleft}{\kern0pt}nn\ {\isacharparenleft}{\kern0pt}{\isasymlambda}{\isacharparenleft}{\kern0pt}n{\isacharcomma}{\kern0pt}\ na{\isacharparenright}{\kern0pt}{\isachardot}{\kern0pt}\ cnj\ {\isacharparenleft}{\kern0pt}SWAP\ {\isachardollar}{\kern0pt}{\isachardollar}{\kern0pt}\ {\isacharparenleft}{\kern0pt}na{\isacharcomma}{\kern0pt}\ n{\isacharparenright}{\kern0pt}{\isacharparenright}{\kern0pt}{\isacharparenright}{\kern0pt}\ {\isacharparenleft}{\kern0pt}{\isasymlambda}{\isacharparenleft}{\kern0pt}n{\isacharcomma}{\kern0pt}\ na{\isacharparenright}{\kern0pt}{\isachardot}{\kern0pt}\ cnj\ {\isacharparenleft}{\kern0pt}SWAP\ {\isachardollar}{\kern0pt}{\isachardollar}{\kern0pt}\ {\isacharparenleft}{\kern0pt}n{\isacharcomma}{\kern0pt}\ na{\isacharparenright}{\kern0pt}{\isacharparenright}{\kern0pt}{\isacharparenright}{\kern0pt}\ {\isadigit{4}}\ {\isadigit{4}}{\isacharcomma}{\kern0pt}\ nna\ {\isacharparenleft}{\kern0pt}{\isasymlambda}{\isacharparenleft}{\kern0pt}n{\isacharcomma}{\kern0pt}\ na{\isacharparenright}{\kern0pt}{\isachardot}{\kern0pt}\ cnj\ {\isacharparenleft}{\kern0pt}SWAP\ {\isachardollar}{\kern0pt}{\isachardollar}{\kern0pt}\ {\isacharparenleft}{\kern0pt}na{\isacharcomma}{\kern0pt}\ n{\isacharparenright}{\kern0pt}{\isacharparenright}{\kern0pt}{\isacharparenright}{\kern0pt}\ {\isacharparenleft}{\kern0pt}{\isasymlambda}{\isacharparenleft}{\kern0pt}n{\isacharcomma}{\kern0pt}\ na{\isacharparenright}{\kern0pt}{\isachardot}{\kern0pt}\ cnj\ {\isacharparenleft}{\kern0pt}SWAP\ {\isachardollar}{\kern0pt}{\isachardollar}{\kern0pt}\ {\isacharparenleft}{\kern0pt}n{\isacharcomma}{\kern0pt}\ na{\isacharparenright}{\kern0pt}{\isacharparenright}{\kern0pt}{\isacharparenright}{\kern0pt}\ {\isadigit{4}}\ {\isadigit{4}}{\isacharparenright}{\kern0pt}\ of\ {\isacharparenleft}{\kern0pt}n{\isacharcomma}{\kern0pt}\ na{\isacharparenright}{\kern0pt}\ {\isasymRightarrow}\ if\ n\ {\isacharequal}{\kern0pt}\ {\isadigit{0}}\ {\isasymand}\ na\ {\isacharequal}{\kern0pt}\ {\isadigit{0}}\ then\ {\isadigit{1}}\ else\ if\ n\ {\isacharequal}{\kern0pt}\ {\isadigit{1}}\ {\isasymand}\ na\ {\isacharequal}{\kern0pt}\ {\isadigit{2}}\ then\ {\isadigit{1}}\ else\ if\ n\ {\isacharequal}{\kern0pt}\ {\isadigit{2}}\ {\isasymand}\ na\ {\isacharequal}{\kern0pt}\ {\isadigit{1}}\ then\ {\isadigit{1}}\ else\ if\ n\ {\isacharequal}{\kern0pt}\ {\isadigit{3}}\ {\isasymand}\ na\ {\isacharequal}{\kern0pt}\ {\isadigit{3}}\ then\ {\isadigit{1}}\ else\ {\isadigit{0}}{\isacharparenright}{\kern0pt}\ {\isasymlongrightarrow}\ {\isacharparenleft}{\kern0pt}{\isacharparenleft}{\kern0pt}if\ nn\ {\isacharparenleft}{\kern0pt}{\isasymlambda}{\isacharparenleft}{\kern0pt}n{\isacharcomma}{\kern0pt}\ na{\isacharparenright}{\kern0pt}{\isachardot}{\kern0pt}\ cnj\ {\isacharparenleft}{\kern0pt}SWAP\ {\isachardollar}{\kern0pt}{\isachardollar}{\kern0pt}\ {\isacharparenleft}{\kern0pt}na{\isacharcomma}{\kern0pt}\ n{\isacharparenright}{\kern0pt}{\isacharparenright}{\kern0pt}{\isacharparenright}{\kern0pt}\ {\isacharparenleft}{\kern0pt}{\isasymlambda}{\isacharparenleft}{\kern0pt}n{\isacharcomma}{\kern0pt}\ na{\isacharparenright}{\kern0pt}{\isachardot}{\kern0pt}\ cnj\ {\isacharparenleft}{\kern0pt}SWAP\ {\isachardollar}{\kern0pt}{\isachardollar}{\kern0pt}\ {\isacharparenleft}{\kern0pt}n{\isacharcomma}{\kern0pt}\ na{\isacharparenright}{\kern0pt}{\isacharparenright}{\kern0pt}{\isacharparenright}{\kern0pt}\ {\isadigit{4}}\ {\isadigit{4}}\ {\isacharequal}{\kern0pt}\ {\isadigit{0}}\ {\isasymand}\ nna\ {\isacharparenleft}{\kern0pt}{\isasymlambda}{\isacharparenleft}{\kern0pt}n{\isacharcomma}{\kern0pt}\ na{\isacharparenright}{\kern0pt}{\isachardot}{\kern0pt}\ cnj\ {\isacharparenleft}{\kern0pt}SWAP\ {\isachardollar}{\kern0pt}{\isachardollar}{\kern0pt}\ {\isacharparenleft}{\kern0pt}na{\isacharcomma}{\kern0pt}\ n{\isacharparenright}{\kern0pt}{\isacharparenright}{\kern0pt}{\isacharparenright}{\kern0pt}\ {\isacharparenleft}{\kern0pt}{\isasymlambda}{\isacharparenleft}{\kern0pt}n{\isacharcomma}{\kern0pt}\ na{\isacharparenright}{\kern0pt}{\isachardot}{\kern0pt}\ cnj\ {\isacharparenleft}{\kern0pt}SWAP\ {\isachardollar}{\kern0pt}{\isachardollar}{\kern0pt}\ {\isacharparenleft}{\kern0pt}n{\isacharcomma}{\kern0pt}\ na{\isacharparenright}{\kern0pt}{\isacharparenright}{\kern0pt}{\isacharparenright}{\kern0pt}\ {\isadigit{4}}\ {\isadigit{4}}\ {\isacharequal}{\kern0pt}\ {\isadigit{0}}\ then\ {\isadigit{1}}{\isacharcolon}{\kern0pt}{\isacharcolon}{\kern0pt}complex\ else\ if\ nn\ {\isacharparenleft}{\kern0pt}{\isasymlambda}{\isacharparenleft}{\kern0pt}n{\isacharcomma}{\kern0pt}\ na{\isacharparenright}{\kern0pt}{\isachardot}{\kern0pt}\ cnj\ {\isacharparenleft}{\kern0pt}SWAP\ {\isachardollar}{\kern0pt}{\isachardollar}{\kern0pt}\ {\isacharparenleft}{\kern0pt}na{\isacharcomma}{\kern0pt}\ n{\isacharparenright}{\kern0pt}{\isacharparenright}{\kern0pt}{\isacharparenright}{\kern0pt}\ {\isacharparenleft}{\kern0pt}{\isasymlambda}{\isacharparenleft}{\kern0pt}n{\isacharcomma}{\kern0pt}\ na{\isacharparenright}{\kern0pt}{\isachardot}{\kern0pt}\ cnj\ {\isacharparenleft}{\kern0pt}SWAP\ {\isachardollar}{\kern0pt}{\isachardollar}{\kern0pt}\ {\isacharparenleft}{\kern0pt}n{\isacharcomma}{\kern0pt}\ na{\isacharparenright}{\kern0pt}{\isacharparenright}{\kern0pt}{\isacharparenright}{\kern0pt}\ {\isadigit{4}}\ {\isadigit{4}}\ {\isacharequal}{\kern0pt}\ {\isadigit{1}}\ {\isasymand}\ nna\ {\isacharparenleft}{\kern0pt}{\isasymlambda}{\isacharparenleft}{\kern0pt}n{\isacharcomma}{\kern0pt}\ na{\isacharparenright}{\kern0pt}{\isachardot}{\kern0pt}\ cnj\ {\isacharparenleft}{\kern0pt}SWAP\ {\isachardollar}{\kern0pt}{\isachardollar}{\kern0pt}\ {\isacharparenleft}{\kern0pt}na{\isacharcomma}{\kern0pt}\ n{\isacharparenright}{\kern0pt}{\isacharparenright}{\kern0pt}{\isacharparenright}{\kern0pt}\ {\isacharparenleft}{\kern0pt}{\isasymlambda}{\isacharparenleft}{\kern0pt}n{\isacharcomma}{\kern0pt}\ na{\isacharparenright}{\kern0pt}{\isachardot}{\kern0pt}\ cnj\ {\isacharparenleft}{\kern0pt}SWAP\ {\isachardollar}{\kern0pt}{\isachardollar}{\kern0pt}\ {\isacharparenleft}{\kern0pt}n{\isacharcomma}{\kern0pt}\ na{\isacharparenright}{\kern0pt}{\isacharparenright}{\kern0pt}{\isacharparenright}{\kern0pt}\ {\isadigit{4}}\ {\isadigit{4}}\ {\isacharequal}{\kern0pt}\ {\isadigit{2}}\ then\ {\isadigit{1}}\ else\ if\ nn\ {\isacharparenleft}{\kern0pt}{\isasymlambda}{\isacharparenleft}{\kern0pt}n{\isacharcomma}{\kern0pt}\ na{\isacharparenright}{\kern0pt}{\isachardot}{\kern0pt}\ cnj\ {\isacharparenleft}{\kern0pt}SWAP\ {\isachardollar}{\kern0pt}{\isachardollar}{\kern0pt}\ {\isacharparenleft}{\kern0pt}na{\isacharcomma}{\kern0pt}\ n{\isacharparenright}{\kern0pt}{\isacharparenright}{\kern0pt}{\isacharparenright}{\kern0pt}\ {\isacharparenleft}{\kern0pt}{\isasymlambda}{\isacharparenleft}{\kern0pt}n{\isacharcomma}{\kern0pt}\ na{\isacharparenright}{\kern0pt}{\isachardot}{\kern0pt}\ cnj\ {\isacharparenleft}{\kern0pt}SWAP\ {\isachardollar}{\kern0pt}{\isachardollar}{\kern0pt}\ {\isacharparenleft}{\kern0pt}n{\isacharcomma}{\kern0pt}\ na{\isacharparenright}{\kern0pt}{\isacharparenright}{\kern0pt}{\isacharparenright}{\kern0pt}\ {\isadigit{4}}\ {\isadigit{4}}\ {\isacharequal}{\kern0pt}\ {\isadigit{2}}\ {\isasymand}\ nna\ {\isacharparenleft}{\kern0pt}{\isasymlambda}{\isacharparenleft}{\kern0pt}n{\isacharcomma}{\kern0pt}\ na{\isacharparenright}{\kern0pt}{\isachardot}{\kern0pt}\ cnj\ {\isacharparenleft}{\kern0pt}SWAP\ {\isachardollar}{\kern0pt}{\isachardollar}{\kern0pt}\ {\isacharparenleft}{\kern0pt}na{\isacharcomma}{\kern0pt}\ n{\isacharparenright}{\kern0pt}{\isacharparenright}{\kern0pt}{\isacharparenright}{\kern0pt}\ {\isacharparenleft}{\kern0pt}{\isasymlambda}{\isacharparenleft}{\kern0pt}n{\isacharcomma}{\kern0pt}\ na{\isacharparenright}{\kern0pt}{\isachardot}{\kern0pt}\ cnj\ {\isacharparenleft}{\kern0pt}SWAP\ {\isachardollar}{\kern0pt}{\isachardollar}{\kern0pt}\ {\isacharparenleft}{\kern0pt}n{\isacharcomma}{\kern0pt}\ na{\isacharparenright}{\kern0pt}{\isacharparenright}{\kern0pt}{\isacharparenright}{\kern0pt}\ {\isadigit{4}}\ {\isadigit{4}}\ {\isacharequal}{\kern0pt}\ {\isadigit{1}}\ then\ {\isadigit{1}}\ else\ if\ nn\ {\isacharparenleft}{\kern0pt}{\isasymlambda}{\isacharparenleft}{\kern0pt}n{\isacharcomma}{\kern0pt}\ na{\isacharparenright}{\kern0pt}{\isachardot}{\kern0pt}\ cnj\ {\isacharparenleft}{\kern0pt}SWAP\ {\isachardollar}{\kern0pt}{\isachardollar}{\kern0pt}\ {\isacharparenleft}{\kern0pt}na{\isacharcomma}{\kern0pt}\ n{\isacharparenright}{\kern0pt}{\isacharparenright}{\kern0pt}{\isacharparenright}{\kern0pt}\ {\isacharparenleft}{\kern0pt}{\isasymlambda}{\isacharparenleft}{\kern0pt}n{\isacharcomma}{\kern0pt}\ na{\isacharparenright}{\kern0pt}{\isachardot}{\kern0pt}\ cnj\ {\isacharparenleft}{\kern0pt}SWAP\ {\isachardollar}{\kern0pt}{\isachardollar}{\kern0pt}\ {\isacharparenleft}{\kern0pt}n{\isacharcomma}{\kern0pt}\ na{\isacharparenright}{\kern0pt}{\isacharparenright}{\kern0pt}{\isacharparenright}{\kern0pt}\ {\isadigit{4}}\ {\isadigit{4}}\ {\isacharequal}{\kern0pt}\ {\isadigit{3}}\ {\isasymand}\ nna\ {\isacharparenleft}{\kern0pt}{\isasymlambda}{\isacharparenleft}{\kern0pt}n{\isacharcomma}{\kern0pt}\ na{\isacharparenright}{\kern0pt}{\isachardot}{\kern0pt}\ cnj\ {\isacharparenleft}{\kern0pt}SWAP\ {\isachardollar}{\kern0pt}{\isachardollar}{\kern0pt}\ {\isacharparenleft}{\kern0pt}na{\isacharcomma}{\kern0pt}\ n{\isacharparenright}{\kern0pt}{\isacharparenright}{\kern0pt}{\isacharparenright}{\kern0pt}\ {\isacharparenleft}{\kern0pt}{\isasymlambda}{\isacharparenleft}{\kern0pt}n{\isacharcomma}{\kern0pt}\ na{\isacharparenright}{\kern0pt}{\isachardot}{\kern0pt}\ cnj\ {\isacharparenleft}{\kern0pt}SWAP\ {\isachardollar}{\kern0pt}{\isachardollar}{\kern0pt}\ {\isacharparenleft}{\kern0pt}n{\isacharcomma}{\kern0pt}\ na{\isacharparenright}{\kern0pt}{\isacharparenright}{\kern0pt}{\isacharparenright}{\kern0pt}\ {\isadigit{4}}\ {\isadigit{4}}\ {\isacharequal}{\kern0pt}\ {\isadigit{3}}\ then\ {\isadigit{1}}\ else\ {\isadigit{0}}{\isacharparenright}{\kern0pt}\ {\isacharequal}{\kern0pt}\ {\isadigit{0}}\ {\isasymand}\ {\isacharparenleft}{\kern0pt}if\ nna\ {\isacharparenleft}{\kern0pt}{\isasymlambda}{\isacharparenleft}{\kern0pt}n{\isacharcomma}{\kern0pt}\ na{\isacharparenright}{\kern0pt}{\isachardot}{\kern0pt}\ cnj\ {\isacharparenleft}{\kern0pt}SWAP\ {\isachardollar}{\kern0pt}{\isachardollar}{\kern0pt}\ {\isacharparenleft}{\kern0pt}na{\isacharcomma}{\kern0pt}\ n{\isacharparenright}{\kern0pt}{\isacharparenright}{\kern0pt}{\isacharparenright}{\kern0pt}\ {\isacharparenleft}{\kern0pt}{\isasymlambda}{\isacharparenleft}{\kern0pt}n{\isacharcomma}{\kern0pt}\ na{\isacharparenright}{\kern0pt}{\isachardot}{\kern0pt}\ cnj\ {\isacharparenleft}{\kern0pt}SWAP\ {\isachardollar}{\kern0pt}{\isachardollar}{\kern0pt}\ {\isacharparenleft}{\kern0pt}n{\isacharcomma}{\kern0pt}\ na{\isacharparenright}{\kern0pt}{\isacharparenright}{\kern0pt}{\isacharparenright}{\kern0pt}\ {\isadigit{4}}\ {\isadigit{4}}\ {\isacharequal}{\kern0pt}\ {\isadigit{0}}\ {\isasymand}\ nn\ {\isacharparenleft}{\kern0pt}{\isasymlambda}{\isacharparenleft}{\kern0pt}n{\isacharcomma}{\kern0pt}\ na{\isacharparenright}{\kern0pt}{\isachardot}{\kern0pt}\ cnj\ {\isacharparenleft}{\kern0pt}SWAP\ {\isachardollar}{\kern0pt}{\isachardollar}{\kern0pt}\ {\isacharparenleft}{\kern0pt}na{\isacharcomma}{\kern0pt}\ n{\isacharparenright}{\kern0pt}{\isacharparenright}{\kern0pt}{\isacharparenright}{\kern0pt}\ {\isacharparenleft}{\kern0pt}{\isasymlambda}{\isacharparenleft}{\kern0pt}n{\isacharcomma}{\kern0pt}\ na{\isacharparenright}{\kern0pt}{\isachardot}{\kern0pt}\ cnj\ {\isacharparenleft}{\kern0pt}SWAP\ {\isachardollar}{\kern0pt}{\isachardollar}{\kern0pt}\ {\isacharparenleft}{\kern0pt}n{\isacharcomma}{\kern0pt}\ na{\isacharparenright}{\kern0pt}{\isacharparenright}{\kern0pt}{\isacharparenright}{\kern0pt}\ {\isadigit{4}}\ {\isadigit{4}}\ {\isacharequal}{\kern0pt}\ {\isadigit{0}}\ then\ {\isadigit{1}}{\isacharcolon}{\kern0pt}{\isacharcolon}{\kern0pt}complex\ else\ if\ nna\ {\isacharparenleft}{\kern0pt}{\isasymlambda}{\isacharparenleft}{\kern0pt}n{\isacharcomma}{\kern0pt}\ na{\isacharparenright}{\kern0pt}{\isachardot}{\kern0pt}\ cnj\ {\isacharparenleft}{\kern0pt}SWAP\ {\isachardollar}{\kern0pt}{\isachardollar}{\kern0pt}\ {\isacharparenleft}{\kern0pt}na{\isacharcomma}{\kern0pt}\ n{\isacharparenright}{\kern0pt}{\isacharparenright}{\kern0pt}{\isacharparenright}{\kern0pt}\ {\isacharparenleft}{\kern0pt}{\isasymlambda}{\isacharparenleft}{\kern0pt}n{\isacharcomma}{\kern0pt}\ na{\isacharparenright}{\kern0pt}{\isachardot}{\kern0pt}\ cnj\ {\isacharparenleft}{\kern0pt}SWAP\ {\isachardollar}{\kern0pt}{\isachardollar}{\kern0pt}\ {\isacharparenleft}{\kern0pt}n{\isacharcomma}{\kern0pt}\ na{\isacharparenright}{\kern0pt}{\isacharparenright}{\kern0pt}{\isacharparenright}{\kern0pt}\ {\isadigit{4}}\ {\isadigit{4}}\ {\isacharequal}{\kern0pt}\ {\isadigit{1}}\ {\isasymand}\ nn\ {\isacharparenleft}{\kern0pt}{\isasymlambda}{\isacharparenleft}{\kern0pt}n{\isacharcomma}{\kern0pt}\ na{\isacharparenright}{\kern0pt}{\isachardot}{\kern0pt}\ cnj\ {\isacharparenleft}{\kern0pt}SWAP\ {\isachardollar}{\kern0pt}{\isachardollar}{\kern0pt}\ {\isacharparenleft}{\kern0pt}na{\isacharcomma}{\kern0pt}\ n{\isacharparenright}{\kern0pt}{\isacharparenright}{\kern0pt}{\isacharparenright}{\kern0pt}\ {\isacharparenleft}{\kern0pt}{\isasymlambda}{\isacharparenleft}{\kern0pt}n{\isacharcomma}{\kern0pt}\ na{\isacharparenright}{\kern0pt}{\isachardot}{\kern0pt}\ cnj\ {\isacharparenleft}{\kern0pt}SWAP\ {\isachardollar}{\kern0pt}{\isachardollar}{\kern0pt}\ {\isacharparenleft}{\kern0pt}n{\isacharcomma}{\kern0pt}\ na{\isacharparenright}{\kern0pt}{\isacharparenright}{\kern0pt}{\isacharparenright}{\kern0pt}\ {\isadigit{4}}\ {\isadigit{4}}\ {\isacharequal}{\kern0pt}\ {\isadigit{2}}\ then\ {\isadigit{1}}\ else\ if\ nna\ {\isacharparenleft}{\kern0pt}{\isasymlambda}{\isacharparenleft}{\kern0pt}n{\isacharcomma}{\kern0pt}\ na{\isacharparenright}{\kern0pt}{\isachardot}{\kern0pt}\ cnj\ {\isacharparenleft}{\kern0pt}SWAP\ {\isachardollar}{\kern0pt}{\isachardollar}{\kern0pt}\ {\isacharparenleft}{\kern0pt}na{\isacharcomma}{\kern0pt}\ n{\isacharparenright}{\kern0pt}{\isacharparenright}{\kern0pt}{\isacharparenright}{\kern0pt}\ {\isacharparenleft}{\kern0pt}{\isasymlambda}{\isacharparenleft}{\kern0pt}n{\isacharcomma}{\kern0pt}\ na{\isacharparenright}{\kern0pt}{\isachardot}{\kern0pt}\ cnj\ {\isacharparenleft}{\kern0pt}SWAP\ {\isachardollar}{\kern0pt}{\isachardollar}{\kern0pt}\ {\isacharparenleft}{\kern0pt}n{\isacharcomma}{\kern0pt}\ na{\isacharparenright}{\kern0pt}{\isacharparenright}{\kern0pt}{\isacharparenright}{\kern0pt}\ {\isadigit{4}}\ {\isadigit{4}}\ {\isacharequal}{\kern0pt}\ {\isadigit{2}}\ {\isasymand}\ nn\ {\isacharparenleft}{\kern0pt}{\isasymlambda}{\isacharparenleft}{\kern0pt}n{\isacharcomma}{\kern0pt}\ na{\isacharparenright}{\kern0pt}{\isachardot}{\kern0pt}\ cnj\ {\isacharparenleft}{\kern0pt}SWAP\ {\isachardollar}{\kern0pt}{\isachardollar}{\kern0pt}\ {\isacharparenleft}{\kern0pt}na{\isacharcomma}{\kern0pt}\ n{\isacharparenright}{\kern0pt}{\isacharparenright}{\kern0pt}{\isacharparenright}{\kern0pt}\ {\isacharparenleft}{\kern0pt}{\isasymlambda}{\isacharparenleft}{\kern0pt}n{\isacharcomma}{\kern0pt}\ na{\isacharparenright}{\kern0pt}{\isachardot}{\kern0pt}\ cnj\ {\isacharparenleft}{\kern0pt}SWAP\ {\isachardollar}{\kern0pt}{\isachardollar}{\kern0pt}\ {\isacharparenleft}{\kern0pt}n{\isacharcomma}{\kern0pt}\ na{\isacharparenright}{\kern0pt}{\isacharparenright}{\kern0pt}{\isacharparenright}{\kern0pt}\ {\isadigit{4}}\ {\isadigit{4}}\ {\isacharequal}{\kern0pt}\ {\isadigit{1}}\ then\ {\isadigit{1}}\ else\ if\ nna\ {\isacharparenleft}{\kern0pt}{\isasymlambda}{\isacharparenleft}{\kern0pt}n{\isacharcomma}{\kern0pt}\ na{\isacharparenright}{\kern0pt}{\isachardot}{\kern0pt}\ cnj\ {\isacharparenleft}{\kern0pt}SWAP\ {\isachardollar}{\kern0pt}{\isachardollar}{\kern0pt}\ {\isacharparenleft}{\kern0pt}na{\isacharcomma}{\kern0pt}\ n{\isacharparenright}{\kern0pt}{\isacharparenright}{\kern0pt}{\isacharparenright}{\kern0pt}\ {\isacharparenleft}{\kern0pt}{\isasymlambda}{\isacharparenleft}{\kern0pt}n{\isacharcomma}{\kern0pt}\ na{\isacharparenright}{\kern0pt}{\isachardot}{\kern0pt}\ cnj\ {\isacharparenleft}{\kern0pt}SWAP\ {\isachardollar}{\kern0pt}{\isachardollar}{\kern0pt}\ {\isacharparenleft}{\kern0pt}n{\isacharcomma}{\kern0pt}\ na{\isacharparenright}{\kern0pt}{\isacharparenright}{\kern0pt}{\isacharparenright}{\kern0pt}\ {\isadigit{4}}\ {\isadigit{4}}\ {\isacharequal}{\kern0pt}\ {\isadigit{3}}\ {\isasymand}\ nn\ {\isacharparenleft}{\kern0pt}{\isasymlambda}{\isacharparenleft}{\kern0pt}n{\isacharcomma}{\kern0pt}\ na{\isacharparenright}{\kern0pt}{\isachardot}{\kern0pt}\ cnj\ {\isacharparenleft}{\kern0pt}SWAP\ {\isachardollar}{\kern0pt}{\isachardollar}{\kern0pt}\ {\isacharparenleft}{\kern0pt}na{\isacharcomma}{\kern0pt}\ n{\isacharparenright}{\kern0pt}{\isacharparenright}{\kern0pt}{\isacharparenright}{\kern0pt}\ {\isacharparenleft}{\kern0pt}{\isasymlambda}{\isacharparenleft}{\kern0pt}n{\isacharcomma}{\kern0pt}\ na{\isacharparenright}{\kern0pt}{\isachardot}{\kern0pt}\ cnj\ {\isacharparenleft}{\kern0pt}SWAP\ {\isachardollar}{\kern0pt}{\isachardollar}{\kern0pt}\ {\isacharparenleft}{\kern0pt}n{\isacharcomma}{\kern0pt}\ na{\isacharparenright}{\kern0pt}{\isacharparenright}{\kern0pt}{\isacharparenright}{\kern0pt}\ {\isadigit{4}}\ {\isadigit{4}}\ {\isacharequal}{\kern0pt}\ {\isadigit{3}}\ then\ {\isadigit{1}}\ else\ {\isadigit{0}}{\isacharparenright}{\kern0pt}\ {\isacharequal}{\kern0pt}\ {\isadigit{0}}{\isacharparenright}{\kern0pt}\ {\isasymand}\ SWAP\ {\isachardollar}{\kern0pt}{\isachardollar}{\kern0pt}\ {\isacharparenleft}{\kern0pt}nna\ {\isacharparenleft}{\kern0pt}{\isasymlambda}{\isacharparenleft}{\kern0pt}n{\isacharcomma}{\kern0pt}\ na{\isacharparenright}{\kern0pt}{\isachardot}{\kern0pt}\ cnj\ {\isacharparenleft}{\kern0pt}SWAP\ {\isachardollar}{\kern0pt}{\isachardollar}{\kern0pt}\ {\isacharparenleft}{\kern0pt}na{\isacharcomma}{\kern0pt}\ n{\isacharparenright}{\kern0pt}{\isacharparenright}{\kern0pt}{\isacharparenright}{\kern0pt}\ {\isacharparenleft}{\kern0pt}{\isasymlambda}{\isacharparenleft}{\kern0pt}n{\isacharcomma}{\kern0pt}\ na{\isacharparenright}{\kern0pt}{\isachardot}{\kern0pt}\ cnj\ {\isacharparenleft}{\kern0pt}SWAP\ {\isachardollar}{\kern0pt}{\isachardollar}{\kern0pt}\ {\isacharparenleft}{\kern0pt}n{\isacharcomma}{\kern0pt}\ na{\isacharparenright}{\kern0pt}{\isacharparenright}{\kern0pt}{\isacharparenright}{\kern0pt}\ {\isadigit{4}}\ {\isadigit{4}}{\isacharcomma}{\kern0pt}\ nn\ {\isacharparenleft}{\kern0pt}{\isasymlambda}{\isacharparenleft}{\kern0pt}n{\isacharcomma}{\kern0pt}\ na{\isacharparenright}{\kern0pt}{\isachardot}{\kern0pt}\ cnj\ {\isacharparenleft}{\kern0pt}SWAP\ {\isachardollar}{\kern0pt}{\isachardollar}{\kern0pt}\ {\isacharparenleft}{\kern0pt}na{\isacharcomma}{\kern0pt}\ n{\isacharparenright}{\kern0pt}{\isacharparenright}{\kern0pt}{\isacharparenright}{\kern0pt}\ {\isacharparenleft}{\kern0pt}{\isasymlambda}{\isacharparenleft}{\kern0pt}n{\isacharcomma}{\kern0pt}\ na{\isacharparenright}{\kern0pt}{\isachardot}{\kern0pt}\ cnj\ {\isacharparenleft}{\kern0pt}SWAP\ {\isachardollar}{\kern0pt}{\isachardollar}{\kern0pt}\ {\isacharparenleft}{\kern0pt}n{\isacharcomma}{\kern0pt}\ na{\isacharparenright}{\kern0pt}{\isacharparenright}{\kern0pt}{\isacharparenright}{\kern0pt}\ {\isadigit{4}}\ {\isadigit{4}}{\isacharparenright}{\kern0pt}\ {\isasymnoteq}\ {\isacharparenleft}{\kern0pt}case\ {\isacharparenleft}{\kern0pt}nn\ {\isacharparenleft}{\kern0pt}{\isasymlambda}{\isacharparenleft}{\kern0pt}n{\isacharcomma}{\kern0pt}\ na{\isacharparenright}{\kern0pt}{\isachardot}{\kern0pt}\ cnj\ {\isacharparenleft}{\kern0pt}SWAP\ {\isachardollar}{\kern0pt}{\isachardollar}{\kern0pt}\ {\isacharparenleft}{\kern0pt}na{\isacharcomma}{\kern0pt}\ n{\isacharparenright}{\kern0pt}{\isacharparenright}{\kern0pt}{\isacharparenright}{\kern0pt}\ {\isacharparenleft}{\kern0pt}{\isasymlambda}{\isacharparenleft}{\kern0pt}n{\isacharcomma}{\kern0pt}\ na{\isacharparenright}{\kern0pt}{\isachardot}{\kern0pt}\ cnj\ {\isacharparenleft}{\kern0pt}SWAP\ {\isachardollar}{\kern0pt}{\isachardollar}{\kern0pt}\ {\isacharparenleft}{\kern0pt}n{\isacharcomma}{\kern0pt}\ na{\isacharparenright}{\kern0pt}{\isacharparenright}{\kern0pt}{\isacharparenright}{\kern0pt}\ {\isadigit{4}}\ {\isadigit{4}}{\isacharcomma}{\kern0pt}\ nna\ {\isacharparenleft}{\kern0pt}{\isasymlambda}{\isacharparenleft}{\kern0pt}n{\isacharcomma}{\kern0pt}\ na{\isacharparenright}{\kern0pt}{\isachardot}{\kern0pt}\ cnj\ {\isacharparenleft}{\kern0pt}SWAP\ {\isachardollar}{\kern0pt}{\isachardollar}{\kern0pt}\ {\isacharparenleft}{\kern0pt}na{\isacharcomma}{\kern0pt}\ n{\isacharparenright}{\kern0pt}{\isacharparenright}{\kern0pt}{\isacharparenright}{\kern0pt}\ {\isacharparenleft}{\kern0pt}{\isasymlambda}{\isacharparenleft}{\kern0pt}n{\isacharcomma}{\kern0pt}\ na{\isacharparenright}{\kern0pt}{\isachardot}{\kern0pt}\ cnj\ {\isacharparenleft}{\kern0pt}SWAP\ {\isachardollar}{\kern0pt}{\isachardollar}{\kern0pt}\ {\isacharparenleft}{\kern0pt}n{\isacharcomma}{\kern0pt}\ na{\isacharparenright}{\kern0pt}{\isacharparenright}{\kern0pt}{\isacharparenright}{\kern0pt}\ {\isadigit{4}}\ {\isadigit{4}}{\isacharparenright}{\kern0pt}\ of\ {\isacharparenleft}{\kern0pt}n{\isacharcomma}{\kern0pt}\ na{\isacharparenright}{\kern0pt}\ {\isasymRightarrow}\ if\ n\ {\isacharequal}{\kern0pt}\ {\isadigit{0}}\ {\isasymand}\ na\ {\isacharequal}{\kern0pt}\ {\isadigit{0}}\ then\ {\isadigit{1}}\ else\ if\ n\ {\isacharequal}{\kern0pt}\ {\isadigit{1}}\ {\isasymand}\ na\ {\isacharequal}{\kern0pt}\ {\isadigit{2}}\ then\ {\isadigit{1}}\ else\ if\ n\ {\isacharequal}{\kern0pt}\ {\isadigit{2}}\ {\isasymand}\ na\ {\isacharequal}{\kern0pt}\ {\isadigit{1}}\ then\ {\isadigit{1}}\ else\ if\ n\ {\isacharequal}{\kern0pt}\ {\isadigit{3}}\ {\isasymand}\ na\ {\isacharequal}{\kern0pt}\ {\isadigit{3}}\ then\ {\isadigit{1}}\ else\ {\isadigit{0}}{\isacharparenright}{\kern0pt}{\isachardoublequoteclose}\isanewline
\ \ \ \ \ \ \ \ \ \ \isacommand{by}\isamarkupfalse%
\ {\isacharparenleft}{\kern0pt}smt\ {\isacharparenleft}{\kern0pt}z{\isadigit{3}}{\isacharparenright}{\kern0pt}\ SWAP{\isacharunderscore}{\kern0pt}def\ old{\isachardot}{\kern0pt}prod{\isachardot}{\kern0pt}case{\isacharparenright}{\kern0pt}\isanewline
\ \ \ \ \ \ \ \ \isacommand{then}\isamarkupfalse%
\ \isacommand{have}\isamarkupfalse%
\ {\isachardoublequoteopen}Matrix{\isachardot}{\kern0pt}mat\ {\isadigit{4}}\ {\isadigit{4}}\ {\isacharparenleft}{\kern0pt}{\isasymlambda}{\isacharparenleft}{\kern0pt}n{\isacharcomma}{\kern0pt}\ na{\isacharparenright}{\kern0pt}{\isachardot}{\kern0pt}\ if\ n\ {\isacharequal}{\kern0pt}\ {\isadigit{0}}\ {\isasymand}\ na\ {\isacharequal}{\kern0pt}\ {\isadigit{0}}\ then\ {\isadigit{1}}{\isacharcolon}{\kern0pt}{\isacharcolon}{\kern0pt}complex\ else\ if\ n\ {\isacharequal}{\kern0pt}\ {\isadigit{1}}\ {\isasymand}\ na\ {\isacharequal}{\kern0pt}\ {\isadigit{2}}\ then\ {\isadigit{1}}\ else\ if\ n\ {\isacharequal}{\kern0pt}\ {\isadigit{2}}\ {\isasymand}\ na\ {\isacharequal}{\kern0pt}\ {\isadigit{1}}\ then\ {\isadigit{1}}\ else\ if\ n\ {\isacharequal}{\kern0pt}\ {\isadigit{3}}\ {\isasymand}\ na\ {\isacharequal}{\kern0pt}\ {\isadigit{3}}\ then\ {\isadigit{1}}\ else\ {\isadigit{0}}{\isacharparenright}{\kern0pt}\ {\isachardollar}{\kern0pt}{\isachardollar}{\kern0pt}\ {\isacharparenleft}{\kern0pt}nn\ {\isacharparenleft}{\kern0pt}{\isasymlambda}{\isacharparenleft}{\kern0pt}n{\isacharcomma}{\kern0pt}\ na{\isacharparenright}{\kern0pt}{\isachardot}{\kern0pt}\ cnj\ {\isacharparenleft}{\kern0pt}SWAP\ {\isachardollar}{\kern0pt}{\isachardollar}{\kern0pt}\ {\isacharparenleft}{\kern0pt}na{\isacharcomma}{\kern0pt}\ n{\isacharparenright}{\kern0pt}{\isacharparenright}{\kern0pt}{\isacharparenright}{\kern0pt}\ {\isacharparenleft}{\kern0pt}{\isasymlambda}{\isacharparenleft}{\kern0pt}n{\isacharcomma}{\kern0pt}\ na{\isacharparenright}{\kern0pt}{\isachardot}{\kern0pt}\ cnj\ {\isacharparenleft}{\kern0pt}SWAP\ {\isachardollar}{\kern0pt}{\isachardollar}{\kern0pt}\ {\isacharparenleft}{\kern0pt}n{\isacharcomma}{\kern0pt}\ na{\isacharparenright}{\kern0pt}{\isacharparenright}{\kern0pt}{\isacharparenright}{\kern0pt}\ {\isadigit{4}}\ {\isadigit{4}}{\isacharcomma}{\kern0pt}\ nna\ {\isacharparenleft}{\kern0pt}{\isasymlambda}{\isacharparenleft}{\kern0pt}n{\isacharcomma}{\kern0pt}\ na{\isacharparenright}{\kern0pt}{\isachardot}{\kern0pt}\ cnj\ {\isacharparenleft}{\kern0pt}SWAP\ {\isachardollar}{\kern0pt}{\isachardollar}{\kern0pt}\ {\isacharparenleft}{\kern0pt}na{\isacharcomma}{\kern0pt}\ n{\isacharparenright}{\kern0pt}{\isacharparenright}{\kern0pt}{\isacharparenright}{\kern0pt}\ {\isacharparenleft}{\kern0pt}{\isasymlambda}{\isacharparenleft}{\kern0pt}n{\isacharcomma}{\kern0pt}\ na{\isacharparenright}{\kern0pt}{\isachardot}{\kern0pt}\ cnj\ {\isacharparenleft}{\kern0pt}SWAP\ {\isachardollar}{\kern0pt}{\isachardollar}{\kern0pt}\ {\isacharparenleft}{\kern0pt}n{\isacharcomma}{\kern0pt}\ na{\isacharparenright}{\kern0pt}{\isacharparenright}{\kern0pt}{\isacharparenright}{\kern0pt}\ {\isadigit{4}}\ {\isadigit{4}}{\isacharparenright}{\kern0pt}\ {\isasymnoteq}\ {\isacharparenleft}{\kern0pt}case\ {\isacharparenleft}{\kern0pt}nn\ {\isacharparenleft}{\kern0pt}{\isasymlambda}{\isacharparenleft}{\kern0pt}n{\isacharcomma}{\kern0pt}\ na{\isacharparenright}{\kern0pt}{\isachardot}{\kern0pt}\ cnj\ {\isacharparenleft}{\kern0pt}SWAP\ {\isachardollar}{\kern0pt}{\isachardollar}{\kern0pt}\ {\isacharparenleft}{\kern0pt}na{\isacharcomma}{\kern0pt}\ n{\isacharparenright}{\kern0pt}{\isacharparenright}{\kern0pt}{\isacharparenright}{\kern0pt}\ {\isacharparenleft}{\kern0pt}{\isasymlambda}{\isacharparenleft}{\kern0pt}n{\isacharcomma}{\kern0pt}\ na{\isacharparenright}{\kern0pt}{\isachardot}{\kern0pt}\ cnj\ {\isacharparenleft}{\kern0pt}SWAP\ {\isachardollar}{\kern0pt}{\isachardollar}{\kern0pt}\ {\isacharparenleft}{\kern0pt}n{\isacharcomma}{\kern0pt}\ na{\isacharparenright}{\kern0pt}{\isacharparenright}{\kern0pt}{\isacharparenright}{\kern0pt}\ {\isadigit{4}}\ {\isadigit{4}}{\isacharcomma}{\kern0pt}\ nna\ {\isacharparenleft}{\kern0pt}{\isasymlambda}{\isacharparenleft}{\kern0pt}n{\isacharcomma}{\kern0pt}\ na{\isacharparenright}{\kern0pt}{\isachardot}{\kern0pt}\ cnj\ {\isacharparenleft}{\kern0pt}SWAP\ {\isachardollar}{\kern0pt}{\isachardollar}{\kern0pt}\ {\isacharparenleft}{\kern0pt}na{\isacharcomma}{\kern0pt}\ n{\isacharparenright}{\kern0pt}{\isacharparenright}{\kern0pt}{\isacharparenright}{\kern0pt}\ {\isacharparenleft}{\kern0pt}{\isasymlambda}{\isacharparenleft}{\kern0pt}n{\isacharcomma}{\kern0pt}\ na{\isacharparenright}{\kern0pt}{\isachardot}{\kern0pt}\ cnj\ {\isacharparenleft}{\kern0pt}SWAP\ {\isachardollar}{\kern0pt}{\isachardollar}{\kern0pt}\ {\isacharparenleft}{\kern0pt}n{\isacharcomma}{\kern0pt}\ na{\isacharparenright}{\kern0pt}{\isacharparenright}{\kern0pt}{\isacharparenright}{\kern0pt}\ {\isadigit{4}}\ {\isadigit{4}}{\isacharparenright}{\kern0pt}\ of\ {\isacharparenleft}{\kern0pt}n{\isacharcomma}{\kern0pt}\ na{\isacharparenright}{\kern0pt}\ {\isasymRightarrow}\ if\ n\ {\isacharequal}{\kern0pt}\ {\isadigit{0}}\ {\isasymand}\ na\ {\isacharequal}{\kern0pt}\ {\isadigit{0}}\ then\ {\isadigit{1}}\ else\ if\ n\ {\isacharequal}{\kern0pt}\ {\isadigit{1}}\ {\isasymand}\ na\ {\isacharequal}{\kern0pt}\ {\isadigit{2}}\ then\ {\isadigit{1}}\ else\ if\ n\ {\isacharequal}{\kern0pt}\ {\isadigit{2}}\ {\isasymand}\ na\ {\isacharequal}{\kern0pt}\ {\isadigit{1}}\ then\ {\isadigit{1}}\ else\ if\ n\ {\isacharequal}{\kern0pt}\ {\isadigit{3}}\ {\isasymand}\ na\ {\isacharequal}{\kern0pt}\ {\isadigit{3}}\ then\ {\isadigit{1}}\ else\ {\isadigit{0}}{\isacharparenright}{\kern0pt}\ {\isasymor}\ SWAP\ {\isachardollar}{\kern0pt}{\isachardollar}{\kern0pt}\ {\isacharparenleft}{\kern0pt}nna\ {\isacharparenleft}{\kern0pt}{\isasymlambda}{\isacharparenleft}{\kern0pt}n{\isacharcomma}{\kern0pt}\ na{\isacharparenright}{\kern0pt}{\isachardot}{\kern0pt}\ cnj\ {\isacharparenleft}{\kern0pt}SWAP\ {\isachardollar}{\kern0pt}{\isachardollar}{\kern0pt}\ {\isacharparenleft}{\kern0pt}na{\isacharcomma}{\kern0pt}\ n{\isacharparenright}{\kern0pt}{\isacharparenright}{\kern0pt}{\isacharparenright}{\kern0pt}\ {\isacharparenleft}{\kern0pt}{\isasymlambda}{\isacharparenleft}{\kern0pt}n{\isacharcomma}{\kern0pt}\ na{\isacharparenright}{\kern0pt}{\isachardot}{\kern0pt}\ cnj\ {\isacharparenleft}{\kern0pt}SWAP\ {\isachardollar}{\kern0pt}{\isachardollar}{\kern0pt}\ {\isacharparenleft}{\kern0pt}n{\isacharcomma}{\kern0pt}\ na{\isacharparenright}{\kern0pt}{\isacharparenright}{\kern0pt}{\isacharparenright}{\kern0pt}\ {\isadigit{4}}\ {\isadigit{4}}{\isacharcomma}{\kern0pt}\ nn\ {\isacharparenleft}{\kern0pt}{\isasymlambda}{\isacharparenleft}{\kern0pt}n{\isacharcomma}{\kern0pt}\ na{\isacharparenright}{\kern0pt}{\isachardot}{\kern0pt}\ cnj\ {\isacharparenleft}{\kern0pt}SWAP\ {\isachardollar}{\kern0pt}{\isachardollar}{\kern0pt}\ {\isacharparenleft}{\kern0pt}na{\isacharcomma}{\kern0pt}\ n{\isacharparenright}{\kern0pt}{\isacharparenright}{\kern0pt}{\isacharparenright}{\kern0pt}\ {\isacharparenleft}{\kern0pt}{\isasymlambda}{\isacharparenleft}{\kern0pt}n{\isacharcomma}{\kern0pt}\ na{\isacharparenright}{\kern0pt}{\isachardot}{\kern0pt}\ cnj\ {\isacharparenleft}{\kern0pt}SWAP\ {\isachardollar}{\kern0pt}{\isachardollar}{\kern0pt}\ {\isacharparenleft}{\kern0pt}n{\isacharcomma}{\kern0pt}\ na{\isacharparenright}{\kern0pt}{\isacharparenright}{\kern0pt}{\isacharparenright}{\kern0pt}\ {\isadigit{4}}\ {\isadigit{4}}{\isacharparenright}{\kern0pt}\ {\isasymnoteq}\ {\isacharparenleft}{\kern0pt}case\ {\isacharparenleft}{\kern0pt}nna\ {\isacharparenleft}{\kern0pt}{\isasymlambda}{\isacharparenleft}{\kern0pt}n{\isacharcomma}{\kern0pt}\ na{\isacharparenright}{\kern0pt}{\isachardot}{\kern0pt}\ cnj\ {\isacharparenleft}{\kern0pt}SWAP\ {\isachardollar}{\kern0pt}{\isachardollar}{\kern0pt}\ {\isacharparenleft}{\kern0pt}na{\isacharcomma}{\kern0pt}\ n{\isacharparenright}{\kern0pt}{\isacharparenright}{\kern0pt}{\isacharparenright}{\kern0pt}\ {\isacharparenleft}{\kern0pt}{\isasymlambda}{\isacharparenleft}{\kern0pt}n{\isacharcomma}{\kern0pt}\ na{\isacharparenright}{\kern0pt}{\isachardot}{\kern0pt}\ cnj\ {\isacharparenleft}{\kern0pt}SWAP\ {\isachardollar}{\kern0pt}{\isachardollar}{\kern0pt}\ {\isacharparenleft}{\kern0pt}n{\isacharcomma}{\kern0pt}\ na{\isacharparenright}{\kern0pt}{\isacharparenright}{\kern0pt}{\isacharparenright}{\kern0pt}\ {\isadigit{4}}\ {\isadigit{4}}{\isacharcomma}{\kern0pt}\ nn\ {\isacharparenleft}{\kern0pt}{\isasymlambda}{\isacharparenleft}{\kern0pt}n{\isacharcomma}{\kern0pt}\ na{\isacharparenright}{\kern0pt}{\isachardot}{\kern0pt}\ cnj\ {\isacharparenleft}{\kern0pt}SWAP\ {\isachardollar}{\kern0pt}{\isachardollar}{\kern0pt}\ {\isacharparenleft}{\kern0pt}na{\isacharcomma}{\kern0pt}\ n{\isacharparenright}{\kern0pt}{\isacharparenright}{\kern0pt}{\isacharparenright}{\kern0pt}\ {\isacharparenleft}{\kern0pt}{\isasymlambda}{\isacharparenleft}{\kern0pt}n{\isacharcomma}{\kern0pt}\ na{\isacharparenright}{\kern0pt}{\isachardot}{\kern0pt}\ cnj\ {\isacharparenleft}{\kern0pt}SWAP\ {\isachardollar}{\kern0pt}{\isachardollar}{\kern0pt}\ {\isacharparenleft}{\kern0pt}n{\isacharcomma}{\kern0pt}\ na{\isacharparenright}{\kern0pt}{\isacharparenright}{\kern0pt}{\isacharparenright}{\kern0pt}\ {\isadigit{4}}\ {\isadigit{4}}{\isacharparenright}{\kern0pt}\ of\ {\isacharparenleft}{\kern0pt}n{\isacharcomma}{\kern0pt}\ na{\isacharparenright}{\kern0pt}\ {\isasymRightarrow}\ if\ n\ {\isacharequal}{\kern0pt}\ {\isadigit{0}}\ {\isasymand}\ na\ {\isacharequal}{\kern0pt}\ {\isadigit{0}}\ then\ {\isadigit{1}}\ else\ if\ n\ {\isacharequal}{\kern0pt}\ {\isadigit{1}}\ {\isasymand}\ na\ {\isacharequal}{\kern0pt}\ {\isadigit{2}}\ then\ {\isadigit{1}}\ else\ if\ n\ {\isacharequal}{\kern0pt}\ {\isadigit{2}}\ {\isasymand}\ na\ {\isacharequal}{\kern0pt}\ {\isadigit{1}}\ then\ {\isadigit{1}}\ else\ if\ n\ {\isacharequal}{\kern0pt}\ {\isadigit{3}}\ {\isasymand}\ na\ {\isacharequal}{\kern0pt}\ {\isadigit{3}}\ then\ {\isadigit{1}}\ else\ {\isadigit{0}}{\isacharparenright}{\kern0pt}{\isachardoublequoteclose}\isanewline
\ \ \ \ \ \ \ \ \ \ \isacommand{by}\isamarkupfalse%
\ fastforce\ \isacommand{{\isacharbraceright}{\kern0pt}}\isamarkupfalse%
\isanewline
\ \ \ \ \ \ \isacommand{ultimately}\isamarkupfalse%
\ \isacommand{have}\isamarkupfalse%
\ {\isachardoublequoteopen}SWAP\ {\isachardollar}{\kern0pt}{\isachardollar}{\kern0pt}\ {\isacharparenleft}{\kern0pt}nna\ {\isacharparenleft}{\kern0pt}{\isasymlambda}{\isacharparenleft}{\kern0pt}n{\isacharcomma}{\kern0pt}\ na{\isacharparenright}{\kern0pt}{\isachardot}{\kern0pt}\ cnj\ {\isacharparenleft}{\kern0pt}SWAP\ {\isachardollar}{\kern0pt}{\isachardollar}{\kern0pt}\ {\isacharparenleft}{\kern0pt}na{\isacharcomma}{\kern0pt}\ n{\isacharparenright}{\kern0pt}{\isacharparenright}{\kern0pt}{\isacharparenright}{\kern0pt}\ {\isacharparenleft}{\kern0pt}{\isasymlambda}{\isacharparenleft}{\kern0pt}n{\isacharcomma}{\kern0pt}\ na{\isacharparenright}{\kern0pt}{\isachardot}{\kern0pt}\ cnj\ {\isacharparenleft}{\kern0pt}SWAP\ {\isachardollar}{\kern0pt}{\isachardollar}{\kern0pt}\ {\isacharparenleft}{\kern0pt}n{\isacharcomma}{\kern0pt}\ na{\isacharparenright}{\kern0pt}{\isacharparenright}{\kern0pt}{\isacharparenright}{\kern0pt}\ {\isadigit{4}}\ {\isadigit{4}}{\isacharcomma}{\kern0pt}\ nn\ {\isacharparenleft}{\kern0pt}{\isasymlambda}{\isacharparenleft}{\kern0pt}n{\isacharcomma}{\kern0pt}\ na{\isacharparenright}{\kern0pt}{\isachardot}{\kern0pt}\ cnj\ {\isacharparenleft}{\kern0pt}SWAP\ {\isachardollar}{\kern0pt}{\isachardollar}{\kern0pt}\ {\isacharparenleft}{\kern0pt}na{\isacharcomma}{\kern0pt}\ n{\isacharparenright}{\kern0pt}{\isacharparenright}{\kern0pt}{\isacharparenright}{\kern0pt}\ {\isacharparenleft}{\kern0pt}{\isasymlambda}{\isacharparenleft}{\kern0pt}n{\isacharcomma}{\kern0pt}\ na{\isacharparenright}{\kern0pt}{\isachardot}{\kern0pt}\ cnj\ {\isacharparenleft}{\kern0pt}SWAP\ {\isachardollar}{\kern0pt}{\isachardollar}{\kern0pt}\ {\isacharparenleft}{\kern0pt}n{\isacharcomma}{\kern0pt}\ na{\isacharparenright}{\kern0pt}{\isacharparenright}{\kern0pt}{\isacharparenright}{\kern0pt}\ {\isadigit{4}}\ {\isadigit{4}}{\isacharparenright}{\kern0pt}\ {\isacharequal}{\kern0pt}\ {\isacharparenleft}{\kern0pt}case\ {\isacharparenleft}{\kern0pt}nna\ {\isacharparenleft}{\kern0pt}{\isasymlambda}{\isacharparenleft}{\kern0pt}n{\isacharcomma}{\kern0pt}\ na{\isacharparenright}{\kern0pt}{\isachardot}{\kern0pt}\ cnj\ {\isacharparenleft}{\kern0pt}SWAP\ {\isachardollar}{\kern0pt}{\isachardollar}{\kern0pt}\ {\isacharparenleft}{\kern0pt}na{\isacharcomma}{\kern0pt}\ n{\isacharparenright}{\kern0pt}{\isacharparenright}{\kern0pt}{\isacharparenright}{\kern0pt}\ {\isacharparenleft}{\kern0pt}{\isasymlambda}{\isacharparenleft}{\kern0pt}n{\isacharcomma}{\kern0pt}\ na{\isacharparenright}{\kern0pt}{\isachardot}{\kern0pt}\ cnj\ {\isacharparenleft}{\kern0pt}SWAP\ {\isachardollar}{\kern0pt}{\isachardollar}{\kern0pt}\ {\isacharparenleft}{\kern0pt}n{\isacharcomma}{\kern0pt}\ na{\isacharparenright}{\kern0pt}{\isacharparenright}{\kern0pt}{\isacharparenright}{\kern0pt}\ {\isadigit{4}}\ {\isadigit{4}}{\isacharcomma}{\kern0pt}\ nn\ {\isacharparenleft}{\kern0pt}{\isasymlambda}{\isacharparenleft}{\kern0pt}n{\isacharcomma}{\kern0pt}\ na{\isacharparenright}{\kern0pt}{\isachardot}{\kern0pt}\ cnj\ {\isacharparenleft}{\kern0pt}SWAP\ {\isachardollar}{\kern0pt}{\isachardollar}{\kern0pt}\ {\isacharparenleft}{\kern0pt}na{\isacharcomma}{\kern0pt}\ n{\isacharparenright}{\kern0pt}{\isacharparenright}{\kern0pt}{\isacharparenright}{\kern0pt}\ {\isacharparenleft}{\kern0pt}{\isasymlambda}{\isacharparenleft}{\kern0pt}n{\isacharcomma}{\kern0pt}\ na{\isacharparenright}{\kern0pt}{\isachardot}{\kern0pt}\ cnj\ {\isacharparenleft}{\kern0pt}SWAP\ {\isachardollar}{\kern0pt}{\isachardollar}{\kern0pt}\ {\isacharparenleft}{\kern0pt}n{\isacharcomma}{\kern0pt}\ na{\isacharparenright}{\kern0pt}{\isacharparenright}{\kern0pt}{\isacharparenright}{\kern0pt}\ {\isadigit{4}}\ {\isadigit{4}}{\isacharparenright}{\kern0pt}\ of\ {\isacharparenleft}{\kern0pt}n{\isacharcomma}{\kern0pt}\ na{\isacharparenright}{\kern0pt}\ {\isasymRightarrow}\ if\ n\ {\isacharequal}{\kern0pt}\ {\isadigit{0}}\ {\isasymand}\ na\ {\isacharequal}{\kern0pt}\ {\isadigit{0}}\ then\ {\isadigit{1}}\ else\ if\ n\ {\isacharequal}{\kern0pt}\ {\isadigit{1}}\ {\isasymand}\ na\ {\isacharequal}{\kern0pt}\ {\isadigit{2}}\ then\ {\isadigit{1}}\ else\ if\ n\ {\isacharequal}{\kern0pt}\ {\isadigit{2}}\ {\isasymand}\ na\ {\isacharequal}{\kern0pt}\ {\isadigit{1}}\ then\ {\isadigit{1}}\ else\ if\ n\ {\isacharequal}{\kern0pt}\ {\isadigit{3}}\ {\isasymand}\ na\ {\isacharequal}{\kern0pt}\ {\isadigit{3}}\ then\ {\isadigit{1}}\ else\ {\isadigit{0}}{\isacharparenright}{\kern0pt}\ {\isasymand}\ Matrix{\isachardot}{\kern0pt}mat\ {\isadigit{4}}\ {\isadigit{4}}\ {\isacharparenleft}{\kern0pt}{\isasymlambda}{\isacharparenleft}{\kern0pt}n{\isacharcomma}{\kern0pt}\ na{\isacharparenright}{\kern0pt}{\isachardot}{\kern0pt}\ if\ n\ {\isacharequal}{\kern0pt}\ {\isadigit{0}}\ {\isasymand}\ na\ {\isacharequal}{\kern0pt}\ {\isadigit{0}}\ then\ {\isadigit{1}}{\isacharcolon}{\kern0pt}{\isacharcolon}{\kern0pt}complex\ else\ if\ n\ {\isacharequal}{\kern0pt}\ {\isadigit{1}}\ {\isasymand}\ na\ {\isacharequal}{\kern0pt}\ {\isadigit{2}}\ then\ {\isadigit{1}}\ else\ if\ n\ {\isacharequal}{\kern0pt}\ {\isadigit{2}}\ {\isasymand}\ na\ {\isacharequal}{\kern0pt}\ {\isadigit{1}}\ then\ {\isadigit{1}}\ else\ if\ n\ {\isacharequal}{\kern0pt}\ {\isadigit{3}}\ {\isasymand}\ na\ {\isacharequal}{\kern0pt}\ {\isadigit{3}}\ then\ {\isadigit{1}}\ else\ {\isadigit{0}}{\isacharparenright}{\kern0pt}\ {\isachardollar}{\kern0pt}{\isachardollar}{\kern0pt}\ {\isacharparenleft}{\kern0pt}nn\ {\isacharparenleft}{\kern0pt}{\isasymlambda}{\isacharparenleft}{\kern0pt}n{\isacharcomma}{\kern0pt}\ na{\isacharparenright}{\kern0pt}{\isachardot}{\kern0pt}\ cnj\ {\isacharparenleft}{\kern0pt}SWAP\ {\isachardollar}{\kern0pt}{\isachardollar}{\kern0pt}\ {\isacharparenleft}{\kern0pt}na{\isacharcomma}{\kern0pt}\ n{\isacharparenright}{\kern0pt}{\isacharparenright}{\kern0pt}{\isacharparenright}{\kern0pt}\ {\isacharparenleft}{\kern0pt}{\isasymlambda}{\isacharparenleft}{\kern0pt}n{\isacharcomma}{\kern0pt}\ na{\isacharparenright}{\kern0pt}{\isachardot}{\kern0pt}\ cnj\ {\isacharparenleft}{\kern0pt}SWAP\ {\isachardollar}{\kern0pt}{\isachardollar}{\kern0pt}\ {\isacharparenleft}{\kern0pt}n{\isacharcomma}{\kern0pt}\ na{\isacharparenright}{\kern0pt}{\isacharparenright}{\kern0pt}{\isacharparenright}{\kern0pt}\ {\isadigit{4}}\ {\isadigit{4}}{\isacharcomma}{\kern0pt}\ nna\ {\isacharparenleft}{\kern0pt}{\isasymlambda}{\isacharparenleft}{\kern0pt}n{\isacharcomma}{\kern0pt}\ na{\isacharparenright}{\kern0pt}{\isachardot}{\kern0pt}\ cnj\ {\isacharparenleft}{\kern0pt}SWAP\ {\isachardollar}{\kern0pt}{\isachardollar}{\kern0pt}\ {\isacharparenleft}{\kern0pt}na{\isacharcomma}{\kern0pt}\ n{\isacharparenright}{\kern0pt}{\isacharparenright}{\kern0pt}{\isacharparenright}{\kern0pt}\ {\isacharparenleft}{\kern0pt}{\isasymlambda}{\isacharparenleft}{\kern0pt}n{\isacharcomma}{\kern0pt}\ na{\isacharparenright}{\kern0pt}{\isachardot}{\kern0pt}\ cnj\ {\isacharparenleft}{\kern0pt}SWAP\ {\isachardollar}{\kern0pt}{\isachardollar}{\kern0pt}\ {\isacharparenleft}{\kern0pt}n{\isacharcomma}{\kern0pt}\ na{\isacharparenright}{\kern0pt}{\isacharparenright}{\kern0pt}{\isacharparenright}{\kern0pt}\ {\isadigit{4}}\ {\isadigit{4}}{\isacharparenright}{\kern0pt}\ {\isacharequal}{\kern0pt}\ {\isacharparenleft}{\kern0pt}case\ {\isacharparenleft}{\kern0pt}nn\ {\isacharparenleft}{\kern0pt}{\isasymlambda}{\isacharparenleft}{\kern0pt}n{\isacharcomma}{\kern0pt}\ na{\isacharparenright}{\kern0pt}{\isachardot}{\kern0pt}\ cnj\ {\isacharparenleft}{\kern0pt}SWAP\ {\isachardollar}{\kern0pt}{\isachardollar}{\kern0pt}\ {\isacharparenleft}{\kern0pt}na{\isacharcomma}{\kern0pt}\ n{\isacharparenright}{\kern0pt}{\isacharparenright}{\kern0pt}{\isacharparenright}{\kern0pt}\ {\isacharparenleft}{\kern0pt}{\isasymlambda}{\isacharparenleft}{\kern0pt}n{\isacharcomma}{\kern0pt}\ na{\isacharparenright}{\kern0pt}{\isachardot}{\kern0pt}\ cnj\ {\isacharparenleft}{\kern0pt}SWAP\ {\isachardollar}{\kern0pt}{\isachardollar}{\kern0pt}\ {\isacharparenleft}{\kern0pt}n{\isacharcomma}{\kern0pt}\ na{\isacharparenright}{\kern0pt}{\isacharparenright}{\kern0pt}{\isacharparenright}{\kern0pt}\ {\isadigit{4}}\ {\isadigit{4}}{\isacharcomma}{\kern0pt}\ nna\ {\isacharparenleft}{\kern0pt}{\isasymlambda}{\isacharparenleft}{\kern0pt}n{\isacharcomma}{\kern0pt}\ na{\isacharparenright}{\kern0pt}{\isachardot}{\kern0pt}\ cnj\ {\isacharparenleft}{\kern0pt}SWAP\ {\isachardollar}{\kern0pt}{\isachardollar}{\kern0pt}\ {\isacharparenleft}{\kern0pt}na{\isacharcomma}{\kern0pt}\ n{\isacharparenright}{\kern0pt}{\isacharparenright}{\kern0pt}{\isacharparenright}{\kern0pt}\ {\isacharparenleft}{\kern0pt}{\isasymlambda}{\isacharparenleft}{\kern0pt}n{\isacharcomma}{\kern0pt}\ na{\isacharparenright}{\kern0pt}{\isachardot}{\kern0pt}\ cnj\ {\isacharparenleft}{\kern0pt}SWAP\ {\isachardollar}{\kern0pt}{\isachardollar}{\kern0pt}\ {\isacharparenleft}{\kern0pt}n{\isacharcomma}{\kern0pt}\ na{\isacharparenright}{\kern0pt}{\isacharparenright}{\kern0pt}{\isacharparenright}{\kern0pt}\ {\isadigit{4}}\ {\isadigit{4}}{\isacharparenright}{\kern0pt}\ of\ {\isacharparenleft}{\kern0pt}n{\isacharcomma}{\kern0pt}\ na{\isacharparenright}{\kern0pt}\ {\isasymRightarrow}\ if\ n\ {\isacharequal}{\kern0pt}\ {\isadigit{0}}\ {\isasymand}\ na\ {\isacharequal}{\kern0pt}\ {\isadigit{0}}\ then\ {\isadigit{1}}\ else\ if\ n\ {\isacharequal}{\kern0pt}\ {\isadigit{1}}\ {\isasymand}\ na\ {\isacharequal}{\kern0pt}\ {\isadigit{2}}\ then\ {\isadigit{1}}\ else\ if\ n\ {\isacharequal}{\kern0pt}\ {\isadigit{2}}\ {\isasymand}\ na\ {\isacharequal}{\kern0pt}\ {\isadigit{1}}\ then\ {\isadigit{1}}\ else\ if\ n\ {\isacharequal}{\kern0pt}\ {\isadigit{3}}\ {\isasymand}\ na\ {\isacharequal}{\kern0pt}\ {\isadigit{3}}\ then\ {\isadigit{1}}\ else\ {\isadigit{0}}{\isacharparenright}{\kern0pt}\ {\isasymlongrightarrow}\ {\isasymnot}\ nn\ {\isacharparenleft}{\kern0pt}{\isasymlambda}{\isacharparenleft}{\kern0pt}n{\isacharcomma}{\kern0pt}\ na{\isacharparenright}{\kern0pt}{\isachardot}{\kern0pt}\ cnj\ {\isacharparenleft}{\kern0pt}SWAP\ {\isachardollar}{\kern0pt}{\isachardollar}{\kern0pt}\ {\isacharparenleft}{\kern0pt}na{\isacharcomma}{\kern0pt}\ n{\isacharparenright}{\kern0pt}{\isacharparenright}{\kern0pt}{\isacharparenright}{\kern0pt}\ {\isacharparenleft}{\kern0pt}{\isasymlambda}{\isacharparenleft}{\kern0pt}n{\isacharcomma}{\kern0pt}\ na{\isacharparenright}{\kern0pt}{\isachardot}{\kern0pt}\ cnj\ {\isacharparenleft}{\kern0pt}SWAP\ {\isachardollar}{\kern0pt}{\isachardollar}{\kern0pt}\ {\isacharparenleft}{\kern0pt}n{\isacharcomma}{\kern0pt}\ na{\isacharparenright}{\kern0pt}{\isacharparenright}{\kern0pt}{\isacharparenright}{\kern0pt}\ {\isadigit{4}}\ {\isadigit{4}}\ {\isacharless}{\kern0pt}\ {\isadigit{4}}\ {\isasymor}\ {\isasymnot}\ nna\ {\isacharparenleft}{\kern0pt}{\isasymlambda}{\isacharparenleft}{\kern0pt}n{\isacharcomma}{\kern0pt}\ na{\isacharparenright}{\kern0pt}{\isachardot}{\kern0pt}\ cnj\ {\isacharparenleft}{\kern0pt}SWAP\ {\isachardollar}{\kern0pt}{\isachardollar}{\kern0pt}\ {\isacharparenleft}{\kern0pt}na{\isacharcomma}{\kern0pt}\ n{\isacharparenright}{\kern0pt}{\isacharparenright}{\kern0pt}{\isacharparenright}{\kern0pt}\ {\isacharparenleft}{\kern0pt}{\isasymlambda}{\isacharparenleft}{\kern0pt}n{\isacharcomma}{\kern0pt}\ na{\isacharparenright}{\kern0pt}{\isachardot}{\kern0pt}\ cnj\ {\isacharparenleft}{\kern0pt}SWAP\ {\isachardollar}{\kern0pt}{\isachardollar}{\kern0pt}\ {\isacharparenleft}{\kern0pt}n{\isacharcomma}{\kern0pt}\ na{\isacharparenright}{\kern0pt}{\isacharparenright}{\kern0pt}{\isacharparenright}{\kern0pt}\ {\isadigit{4}}\ {\isadigit{4}}\ {\isacharless}{\kern0pt}\ {\isadigit{4}}\ {\isasymor}\ {\isacharparenleft}{\kern0pt}case\ {\isacharparenleft}{\kern0pt}nn\ {\isacharparenleft}{\kern0pt}{\isasymlambda}{\isacharparenleft}{\kern0pt}n{\isacharcomma}{\kern0pt}\ na{\isacharparenright}{\kern0pt}{\isachardot}{\kern0pt}\ cnj\ {\isacharparenleft}{\kern0pt}SWAP\ {\isachardollar}{\kern0pt}{\isachardollar}{\kern0pt}\ {\isacharparenleft}{\kern0pt}na{\isacharcomma}{\kern0pt}\ n{\isacharparenright}{\kern0pt}{\isacharparenright}{\kern0pt}{\isacharparenright}{\kern0pt}\ {\isacharparenleft}{\kern0pt}{\isasymlambda}{\isacharparenleft}{\kern0pt}n{\isacharcomma}{\kern0pt}\ na{\isacharparenright}{\kern0pt}{\isachardot}{\kern0pt}\ cnj\ {\isacharparenleft}{\kern0pt}SWAP\ {\isachardollar}{\kern0pt}{\isachardollar}{\kern0pt}\ {\isacharparenleft}{\kern0pt}n{\isacharcomma}{\kern0pt}\ na{\isacharparenright}{\kern0pt}{\isacharparenright}{\kern0pt}{\isacharparenright}{\kern0pt}\ {\isadigit{4}}\ {\isadigit{4}}{\isacharcomma}{\kern0pt}\ nna\ {\isacharparenleft}{\kern0pt}{\isasymlambda}{\isacharparenleft}{\kern0pt}n{\isacharcomma}{\kern0pt}\ na{\isacharparenright}{\kern0pt}{\isachardot}{\kern0pt}\ cnj\ {\isacharparenleft}{\kern0pt}SWAP\ {\isachardollar}{\kern0pt}{\isachardollar}{\kern0pt}\ {\isacharparenleft}{\kern0pt}na{\isacharcomma}{\kern0pt}\ n{\isacharparenright}{\kern0pt}{\isacharparenright}{\kern0pt}{\isacharparenright}{\kern0pt}\ {\isacharparenleft}{\kern0pt}{\isasymlambda}{\isacharparenleft}{\kern0pt}n{\isacharcomma}{\kern0pt}\ na{\isacharparenright}{\kern0pt}{\isachardot}{\kern0pt}\ cnj\ {\isacharparenleft}{\kern0pt}SWAP\ {\isachardollar}{\kern0pt}{\isachardollar}{\kern0pt}\ {\isacharparenleft}{\kern0pt}n{\isacharcomma}{\kern0pt}\ na{\isacharparenright}{\kern0pt}{\isacharparenright}{\kern0pt}{\isacharparenright}{\kern0pt}\ {\isadigit{4}}\ {\isadigit{4}}{\isacharparenright}{\kern0pt}\ of\ {\isacharparenleft}{\kern0pt}n{\isacharcomma}{\kern0pt}\ na{\isacharparenright}{\kern0pt}\ {\isasymRightarrow}\ cnj\ {\isacharparenleft}{\kern0pt}SWAP\ {\isachardollar}{\kern0pt}{\isachardollar}{\kern0pt}\ {\isacharparenleft}{\kern0pt}n{\isacharcomma}{\kern0pt}\ na{\isacharparenright}{\kern0pt}{\isacharparenright}{\kern0pt}{\isacharparenright}{\kern0pt}\ {\isacharequal}{\kern0pt}\ {\isacharparenleft}{\kern0pt}case\ {\isacharparenleft}{\kern0pt}nn\ {\isacharparenleft}{\kern0pt}{\isasymlambda}{\isacharparenleft}{\kern0pt}n{\isacharcomma}{\kern0pt}\ na{\isacharparenright}{\kern0pt}{\isachardot}{\kern0pt}\ cnj\ {\isacharparenleft}{\kern0pt}SWAP\ {\isachardollar}{\kern0pt}{\isachardollar}{\kern0pt}\ {\isacharparenleft}{\kern0pt}na{\isacharcomma}{\kern0pt}\ n{\isacharparenright}{\kern0pt}{\isacharparenright}{\kern0pt}{\isacharparenright}{\kern0pt}\ {\isacharparenleft}{\kern0pt}{\isasymlambda}{\isacharparenleft}{\kern0pt}n{\isacharcomma}{\kern0pt}\ na{\isacharparenright}{\kern0pt}{\isachardot}{\kern0pt}\ cnj\ {\isacharparenleft}{\kern0pt}SWAP\ {\isachardollar}{\kern0pt}{\isachardollar}{\kern0pt}\ {\isacharparenleft}{\kern0pt}n{\isacharcomma}{\kern0pt}\ na{\isacharparenright}{\kern0pt}{\isacharparenright}{\kern0pt}{\isacharparenright}{\kern0pt}\ {\isadigit{4}}\ {\isadigit{4}}{\isacharcomma}{\kern0pt}\ nna\ {\isacharparenleft}{\kern0pt}{\isasymlambda}{\isacharparenleft}{\kern0pt}n{\isacharcomma}{\kern0pt}\ na{\isacharparenright}{\kern0pt}{\isachardot}{\kern0pt}\ cnj\ {\isacharparenleft}{\kern0pt}SWAP\ {\isachardollar}{\kern0pt}{\isachardollar}{\kern0pt}\ {\isacharparenleft}{\kern0pt}na{\isacharcomma}{\kern0pt}\ n{\isacharparenright}{\kern0pt}{\isacharparenright}{\kern0pt}{\isacharparenright}{\kern0pt}\ {\isacharparenleft}{\kern0pt}{\isasymlambda}{\isacharparenleft}{\kern0pt}n{\isacharcomma}{\kern0pt}\ na{\isacharparenright}{\kern0pt}{\isachardot}{\kern0pt}\ cnj\ {\isacharparenleft}{\kern0pt}SWAP\ {\isachardollar}{\kern0pt}{\isachardollar}{\kern0pt}\ {\isacharparenleft}{\kern0pt}n{\isacharcomma}{\kern0pt}\ na{\isacharparenright}{\kern0pt}{\isacharparenright}{\kern0pt}{\isacharparenright}{\kern0pt}\ {\isadigit{4}}\ {\isadigit{4}}{\isacharparenright}{\kern0pt}\ of\ {\isacharparenleft}{\kern0pt}n{\isacharcomma}{\kern0pt}\ na{\isacharparenright}{\kern0pt}\ {\isasymRightarrow}\ cnj\ {\isacharparenleft}{\kern0pt}SWAP\ {\isachardollar}{\kern0pt}{\isachardollar}{\kern0pt}\ {\isacharparenleft}{\kern0pt}na{\isacharcomma}{\kern0pt}\ n{\isacharparenright}{\kern0pt}{\isacharparenright}{\kern0pt}{\isacharparenright}{\kern0pt}{\isachardoublequoteclose}\isanewline
\ \ \ \ \ \ \ \ \isacommand{by}\isamarkupfalse%
\ blast\ \isacommand{{\isacharbraceright}{\kern0pt}}\isamarkupfalse%
\isanewline
\ \ \ \ \isacommand{moreover}\isamarkupfalse%
\isanewline
\ \ \ \ \isacommand{{\isacharbraceleft}{\kern0pt}}\isamarkupfalse%
\ \isacommand{assume}\isamarkupfalse%
\ {\isachardoublequoteopen}{\isacharparenleft}{\kern0pt}if\ nn\ {\isacharparenleft}{\kern0pt}{\isasymlambda}{\isacharparenleft}{\kern0pt}na{\isacharcomma}{\kern0pt}\ n{\isacharparenright}{\kern0pt}{\isachardot}{\kern0pt}\ cnj\ {\isacharparenleft}{\kern0pt}SWAP\ {\isachardollar}{\kern0pt}{\isachardollar}{\kern0pt}\ {\isacharparenleft}{\kern0pt}n{\isacharcomma}{\kern0pt}\ na{\isacharparenright}{\kern0pt}{\isacharparenright}{\kern0pt}{\isacharparenright}{\kern0pt}\ {\isacharparenleft}{\kern0pt}{\isasymlambda}{\isacharparenleft}{\kern0pt}na{\isacharcomma}{\kern0pt}\ n{\isacharparenright}{\kern0pt}{\isachardot}{\kern0pt}\ cnj\ {\isacharparenleft}{\kern0pt}SWAP\ {\isachardollar}{\kern0pt}{\isachardollar}{\kern0pt}\ {\isacharparenleft}{\kern0pt}na{\isacharcomma}{\kern0pt}\ n{\isacharparenright}{\kern0pt}{\isacharparenright}{\kern0pt}{\isacharparenright}{\kern0pt}\ {\isadigit{4}}\ {\isadigit{4}}\ {\isacharequal}{\kern0pt}\ {\isadigit{0}}\ {\isasymand}\ nna\ {\isacharparenleft}{\kern0pt}{\isasymlambda}{\isacharparenleft}{\kern0pt}na{\isacharcomma}{\kern0pt}\ n{\isacharparenright}{\kern0pt}{\isachardot}{\kern0pt}\ cnj\ {\isacharparenleft}{\kern0pt}SWAP\ {\isachardollar}{\kern0pt}{\isachardollar}{\kern0pt}\ {\isacharparenleft}{\kern0pt}n{\isacharcomma}{\kern0pt}\ na{\isacharparenright}{\kern0pt}{\isacharparenright}{\kern0pt}{\isacharparenright}{\kern0pt}\ {\isacharparenleft}{\kern0pt}{\isasymlambda}{\isacharparenleft}{\kern0pt}na{\isacharcomma}{\kern0pt}\ n{\isacharparenright}{\kern0pt}{\isachardot}{\kern0pt}\ cnj\ {\isacharparenleft}{\kern0pt}SWAP\ {\isachardollar}{\kern0pt}{\isachardollar}{\kern0pt}\ {\isacharparenleft}{\kern0pt}na{\isacharcomma}{\kern0pt}\ n{\isacharparenright}{\kern0pt}{\isacharparenright}{\kern0pt}{\isacharparenright}{\kern0pt}\ {\isadigit{4}}\ {\isadigit{4}}\ {\isacharequal}{\kern0pt}\ {\isadigit{0}}\ then\ {\isadigit{1}}{\isacharcolon}{\kern0pt}{\isacharcolon}{\kern0pt}complex\ else\ if\ nn\ {\isacharparenleft}{\kern0pt}{\isasymlambda}{\isacharparenleft}{\kern0pt}na{\isacharcomma}{\kern0pt}\ n{\isacharparenright}{\kern0pt}{\isachardot}{\kern0pt}\ cnj\ {\isacharparenleft}{\kern0pt}SWAP\ {\isachardollar}{\kern0pt}{\isachardollar}{\kern0pt}\ {\isacharparenleft}{\kern0pt}n{\isacharcomma}{\kern0pt}\ na{\isacharparenright}{\kern0pt}{\isacharparenright}{\kern0pt}{\isacharparenright}{\kern0pt}\ {\isacharparenleft}{\kern0pt}{\isasymlambda}{\isacharparenleft}{\kern0pt}na{\isacharcomma}{\kern0pt}\ n{\isacharparenright}{\kern0pt}{\isachardot}{\kern0pt}\ cnj\ {\isacharparenleft}{\kern0pt}SWAP\ {\isachardollar}{\kern0pt}{\isachardollar}{\kern0pt}\ {\isacharparenleft}{\kern0pt}na{\isacharcomma}{\kern0pt}\ n{\isacharparenright}{\kern0pt}{\isacharparenright}{\kern0pt}{\isacharparenright}{\kern0pt}\ {\isadigit{4}}\ {\isadigit{4}}\ {\isacharequal}{\kern0pt}\ {\isadigit{1}}\ {\isasymand}\ nna\ {\isacharparenleft}{\kern0pt}{\isasymlambda}{\isacharparenleft}{\kern0pt}na{\isacharcomma}{\kern0pt}\ n{\isacharparenright}{\kern0pt}{\isachardot}{\kern0pt}\ cnj\ {\isacharparenleft}{\kern0pt}SWAP\ {\isachardollar}{\kern0pt}{\isachardollar}{\kern0pt}\ {\isacharparenleft}{\kern0pt}n{\isacharcomma}{\kern0pt}\ na{\isacharparenright}{\kern0pt}{\isacharparenright}{\kern0pt}{\isacharparenright}{\kern0pt}\ {\isacharparenleft}{\kern0pt}{\isasymlambda}{\isacharparenleft}{\kern0pt}na{\isacharcomma}{\kern0pt}\ n{\isacharparenright}{\kern0pt}{\isachardot}{\kern0pt}\ cnj\ {\isacharparenleft}{\kern0pt}SWAP\ {\isachardollar}{\kern0pt}{\isachardollar}{\kern0pt}\ {\isacharparenleft}{\kern0pt}na{\isacharcomma}{\kern0pt}\ n{\isacharparenright}{\kern0pt}{\isacharparenright}{\kern0pt}{\isacharparenright}{\kern0pt}\ {\isadigit{4}}\ {\isadigit{4}}\ {\isacharequal}{\kern0pt}\ {\isadigit{2}}\ then\ {\isadigit{1}}\ else\ if\ nn\ {\isacharparenleft}{\kern0pt}{\isasymlambda}{\isacharparenleft}{\kern0pt}na{\isacharcomma}{\kern0pt}\ n{\isacharparenright}{\kern0pt}{\isachardot}{\kern0pt}\ cnj\ {\isacharparenleft}{\kern0pt}SWAP\ {\isachardollar}{\kern0pt}{\isachardollar}{\kern0pt}\ {\isacharparenleft}{\kern0pt}n{\isacharcomma}{\kern0pt}\ na{\isacharparenright}{\kern0pt}{\isacharparenright}{\kern0pt}{\isacharparenright}{\kern0pt}\ {\isacharparenleft}{\kern0pt}{\isasymlambda}{\isacharparenleft}{\kern0pt}na{\isacharcomma}{\kern0pt}\ n{\isacharparenright}{\kern0pt}{\isachardot}{\kern0pt}\ cnj\ {\isacharparenleft}{\kern0pt}SWAP\ {\isachardollar}{\kern0pt}{\isachardollar}{\kern0pt}\ {\isacharparenleft}{\kern0pt}na{\isacharcomma}{\kern0pt}\ n{\isacharparenright}{\kern0pt}{\isacharparenright}{\kern0pt}{\isacharparenright}{\kern0pt}\ {\isadigit{4}}\ {\isadigit{4}}\ {\isacharequal}{\kern0pt}\ {\isadigit{2}}\ {\isasymand}\ nna\ {\isacharparenleft}{\kern0pt}{\isasymlambda}{\isacharparenleft}{\kern0pt}na{\isacharcomma}{\kern0pt}\ n{\isacharparenright}{\kern0pt}{\isachardot}{\kern0pt}\ cnj\ {\isacharparenleft}{\kern0pt}SWAP\ {\isachardollar}{\kern0pt}{\isachardollar}{\kern0pt}\ {\isacharparenleft}{\kern0pt}n{\isacharcomma}{\kern0pt}\ na{\isacharparenright}{\kern0pt}{\isacharparenright}{\kern0pt}{\isacharparenright}{\kern0pt}\ {\isacharparenleft}{\kern0pt}{\isasymlambda}{\isacharparenleft}{\kern0pt}na{\isacharcomma}{\kern0pt}\ n{\isacharparenright}{\kern0pt}{\isachardot}{\kern0pt}\ cnj\ {\isacharparenleft}{\kern0pt}SWAP\ {\isachardollar}{\kern0pt}{\isachardollar}{\kern0pt}\ {\isacharparenleft}{\kern0pt}na{\isacharcomma}{\kern0pt}\ n{\isacharparenright}{\kern0pt}{\isacharparenright}{\kern0pt}{\isacharparenright}{\kern0pt}\ {\isadigit{4}}\ {\isadigit{4}}\ {\isacharequal}{\kern0pt}\ {\isadigit{1}}\ then\ {\isadigit{1}}\ else\ if\ nn\ {\isacharparenleft}{\kern0pt}{\isasymlambda}{\isacharparenleft}{\kern0pt}na{\isacharcomma}{\kern0pt}\ n{\isacharparenright}{\kern0pt}{\isachardot}{\kern0pt}\ cnj\ {\isacharparenleft}{\kern0pt}SWAP\ {\isachardollar}{\kern0pt}{\isachardollar}{\kern0pt}\ {\isacharparenleft}{\kern0pt}n{\isacharcomma}{\kern0pt}\ na{\isacharparenright}{\kern0pt}{\isacharparenright}{\kern0pt}{\isacharparenright}{\kern0pt}\ {\isacharparenleft}{\kern0pt}{\isasymlambda}{\isacharparenleft}{\kern0pt}na{\isacharcomma}{\kern0pt}\ n{\isacharparenright}{\kern0pt}{\isachardot}{\kern0pt}\ cnj\ {\isacharparenleft}{\kern0pt}SWAP\ {\isachardollar}{\kern0pt}{\isachardollar}{\kern0pt}\ {\isacharparenleft}{\kern0pt}na{\isacharcomma}{\kern0pt}\ n{\isacharparenright}{\kern0pt}{\isacharparenright}{\kern0pt}{\isacharparenright}{\kern0pt}\ {\isadigit{4}}\ {\isadigit{4}}\ {\isacharequal}{\kern0pt}\ {\isadigit{3}}\ {\isasymand}\ nna\ {\isacharparenleft}{\kern0pt}{\isasymlambda}{\isacharparenleft}{\kern0pt}na{\isacharcomma}{\kern0pt}\ n{\isacharparenright}{\kern0pt}{\isachardot}{\kern0pt}\ cnj\ {\isacharparenleft}{\kern0pt}SWAP\ {\isachardollar}{\kern0pt}{\isachardollar}{\kern0pt}\ {\isacharparenleft}{\kern0pt}n{\isacharcomma}{\kern0pt}\ na{\isacharparenright}{\kern0pt}{\isacharparenright}{\kern0pt}{\isacharparenright}{\kern0pt}\ {\isacharparenleft}{\kern0pt}{\isasymlambda}{\isacharparenleft}{\kern0pt}na{\isacharcomma}{\kern0pt}\ n{\isacharparenright}{\kern0pt}{\isachardot}{\kern0pt}\ cnj\ {\isacharparenleft}{\kern0pt}SWAP\ {\isachardollar}{\kern0pt}{\isachardollar}{\kern0pt}\ {\isacharparenleft}{\kern0pt}na{\isacharcomma}{\kern0pt}\ n{\isacharparenright}{\kern0pt}{\isacharparenright}{\kern0pt}{\isacharparenright}{\kern0pt}\ {\isadigit{4}}\ {\isadigit{4}}\ {\isacharequal}{\kern0pt}\ {\isadigit{3}}\ then\ {\isadigit{1}}\ else\ {\isadigit{0}}{\isacharparenright}{\kern0pt}\ {\isacharequal}{\kern0pt}\ {\isadigit{1}}\ {\isasymand}\ {\isacharparenleft}{\kern0pt}if\ nna\ {\isacharparenleft}{\kern0pt}{\isasymlambda}{\isacharparenleft}{\kern0pt}na{\isacharcomma}{\kern0pt}\ n{\isacharparenright}{\kern0pt}{\isachardot}{\kern0pt}\ cnj\ {\isacharparenleft}{\kern0pt}SWAP\ {\isachardollar}{\kern0pt}{\isachardollar}{\kern0pt}\ {\isacharparenleft}{\kern0pt}n{\isacharcomma}{\kern0pt}\ na{\isacharparenright}{\kern0pt}{\isacharparenright}{\kern0pt}{\isacharparenright}{\kern0pt}\ {\isacharparenleft}{\kern0pt}{\isasymlambda}{\isacharparenleft}{\kern0pt}na{\isacharcomma}{\kern0pt}\ n{\isacharparenright}{\kern0pt}{\isachardot}{\kern0pt}\ cnj\ {\isacharparenleft}{\kern0pt}SWAP\ {\isachardollar}{\kern0pt}{\isachardollar}{\kern0pt}\ {\isacharparenleft}{\kern0pt}na{\isacharcomma}{\kern0pt}\ n{\isacharparenright}{\kern0pt}{\isacharparenright}{\kern0pt}{\isacharparenright}{\kern0pt}\ {\isadigit{4}}\ {\isadigit{4}}\ {\isacharequal}{\kern0pt}\ {\isadigit{0}}\ {\isasymand}\ nn\ {\isacharparenleft}{\kern0pt}{\isasymlambda}{\isacharparenleft}{\kern0pt}na{\isacharcomma}{\kern0pt}\ n{\isacharparenright}{\kern0pt}{\isachardot}{\kern0pt}\ cnj\ {\isacharparenleft}{\kern0pt}SWAP\ {\isachardollar}{\kern0pt}{\isachardollar}{\kern0pt}\ {\isacharparenleft}{\kern0pt}n{\isacharcomma}{\kern0pt}\ na{\isacharparenright}{\kern0pt}{\isacharparenright}{\kern0pt}{\isacharparenright}{\kern0pt}\ {\isacharparenleft}{\kern0pt}{\isasymlambda}{\isacharparenleft}{\kern0pt}na{\isacharcomma}{\kern0pt}\ n{\isacharparenright}{\kern0pt}{\isachardot}{\kern0pt}\ cnj\ {\isacharparenleft}{\kern0pt}SWAP\ {\isachardollar}{\kern0pt}{\isachardollar}{\kern0pt}\ {\isacharparenleft}{\kern0pt}na{\isacharcomma}{\kern0pt}\ n{\isacharparenright}{\kern0pt}{\isacharparenright}{\kern0pt}{\isacharparenright}{\kern0pt}\ {\isadigit{4}}\ {\isadigit{4}}\ {\isacharequal}{\kern0pt}\ {\isadigit{0}}\ then\ {\isadigit{1}}{\isacharcolon}{\kern0pt}{\isacharcolon}{\kern0pt}complex\ else\ if\ nna\ {\isacharparenleft}{\kern0pt}{\isasymlambda}{\isacharparenleft}{\kern0pt}na{\isacharcomma}{\kern0pt}\ n{\isacharparenright}{\kern0pt}{\isachardot}{\kern0pt}\ cnj\ {\isacharparenleft}{\kern0pt}SWAP\ {\isachardollar}{\kern0pt}{\isachardollar}{\kern0pt}\ {\isacharparenleft}{\kern0pt}n{\isacharcomma}{\kern0pt}\ na{\isacharparenright}{\kern0pt}{\isacharparenright}{\kern0pt}{\isacharparenright}{\kern0pt}\ {\isacharparenleft}{\kern0pt}{\isasymlambda}{\isacharparenleft}{\kern0pt}na{\isacharcomma}{\kern0pt}\ n{\isacharparenright}{\kern0pt}{\isachardot}{\kern0pt}\ cnj\ {\isacharparenleft}{\kern0pt}SWAP\ {\isachardollar}{\kern0pt}{\isachardollar}{\kern0pt}\ {\isacharparenleft}{\kern0pt}na{\isacharcomma}{\kern0pt}\ n{\isacharparenright}{\kern0pt}{\isacharparenright}{\kern0pt}{\isacharparenright}{\kern0pt}\ {\isadigit{4}}\ {\isadigit{4}}\ {\isacharequal}{\kern0pt}\ {\isadigit{1}}\ {\isasymand}\ nn\ {\isacharparenleft}{\kern0pt}{\isasymlambda}{\isacharparenleft}{\kern0pt}na{\isacharcomma}{\kern0pt}\ n{\isacharparenright}{\kern0pt}{\isachardot}{\kern0pt}\ cnj\ {\isacharparenleft}{\kern0pt}SWAP\ {\isachardollar}{\kern0pt}{\isachardollar}{\kern0pt}\ {\isacharparenleft}{\kern0pt}n{\isacharcomma}{\kern0pt}\ na{\isacharparenright}{\kern0pt}{\isacharparenright}{\kern0pt}{\isacharparenright}{\kern0pt}\ {\isacharparenleft}{\kern0pt}{\isasymlambda}{\isacharparenleft}{\kern0pt}na{\isacharcomma}{\kern0pt}\ n{\isacharparenright}{\kern0pt}{\isachardot}{\kern0pt}\ cnj\ {\isacharparenleft}{\kern0pt}SWAP\ {\isachardollar}{\kern0pt}{\isachardollar}{\kern0pt}\ {\isacharparenleft}{\kern0pt}na{\isacharcomma}{\kern0pt}\ n{\isacharparenright}{\kern0pt}{\isacharparenright}{\kern0pt}{\isacharparenright}{\kern0pt}\ {\isadigit{4}}\ {\isadigit{4}}\ {\isacharequal}{\kern0pt}\ {\isadigit{2}}\ then\ {\isadigit{1}}\ else\ if\ nna\ {\isacharparenleft}{\kern0pt}{\isasymlambda}{\isacharparenleft}{\kern0pt}na{\isacharcomma}{\kern0pt}\ n{\isacharparenright}{\kern0pt}{\isachardot}{\kern0pt}\ cnj\ {\isacharparenleft}{\kern0pt}SWAP\ {\isachardollar}{\kern0pt}{\isachardollar}{\kern0pt}\ {\isacharparenleft}{\kern0pt}n{\isacharcomma}{\kern0pt}\ na{\isacharparenright}{\kern0pt}{\isacharparenright}{\kern0pt}{\isacharparenright}{\kern0pt}\ {\isacharparenleft}{\kern0pt}{\isasymlambda}{\isacharparenleft}{\kern0pt}na{\isacharcomma}{\kern0pt}\ n{\isacharparenright}{\kern0pt}{\isachardot}{\kern0pt}\ cnj\ {\isacharparenleft}{\kern0pt}SWAP\ {\isachardollar}{\kern0pt}{\isachardollar}{\kern0pt}\ {\isacharparenleft}{\kern0pt}na{\isacharcomma}{\kern0pt}\ n{\isacharparenright}{\kern0pt}{\isacharparenright}{\kern0pt}{\isacharparenright}{\kern0pt}\ {\isadigit{4}}\ {\isadigit{4}}\ {\isacharequal}{\kern0pt}\ {\isadigit{2}}\ {\isasymand}\ nn\ {\isacharparenleft}{\kern0pt}{\isasymlambda}{\isacharparenleft}{\kern0pt}na{\isacharcomma}{\kern0pt}\ n{\isacharparenright}{\kern0pt}{\isachardot}{\kern0pt}\ cnj\ {\isacharparenleft}{\kern0pt}SWAP\ {\isachardollar}{\kern0pt}{\isachardollar}{\kern0pt}\ {\isacharparenleft}{\kern0pt}n{\isacharcomma}{\kern0pt}\ na{\isacharparenright}{\kern0pt}{\isacharparenright}{\kern0pt}{\isacharparenright}{\kern0pt}\ {\isacharparenleft}{\kern0pt}{\isasymlambda}{\isacharparenleft}{\kern0pt}na{\isacharcomma}{\kern0pt}\ n{\isacharparenright}{\kern0pt}{\isachardot}{\kern0pt}\ cnj\ {\isacharparenleft}{\kern0pt}SWAP\ {\isachardollar}{\kern0pt}{\isachardollar}{\kern0pt}\ {\isacharparenleft}{\kern0pt}na{\isacharcomma}{\kern0pt}\ n{\isacharparenright}{\kern0pt}{\isacharparenright}{\kern0pt}{\isacharparenright}{\kern0pt}\ {\isadigit{4}}\ {\isadigit{4}}\ {\isacharequal}{\kern0pt}\ {\isadigit{1}}\ then\ {\isadigit{1}}\ else\ if\ nna\ {\isacharparenleft}{\kern0pt}{\isasymlambda}{\isacharparenleft}{\kern0pt}na{\isacharcomma}{\kern0pt}\ n{\isacharparenright}{\kern0pt}{\isachardot}{\kern0pt}\ cnj\ {\isacharparenleft}{\kern0pt}SWAP\ {\isachardollar}{\kern0pt}{\isachardollar}{\kern0pt}\ {\isacharparenleft}{\kern0pt}n{\isacharcomma}{\kern0pt}\ na{\isacharparenright}{\kern0pt}{\isacharparenright}{\kern0pt}{\isacharparenright}{\kern0pt}\ {\isacharparenleft}{\kern0pt}{\isasymlambda}{\isacharparenleft}{\kern0pt}na{\isacharcomma}{\kern0pt}\ n{\isacharparenright}{\kern0pt}{\isachardot}{\kern0pt}\ cnj\ {\isacharparenleft}{\kern0pt}SWAP\ {\isachardollar}{\kern0pt}{\isachardollar}{\kern0pt}\ {\isacharparenleft}{\kern0pt}na{\isacharcomma}{\kern0pt}\ n{\isacharparenright}{\kern0pt}{\isacharparenright}{\kern0pt}{\isacharparenright}{\kern0pt}\ {\isadigit{4}}\ {\isadigit{4}}\ {\isacharequal}{\kern0pt}\ {\isadigit{3}}\ {\isasymand}\ nn\ {\isacharparenleft}{\kern0pt}{\isasymlambda}{\isacharparenleft}{\kern0pt}na{\isacharcomma}{\kern0pt}\ n{\isacharparenright}{\kern0pt}{\isachardot}{\kern0pt}\ cnj\ {\isacharparenleft}{\kern0pt}SWAP\ {\isachardollar}{\kern0pt}{\isachardollar}{\kern0pt}\ {\isacharparenleft}{\kern0pt}n{\isacharcomma}{\kern0pt}\ na{\isacharparenright}{\kern0pt}{\isacharparenright}{\kern0pt}{\isacharparenright}{\kern0pt}\ {\isacharparenleft}{\kern0pt}{\isasymlambda}{\isacharparenleft}{\kern0pt}na{\isacharcomma}{\kern0pt}\ n{\isacharparenright}{\kern0pt}{\isachardot}{\kern0pt}\ cnj\ {\isacharparenleft}{\kern0pt}SWAP\ {\isachardollar}{\kern0pt}{\isachardollar}{\kern0pt}\ {\isacharparenleft}{\kern0pt}na{\isacharcomma}{\kern0pt}\ n{\isacharparenright}{\kern0pt}{\isacharparenright}{\kern0pt}{\isacharparenright}{\kern0pt}\ {\isadigit{4}}\ {\isadigit{4}}\ {\isacharequal}{\kern0pt}\ {\isadigit{3}}\ then\ {\isadigit{1}}\ else\ {\isadigit{0}}{\isacharparenright}{\kern0pt}\ {\isacharequal}{\kern0pt}\ {\isadigit{1}}{\isachardoublequoteclose}\isanewline
\ \ \ \ \ \ \isacommand{moreover}\isamarkupfalse%
\isanewline
\ \ \ \ \ \ \isacommand{{\isacharbraceleft}{\kern0pt}}\isamarkupfalse%
\ \isacommand{assume}\isamarkupfalse%
\ {\isachardoublequoteopen}{\isacharparenleft}{\kern0pt}{\isacharparenleft}{\kern0pt}if\ nn\ {\isacharparenleft}{\kern0pt}{\isasymlambda}{\isacharparenleft}{\kern0pt}n{\isacharcomma}{\kern0pt}\ na{\isacharparenright}{\kern0pt}{\isachardot}{\kern0pt}\ cnj\ {\isacharparenleft}{\kern0pt}SWAP\ {\isachardollar}{\kern0pt}{\isachardollar}{\kern0pt}\ {\isacharparenleft}{\kern0pt}na{\isacharcomma}{\kern0pt}\ n{\isacharparenright}{\kern0pt}{\isacharparenright}{\kern0pt}{\isacharparenright}{\kern0pt}\ {\isacharparenleft}{\kern0pt}{\isasymlambda}{\isacharparenleft}{\kern0pt}n{\isacharcomma}{\kern0pt}\ na{\isacharparenright}{\kern0pt}{\isachardot}{\kern0pt}\ cnj\ {\isacharparenleft}{\kern0pt}SWAP\ {\isachardollar}{\kern0pt}{\isachardollar}{\kern0pt}\ {\isacharparenleft}{\kern0pt}n{\isacharcomma}{\kern0pt}\ na{\isacharparenright}{\kern0pt}{\isacharparenright}{\kern0pt}{\isacharparenright}{\kern0pt}\ {\isadigit{4}}\ {\isadigit{4}}\ {\isacharequal}{\kern0pt}\ {\isadigit{0}}\ {\isasymand}\ nna\ {\isacharparenleft}{\kern0pt}{\isasymlambda}{\isacharparenleft}{\kern0pt}n{\isacharcomma}{\kern0pt}\ na{\isacharparenright}{\kern0pt}{\isachardot}{\kern0pt}\ cnj\ {\isacharparenleft}{\kern0pt}SWAP\ {\isachardollar}{\kern0pt}{\isachardollar}{\kern0pt}\ {\isacharparenleft}{\kern0pt}na{\isacharcomma}{\kern0pt}\ n{\isacharparenright}{\kern0pt}{\isacharparenright}{\kern0pt}{\isacharparenright}{\kern0pt}\ {\isacharparenleft}{\kern0pt}{\isasymlambda}{\isacharparenleft}{\kern0pt}n{\isacharcomma}{\kern0pt}\ na{\isacharparenright}{\kern0pt}{\isachardot}{\kern0pt}\ cnj\ {\isacharparenleft}{\kern0pt}SWAP\ {\isachardollar}{\kern0pt}{\isachardollar}{\kern0pt}\ {\isacharparenleft}{\kern0pt}n{\isacharcomma}{\kern0pt}\ na{\isacharparenright}{\kern0pt}{\isacharparenright}{\kern0pt}{\isacharparenright}{\kern0pt}\ {\isadigit{4}}\ {\isadigit{4}}\ {\isacharequal}{\kern0pt}\ {\isadigit{0}}\ then\ {\isadigit{1}}{\isacharcolon}{\kern0pt}{\isacharcolon}{\kern0pt}complex\ else\ if\ nn\ {\isacharparenleft}{\kern0pt}{\isasymlambda}{\isacharparenleft}{\kern0pt}n{\isacharcomma}{\kern0pt}\ na{\isacharparenright}{\kern0pt}{\isachardot}{\kern0pt}\ cnj\ {\isacharparenleft}{\kern0pt}SWAP\ {\isachardollar}{\kern0pt}{\isachardollar}{\kern0pt}\ {\isacharparenleft}{\kern0pt}na{\isacharcomma}{\kern0pt}\ n{\isacharparenright}{\kern0pt}{\isacharparenright}{\kern0pt}{\isacharparenright}{\kern0pt}\ {\isacharparenleft}{\kern0pt}{\isasymlambda}{\isacharparenleft}{\kern0pt}n{\isacharcomma}{\kern0pt}\ na{\isacharparenright}{\kern0pt}{\isachardot}{\kern0pt}\ cnj\ {\isacharparenleft}{\kern0pt}SWAP\ {\isachardollar}{\kern0pt}{\isachardollar}{\kern0pt}\ {\isacharparenleft}{\kern0pt}n{\isacharcomma}{\kern0pt}\ na{\isacharparenright}{\kern0pt}{\isacharparenright}{\kern0pt}{\isacharparenright}{\kern0pt}\ {\isadigit{4}}\ {\isadigit{4}}\ {\isacharequal}{\kern0pt}\ {\isadigit{1}}\ {\isasymand}\ nna\ {\isacharparenleft}{\kern0pt}{\isasymlambda}{\isacharparenleft}{\kern0pt}n{\isacharcomma}{\kern0pt}\ na{\isacharparenright}{\kern0pt}{\isachardot}{\kern0pt}\ cnj\ {\isacharparenleft}{\kern0pt}SWAP\ {\isachardollar}{\kern0pt}{\isachardollar}{\kern0pt}\ {\isacharparenleft}{\kern0pt}na{\isacharcomma}{\kern0pt}\ n{\isacharparenright}{\kern0pt}{\isacharparenright}{\kern0pt}{\isacharparenright}{\kern0pt}\ {\isacharparenleft}{\kern0pt}{\isasymlambda}{\isacharparenleft}{\kern0pt}n{\isacharcomma}{\kern0pt}\ na{\isacharparenright}{\kern0pt}{\isachardot}{\kern0pt}\ cnj\ {\isacharparenleft}{\kern0pt}SWAP\ {\isachardollar}{\kern0pt}{\isachardollar}{\kern0pt}\ {\isacharparenleft}{\kern0pt}n{\isacharcomma}{\kern0pt}\ na{\isacharparenright}{\kern0pt}{\isacharparenright}{\kern0pt}{\isacharparenright}{\kern0pt}\ {\isadigit{4}}\ {\isadigit{4}}\ {\isacharequal}{\kern0pt}\ {\isadigit{2}}\ then\ {\isadigit{1}}\ else\ if\ nn\ {\isacharparenleft}{\kern0pt}{\isasymlambda}{\isacharparenleft}{\kern0pt}n{\isacharcomma}{\kern0pt}\ na{\isacharparenright}{\kern0pt}{\isachardot}{\kern0pt}\ cnj\ {\isacharparenleft}{\kern0pt}SWAP\ {\isachardollar}{\kern0pt}{\isachardollar}{\kern0pt}\ {\isacharparenleft}{\kern0pt}na{\isacharcomma}{\kern0pt}\ n{\isacharparenright}{\kern0pt}{\isacharparenright}{\kern0pt}{\isacharparenright}{\kern0pt}\ {\isacharparenleft}{\kern0pt}{\isasymlambda}{\isacharparenleft}{\kern0pt}n{\isacharcomma}{\kern0pt}\ na{\isacharparenright}{\kern0pt}{\isachardot}{\kern0pt}\ cnj\ {\isacharparenleft}{\kern0pt}SWAP\ {\isachardollar}{\kern0pt}{\isachardollar}{\kern0pt}\ {\isacharparenleft}{\kern0pt}n{\isacharcomma}{\kern0pt}\ na{\isacharparenright}{\kern0pt}{\isacharparenright}{\kern0pt}{\isacharparenright}{\kern0pt}\ {\isadigit{4}}\ {\isadigit{4}}\ {\isacharequal}{\kern0pt}\ {\isadigit{2}}\ {\isasymand}\ nna\ {\isacharparenleft}{\kern0pt}{\isasymlambda}{\isacharparenleft}{\kern0pt}n{\isacharcomma}{\kern0pt}\ na{\isacharparenright}{\kern0pt}{\isachardot}{\kern0pt}\ cnj\ {\isacharparenleft}{\kern0pt}SWAP\ {\isachardollar}{\kern0pt}{\isachardollar}{\kern0pt}\ {\isacharparenleft}{\kern0pt}na{\isacharcomma}{\kern0pt}\ n{\isacharparenright}{\kern0pt}{\isacharparenright}{\kern0pt}{\isacharparenright}{\kern0pt}\ {\isacharparenleft}{\kern0pt}{\isasymlambda}{\isacharparenleft}{\kern0pt}n{\isacharcomma}{\kern0pt}\ na{\isacharparenright}{\kern0pt}{\isachardot}{\kern0pt}\ cnj\ {\isacharparenleft}{\kern0pt}SWAP\ {\isachardollar}{\kern0pt}{\isachardollar}{\kern0pt}\ {\isacharparenleft}{\kern0pt}n{\isacharcomma}{\kern0pt}\ na{\isacharparenright}{\kern0pt}{\isacharparenright}{\kern0pt}{\isacharparenright}{\kern0pt}\ {\isadigit{4}}\ {\isadigit{4}}\ {\isacharequal}{\kern0pt}\ {\isadigit{1}}\ then\ {\isadigit{1}}\ else\ if\ nn\ {\isacharparenleft}{\kern0pt}{\isasymlambda}{\isacharparenleft}{\kern0pt}n{\isacharcomma}{\kern0pt}\ na{\isacharparenright}{\kern0pt}{\isachardot}{\kern0pt}\ cnj\ {\isacharparenleft}{\kern0pt}SWAP\ {\isachardollar}{\kern0pt}{\isachardollar}{\kern0pt}\ {\isacharparenleft}{\kern0pt}na{\isacharcomma}{\kern0pt}\ n{\isacharparenright}{\kern0pt}{\isacharparenright}{\kern0pt}{\isacharparenright}{\kern0pt}\ {\isacharparenleft}{\kern0pt}{\isasymlambda}{\isacharparenleft}{\kern0pt}n{\isacharcomma}{\kern0pt}\ na{\isacharparenright}{\kern0pt}{\isachardot}{\kern0pt}\ cnj\ {\isacharparenleft}{\kern0pt}SWAP\ {\isachardollar}{\kern0pt}{\isachardollar}{\kern0pt}\ {\isacharparenleft}{\kern0pt}n{\isacharcomma}{\kern0pt}\ na{\isacharparenright}{\kern0pt}{\isacharparenright}{\kern0pt}{\isacharparenright}{\kern0pt}\ {\isadigit{4}}\ {\isadigit{4}}\ {\isacharequal}{\kern0pt}\ {\isadigit{3}}\ {\isasymand}\ nna\ {\isacharparenleft}{\kern0pt}{\isasymlambda}{\isacharparenleft}{\kern0pt}n{\isacharcomma}{\kern0pt}\ na{\isacharparenright}{\kern0pt}{\isachardot}{\kern0pt}\ cnj\ {\isacharparenleft}{\kern0pt}SWAP\ {\isachardollar}{\kern0pt}{\isachardollar}{\kern0pt}\ {\isacharparenleft}{\kern0pt}na{\isacharcomma}{\kern0pt}\ n{\isacharparenright}{\kern0pt}{\isacharparenright}{\kern0pt}{\isacharparenright}{\kern0pt}\ {\isacharparenleft}{\kern0pt}{\isasymlambda}{\isacharparenleft}{\kern0pt}n{\isacharcomma}{\kern0pt}\ na{\isacharparenright}{\kern0pt}{\isachardot}{\kern0pt}\ cnj\ {\isacharparenleft}{\kern0pt}SWAP\ {\isachardollar}{\kern0pt}{\isachardollar}{\kern0pt}\ {\isacharparenleft}{\kern0pt}n{\isacharcomma}{\kern0pt}\ na{\isacharparenright}{\kern0pt}{\isacharparenright}{\kern0pt}{\isacharparenright}{\kern0pt}\ {\isadigit{4}}\ {\isadigit{4}}\ {\isacharequal}{\kern0pt}\ {\isadigit{3}}\ then\ {\isadigit{1}}\ else\ {\isadigit{0}}{\isacharparenright}{\kern0pt}\ {\isacharequal}{\kern0pt}\ {\isadigit{1}}\ {\isasymand}\ {\isacharparenleft}{\kern0pt}if\ nna\ {\isacharparenleft}{\kern0pt}{\isasymlambda}{\isacharparenleft}{\kern0pt}n{\isacharcomma}{\kern0pt}\ na{\isacharparenright}{\kern0pt}{\isachardot}{\kern0pt}\ cnj\ {\isacharparenleft}{\kern0pt}SWAP\ {\isachardollar}{\kern0pt}{\isachardollar}{\kern0pt}\ {\isacharparenleft}{\kern0pt}na{\isacharcomma}{\kern0pt}\ n{\isacharparenright}{\kern0pt}{\isacharparenright}{\kern0pt}{\isacharparenright}{\kern0pt}\ {\isacharparenleft}{\kern0pt}{\isasymlambda}{\isacharparenleft}{\kern0pt}n{\isacharcomma}{\kern0pt}\ na{\isacharparenright}{\kern0pt}{\isachardot}{\kern0pt}\ cnj\ {\isacharparenleft}{\kern0pt}SWAP\ {\isachardollar}{\kern0pt}{\isachardollar}{\kern0pt}\ {\isacharparenleft}{\kern0pt}n{\isacharcomma}{\kern0pt}\ na{\isacharparenright}{\kern0pt}{\isacharparenright}{\kern0pt}{\isacharparenright}{\kern0pt}\ {\isadigit{4}}\ {\isadigit{4}}\ {\isacharequal}{\kern0pt}\ {\isadigit{0}}\ {\isasymand}\ nn\ {\isacharparenleft}{\kern0pt}{\isasymlambda}{\isacharparenleft}{\kern0pt}n{\isacharcomma}{\kern0pt}\ na{\isacharparenright}{\kern0pt}{\isachardot}{\kern0pt}\ cnj\ {\isacharparenleft}{\kern0pt}SWAP\ {\isachardollar}{\kern0pt}{\isachardollar}{\kern0pt}\ {\isacharparenleft}{\kern0pt}na{\isacharcomma}{\kern0pt}\ n{\isacharparenright}{\kern0pt}{\isacharparenright}{\kern0pt}{\isacharparenright}{\kern0pt}\ {\isacharparenleft}{\kern0pt}{\isasymlambda}{\isacharparenleft}{\kern0pt}n{\isacharcomma}{\kern0pt}\ na{\isacharparenright}{\kern0pt}{\isachardot}{\kern0pt}\ cnj\ {\isacharparenleft}{\kern0pt}SWAP\ {\isachardollar}{\kern0pt}{\isachardollar}{\kern0pt}\ {\isacharparenleft}{\kern0pt}n{\isacharcomma}{\kern0pt}\ na{\isacharparenright}{\kern0pt}{\isacharparenright}{\kern0pt}{\isacharparenright}{\kern0pt}\ {\isadigit{4}}\ {\isadigit{4}}\ {\isacharequal}{\kern0pt}\ {\isadigit{0}}\ then\ {\isadigit{1}}{\isacharcolon}{\kern0pt}{\isacharcolon}{\kern0pt}complex\ else\ if\ nna\ {\isacharparenleft}{\kern0pt}{\isasymlambda}{\isacharparenleft}{\kern0pt}n{\isacharcomma}{\kern0pt}\ na{\isacharparenright}{\kern0pt}{\isachardot}{\kern0pt}\ cnj\ {\isacharparenleft}{\kern0pt}SWAP\ {\isachardollar}{\kern0pt}{\isachardollar}{\kern0pt}\ {\isacharparenleft}{\kern0pt}na{\isacharcomma}{\kern0pt}\ n{\isacharparenright}{\kern0pt}{\isacharparenright}{\kern0pt}{\isacharparenright}{\kern0pt}\ {\isacharparenleft}{\kern0pt}{\isasymlambda}{\isacharparenleft}{\kern0pt}n{\isacharcomma}{\kern0pt}\ na{\isacharparenright}{\kern0pt}{\isachardot}{\kern0pt}\ cnj\ {\isacharparenleft}{\kern0pt}SWAP\ {\isachardollar}{\kern0pt}{\isachardollar}{\kern0pt}\ {\isacharparenleft}{\kern0pt}n{\isacharcomma}{\kern0pt}\ na{\isacharparenright}{\kern0pt}{\isacharparenright}{\kern0pt}{\isacharparenright}{\kern0pt}\ {\isadigit{4}}\ {\isadigit{4}}\ {\isacharequal}{\kern0pt}\ {\isadigit{1}}\ {\isasymand}\ nn\ {\isacharparenleft}{\kern0pt}{\isasymlambda}{\isacharparenleft}{\kern0pt}n{\isacharcomma}{\kern0pt}\ na{\isacharparenright}{\kern0pt}{\isachardot}{\kern0pt}\ cnj\ {\isacharparenleft}{\kern0pt}SWAP\ {\isachardollar}{\kern0pt}{\isachardollar}{\kern0pt}\ {\isacharparenleft}{\kern0pt}na{\isacharcomma}{\kern0pt}\ n{\isacharparenright}{\kern0pt}{\isacharparenright}{\kern0pt}{\isacharparenright}{\kern0pt}\ {\isacharparenleft}{\kern0pt}{\isasymlambda}{\isacharparenleft}{\kern0pt}n{\isacharcomma}{\kern0pt}\ na{\isacharparenright}{\kern0pt}{\isachardot}{\kern0pt}\ cnj\ {\isacharparenleft}{\kern0pt}SWAP\ {\isachardollar}{\kern0pt}{\isachardollar}{\kern0pt}\ {\isacharparenleft}{\kern0pt}n{\isacharcomma}{\kern0pt}\ na{\isacharparenright}{\kern0pt}{\isacharparenright}{\kern0pt}{\isacharparenright}{\kern0pt}\ {\isadigit{4}}\ {\isadigit{4}}\ {\isacharequal}{\kern0pt}\ {\isadigit{2}}\ then\ {\isadigit{1}}\ else\ if\ nna\ {\isacharparenleft}{\kern0pt}{\isasymlambda}{\isacharparenleft}{\kern0pt}n{\isacharcomma}{\kern0pt}\ na{\isacharparenright}{\kern0pt}{\isachardot}{\kern0pt}\ cnj\ {\isacharparenleft}{\kern0pt}SWAP\ {\isachardollar}{\kern0pt}{\isachardollar}{\kern0pt}\ {\isacharparenleft}{\kern0pt}na{\isacharcomma}{\kern0pt}\ n{\isacharparenright}{\kern0pt}{\isacharparenright}{\kern0pt}{\isacharparenright}{\kern0pt}\ {\isacharparenleft}{\kern0pt}{\isasymlambda}{\isacharparenleft}{\kern0pt}n{\isacharcomma}{\kern0pt}\ na{\isacharparenright}{\kern0pt}{\isachardot}{\kern0pt}\ cnj\ {\isacharparenleft}{\kern0pt}SWAP\ {\isachardollar}{\kern0pt}{\isachardollar}{\kern0pt}\ {\isacharparenleft}{\kern0pt}n{\isacharcomma}{\kern0pt}\ na{\isacharparenright}{\kern0pt}{\isacharparenright}{\kern0pt}{\isacharparenright}{\kern0pt}\ {\isadigit{4}}\ {\isadigit{4}}\ {\isacharequal}{\kern0pt}\ {\isadigit{2}}\ {\isasymand}\ nn\ {\isacharparenleft}{\kern0pt}{\isasymlambda}{\isacharparenleft}{\kern0pt}n{\isacharcomma}{\kern0pt}\ na{\isacharparenright}{\kern0pt}{\isachardot}{\kern0pt}\ cnj\ {\isacharparenleft}{\kern0pt}SWAP\ {\isachardollar}{\kern0pt}{\isachardollar}{\kern0pt}\ {\isacharparenleft}{\kern0pt}na{\isacharcomma}{\kern0pt}\ n{\isacharparenright}{\kern0pt}{\isacharparenright}{\kern0pt}{\isacharparenright}{\kern0pt}\ {\isacharparenleft}{\kern0pt}{\isasymlambda}{\isacharparenleft}{\kern0pt}n{\isacharcomma}{\kern0pt}\ na{\isacharparenright}{\kern0pt}{\isachardot}{\kern0pt}\ cnj\ {\isacharparenleft}{\kern0pt}SWAP\ {\isachardollar}{\kern0pt}{\isachardollar}{\kern0pt}\ {\isacharparenleft}{\kern0pt}n{\isacharcomma}{\kern0pt}\ na{\isacharparenright}{\kern0pt}{\isacharparenright}{\kern0pt}{\isacharparenright}{\kern0pt}\ {\isadigit{4}}\ {\isadigit{4}}\ {\isacharequal}{\kern0pt}\ {\isadigit{1}}\ then\ {\isadigit{1}}\ else\ if\ nna\ {\isacharparenleft}{\kern0pt}{\isasymlambda}{\isacharparenleft}{\kern0pt}n{\isacharcomma}{\kern0pt}\ na{\isacharparenright}{\kern0pt}{\isachardot}{\kern0pt}\ cnj\ {\isacharparenleft}{\kern0pt}SWAP\ {\isachardollar}{\kern0pt}{\isachardollar}{\kern0pt}\ {\isacharparenleft}{\kern0pt}na{\isacharcomma}{\kern0pt}\ n{\isacharparenright}{\kern0pt}{\isacharparenright}{\kern0pt}{\isacharparenright}{\kern0pt}\ {\isacharparenleft}{\kern0pt}{\isasymlambda}{\isacharparenleft}{\kern0pt}n{\isacharcomma}{\kern0pt}\ na{\isacharparenright}{\kern0pt}{\isachardot}{\kern0pt}\ cnj\ {\isacharparenleft}{\kern0pt}SWAP\ {\isachardollar}{\kern0pt}{\isachardollar}{\kern0pt}\ {\isacharparenleft}{\kern0pt}n{\isacharcomma}{\kern0pt}\ na{\isacharparenright}{\kern0pt}{\isacharparenright}{\kern0pt}{\isacharparenright}{\kern0pt}\ {\isadigit{4}}\ {\isadigit{4}}\ {\isacharequal}{\kern0pt}\ {\isadigit{3}}\ {\isasymand}\ nn\ {\isacharparenleft}{\kern0pt}{\isasymlambda}{\isacharparenleft}{\kern0pt}n{\isacharcomma}{\kern0pt}\ na{\isacharparenright}{\kern0pt}{\isachardot}{\kern0pt}\ cnj\ {\isacharparenleft}{\kern0pt}SWAP\ {\isachardollar}{\kern0pt}{\isachardollar}{\kern0pt}\ {\isacharparenleft}{\kern0pt}na{\isacharcomma}{\kern0pt}\ n{\isacharparenright}{\kern0pt}{\isacharparenright}{\kern0pt}{\isacharparenright}{\kern0pt}\ {\isacharparenleft}{\kern0pt}{\isasymlambda}{\isacharparenleft}{\kern0pt}n{\isacharcomma}{\kern0pt}\ na{\isacharparenright}{\kern0pt}{\isachardot}{\kern0pt}\ cnj\ {\isacharparenleft}{\kern0pt}SWAP\ {\isachardollar}{\kern0pt}{\isachardollar}{\kern0pt}\ {\isacharparenleft}{\kern0pt}n{\isacharcomma}{\kern0pt}\ na{\isacharparenright}{\kern0pt}{\isacharparenright}{\kern0pt}{\isacharparenright}{\kern0pt}\ {\isadigit{4}}\ {\isadigit{4}}\ {\isacharequal}{\kern0pt}\ {\isadigit{3}}\ then\ {\isadigit{1}}\ else\ {\isadigit{0}}{\isacharparenright}{\kern0pt}\ {\isacharequal}{\kern0pt}\ {\isadigit{1}}{\isacharparenright}{\kern0pt}\ {\isasymand}\ {\isacharparenleft}{\kern0pt}case\ {\isacharparenleft}{\kern0pt}nn\ {\isacharparenleft}{\kern0pt}{\isasymlambda}{\isacharparenleft}{\kern0pt}n{\isacharcomma}{\kern0pt}\ na{\isacharparenright}{\kern0pt}{\isachardot}{\kern0pt}\ cnj\ {\isacharparenleft}{\kern0pt}SWAP\ {\isachardollar}{\kern0pt}{\isachardollar}{\kern0pt}\ {\isacharparenleft}{\kern0pt}na{\isacharcomma}{\kern0pt}\ n{\isacharparenright}{\kern0pt}{\isacharparenright}{\kern0pt}{\isacharparenright}{\kern0pt}\ {\isacharparenleft}{\kern0pt}{\isasymlambda}{\isacharparenleft}{\kern0pt}n{\isacharcomma}{\kern0pt}\ na{\isacharparenright}{\kern0pt}{\isachardot}{\kern0pt}\ cnj\ {\isacharparenleft}{\kern0pt}SWAP\ {\isachardollar}{\kern0pt}{\isachardollar}{\kern0pt}\ {\isacharparenleft}{\kern0pt}n{\isacharcomma}{\kern0pt}\ na{\isacharparenright}{\kern0pt}{\isacharparenright}{\kern0pt}{\isacharparenright}{\kern0pt}\ {\isadigit{4}}\ {\isadigit{4}}{\isacharcomma}{\kern0pt}\ nna\ {\isacharparenleft}{\kern0pt}{\isasymlambda}{\isacharparenleft}{\kern0pt}n{\isacharcomma}{\kern0pt}\ na{\isacharparenright}{\kern0pt}{\isachardot}{\kern0pt}\ cnj\ {\isacharparenleft}{\kern0pt}SWAP\ {\isachardollar}{\kern0pt}{\isachardollar}{\kern0pt}\ {\isacharparenleft}{\kern0pt}na{\isacharcomma}{\kern0pt}\ n{\isacharparenright}{\kern0pt}{\isacharparenright}{\kern0pt}{\isacharparenright}{\kern0pt}\ {\isacharparenleft}{\kern0pt}{\isasymlambda}{\isacharparenleft}{\kern0pt}n{\isacharcomma}{\kern0pt}\ na{\isacharparenright}{\kern0pt}{\isachardot}{\kern0pt}\ cnj\ {\isacharparenleft}{\kern0pt}SWAP\ {\isachardollar}{\kern0pt}{\isachardollar}{\kern0pt}\ {\isacharparenleft}{\kern0pt}n{\isacharcomma}{\kern0pt}\ na{\isacharparenright}{\kern0pt}{\isacharparenright}{\kern0pt}{\isacharparenright}{\kern0pt}\ {\isadigit{4}}\ {\isadigit{4}}{\isacharparenright}{\kern0pt}\ of\ {\isacharparenleft}{\kern0pt}n{\isacharcomma}{\kern0pt}\ na{\isacharparenright}{\kern0pt}\ {\isasymRightarrow}\ cnj\ {\isacharparenleft}{\kern0pt}SWAP\ {\isachardollar}{\kern0pt}{\isachardollar}{\kern0pt}\ {\isacharparenleft}{\kern0pt}n{\isacharcomma}{\kern0pt}\ na{\isacharparenright}{\kern0pt}{\isacharparenright}{\kern0pt}{\isacharparenright}{\kern0pt}\ {\isasymnoteq}\ {\isacharparenleft}{\kern0pt}case\ {\isacharparenleft}{\kern0pt}nn\ {\isacharparenleft}{\kern0pt}{\isasymlambda}{\isacharparenleft}{\kern0pt}n{\isacharcomma}{\kern0pt}\ na{\isacharparenright}{\kern0pt}{\isachardot}{\kern0pt}\ cnj\ {\isacharparenleft}{\kern0pt}SWAP\ {\isachardollar}{\kern0pt}{\isachardollar}{\kern0pt}\ {\isacharparenleft}{\kern0pt}na{\isacharcomma}{\kern0pt}\ n{\isacharparenright}{\kern0pt}{\isacharparenright}{\kern0pt}{\isacharparenright}{\kern0pt}\ {\isacharparenleft}{\kern0pt}{\isasymlambda}{\isacharparenleft}{\kern0pt}n{\isacharcomma}{\kern0pt}\ na{\isacharparenright}{\kern0pt}{\isachardot}{\kern0pt}\ cnj\ {\isacharparenleft}{\kern0pt}SWAP\ {\isachardollar}{\kern0pt}{\isachardollar}{\kern0pt}\ {\isacharparenleft}{\kern0pt}n{\isacharcomma}{\kern0pt}\ na{\isacharparenright}{\kern0pt}{\isacharparenright}{\kern0pt}{\isacharparenright}{\kern0pt}\ {\isadigit{4}}\ {\isadigit{4}}{\isacharcomma}{\kern0pt}\ nna\ {\isacharparenleft}{\kern0pt}{\isasymlambda}{\isacharparenleft}{\kern0pt}n{\isacharcomma}{\kern0pt}\ na{\isacharparenright}{\kern0pt}{\isachardot}{\kern0pt}\ cnj\ {\isacharparenleft}{\kern0pt}SWAP\ {\isachardollar}{\kern0pt}{\isachardollar}{\kern0pt}\ {\isacharparenleft}{\kern0pt}na{\isacharcomma}{\kern0pt}\ n{\isacharparenright}{\kern0pt}{\isacharparenright}{\kern0pt}{\isacharparenright}{\kern0pt}\ {\isacharparenleft}{\kern0pt}{\isasymlambda}{\isacharparenleft}{\kern0pt}n{\isacharcomma}{\kern0pt}\ na{\isacharparenright}{\kern0pt}{\isachardot}{\kern0pt}\ cnj\ {\isacharparenleft}{\kern0pt}SWAP\ {\isachardollar}{\kern0pt}{\isachardollar}{\kern0pt}\ {\isacharparenleft}{\kern0pt}n{\isacharcomma}{\kern0pt}\ na{\isacharparenright}{\kern0pt}{\isacharparenright}{\kern0pt}{\isacharparenright}{\kern0pt}\ {\isadigit{4}}\ {\isadigit{4}}{\isacharparenright}{\kern0pt}\ of\ {\isacharparenleft}{\kern0pt}n{\isacharcomma}{\kern0pt}\ na{\isacharparenright}{\kern0pt}\ {\isasymRightarrow}\ cnj\ {\isacharparenleft}{\kern0pt}SWAP\ {\isachardollar}{\kern0pt}{\isachardollar}{\kern0pt}\ {\isacharparenleft}{\kern0pt}na{\isacharcomma}{\kern0pt}\ n{\isacharparenright}{\kern0pt}{\isacharparenright}{\kern0pt}{\isacharparenright}{\kern0pt}{\isachardoublequoteclose}\isanewline
\ \ \ \ \ \ \ \ \isacommand{then}\isamarkupfalse%
\ \isacommand{have}\isamarkupfalse%
\ {\isachardoublequoteopen}{\isacharparenleft}{\kern0pt}{\isacharparenleft}{\kern0pt}if\ nn\ {\isacharparenleft}{\kern0pt}{\isasymlambda}{\isacharparenleft}{\kern0pt}na{\isacharcomma}{\kern0pt}\ n{\isacharparenright}{\kern0pt}{\isachardot}{\kern0pt}\ cnj\ {\isacharparenleft}{\kern0pt}SWAP\ {\isachardollar}{\kern0pt}{\isachardollar}{\kern0pt}\ {\isacharparenleft}{\kern0pt}n{\isacharcomma}{\kern0pt}\ na{\isacharparenright}{\kern0pt}{\isacharparenright}{\kern0pt}{\isacharparenright}{\kern0pt}\ {\isacharparenleft}{\kern0pt}{\isasymlambda}{\isacharparenleft}{\kern0pt}na{\isacharcomma}{\kern0pt}\ n{\isacharparenright}{\kern0pt}{\isachardot}{\kern0pt}\ cnj\ {\isacharparenleft}{\kern0pt}SWAP\ {\isachardollar}{\kern0pt}{\isachardollar}{\kern0pt}\ {\isacharparenleft}{\kern0pt}na{\isacharcomma}{\kern0pt}\ n{\isacharparenright}{\kern0pt}{\isacharparenright}{\kern0pt}{\isacharparenright}{\kern0pt}\ {\isadigit{4}}\ {\isadigit{4}}\ {\isacharequal}{\kern0pt}\ {\isadigit{0}}\ {\isasymand}\ nna\ {\isacharparenleft}{\kern0pt}{\isasymlambda}{\isacharparenleft}{\kern0pt}na{\isacharcomma}{\kern0pt}\ n{\isacharparenright}{\kern0pt}{\isachardot}{\kern0pt}\ cnj\ {\isacharparenleft}{\kern0pt}SWAP\ {\isachardollar}{\kern0pt}{\isachardollar}{\kern0pt}\ {\isacharparenleft}{\kern0pt}n{\isacharcomma}{\kern0pt}\ na{\isacharparenright}{\kern0pt}{\isacharparenright}{\kern0pt}{\isacharparenright}{\kern0pt}\ {\isacharparenleft}{\kern0pt}{\isasymlambda}{\isacharparenleft}{\kern0pt}na{\isacharcomma}{\kern0pt}\ n{\isacharparenright}{\kern0pt}{\isachardot}{\kern0pt}\ cnj\ {\isacharparenleft}{\kern0pt}SWAP\ {\isachardollar}{\kern0pt}{\isachardollar}{\kern0pt}\ {\isacharparenleft}{\kern0pt}na{\isacharcomma}{\kern0pt}\ n{\isacharparenright}{\kern0pt}{\isacharparenright}{\kern0pt}{\isacharparenright}{\kern0pt}\ {\isadigit{4}}\ {\isadigit{4}}\ {\isacharequal}{\kern0pt}\ {\isadigit{0}}\ then\ {\isadigit{1}}{\isacharcolon}{\kern0pt}{\isacharcolon}{\kern0pt}complex\ else\ if\ nn\ {\isacharparenleft}{\kern0pt}{\isasymlambda}{\isacharparenleft}{\kern0pt}na{\isacharcomma}{\kern0pt}\ n{\isacharparenright}{\kern0pt}{\isachardot}{\kern0pt}\ cnj\ {\isacharparenleft}{\kern0pt}SWAP\ {\isachardollar}{\kern0pt}{\isachardollar}{\kern0pt}\ {\isacharparenleft}{\kern0pt}n{\isacharcomma}{\kern0pt}\ na{\isacharparenright}{\kern0pt}{\isacharparenright}{\kern0pt}{\isacharparenright}{\kern0pt}\ {\isacharparenleft}{\kern0pt}{\isasymlambda}{\isacharparenleft}{\kern0pt}na{\isacharcomma}{\kern0pt}\ n{\isacharparenright}{\kern0pt}{\isachardot}{\kern0pt}\ cnj\ {\isacharparenleft}{\kern0pt}SWAP\ {\isachardollar}{\kern0pt}{\isachardollar}{\kern0pt}\ {\isacharparenleft}{\kern0pt}na{\isacharcomma}{\kern0pt}\ n{\isacharparenright}{\kern0pt}{\isacharparenright}{\kern0pt}{\isacharparenright}{\kern0pt}\ {\isadigit{4}}\ {\isadigit{4}}\ {\isacharequal}{\kern0pt}\ {\isadigit{1}}\ {\isasymand}\ nna\ {\isacharparenleft}{\kern0pt}{\isasymlambda}{\isacharparenleft}{\kern0pt}na{\isacharcomma}{\kern0pt}\ n{\isacharparenright}{\kern0pt}{\isachardot}{\kern0pt}\ cnj\ {\isacharparenleft}{\kern0pt}SWAP\ {\isachardollar}{\kern0pt}{\isachardollar}{\kern0pt}\ {\isacharparenleft}{\kern0pt}n{\isacharcomma}{\kern0pt}\ na{\isacharparenright}{\kern0pt}{\isacharparenright}{\kern0pt}{\isacharparenright}{\kern0pt}\ {\isacharparenleft}{\kern0pt}{\isasymlambda}{\isacharparenleft}{\kern0pt}na{\isacharcomma}{\kern0pt}\ n{\isacharparenright}{\kern0pt}{\isachardot}{\kern0pt}\ cnj\ {\isacharparenleft}{\kern0pt}SWAP\ {\isachardollar}{\kern0pt}{\isachardollar}{\kern0pt}\ {\isacharparenleft}{\kern0pt}na{\isacharcomma}{\kern0pt}\ n{\isacharparenright}{\kern0pt}{\isacharparenright}{\kern0pt}{\isacharparenright}{\kern0pt}\ {\isadigit{4}}\ {\isadigit{4}}\ {\isacharequal}{\kern0pt}\ {\isadigit{2}}\ then\ {\isadigit{1}}\ else\ if\ nn\ {\isacharparenleft}{\kern0pt}{\isasymlambda}{\isacharparenleft}{\kern0pt}na{\isacharcomma}{\kern0pt}\ n{\isacharparenright}{\kern0pt}{\isachardot}{\kern0pt}\ cnj\ {\isacharparenleft}{\kern0pt}SWAP\ {\isachardollar}{\kern0pt}{\isachardollar}{\kern0pt}\ {\isacharparenleft}{\kern0pt}n{\isacharcomma}{\kern0pt}\ na{\isacharparenright}{\kern0pt}{\isacharparenright}{\kern0pt}{\isacharparenright}{\kern0pt}\ {\isacharparenleft}{\kern0pt}{\isasymlambda}{\isacharparenleft}{\kern0pt}na{\isacharcomma}{\kern0pt}\ n{\isacharparenright}{\kern0pt}{\isachardot}{\kern0pt}\ cnj\ {\isacharparenleft}{\kern0pt}SWAP\ {\isachardollar}{\kern0pt}{\isachardollar}{\kern0pt}\ {\isacharparenleft}{\kern0pt}na{\isacharcomma}{\kern0pt}\ n{\isacharparenright}{\kern0pt}{\isacharparenright}{\kern0pt}{\isacharparenright}{\kern0pt}\ {\isadigit{4}}\ {\isadigit{4}}\ {\isacharequal}{\kern0pt}\ {\isadigit{2}}\ {\isasymand}\ nna\ {\isacharparenleft}{\kern0pt}{\isasymlambda}{\isacharparenleft}{\kern0pt}na{\isacharcomma}{\kern0pt}\ n{\isacharparenright}{\kern0pt}{\isachardot}{\kern0pt}\ cnj\ {\isacharparenleft}{\kern0pt}SWAP\ {\isachardollar}{\kern0pt}{\isachardollar}{\kern0pt}\ {\isacharparenleft}{\kern0pt}n{\isacharcomma}{\kern0pt}\ na{\isacharparenright}{\kern0pt}{\isacharparenright}{\kern0pt}{\isacharparenright}{\kern0pt}\ {\isacharparenleft}{\kern0pt}{\isasymlambda}{\isacharparenleft}{\kern0pt}na{\isacharcomma}{\kern0pt}\ n{\isacharparenright}{\kern0pt}{\isachardot}{\kern0pt}\ cnj\ {\isacharparenleft}{\kern0pt}SWAP\ {\isachardollar}{\kern0pt}{\isachardollar}{\kern0pt}\ {\isacharparenleft}{\kern0pt}na{\isacharcomma}{\kern0pt}\ n{\isacharparenright}{\kern0pt}{\isacharparenright}{\kern0pt}{\isacharparenright}{\kern0pt}\ {\isadigit{4}}\ {\isadigit{4}}\ {\isacharequal}{\kern0pt}\ {\isadigit{1}}\ then\ {\isadigit{1}}\ else\ if\ nn\ {\isacharparenleft}{\kern0pt}{\isasymlambda}{\isacharparenleft}{\kern0pt}na{\isacharcomma}{\kern0pt}\ n{\isacharparenright}{\kern0pt}{\isachardot}{\kern0pt}\ cnj\ {\isacharparenleft}{\kern0pt}SWAP\ {\isachardollar}{\kern0pt}{\isachardollar}{\kern0pt}\ {\isacharparenleft}{\kern0pt}n{\isacharcomma}{\kern0pt}\ na{\isacharparenright}{\kern0pt}{\isacharparenright}{\kern0pt}{\isacharparenright}{\kern0pt}\ {\isacharparenleft}{\kern0pt}{\isasymlambda}{\isacharparenleft}{\kern0pt}na{\isacharcomma}{\kern0pt}\ n{\isacharparenright}{\kern0pt}{\isachardot}{\kern0pt}\ cnj\ {\isacharparenleft}{\kern0pt}SWAP\ {\isachardollar}{\kern0pt}{\isachardollar}{\kern0pt}\ {\isacharparenleft}{\kern0pt}na{\isacharcomma}{\kern0pt}\ n{\isacharparenright}{\kern0pt}{\isacharparenright}{\kern0pt}{\isacharparenright}{\kern0pt}\ {\isadigit{4}}\ {\isadigit{4}}\ {\isacharequal}{\kern0pt}\ {\isadigit{3}}\ {\isasymand}\ nna\ {\isacharparenleft}{\kern0pt}{\isasymlambda}{\isacharparenleft}{\kern0pt}na{\isacharcomma}{\kern0pt}\ n{\isacharparenright}{\kern0pt}{\isachardot}{\kern0pt}\ cnj\ {\isacharparenleft}{\kern0pt}SWAP\ {\isachardollar}{\kern0pt}{\isachardollar}{\kern0pt}\ {\isacharparenleft}{\kern0pt}n{\isacharcomma}{\kern0pt}\ na{\isacharparenright}{\kern0pt}{\isacharparenright}{\kern0pt}{\isacharparenright}{\kern0pt}\ {\isacharparenleft}{\kern0pt}{\isasymlambda}{\isacharparenleft}{\kern0pt}na{\isacharcomma}{\kern0pt}\ n{\isacharparenright}{\kern0pt}{\isachardot}{\kern0pt}\ cnj\ {\isacharparenleft}{\kern0pt}SWAP\ {\isachardollar}{\kern0pt}{\isachardollar}{\kern0pt}\ {\isacharparenleft}{\kern0pt}na{\isacharcomma}{\kern0pt}\ n{\isacharparenright}{\kern0pt}{\isacharparenright}{\kern0pt}{\isacharparenright}{\kern0pt}\ {\isadigit{4}}\ {\isadigit{4}}\ {\isacharequal}{\kern0pt}\ {\isadigit{3}}\ then\ {\isadigit{1}}\ else\ {\isadigit{0}}{\isacharparenright}{\kern0pt}\ {\isacharequal}{\kern0pt}\ {\isadigit{1}}\ {\isasymand}\ {\isacharparenleft}{\kern0pt}if\ nna\ {\isacharparenleft}{\kern0pt}{\isasymlambda}{\isacharparenleft}{\kern0pt}na{\isacharcomma}{\kern0pt}\ n{\isacharparenright}{\kern0pt}{\isachardot}{\kern0pt}\ cnj\ {\isacharparenleft}{\kern0pt}SWAP\ {\isachardollar}{\kern0pt}{\isachardollar}{\kern0pt}\ {\isacharparenleft}{\kern0pt}n{\isacharcomma}{\kern0pt}\ na{\isacharparenright}{\kern0pt}{\isacharparenright}{\kern0pt}{\isacharparenright}{\kern0pt}\ {\isacharparenleft}{\kern0pt}{\isasymlambda}{\isacharparenleft}{\kern0pt}na{\isacharcomma}{\kern0pt}\ n{\isacharparenright}{\kern0pt}{\isachardot}{\kern0pt}\ cnj\ {\isacharparenleft}{\kern0pt}SWAP\ {\isachardollar}{\kern0pt}{\isachardollar}{\kern0pt}\ {\isacharparenleft}{\kern0pt}na{\isacharcomma}{\kern0pt}\ n{\isacharparenright}{\kern0pt}{\isacharparenright}{\kern0pt}{\isacharparenright}{\kern0pt}\ {\isadigit{4}}\ {\isadigit{4}}\ {\isacharequal}{\kern0pt}\ {\isadigit{0}}\ {\isasymand}\ nn\ {\isacharparenleft}{\kern0pt}{\isasymlambda}{\isacharparenleft}{\kern0pt}na{\isacharcomma}{\kern0pt}\ n{\isacharparenright}{\kern0pt}{\isachardot}{\kern0pt}\ cnj\ {\isacharparenleft}{\kern0pt}SWAP\ {\isachardollar}{\kern0pt}{\isachardollar}{\kern0pt}\ {\isacharparenleft}{\kern0pt}n{\isacharcomma}{\kern0pt}\ na{\isacharparenright}{\kern0pt}{\isacharparenright}{\kern0pt}{\isacharparenright}{\kern0pt}\ {\isacharparenleft}{\kern0pt}{\isasymlambda}{\isacharparenleft}{\kern0pt}na{\isacharcomma}{\kern0pt}\ n{\isacharparenright}{\kern0pt}{\isachardot}{\kern0pt}\ cnj\ {\isacharparenleft}{\kern0pt}SWAP\ {\isachardollar}{\kern0pt}{\isachardollar}{\kern0pt}\ {\isacharparenleft}{\kern0pt}na{\isacharcomma}{\kern0pt}\ n{\isacharparenright}{\kern0pt}{\isacharparenright}{\kern0pt}{\isacharparenright}{\kern0pt}\ {\isadigit{4}}\ {\isadigit{4}}\ {\isacharequal}{\kern0pt}\ {\isadigit{0}}\ then\ {\isadigit{1}}{\isacharcolon}{\kern0pt}{\isacharcolon}{\kern0pt}complex\ else\ if\ nna\ {\isacharparenleft}{\kern0pt}{\isasymlambda}{\isacharparenleft}{\kern0pt}na{\isacharcomma}{\kern0pt}\ n{\isacharparenright}{\kern0pt}{\isachardot}{\kern0pt}\ cnj\ {\isacharparenleft}{\kern0pt}SWAP\ {\isachardollar}{\kern0pt}{\isachardollar}{\kern0pt}\ {\isacharparenleft}{\kern0pt}n{\isacharcomma}{\kern0pt}\ na{\isacharparenright}{\kern0pt}{\isacharparenright}{\kern0pt}{\isacharparenright}{\kern0pt}\ {\isacharparenleft}{\kern0pt}{\isasymlambda}{\isacharparenleft}{\kern0pt}na{\isacharcomma}{\kern0pt}\ n{\isacharparenright}{\kern0pt}{\isachardot}{\kern0pt}\ cnj\ {\isacharparenleft}{\kern0pt}SWAP\ {\isachardollar}{\kern0pt}{\isachardollar}{\kern0pt}\ {\isacharparenleft}{\kern0pt}na{\isacharcomma}{\kern0pt}\ n{\isacharparenright}{\kern0pt}{\isacharparenright}{\kern0pt}{\isacharparenright}{\kern0pt}\ {\isadigit{4}}\ {\isadigit{4}}\ {\isacharequal}{\kern0pt}\ {\isadigit{1}}\ {\isasymand}\ nn\ {\isacharparenleft}{\kern0pt}{\isasymlambda}{\isacharparenleft}{\kern0pt}na{\isacharcomma}{\kern0pt}\ n{\isacharparenright}{\kern0pt}{\isachardot}{\kern0pt}\ cnj\ {\isacharparenleft}{\kern0pt}SWAP\ {\isachardollar}{\kern0pt}{\isachardollar}{\kern0pt}\ {\isacharparenleft}{\kern0pt}n{\isacharcomma}{\kern0pt}\ na{\isacharparenright}{\kern0pt}{\isacharparenright}{\kern0pt}{\isacharparenright}{\kern0pt}\ {\isacharparenleft}{\kern0pt}{\isasymlambda}{\isacharparenleft}{\kern0pt}na{\isacharcomma}{\kern0pt}\ n{\isacharparenright}{\kern0pt}{\isachardot}{\kern0pt}\ cnj\ {\isacharparenleft}{\kern0pt}SWAP\ {\isachardollar}{\kern0pt}{\isachardollar}{\kern0pt}\ {\isacharparenleft}{\kern0pt}na{\isacharcomma}{\kern0pt}\ n{\isacharparenright}{\kern0pt}{\isacharparenright}{\kern0pt}{\isacharparenright}{\kern0pt}\ {\isadigit{4}}\ {\isadigit{4}}\ {\isacharequal}{\kern0pt}\ {\isadigit{2}}\ then\ {\isadigit{1}}\ else\ if\ nna\ {\isacharparenleft}{\kern0pt}{\isasymlambda}{\isacharparenleft}{\kern0pt}na{\isacharcomma}{\kern0pt}\ n{\isacharparenright}{\kern0pt}{\isachardot}{\kern0pt}\ cnj\ {\isacharparenleft}{\kern0pt}SWAP\ {\isachardollar}{\kern0pt}{\isachardollar}{\kern0pt}\ {\isacharparenleft}{\kern0pt}n{\isacharcomma}{\kern0pt}\ na{\isacharparenright}{\kern0pt}{\isacharparenright}{\kern0pt}{\isacharparenright}{\kern0pt}\ {\isacharparenleft}{\kern0pt}{\isasymlambda}{\isacharparenleft}{\kern0pt}na{\isacharcomma}{\kern0pt}\ n{\isacharparenright}{\kern0pt}{\isachardot}{\kern0pt}\ cnj\ {\isacharparenleft}{\kern0pt}SWAP\ {\isachardollar}{\kern0pt}{\isachardollar}{\kern0pt}\ {\isacharparenleft}{\kern0pt}na{\isacharcomma}{\kern0pt}\ n{\isacharparenright}{\kern0pt}{\isacharparenright}{\kern0pt}{\isacharparenright}{\kern0pt}\ {\isadigit{4}}\ {\isadigit{4}}\ {\isacharequal}{\kern0pt}\ {\isadigit{2}}\ {\isasymand}\ nn\ {\isacharparenleft}{\kern0pt}{\isasymlambda}{\isacharparenleft}{\kern0pt}na{\isacharcomma}{\kern0pt}\ n{\isacharparenright}{\kern0pt}{\isachardot}{\kern0pt}\ cnj\ {\isacharparenleft}{\kern0pt}SWAP\ {\isachardollar}{\kern0pt}{\isachardollar}{\kern0pt}\ {\isacharparenleft}{\kern0pt}n{\isacharcomma}{\kern0pt}\ na{\isacharparenright}{\kern0pt}{\isacharparenright}{\kern0pt}{\isacharparenright}{\kern0pt}\ {\isacharparenleft}{\kern0pt}{\isasymlambda}{\isacharparenleft}{\kern0pt}na{\isacharcomma}{\kern0pt}\ n{\isacharparenright}{\kern0pt}{\isachardot}{\kern0pt}\ cnj\ {\isacharparenleft}{\kern0pt}SWAP\ {\isachardollar}{\kern0pt}{\isachardollar}{\kern0pt}\ {\isacharparenleft}{\kern0pt}na{\isacharcomma}{\kern0pt}\ n{\isacharparenright}{\kern0pt}{\isacharparenright}{\kern0pt}{\isacharparenright}{\kern0pt}\ {\isadigit{4}}\ {\isadigit{4}}\ {\isacharequal}{\kern0pt}\ {\isadigit{1}}\ then\ {\isadigit{1}}\ else\ if\ nna\ {\isacharparenleft}{\kern0pt}{\isasymlambda}{\isacharparenleft}{\kern0pt}na{\isacharcomma}{\kern0pt}\ n{\isacharparenright}{\kern0pt}{\isachardot}{\kern0pt}\ cnj\ {\isacharparenleft}{\kern0pt}SWAP\ {\isachardollar}{\kern0pt}{\isachardollar}{\kern0pt}\ {\isacharparenleft}{\kern0pt}n{\isacharcomma}{\kern0pt}\ na{\isacharparenright}{\kern0pt}{\isacharparenright}{\kern0pt}{\isacharparenright}{\kern0pt}\ {\isacharparenleft}{\kern0pt}{\isasymlambda}{\isacharparenleft}{\kern0pt}na{\isacharcomma}{\kern0pt}\ n{\isacharparenright}{\kern0pt}{\isachardot}{\kern0pt}\ cnj\ {\isacharparenleft}{\kern0pt}SWAP\ {\isachardollar}{\kern0pt}{\isachardollar}{\kern0pt}\ {\isacharparenleft}{\kern0pt}na{\isacharcomma}{\kern0pt}\ n{\isacharparenright}{\kern0pt}{\isacharparenright}{\kern0pt}{\isacharparenright}{\kern0pt}\ {\isadigit{4}}\ {\isadigit{4}}\ {\isacharequal}{\kern0pt}\ {\isadigit{3}}\ {\isasymand}\ nn\ {\isacharparenleft}{\kern0pt}{\isasymlambda}{\isacharparenleft}{\kern0pt}na{\isacharcomma}{\kern0pt}\ n{\isacharparenright}{\kern0pt}{\isachardot}{\kern0pt}\ cnj\ {\isacharparenleft}{\kern0pt}SWAP\ {\isachardollar}{\kern0pt}{\isachardollar}{\kern0pt}\ {\isacharparenleft}{\kern0pt}n{\isacharcomma}{\kern0pt}\ na{\isacharparenright}{\kern0pt}{\isacharparenright}{\kern0pt}{\isacharparenright}{\kern0pt}\ {\isacharparenleft}{\kern0pt}{\isasymlambda}{\isacharparenleft}{\kern0pt}na{\isacharcomma}{\kern0pt}\ n{\isacharparenright}{\kern0pt}{\isachardot}{\kern0pt}\ cnj\ {\isacharparenleft}{\kern0pt}SWAP\ {\isachardollar}{\kern0pt}{\isachardollar}{\kern0pt}\ {\isacharparenleft}{\kern0pt}na{\isacharcomma}{\kern0pt}\ n{\isacharparenright}{\kern0pt}{\isacharparenright}{\kern0pt}{\isacharparenright}{\kern0pt}\ {\isadigit{4}}\ {\isadigit{4}}\ {\isacharequal}{\kern0pt}\ {\isadigit{3}}\ then\ {\isadigit{1}}\ else\ {\isadigit{0}}{\isacharparenright}{\kern0pt}\ {\isacharequal}{\kern0pt}\ {\isadigit{1}}{\isacharparenright}{\kern0pt}\ {\isasymand}\ SWAP\ {\isachardollar}{\kern0pt}{\isachardollar}{\kern0pt}\ {\isacharparenleft}{\kern0pt}nna\ {\isacharparenleft}{\kern0pt}{\isasymlambda}{\isacharparenleft}{\kern0pt}na{\isacharcomma}{\kern0pt}\ n{\isacharparenright}{\kern0pt}{\isachardot}{\kern0pt}\ cnj\ {\isacharparenleft}{\kern0pt}SWAP\ {\isachardollar}{\kern0pt}{\isachardollar}{\kern0pt}\ {\isacharparenleft}{\kern0pt}n{\isacharcomma}{\kern0pt}\ na{\isacharparenright}{\kern0pt}{\isacharparenright}{\kern0pt}{\isacharparenright}{\kern0pt}\ {\isacharparenleft}{\kern0pt}{\isasymlambda}{\isacharparenleft}{\kern0pt}na{\isacharcomma}{\kern0pt}\ n{\isacharparenright}{\kern0pt}{\isachardot}{\kern0pt}\ cnj\ {\isacharparenleft}{\kern0pt}SWAP\ {\isachardollar}{\kern0pt}{\isachardollar}{\kern0pt}\ {\isacharparenleft}{\kern0pt}na{\isacharcomma}{\kern0pt}\ n{\isacharparenright}{\kern0pt}{\isacharparenright}{\kern0pt}{\isacharparenright}{\kern0pt}\ {\isadigit{4}}\ {\isadigit{4}}{\isacharcomma}{\kern0pt}\ nn\ {\isacharparenleft}{\kern0pt}{\isasymlambda}{\isacharparenleft}{\kern0pt}na{\isacharcomma}{\kern0pt}\ n{\isacharparenright}{\kern0pt}{\isachardot}{\kern0pt}\ cnj\ {\isacharparenleft}{\kern0pt}SWAP\ {\isachardollar}{\kern0pt}{\isachardollar}{\kern0pt}\ {\isacharparenleft}{\kern0pt}n{\isacharcomma}{\kern0pt}\ na{\isacharparenright}{\kern0pt}{\isacharparenright}{\kern0pt}{\isacharparenright}{\kern0pt}\ {\isacharparenleft}{\kern0pt}{\isasymlambda}{\isacharparenleft}{\kern0pt}na{\isacharcomma}{\kern0pt}\ n{\isacharparenright}{\kern0pt}{\isachardot}{\kern0pt}\ cnj\ {\isacharparenleft}{\kern0pt}SWAP\ {\isachardollar}{\kern0pt}{\isachardollar}{\kern0pt}\ {\isacharparenleft}{\kern0pt}na{\isacharcomma}{\kern0pt}\ n{\isacharparenright}{\kern0pt}{\isacharparenright}{\kern0pt}{\isacharparenright}{\kern0pt}\ {\isadigit{4}}\ {\isadigit{4}}{\isacharparenright}{\kern0pt}\ {\isasymnoteq}\ SWAP\ {\isachardollar}{\kern0pt}{\isachardollar}{\kern0pt}\ {\isacharparenleft}{\kern0pt}nn\ {\isacharparenleft}{\kern0pt}{\isasymlambda}{\isacharparenleft}{\kern0pt}na{\isacharcomma}{\kern0pt}\ n{\isacharparenright}{\kern0pt}{\isachardot}{\kern0pt}\ cnj\ {\isacharparenleft}{\kern0pt}SWAP\ {\isachardollar}{\kern0pt}{\isachardollar}{\kern0pt}\ {\isacharparenleft}{\kern0pt}n{\isacharcomma}{\kern0pt}\ na{\isacharparenright}{\kern0pt}{\isacharparenright}{\kern0pt}{\isacharparenright}{\kern0pt}\ {\isacharparenleft}{\kern0pt}{\isasymlambda}{\isacharparenleft}{\kern0pt}na{\isacharcomma}{\kern0pt}\ n{\isacharparenright}{\kern0pt}{\isachardot}{\kern0pt}\ cnj\ {\isacharparenleft}{\kern0pt}SWAP\ {\isachardollar}{\kern0pt}{\isachardollar}{\kern0pt}\ {\isacharparenleft}{\kern0pt}na{\isacharcomma}{\kern0pt}\ n{\isacharparenright}{\kern0pt}{\isacharparenright}{\kern0pt}{\isacharparenright}{\kern0pt}\ {\isadigit{4}}\ {\isadigit{4}}{\isacharcomma}{\kern0pt}\ nna\ {\isacharparenleft}{\kern0pt}{\isasymlambda}{\isacharparenleft}{\kern0pt}na{\isacharcomma}{\kern0pt}\ n{\isacharparenright}{\kern0pt}{\isachardot}{\kern0pt}\ cnj\ {\isacharparenleft}{\kern0pt}SWAP\ {\isachardollar}{\kern0pt}{\isachardollar}{\kern0pt}\ {\isacharparenleft}{\kern0pt}n{\isacharcomma}{\kern0pt}\ na{\isacharparenright}{\kern0pt}{\isacharparenright}{\kern0pt}{\isacharparenright}{\kern0pt}\ {\isacharparenleft}{\kern0pt}{\isasymlambda}{\isacharparenleft}{\kern0pt}na{\isacharcomma}{\kern0pt}\ n{\isacharparenright}{\kern0pt}{\isachardot}{\kern0pt}\ cnj\ {\isacharparenleft}{\kern0pt}SWAP\ {\isachardollar}{\kern0pt}{\isachardollar}{\kern0pt}\ {\isacharparenleft}{\kern0pt}na{\isacharcomma}{\kern0pt}\ n{\isacharparenright}{\kern0pt}{\isacharparenright}{\kern0pt}{\isacharparenright}{\kern0pt}\ {\isadigit{4}}\ {\isadigit{4}}{\isacharparenright}{\kern0pt}{\isachardoublequoteclose}\isanewline
\ \ \ \ \ \ \ \ \ \ \isacommand{by}\isamarkupfalse%
\ {\isacharparenleft}{\kern0pt}smt\ {\isacharparenleft}{\kern0pt}z{\isadigit{3}}{\isacharparenright}{\kern0pt}\ old{\isachardot}{\kern0pt}prod{\isachardot}{\kern0pt}case{\isacharparenright}{\kern0pt}\isanewline
\ \ \ \ \ \ \ \ \isacommand{then}\isamarkupfalse%
\ \isacommand{have}\isamarkupfalse%
\ {\isachardoublequoteopen}Matrix{\isachardot}{\kern0pt}mat\ {\isadigit{4}}\ {\isadigit{4}}\ {\isacharparenleft}{\kern0pt}{\isasymlambda}{\isacharparenleft}{\kern0pt}n{\isacharcomma}{\kern0pt}\ na{\isacharparenright}{\kern0pt}{\isachardot}{\kern0pt}\ if\ n\ {\isacharequal}{\kern0pt}\ {\isadigit{0}}\ {\isasymand}\ na\ {\isacharequal}{\kern0pt}\ {\isadigit{0}}\ then\ {\isadigit{1}}{\isacharcolon}{\kern0pt}{\isacharcolon}{\kern0pt}complex\ else\ if\ n\ {\isacharequal}{\kern0pt}\ {\isadigit{1}}\ {\isasymand}\ na\ {\isacharequal}{\kern0pt}\ {\isadigit{2}}\ then\ {\isadigit{1}}\ else\ if\ n\ {\isacharequal}{\kern0pt}\ {\isadigit{2}}\ {\isasymand}\ na\ {\isacharequal}{\kern0pt}\ {\isadigit{1}}\ then\ {\isadigit{1}}\ else\ if\ n\ {\isacharequal}{\kern0pt}\ {\isadigit{3}}\ {\isasymand}\ na\ {\isacharequal}{\kern0pt}\ {\isadigit{3}}\ then\ {\isadigit{1}}\ else\ {\isadigit{0}}{\isacharparenright}{\kern0pt}\ {\isachardollar}{\kern0pt}{\isachardollar}{\kern0pt}\ {\isacharparenleft}{\kern0pt}nn\ {\isacharparenleft}{\kern0pt}{\isasymlambda}{\isacharparenleft}{\kern0pt}n{\isacharcomma}{\kern0pt}\ na{\isacharparenright}{\kern0pt}{\isachardot}{\kern0pt}\ cnj\ {\isacharparenleft}{\kern0pt}SWAP\ {\isachardollar}{\kern0pt}{\isachardollar}{\kern0pt}\ {\isacharparenleft}{\kern0pt}na{\isacharcomma}{\kern0pt}\ n{\isacharparenright}{\kern0pt}{\isacharparenright}{\kern0pt}{\isacharparenright}{\kern0pt}\ {\isacharparenleft}{\kern0pt}{\isasymlambda}{\isacharparenleft}{\kern0pt}n{\isacharcomma}{\kern0pt}\ na{\isacharparenright}{\kern0pt}{\isachardot}{\kern0pt}\ cnj\ {\isacharparenleft}{\kern0pt}SWAP\ {\isachardollar}{\kern0pt}{\isachardollar}{\kern0pt}\ {\isacharparenleft}{\kern0pt}n{\isacharcomma}{\kern0pt}\ na{\isacharparenright}{\kern0pt}{\isacharparenright}{\kern0pt}{\isacharparenright}{\kern0pt}\ {\isadigit{4}}\ {\isadigit{4}}{\isacharcomma}{\kern0pt}\ nna\ {\isacharparenleft}{\kern0pt}{\isasymlambda}{\isacharparenleft}{\kern0pt}n{\isacharcomma}{\kern0pt}\ na{\isacharparenright}{\kern0pt}{\isachardot}{\kern0pt}\ cnj\ {\isacharparenleft}{\kern0pt}SWAP\ {\isachardollar}{\kern0pt}{\isachardollar}{\kern0pt}\ {\isacharparenleft}{\kern0pt}na{\isacharcomma}{\kern0pt}\ n{\isacharparenright}{\kern0pt}{\isacharparenright}{\kern0pt}{\isacharparenright}{\kern0pt}\ {\isacharparenleft}{\kern0pt}{\isasymlambda}{\isacharparenleft}{\kern0pt}n{\isacharcomma}{\kern0pt}\ na{\isacharparenright}{\kern0pt}{\isachardot}{\kern0pt}\ cnj\ {\isacharparenleft}{\kern0pt}SWAP\ {\isachardollar}{\kern0pt}{\isachardollar}{\kern0pt}\ {\isacharparenleft}{\kern0pt}n{\isacharcomma}{\kern0pt}\ na{\isacharparenright}{\kern0pt}{\isacharparenright}{\kern0pt}{\isacharparenright}{\kern0pt}\ {\isadigit{4}}\ {\isadigit{4}}{\isacharparenright}{\kern0pt}\ {\isasymnoteq}\ {\isacharparenleft}{\kern0pt}case\ {\isacharparenleft}{\kern0pt}nn\ {\isacharparenleft}{\kern0pt}{\isasymlambda}{\isacharparenleft}{\kern0pt}n{\isacharcomma}{\kern0pt}\ na{\isacharparenright}{\kern0pt}{\isachardot}{\kern0pt}\ cnj\ {\isacharparenleft}{\kern0pt}SWAP\ {\isachardollar}{\kern0pt}{\isachardollar}{\kern0pt}\ {\isacharparenleft}{\kern0pt}na{\isacharcomma}{\kern0pt}\ n{\isacharparenright}{\kern0pt}{\isacharparenright}{\kern0pt}{\isacharparenright}{\kern0pt}\ {\isacharparenleft}{\kern0pt}{\isasymlambda}{\isacharparenleft}{\kern0pt}n{\isacharcomma}{\kern0pt}\ na{\isacharparenright}{\kern0pt}{\isachardot}{\kern0pt}\ cnj\ {\isacharparenleft}{\kern0pt}SWAP\ {\isachardollar}{\kern0pt}{\isachardollar}{\kern0pt}\ {\isacharparenleft}{\kern0pt}n{\isacharcomma}{\kern0pt}\ na{\isacharparenright}{\kern0pt}{\isacharparenright}{\kern0pt}{\isacharparenright}{\kern0pt}\ {\isadigit{4}}\ {\isadigit{4}}{\isacharcomma}{\kern0pt}\ nna\ {\isacharparenleft}{\kern0pt}{\isasymlambda}{\isacharparenleft}{\kern0pt}n{\isacharcomma}{\kern0pt}\ na{\isacharparenright}{\kern0pt}{\isachardot}{\kern0pt}\ cnj\ {\isacharparenleft}{\kern0pt}SWAP\ {\isachardollar}{\kern0pt}{\isachardollar}{\kern0pt}\ {\isacharparenleft}{\kern0pt}na{\isacharcomma}{\kern0pt}\ n{\isacharparenright}{\kern0pt}{\isacharparenright}{\kern0pt}{\isacharparenright}{\kern0pt}\ {\isacharparenleft}{\kern0pt}{\isasymlambda}{\isacharparenleft}{\kern0pt}n{\isacharcomma}{\kern0pt}\ na{\isacharparenright}{\kern0pt}{\isachardot}{\kern0pt}\ cnj\ {\isacharparenleft}{\kern0pt}SWAP\ {\isachardollar}{\kern0pt}{\isachardollar}{\kern0pt}\ {\isacharparenleft}{\kern0pt}n{\isacharcomma}{\kern0pt}\ na{\isacharparenright}{\kern0pt}{\isacharparenright}{\kern0pt}{\isacharparenright}{\kern0pt}\ {\isadigit{4}}\ {\isadigit{4}}{\isacharparenright}{\kern0pt}\ of\ {\isacharparenleft}{\kern0pt}n{\isacharcomma}{\kern0pt}\ na{\isacharparenright}{\kern0pt}\ {\isasymRightarrow}\ if\ n\ {\isacharequal}{\kern0pt}\ {\isadigit{0}}\ {\isasymand}\ na\ {\isacharequal}{\kern0pt}\ {\isadigit{0}}\ then\ {\isadigit{1}}\ else\ if\ n\ {\isacharequal}{\kern0pt}\ {\isadigit{1}}\ {\isasymand}\ na\ {\isacharequal}{\kern0pt}\ {\isadigit{2}}\ then\ {\isadigit{1}}\ else\ if\ n\ {\isacharequal}{\kern0pt}\ {\isadigit{2}}\ {\isasymand}\ na\ {\isacharequal}{\kern0pt}\ {\isadigit{1}}\ then\ {\isadigit{1}}\ else\ if\ n\ {\isacharequal}{\kern0pt}\ {\isadigit{3}}\ {\isasymand}\ na\ {\isacharequal}{\kern0pt}\ {\isadigit{3}}\ then\ {\isadigit{1}}\ else\ {\isadigit{0}}{\isacharparenright}{\kern0pt}\ {\isasymor}\ SWAP\ {\isachardollar}{\kern0pt}{\isachardollar}{\kern0pt}\ {\isacharparenleft}{\kern0pt}nna\ {\isacharparenleft}{\kern0pt}{\isasymlambda}{\isacharparenleft}{\kern0pt}n{\isacharcomma}{\kern0pt}\ na{\isacharparenright}{\kern0pt}{\isachardot}{\kern0pt}\ cnj\ {\isacharparenleft}{\kern0pt}SWAP\ {\isachardollar}{\kern0pt}{\isachardollar}{\kern0pt}\ {\isacharparenleft}{\kern0pt}na{\isacharcomma}{\kern0pt}\ n{\isacharparenright}{\kern0pt}{\isacharparenright}{\kern0pt}{\isacharparenright}{\kern0pt}\ {\isacharparenleft}{\kern0pt}{\isasymlambda}{\isacharparenleft}{\kern0pt}n{\isacharcomma}{\kern0pt}\ na{\isacharparenright}{\kern0pt}{\isachardot}{\kern0pt}\ cnj\ {\isacharparenleft}{\kern0pt}SWAP\ {\isachardollar}{\kern0pt}{\isachardollar}{\kern0pt}\ {\isacharparenleft}{\kern0pt}n{\isacharcomma}{\kern0pt}\ na{\isacharparenright}{\kern0pt}{\isacharparenright}{\kern0pt}{\isacharparenright}{\kern0pt}\ {\isadigit{4}}\ {\isadigit{4}}{\isacharcomma}{\kern0pt}\ nn\ {\isacharparenleft}{\kern0pt}{\isasymlambda}{\isacharparenleft}{\kern0pt}n{\isacharcomma}{\kern0pt}\ na{\isacharparenright}{\kern0pt}{\isachardot}{\kern0pt}\ cnj\ {\isacharparenleft}{\kern0pt}SWAP\ {\isachardollar}{\kern0pt}{\isachardollar}{\kern0pt}\ {\isacharparenleft}{\kern0pt}na{\isacharcomma}{\kern0pt}\ n{\isacharparenright}{\kern0pt}{\isacharparenright}{\kern0pt}{\isacharparenright}{\kern0pt}\ {\isacharparenleft}{\kern0pt}{\isasymlambda}{\isacharparenleft}{\kern0pt}n{\isacharcomma}{\kern0pt}\ na{\isacharparenright}{\kern0pt}{\isachardot}{\kern0pt}\ cnj\ {\isacharparenleft}{\kern0pt}SWAP\ {\isachardollar}{\kern0pt}{\isachardollar}{\kern0pt}\ {\isacharparenleft}{\kern0pt}n{\isacharcomma}{\kern0pt}\ na{\isacharparenright}{\kern0pt}{\isacharparenright}{\kern0pt}{\isacharparenright}{\kern0pt}\ {\isadigit{4}}\ {\isadigit{4}}{\isacharparenright}{\kern0pt}\ {\isasymnoteq}\ {\isacharparenleft}{\kern0pt}case\ {\isacharparenleft}{\kern0pt}nna\ {\isacharparenleft}{\kern0pt}{\isasymlambda}{\isacharparenleft}{\kern0pt}n{\isacharcomma}{\kern0pt}\ na{\isacharparenright}{\kern0pt}{\isachardot}{\kern0pt}\ cnj\ {\isacharparenleft}{\kern0pt}SWAP\ {\isachardollar}{\kern0pt}{\isachardollar}{\kern0pt}\ {\isacharparenleft}{\kern0pt}na{\isacharcomma}{\kern0pt}\ n{\isacharparenright}{\kern0pt}{\isacharparenright}{\kern0pt}{\isacharparenright}{\kern0pt}\ {\isacharparenleft}{\kern0pt}{\isasymlambda}{\isacharparenleft}{\kern0pt}n{\isacharcomma}{\kern0pt}\ na{\isacharparenright}{\kern0pt}{\isachardot}{\kern0pt}\ cnj\ {\isacharparenleft}{\kern0pt}SWAP\ {\isachardollar}{\kern0pt}{\isachardollar}{\kern0pt}\ {\isacharparenleft}{\kern0pt}n{\isacharcomma}{\kern0pt}\ na{\isacharparenright}{\kern0pt}{\isacharparenright}{\kern0pt}{\isacharparenright}{\kern0pt}\ {\isadigit{4}}\ {\isadigit{4}}{\isacharcomma}{\kern0pt}\ nn\ {\isacharparenleft}{\kern0pt}{\isasymlambda}{\isacharparenleft}{\kern0pt}n{\isacharcomma}{\kern0pt}\ na{\isacharparenright}{\kern0pt}{\isachardot}{\kern0pt}\ cnj\ {\isacharparenleft}{\kern0pt}SWAP\ {\isachardollar}{\kern0pt}{\isachardollar}{\kern0pt}\ {\isacharparenleft}{\kern0pt}na{\isacharcomma}{\kern0pt}\ n{\isacharparenright}{\kern0pt}{\isacharparenright}{\kern0pt}{\isacharparenright}{\kern0pt}\ {\isacharparenleft}{\kern0pt}{\isasymlambda}{\isacharparenleft}{\kern0pt}n{\isacharcomma}{\kern0pt}\ na{\isacharparenright}{\kern0pt}{\isachardot}{\kern0pt}\ cnj\ {\isacharparenleft}{\kern0pt}SWAP\ {\isachardollar}{\kern0pt}{\isachardollar}{\kern0pt}\ {\isacharparenleft}{\kern0pt}n{\isacharcomma}{\kern0pt}\ na{\isacharparenright}{\kern0pt}{\isacharparenright}{\kern0pt}{\isacharparenright}{\kern0pt}\ {\isadigit{4}}\ {\isadigit{4}}{\isacharparenright}{\kern0pt}\ of\ {\isacharparenleft}{\kern0pt}n{\isacharcomma}{\kern0pt}\ na{\isacharparenright}{\kern0pt}\ {\isasymRightarrow}\ if\ n\ {\isacharequal}{\kern0pt}\ {\isadigit{0}}\ {\isasymand}\ na\ {\isacharequal}{\kern0pt}\ {\isadigit{0}}\ then\ {\isadigit{1}}\ else\ if\ n\ {\isacharequal}{\kern0pt}\ {\isadigit{1}}\ {\isasymand}\ na\ {\isacharequal}{\kern0pt}\ {\isadigit{2}}\ then\ {\isadigit{1}}\ else\ if\ n\ {\isacharequal}{\kern0pt}\ {\isadigit{2}}\ {\isasymand}\ na\ {\isacharequal}{\kern0pt}\ {\isadigit{1}}\ then\ {\isadigit{1}}\ else\ if\ n\ {\isacharequal}{\kern0pt}\ {\isadigit{3}}\ {\isasymand}\ na\ {\isacharequal}{\kern0pt}\ {\isadigit{3}}\ then\ {\isadigit{1}}\ else\ {\isadigit{0}}{\isacharparenright}{\kern0pt}{\isachardoublequoteclose}\isanewline
\ \ \ \ \ \ \ \ \ \ \isacommand{using}\isamarkupfalse%
\ SWAP{\isacharunderscore}{\kern0pt}def\ \isacommand{by}\isamarkupfalse%
\ auto\ \isacommand{{\isacharbraceright}{\kern0pt}}\isamarkupfalse%
\isanewline
\ \ \ \ \ \ \isacommand{ultimately}\isamarkupfalse%
\ \isacommand{have}\isamarkupfalse%
\ {\isachardoublequoteopen}SWAP\ {\isachardollar}{\kern0pt}{\isachardollar}{\kern0pt}\ {\isacharparenleft}{\kern0pt}nna\ {\isacharparenleft}{\kern0pt}{\isasymlambda}{\isacharparenleft}{\kern0pt}n{\isacharcomma}{\kern0pt}\ na{\isacharparenright}{\kern0pt}{\isachardot}{\kern0pt}\ cnj\ {\isacharparenleft}{\kern0pt}SWAP\ {\isachardollar}{\kern0pt}{\isachardollar}{\kern0pt}\ {\isacharparenleft}{\kern0pt}na{\isacharcomma}{\kern0pt}\ n{\isacharparenright}{\kern0pt}{\isacharparenright}{\kern0pt}{\isacharparenright}{\kern0pt}\ {\isacharparenleft}{\kern0pt}{\isasymlambda}{\isacharparenleft}{\kern0pt}n{\isacharcomma}{\kern0pt}\ na{\isacharparenright}{\kern0pt}{\isachardot}{\kern0pt}\ cnj\ {\isacharparenleft}{\kern0pt}SWAP\ {\isachardollar}{\kern0pt}{\isachardollar}{\kern0pt}\ {\isacharparenleft}{\kern0pt}n{\isacharcomma}{\kern0pt}\ na{\isacharparenright}{\kern0pt}{\isacharparenright}{\kern0pt}{\isacharparenright}{\kern0pt}\ {\isadigit{4}}\ {\isadigit{4}}{\isacharcomma}{\kern0pt}\ nn\ {\isacharparenleft}{\kern0pt}{\isasymlambda}{\isacharparenleft}{\kern0pt}n{\isacharcomma}{\kern0pt}\ na{\isacharparenright}{\kern0pt}{\isachardot}{\kern0pt}\ cnj\ {\isacharparenleft}{\kern0pt}SWAP\ {\isachardollar}{\kern0pt}{\isachardollar}{\kern0pt}\ {\isacharparenleft}{\kern0pt}na{\isacharcomma}{\kern0pt}\ n{\isacharparenright}{\kern0pt}{\isacharparenright}{\kern0pt}{\isacharparenright}{\kern0pt}\ {\isacharparenleft}{\kern0pt}{\isasymlambda}{\isacharparenleft}{\kern0pt}n{\isacharcomma}{\kern0pt}\ na{\isacharparenright}{\kern0pt}{\isachardot}{\kern0pt}\ cnj\ {\isacharparenleft}{\kern0pt}SWAP\ {\isachardollar}{\kern0pt}{\isachardollar}{\kern0pt}\ {\isacharparenleft}{\kern0pt}n{\isacharcomma}{\kern0pt}\ na{\isacharparenright}{\kern0pt}{\isacharparenright}{\kern0pt}{\isacharparenright}{\kern0pt}\ {\isadigit{4}}\ {\isadigit{4}}{\isacharparenright}{\kern0pt}\ {\isacharequal}{\kern0pt}\ {\isacharparenleft}{\kern0pt}case\ {\isacharparenleft}{\kern0pt}nna\ {\isacharparenleft}{\kern0pt}{\isasymlambda}{\isacharparenleft}{\kern0pt}n{\isacharcomma}{\kern0pt}\ na{\isacharparenright}{\kern0pt}{\isachardot}{\kern0pt}\ cnj\ {\isacharparenleft}{\kern0pt}SWAP\ {\isachardollar}{\kern0pt}{\isachardollar}{\kern0pt}\ {\isacharparenleft}{\kern0pt}na{\isacharcomma}{\kern0pt}\ n{\isacharparenright}{\kern0pt}{\isacharparenright}{\kern0pt}{\isacharparenright}{\kern0pt}\ {\isacharparenleft}{\kern0pt}{\isasymlambda}{\isacharparenleft}{\kern0pt}n{\isacharcomma}{\kern0pt}\ na{\isacharparenright}{\kern0pt}{\isachardot}{\kern0pt}\ cnj\ {\isacharparenleft}{\kern0pt}SWAP\ {\isachardollar}{\kern0pt}{\isachardollar}{\kern0pt}\ {\isacharparenleft}{\kern0pt}n{\isacharcomma}{\kern0pt}\ na{\isacharparenright}{\kern0pt}{\isacharparenright}{\kern0pt}{\isacharparenright}{\kern0pt}\ {\isadigit{4}}\ {\isadigit{4}}{\isacharcomma}{\kern0pt}\ nn\ {\isacharparenleft}{\kern0pt}{\isasymlambda}{\isacharparenleft}{\kern0pt}n{\isacharcomma}{\kern0pt}\ na{\isacharparenright}{\kern0pt}{\isachardot}{\kern0pt}\ cnj\ {\isacharparenleft}{\kern0pt}SWAP\ {\isachardollar}{\kern0pt}{\isachardollar}{\kern0pt}\ {\isacharparenleft}{\kern0pt}na{\isacharcomma}{\kern0pt}\ n{\isacharparenright}{\kern0pt}{\isacharparenright}{\kern0pt}{\isacharparenright}{\kern0pt}\ {\isacharparenleft}{\kern0pt}{\isasymlambda}{\isacharparenleft}{\kern0pt}n{\isacharcomma}{\kern0pt}\ na{\isacharparenright}{\kern0pt}{\isachardot}{\kern0pt}\ cnj\ {\isacharparenleft}{\kern0pt}SWAP\ {\isachardollar}{\kern0pt}{\isachardollar}{\kern0pt}\ {\isacharparenleft}{\kern0pt}n{\isacharcomma}{\kern0pt}\ na{\isacharparenright}{\kern0pt}{\isacharparenright}{\kern0pt}{\isacharparenright}{\kern0pt}\ {\isadigit{4}}\ {\isadigit{4}}{\isacharparenright}{\kern0pt}\ of\ {\isacharparenleft}{\kern0pt}n{\isacharcomma}{\kern0pt}\ na{\isacharparenright}{\kern0pt}\ {\isasymRightarrow}\ if\ n\ {\isacharequal}{\kern0pt}\ {\isadigit{0}}\ {\isasymand}\ na\ {\isacharequal}{\kern0pt}\ {\isadigit{0}}\ then\ {\isadigit{1}}\ else\ if\ n\ {\isacharequal}{\kern0pt}\ {\isadigit{1}}\ {\isasymand}\ na\ {\isacharequal}{\kern0pt}\ {\isadigit{2}}\ then\ {\isadigit{1}}\ else\ if\ n\ {\isacharequal}{\kern0pt}\ {\isadigit{2}}\ {\isasymand}\ na\ {\isacharequal}{\kern0pt}\ {\isadigit{1}}\ then\ {\isadigit{1}}\ else\ if\ n\ {\isacharequal}{\kern0pt}\ {\isadigit{3}}\ {\isasymand}\ na\ {\isacharequal}{\kern0pt}\ {\isadigit{3}}\ then\ {\isadigit{1}}\ else\ {\isadigit{0}}{\isacharparenright}{\kern0pt}\ {\isasymand}\ Matrix{\isachardot}{\kern0pt}mat\ {\isadigit{4}}\ {\isadigit{4}}\ {\isacharparenleft}{\kern0pt}{\isasymlambda}{\isacharparenleft}{\kern0pt}n{\isacharcomma}{\kern0pt}\ na{\isacharparenright}{\kern0pt}{\isachardot}{\kern0pt}\ if\ n\ {\isacharequal}{\kern0pt}\ {\isadigit{0}}\ {\isasymand}\ na\ {\isacharequal}{\kern0pt}\ {\isadigit{0}}\ then\ {\isadigit{1}}{\isacharcolon}{\kern0pt}{\isacharcolon}{\kern0pt}complex\ else\ if\ n\ {\isacharequal}{\kern0pt}\ {\isadigit{1}}\ {\isasymand}\ na\ {\isacharequal}{\kern0pt}\ {\isadigit{2}}\ then\ {\isadigit{1}}\ else\ if\ n\ {\isacharequal}{\kern0pt}\ {\isadigit{2}}\ {\isasymand}\ na\ {\isacharequal}{\kern0pt}\ {\isadigit{1}}\ then\ {\isadigit{1}}\ else\ if\ n\ {\isacharequal}{\kern0pt}\ {\isadigit{3}}\ {\isasymand}\ na\ {\isacharequal}{\kern0pt}\ {\isadigit{3}}\ then\ {\isadigit{1}}\ else\ {\isadigit{0}}{\isacharparenright}{\kern0pt}\ {\isachardollar}{\kern0pt}{\isachardollar}{\kern0pt}\ {\isacharparenleft}{\kern0pt}nn\ {\isacharparenleft}{\kern0pt}{\isasymlambda}{\isacharparenleft}{\kern0pt}n{\isacharcomma}{\kern0pt}\ na{\isacharparenright}{\kern0pt}{\isachardot}{\kern0pt}\ cnj\ {\isacharparenleft}{\kern0pt}SWAP\ {\isachardollar}{\kern0pt}{\isachardollar}{\kern0pt}\ {\isacharparenleft}{\kern0pt}na{\isacharcomma}{\kern0pt}\ n{\isacharparenright}{\kern0pt}{\isacharparenright}{\kern0pt}{\isacharparenright}{\kern0pt}\ {\isacharparenleft}{\kern0pt}{\isasymlambda}{\isacharparenleft}{\kern0pt}n{\isacharcomma}{\kern0pt}\ na{\isacharparenright}{\kern0pt}{\isachardot}{\kern0pt}\ cnj\ {\isacharparenleft}{\kern0pt}SWAP\ {\isachardollar}{\kern0pt}{\isachardollar}{\kern0pt}\ {\isacharparenleft}{\kern0pt}n{\isacharcomma}{\kern0pt}\ na{\isacharparenright}{\kern0pt}{\isacharparenright}{\kern0pt}{\isacharparenright}{\kern0pt}\ {\isadigit{4}}\ {\isadigit{4}}{\isacharcomma}{\kern0pt}\ nna\ {\isacharparenleft}{\kern0pt}{\isasymlambda}{\isacharparenleft}{\kern0pt}n{\isacharcomma}{\kern0pt}\ na{\isacharparenright}{\kern0pt}{\isachardot}{\kern0pt}\ cnj\ {\isacharparenleft}{\kern0pt}SWAP\ {\isachardollar}{\kern0pt}{\isachardollar}{\kern0pt}\ {\isacharparenleft}{\kern0pt}na{\isacharcomma}{\kern0pt}\ n{\isacharparenright}{\kern0pt}{\isacharparenright}{\kern0pt}{\isacharparenright}{\kern0pt}\ {\isacharparenleft}{\kern0pt}{\isasymlambda}{\isacharparenleft}{\kern0pt}n{\isacharcomma}{\kern0pt}\ na{\isacharparenright}{\kern0pt}{\isachardot}{\kern0pt}\ cnj\ {\isacharparenleft}{\kern0pt}SWAP\ {\isachardollar}{\kern0pt}{\isachardollar}{\kern0pt}\ {\isacharparenleft}{\kern0pt}n{\isacharcomma}{\kern0pt}\ na{\isacharparenright}{\kern0pt}{\isacharparenright}{\kern0pt}{\isacharparenright}{\kern0pt}\ {\isadigit{4}}\ {\isadigit{4}}{\isacharparenright}{\kern0pt}\ {\isacharequal}{\kern0pt}\ {\isacharparenleft}{\kern0pt}case\ {\isacharparenleft}{\kern0pt}nn\ {\isacharparenleft}{\kern0pt}{\isasymlambda}{\isacharparenleft}{\kern0pt}n{\isacharcomma}{\kern0pt}\ na{\isacharparenright}{\kern0pt}{\isachardot}{\kern0pt}\ cnj\ {\isacharparenleft}{\kern0pt}SWAP\ {\isachardollar}{\kern0pt}{\isachardollar}{\kern0pt}\ {\isacharparenleft}{\kern0pt}na{\isacharcomma}{\kern0pt}\ n{\isacharparenright}{\kern0pt}{\isacharparenright}{\kern0pt}{\isacharparenright}{\kern0pt}\ {\isacharparenleft}{\kern0pt}{\isasymlambda}{\isacharparenleft}{\kern0pt}n{\isacharcomma}{\kern0pt}\ na{\isacharparenright}{\kern0pt}{\isachardot}{\kern0pt}\ cnj\ {\isacharparenleft}{\kern0pt}SWAP\ {\isachardollar}{\kern0pt}{\isachardollar}{\kern0pt}\ {\isacharparenleft}{\kern0pt}n{\isacharcomma}{\kern0pt}\ na{\isacharparenright}{\kern0pt}{\isacharparenright}{\kern0pt}{\isacharparenright}{\kern0pt}\ {\isadigit{4}}\ {\isadigit{4}}{\isacharcomma}{\kern0pt}\ nna\ {\isacharparenleft}{\kern0pt}{\isasymlambda}{\isacharparenleft}{\kern0pt}n{\isacharcomma}{\kern0pt}\ na{\isacharparenright}{\kern0pt}{\isachardot}{\kern0pt}\ cnj\ {\isacharparenleft}{\kern0pt}SWAP\ {\isachardollar}{\kern0pt}{\isachardollar}{\kern0pt}\ {\isacharparenleft}{\kern0pt}na{\isacharcomma}{\kern0pt}\ n{\isacharparenright}{\kern0pt}{\isacharparenright}{\kern0pt}{\isacharparenright}{\kern0pt}\ {\isacharparenleft}{\kern0pt}{\isasymlambda}{\isacharparenleft}{\kern0pt}n{\isacharcomma}{\kern0pt}\ na{\isacharparenright}{\kern0pt}{\isachardot}{\kern0pt}\ cnj\ {\isacharparenleft}{\kern0pt}SWAP\ {\isachardollar}{\kern0pt}{\isachardollar}{\kern0pt}\ {\isacharparenleft}{\kern0pt}n{\isacharcomma}{\kern0pt}\ na{\isacharparenright}{\kern0pt}{\isacharparenright}{\kern0pt}{\isacharparenright}{\kern0pt}\ {\isadigit{4}}\ {\isadigit{4}}{\isacharparenright}{\kern0pt}\ of\ {\isacharparenleft}{\kern0pt}n{\isacharcomma}{\kern0pt}\ na{\isacharparenright}{\kern0pt}\ {\isasymRightarrow}\ if\ n\ {\isacharequal}{\kern0pt}\ {\isadigit{0}}\ {\isasymand}\ na\ {\isacharequal}{\kern0pt}\ {\isadigit{0}}\ then\ {\isadigit{1}}\ else\ if\ n\ {\isacharequal}{\kern0pt}\ {\isadigit{1}}\ {\isasymand}\ na\ {\isacharequal}{\kern0pt}\ {\isadigit{2}}\ then\ {\isadigit{1}}\ else\ if\ n\ {\isacharequal}{\kern0pt}\ {\isadigit{2}}\ {\isasymand}\ na\ {\isacharequal}{\kern0pt}\ {\isadigit{1}}\ then\ {\isadigit{1}}\ else\ if\ n\ {\isacharequal}{\kern0pt}\ {\isadigit{3}}\ {\isasymand}\ na\ {\isacharequal}{\kern0pt}\ {\isadigit{3}}\ then\ {\isadigit{1}}\ else\ {\isadigit{0}}{\isacharparenright}{\kern0pt}\ {\isasymlongrightarrow}\ {\isasymnot}\ nn\ {\isacharparenleft}{\kern0pt}{\isasymlambda}{\isacharparenleft}{\kern0pt}n{\isacharcomma}{\kern0pt}\ na{\isacharparenright}{\kern0pt}{\isachardot}{\kern0pt}\ cnj\ {\isacharparenleft}{\kern0pt}SWAP\ {\isachardollar}{\kern0pt}{\isachardollar}{\kern0pt}\ {\isacharparenleft}{\kern0pt}na{\isacharcomma}{\kern0pt}\ n{\isacharparenright}{\kern0pt}{\isacharparenright}{\kern0pt}{\isacharparenright}{\kern0pt}\ {\isacharparenleft}{\kern0pt}{\isasymlambda}{\isacharparenleft}{\kern0pt}n{\isacharcomma}{\kern0pt}\ na{\isacharparenright}{\kern0pt}{\isachardot}{\kern0pt}\ cnj\ {\isacharparenleft}{\kern0pt}SWAP\ {\isachardollar}{\kern0pt}{\isachardollar}{\kern0pt}\ {\isacharparenleft}{\kern0pt}n{\isacharcomma}{\kern0pt}\ na{\isacharparenright}{\kern0pt}{\isacharparenright}{\kern0pt}{\isacharparenright}{\kern0pt}\ {\isadigit{4}}\ {\isadigit{4}}\ {\isacharless}{\kern0pt}\ {\isadigit{4}}\ {\isasymor}\ {\isasymnot}\ nna\ {\isacharparenleft}{\kern0pt}{\isasymlambda}{\isacharparenleft}{\kern0pt}n{\isacharcomma}{\kern0pt}\ na{\isacharparenright}{\kern0pt}{\isachardot}{\kern0pt}\ cnj\ {\isacharparenleft}{\kern0pt}SWAP\ {\isachardollar}{\kern0pt}{\isachardollar}{\kern0pt}\ {\isacharparenleft}{\kern0pt}na{\isacharcomma}{\kern0pt}\ n{\isacharparenright}{\kern0pt}{\isacharparenright}{\kern0pt}{\isacharparenright}{\kern0pt}\ {\isacharparenleft}{\kern0pt}{\isasymlambda}{\isacharparenleft}{\kern0pt}n{\isacharcomma}{\kern0pt}\ na{\isacharparenright}{\kern0pt}{\isachardot}{\kern0pt}\ cnj\ {\isacharparenleft}{\kern0pt}SWAP\ {\isachardollar}{\kern0pt}{\isachardollar}{\kern0pt}\ {\isacharparenleft}{\kern0pt}n{\isacharcomma}{\kern0pt}\ na{\isacharparenright}{\kern0pt}{\isacharparenright}{\kern0pt}{\isacharparenright}{\kern0pt}\ {\isadigit{4}}\ {\isadigit{4}}\ {\isacharless}{\kern0pt}\ {\isadigit{4}}\ {\isasymor}\ {\isacharparenleft}{\kern0pt}case\ {\isacharparenleft}{\kern0pt}nn\ {\isacharparenleft}{\kern0pt}{\isasymlambda}{\isacharparenleft}{\kern0pt}n{\isacharcomma}{\kern0pt}\ na{\isacharparenright}{\kern0pt}{\isachardot}{\kern0pt}\ cnj\ {\isacharparenleft}{\kern0pt}SWAP\ {\isachardollar}{\kern0pt}{\isachardollar}{\kern0pt}\ {\isacharparenleft}{\kern0pt}na{\isacharcomma}{\kern0pt}\ n{\isacharparenright}{\kern0pt}{\isacharparenright}{\kern0pt}{\isacharparenright}{\kern0pt}\ {\isacharparenleft}{\kern0pt}{\isasymlambda}{\isacharparenleft}{\kern0pt}n{\isacharcomma}{\kern0pt}\ na{\isacharparenright}{\kern0pt}{\isachardot}{\kern0pt}\ cnj\ {\isacharparenleft}{\kern0pt}SWAP\ {\isachardollar}{\kern0pt}{\isachardollar}{\kern0pt}\ {\isacharparenleft}{\kern0pt}n{\isacharcomma}{\kern0pt}\ na{\isacharparenright}{\kern0pt}{\isacharparenright}{\kern0pt}{\isacharparenright}{\kern0pt}\ {\isadigit{4}}\ {\isadigit{4}}{\isacharcomma}{\kern0pt}\ nna\ {\isacharparenleft}{\kern0pt}{\isasymlambda}{\isacharparenleft}{\kern0pt}n{\isacharcomma}{\kern0pt}\ na{\isacharparenright}{\kern0pt}{\isachardot}{\kern0pt}\ cnj\ {\isacharparenleft}{\kern0pt}SWAP\ {\isachardollar}{\kern0pt}{\isachardollar}{\kern0pt}\ {\isacharparenleft}{\kern0pt}na{\isacharcomma}{\kern0pt}\ n{\isacharparenright}{\kern0pt}{\isacharparenright}{\kern0pt}{\isacharparenright}{\kern0pt}\ {\isacharparenleft}{\kern0pt}{\isasymlambda}{\isacharparenleft}{\kern0pt}n{\isacharcomma}{\kern0pt}\ na{\isacharparenright}{\kern0pt}{\isachardot}{\kern0pt}\ cnj\ {\isacharparenleft}{\kern0pt}SWAP\ {\isachardollar}{\kern0pt}{\isachardollar}{\kern0pt}\ {\isacharparenleft}{\kern0pt}n{\isacharcomma}{\kern0pt}\ na{\isacharparenright}{\kern0pt}{\isacharparenright}{\kern0pt}{\isacharparenright}{\kern0pt}\ {\isadigit{4}}\ {\isadigit{4}}{\isacharparenright}{\kern0pt}\ of\ {\isacharparenleft}{\kern0pt}n{\isacharcomma}{\kern0pt}\ na{\isacharparenright}{\kern0pt}\ {\isasymRightarrow}\ cnj\ {\isacharparenleft}{\kern0pt}SWAP\ {\isachardollar}{\kern0pt}{\isachardollar}{\kern0pt}\ {\isacharparenleft}{\kern0pt}n{\isacharcomma}{\kern0pt}\ na{\isacharparenright}{\kern0pt}{\isacharparenright}{\kern0pt}{\isacharparenright}{\kern0pt}\ {\isacharequal}{\kern0pt}\ {\isacharparenleft}{\kern0pt}case\ {\isacharparenleft}{\kern0pt}nn\ {\isacharparenleft}{\kern0pt}{\isasymlambda}{\isacharparenleft}{\kern0pt}n{\isacharcomma}{\kern0pt}\ na{\isacharparenright}{\kern0pt}{\isachardot}{\kern0pt}\ cnj\ {\isacharparenleft}{\kern0pt}SWAP\ {\isachardollar}{\kern0pt}{\isachardollar}{\kern0pt}\ {\isacharparenleft}{\kern0pt}na{\isacharcomma}{\kern0pt}\ n{\isacharparenright}{\kern0pt}{\isacharparenright}{\kern0pt}{\isacharparenright}{\kern0pt}\ {\isacharparenleft}{\kern0pt}{\isasymlambda}{\isacharparenleft}{\kern0pt}n{\isacharcomma}{\kern0pt}\ na{\isacharparenright}{\kern0pt}{\isachardot}{\kern0pt}\ cnj\ {\isacharparenleft}{\kern0pt}SWAP\ {\isachardollar}{\kern0pt}{\isachardollar}{\kern0pt}\ {\isacharparenleft}{\kern0pt}n{\isacharcomma}{\kern0pt}\ na{\isacharparenright}{\kern0pt}{\isacharparenright}{\kern0pt}{\isacharparenright}{\kern0pt}\ {\isadigit{4}}\ {\isadigit{4}}{\isacharcomma}{\kern0pt}\ nna\ {\isacharparenleft}{\kern0pt}{\isasymlambda}{\isacharparenleft}{\kern0pt}n{\isacharcomma}{\kern0pt}\ na{\isacharparenright}{\kern0pt}{\isachardot}{\kern0pt}\ cnj\ {\isacharparenleft}{\kern0pt}SWAP\ {\isachardollar}{\kern0pt}{\isachardollar}{\kern0pt}\ {\isacharparenleft}{\kern0pt}na{\isacharcomma}{\kern0pt}\ n{\isacharparenright}{\kern0pt}{\isacharparenright}{\kern0pt}{\isacharparenright}{\kern0pt}\ {\isacharparenleft}{\kern0pt}{\isasymlambda}{\isacharparenleft}{\kern0pt}n{\isacharcomma}{\kern0pt}\ na{\isacharparenright}{\kern0pt}{\isachardot}{\kern0pt}\ cnj\ {\isacharparenleft}{\kern0pt}SWAP\ {\isachardollar}{\kern0pt}{\isachardollar}{\kern0pt}\ {\isacharparenleft}{\kern0pt}n{\isacharcomma}{\kern0pt}\ na{\isacharparenright}{\kern0pt}{\isacharparenright}{\kern0pt}{\isacharparenright}{\kern0pt}\ {\isadigit{4}}\ {\isadigit{4}}{\isacharparenright}{\kern0pt}\ of\ {\isacharparenleft}{\kern0pt}n{\isacharcomma}{\kern0pt}\ na{\isacharparenright}{\kern0pt}\ {\isasymRightarrow}\ cnj\ {\isacharparenleft}{\kern0pt}SWAP\ {\isachardollar}{\kern0pt}{\isachardollar}{\kern0pt}\ {\isacharparenleft}{\kern0pt}na{\isacharcomma}{\kern0pt}\ n{\isacharparenright}{\kern0pt}{\isacharparenright}{\kern0pt}{\isacharparenright}{\kern0pt}{\isachardoublequoteclose}\isanewline
\ \ \ \ \ \ \ \ \isacommand{by}\isamarkupfalse%
\ linarith\ \isacommand{{\isacharbraceright}{\kern0pt}}\isamarkupfalse%
\isanewline
\ \ \ \ \isacommand{ultimately}\isamarkupfalse%
\ \isacommand{show}\isamarkupfalse%
\ {\isacharquery}{\kern0pt}thesis\isanewline
\ \ \ \ \ \ \isacommand{by}\isamarkupfalse%
\ {\isacharparenleft}{\kern0pt}smt\ {\isacharparenleft}{\kern0pt}z{\isadigit{3}}{\isacharparenright}{\kern0pt}\ SWAP{\isacharunderscore}{\kern0pt}def\ index{\isacharunderscore}{\kern0pt}mat{\isacharparenleft}{\kern0pt}{\isadigit{1}}{\isacharparenright}{\kern0pt}{\isacharparenright}{\kern0pt}\isanewline
\ \ \isacommand{qed}\isamarkupfalse%
\isanewline
\ \ \isacommand{also}\isamarkupfalse%
\ \isacommand{have}\isamarkupfalse%
\ {\isachardoublequoteopen}{\isasymdots}\ {\isacharequal}{\kern0pt}\ SWAP{\isachardoublequoteclose}\ \isacommand{using}\isamarkupfalse%
\ SWAP{\isacharunderscore}{\kern0pt}def\ SWAP{\isacharunderscore}{\kern0pt}index\isanewline
\ \ \ \ \isacommand{by}\isamarkupfalse%
\ {\isacharparenleft}{\kern0pt}smt\ {\isacharparenleft}{\kern0pt}verit{\isacharcomma}{\kern0pt}\ ccfv{\isacharunderscore}{\kern0pt}SIG{\isacharparenright}{\kern0pt}\ case{\isacharunderscore}{\kern0pt}prod{\isacharunderscore}{\kern0pt}conv\ complex{\isacharunderscore}{\kern0pt}cnj{\isacharunderscore}{\kern0pt}one\ complex{\isacharunderscore}{\kern0pt}cnj{\isacharunderscore}{\kern0pt}zero\ cong{\isacharunderscore}{\kern0pt}mat\ index{\isacharunderscore}{\kern0pt}mat{\isacharparenleft}{\kern0pt}{\isadigit{1}}{\isacharparenright}{\kern0pt}{\isacharparenright}{\kern0pt}\isanewline
\ \ \isacommand{finally}\isamarkupfalse%
\ \isacommand{show}\isamarkupfalse%
\ {\isacharquery}{\kern0pt}thesis\ \isacommand{by}\isamarkupfalse%
\ this\isanewline
\isacommand{qed}\isamarkupfalse%
%
\endisatagproof
{\isafoldproof}%
%
\isadelimproof
\isanewline
%
\endisadelimproof
\isanewline
\isacommand{lemma}\isamarkupfalse%
\ SWAP{\isacharunderscore}{\kern0pt}inv{\isacharcolon}{\kern0pt}\isanewline
\ \ \isakeyword{shows}\ {\isachardoublequoteopen}SWAP\ {\isacharasterisk}{\kern0pt}\ {\isacharparenleft}{\kern0pt}SWAP\isactrlsup {\isasymdagger}{\isacharparenright}{\kern0pt}\ {\isacharequal}{\kern0pt}\ {\isadigit{1}}\isactrlsub m\ {\isadigit{4}}{\isachardoublequoteclose}\isanewline
%
\isadelimproof
\ \ %
\endisadelimproof
%
\isatagproof
\isacommand{apply}\isamarkupfalse%
\ {\isacharparenleft}{\kern0pt}simp\ add{\isacharcolon}{\kern0pt}\ SWAP{\isacharunderscore}{\kern0pt}def\ times{\isacharunderscore}{\kern0pt}mat{\isacharunderscore}{\kern0pt}def\ one{\isacharunderscore}{\kern0pt}mat{\isacharunderscore}{\kern0pt}def{\isacharparenright}{\kern0pt}\isanewline
\ \ \isacommand{apply}\isamarkupfalse%
\ {\isacharparenleft}{\kern0pt}rule\ cong{\isacharunderscore}{\kern0pt}mat{\isacharparenright}{\kern0pt}\isanewline
\ \ \isacommand{by}\isamarkupfalse%
\ {\isacharparenleft}{\kern0pt}auto\ simp{\isacharcolon}{\kern0pt}\ scalar{\isacharunderscore}{\kern0pt}prod{\isacharunderscore}{\kern0pt}def\ complex{\isacharunderscore}{\kern0pt}eqI{\isacharparenright}{\kern0pt}%
\endisatagproof
{\isafoldproof}%
%
\isadelimproof
\isanewline
%
\endisadelimproof
\isanewline
\isacommand{lemma}\isamarkupfalse%
\ SWAP{\isacharunderscore}{\kern0pt}inv{\isacharprime}{\kern0pt}{\isacharcolon}{\kern0pt}\isanewline
\ \ \isakeyword{shows}\ {\isachardoublequoteopen}{\isacharparenleft}{\kern0pt}SWAP\isactrlsup {\isasymdagger}{\isacharparenright}{\kern0pt}\ {\isacharasterisk}{\kern0pt}\ SWAP\ {\isacharequal}{\kern0pt}\ {\isadigit{1}}\isactrlsub m\ {\isadigit{4}}{\isachardoublequoteclose}\isanewline
%
\isadelimproof
\ \ %
\endisadelimproof
%
\isatagproof
\isacommand{apply}\isamarkupfalse%
\ {\isacharparenleft}{\kern0pt}simp\ add{\isacharcolon}{\kern0pt}\ SWAP{\isacharunderscore}{\kern0pt}def\ times{\isacharunderscore}{\kern0pt}mat{\isacharunderscore}{\kern0pt}def\ one{\isacharunderscore}{\kern0pt}mat{\isacharunderscore}{\kern0pt}def{\isacharparenright}{\kern0pt}\isanewline
\ \ \isacommand{apply}\isamarkupfalse%
\ {\isacharparenleft}{\kern0pt}rule\ cong{\isacharunderscore}{\kern0pt}mat{\isacharparenright}{\kern0pt}\isanewline
\ \ \isacommand{by}\isamarkupfalse%
\ {\isacharparenleft}{\kern0pt}auto\ simp{\isacharcolon}{\kern0pt}\ scalar{\isacharunderscore}{\kern0pt}prod{\isacharunderscore}{\kern0pt}def\ complex{\isacharunderscore}{\kern0pt}eqI{\isacharparenright}{\kern0pt}%
\endisatagproof
{\isafoldproof}%
%
\isadelimproof
\isanewline
%
\endisadelimproof
\ \isanewline
\isacommand{lemma}\isamarkupfalse%
\ SWAP{\isacharunderscore}{\kern0pt}is{\isacharunderscore}{\kern0pt}gate{\isacharcolon}{\kern0pt}\isanewline
\ \ \isakeyword{shows}\ {\isachardoublequoteopen}gate\ {\isadigit{2}}\ SWAP{\isachardoublequoteclose}\isanewline
%
\isadelimproof
%
\endisadelimproof
%
\isatagproof
\isacommand{proof}\isamarkupfalse%
\isanewline
\ \ \isacommand{show}\isamarkupfalse%
\ {\isachardoublequoteopen}dim{\isacharunderscore}{\kern0pt}row\ SWAP\ {\isacharequal}{\kern0pt}\ {\isadigit{2}}\isactrlsup {\isadigit{2}}{\isachardoublequoteclose}\ \isacommand{using}\isamarkupfalse%
\ SWAP{\isacharunderscore}{\kern0pt}carrier{\isacharunderscore}{\kern0pt}mat\ \isacommand{by}\isamarkupfalse%
\ {\isacharparenleft}{\kern0pt}simp\ add{\isacharcolon}{\kern0pt}\ numeral{\isacharunderscore}{\kern0pt}Bit{\isadigit{0}}{\isacharparenright}{\kern0pt}\isanewline
\isacommand{next}\isamarkupfalse%
\isanewline
\ \ \isacommand{show}\isamarkupfalse%
\ {\isachardoublequoteopen}square{\isacharunderscore}{\kern0pt}mat\ SWAP{\isachardoublequoteclose}\ \isacommand{using}\isamarkupfalse%
\ SWAP{\isacharunderscore}{\kern0pt}carrier{\isacharunderscore}{\kern0pt}mat\ \isacommand{by}\isamarkupfalse%
\ {\isacharparenleft}{\kern0pt}simp\ add{\isacharcolon}{\kern0pt}\ numeral{\isacharunderscore}{\kern0pt}Bit{\isadigit{0}}{\isacharparenright}{\kern0pt}\isanewline
\isacommand{next}\isamarkupfalse%
\isanewline
\ \ \isacommand{show}\isamarkupfalse%
\ {\isachardoublequoteopen}unitary\ SWAP{\isachardoublequoteclose}\isanewline
\ \ \ \ \isacommand{using}\isamarkupfalse%
\ unitary{\isacharunderscore}{\kern0pt}def\ SWAP{\isacharunderscore}{\kern0pt}inv\ SWAP{\isacharunderscore}{\kern0pt}inv{\isacharprime}{\kern0pt}\ SWAP{\isacharunderscore}{\kern0pt}ncols\ SWAP{\isacharunderscore}{\kern0pt}nrows\ \isacommand{by}\isamarkupfalse%
\ presburger\isanewline
\isacommand{qed}\isamarkupfalse%
%
\endisatagproof
{\isafoldproof}%
%
\isadelimproof
\isanewline
%
\endisadelimproof
\isanewline
\isanewline
\isacommand{lemma}\isamarkupfalse%
\ control{\isadigit{2}}{\isacharunderscore}{\kern0pt}inv{\isacharcolon}{\kern0pt}\isanewline
\ \ \isakeyword{assumes}\ {\isachardoublequoteopen}gate\ {\isadigit{1}}\ U{\isachardoublequoteclose}\isanewline
\ \ \isakeyword{shows}\ {\isachardoublequoteopen}{\isacharparenleft}{\kern0pt}control{\isadigit{2}}\ U{\isacharparenright}{\kern0pt}\ {\isacharasterisk}{\kern0pt}\ {\isacharparenleft}{\kern0pt}{\isacharparenleft}{\kern0pt}control{\isadigit{2}}\ U{\isacharparenright}{\kern0pt}\isactrlsup {\isasymdagger}{\isacharparenright}{\kern0pt}\ {\isacharequal}{\kern0pt}\ {\isadigit{1}}\isactrlsub m\ {\isadigit{4}}{\isachardoublequoteclose}\isanewline
%
\isadelimproof
%
\endisadelimproof
%
\isatagproof
\isacommand{proof}\isamarkupfalse%
\ \isanewline
\ \ \isacommand{show}\isamarkupfalse%
\ {\isachardoublequoteopen}{\isasymAnd}i\ j{\isachardot}{\kern0pt}\ i\ {\isacharless}{\kern0pt}\ dim{\isacharunderscore}{\kern0pt}row\ {\isacharparenleft}{\kern0pt}{\isadigit{1}}\isactrlsub m\ {\isadigit{4}}{\isacharparenright}{\kern0pt}\ {\isasymLongrightarrow}\ j\ {\isacharless}{\kern0pt}\ dim{\isacharunderscore}{\kern0pt}col\ {\isacharparenleft}{\kern0pt}{\isadigit{1}}\isactrlsub m\ {\isadigit{4}}{\isacharparenright}{\kern0pt}\ {\isasymLongrightarrow}\isanewline
\ \ \ \ \ \ \ \ \ \ \ {\isacharparenleft}{\kern0pt}control{\isadigit{2}}\ U\ {\isacharasterisk}{\kern0pt}\ {\isacharparenleft}{\kern0pt}{\isacharparenleft}{\kern0pt}control{\isadigit{2}}\ U{\isacharparenright}{\kern0pt}\isactrlsup {\isasymdagger}{\isacharparenright}{\kern0pt}{\isacharparenright}{\kern0pt}\ {\isachardollar}{\kern0pt}{\isachardollar}{\kern0pt}\ {\isacharparenleft}{\kern0pt}i{\isacharcomma}{\kern0pt}\ j{\isacharparenright}{\kern0pt}\ {\isacharequal}{\kern0pt}\ {\isadigit{1}}\isactrlsub m\ {\isadigit{4}}\ {\isachardollar}{\kern0pt}{\isachardollar}{\kern0pt}\ {\isacharparenleft}{\kern0pt}i{\isacharcomma}{\kern0pt}\ j{\isacharparenright}{\kern0pt}{\isachardoublequoteclose}\isanewline
\ \ \isacommand{proof}\isamarkupfalse%
\ {\isacharminus}{\kern0pt}\isanewline
\ \ \ \ \isacommand{fix}\isamarkupfalse%
\ i\ j\isanewline
\ \ \ \ \isacommand{assume}\isamarkupfalse%
\ {\isachardoublequoteopen}i\ {\isacharless}{\kern0pt}\ dim{\isacharunderscore}{\kern0pt}row\ {\isacharparenleft}{\kern0pt}{\isadigit{1}}\isactrlsub m\ {\isadigit{4}}{\isacharparenright}{\kern0pt}{\isachardoublequoteclose}\isanewline
\ \ \ \ \isacommand{hence}\isamarkupfalse%
\ i{\isadigit{4}}{\isacharcolon}{\kern0pt}{\isachardoublequoteopen}i\ {\isacharless}{\kern0pt}\ {\isadigit{4}}{\isachardoublequoteclose}\ \isacommand{by}\isamarkupfalse%
\ auto\isanewline
\ \ \ \ \isacommand{assume}\isamarkupfalse%
\ {\isachardoublequoteopen}j\ {\isacharless}{\kern0pt}\ dim{\isacharunderscore}{\kern0pt}col\ {\isacharparenleft}{\kern0pt}{\isadigit{1}}\isactrlsub m\ {\isadigit{4}}{\isacharparenright}{\kern0pt}{\isachardoublequoteclose}\isanewline
\ \ \ \ \isacommand{hence}\isamarkupfalse%
\ j{\isadigit{4}}{\isacharcolon}{\kern0pt}{\isachardoublequoteopen}j\ {\isacharless}{\kern0pt}\ {\isadigit{4}}{\isachardoublequoteclose}\ \isacommand{by}\isamarkupfalse%
\ auto\isanewline
\ \ \ \ \isacommand{show}\isamarkupfalse%
\ {\isachardoublequoteopen}{\isacharparenleft}{\kern0pt}control{\isadigit{2}}\ U\ {\isacharasterisk}{\kern0pt}\ {\isacharparenleft}{\kern0pt}{\isacharparenleft}{\kern0pt}control{\isadigit{2}}\ U{\isacharparenright}{\kern0pt}\isactrlsup {\isasymdagger}{\isacharparenright}{\kern0pt}{\isacharparenright}{\kern0pt}\ {\isachardollar}{\kern0pt}{\isachardollar}{\kern0pt}\ {\isacharparenleft}{\kern0pt}i{\isacharcomma}{\kern0pt}\ j{\isacharparenright}{\kern0pt}\ {\isacharequal}{\kern0pt}\ {\isadigit{1}}\isactrlsub m\ {\isadigit{4}}\ {\isachardollar}{\kern0pt}{\isachardollar}{\kern0pt}\ {\isacharparenleft}{\kern0pt}i{\isacharcomma}{\kern0pt}\ j{\isacharparenright}{\kern0pt}{\isachardoublequoteclose}\isanewline
\ \ \ \ \isacommand{proof}\isamarkupfalse%
\ {\isacharparenleft}{\kern0pt}rule\ disjE{\isacharparenright}{\kern0pt}\isanewline
\ \ \ \ \ \ \isacommand{show}\isamarkupfalse%
\ {\isachardoublequoteopen}i\ {\isacharequal}{\kern0pt}\ {\isadigit{0}}\ {\isasymor}\ i\ {\isacharequal}{\kern0pt}\ {\isadigit{1}}\ {\isasymor}\ i\ {\isacharequal}{\kern0pt}\ {\isadigit{2}}\ {\isasymor}\ i\ {\isacharequal}{\kern0pt}\ {\isadigit{3}}{\isachardoublequoteclose}\ \isacommand{using}\isamarkupfalse%
\ i{\isadigit{4}}\ \isacommand{by}\isamarkupfalse%
\ auto\isanewline
\ \ \ \ \isacommand{next}\isamarkupfalse%
\isanewline
\ \ \ \ \ \ \isacommand{assume}\isamarkupfalse%
\ i{\isadigit{0}}{\isacharcolon}{\kern0pt}{\isachardoublequoteopen}i\ {\isacharequal}{\kern0pt}\ {\isadigit{0}}{\isachardoublequoteclose}\isanewline
\ \ \ \ \ \ \isacommand{show}\isamarkupfalse%
\ {\isachardoublequoteopen}{\isacharparenleft}{\kern0pt}control{\isadigit{2}}\ U\ {\isacharasterisk}{\kern0pt}\ {\isacharparenleft}{\kern0pt}{\isacharparenleft}{\kern0pt}control{\isadigit{2}}\ U{\isacharparenright}{\kern0pt}\isactrlsup {\isasymdagger}{\isacharparenright}{\kern0pt}{\isacharparenright}{\kern0pt}\ {\isachardollar}{\kern0pt}{\isachardollar}{\kern0pt}\ {\isacharparenleft}{\kern0pt}i{\isacharcomma}{\kern0pt}\ j{\isacharparenright}{\kern0pt}\ {\isacharequal}{\kern0pt}\ {\isadigit{1}}\isactrlsub m\ {\isadigit{4}}\ {\isachardollar}{\kern0pt}{\isachardollar}{\kern0pt}\ {\isacharparenleft}{\kern0pt}i{\isacharcomma}{\kern0pt}\ j{\isacharparenright}{\kern0pt}{\isachardoublequoteclose}\isanewline
\ \ \ \ \ \ \isacommand{proof}\isamarkupfalse%
\ {\isacharparenleft}{\kern0pt}rule\ disjE{\isacharparenright}{\kern0pt}\isanewline
\ \ \ \ \ \ \ \ \isacommand{show}\isamarkupfalse%
\ {\isachardoublequoteopen}j\ {\isacharequal}{\kern0pt}\ {\isadigit{0}}\ {\isasymor}\ j\ {\isacharequal}{\kern0pt}\ {\isadigit{1}}\ {\isasymor}\ j\ {\isacharequal}{\kern0pt}\ {\isadigit{2}}\ {\isasymor}\ j\ {\isacharequal}{\kern0pt}\ {\isadigit{3}}{\isachardoublequoteclose}\ \isacommand{using}\isamarkupfalse%
\ j{\isadigit{4}}\ \isacommand{by}\isamarkupfalse%
\ auto\isanewline
\ \ \ \ \ \ \isacommand{next}\isamarkupfalse%
\isanewline
\ \ \ \ \ \ \ \ \isacommand{assume}\isamarkupfalse%
\ j{\isadigit{0}}{\isacharcolon}{\kern0pt}{\isachardoublequoteopen}j\ {\isacharequal}{\kern0pt}\ {\isadigit{0}}{\isachardoublequoteclose}\isanewline
\ \ \ \ \ \ \ \ \isacommand{show}\isamarkupfalse%
\ {\isachardoublequoteopen}{\isacharparenleft}{\kern0pt}control{\isadigit{2}}\ U\ {\isacharasterisk}{\kern0pt}\ {\isacharparenleft}{\kern0pt}{\isacharparenleft}{\kern0pt}control{\isadigit{2}}\ U{\isacharparenright}{\kern0pt}\isactrlsup {\isasymdagger}{\isacharparenright}{\kern0pt}{\isacharparenright}{\kern0pt}\ {\isachardollar}{\kern0pt}{\isachardollar}{\kern0pt}\ {\isacharparenleft}{\kern0pt}i{\isacharcomma}{\kern0pt}\ j{\isacharparenright}{\kern0pt}\ {\isacharequal}{\kern0pt}\ {\isadigit{1}}\isactrlsub m\ {\isadigit{4}}\ {\isachardollar}{\kern0pt}{\isachardollar}{\kern0pt}\ {\isacharparenleft}{\kern0pt}i{\isacharcomma}{\kern0pt}\ j{\isacharparenright}{\kern0pt}{\isachardoublequoteclose}\isanewline
\ \ \ \ \ \ \ \ \isacommand{proof}\isamarkupfalse%
\ {\isacharminus}{\kern0pt}\isanewline
\ \ \ \ \ \ \ \ \ \ \isacommand{have}\isamarkupfalse%
\ {\isachardoublequoteopen}{\isacharparenleft}{\kern0pt}control{\isadigit{2}}\ U\ {\isacharasterisk}{\kern0pt}\ {\isacharparenleft}{\kern0pt}{\isacharparenleft}{\kern0pt}control{\isadigit{2}}\ U{\isacharparenright}{\kern0pt}\isactrlsup {\isasymdagger}{\isacharparenright}{\kern0pt}{\isacharparenright}{\kern0pt}\ {\isachardollar}{\kern0pt}{\isachardollar}{\kern0pt}\ {\isacharparenleft}{\kern0pt}{\isadigit{0}}{\isacharcomma}{\kern0pt}{\isadigit{0}}{\isacharparenright}{\kern0pt}\ {\isacharequal}{\kern0pt}\ \isanewline
\ \ \ \ \ \ \ \ \ \ \ \ \ \ \ \ {\isacharparenleft}{\kern0pt}control{\isadigit{2}}\ U{\isacharparenright}{\kern0pt}\ {\isachardollar}{\kern0pt}{\isachardollar}{\kern0pt}\ {\isacharparenleft}{\kern0pt}{\isadigit{0}}{\isacharcomma}{\kern0pt}{\isadigit{0}}{\isacharparenright}{\kern0pt}\ {\isacharasterisk}{\kern0pt}\ {\isacharparenleft}{\kern0pt}{\isacharparenleft}{\kern0pt}control{\isadigit{2}}\ U{\isacharparenright}{\kern0pt}\isactrlsup {\isasymdagger}{\isacharparenright}{\kern0pt}\ {\isachardollar}{\kern0pt}{\isachardollar}{\kern0pt}\ {\isacharparenleft}{\kern0pt}{\isadigit{0}}{\isacharcomma}{\kern0pt}{\isadigit{0}}{\isacharparenright}{\kern0pt}\ {\isacharplus}{\kern0pt}\isanewline
\ \ \ \ \ \ \ \ \ \ \ \ \ \ \ \ {\isacharparenleft}{\kern0pt}control{\isadigit{2}}\ U{\isacharparenright}{\kern0pt}\ {\isachardollar}{\kern0pt}{\isachardollar}{\kern0pt}\ {\isacharparenleft}{\kern0pt}{\isadigit{0}}{\isacharcomma}{\kern0pt}{\isadigit{1}}{\isacharparenright}{\kern0pt}\ {\isacharasterisk}{\kern0pt}\ {\isacharparenleft}{\kern0pt}{\isacharparenleft}{\kern0pt}control{\isadigit{2}}\ U{\isacharparenright}{\kern0pt}\isactrlsup {\isasymdagger}{\isacharparenright}{\kern0pt}\ {\isachardollar}{\kern0pt}{\isachardollar}{\kern0pt}\ {\isacharparenleft}{\kern0pt}{\isadigit{1}}{\isacharcomma}{\kern0pt}{\isadigit{0}}{\isacharparenright}{\kern0pt}\ {\isacharplus}{\kern0pt}\isanewline
\ \ \ \ \ \ \ \ \ \ \ \ \ \ \ \ {\isacharparenleft}{\kern0pt}control{\isadigit{2}}\ U{\isacharparenright}{\kern0pt}\ {\isachardollar}{\kern0pt}{\isachardollar}{\kern0pt}\ {\isacharparenleft}{\kern0pt}{\isadigit{0}}{\isacharcomma}{\kern0pt}{\isadigit{2}}{\isacharparenright}{\kern0pt}\ {\isacharasterisk}{\kern0pt}\ {\isacharparenleft}{\kern0pt}{\isacharparenleft}{\kern0pt}control{\isadigit{2}}\ U{\isacharparenright}{\kern0pt}\isactrlsup {\isasymdagger}{\isacharparenright}{\kern0pt}\ {\isachardollar}{\kern0pt}{\isachardollar}{\kern0pt}\ {\isacharparenleft}{\kern0pt}{\isadigit{2}}{\isacharcomma}{\kern0pt}{\isadigit{0}}{\isacharparenright}{\kern0pt}\ {\isacharplus}{\kern0pt}\isanewline
\ \ \ \ \ \ \ \ \ \ \ \ \ \ \ \ {\isacharparenleft}{\kern0pt}control{\isadigit{2}}\ U{\isacharparenright}{\kern0pt}\ {\isachardollar}{\kern0pt}{\isachardollar}{\kern0pt}\ {\isacharparenleft}{\kern0pt}{\isadigit{0}}{\isacharcomma}{\kern0pt}{\isadigit{3}}{\isacharparenright}{\kern0pt}\ {\isacharasterisk}{\kern0pt}\ {\isacharparenleft}{\kern0pt}{\isacharparenleft}{\kern0pt}control{\isadigit{2}}\ U{\isacharparenright}{\kern0pt}\isactrlsup {\isasymdagger}{\isacharparenright}{\kern0pt}\ {\isachardollar}{\kern0pt}{\isachardollar}{\kern0pt}\ {\isacharparenleft}{\kern0pt}{\isadigit{3}}{\isacharcomma}{\kern0pt}{\isadigit{0}}{\isacharparenright}{\kern0pt}{\isachardoublequoteclose}\isanewline
\ \ \ \ \ \ \ \ \ \ \ \ \isacommand{using}\isamarkupfalse%
\ times{\isacharunderscore}{\kern0pt}mat{\isacharunderscore}{\kern0pt}def\ sumof{\isadigit{4}}\isanewline
\ \ \ \ \ \ \ \ \ \ \ \ \isacommand{by}\isamarkupfalse%
\ {\isacharparenleft}{\kern0pt}smt\ {\isacharparenleft}{\kern0pt}z{\isadigit{3}}{\isacharparenright}{\kern0pt}\ carrier{\isacharunderscore}{\kern0pt}matD{\isacharparenleft}{\kern0pt}{\isadigit{1}}{\isacharparenright}{\kern0pt}\ carrier{\isacharunderscore}{\kern0pt}matD{\isacharparenleft}{\kern0pt}{\isadigit{2}}{\isacharparenright}{\kern0pt}\ control{\isadigit{2}}{\isacharunderscore}{\kern0pt}carrier{\isacharunderscore}{\kern0pt}mat\ dagger{\isacharunderscore}{\kern0pt}def\ \isanewline
\ \ \ \ \ \ \ \ \ \ \ \ \ \ \ \ dim{\isacharunderscore}{\kern0pt}col{\isacharunderscore}{\kern0pt}of{\isacharunderscore}{\kern0pt}dagger\ dim{\isacharunderscore}{\kern0pt}row{\isacharunderscore}{\kern0pt}mat{\isacharparenleft}{\kern0pt}{\isadigit{1}}{\isacharparenright}{\kern0pt}\ i{\isadigit{0}}\ i{\isadigit{4}}\ index{\isacharunderscore}{\kern0pt}matrix{\isacharunderscore}{\kern0pt}prod{\isacharparenright}{\kern0pt}\isanewline
\ \ \ \ \ \ \ \ \ \ \isacommand{also}\isamarkupfalse%
\ \isacommand{have}\isamarkupfalse%
\ {\isachardoublequoteopen}{\isasymdots}\ {\isacharequal}{\kern0pt}\ {\isacharparenleft}{\kern0pt}{\isacharparenleft}{\kern0pt}control{\isadigit{2}}\ U{\isacharparenright}{\kern0pt}\isactrlsup {\isasymdagger}{\isacharparenright}{\kern0pt}\ {\isachardollar}{\kern0pt}{\isachardollar}{\kern0pt}\ {\isacharparenleft}{\kern0pt}{\isadigit{0}}{\isacharcomma}{\kern0pt}{\isadigit{0}}{\isacharparenright}{\kern0pt}{\isachardoublequoteclose}\isanewline
\ \ \ \ \ \ \ \ \ \ \ \ \isacommand{using}\isamarkupfalse%
\ control{\isadigit{2}}{\isacharunderscore}{\kern0pt}def\ index{\isacharunderscore}{\kern0pt}mat{\isacharunderscore}{\kern0pt}of{\isacharunderscore}{\kern0pt}cols{\isacharunderscore}{\kern0pt}list\ \isacommand{by}\isamarkupfalse%
\ force\isanewline
\ \ \ \ \ \ \ \ \ \ \isacommand{also}\isamarkupfalse%
\ \isacommand{have}\isamarkupfalse%
\ {\isachardoublequoteopen}{\isasymdots}\ {\isacharequal}{\kern0pt}\ cnj\ {\isacharparenleft}{\kern0pt}{\isacharparenleft}{\kern0pt}control{\isadigit{2}}\ U{\isacharparenright}{\kern0pt}\ {\isachardollar}{\kern0pt}{\isachardollar}{\kern0pt}\ {\isacharparenleft}{\kern0pt}{\isadigit{0}}{\isacharcomma}{\kern0pt}{\isadigit{0}}{\isacharparenright}{\kern0pt}{\isacharparenright}{\kern0pt}{\isachardoublequoteclose}\isanewline
\ \ \ \ \ \ \ \ \ \ \ \ \isacommand{using}\isamarkupfalse%
\ dagger{\isacharunderscore}{\kern0pt}def\ \isanewline
\ \ \ \ \ \ \ \ \ \ \ \ \isacommand{by}\isamarkupfalse%
\ {\isacharparenleft}{\kern0pt}metis\ carrier{\isacharunderscore}{\kern0pt}matD{\isacharparenleft}{\kern0pt}{\isadigit{1}}{\isacharparenright}{\kern0pt}\ carrier{\isacharunderscore}{\kern0pt}matD{\isacharparenleft}{\kern0pt}{\isadigit{2}}{\isacharparenright}{\kern0pt}\ control{\isadigit{2}}{\isacharunderscore}{\kern0pt}carrier{\isacharunderscore}{\kern0pt}mat\ i{\isadigit{0}}\ i{\isadigit{4}}\ index{\isacharunderscore}{\kern0pt}mat{\isacharparenleft}{\kern0pt}{\isadigit{1}}{\isacharparenright}{\kern0pt}\ \isanewline
\ \ \ \ \ \ \ \ \ \ \ \ \ \ \ \ old{\isachardot}{\kern0pt}prod{\isachardot}{\kern0pt}case{\isacharparenright}{\kern0pt}\isanewline
\ \ \ \ \ \ \ \ \ \ \isacommand{also}\isamarkupfalse%
\ \isacommand{have}\isamarkupfalse%
\ {\isachardoublequoteopen}{\isasymdots}\ {\isacharequal}{\kern0pt}\ {\isadigit{1}}{\isachardoublequoteclose}\ \isacommand{using}\isamarkupfalse%
\ control{\isadigit{2}}{\isacharunderscore}{\kern0pt}def\ index{\isacharunderscore}{\kern0pt}mat{\isacharunderscore}{\kern0pt}of{\isacharunderscore}{\kern0pt}cols{\isacharunderscore}{\kern0pt}list\ \isacommand{by}\isamarkupfalse%
\ auto\isanewline
\ \ \ \ \ \ \ \ \ \ \isacommand{also}\isamarkupfalse%
\ \isacommand{have}\isamarkupfalse%
\ {\isachardoublequoteopen}{\isasymdots}\ {\isacharequal}{\kern0pt}\ {\isadigit{1}}\isactrlsub m\ {\isadigit{4}}\ {\isachardollar}{\kern0pt}{\isachardollar}{\kern0pt}\ {\isacharparenleft}{\kern0pt}{\isadigit{0}}{\isacharcomma}{\kern0pt}{\isadigit{0}}{\isacharparenright}{\kern0pt}{\isachardoublequoteclose}\ \isacommand{by}\isamarkupfalse%
\ simp\isanewline
\ \ \ \ \ \ \ \ \ \ \isacommand{finally}\isamarkupfalse%
\ \isacommand{show}\isamarkupfalse%
\ {\isacharquery}{\kern0pt}thesis\ \isacommand{using}\isamarkupfalse%
\ i{\isadigit{0}}\ j{\isadigit{0}}\ \isacommand{by}\isamarkupfalse%
\ simp\isanewline
\ \ \ \ \ \ \ \ \isacommand{qed}\isamarkupfalse%
\isanewline
\ \ \ \ \ \ \isacommand{next}\isamarkupfalse%
\isanewline
\ \ \ \ \ \ \ \ \isacommand{assume}\isamarkupfalse%
\ jl{\isadigit{3}}{\isacharcolon}{\kern0pt}{\isachardoublequoteopen}j\ {\isacharequal}{\kern0pt}\ {\isadigit{1}}\ {\isasymor}\ j\ {\isacharequal}{\kern0pt}\ {\isadigit{2}}\ {\isasymor}\ j\ {\isacharequal}{\kern0pt}\ {\isadigit{3}}{\isachardoublequoteclose}\isanewline
\ \ \ \ \ \ \ \ \isacommand{show}\isamarkupfalse%
\ {\isachardoublequoteopen}{\isacharparenleft}{\kern0pt}control{\isadigit{2}}\ U\ {\isacharasterisk}{\kern0pt}\ {\isacharparenleft}{\kern0pt}{\isacharparenleft}{\kern0pt}control{\isadigit{2}}\ U{\isacharparenright}{\kern0pt}\isactrlsup {\isasymdagger}{\isacharparenright}{\kern0pt}{\isacharparenright}{\kern0pt}\ {\isachardollar}{\kern0pt}{\isachardollar}{\kern0pt}\ {\isacharparenleft}{\kern0pt}i{\isacharcomma}{\kern0pt}\ j{\isacharparenright}{\kern0pt}\ {\isacharequal}{\kern0pt}\ {\isadigit{1}}\isactrlsub m\ {\isadigit{4}}\ {\isachardollar}{\kern0pt}{\isachardollar}{\kern0pt}\ {\isacharparenleft}{\kern0pt}i{\isacharcomma}{\kern0pt}\ j{\isacharparenright}{\kern0pt}{\isachardoublequoteclose}\isanewline
\ \ \ \ \ \ \ \ \isacommand{proof}\isamarkupfalse%
\ {\isacharparenleft}{\kern0pt}rule\ disjE{\isacharparenright}{\kern0pt}\isanewline
\ \ \ \ \ \ \ \ \ \ \isacommand{show}\isamarkupfalse%
\ {\isachardoublequoteopen}j\ {\isacharequal}{\kern0pt}\ {\isadigit{1}}\ {\isasymor}\ j\ {\isacharequal}{\kern0pt}\ {\isadigit{2}}\ {\isasymor}\ j\ {\isacharequal}{\kern0pt}\ {\isadigit{3}}{\isachardoublequoteclose}\ \isacommand{using}\isamarkupfalse%
\ jl{\isadigit{3}}\ \isacommand{by}\isamarkupfalse%
\ this\isanewline
\ \ \ \ \ \ \ \ \isacommand{next}\isamarkupfalse%
\isanewline
\ \ \ \ \ \ \ \ \ \ \isacommand{assume}\isamarkupfalse%
\ j{\isadigit{1}}{\isacharcolon}{\kern0pt}{\isachardoublequoteopen}j\ {\isacharequal}{\kern0pt}\ {\isadigit{1}}{\isachardoublequoteclose}\isanewline
\ \ \ \ \ \ \ \ \ \ \isacommand{show}\isamarkupfalse%
\ {\isachardoublequoteopen}{\isacharparenleft}{\kern0pt}control{\isadigit{2}}\ U\ {\isacharasterisk}{\kern0pt}\ {\isacharparenleft}{\kern0pt}{\isacharparenleft}{\kern0pt}control{\isadigit{2}}\ U{\isacharparenright}{\kern0pt}\isactrlsup {\isasymdagger}{\isacharparenright}{\kern0pt}{\isacharparenright}{\kern0pt}\ {\isachardollar}{\kern0pt}{\isachardollar}{\kern0pt}\ {\isacharparenleft}{\kern0pt}i{\isacharcomma}{\kern0pt}\ j{\isacharparenright}{\kern0pt}\ {\isacharequal}{\kern0pt}\ {\isadigit{1}}\isactrlsub m\ {\isadigit{4}}\ {\isachardollar}{\kern0pt}{\isachardollar}{\kern0pt}\ {\isacharparenleft}{\kern0pt}i{\isacharcomma}{\kern0pt}\ j{\isacharparenright}{\kern0pt}{\isachardoublequoteclose}\isanewline
\ \ \ \ \ \ \ \ \ \ \isacommand{proof}\isamarkupfalse%
\ {\isacharminus}{\kern0pt}\isanewline
\ \ \ \ \ \ \ \ \ \ \ \ \isacommand{have}\isamarkupfalse%
\ {\isachardoublequoteopen}{\isacharparenleft}{\kern0pt}control{\isadigit{2}}\ U\ {\isacharasterisk}{\kern0pt}\ {\isacharparenleft}{\kern0pt}{\isacharparenleft}{\kern0pt}control{\isadigit{2}}\ U{\isacharparenright}{\kern0pt}\isactrlsup {\isasymdagger}{\isacharparenright}{\kern0pt}{\isacharparenright}{\kern0pt}\ {\isachardollar}{\kern0pt}{\isachardollar}{\kern0pt}\ {\isacharparenleft}{\kern0pt}{\isadigit{0}}{\isacharcomma}{\kern0pt}{\isadigit{1}}{\isacharparenright}{\kern0pt}\ {\isacharequal}{\kern0pt}\ \isanewline
\ \ \ \ \ \ \ \ \ \ \ \ \ \ \ \ \ \ {\isacharparenleft}{\kern0pt}control{\isadigit{2}}\ U{\isacharparenright}{\kern0pt}\ {\isachardollar}{\kern0pt}{\isachardollar}{\kern0pt}\ {\isacharparenleft}{\kern0pt}{\isadigit{0}}{\isacharcomma}{\kern0pt}{\isadigit{0}}{\isacharparenright}{\kern0pt}\ {\isacharasterisk}{\kern0pt}\ {\isacharparenleft}{\kern0pt}{\isacharparenleft}{\kern0pt}control{\isadigit{2}}\ U{\isacharparenright}{\kern0pt}\isactrlsup {\isasymdagger}{\isacharparenright}{\kern0pt}\ {\isachardollar}{\kern0pt}{\isachardollar}{\kern0pt}\ {\isacharparenleft}{\kern0pt}{\isadigit{0}}{\isacharcomma}{\kern0pt}{\isadigit{1}}{\isacharparenright}{\kern0pt}\ {\isacharplus}{\kern0pt}\isanewline
\ \ \ \ \ \ \ \ \ \ \ \ \ \ \ \ \ \ {\isacharparenleft}{\kern0pt}control{\isadigit{2}}\ U{\isacharparenright}{\kern0pt}\ {\isachardollar}{\kern0pt}{\isachardollar}{\kern0pt}\ {\isacharparenleft}{\kern0pt}{\isadigit{0}}{\isacharcomma}{\kern0pt}{\isadigit{1}}{\isacharparenright}{\kern0pt}\ {\isacharasterisk}{\kern0pt}\ {\isacharparenleft}{\kern0pt}{\isacharparenleft}{\kern0pt}control{\isadigit{2}}\ U{\isacharparenright}{\kern0pt}\isactrlsup {\isasymdagger}{\isacharparenright}{\kern0pt}\ {\isachardollar}{\kern0pt}{\isachardollar}{\kern0pt}\ {\isacharparenleft}{\kern0pt}{\isadigit{1}}{\isacharcomma}{\kern0pt}{\isadigit{1}}{\isacharparenright}{\kern0pt}\ {\isacharplus}{\kern0pt}\isanewline
\ \ \ \ \ \ \ \ \ \ \ \ \ \ \ \ \ \ {\isacharparenleft}{\kern0pt}control{\isadigit{2}}\ U{\isacharparenright}{\kern0pt}\ {\isachardollar}{\kern0pt}{\isachardollar}{\kern0pt}\ {\isacharparenleft}{\kern0pt}{\isadigit{0}}{\isacharcomma}{\kern0pt}{\isadigit{2}}{\isacharparenright}{\kern0pt}\ {\isacharasterisk}{\kern0pt}\ {\isacharparenleft}{\kern0pt}{\isacharparenleft}{\kern0pt}control{\isadigit{2}}\ U{\isacharparenright}{\kern0pt}\isactrlsup {\isasymdagger}{\isacharparenright}{\kern0pt}\ {\isachardollar}{\kern0pt}{\isachardollar}{\kern0pt}\ {\isacharparenleft}{\kern0pt}{\isadigit{2}}{\isacharcomma}{\kern0pt}{\isadigit{1}}{\isacharparenright}{\kern0pt}\ {\isacharplus}{\kern0pt}\isanewline
\ \ \ \ \ \ \ \ \ \ \ \ \ \ \ \ \ \ {\isacharparenleft}{\kern0pt}control{\isadigit{2}}\ U{\isacharparenright}{\kern0pt}\ {\isachardollar}{\kern0pt}{\isachardollar}{\kern0pt}\ {\isacharparenleft}{\kern0pt}{\isadigit{0}}{\isacharcomma}{\kern0pt}{\isadigit{3}}{\isacharparenright}{\kern0pt}\ {\isacharasterisk}{\kern0pt}\ {\isacharparenleft}{\kern0pt}{\isacharparenleft}{\kern0pt}control{\isadigit{2}}\ U{\isacharparenright}{\kern0pt}\isactrlsup {\isasymdagger}{\isacharparenright}{\kern0pt}\ {\isachardollar}{\kern0pt}{\isachardollar}{\kern0pt}\ {\isacharparenleft}{\kern0pt}{\isadigit{3}}{\isacharcomma}{\kern0pt}{\isadigit{1}}{\isacharparenright}{\kern0pt}{\isachardoublequoteclose}\isanewline
\ \ \ \ \ \ \ \ \ \ \ \ \ \ \isacommand{using}\isamarkupfalse%
\ times{\isacharunderscore}{\kern0pt}mat{\isacharunderscore}{\kern0pt}def\ sumof{\isadigit{4}}\isanewline
\ \ \ \ \ \ \ \ \ \ \ \ \ \ \isacommand{by}\isamarkupfalse%
\ {\isacharparenleft}{\kern0pt}smt\ {\isacharparenleft}{\kern0pt}z{\isadigit{3}}{\isacharparenright}{\kern0pt}\ carrier{\isacharunderscore}{\kern0pt}matD{\isacharparenleft}{\kern0pt}{\isadigit{1}}{\isacharparenright}{\kern0pt}\ carrier{\isacharunderscore}{\kern0pt}matD{\isacharparenleft}{\kern0pt}{\isadigit{2}}{\isacharparenright}{\kern0pt}\ control{\isadigit{2}}{\isacharunderscore}{\kern0pt}carrier{\isacharunderscore}{\kern0pt}mat\ dim{\isacharunderscore}{\kern0pt}col{\isacharunderscore}{\kern0pt}of{\isacharunderscore}{\kern0pt}dagger\ \isanewline
\ \ \ \ \ \ \ \ \ \ \ \ \ \ \ \ \ \ \ \ dim{\isacharunderscore}{\kern0pt}row{\isacharunderscore}{\kern0pt}of{\isacharunderscore}{\kern0pt}dagger\ i{\isadigit{0}}\ i{\isadigit{4}}\ index{\isacharunderscore}{\kern0pt}matrix{\isacharunderscore}{\kern0pt}prod\ j{\isadigit{1}}\ j{\isadigit{4}}{\isacharparenright}{\kern0pt}\isanewline
\ \ \ \ \ \ \ \ \ \ \ \ \isacommand{also}\isamarkupfalse%
\ \isacommand{have}\isamarkupfalse%
\ {\isachardoublequoteopen}{\isasymdots}\ {\isacharequal}{\kern0pt}\ {\isacharparenleft}{\kern0pt}{\isacharparenleft}{\kern0pt}control{\isadigit{2}}\ U{\isacharparenright}{\kern0pt}\isactrlsup {\isasymdagger}{\isacharparenright}{\kern0pt}\ {\isachardollar}{\kern0pt}{\isachardollar}{\kern0pt}\ {\isacharparenleft}{\kern0pt}{\isadigit{0}}{\isacharcomma}{\kern0pt}{\isadigit{1}}{\isacharparenright}{\kern0pt}{\isachardoublequoteclose}\isanewline
\ \ \ \ \ \ \ \ \ \ \ \ \ \ \isacommand{using}\isamarkupfalse%
\ control{\isadigit{2}}{\isacharunderscore}{\kern0pt}def\ index{\isacharunderscore}{\kern0pt}mat{\isacharunderscore}{\kern0pt}of{\isacharunderscore}{\kern0pt}cols{\isacharunderscore}{\kern0pt}list\ \isacommand{by}\isamarkupfalse%
\ force\isanewline
\ \ \ \ \ \ \ \ \ \ \ \ \isacommand{also}\isamarkupfalse%
\ \isacommand{have}\isamarkupfalse%
\ {\isachardoublequoteopen}{\isasymdots}\ {\isacharequal}{\kern0pt}\ cnj\ {\isacharparenleft}{\kern0pt}{\isacharparenleft}{\kern0pt}control{\isadigit{2}}\ U{\isacharparenright}{\kern0pt}\ {\isachardollar}{\kern0pt}{\isachardollar}{\kern0pt}\ {\isacharparenleft}{\kern0pt}{\isadigit{1}}{\isacharcomma}{\kern0pt}{\isadigit{0}}{\isacharparenright}{\kern0pt}{\isacharparenright}{\kern0pt}{\isachardoublequoteclose}\isanewline
\ \ \ \ \ \ \ \ \ \ \ \ \ \ \isacommand{using}\isamarkupfalse%
\ dagger{\isacharunderscore}{\kern0pt}def\ \isanewline
\ \ \ \ \ \ \ \ \ \ \ \ \ \ \isacommand{by}\isamarkupfalse%
\ {\isacharparenleft}{\kern0pt}metis\ {\isacharparenleft}{\kern0pt}mono{\isacharunderscore}{\kern0pt}tags{\isacharcomma}{\kern0pt}\ lifting{\isacharparenright}{\kern0pt}\ One{\isacharunderscore}{\kern0pt}nat{\isacharunderscore}{\kern0pt}def\ Suc{\isacharunderscore}{\kern0pt}{\isadigit{1}}\ add{\isacharunderscore}{\kern0pt}Suc{\isacharunderscore}{\kern0pt}right\ carrier{\isacharunderscore}{\kern0pt}matD{\isacharparenleft}{\kern0pt}{\isadigit{1}}{\isacharparenright}{\kern0pt}\ \isanewline
\ \ \ \ \ \ \ \ \ \ \ \ \ \ \ \ \ \ carrier{\isacharunderscore}{\kern0pt}matD{\isacharparenleft}{\kern0pt}{\isadigit{2}}{\isacharparenright}{\kern0pt}\ control{\isadigit{2}}{\isacharunderscore}{\kern0pt}carrier{\isacharunderscore}{\kern0pt}mat\ index{\isacharunderscore}{\kern0pt}mat{\isacharparenleft}{\kern0pt}{\isadigit{1}}{\isacharparenright}{\kern0pt}\ less{\isacharunderscore}{\kern0pt}Suc{\isacharunderscore}{\kern0pt}eq{\isacharunderscore}{\kern0pt}{\isadigit{0}}{\isacharunderscore}{\kern0pt}disj\ numeral{\isacharunderscore}{\kern0pt}Bit{\isadigit{0}}\isanewline
\ \ \ \ \ \ \ \ \ \ \ \ \ \ \ \ \ \ prod{\isachardot}{\kern0pt}simps{\isacharparenleft}{\kern0pt}{\isadigit{2}}{\isacharparenright}{\kern0pt}{\isacharparenright}{\kern0pt}\isanewline
\ \ \ \ \ \ \ \ \ \ \ \ \isacommand{also}\isamarkupfalse%
\ \isacommand{have}\isamarkupfalse%
\ {\isachardoublequoteopen}{\isasymdots}\ {\isacharequal}{\kern0pt}\ {\isadigit{0}}{\isachardoublequoteclose}\ \isacommand{using}\isamarkupfalse%
\ control{\isadigit{2}}{\isacharunderscore}{\kern0pt}def\ index{\isacharunderscore}{\kern0pt}mat{\isacharunderscore}{\kern0pt}of{\isacharunderscore}{\kern0pt}cols{\isacharunderscore}{\kern0pt}list\ \isacommand{by}\isamarkupfalse%
\ auto\isanewline
\ \ \ \ \ \ \ \ \ \ \ \ \isacommand{also}\isamarkupfalse%
\ \isacommand{have}\isamarkupfalse%
\ {\isachardoublequoteopen}{\isasymdots}\ {\isacharequal}{\kern0pt}\ {\isadigit{1}}\isactrlsub m\ {\isadigit{4}}\ {\isachardollar}{\kern0pt}{\isachardollar}{\kern0pt}\ {\isacharparenleft}{\kern0pt}{\isadigit{0}}{\isacharcomma}{\kern0pt}{\isadigit{1}}{\isacharparenright}{\kern0pt}{\isachardoublequoteclose}\ \isacommand{by}\isamarkupfalse%
\ simp\isanewline
\ \ \ \ \ \ \ \ \ \ \ \ \isacommand{finally}\isamarkupfalse%
\ \isacommand{show}\isamarkupfalse%
\ {\isacharquery}{\kern0pt}thesis\ \isacommand{using}\isamarkupfalse%
\ i{\isadigit{0}}\ j{\isadigit{1}}\ \isacommand{by}\isamarkupfalse%
\ simp\isanewline
\ \ \ \ \ \ \ \ \ \ \isacommand{qed}\isamarkupfalse%
\isanewline
\ \ \ \ \ \ \ \ \isacommand{next}\isamarkupfalse%
\isanewline
\ \ \ \ \ \ \ \ \ \ \isacommand{assume}\isamarkupfalse%
\ jl{\isadigit{2}}{\isacharcolon}{\kern0pt}{\isachardoublequoteopen}j\ {\isacharequal}{\kern0pt}\ {\isadigit{2}}\ {\isasymor}\ j\ {\isacharequal}{\kern0pt}\ {\isadigit{3}}{\isachardoublequoteclose}\isanewline
\ \ \ \ \ \ \ \ \ \ \isacommand{show}\isamarkupfalse%
\ {\isachardoublequoteopen}{\isacharparenleft}{\kern0pt}control{\isadigit{2}}\ U\ {\isacharasterisk}{\kern0pt}\ {\isacharparenleft}{\kern0pt}{\isacharparenleft}{\kern0pt}control{\isadigit{2}}\ U{\isacharparenright}{\kern0pt}\isactrlsup {\isasymdagger}{\isacharparenright}{\kern0pt}{\isacharparenright}{\kern0pt}\ {\isachardollar}{\kern0pt}{\isachardollar}{\kern0pt}\ {\isacharparenleft}{\kern0pt}i{\isacharcomma}{\kern0pt}\ j{\isacharparenright}{\kern0pt}\ {\isacharequal}{\kern0pt}\ {\isadigit{1}}\isactrlsub m\ {\isadigit{4}}\ {\isachardollar}{\kern0pt}{\isachardollar}{\kern0pt}\ {\isacharparenleft}{\kern0pt}i{\isacharcomma}{\kern0pt}\ j{\isacharparenright}{\kern0pt}{\isachardoublequoteclose}\isanewline
\ \ \ \ \ \ \ \ \ \ \isacommand{proof}\isamarkupfalse%
\ {\isacharparenleft}{\kern0pt}rule\ disjE{\isacharparenright}{\kern0pt}\isanewline
\ \ \ \ \ \ \ \ \ \ \ \ \isacommand{show}\isamarkupfalse%
\ {\isachardoublequoteopen}j\ {\isacharequal}{\kern0pt}\ {\isadigit{2}}\ {\isasymor}\ j\ {\isacharequal}{\kern0pt}\ {\isadigit{3}}{\isachardoublequoteclose}\ \isacommand{using}\isamarkupfalse%
\ jl{\isadigit{2}}\ \isacommand{by}\isamarkupfalse%
\ this\isanewline
\ \ \ \ \ \ \ \ \ \ \isacommand{next}\isamarkupfalse%
\isanewline
\ \ \ \ \ \ \ \ \ \ \ \ \isacommand{assume}\isamarkupfalse%
\ j{\isadigit{2}}{\isacharcolon}{\kern0pt}{\isachardoublequoteopen}j\ {\isacharequal}{\kern0pt}\ {\isadigit{2}}{\isachardoublequoteclose}\isanewline
\ \ \ \ \ \ \ \ \ \ \ \ \isacommand{show}\isamarkupfalse%
\ {\isachardoublequoteopen}{\isacharparenleft}{\kern0pt}control{\isadigit{2}}\ U\ {\isacharasterisk}{\kern0pt}\ {\isacharparenleft}{\kern0pt}{\isacharparenleft}{\kern0pt}control{\isadigit{2}}\ U{\isacharparenright}{\kern0pt}\isactrlsup {\isasymdagger}{\isacharparenright}{\kern0pt}{\isacharparenright}{\kern0pt}\ {\isachardollar}{\kern0pt}{\isachardollar}{\kern0pt}\ {\isacharparenleft}{\kern0pt}i{\isacharcomma}{\kern0pt}\ j{\isacharparenright}{\kern0pt}\ {\isacharequal}{\kern0pt}\ {\isadigit{1}}\isactrlsub m\ {\isadigit{4}}\ {\isachardollar}{\kern0pt}{\isachardollar}{\kern0pt}\ {\isacharparenleft}{\kern0pt}i{\isacharcomma}{\kern0pt}\ j{\isacharparenright}{\kern0pt}{\isachardoublequoteclose}\isanewline
\ \ \ \ \ \ \ \ \ \ \ \ \isacommand{proof}\isamarkupfalse%
\ {\isacharminus}{\kern0pt}\isanewline
\ \ \ \ \ \ \ \ \ \ \ \ \ \ \isacommand{have}\isamarkupfalse%
\ {\isachardoublequoteopen}{\isacharparenleft}{\kern0pt}control{\isadigit{2}}\ U\ {\isacharasterisk}{\kern0pt}\ {\isacharparenleft}{\kern0pt}{\isacharparenleft}{\kern0pt}control{\isadigit{2}}\ U{\isacharparenright}{\kern0pt}\isactrlsup {\isasymdagger}{\isacharparenright}{\kern0pt}{\isacharparenright}{\kern0pt}\ {\isachardollar}{\kern0pt}{\isachardollar}{\kern0pt}\ {\isacharparenleft}{\kern0pt}{\isadigit{0}}{\isacharcomma}{\kern0pt}{\isadigit{2}}{\isacharparenright}{\kern0pt}\ {\isacharequal}{\kern0pt}\ \isanewline
\ \ \ \ \ \ \ \ \ \ \ \ \ \ \ \ \ \ \ \ {\isacharparenleft}{\kern0pt}control{\isadigit{2}}\ U{\isacharparenright}{\kern0pt}\ {\isachardollar}{\kern0pt}{\isachardollar}{\kern0pt}\ {\isacharparenleft}{\kern0pt}{\isadigit{0}}{\isacharcomma}{\kern0pt}{\isadigit{0}}{\isacharparenright}{\kern0pt}\ {\isacharasterisk}{\kern0pt}\ {\isacharparenleft}{\kern0pt}{\isacharparenleft}{\kern0pt}control{\isadigit{2}}\ U{\isacharparenright}{\kern0pt}\isactrlsup {\isasymdagger}{\isacharparenright}{\kern0pt}\ {\isachardollar}{\kern0pt}{\isachardollar}{\kern0pt}\ {\isacharparenleft}{\kern0pt}{\isadigit{0}}{\isacharcomma}{\kern0pt}{\isadigit{2}}{\isacharparenright}{\kern0pt}\ {\isacharplus}{\kern0pt}\isanewline
\ \ \ \ \ \ \ \ \ \ \ \ \ \ \ \ \ \ \ \ {\isacharparenleft}{\kern0pt}control{\isadigit{2}}\ U{\isacharparenright}{\kern0pt}\ {\isachardollar}{\kern0pt}{\isachardollar}{\kern0pt}\ {\isacharparenleft}{\kern0pt}{\isadigit{0}}{\isacharcomma}{\kern0pt}{\isadigit{1}}{\isacharparenright}{\kern0pt}\ {\isacharasterisk}{\kern0pt}\ {\isacharparenleft}{\kern0pt}{\isacharparenleft}{\kern0pt}control{\isadigit{2}}\ U{\isacharparenright}{\kern0pt}\isactrlsup {\isasymdagger}{\isacharparenright}{\kern0pt}\ {\isachardollar}{\kern0pt}{\isachardollar}{\kern0pt}\ {\isacharparenleft}{\kern0pt}{\isadigit{1}}{\isacharcomma}{\kern0pt}{\isadigit{2}}{\isacharparenright}{\kern0pt}\ {\isacharplus}{\kern0pt}\isanewline
\ \ \ \ \ \ \ \ \ \ \ \ \ \ \ \ \ \ \ \ {\isacharparenleft}{\kern0pt}control{\isadigit{2}}\ U{\isacharparenright}{\kern0pt}\ {\isachardollar}{\kern0pt}{\isachardollar}{\kern0pt}\ {\isacharparenleft}{\kern0pt}{\isadigit{0}}{\isacharcomma}{\kern0pt}{\isadigit{2}}{\isacharparenright}{\kern0pt}\ {\isacharasterisk}{\kern0pt}\ {\isacharparenleft}{\kern0pt}{\isacharparenleft}{\kern0pt}control{\isadigit{2}}\ U{\isacharparenright}{\kern0pt}\isactrlsup {\isasymdagger}{\isacharparenright}{\kern0pt}\ {\isachardollar}{\kern0pt}{\isachardollar}{\kern0pt}\ {\isacharparenleft}{\kern0pt}{\isadigit{2}}{\isacharcomma}{\kern0pt}{\isadigit{2}}{\isacharparenright}{\kern0pt}\ {\isacharplus}{\kern0pt}\isanewline
\ \ \ \ \ \ \ \ \ \ \ \ \ \ \ \ \ \ \ \ {\isacharparenleft}{\kern0pt}control{\isadigit{2}}\ U{\isacharparenright}{\kern0pt}\ {\isachardollar}{\kern0pt}{\isachardollar}{\kern0pt}\ {\isacharparenleft}{\kern0pt}{\isadigit{0}}{\isacharcomma}{\kern0pt}{\isadigit{3}}{\isacharparenright}{\kern0pt}\ {\isacharasterisk}{\kern0pt}\ {\isacharparenleft}{\kern0pt}{\isacharparenleft}{\kern0pt}control{\isadigit{2}}\ U{\isacharparenright}{\kern0pt}\isactrlsup {\isasymdagger}{\isacharparenright}{\kern0pt}\ {\isachardollar}{\kern0pt}{\isachardollar}{\kern0pt}\ {\isacharparenleft}{\kern0pt}{\isadigit{3}}{\isacharcomma}{\kern0pt}{\isadigit{2}}{\isacharparenright}{\kern0pt}{\isachardoublequoteclose}\isanewline
\ \ \ \ \ \ \ \ \ \ \ \ \ \ \ \ \isacommand{using}\isamarkupfalse%
\ times{\isacharunderscore}{\kern0pt}mat{\isacharunderscore}{\kern0pt}def\ sumof{\isadigit{4}}\isanewline
\ \ \ \ \ \ \ \ \ \ \ \ \ \ \ \ \isacommand{by}\isamarkupfalse%
\ {\isacharparenleft}{\kern0pt}smt\ {\isacharparenleft}{\kern0pt}z{\isadigit{3}}{\isacharparenright}{\kern0pt}\ carrier{\isacharunderscore}{\kern0pt}matD{\isacharparenleft}{\kern0pt}{\isadigit{1}}{\isacharparenright}{\kern0pt}\ carrier{\isacharunderscore}{\kern0pt}matD{\isacharparenleft}{\kern0pt}{\isadigit{2}}{\isacharparenright}{\kern0pt}\ control{\isadigit{2}}{\isacharunderscore}{\kern0pt}carrier{\isacharunderscore}{\kern0pt}mat\ dim{\isacharunderscore}{\kern0pt}col{\isacharunderscore}{\kern0pt}of{\isacharunderscore}{\kern0pt}dagger\ \isanewline
\ \ \ \ \ \ \ \ \ \ \ \ \ \ \ \ \ \ \ \ \ \ dim{\isacharunderscore}{\kern0pt}row{\isacharunderscore}{\kern0pt}of{\isacharunderscore}{\kern0pt}dagger\ i{\isadigit{0}}\ i{\isadigit{4}}\ index{\isacharunderscore}{\kern0pt}matrix{\isacharunderscore}{\kern0pt}prod\ j{\isadigit{2}}\ j{\isadigit{4}}{\isacharparenright}{\kern0pt}\isanewline
\ \ \ \ \ \ \ \ \ \ \ \ \ \ \isacommand{also}\isamarkupfalse%
\ \isacommand{have}\isamarkupfalse%
\ {\isachardoublequoteopen}{\isasymdots}\ {\isacharequal}{\kern0pt}\ {\isacharparenleft}{\kern0pt}{\isacharparenleft}{\kern0pt}control{\isadigit{2}}\ U{\isacharparenright}{\kern0pt}\isactrlsup {\isasymdagger}{\isacharparenright}{\kern0pt}\ {\isachardollar}{\kern0pt}{\isachardollar}{\kern0pt}\ {\isacharparenleft}{\kern0pt}{\isadigit{0}}{\isacharcomma}{\kern0pt}{\isadigit{2}}{\isacharparenright}{\kern0pt}{\isachardoublequoteclose}\isanewline
\ \ \ \ \ \ \ \ \ \ \ \ \ \ \ \ \isacommand{using}\isamarkupfalse%
\ control{\isadigit{2}}{\isacharunderscore}{\kern0pt}def\ index{\isacharunderscore}{\kern0pt}mat{\isacharunderscore}{\kern0pt}of{\isacharunderscore}{\kern0pt}cols{\isacharunderscore}{\kern0pt}list\ \isacommand{by}\isamarkupfalse%
\ force\isanewline
\ \ \ \ \ \ \ \ \ \ \ \ \ \ \isacommand{also}\isamarkupfalse%
\ \isacommand{have}\isamarkupfalse%
\ {\isachardoublequoteopen}{\isasymdots}\ {\isacharequal}{\kern0pt}\ cnj\ {\isacharparenleft}{\kern0pt}{\isacharparenleft}{\kern0pt}control{\isadigit{2}}\ U{\isacharparenright}{\kern0pt}\ {\isachardollar}{\kern0pt}{\isachardollar}{\kern0pt}\ {\isacharparenleft}{\kern0pt}{\isadigit{2}}{\isacharcomma}{\kern0pt}{\isadigit{0}}{\isacharparenright}{\kern0pt}{\isacharparenright}{\kern0pt}{\isachardoublequoteclose}\isanewline
\ \ \ \ \ \ \ \ \ \ \ \ \ \ \ \ \isacommand{using}\isamarkupfalse%
\ dagger{\isacharunderscore}{\kern0pt}def\ \isanewline
\ \ \ \ \ \ \ \ \ \ \ \ \ \ \ \ \isacommand{by}\isamarkupfalse%
\ {\isacharparenleft}{\kern0pt}smt\ {\isacharparenleft}{\kern0pt}verit{\isacharcomma}{\kern0pt}\ del{\isacharunderscore}{\kern0pt}insts{\isacharparenright}{\kern0pt}\ carrier{\isacharunderscore}{\kern0pt}matD{\isacharparenleft}{\kern0pt}{\isadigit{1}}{\isacharparenright}{\kern0pt}\ carrier{\isacharunderscore}{\kern0pt}matD{\isacharparenleft}{\kern0pt}{\isadigit{2}}{\isacharparenright}{\kern0pt}\ control{\isadigit{2}}{\isacharunderscore}{\kern0pt}carrier{\isacharunderscore}{\kern0pt}mat\ \isanewline
\ \ \ \ \ \ \ \ \ \ \ \ \ \ \ \ \ \ \ \ index{\isacharunderscore}{\kern0pt}mat{\isacharparenleft}{\kern0pt}{\isadigit{1}}{\isacharparenright}{\kern0pt}\ less{\isacharunderscore}{\kern0pt}add{\isacharunderscore}{\kern0pt}same{\isacharunderscore}{\kern0pt}cancel{\isadigit{2}}\ numeral{\isacharunderscore}{\kern0pt}Bit{\isadigit{0}}\ prod{\isachardot}{\kern0pt}simps{\isacharparenleft}{\kern0pt}{\isadigit{2}}{\isacharparenright}{\kern0pt}\ zero{\isacharunderscore}{\kern0pt}less{\isacharunderscore}{\kern0pt}numeral{\isacharparenright}{\kern0pt}\isanewline
\ \ \ \ \ \ \ \ \ \ \ \ \ \ \isacommand{also}\isamarkupfalse%
\ \isacommand{have}\isamarkupfalse%
\ {\isachardoublequoteopen}{\isasymdots}\ {\isacharequal}{\kern0pt}\ {\isadigit{0}}{\isachardoublequoteclose}\ \isacommand{using}\isamarkupfalse%
\ control{\isadigit{2}}{\isacharunderscore}{\kern0pt}def\ index{\isacharunderscore}{\kern0pt}mat{\isacharunderscore}{\kern0pt}of{\isacharunderscore}{\kern0pt}cols{\isacharunderscore}{\kern0pt}list\ \isacommand{by}\isamarkupfalse%
\ auto\isanewline
\ \ \ \ \ \ \ \ \ \ \ \ \ \ \isacommand{also}\isamarkupfalse%
\ \isacommand{have}\isamarkupfalse%
\ {\isachardoublequoteopen}{\isasymdots}\ {\isacharequal}{\kern0pt}\ {\isadigit{1}}\isactrlsub m\ {\isadigit{4}}\ {\isachardollar}{\kern0pt}{\isachardollar}{\kern0pt}\ {\isacharparenleft}{\kern0pt}{\isadigit{0}}{\isacharcomma}{\kern0pt}{\isadigit{2}}{\isacharparenright}{\kern0pt}{\isachardoublequoteclose}\ \isacommand{by}\isamarkupfalse%
\ simp\isanewline
\ \ \ \ \ \ \ \ \ \ \ \ \ \ \isacommand{finally}\isamarkupfalse%
\ \isacommand{show}\isamarkupfalse%
\ {\isacharquery}{\kern0pt}thesis\ \isacommand{using}\isamarkupfalse%
\ i{\isadigit{0}}\ j{\isadigit{2}}\ \isacommand{by}\isamarkupfalse%
\ simp\isanewline
\ \ \ \ \ \ \ \ \ \ \ \ \isacommand{qed}\isamarkupfalse%
\isanewline
\ \ \ \ \ \ \ \ \ \ \isacommand{next}\isamarkupfalse%
\isanewline
\ \ \ \ \ \ \ \ \ \ \ \ \isacommand{assume}\isamarkupfalse%
\ j{\isadigit{3}}{\isacharcolon}{\kern0pt}{\isachardoublequoteopen}j\ {\isacharequal}{\kern0pt}\ {\isadigit{3}}{\isachardoublequoteclose}\isanewline
\ \ \ \ \ \ \ \ \ \ \ \ \isacommand{show}\isamarkupfalse%
\ {\isachardoublequoteopen}{\isacharparenleft}{\kern0pt}control{\isadigit{2}}\ U\ {\isacharasterisk}{\kern0pt}\ {\isacharparenleft}{\kern0pt}{\isacharparenleft}{\kern0pt}control{\isadigit{2}}\ U{\isacharparenright}{\kern0pt}\isactrlsup {\isasymdagger}{\isacharparenright}{\kern0pt}{\isacharparenright}{\kern0pt}\ {\isachardollar}{\kern0pt}{\isachardollar}{\kern0pt}\ {\isacharparenleft}{\kern0pt}i{\isacharcomma}{\kern0pt}\ j{\isacharparenright}{\kern0pt}\ {\isacharequal}{\kern0pt}\ {\isadigit{1}}\isactrlsub m\ {\isadigit{4}}\ {\isachardollar}{\kern0pt}{\isachardollar}{\kern0pt}\ {\isacharparenleft}{\kern0pt}i{\isacharcomma}{\kern0pt}\ j{\isacharparenright}{\kern0pt}{\isachardoublequoteclose}\isanewline
\ \ \ \ \ \ \ \ \ \ \ \ \isacommand{proof}\isamarkupfalse%
\ {\isacharminus}{\kern0pt}\isanewline
\ \ \ \ \ \ \ \ \ \ \ \ \ \ \isacommand{have}\isamarkupfalse%
\ {\isachardoublequoteopen}{\isacharparenleft}{\kern0pt}control{\isadigit{2}}\ U\ {\isacharasterisk}{\kern0pt}\ {\isacharparenleft}{\kern0pt}{\isacharparenleft}{\kern0pt}control{\isadigit{2}}\ U{\isacharparenright}{\kern0pt}\isactrlsup {\isasymdagger}{\isacharparenright}{\kern0pt}{\isacharparenright}{\kern0pt}\ {\isachardollar}{\kern0pt}{\isachardollar}{\kern0pt}\ {\isacharparenleft}{\kern0pt}{\isadigit{0}}{\isacharcomma}{\kern0pt}{\isadigit{3}}{\isacharparenright}{\kern0pt}\ {\isacharequal}{\kern0pt}\ \isanewline
\ \ \ \ \ \ \ \ \ \ \ \ \ \ \ \ \ \ \ \ {\isacharparenleft}{\kern0pt}control{\isadigit{2}}\ U{\isacharparenright}{\kern0pt}\ {\isachardollar}{\kern0pt}{\isachardollar}{\kern0pt}\ {\isacharparenleft}{\kern0pt}{\isadigit{0}}{\isacharcomma}{\kern0pt}{\isadigit{0}}{\isacharparenright}{\kern0pt}\ {\isacharasterisk}{\kern0pt}\ {\isacharparenleft}{\kern0pt}{\isacharparenleft}{\kern0pt}control{\isadigit{2}}\ U{\isacharparenright}{\kern0pt}\isactrlsup {\isasymdagger}{\isacharparenright}{\kern0pt}\ {\isachardollar}{\kern0pt}{\isachardollar}{\kern0pt}\ {\isacharparenleft}{\kern0pt}{\isadigit{0}}{\isacharcomma}{\kern0pt}{\isadigit{3}}{\isacharparenright}{\kern0pt}\ {\isacharplus}{\kern0pt}\isanewline
\ \ \ \ \ \ \ \ \ \ \ \ \ \ \ \ \ \ \ \ {\isacharparenleft}{\kern0pt}control{\isadigit{2}}\ U{\isacharparenright}{\kern0pt}\ {\isachardollar}{\kern0pt}{\isachardollar}{\kern0pt}\ {\isacharparenleft}{\kern0pt}{\isadigit{0}}{\isacharcomma}{\kern0pt}{\isadigit{1}}{\isacharparenright}{\kern0pt}\ {\isacharasterisk}{\kern0pt}\ {\isacharparenleft}{\kern0pt}{\isacharparenleft}{\kern0pt}control{\isadigit{2}}\ U{\isacharparenright}{\kern0pt}\isactrlsup {\isasymdagger}{\isacharparenright}{\kern0pt}\ {\isachardollar}{\kern0pt}{\isachardollar}{\kern0pt}\ {\isacharparenleft}{\kern0pt}{\isadigit{1}}{\isacharcomma}{\kern0pt}{\isadigit{3}}{\isacharparenright}{\kern0pt}\ {\isacharplus}{\kern0pt}\isanewline
\ \ \ \ \ \ \ \ \ \ \ \ \ \ \ \ \ \ \ \ {\isacharparenleft}{\kern0pt}control{\isadigit{2}}\ U{\isacharparenright}{\kern0pt}\ {\isachardollar}{\kern0pt}{\isachardollar}{\kern0pt}\ {\isacharparenleft}{\kern0pt}{\isadigit{0}}{\isacharcomma}{\kern0pt}{\isadigit{2}}{\isacharparenright}{\kern0pt}\ {\isacharasterisk}{\kern0pt}\ {\isacharparenleft}{\kern0pt}{\isacharparenleft}{\kern0pt}control{\isadigit{2}}\ U{\isacharparenright}{\kern0pt}\isactrlsup {\isasymdagger}{\isacharparenright}{\kern0pt}\ {\isachardollar}{\kern0pt}{\isachardollar}{\kern0pt}\ {\isacharparenleft}{\kern0pt}{\isadigit{2}}{\isacharcomma}{\kern0pt}{\isadigit{3}}{\isacharparenright}{\kern0pt}\ {\isacharplus}{\kern0pt}\isanewline
\ \ \ \ \ \ \ \ \ \ \ \ \ \ \ \ \ \ \ \ {\isacharparenleft}{\kern0pt}control{\isadigit{2}}\ U{\isacharparenright}{\kern0pt}\ {\isachardollar}{\kern0pt}{\isachardollar}{\kern0pt}\ {\isacharparenleft}{\kern0pt}{\isadigit{0}}{\isacharcomma}{\kern0pt}{\isadigit{3}}{\isacharparenright}{\kern0pt}\ {\isacharasterisk}{\kern0pt}\ {\isacharparenleft}{\kern0pt}{\isacharparenleft}{\kern0pt}control{\isadigit{2}}\ U{\isacharparenright}{\kern0pt}\isactrlsup {\isasymdagger}{\isacharparenright}{\kern0pt}\ {\isachardollar}{\kern0pt}{\isachardollar}{\kern0pt}\ {\isacharparenleft}{\kern0pt}{\isadigit{3}}{\isacharcomma}{\kern0pt}{\isadigit{3}}{\isacharparenright}{\kern0pt}{\isachardoublequoteclose}\isanewline
\ \ \ \ \ \ \ \ \ \ \ \ \ \ \ \ \isacommand{using}\isamarkupfalse%
\ times{\isacharunderscore}{\kern0pt}mat{\isacharunderscore}{\kern0pt}def\ sumof{\isadigit{4}}\isanewline
\ \ \ \ \ \ \ \ \ \ \ \ \ \ \ \ \isacommand{by}\isamarkupfalse%
\ {\isacharparenleft}{\kern0pt}smt\ {\isacharparenleft}{\kern0pt}z{\isadigit{3}}{\isacharparenright}{\kern0pt}\ carrier{\isacharunderscore}{\kern0pt}matD{\isacharparenleft}{\kern0pt}{\isadigit{1}}{\isacharparenright}{\kern0pt}\ carrier{\isacharunderscore}{\kern0pt}matD{\isacharparenleft}{\kern0pt}{\isadigit{2}}{\isacharparenright}{\kern0pt}\ control{\isadigit{2}}{\isacharunderscore}{\kern0pt}carrier{\isacharunderscore}{\kern0pt}mat\ dim{\isacharunderscore}{\kern0pt}col{\isacharunderscore}{\kern0pt}of{\isacharunderscore}{\kern0pt}dagger\ \isanewline
\ \ \ \ \ \ \ \ \ \ \ \ \ \ \ \ \ \ \ \ \ \ dim{\isacharunderscore}{\kern0pt}row{\isacharunderscore}{\kern0pt}of{\isacharunderscore}{\kern0pt}dagger\ i{\isadigit{0}}\ i{\isadigit{4}}\ index{\isacharunderscore}{\kern0pt}matrix{\isacharunderscore}{\kern0pt}prod\ j{\isadigit{3}}\ j{\isadigit{4}}{\isacharparenright}{\kern0pt}\isanewline
\ \ \ \ \ \ \ \ \ \ \ \ \ \ \isacommand{also}\isamarkupfalse%
\ \isacommand{have}\isamarkupfalse%
\ {\isachardoublequoteopen}{\isasymdots}\ {\isacharequal}{\kern0pt}\ {\isacharparenleft}{\kern0pt}{\isacharparenleft}{\kern0pt}control{\isadigit{2}}\ U{\isacharparenright}{\kern0pt}\isactrlsup {\isasymdagger}{\isacharparenright}{\kern0pt}\ {\isachardollar}{\kern0pt}{\isachardollar}{\kern0pt}\ {\isacharparenleft}{\kern0pt}{\isadigit{0}}{\isacharcomma}{\kern0pt}{\isadigit{3}}{\isacharparenright}{\kern0pt}{\isachardoublequoteclose}\isanewline
\ \ \ \ \ \ \ \ \ \ \ \ \ \ \ \ \isacommand{using}\isamarkupfalse%
\ control{\isadigit{2}}{\isacharunderscore}{\kern0pt}def\ index{\isacharunderscore}{\kern0pt}mat{\isacharunderscore}{\kern0pt}of{\isacharunderscore}{\kern0pt}cols{\isacharunderscore}{\kern0pt}list\ \isacommand{by}\isamarkupfalse%
\ force\isanewline
\ \ \ \ \ \ \ \ \ \ \ \ \ \ \isacommand{also}\isamarkupfalse%
\ \isacommand{have}\isamarkupfalse%
\ {\isachardoublequoteopen}{\isasymdots}\ {\isacharequal}{\kern0pt}\ cnj\ {\isacharparenleft}{\kern0pt}{\isacharparenleft}{\kern0pt}control{\isadigit{2}}\ U{\isacharparenright}{\kern0pt}\ {\isachardollar}{\kern0pt}{\isachardollar}{\kern0pt}\ {\isacharparenleft}{\kern0pt}{\isadigit{3}}{\isacharcomma}{\kern0pt}{\isadigit{0}}{\isacharparenright}{\kern0pt}{\isacharparenright}{\kern0pt}{\isachardoublequoteclose}\isanewline
\ \ \ \ \ \ \ \ \ \ \ \ \ \ \ \ \isacommand{using}\isamarkupfalse%
\ dagger{\isacharunderscore}{\kern0pt}def\ \isanewline
\ \ \ \ \ \ \ \ \ \ \ \ \ \ \ \ \isacommand{by}\isamarkupfalse%
\ {\isacharparenleft}{\kern0pt}metis\ carrier{\isacharunderscore}{\kern0pt}matD{\isacharparenleft}{\kern0pt}{\isadigit{1}}{\isacharparenright}{\kern0pt}\ carrier{\isacharunderscore}{\kern0pt}matD{\isacharparenleft}{\kern0pt}{\isadigit{2}}{\isacharparenright}{\kern0pt}\ control{\isadigit{2}}{\isacharunderscore}{\kern0pt}carrier{\isacharunderscore}{\kern0pt}mat\ index{\isacharunderscore}{\kern0pt}mat{\isacharparenleft}{\kern0pt}{\isadigit{1}}{\isacharparenright}{\kern0pt}\ j{\isadigit{3}}\ j{\isadigit{4}}\ \isanewline
\ \ \ \ \ \ \ \ \ \ \ \ \ \ \ \ \ \ \ \ prod{\isachardot}{\kern0pt}simps{\isacharparenleft}{\kern0pt}{\isadigit{2}}{\isacharparenright}{\kern0pt}\ zero{\isacharunderscore}{\kern0pt}less{\isacharunderscore}{\kern0pt}numeral{\isacharparenright}{\kern0pt}\isanewline
\ \ \ \ \ \ \ \ \ \ \ \ \ \ \isacommand{also}\isamarkupfalse%
\ \isacommand{have}\isamarkupfalse%
\ {\isachardoublequoteopen}{\isasymdots}\ {\isacharequal}{\kern0pt}\ {\isadigit{0}}{\isachardoublequoteclose}\ \isacommand{using}\isamarkupfalse%
\ control{\isadigit{2}}{\isacharunderscore}{\kern0pt}def\ index{\isacharunderscore}{\kern0pt}mat{\isacharunderscore}{\kern0pt}of{\isacharunderscore}{\kern0pt}cols{\isacharunderscore}{\kern0pt}list\ \isacommand{by}\isamarkupfalse%
\ auto\isanewline
\ \ \ \ \ \ \ \ \ \ \ \ \ \ \isacommand{also}\isamarkupfalse%
\ \isacommand{have}\isamarkupfalse%
\ {\isachardoublequoteopen}{\isasymdots}\ {\isacharequal}{\kern0pt}\ {\isadigit{1}}\isactrlsub m\ {\isadigit{4}}\ {\isachardollar}{\kern0pt}{\isachardollar}{\kern0pt}\ {\isacharparenleft}{\kern0pt}{\isadigit{0}}{\isacharcomma}{\kern0pt}{\isadigit{3}}{\isacharparenright}{\kern0pt}{\isachardoublequoteclose}\ \isacommand{by}\isamarkupfalse%
\ simp\isanewline
\ \ \ \ \ \ \ \ \ \ \ \ \ \ \isacommand{finally}\isamarkupfalse%
\ \isacommand{show}\isamarkupfalse%
\ {\isacharquery}{\kern0pt}thesis\ \isacommand{using}\isamarkupfalse%
\ i{\isadigit{0}}\ j{\isadigit{3}}\ \isacommand{by}\isamarkupfalse%
\ simp\isanewline
\ \ \ \ \ \ \ \ \ \ \ \ \isacommand{qed}\isamarkupfalse%
\isanewline
\ \ \ \ \ \ \ \ \ \ \isacommand{qed}\isamarkupfalse%
\isanewline
\ \ \ \ \ \ \ \ \isacommand{qed}\isamarkupfalse%
\isanewline
\ \ \ \ \ \ \isacommand{qed}\isamarkupfalse%
\isanewline
\ \ \ \ \isacommand{next}\isamarkupfalse%
\isanewline
\ \ \ \ \ \ \isacommand{assume}\isamarkupfalse%
\ il{\isadigit{3}}{\isacharcolon}{\kern0pt}{\isachardoublequoteopen}i\ {\isacharequal}{\kern0pt}\ {\isadigit{1}}\ {\isasymor}\ i\ {\isacharequal}{\kern0pt}\ {\isadigit{2}}\ {\isasymor}\ i\ {\isacharequal}{\kern0pt}\ {\isadigit{3}}{\isachardoublequoteclose}\isanewline
\ \ \ \ \ \ \isacommand{show}\isamarkupfalse%
\ {\isachardoublequoteopen}{\isacharparenleft}{\kern0pt}control{\isadigit{2}}\ U\ {\isacharasterisk}{\kern0pt}\ {\isacharparenleft}{\kern0pt}{\isacharparenleft}{\kern0pt}control{\isadigit{2}}\ U{\isacharparenright}{\kern0pt}\isactrlsup {\isasymdagger}{\isacharparenright}{\kern0pt}{\isacharparenright}{\kern0pt}\ {\isachardollar}{\kern0pt}{\isachardollar}{\kern0pt}\ {\isacharparenleft}{\kern0pt}i{\isacharcomma}{\kern0pt}\ j{\isacharparenright}{\kern0pt}\ {\isacharequal}{\kern0pt}\ {\isadigit{1}}\isactrlsub m\ {\isadigit{4}}\ {\isachardollar}{\kern0pt}{\isachardollar}{\kern0pt}\ {\isacharparenleft}{\kern0pt}i{\isacharcomma}{\kern0pt}\ j{\isacharparenright}{\kern0pt}{\isachardoublequoteclose}\isanewline
\ \ \ \ \ \ \isacommand{proof}\isamarkupfalse%
\ {\isacharparenleft}{\kern0pt}rule\ disjE{\isacharparenright}{\kern0pt}\isanewline
\ \ \ \ \ \ \ \ \isacommand{show}\isamarkupfalse%
\ {\isachardoublequoteopen}i\ {\isacharequal}{\kern0pt}\ {\isadigit{1}}\ {\isasymor}\ i\ {\isacharequal}{\kern0pt}\ {\isadigit{2}}\ {\isasymor}\ i\ {\isacharequal}{\kern0pt}\ {\isadigit{3}}{\isachardoublequoteclose}\ \isacommand{using}\isamarkupfalse%
\ il{\isadigit{3}}\ \isacommand{by}\isamarkupfalse%
\ this\isanewline
\ \ \ \ \ \ \isacommand{next}\isamarkupfalse%
\isanewline
\ \ \ \ \ \ \ \ \isacommand{assume}\isamarkupfalse%
\ i{\isadigit{1}}{\isacharcolon}{\kern0pt}{\isachardoublequoteopen}i\ {\isacharequal}{\kern0pt}\ {\isadigit{1}}{\isachardoublequoteclose}\isanewline
\ \ \ \ \ \ \ \ \isacommand{show}\isamarkupfalse%
\ {\isachardoublequoteopen}{\isacharparenleft}{\kern0pt}control{\isadigit{2}}\ U\ {\isacharasterisk}{\kern0pt}\ {\isacharparenleft}{\kern0pt}{\isacharparenleft}{\kern0pt}control{\isadigit{2}}\ U{\isacharparenright}{\kern0pt}\isactrlsup {\isasymdagger}{\isacharparenright}{\kern0pt}{\isacharparenright}{\kern0pt}\ {\isachardollar}{\kern0pt}{\isachardollar}{\kern0pt}\ {\isacharparenleft}{\kern0pt}i{\isacharcomma}{\kern0pt}\ j{\isacharparenright}{\kern0pt}\ {\isacharequal}{\kern0pt}\ {\isadigit{1}}\isactrlsub m\ {\isadigit{4}}\ {\isachardollar}{\kern0pt}{\isachardollar}{\kern0pt}\ {\isacharparenleft}{\kern0pt}i{\isacharcomma}{\kern0pt}\ j{\isacharparenright}{\kern0pt}{\isachardoublequoteclose}\isanewline
\ \ \ \ \ \ \ \ \isacommand{proof}\isamarkupfalse%
\ {\isacharparenleft}{\kern0pt}rule\ disjE{\isacharparenright}{\kern0pt}\isanewline
\ \ \ \ \ \ \ \ \ \ \isacommand{show}\isamarkupfalse%
\ jl{\isadigit{4}}{\isacharcolon}{\kern0pt}{\isachardoublequoteopen}j\ {\isacharequal}{\kern0pt}\ {\isadigit{0}}\ {\isasymor}\ j\ {\isacharequal}{\kern0pt}\ {\isadigit{1}}\ {\isasymor}\ j\ {\isacharequal}{\kern0pt}\ {\isadigit{2}}\ {\isasymor}\ j\ {\isacharequal}{\kern0pt}\ {\isadigit{3}}{\isachardoublequoteclose}\ \isacommand{using}\isamarkupfalse%
\ j{\isadigit{4}}\ \isacommand{by}\isamarkupfalse%
\ auto\isanewline
\ \ \ \ \ \ \ \ \isacommand{next}\isamarkupfalse%
\isanewline
\ \ \ \ \ \ \ \ \ \ \isacommand{assume}\isamarkupfalse%
\ j{\isadigit{0}}{\isacharcolon}{\kern0pt}{\isachardoublequoteopen}j\ {\isacharequal}{\kern0pt}\ {\isadigit{0}}{\isachardoublequoteclose}\isanewline
\ \ \ \ \ \ \ \ \ \ \isacommand{show}\isamarkupfalse%
\ {\isachardoublequoteopen}{\isacharparenleft}{\kern0pt}control{\isadigit{2}}\ U\ {\isacharasterisk}{\kern0pt}\ {\isacharparenleft}{\kern0pt}{\isacharparenleft}{\kern0pt}control{\isadigit{2}}\ U{\isacharparenright}{\kern0pt}\isactrlsup {\isasymdagger}{\isacharparenright}{\kern0pt}{\isacharparenright}{\kern0pt}\ {\isachardollar}{\kern0pt}{\isachardollar}{\kern0pt}\ {\isacharparenleft}{\kern0pt}i{\isacharcomma}{\kern0pt}\ j{\isacharparenright}{\kern0pt}\ {\isacharequal}{\kern0pt}\ {\isadigit{1}}\isactrlsub m\ {\isadigit{4}}\ {\isachardollar}{\kern0pt}{\isachardollar}{\kern0pt}\ {\isacharparenleft}{\kern0pt}i{\isacharcomma}{\kern0pt}\ j{\isacharparenright}{\kern0pt}{\isachardoublequoteclose}\isanewline
\ \ \ \ \ \ \ \ \ \ \isacommand{proof}\isamarkupfalse%
\ {\isacharminus}{\kern0pt}\isanewline
\ \ \ \ \ \ \ \ \ \ \ \ \isacommand{have}\isamarkupfalse%
\ {\isachardoublequoteopen}{\isacharparenleft}{\kern0pt}control{\isadigit{2}}\ U\ {\isacharasterisk}{\kern0pt}\ {\isacharparenleft}{\kern0pt}{\isacharparenleft}{\kern0pt}control{\isadigit{2}}\ U{\isacharparenright}{\kern0pt}\isactrlsup {\isasymdagger}{\isacharparenright}{\kern0pt}{\isacharparenright}{\kern0pt}\ {\isachardollar}{\kern0pt}{\isachardollar}{\kern0pt}\ {\isacharparenleft}{\kern0pt}{\isadigit{1}}{\isacharcomma}{\kern0pt}{\isadigit{0}}{\isacharparenright}{\kern0pt}\ {\isacharequal}{\kern0pt}\ \isanewline
\ \ \ \ \ \ \ \ \ \ \ \ \ \ \ \ \ \ \ \ {\isacharparenleft}{\kern0pt}control{\isadigit{2}}\ U{\isacharparenright}{\kern0pt}\ {\isachardollar}{\kern0pt}{\isachardollar}{\kern0pt}\ {\isacharparenleft}{\kern0pt}{\isadigit{1}}{\isacharcomma}{\kern0pt}{\isadigit{0}}{\isacharparenright}{\kern0pt}\ {\isacharasterisk}{\kern0pt}\ {\isacharparenleft}{\kern0pt}{\isacharparenleft}{\kern0pt}control{\isadigit{2}}\ U{\isacharparenright}{\kern0pt}\isactrlsup {\isasymdagger}{\isacharparenright}{\kern0pt}\ {\isachardollar}{\kern0pt}{\isachardollar}{\kern0pt}\ {\isacharparenleft}{\kern0pt}{\isadigit{0}}{\isacharcomma}{\kern0pt}{\isadigit{0}}{\isacharparenright}{\kern0pt}\ {\isacharplus}{\kern0pt}\isanewline
\ \ \ \ \ \ \ \ \ \ \ \ \ \ \ \ \ \ \ \ {\isacharparenleft}{\kern0pt}control{\isadigit{2}}\ U{\isacharparenright}{\kern0pt}\ {\isachardollar}{\kern0pt}{\isachardollar}{\kern0pt}\ {\isacharparenleft}{\kern0pt}{\isadigit{1}}{\isacharcomma}{\kern0pt}{\isadigit{1}}{\isacharparenright}{\kern0pt}\ {\isacharasterisk}{\kern0pt}\ {\isacharparenleft}{\kern0pt}{\isacharparenleft}{\kern0pt}control{\isadigit{2}}\ U{\isacharparenright}{\kern0pt}\isactrlsup {\isasymdagger}{\isacharparenright}{\kern0pt}\ {\isachardollar}{\kern0pt}{\isachardollar}{\kern0pt}\ {\isacharparenleft}{\kern0pt}{\isadigit{1}}{\isacharcomma}{\kern0pt}{\isadigit{0}}{\isacharparenright}{\kern0pt}\ {\isacharplus}{\kern0pt}\isanewline
\ \ \ \ \ \ \ \ \ \ \ \ \ \ \ \ \ \ \ \ {\isacharparenleft}{\kern0pt}control{\isadigit{2}}\ U{\isacharparenright}{\kern0pt}\ {\isachardollar}{\kern0pt}{\isachardollar}{\kern0pt}\ {\isacharparenleft}{\kern0pt}{\isadigit{1}}{\isacharcomma}{\kern0pt}{\isadigit{2}}{\isacharparenright}{\kern0pt}\ {\isacharasterisk}{\kern0pt}\ {\isacharparenleft}{\kern0pt}{\isacharparenleft}{\kern0pt}control{\isadigit{2}}\ U{\isacharparenright}{\kern0pt}\isactrlsup {\isasymdagger}{\isacharparenright}{\kern0pt}\ {\isachardollar}{\kern0pt}{\isachardollar}{\kern0pt}\ {\isacharparenleft}{\kern0pt}{\isadigit{2}}{\isacharcomma}{\kern0pt}{\isadigit{0}}{\isacharparenright}{\kern0pt}\ {\isacharplus}{\kern0pt}\isanewline
\ \ \ \ \ \ \ \ \ \ \ \ \ \ \ \ \ \ \ \ {\isacharparenleft}{\kern0pt}control{\isadigit{2}}\ U{\isacharparenright}{\kern0pt}\ {\isachardollar}{\kern0pt}{\isachardollar}{\kern0pt}\ {\isacharparenleft}{\kern0pt}{\isadigit{1}}{\isacharcomma}{\kern0pt}{\isadigit{3}}{\isacharparenright}{\kern0pt}\ {\isacharasterisk}{\kern0pt}\ {\isacharparenleft}{\kern0pt}{\isacharparenleft}{\kern0pt}control{\isadigit{2}}\ U{\isacharparenright}{\kern0pt}\isactrlsup {\isasymdagger}{\isacharparenright}{\kern0pt}\ {\isachardollar}{\kern0pt}{\isachardollar}{\kern0pt}\ {\isacharparenleft}{\kern0pt}{\isadigit{3}}{\isacharcomma}{\kern0pt}{\isadigit{0}}{\isacharparenright}{\kern0pt}{\isachardoublequoteclose}\isanewline
\ \ \ \ \ \ \ \ \ \ \ \ \ \ \isacommand{using}\isamarkupfalse%
\ times{\isacharunderscore}{\kern0pt}mat{\isacharunderscore}{\kern0pt}def\ sumof{\isadigit{4}}\isanewline
\ \ \ \ \ \ \ \ \ \ \ \ \ \ \ \ \isacommand{by}\isamarkupfalse%
\ {\isacharparenleft}{\kern0pt}smt\ {\isacharparenleft}{\kern0pt}z{\isadigit{3}}{\isacharparenright}{\kern0pt}\ carrier{\isacharunderscore}{\kern0pt}matD{\isacharparenleft}{\kern0pt}{\isadigit{1}}{\isacharparenright}{\kern0pt}\ carrier{\isacharunderscore}{\kern0pt}matD{\isacharparenleft}{\kern0pt}{\isadigit{2}}{\isacharparenright}{\kern0pt}\ control{\isadigit{2}}{\isacharunderscore}{\kern0pt}carrier{\isacharunderscore}{\kern0pt}mat\ dim{\isacharunderscore}{\kern0pt}col{\isacharunderscore}{\kern0pt}of{\isacharunderscore}{\kern0pt}dagger\ \isanewline
\ \ \ \ \ \ \ \ \ \ \ \ \ \ \ \ \ \ \ \ \ \ dim{\isacharunderscore}{\kern0pt}row{\isacharunderscore}{\kern0pt}of{\isacharunderscore}{\kern0pt}dagger\ i{\isadigit{1}}\ i{\isadigit{4}}\ index{\isacharunderscore}{\kern0pt}matrix{\isacharunderscore}{\kern0pt}prod\ j{\isadigit{0}}\ j{\isadigit{4}}{\isacharparenright}{\kern0pt}\isanewline
\ \ \ \ \ \ \ \ \ \ \ \ \isacommand{also}\isamarkupfalse%
\ \isacommand{have}\isamarkupfalse%
\ {\isachardoublequoteopen}{\isasymdots}\ {\isacharequal}{\kern0pt}\ {\isacharparenleft}{\kern0pt}control{\isadigit{2}}\ U{\isacharparenright}{\kern0pt}\ {\isachardollar}{\kern0pt}{\isachardollar}{\kern0pt}\ {\isacharparenleft}{\kern0pt}{\isadigit{1}}{\isacharcomma}{\kern0pt}{\isadigit{1}}{\isacharparenright}{\kern0pt}\ {\isacharasterisk}{\kern0pt}\ {\isacharparenleft}{\kern0pt}{\isacharparenleft}{\kern0pt}control{\isadigit{2}}\ U{\isacharparenright}{\kern0pt}\isactrlsup {\isasymdagger}{\isacharparenright}{\kern0pt}\ {\isachardollar}{\kern0pt}{\isachardollar}{\kern0pt}\ {\isacharparenleft}{\kern0pt}{\isadigit{1}}{\isacharcomma}{\kern0pt}{\isadigit{0}}{\isacharparenright}{\kern0pt}\ {\isacharplus}{\kern0pt}\ \isanewline
\ \ \ \ \ \ \ \ \ \ \ \ \ \ \ \ \ \ \ \ \ \ \ \ \ \ \ \ \ \ {\isacharparenleft}{\kern0pt}control{\isadigit{2}}\ U{\isacharparenright}{\kern0pt}\ {\isachardollar}{\kern0pt}{\isachardollar}{\kern0pt}\ {\isacharparenleft}{\kern0pt}{\isadigit{1}}{\isacharcomma}{\kern0pt}{\isadigit{3}}{\isacharparenright}{\kern0pt}\ {\isacharasterisk}{\kern0pt}\ {\isacharparenleft}{\kern0pt}{\isacharparenleft}{\kern0pt}control{\isadigit{2}}\ U{\isacharparenright}{\kern0pt}\isactrlsup {\isasymdagger}{\isacharparenright}{\kern0pt}\ {\isachardollar}{\kern0pt}{\isachardollar}{\kern0pt}\ {\isacharparenleft}{\kern0pt}{\isadigit{3}}{\isacharcomma}{\kern0pt}{\isadigit{0}}{\isacharparenright}{\kern0pt}{\isachardoublequoteclose}\isanewline
\ \ \ \ \ \ \ \ \ \ \ \ \ \ \ \ \isacommand{using}\isamarkupfalse%
\ control{\isadigit{2}}{\isacharunderscore}{\kern0pt}def\ index{\isacharunderscore}{\kern0pt}mat{\isacharunderscore}{\kern0pt}of{\isacharunderscore}{\kern0pt}cols{\isacharunderscore}{\kern0pt}list\ \isacommand{by}\isamarkupfalse%
\ force\isanewline
\ \ \ \ \ \ \ \ \ \ \ \ \isacommand{also}\isamarkupfalse%
\ \isacommand{have}\isamarkupfalse%
\ {\isachardoublequoteopen}{\isasymdots}\ {\isacharequal}{\kern0pt}\ {\isacharparenleft}{\kern0pt}control{\isadigit{2}}\ U{\isacharparenright}{\kern0pt}\ {\isachardollar}{\kern0pt}{\isachardollar}{\kern0pt}\ {\isacharparenleft}{\kern0pt}{\isadigit{1}}{\isacharcomma}{\kern0pt}{\isadigit{1}}{\isacharparenright}{\kern0pt}\ {\isacharasterisk}{\kern0pt}\ {\isacharparenleft}{\kern0pt}cnj\ {\isacharparenleft}{\kern0pt}{\isacharparenleft}{\kern0pt}control{\isadigit{2}}\ U{\isacharparenright}{\kern0pt}\ {\isachardollar}{\kern0pt}{\isachardollar}{\kern0pt}\ {\isacharparenleft}{\kern0pt}{\isadigit{0}}{\isacharcomma}{\kern0pt}{\isadigit{1}}{\isacharparenright}{\kern0pt}{\isacharparenright}{\kern0pt}{\isacharparenright}{\kern0pt}\ {\isacharplus}{\kern0pt}\ \isanewline
\ \ \ \ \ \ \ \ \ \ \ \ \ \ \ \ \ \ \ \ \ \ \ \ \ \ \ \ \ \ {\isacharparenleft}{\kern0pt}control{\isadigit{2}}\ U{\isacharparenright}{\kern0pt}\ {\isachardollar}{\kern0pt}{\isachardollar}{\kern0pt}\ {\isacharparenleft}{\kern0pt}{\isadigit{1}}{\isacharcomma}{\kern0pt}{\isadigit{3}}{\isacharparenright}{\kern0pt}\ {\isacharasterisk}{\kern0pt}\ {\isacharparenleft}{\kern0pt}cnj\ {\isacharparenleft}{\kern0pt}{\isacharparenleft}{\kern0pt}control{\isadigit{2}}\ U{\isacharparenright}{\kern0pt}\ {\isachardollar}{\kern0pt}{\isachardollar}{\kern0pt}\ {\isacharparenleft}{\kern0pt}{\isadigit{0}}{\isacharcomma}{\kern0pt}{\isadigit{3}}{\isacharparenright}{\kern0pt}{\isacharparenright}{\kern0pt}{\isacharparenright}{\kern0pt}{\isachardoublequoteclose}\isanewline
\ \ \ \ \ \ \ \ \ \ \ \ \ \ \ \ \isacommand{using}\isamarkupfalse%
\ dagger{\isacharunderscore}{\kern0pt}def\isanewline
\ \ \ \ \ \ \ \ \ \ \ \ \ \ \ \ \isacommand{by}\isamarkupfalse%
\ {\isacharparenleft}{\kern0pt}smt\ {\isacharparenleft}{\kern0pt}verit{\isacharcomma}{\kern0pt}\ ccfv{\isacharunderscore}{\kern0pt}threshold{\isacharparenright}{\kern0pt}\ One{\isacharunderscore}{\kern0pt}nat{\isacharunderscore}{\kern0pt}def\ Suc{\isacharunderscore}{\kern0pt}{\isadigit{1}}\ add{\isachardot}{\kern0pt}commute\ add{\isacharunderscore}{\kern0pt}Suc{\isacharunderscore}{\kern0pt}right\ \isanewline
\ \ \ \ \ \ \ \ \ \ \ \ \ \ \ \ \ \ \ \ carrier{\isacharunderscore}{\kern0pt}matD{\isacharparenleft}{\kern0pt}{\isadigit{1}}{\isacharparenright}{\kern0pt}\ carrier{\isacharunderscore}{\kern0pt}matD{\isacharparenleft}{\kern0pt}{\isadigit{2}}{\isacharparenright}{\kern0pt}\ control{\isadigit{2}}{\isacharunderscore}{\kern0pt}carrier{\isacharunderscore}{\kern0pt}mat\ i{\isadigit{1}}\ i{\isadigit{4}}\ index{\isacharunderscore}{\kern0pt}mat{\isacharparenleft}{\kern0pt}{\isadigit{1}}{\isacharparenright}{\kern0pt}\ j{\isadigit{0}}\ j{\isadigit{4}}\ \isanewline
\ \ \ \ \ \ \ \ \ \ \ \ \ \ \ \ \ \ \ \ lessI\ numeral{\isacharunderscore}{\kern0pt}{\isadigit{3}}{\isacharunderscore}{\kern0pt}eq{\isacharunderscore}{\kern0pt}{\isadigit{3}}\ numeral{\isacharunderscore}{\kern0pt}Bit{\isadigit{0}}\ plus{\isacharunderscore}{\kern0pt}{\isadigit{1}}{\isacharunderscore}{\kern0pt}eq{\isacharunderscore}{\kern0pt}Suc\ prod{\isachardot}{\kern0pt}simps{\isacharparenleft}{\kern0pt}{\isadigit{2}}{\isacharparenright}{\kern0pt}{\isacharparenright}{\kern0pt}\isanewline
\ \ \ \ \ \ \ \ \ \ \ \ \isacommand{also}\isamarkupfalse%
\ \isacommand{have}\isamarkupfalse%
\ {\isachardoublequoteopen}{\isasymdots}\ {\isacharequal}{\kern0pt}\ {\isacharparenleft}{\kern0pt}control{\isadigit{2}}\ U{\isacharparenright}{\kern0pt}\ {\isachardollar}{\kern0pt}{\isachardollar}{\kern0pt}\ {\isacharparenleft}{\kern0pt}{\isadigit{1}}{\isacharcomma}{\kern0pt}{\isadigit{1}}{\isacharparenright}{\kern0pt}\ {\isacharasterisk}{\kern0pt}\ {\isacharparenleft}{\kern0pt}cnj\ {\isadigit{0}}{\isacharparenright}{\kern0pt}\ {\isacharplus}{\kern0pt}\isanewline
\ \ \ \ \ \ \ \ \ \ \ \ \ \ \ \ \ \ \ \ \ \ \ \ \ \ \ \ \ \ {\isacharparenleft}{\kern0pt}control{\isadigit{2}}\ U{\isacharparenright}{\kern0pt}\ {\isachardollar}{\kern0pt}{\isachardollar}{\kern0pt}\ {\isacharparenleft}{\kern0pt}{\isadigit{1}}{\isacharcomma}{\kern0pt}{\isadigit{3}}{\isacharparenright}{\kern0pt}\ {\isacharasterisk}{\kern0pt}\ {\isacharparenleft}{\kern0pt}cnj\ {\isadigit{0}}{\isacharparenright}{\kern0pt}{\isachardoublequoteclose}\isanewline
\ \ \ \ \ \ \ \ \ \ \ \ \ \ \ \ \isacommand{using}\isamarkupfalse%
\ control{\isadigit{2}}{\isacharunderscore}{\kern0pt}def\ index{\isacharunderscore}{\kern0pt}mat{\isacharunderscore}{\kern0pt}of{\isacharunderscore}{\kern0pt}cols{\isacharunderscore}{\kern0pt}list\ \isacommand{by}\isamarkupfalse%
\ simp\isanewline
\ \ \ \ \ \ \ \ \ \ \ \ \isacommand{also}\isamarkupfalse%
\ \isacommand{have}\isamarkupfalse%
\ {\isachardoublequoteopen}{\isasymdots}\ {\isacharequal}{\kern0pt}\ {\isadigit{0}}{\isachardoublequoteclose}\ \isacommand{by}\isamarkupfalse%
\ auto\isanewline
\ \ \ \ \ \ \ \ \ \ \ \ \isacommand{also}\isamarkupfalse%
\ \isacommand{have}\isamarkupfalse%
\ {\isachardoublequoteopen}{\isasymdots}\ {\isacharequal}{\kern0pt}\ {\isadigit{1}}\isactrlsub m\ {\isadigit{4}}\ {\isachardollar}{\kern0pt}{\isachardollar}{\kern0pt}\ {\isacharparenleft}{\kern0pt}{\isadigit{1}}{\isacharcomma}{\kern0pt}{\isadigit{0}}{\isacharparenright}{\kern0pt}{\isachardoublequoteclose}\ \isacommand{by}\isamarkupfalse%
\ simp\isanewline
\ \ \ \ \ \ \ \ \ \ \ \ \isacommand{finally}\isamarkupfalse%
\ \isacommand{show}\isamarkupfalse%
\ {\isacharquery}{\kern0pt}thesis\ \isacommand{using}\isamarkupfalse%
\ i{\isadigit{1}}\ j{\isadigit{0}}\ \isacommand{by}\isamarkupfalse%
\ simp\isanewline
\ \ \ \ \ \ \ \ \ \ \isacommand{qed}\isamarkupfalse%
\isanewline
\ \ \ \ \ \ \ \ \isacommand{next}\isamarkupfalse%
\isanewline
\ \ \ \ \ \ \ \ \ \ \isacommand{assume}\isamarkupfalse%
\ jl{\isadigit{3}}{\isacharcolon}{\kern0pt}{\isachardoublequoteopen}j\ {\isacharequal}{\kern0pt}\ {\isadigit{1}}\ {\isasymor}\ j\ {\isacharequal}{\kern0pt}\ {\isadigit{2}}\ {\isasymor}\ j\ {\isacharequal}{\kern0pt}\ {\isadigit{3}}{\isachardoublequoteclose}\isanewline
\ \ \ \ \ \ \ \ \ \ \isacommand{show}\isamarkupfalse%
\ {\isachardoublequoteopen}{\isacharparenleft}{\kern0pt}control{\isadigit{2}}\ U\ {\isacharasterisk}{\kern0pt}\ {\isacharparenleft}{\kern0pt}{\isacharparenleft}{\kern0pt}control{\isadigit{2}}\ U{\isacharparenright}{\kern0pt}\isactrlsup {\isasymdagger}{\isacharparenright}{\kern0pt}{\isacharparenright}{\kern0pt}\ {\isachardollar}{\kern0pt}{\isachardollar}{\kern0pt}\ {\isacharparenleft}{\kern0pt}i{\isacharcomma}{\kern0pt}\ j{\isacharparenright}{\kern0pt}\ {\isacharequal}{\kern0pt}\ {\isadigit{1}}\isactrlsub m\ {\isadigit{4}}\ {\isachardollar}{\kern0pt}{\isachardollar}{\kern0pt}\ {\isacharparenleft}{\kern0pt}i{\isacharcomma}{\kern0pt}\ j{\isacharparenright}{\kern0pt}{\isachardoublequoteclose}\isanewline
\ \ \ \ \ \ \ \ \ \ \isacommand{proof}\isamarkupfalse%
\ {\isacharparenleft}{\kern0pt}rule\ disjE{\isacharparenright}{\kern0pt}\isanewline
\ \ \ \ \ \ \ \ \ \ \ \ \isacommand{show}\isamarkupfalse%
\ {\isachardoublequoteopen}j\ {\isacharequal}{\kern0pt}\ {\isadigit{1}}\ {\isasymor}\ j\ {\isacharequal}{\kern0pt}\ {\isadigit{2}}\ {\isasymor}\ j\ {\isacharequal}{\kern0pt}\ {\isadigit{3}}{\isachardoublequoteclose}\ \isacommand{using}\isamarkupfalse%
\ jl{\isadigit{3}}\ \isacommand{by}\isamarkupfalse%
\ this\isanewline
\ \ \ \ \ \ \ \ \ \ \isacommand{next}\isamarkupfalse%
\isanewline
\ \ \ \ \ \ \ \ \ \ \ \ \isacommand{assume}\isamarkupfalse%
\ j{\isadigit{1}}{\isacharcolon}{\kern0pt}{\isachardoublequoteopen}j\ {\isacharequal}{\kern0pt}\ {\isadigit{1}}{\isachardoublequoteclose}\isanewline
\ \ \ \ \ \ \ \ \ \ \ \ \isacommand{show}\isamarkupfalse%
\ {\isachardoublequoteopen}{\isacharparenleft}{\kern0pt}control{\isadigit{2}}\ U\ {\isacharasterisk}{\kern0pt}\ {\isacharparenleft}{\kern0pt}{\isacharparenleft}{\kern0pt}control{\isadigit{2}}\ U{\isacharparenright}{\kern0pt}\isactrlsup {\isasymdagger}{\isacharparenright}{\kern0pt}{\isacharparenright}{\kern0pt}\ {\isachardollar}{\kern0pt}{\isachardollar}{\kern0pt}\ {\isacharparenleft}{\kern0pt}i{\isacharcomma}{\kern0pt}\ j{\isacharparenright}{\kern0pt}\ {\isacharequal}{\kern0pt}\ {\isadigit{1}}\isactrlsub m\ {\isadigit{4}}\ {\isachardollar}{\kern0pt}{\isachardollar}{\kern0pt}\ {\isacharparenleft}{\kern0pt}i{\isacharcomma}{\kern0pt}\ j{\isacharparenright}{\kern0pt}{\isachardoublequoteclose}\isanewline
\ \ \ \ \ \ \ \ \ \ \ \ \isacommand{proof}\isamarkupfalse%
\ {\isacharminus}{\kern0pt}\isanewline
\ \ \ \ \ \ \ \ \ \ \ \ \ \ \isacommand{have}\isamarkupfalse%
\ {\isachardoublequoteopen}{\isacharparenleft}{\kern0pt}control{\isadigit{2}}\ U\ {\isacharasterisk}{\kern0pt}\ {\isacharparenleft}{\kern0pt}{\isacharparenleft}{\kern0pt}control{\isadigit{2}}\ U{\isacharparenright}{\kern0pt}\isactrlsup {\isasymdagger}{\isacharparenright}{\kern0pt}{\isacharparenright}{\kern0pt}\ {\isachardollar}{\kern0pt}{\isachardollar}{\kern0pt}\ {\isacharparenleft}{\kern0pt}{\isadigit{1}}{\isacharcomma}{\kern0pt}{\isadigit{1}}{\isacharparenright}{\kern0pt}\ {\isacharequal}{\kern0pt}\ \isanewline
\ \ \ \ \ \ \ \ \ \ \ \ \ \ \ \ \ \ \ \ {\isacharparenleft}{\kern0pt}control{\isadigit{2}}\ U{\isacharparenright}{\kern0pt}\ {\isachardollar}{\kern0pt}{\isachardollar}{\kern0pt}\ {\isacharparenleft}{\kern0pt}{\isadigit{1}}{\isacharcomma}{\kern0pt}{\isadigit{0}}{\isacharparenright}{\kern0pt}\ {\isacharasterisk}{\kern0pt}\ {\isacharparenleft}{\kern0pt}{\isacharparenleft}{\kern0pt}control{\isadigit{2}}\ U{\isacharparenright}{\kern0pt}\isactrlsup {\isasymdagger}{\isacharparenright}{\kern0pt}\ {\isachardollar}{\kern0pt}{\isachardollar}{\kern0pt}\ {\isacharparenleft}{\kern0pt}{\isadigit{0}}{\isacharcomma}{\kern0pt}{\isadigit{1}}{\isacharparenright}{\kern0pt}\ {\isacharplus}{\kern0pt}\isanewline
\ \ \ \ \ \ \ \ \ \ \ \ \ \ \ \ \ \ \ \ {\isacharparenleft}{\kern0pt}control{\isadigit{2}}\ U{\isacharparenright}{\kern0pt}\ {\isachardollar}{\kern0pt}{\isachardollar}{\kern0pt}\ {\isacharparenleft}{\kern0pt}{\isadigit{1}}{\isacharcomma}{\kern0pt}{\isadigit{1}}{\isacharparenright}{\kern0pt}\ {\isacharasterisk}{\kern0pt}\ {\isacharparenleft}{\kern0pt}{\isacharparenleft}{\kern0pt}control{\isadigit{2}}\ U{\isacharparenright}{\kern0pt}\isactrlsup {\isasymdagger}{\isacharparenright}{\kern0pt}\ {\isachardollar}{\kern0pt}{\isachardollar}{\kern0pt}\ {\isacharparenleft}{\kern0pt}{\isadigit{1}}{\isacharcomma}{\kern0pt}{\isadigit{1}}{\isacharparenright}{\kern0pt}\ {\isacharplus}{\kern0pt}\isanewline
\ \ \ \ \ \ \ \ \ \ \ \ \ \ \ \ \ \ \ \ {\isacharparenleft}{\kern0pt}control{\isadigit{2}}\ U{\isacharparenright}{\kern0pt}\ {\isachardollar}{\kern0pt}{\isachardollar}{\kern0pt}\ {\isacharparenleft}{\kern0pt}{\isadigit{1}}{\isacharcomma}{\kern0pt}{\isadigit{2}}{\isacharparenright}{\kern0pt}\ {\isacharasterisk}{\kern0pt}\ {\isacharparenleft}{\kern0pt}{\isacharparenleft}{\kern0pt}control{\isadigit{2}}\ U{\isacharparenright}{\kern0pt}\isactrlsup {\isasymdagger}{\isacharparenright}{\kern0pt}\ {\isachardollar}{\kern0pt}{\isachardollar}{\kern0pt}\ {\isacharparenleft}{\kern0pt}{\isadigit{2}}{\isacharcomma}{\kern0pt}{\isadigit{1}}{\isacharparenright}{\kern0pt}\ {\isacharplus}{\kern0pt}\isanewline
\ \ \ \ \ \ \ \ \ \ \ \ \ \ \ \ \ \ \ \ {\isacharparenleft}{\kern0pt}control{\isadigit{2}}\ U{\isacharparenright}{\kern0pt}\ {\isachardollar}{\kern0pt}{\isachardollar}{\kern0pt}\ {\isacharparenleft}{\kern0pt}{\isadigit{1}}{\isacharcomma}{\kern0pt}{\isadigit{3}}{\isacharparenright}{\kern0pt}\ {\isacharasterisk}{\kern0pt}\ {\isacharparenleft}{\kern0pt}{\isacharparenleft}{\kern0pt}control{\isadigit{2}}\ U{\isacharparenright}{\kern0pt}\isactrlsup {\isasymdagger}{\isacharparenright}{\kern0pt}\ {\isachardollar}{\kern0pt}{\isachardollar}{\kern0pt}\ {\isacharparenleft}{\kern0pt}{\isadigit{3}}{\isacharcomma}{\kern0pt}{\isadigit{1}}{\isacharparenright}{\kern0pt}{\isachardoublequoteclose}\isanewline
\ \ \ \ \ \ \ \ \ \ \ \ \ \ \ \ \isacommand{using}\isamarkupfalse%
\ times{\isacharunderscore}{\kern0pt}mat{\isacharunderscore}{\kern0pt}def\ sumof{\isadigit{4}}\isanewline
\ \ \ \ \ \ \ \ \ \ \ \ \ \ \ \ \isacommand{by}\isamarkupfalse%
\ {\isacharparenleft}{\kern0pt}smt\ {\isacharparenleft}{\kern0pt}z{\isadigit{3}}{\isacharparenright}{\kern0pt}\ carrier{\isacharunderscore}{\kern0pt}matD{\isacharparenleft}{\kern0pt}{\isadigit{1}}{\isacharparenright}{\kern0pt}\ carrier{\isacharunderscore}{\kern0pt}matD{\isacharparenleft}{\kern0pt}{\isadigit{2}}{\isacharparenright}{\kern0pt}\ control{\isadigit{2}}{\isacharunderscore}{\kern0pt}carrier{\isacharunderscore}{\kern0pt}mat\ dim{\isacharunderscore}{\kern0pt}col{\isacharunderscore}{\kern0pt}of{\isacharunderscore}{\kern0pt}dagger\ \isanewline
\ \ \ \ \ \ \ \ \ \ \ \ \ \ \ \ \ \ \ \ \ \ \ \ dim{\isacharunderscore}{\kern0pt}row{\isacharunderscore}{\kern0pt}of{\isacharunderscore}{\kern0pt}dagger\ i{\isadigit{1}}\ i{\isadigit{4}}\ index{\isacharunderscore}{\kern0pt}matrix{\isacharunderscore}{\kern0pt}prod\ j{\isadigit{1}}\ j{\isadigit{4}}{\isacharparenright}{\kern0pt}\isanewline
\ \ \ \ \ \ \ \ \ \ \ \ \ \ \isacommand{also}\isamarkupfalse%
\ \isacommand{have}\isamarkupfalse%
\ {\isachardoublequoteopen}{\isasymdots}\ {\isacharequal}{\kern0pt}\ {\isacharparenleft}{\kern0pt}control{\isadigit{2}}\ U{\isacharparenright}{\kern0pt}\ {\isachardollar}{\kern0pt}{\isachardollar}{\kern0pt}\ {\isacharparenleft}{\kern0pt}{\isadigit{1}}{\isacharcomma}{\kern0pt}{\isadigit{1}}{\isacharparenright}{\kern0pt}\ {\isacharasterisk}{\kern0pt}\ {\isacharparenleft}{\kern0pt}{\isacharparenleft}{\kern0pt}control{\isadigit{2}}\ U{\isacharparenright}{\kern0pt}\isactrlsup {\isasymdagger}{\isacharparenright}{\kern0pt}\ {\isachardollar}{\kern0pt}{\isachardollar}{\kern0pt}\ {\isacharparenleft}{\kern0pt}{\isadigit{1}}{\isacharcomma}{\kern0pt}{\isadigit{1}}{\isacharparenright}{\kern0pt}\ {\isacharplus}{\kern0pt}\ \isanewline
\ \ \ \ \ \ \ \ \ \ \ \ \ \ \ \ \ \ \ \ \ \ \ \ \ \ \ \ \ \ \ \ {\isacharparenleft}{\kern0pt}control{\isadigit{2}}\ U{\isacharparenright}{\kern0pt}\ {\isachardollar}{\kern0pt}{\isachardollar}{\kern0pt}\ {\isacharparenleft}{\kern0pt}{\isadigit{1}}{\isacharcomma}{\kern0pt}{\isadigit{3}}{\isacharparenright}{\kern0pt}\ {\isacharasterisk}{\kern0pt}\ {\isacharparenleft}{\kern0pt}{\isacharparenleft}{\kern0pt}control{\isadigit{2}}\ U{\isacharparenright}{\kern0pt}\isactrlsup {\isasymdagger}{\isacharparenright}{\kern0pt}\ {\isachardollar}{\kern0pt}{\isachardollar}{\kern0pt}\ {\isacharparenleft}{\kern0pt}{\isadigit{3}}{\isacharcomma}{\kern0pt}{\isadigit{1}}{\isacharparenright}{\kern0pt}{\isachardoublequoteclose}\isanewline
\ \ \ \ \ \ \ \ \ \ \ \ \ \ \ \ \ \ \isacommand{using}\isamarkupfalse%
\ control{\isadigit{2}}{\isacharunderscore}{\kern0pt}def\ index{\isacharunderscore}{\kern0pt}mat{\isacharunderscore}{\kern0pt}of{\isacharunderscore}{\kern0pt}cols{\isacharunderscore}{\kern0pt}list\ \isacommand{by}\isamarkupfalse%
\ force\isanewline
\ \ \ \ \ \ \ \ \ \ \ \ \ \ \isacommand{also}\isamarkupfalse%
\ \isacommand{have}\isamarkupfalse%
\ {\isachardoublequoteopen}{\isasymdots}\ {\isacharequal}{\kern0pt}\ {\isacharparenleft}{\kern0pt}control{\isadigit{2}}\ U{\isacharparenright}{\kern0pt}\ {\isachardollar}{\kern0pt}{\isachardollar}{\kern0pt}\ {\isacharparenleft}{\kern0pt}{\isadigit{1}}{\isacharcomma}{\kern0pt}{\isadigit{1}}{\isacharparenright}{\kern0pt}\ {\isacharasterisk}{\kern0pt}\ {\isacharparenleft}{\kern0pt}cnj\ {\isacharparenleft}{\kern0pt}{\isacharparenleft}{\kern0pt}control{\isadigit{2}}\ U{\isacharparenright}{\kern0pt}\ {\isachardollar}{\kern0pt}{\isachardollar}{\kern0pt}\ {\isacharparenleft}{\kern0pt}{\isadigit{1}}{\isacharcomma}{\kern0pt}{\isadigit{1}}{\isacharparenright}{\kern0pt}{\isacharparenright}{\kern0pt}{\isacharparenright}{\kern0pt}\ {\isacharplus}{\kern0pt}\ \isanewline
\ \ \ \ \ \ \ \ \ \ \ \ \ \ \ \ \ \ \ \ \ \ \ \ \ \ \ \ \ \ \ \ {\isacharparenleft}{\kern0pt}control{\isadigit{2}}\ U{\isacharparenright}{\kern0pt}\ {\isachardollar}{\kern0pt}{\isachardollar}{\kern0pt}\ {\isacharparenleft}{\kern0pt}{\isadigit{1}}{\isacharcomma}{\kern0pt}{\isadigit{3}}{\isacharparenright}{\kern0pt}\ {\isacharasterisk}{\kern0pt}\ {\isacharparenleft}{\kern0pt}cnj\ {\isacharparenleft}{\kern0pt}{\isacharparenleft}{\kern0pt}control{\isadigit{2}}\ U{\isacharparenright}{\kern0pt}\ {\isachardollar}{\kern0pt}{\isachardollar}{\kern0pt}\ {\isacharparenleft}{\kern0pt}{\isadigit{1}}{\isacharcomma}{\kern0pt}{\isadigit{3}}{\isacharparenright}{\kern0pt}{\isacharparenright}{\kern0pt}{\isacharparenright}{\kern0pt}{\isachardoublequoteclose}\isanewline
\ \ \ \ \ \ \ \ \ \ \ \ \ \ \ \ \ \ \isacommand{using}\isamarkupfalse%
\ dagger{\isacharunderscore}{\kern0pt}def\isanewline
\ \ \ \ \ \ \ \ \ \ \ \ \ \ \ \ \ \ \isacommand{by}\isamarkupfalse%
\ {\isacharparenleft}{\kern0pt}smt\ {\isacharparenleft}{\kern0pt}verit{\isacharcomma}{\kern0pt}\ best{\isacharparenright}{\kern0pt}\ One{\isacharunderscore}{\kern0pt}nat{\isacharunderscore}{\kern0pt}def\ Suc{\isacharunderscore}{\kern0pt}{\isadigit{1}}\ add{\isachardot}{\kern0pt}commute\ add{\isacharunderscore}{\kern0pt}Suc{\isacharunderscore}{\kern0pt}right\ carrier{\isacharunderscore}{\kern0pt}matD{\isacharparenleft}{\kern0pt}{\isadigit{1}}{\isacharparenright}{\kern0pt}\isanewline
\ \ \ \ \ \ \ \ \ \ \ \ \ \ \ \ \ \ \ \ \ \ carrier{\isacharunderscore}{\kern0pt}matD{\isacharparenleft}{\kern0pt}{\isadigit{2}}{\isacharparenright}{\kern0pt}\ control{\isadigit{2}}{\isacharunderscore}{\kern0pt}carrier{\isacharunderscore}{\kern0pt}mat\ i{\isadigit{1}}\ i{\isadigit{4}}\ index{\isacharunderscore}{\kern0pt}mat{\isacharparenleft}{\kern0pt}{\isadigit{1}}{\isacharparenright}{\kern0pt}\ lessI\ numeral{\isacharunderscore}{\kern0pt}{\isadigit{3}}{\isacharunderscore}{\kern0pt}eq{\isacharunderscore}{\kern0pt}{\isadigit{3}}\ \isanewline
\ \ \ \ \ \ \ \ \ \ \ \ \ \ \ \ \ \ \ \ \ \ numeral{\isacharunderscore}{\kern0pt}Bit{\isadigit{0}}\ plus{\isacharunderscore}{\kern0pt}{\isadigit{1}}{\isacharunderscore}{\kern0pt}eq{\isacharunderscore}{\kern0pt}Suc\ prod{\isachardot}{\kern0pt}simps{\isacharparenleft}{\kern0pt}{\isadigit{2}}{\isacharparenright}{\kern0pt}{\isacharparenright}{\kern0pt}\isanewline
\ \ \ \ \ \ \ \ \ \ \ \ \ \ \isacommand{also}\isamarkupfalse%
\ \isacommand{have}\isamarkupfalse%
\ {\isachardoublequoteopen}{\isasymdots}\ {\isacharequal}{\kern0pt}\ U\ {\isachardollar}{\kern0pt}{\isachardollar}{\kern0pt}\ {\isacharparenleft}{\kern0pt}{\isadigit{0}}{\isacharcomma}{\kern0pt}{\isadigit{0}}{\isacharparenright}{\kern0pt}\ {\isacharasterisk}{\kern0pt}\ {\isacharparenleft}{\kern0pt}cnj\ {\isacharparenleft}{\kern0pt}U\ {\isachardollar}{\kern0pt}{\isachardollar}{\kern0pt}\ {\isacharparenleft}{\kern0pt}{\isadigit{0}}{\isacharcomma}{\kern0pt}{\isadigit{0}}{\isacharparenright}{\kern0pt}{\isacharparenright}{\kern0pt}{\isacharparenright}{\kern0pt}\ {\isacharplus}{\kern0pt}\isanewline
\ \ \ \ \ \ \ \ \ \ \ \ \ \ \ \ \ \ \ \ \ \ \ \ \ \ \ \ \ \ U\ {\isachardollar}{\kern0pt}{\isachardollar}{\kern0pt}\ {\isacharparenleft}{\kern0pt}{\isadigit{0}}{\isacharcomma}{\kern0pt}{\isadigit{1}}{\isacharparenright}{\kern0pt}\ {\isacharasterisk}{\kern0pt}\ {\isacharparenleft}{\kern0pt}cnj\ {\isacharparenleft}{\kern0pt}U\ {\isachardollar}{\kern0pt}{\isachardollar}{\kern0pt}\ {\isacharparenleft}{\kern0pt}{\isadigit{0}}{\isacharcomma}{\kern0pt}{\isadigit{1}}{\isacharparenright}{\kern0pt}{\isacharparenright}{\kern0pt}{\isacharparenright}{\kern0pt}{\isachardoublequoteclose}\isanewline
\ \ \ \ \ \ \ \ \ \ \ \ \ \ \ \ \isacommand{using}\isamarkupfalse%
\ control{\isadigit{2}}{\isacharunderscore}{\kern0pt}def\ index{\isacharunderscore}{\kern0pt}mat{\isacharunderscore}{\kern0pt}of{\isacharunderscore}{\kern0pt}cols{\isacharunderscore}{\kern0pt}list\ \isacommand{by}\isamarkupfalse%
\ simp\isanewline
\ \ \ \ \ \ \ \ \ \ \ \ \ \ \isacommand{also}\isamarkupfalse%
\ \isacommand{have}\isamarkupfalse%
\ {\isachardoublequoteopen}{\isasymdots}\ {\isacharequal}{\kern0pt}\ {\isacharparenleft}{\kern0pt}U\ {\isachardollar}{\kern0pt}{\isachardollar}{\kern0pt}\ {\isacharparenleft}{\kern0pt}{\isadigit{0}}{\isacharcomma}{\kern0pt}{\isadigit{0}}{\isacharparenright}{\kern0pt}{\isacharparenright}{\kern0pt}\ {\isacharasterisk}{\kern0pt}\ {\isacharparenleft}{\kern0pt}{\isacharparenleft}{\kern0pt}U\isactrlsup {\isasymdagger}{\isacharparenright}{\kern0pt}\ {\isachardollar}{\kern0pt}{\isachardollar}{\kern0pt}\ {\isacharparenleft}{\kern0pt}{\isadigit{0}}{\isacharcomma}{\kern0pt}{\isadigit{0}}{\isacharparenright}{\kern0pt}{\isacharparenright}{\kern0pt}\ {\isacharplus}{\kern0pt}\isanewline
\ \ \ \ \ \ \ \ \ \ \ \ \ \ \ \ \ \ \ \ \ \ \ \ \ \ \ \ \ \ {\isacharparenleft}{\kern0pt}U\ {\isachardollar}{\kern0pt}{\isachardollar}{\kern0pt}\ {\isacharparenleft}{\kern0pt}{\isadigit{0}}{\isacharcomma}{\kern0pt}{\isadigit{1}}{\isacharparenright}{\kern0pt}{\isacharparenright}{\kern0pt}\ {\isacharasterisk}{\kern0pt}\ {\isacharparenleft}{\kern0pt}{\isacharparenleft}{\kern0pt}U\isactrlsup {\isasymdagger}{\isacharparenright}{\kern0pt}\ {\isachardollar}{\kern0pt}{\isachardollar}{\kern0pt}\ {\isacharparenleft}{\kern0pt}{\isadigit{1}}{\isacharcomma}{\kern0pt}{\isadigit{0}}{\isacharparenright}{\kern0pt}{\isacharparenright}{\kern0pt}{\isachardoublequoteclose}\isanewline
\ \ \ \ \ \ \ \ \ \ \ \ \ \ \ \ \isacommand{using}\isamarkupfalse%
\ dagger{\isacharunderscore}{\kern0pt}def\ assms{\isacharparenleft}{\kern0pt}{\isadigit{1}}{\isacharparenright}{\kern0pt}\ gate{\isacharunderscore}{\kern0pt}def\ \isacommand{by}\isamarkupfalse%
\ force\isanewline
\ \ \ \ \ \ \ \ \ \ \ \ \ \ \isacommand{also}\isamarkupfalse%
\ \isacommand{have}\isamarkupfalse%
\ {\isachardoublequoteopen}{\isasymdots}\ {\isacharequal}{\kern0pt}\ {\isacharparenleft}{\kern0pt}U\ {\isacharasterisk}{\kern0pt}\ {\isacharparenleft}{\kern0pt}U\isactrlsup {\isasymdagger}{\isacharparenright}{\kern0pt}{\isacharparenright}{\kern0pt}\ {\isachardollar}{\kern0pt}{\isachardollar}{\kern0pt}\ {\isacharparenleft}{\kern0pt}{\isadigit{0}}{\isacharcomma}{\kern0pt}{\isadigit{0}}{\isacharparenright}{\kern0pt}{\isachardoublequoteclose}\ \isanewline
\ \ \ \ \ \ \ \ \ \ \ \ \ \ \ \ \isacommand{using}\isamarkupfalse%
\ times{\isacharunderscore}{\kern0pt}mat{\isacharunderscore}{\kern0pt}def\ assms{\isacharparenleft}{\kern0pt}{\isadigit{1}}{\isacharparenright}{\kern0pt}\ gate{\isacharunderscore}{\kern0pt}carrier{\isacharunderscore}{\kern0pt}mat\ sumof{\isadigit{2}}\isanewline
\ \ \ \ \ \ \ \ \ \ \ \ \ \ \ \ \isacommand{by}\isamarkupfalse%
\ {\isacharparenleft}{\kern0pt}smt\ {\isacharparenleft}{\kern0pt}z{\isadigit{3}}{\isacharparenright}{\kern0pt}\ carrier{\isacharunderscore}{\kern0pt}matD{\isacharparenleft}{\kern0pt}{\isadigit{2}}{\isacharparenright}{\kern0pt}\ dagger{\isacharunderscore}{\kern0pt}def\ dim{\isacharunderscore}{\kern0pt}col{\isacharunderscore}{\kern0pt}mat{\isacharparenleft}{\kern0pt}{\isadigit{1}}{\isacharparenright}{\kern0pt}\ dim{\isacharunderscore}{\kern0pt}row{\isacharunderscore}{\kern0pt}of{\isacharunderscore}{\kern0pt}dagger\ \isanewline
\ \ \ \ \ \ \ \ \ \ \ \ \ \ \ \ \ \ \ \ gate{\isachardot}{\kern0pt}dim{\isacharunderscore}{\kern0pt}row\ index{\isacharunderscore}{\kern0pt}matrix{\isacharunderscore}{\kern0pt}prod\ pos{\isadigit{2}}\ power{\isacharunderscore}{\kern0pt}one{\isacharunderscore}{\kern0pt}right{\isacharparenright}{\kern0pt}\isanewline
\ \ \ \ \ \ \ \ \ \ \ \ \ \ \isacommand{also}\isamarkupfalse%
\ \isacommand{have}\isamarkupfalse%
\ {\isachardoublequoteopen}{\isasymdots}\ {\isacharequal}{\kern0pt}\ {\isacharparenleft}{\kern0pt}{\isadigit{1}}\isactrlsub m\ {\isadigit{2}}{\isacharparenright}{\kern0pt}\ {\isachardollar}{\kern0pt}{\isachardollar}{\kern0pt}\ {\isacharparenleft}{\kern0pt}{\isadigit{0}}{\isacharcomma}{\kern0pt}{\isadigit{0}}{\isacharparenright}{\kern0pt}{\isachardoublequoteclose}\ \isacommand{using}\isamarkupfalse%
\ assms{\isacharparenleft}{\kern0pt}{\isadigit{1}}{\isacharparenright}{\kern0pt}\ gate{\isacharunderscore}{\kern0pt}def\ unitary{\isacharunderscore}{\kern0pt}def\ \isacommand{by}\isamarkupfalse%
\ auto\isanewline
\ \ \ \ \ \ \ \ \ \ \ \ \ \ \isacommand{also}\isamarkupfalse%
\ \isacommand{have}\isamarkupfalse%
\ {\isachardoublequoteopen}{\isasymdots}\ {\isacharequal}{\kern0pt}\ {\isadigit{1}}{\isachardoublequoteclose}\ \isacommand{by}\isamarkupfalse%
\ auto\isanewline
\ \ \ \ \ \ \ \ \ \ \ \ \ \ \isacommand{also}\isamarkupfalse%
\ \isacommand{have}\isamarkupfalse%
\ {\isachardoublequoteopen}{\isasymdots}\ {\isacharequal}{\kern0pt}\ {\isadigit{1}}\isactrlsub m\ {\isadigit{4}}\ {\isachardollar}{\kern0pt}{\isachardollar}{\kern0pt}\ {\isacharparenleft}{\kern0pt}{\isadigit{1}}{\isacharcomma}{\kern0pt}{\isadigit{1}}{\isacharparenright}{\kern0pt}{\isachardoublequoteclose}\ \isacommand{by}\isamarkupfalse%
\ simp\isanewline
\ \ \ \ \ \ \ \ \ \ \ \ \ \ \isacommand{finally}\isamarkupfalse%
\ \isacommand{show}\isamarkupfalse%
\ {\isacharquery}{\kern0pt}thesis\ \isacommand{using}\isamarkupfalse%
\ i{\isadigit{1}}\ j{\isadigit{1}}\ \isacommand{by}\isamarkupfalse%
\ simp\isanewline
\ \ \ \ \ \ \ \ \ \ \ \ \isacommand{qed}\isamarkupfalse%
\isanewline
\ \ \ \ \ \ \ \ \ \ \isacommand{next}\isamarkupfalse%
\isanewline
\ \ \ \ \ \ \ \ \ \ \ \ \isacommand{assume}\isamarkupfalse%
\ jl{\isadigit{2}}{\isacharcolon}{\kern0pt}{\isachardoublequoteopen}j\ {\isacharequal}{\kern0pt}\ {\isadigit{2}}\ {\isasymor}\ j\ {\isacharequal}{\kern0pt}\ {\isadigit{3}}{\isachardoublequoteclose}\isanewline
\ \ \ \ \ \ \ \ \ \ \ \ \isacommand{show}\isamarkupfalse%
\ {\isachardoublequoteopen}{\isacharparenleft}{\kern0pt}control{\isadigit{2}}\ U\ {\isacharasterisk}{\kern0pt}\ {\isacharparenleft}{\kern0pt}{\isacharparenleft}{\kern0pt}control{\isadigit{2}}\ U{\isacharparenright}{\kern0pt}\isactrlsup {\isasymdagger}{\isacharparenright}{\kern0pt}{\isacharparenright}{\kern0pt}\ {\isachardollar}{\kern0pt}{\isachardollar}{\kern0pt}\ {\isacharparenleft}{\kern0pt}i{\isacharcomma}{\kern0pt}\ j{\isacharparenright}{\kern0pt}\ {\isacharequal}{\kern0pt}\ {\isadigit{1}}\isactrlsub m\ {\isadigit{4}}\ {\isachardollar}{\kern0pt}{\isachardollar}{\kern0pt}\ {\isacharparenleft}{\kern0pt}i{\isacharcomma}{\kern0pt}\ j{\isacharparenright}{\kern0pt}{\isachardoublequoteclose}\isanewline
\ \ \ \ \ \ \ \ \ \ \ \ \isacommand{proof}\isamarkupfalse%
\ {\isacharparenleft}{\kern0pt}rule\ disjE{\isacharparenright}{\kern0pt}\isanewline
\ \ \ \ \ \ \ \ \ \ \ \ \ \ \isacommand{show}\isamarkupfalse%
\ {\isachardoublequoteopen}j\ {\isacharequal}{\kern0pt}\ {\isadigit{2}}\ {\isasymor}\ j\ {\isacharequal}{\kern0pt}\ {\isadigit{3}}{\isachardoublequoteclose}\ \isacommand{using}\isamarkupfalse%
\ jl{\isadigit{2}}\ \isacommand{by}\isamarkupfalse%
\ this\isanewline
\ \ \ \ \ \ \ \ \ \ \ \ \isacommand{next}\isamarkupfalse%
\isanewline
\ \ \ \ \ \ \ \ \ \ \ \ \ \ \isacommand{assume}\isamarkupfalse%
\ j{\isadigit{2}}{\isacharcolon}{\kern0pt}{\isachardoublequoteopen}j\ {\isacharequal}{\kern0pt}\ {\isadigit{2}}{\isachardoublequoteclose}\isanewline
\ \ \ \ \ \ \ \ \ \ \ \ \ \ \isacommand{show}\isamarkupfalse%
\ {\isachardoublequoteopen}{\isacharparenleft}{\kern0pt}control{\isadigit{2}}\ U\ {\isacharasterisk}{\kern0pt}\ {\isacharparenleft}{\kern0pt}{\isacharparenleft}{\kern0pt}control{\isadigit{2}}\ U{\isacharparenright}{\kern0pt}\isactrlsup {\isasymdagger}{\isacharparenright}{\kern0pt}{\isacharparenright}{\kern0pt}\ {\isachardollar}{\kern0pt}{\isachardollar}{\kern0pt}\ {\isacharparenleft}{\kern0pt}i{\isacharcomma}{\kern0pt}\ j{\isacharparenright}{\kern0pt}\ {\isacharequal}{\kern0pt}\ {\isadigit{1}}\isactrlsub m\ {\isadigit{4}}\ {\isachardollar}{\kern0pt}{\isachardollar}{\kern0pt}\ {\isacharparenleft}{\kern0pt}i{\isacharcomma}{\kern0pt}\ j{\isacharparenright}{\kern0pt}{\isachardoublequoteclose}\isanewline
\ \ \ \ \ \ \ \ \ \ \ \ \ \ \isacommand{proof}\isamarkupfalse%
\ {\isacharminus}{\kern0pt}\isanewline
\ \ \ \ \ \ \ \ \ \ \ \ \ \ \ \ \isacommand{have}\isamarkupfalse%
\ {\isachardoublequoteopen}{\isacharparenleft}{\kern0pt}control{\isadigit{2}}\ U\ {\isacharasterisk}{\kern0pt}\ {\isacharparenleft}{\kern0pt}{\isacharparenleft}{\kern0pt}control{\isadigit{2}}\ U{\isacharparenright}{\kern0pt}\isactrlsup {\isasymdagger}{\isacharparenright}{\kern0pt}{\isacharparenright}{\kern0pt}\ {\isachardollar}{\kern0pt}{\isachardollar}{\kern0pt}\ {\isacharparenleft}{\kern0pt}{\isadigit{1}}{\isacharcomma}{\kern0pt}{\isadigit{2}}{\isacharparenright}{\kern0pt}\ {\isacharequal}{\kern0pt}\ \isanewline
\ \ \ \ \ \ \ \ \ \ \ \ \ \ \ \ \ \ \ \ {\isacharparenleft}{\kern0pt}control{\isadigit{2}}\ U{\isacharparenright}{\kern0pt}\ {\isachardollar}{\kern0pt}{\isachardollar}{\kern0pt}\ {\isacharparenleft}{\kern0pt}{\isadigit{1}}{\isacharcomma}{\kern0pt}{\isadigit{0}}{\isacharparenright}{\kern0pt}\ {\isacharasterisk}{\kern0pt}\ {\isacharparenleft}{\kern0pt}{\isacharparenleft}{\kern0pt}control{\isadigit{2}}\ U{\isacharparenright}{\kern0pt}\isactrlsup {\isasymdagger}{\isacharparenright}{\kern0pt}\ {\isachardollar}{\kern0pt}{\isachardollar}{\kern0pt}\ {\isacharparenleft}{\kern0pt}{\isadigit{0}}{\isacharcomma}{\kern0pt}{\isadigit{2}}{\isacharparenright}{\kern0pt}\ {\isacharplus}{\kern0pt}\isanewline
\ \ \ \ \ \ \ \ \ \ \ \ \ \ \ \ \ \ \ \ {\isacharparenleft}{\kern0pt}control{\isadigit{2}}\ U{\isacharparenright}{\kern0pt}\ {\isachardollar}{\kern0pt}{\isachardollar}{\kern0pt}\ {\isacharparenleft}{\kern0pt}{\isadigit{1}}{\isacharcomma}{\kern0pt}{\isadigit{1}}{\isacharparenright}{\kern0pt}\ {\isacharasterisk}{\kern0pt}\ {\isacharparenleft}{\kern0pt}{\isacharparenleft}{\kern0pt}control{\isadigit{2}}\ U{\isacharparenright}{\kern0pt}\isactrlsup {\isasymdagger}{\isacharparenright}{\kern0pt}\ {\isachardollar}{\kern0pt}{\isachardollar}{\kern0pt}\ {\isacharparenleft}{\kern0pt}{\isadigit{1}}{\isacharcomma}{\kern0pt}{\isadigit{2}}{\isacharparenright}{\kern0pt}\ {\isacharplus}{\kern0pt}\isanewline
\ \ \ \ \ \ \ \ \ \ \ \ \ \ \ \ \ \ \ \ {\isacharparenleft}{\kern0pt}control{\isadigit{2}}\ U{\isacharparenright}{\kern0pt}\ {\isachardollar}{\kern0pt}{\isachardollar}{\kern0pt}\ {\isacharparenleft}{\kern0pt}{\isadigit{1}}{\isacharcomma}{\kern0pt}{\isadigit{2}}{\isacharparenright}{\kern0pt}\ {\isacharasterisk}{\kern0pt}\ {\isacharparenleft}{\kern0pt}{\isacharparenleft}{\kern0pt}control{\isadigit{2}}\ U{\isacharparenright}{\kern0pt}\isactrlsup {\isasymdagger}{\isacharparenright}{\kern0pt}\ {\isachardollar}{\kern0pt}{\isachardollar}{\kern0pt}\ {\isacharparenleft}{\kern0pt}{\isadigit{2}}{\isacharcomma}{\kern0pt}{\isadigit{2}}{\isacharparenright}{\kern0pt}\ {\isacharplus}{\kern0pt}\isanewline
\ \ \ \ \ \ \ \ \ \ \ \ \ \ \ \ \ \ \ \ {\isacharparenleft}{\kern0pt}control{\isadigit{2}}\ U{\isacharparenright}{\kern0pt}\ {\isachardollar}{\kern0pt}{\isachardollar}{\kern0pt}\ {\isacharparenleft}{\kern0pt}{\isadigit{1}}{\isacharcomma}{\kern0pt}{\isadigit{3}}{\isacharparenright}{\kern0pt}\ {\isacharasterisk}{\kern0pt}\ {\isacharparenleft}{\kern0pt}{\isacharparenleft}{\kern0pt}control{\isadigit{2}}\ U{\isacharparenright}{\kern0pt}\isactrlsup {\isasymdagger}{\isacharparenright}{\kern0pt}\ {\isachardollar}{\kern0pt}{\isachardollar}{\kern0pt}\ {\isacharparenleft}{\kern0pt}{\isadigit{3}}{\isacharcomma}{\kern0pt}{\isadigit{2}}{\isacharparenright}{\kern0pt}{\isachardoublequoteclose}\isanewline
\ \ \ \ \ \ \ \ \ \ \ \ \ \ \ \ \isacommand{using}\isamarkupfalse%
\ times{\isacharunderscore}{\kern0pt}mat{\isacharunderscore}{\kern0pt}def\ sumof{\isadigit{4}}\isanewline
\ \ \ \ \ \ \ \ \ \ \ \ \ \ \ \ \ \ \isacommand{by}\isamarkupfalse%
\ {\isacharparenleft}{\kern0pt}smt\ {\isacharparenleft}{\kern0pt}z{\isadigit{3}}{\isacharparenright}{\kern0pt}\ carrier{\isacharunderscore}{\kern0pt}matD{\isacharparenleft}{\kern0pt}{\isadigit{1}}{\isacharparenright}{\kern0pt}\ carrier{\isacharunderscore}{\kern0pt}matD{\isacharparenleft}{\kern0pt}{\isadigit{2}}{\isacharparenright}{\kern0pt}\ control{\isadigit{2}}{\isacharunderscore}{\kern0pt}carrier{\isacharunderscore}{\kern0pt}mat\ dim{\isacharunderscore}{\kern0pt}col{\isacharunderscore}{\kern0pt}of{\isacharunderscore}{\kern0pt}dagger\ \isanewline
\ \ \ \ \ \ \ \ \ \ \ \ \ \ \ \ \ \ \ \ \ \ \ \ dim{\isacharunderscore}{\kern0pt}row{\isacharunderscore}{\kern0pt}of{\isacharunderscore}{\kern0pt}dagger\ i{\isadigit{1}}\ i{\isadigit{4}}\ index{\isacharunderscore}{\kern0pt}matrix{\isacharunderscore}{\kern0pt}prod\ j{\isadigit{2}}\ j{\isadigit{4}}{\isacharparenright}{\kern0pt}\isanewline
\ \ \ \ \ \ \ \ \ \ \ \ \ \ \isacommand{also}\isamarkupfalse%
\ \isacommand{have}\isamarkupfalse%
\ {\isachardoublequoteopen}{\isasymdots}\ {\isacharequal}{\kern0pt}\ {\isacharparenleft}{\kern0pt}control{\isadigit{2}}\ U{\isacharparenright}{\kern0pt}\ {\isachardollar}{\kern0pt}{\isachardollar}{\kern0pt}\ {\isacharparenleft}{\kern0pt}{\isadigit{1}}{\isacharcomma}{\kern0pt}{\isadigit{1}}{\isacharparenright}{\kern0pt}\ {\isacharasterisk}{\kern0pt}\ {\isacharparenleft}{\kern0pt}{\isacharparenleft}{\kern0pt}control{\isadigit{2}}\ U{\isacharparenright}{\kern0pt}\isactrlsup {\isasymdagger}{\isacharparenright}{\kern0pt}\ {\isachardollar}{\kern0pt}{\isachardollar}{\kern0pt}\ {\isacharparenleft}{\kern0pt}{\isadigit{1}}{\isacharcomma}{\kern0pt}{\isadigit{2}}{\isacharparenright}{\kern0pt}\ {\isacharplus}{\kern0pt}\ \isanewline
\ \ \ \ \ \ \ \ \ \ \ \ \ \ \ \ \ \ \ \ \ \ \ \ \ \ \ \ \ \ \ \ {\isacharparenleft}{\kern0pt}control{\isadigit{2}}\ U{\isacharparenright}{\kern0pt}\ {\isachardollar}{\kern0pt}{\isachardollar}{\kern0pt}\ {\isacharparenleft}{\kern0pt}{\isadigit{1}}{\isacharcomma}{\kern0pt}{\isadigit{3}}{\isacharparenright}{\kern0pt}\ {\isacharasterisk}{\kern0pt}\ {\isacharparenleft}{\kern0pt}{\isacharparenleft}{\kern0pt}control{\isadigit{2}}\ U{\isacharparenright}{\kern0pt}\isactrlsup {\isasymdagger}{\isacharparenright}{\kern0pt}\ {\isachardollar}{\kern0pt}{\isachardollar}{\kern0pt}\ {\isacharparenleft}{\kern0pt}{\isadigit{3}}{\isacharcomma}{\kern0pt}{\isadigit{2}}{\isacharparenright}{\kern0pt}{\isachardoublequoteclose}\isanewline
\ \ \ \ \ \ \ \ \ \ \ \ \ \ \ \ \ \ \isacommand{using}\isamarkupfalse%
\ control{\isadigit{2}}{\isacharunderscore}{\kern0pt}def\ index{\isacharunderscore}{\kern0pt}mat{\isacharunderscore}{\kern0pt}of{\isacharunderscore}{\kern0pt}cols{\isacharunderscore}{\kern0pt}list\ \isacommand{by}\isamarkupfalse%
\ force\isanewline
\ \ \ \ \ \ \ \ \ \ \ \ \ \ \isacommand{also}\isamarkupfalse%
\ \isacommand{have}\isamarkupfalse%
\ {\isachardoublequoteopen}{\isasymdots}\ {\isacharequal}{\kern0pt}\ {\isacharparenleft}{\kern0pt}control{\isadigit{2}}\ U{\isacharparenright}{\kern0pt}\ {\isachardollar}{\kern0pt}{\isachardollar}{\kern0pt}\ {\isacharparenleft}{\kern0pt}{\isadigit{1}}{\isacharcomma}{\kern0pt}{\isadigit{1}}{\isacharparenright}{\kern0pt}\ {\isacharasterisk}{\kern0pt}\ {\isacharparenleft}{\kern0pt}cnj\ {\isacharparenleft}{\kern0pt}{\isacharparenleft}{\kern0pt}control{\isadigit{2}}\ U{\isacharparenright}{\kern0pt}\ {\isachardollar}{\kern0pt}{\isachardollar}{\kern0pt}\ {\isacharparenleft}{\kern0pt}{\isadigit{2}}{\isacharcomma}{\kern0pt}{\isadigit{1}}{\isacharparenright}{\kern0pt}{\isacharparenright}{\kern0pt}{\isacharparenright}{\kern0pt}\ {\isacharplus}{\kern0pt}\ \isanewline
\ \ \ \ \ \ \ \ \ \ \ \ \ \ \ \ \ \ \ \ \ \ \ \ \ \ \ \ \ \ \ \ {\isacharparenleft}{\kern0pt}control{\isadigit{2}}\ U{\isacharparenright}{\kern0pt}\ {\isachardollar}{\kern0pt}{\isachardollar}{\kern0pt}\ {\isacharparenleft}{\kern0pt}{\isadigit{1}}{\isacharcomma}{\kern0pt}{\isadigit{3}}{\isacharparenright}{\kern0pt}\ {\isacharasterisk}{\kern0pt}\ {\isacharparenleft}{\kern0pt}cnj\ {\isacharparenleft}{\kern0pt}{\isacharparenleft}{\kern0pt}control{\isadigit{2}}\ U{\isacharparenright}{\kern0pt}\ {\isachardollar}{\kern0pt}{\isachardollar}{\kern0pt}\ {\isacharparenleft}{\kern0pt}{\isadigit{2}}{\isacharcomma}{\kern0pt}{\isadigit{3}}{\isacharparenright}{\kern0pt}{\isacharparenright}{\kern0pt}{\isacharparenright}{\kern0pt}{\isachardoublequoteclose}\isanewline
\ \ \ \ \ \ \ \ \ \ \ \ \ \ \ \ \ \ \isacommand{using}\isamarkupfalse%
\ dagger{\isacharunderscore}{\kern0pt}def\isanewline
\ \ \ \ \ \ \ \ \ \ \ \ \ \ \ \ \ \ \isacommand{by}\isamarkupfalse%
\ {\isacharparenleft}{\kern0pt}smt\ {\isacharparenleft}{\kern0pt}verit{\isacharcomma}{\kern0pt}\ ccfv{\isacharunderscore}{\kern0pt}threshold{\isacharparenright}{\kern0pt}\ One{\isacharunderscore}{\kern0pt}nat{\isacharunderscore}{\kern0pt}def\ Suc{\isacharunderscore}{\kern0pt}{\isadigit{1}}\ add{\isachardot}{\kern0pt}commute\ add{\isacharunderscore}{\kern0pt}Suc{\isacharunderscore}{\kern0pt}right\ \isanewline
\ \ \ \ \ \ \ \ \ \ \ \ \ \ \ \ \ \ \ \ \ \ carrier{\isacharunderscore}{\kern0pt}matD{\isacharparenleft}{\kern0pt}{\isadigit{1}}{\isacharparenright}{\kern0pt}\ carrier{\isacharunderscore}{\kern0pt}matD{\isacharparenleft}{\kern0pt}{\isadigit{2}}{\isacharparenright}{\kern0pt}\ control{\isadigit{2}}{\isacharunderscore}{\kern0pt}carrier{\isacharunderscore}{\kern0pt}mat\ i{\isadigit{1}}\ i{\isadigit{4}}\ index{\isacharunderscore}{\kern0pt}mat{\isacharparenleft}{\kern0pt}{\isadigit{1}}{\isacharparenright}{\kern0pt}\ j{\isadigit{2}}\ j{\isadigit{4}}\ \isanewline
\ \ \ \ \ \ \ \ \ \ \ \ \ \ \ \ \ \ \ \ \ \ lessI\ numeral{\isacharunderscore}{\kern0pt}{\isadigit{3}}{\isacharunderscore}{\kern0pt}eq{\isacharunderscore}{\kern0pt}{\isadigit{3}}\ numeral{\isacharunderscore}{\kern0pt}Bit{\isadigit{0}}\ plus{\isacharunderscore}{\kern0pt}{\isadigit{1}}{\isacharunderscore}{\kern0pt}eq{\isacharunderscore}{\kern0pt}Suc\ prod{\isachardot}{\kern0pt}simps{\isacharparenleft}{\kern0pt}{\isadigit{2}}{\isacharparenright}{\kern0pt}{\isacharparenright}{\kern0pt}\isanewline
\ \ \ \ \ \ \ \ \ \ \ \ \ \ \isacommand{also}\isamarkupfalse%
\ \isacommand{have}\isamarkupfalse%
\ {\isachardoublequoteopen}{\isasymdots}\ {\isacharequal}{\kern0pt}\ {\isacharparenleft}{\kern0pt}control{\isadigit{2}}\ U{\isacharparenright}{\kern0pt}\ {\isachardollar}{\kern0pt}{\isachardollar}{\kern0pt}\ {\isacharparenleft}{\kern0pt}{\isadigit{1}}{\isacharcomma}{\kern0pt}{\isadigit{1}}{\isacharparenright}{\kern0pt}\ {\isacharasterisk}{\kern0pt}\ {\isacharparenleft}{\kern0pt}cnj\ {\isadigit{0}}{\isacharparenright}{\kern0pt}\ {\isacharplus}{\kern0pt}\isanewline
\ \ \ \ \ \ \ \ \ \ \ \ \ \ \ \ \ \ \ \ \ \ \ \ \ \ \ \ \ \ \ \ {\isacharparenleft}{\kern0pt}control{\isadigit{2}}\ U{\isacharparenright}{\kern0pt}\ {\isachardollar}{\kern0pt}{\isachardollar}{\kern0pt}\ {\isacharparenleft}{\kern0pt}{\isadigit{1}}{\isacharcomma}{\kern0pt}{\isadigit{3}}{\isacharparenright}{\kern0pt}\ {\isacharasterisk}{\kern0pt}\ {\isacharparenleft}{\kern0pt}cnj\ {\isadigit{0}}{\isacharparenright}{\kern0pt}{\isachardoublequoteclose}\isanewline
\ \ \ \ \ \ \ \ \ \ \ \ \ \ \ \ \ \ \isacommand{using}\isamarkupfalse%
\ control{\isadigit{2}}{\isacharunderscore}{\kern0pt}def\ index{\isacharunderscore}{\kern0pt}mat{\isacharunderscore}{\kern0pt}of{\isacharunderscore}{\kern0pt}cols{\isacharunderscore}{\kern0pt}list\ \isacommand{by}\isamarkupfalse%
\ simp\isanewline
\ \ \ \ \ \ \ \ \ \ \ \ \ \ \isacommand{also}\isamarkupfalse%
\ \isacommand{have}\isamarkupfalse%
\ {\isachardoublequoteopen}{\isasymdots}\ {\isacharequal}{\kern0pt}\ {\isadigit{0}}{\isachardoublequoteclose}\ \isacommand{by}\isamarkupfalse%
\ auto\isanewline
\ \ \ \ \ \ \ \ \ \ \ \ \ \ \isacommand{also}\isamarkupfalse%
\ \isacommand{have}\isamarkupfalse%
\ {\isachardoublequoteopen}{\isasymdots}\ {\isacharequal}{\kern0pt}\ {\isadigit{1}}\isactrlsub m\ {\isadigit{4}}\ {\isachardollar}{\kern0pt}{\isachardollar}{\kern0pt}\ {\isacharparenleft}{\kern0pt}{\isadigit{1}}{\isacharcomma}{\kern0pt}{\isadigit{2}}{\isacharparenright}{\kern0pt}{\isachardoublequoteclose}\ \isacommand{by}\isamarkupfalse%
\ simp\isanewline
\ \ \ \ \ \ \ \ \ \ \ \ \ \ \isacommand{finally}\isamarkupfalse%
\ \isacommand{show}\isamarkupfalse%
\ {\isacharquery}{\kern0pt}thesis\ \isacommand{using}\isamarkupfalse%
\ i{\isadigit{1}}\ j{\isadigit{2}}\ \isacommand{by}\isamarkupfalse%
\ simp\isanewline
\ \ \ \ \ \ \ \ \ \ \ \ \isacommand{qed}\isamarkupfalse%
\isanewline
\ \ \ \ \ \ \ \ \ \ \isacommand{next}\isamarkupfalse%
\isanewline
\ \ \ \ \ \ \ \ \ \ \ \ \isacommand{assume}\isamarkupfalse%
\ j{\isadigit{3}}{\isacharcolon}{\kern0pt}{\isachardoublequoteopen}j\ {\isacharequal}{\kern0pt}\ {\isadigit{3}}{\isachardoublequoteclose}\isanewline
\ \ \ \ \ \ \ \ \ \ \ \ \isacommand{show}\isamarkupfalse%
\ {\isachardoublequoteopen}{\isacharparenleft}{\kern0pt}control{\isadigit{2}}\ U\ {\isacharasterisk}{\kern0pt}\ {\isacharparenleft}{\kern0pt}{\isacharparenleft}{\kern0pt}control{\isadigit{2}}\ U{\isacharparenright}{\kern0pt}\isactrlsup {\isasymdagger}{\isacharparenright}{\kern0pt}{\isacharparenright}{\kern0pt}\ {\isachardollar}{\kern0pt}{\isachardollar}{\kern0pt}\ {\isacharparenleft}{\kern0pt}i{\isacharcomma}{\kern0pt}\ j{\isacharparenright}{\kern0pt}\ {\isacharequal}{\kern0pt}\ {\isadigit{1}}\isactrlsub m\ {\isadigit{4}}\ {\isachardollar}{\kern0pt}{\isachardollar}{\kern0pt}\ {\isacharparenleft}{\kern0pt}i{\isacharcomma}{\kern0pt}\ j{\isacharparenright}{\kern0pt}{\isachardoublequoteclose}\isanewline
\ \ \ \ \ \ \ \ \ \ \ \ \isacommand{proof}\isamarkupfalse%
\ {\isacharminus}{\kern0pt}\isanewline
\ \ \ \ \ \ \ \ \ \ \ \ \ \ \isacommand{have}\isamarkupfalse%
\ {\isachardoublequoteopen}{\isacharparenleft}{\kern0pt}control{\isadigit{2}}\ U\ {\isacharasterisk}{\kern0pt}\ {\isacharparenleft}{\kern0pt}{\isacharparenleft}{\kern0pt}control{\isadigit{2}}\ U{\isacharparenright}{\kern0pt}\isactrlsup {\isasymdagger}{\isacharparenright}{\kern0pt}{\isacharparenright}{\kern0pt}\ {\isachardollar}{\kern0pt}{\isachardollar}{\kern0pt}\ {\isacharparenleft}{\kern0pt}{\isadigit{1}}{\isacharcomma}{\kern0pt}{\isadigit{3}}{\isacharparenright}{\kern0pt}\ {\isacharequal}{\kern0pt}\ \isanewline
\ \ \ \ \ \ \ \ \ \ \ \ \ \ \ \ \ \ \ \ {\isacharparenleft}{\kern0pt}control{\isadigit{2}}\ U{\isacharparenright}{\kern0pt}\ {\isachardollar}{\kern0pt}{\isachardollar}{\kern0pt}\ {\isacharparenleft}{\kern0pt}{\isadigit{1}}{\isacharcomma}{\kern0pt}{\isadigit{0}}{\isacharparenright}{\kern0pt}\ {\isacharasterisk}{\kern0pt}\ {\isacharparenleft}{\kern0pt}{\isacharparenleft}{\kern0pt}control{\isadigit{2}}\ U{\isacharparenright}{\kern0pt}\isactrlsup {\isasymdagger}{\isacharparenright}{\kern0pt}\ {\isachardollar}{\kern0pt}{\isachardollar}{\kern0pt}\ {\isacharparenleft}{\kern0pt}{\isadigit{0}}{\isacharcomma}{\kern0pt}{\isadigit{3}}{\isacharparenright}{\kern0pt}\ {\isacharplus}{\kern0pt}\isanewline
\ \ \ \ \ \ \ \ \ \ \ \ \ \ \ \ \ \ \ \ {\isacharparenleft}{\kern0pt}control{\isadigit{2}}\ U{\isacharparenright}{\kern0pt}\ {\isachardollar}{\kern0pt}{\isachardollar}{\kern0pt}\ {\isacharparenleft}{\kern0pt}{\isadigit{1}}{\isacharcomma}{\kern0pt}{\isadigit{1}}{\isacharparenright}{\kern0pt}\ {\isacharasterisk}{\kern0pt}\ {\isacharparenleft}{\kern0pt}{\isacharparenleft}{\kern0pt}control{\isadigit{2}}\ U{\isacharparenright}{\kern0pt}\isactrlsup {\isasymdagger}{\isacharparenright}{\kern0pt}\ {\isachardollar}{\kern0pt}{\isachardollar}{\kern0pt}\ {\isacharparenleft}{\kern0pt}{\isadigit{1}}{\isacharcomma}{\kern0pt}{\isadigit{3}}{\isacharparenright}{\kern0pt}\ {\isacharplus}{\kern0pt}\isanewline
\ \ \ \ \ \ \ \ \ \ \ \ \ \ \ \ \ \ \ \ {\isacharparenleft}{\kern0pt}control{\isadigit{2}}\ U{\isacharparenright}{\kern0pt}\ {\isachardollar}{\kern0pt}{\isachardollar}{\kern0pt}\ {\isacharparenleft}{\kern0pt}{\isadigit{1}}{\isacharcomma}{\kern0pt}{\isadigit{2}}{\isacharparenright}{\kern0pt}\ {\isacharasterisk}{\kern0pt}\ {\isacharparenleft}{\kern0pt}{\isacharparenleft}{\kern0pt}control{\isadigit{2}}\ U{\isacharparenright}{\kern0pt}\isactrlsup {\isasymdagger}{\isacharparenright}{\kern0pt}\ {\isachardollar}{\kern0pt}{\isachardollar}{\kern0pt}\ {\isacharparenleft}{\kern0pt}{\isadigit{2}}{\isacharcomma}{\kern0pt}{\isadigit{3}}{\isacharparenright}{\kern0pt}\ {\isacharplus}{\kern0pt}\isanewline
\ \ \ \ \ \ \ \ \ \ \ \ \ \ \ \ \ \ \ \ {\isacharparenleft}{\kern0pt}control{\isadigit{2}}\ U{\isacharparenright}{\kern0pt}\ {\isachardollar}{\kern0pt}{\isachardollar}{\kern0pt}\ {\isacharparenleft}{\kern0pt}{\isadigit{1}}{\isacharcomma}{\kern0pt}{\isadigit{3}}{\isacharparenright}{\kern0pt}\ {\isacharasterisk}{\kern0pt}\ {\isacharparenleft}{\kern0pt}{\isacharparenleft}{\kern0pt}control{\isadigit{2}}\ U{\isacharparenright}{\kern0pt}\isactrlsup {\isasymdagger}{\isacharparenright}{\kern0pt}\ {\isachardollar}{\kern0pt}{\isachardollar}{\kern0pt}\ {\isacharparenleft}{\kern0pt}{\isadigit{3}}{\isacharcomma}{\kern0pt}{\isadigit{3}}{\isacharparenright}{\kern0pt}{\isachardoublequoteclose}\isanewline
\ \ \ \ \ \ \ \ \ \ \ \ \ \ \ \ \isacommand{using}\isamarkupfalse%
\ times{\isacharunderscore}{\kern0pt}mat{\isacharunderscore}{\kern0pt}def\ sumof{\isadigit{4}}\isanewline
\ \ \ \ \ \ \ \ \ \ \ \ \ \ \ \ \isacommand{by}\isamarkupfalse%
\ {\isacharparenleft}{\kern0pt}smt\ {\isacharparenleft}{\kern0pt}z{\isadigit{3}}{\isacharparenright}{\kern0pt}\ carrier{\isacharunderscore}{\kern0pt}matD{\isacharparenleft}{\kern0pt}{\isadigit{1}}{\isacharparenright}{\kern0pt}\ carrier{\isacharunderscore}{\kern0pt}matD{\isacharparenleft}{\kern0pt}{\isadigit{2}}{\isacharparenright}{\kern0pt}\ control{\isadigit{2}}{\isacharunderscore}{\kern0pt}carrier{\isacharunderscore}{\kern0pt}mat\ dim{\isacharunderscore}{\kern0pt}col{\isacharunderscore}{\kern0pt}of{\isacharunderscore}{\kern0pt}dagger\ \isanewline
\ \ \ \ \ \ \ \ \ \ \ \ \ \ \ \ \ \ \ \ \ \ \ \ dim{\isacharunderscore}{\kern0pt}row{\isacharunderscore}{\kern0pt}of{\isacharunderscore}{\kern0pt}dagger\ i{\isadigit{1}}\ i{\isadigit{4}}\ index{\isacharunderscore}{\kern0pt}matrix{\isacharunderscore}{\kern0pt}prod\ j{\isadigit{3}}\ j{\isadigit{4}}{\isacharparenright}{\kern0pt}\isanewline
\ \ \ \ \ \ \ \ \ \ \ \ \ \ \isacommand{also}\isamarkupfalse%
\ \isacommand{have}\isamarkupfalse%
\ {\isachardoublequoteopen}{\isasymdots}\ {\isacharequal}{\kern0pt}\ {\isacharparenleft}{\kern0pt}control{\isadigit{2}}\ U{\isacharparenright}{\kern0pt}\ {\isachardollar}{\kern0pt}{\isachardollar}{\kern0pt}\ {\isacharparenleft}{\kern0pt}{\isadigit{1}}{\isacharcomma}{\kern0pt}{\isadigit{1}}{\isacharparenright}{\kern0pt}\ {\isacharasterisk}{\kern0pt}\ {\isacharparenleft}{\kern0pt}{\isacharparenleft}{\kern0pt}control{\isadigit{2}}\ U{\isacharparenright}{\kern0pt}\isactrlsup {\isasymdagger}{\isacharparenright}{\kern0pt}\ {\isachardollar}{\kern0pt}{\isachardollar}{\kern0pt}\ {\isacharparenleft}{\kern0pt}{\isadigit{1}}{\isacharcomma}{\kern0pt}{\isadigit{3}}{\isacharparenright}{\kern0pt}\ {\isacharplus}{\kern0pt}\ \isanewline
\ \ \ \ \ \ \ \ \ \ \ \ \ \ \ \ \ \ \ \ \ \ \ \ \ \ \ \ \ \ \ \ {\isacharparenleft}{\kern0pt}control{\isadigit{2}}\ U{\isacharparenright}{\kern0pt}\ {\isachardollar}{\kern0pt}{\isachardollar}{\kern0pt}\ {\isacharparenleft}{\kern0pt}{\isadigit{1}}{\isacharcomma}{\kern0pt}{\isadigit{3}}{\isacharparenright}{\kern0pt}\ {\isacharasterisk}{\kern0pt}\ {\isacharparenleft}{\kern0pt}{\isacharparenleft}{\kern0pt}control{\isadigit{2}}\ U{\isacharparenright}{\kern0pt}\isactrlsup {\isasymdagger}{\isacharparenright}{\kern0pt}\ {\isachardollar}{\kern0pt}{\isachardollar}{\kern0pt}\ {\isacharparenleft}{\kern0pt}{\isadigit{3}}{\isacharcomma}{\kern0pt}{\isadigit{3}}{\isacharparenright}{\kern0pt}{\isachardoublequoteclose}\isanewline
\ \ \ \ \ \ \ \ \ \ \ \ \ \ \ \ \ \ \isacommand{using}\isamarkupfalse%
\ control{\isadigit{2}}{\isacharunderscore}{\kern0pt}def\ index{\isacharunderscore}{\kern0pt}mat{\isacharunderscore}{\kern0pt}of{\isacharunderscore}{\kern0pt}cols{\isacharunderscore}{\kern0pt}list\ \isacommand{by}\isamarkupfalse%
\ force\isanewline
\ \ \ \ \ \ \ \ \ \ \ \ \ \ \isacommand{also}\isamarkupfalse%
\ \isacommand{have}\isamarkupfalse%
\ {\isachardoublequoteopen}{\isasymdots}\ {\isacharequal}{\kern0pt}\ {\isacharparenleft}{\kern0pt}control{\isadigit{2}}\ U{\isacharparenright}{\kern0pt}\ {\isachardollar}{\kern0pt}{\isachardollar}{\kern0pt}\ {\isacharparenleft}{\kern0pt}{\isadigit{1}}{\isacharcomma}{\kern0pt}{\isadigit{1}}{\isacharparenright}{\kern0pt}\ {\isacharasterisk}{\kern0pt}\ {\isacharparenleft}{\kern0pt}cnj\ {\isacharparenleft}{\kern0pt}{\isacharparenleft}{\kern0pt}control{\isadigit{2}}\ U{\isacharparenright}{\kern0pt}\ {\isachardollar}{\kern0pt}{\isachardollar}{\kern0pt}\ {\isacharparenleft}{\kern0pt}{\isadigit{3}}{\isacharcomma}{\kern0pt}{\isadigit{1}}{\isacharparenright}{\kern0pt}{\isacharparenright}{\kern0pt}{\isacharparenright}{\kern0pt}\ {\isacharplus}{\kern0pt}\ \isanewline
\ \ \ \ \ \ \ \ \ \ \ \ \ \ \ \ \ \ \ \ \ \ \ \ \ \ \ \ \ \ \ \ {\isacharparenleft}{\kern0pt}control{\isadigit{2}}\ U{\isacharparenright}{\kern0pt}\ {\isachardollar}{\kern0pt}{\isachardollar}{\kern0pt}\ {\isacharparenleft}{\kern0pt}{\isadigit{1}}{\isacharcomma}{\kern0pt}{\isadigit{3}}{\isacharparenright}{\kern0pt}\ {\isacharasterisk}{\kern0pt}\ {\isacharparenleft}{\kern0pt}cnj\ {\isacharparenleft}{\kern0pt}{\isacharparenleft}{\kern0pt}control{\isadigit{2}}\ U{\isacharparenright}{\kern0pt}\ {\isachardollar}{\kern0pt}{\isachardollar}{\kern0pt}\ {\isacharparenleft}{\kern0pt}{\isadigit{3}}{\isacharcomma}{\kern0pt}{\isadigit{3}}{\isacharparenright}{\kern0pt}{\isacharparenright}{\kern0pt}{\isacharparenright}{\kern0pt}{\isachardoublequoteclose}\isanewline
\ \ \ \ \ \ \ \ \ \ \ \ \ \ \ \ \ \ \isacommand{using}\isamarkupfalse%
\ dagger{\isacharunderscore}{\kern0pt}def\isanewline
\ \ \ \ \ \ \ \ \ \ \ \ \ \ \ \ \ \ \isacommand{by}\isamarkupfalse%
\ {\isacharparenleft}{\kern0pt}smt\ {\isacharparenleft}{\kern0pt}verit{\isacharcomma}{\kern0pt}\ best{\isacharparenright}{\kern0pt}\ One{\isacharunderscore}{\kern0pt}nat{\isacharunderscore}{\kern0pt}def\ Suc{\isacharunderscore}{\kern0pt}{\isadigit{1}}\ add{\isachardot}{\kern0pt}commute\ add{\isacharunderscore}{\kern0pt}Suc{\isacharunderscore}{\kern0pt}right\ carrier{\isacharunderscore}{\kern0pt}matD{\isacharparenleft}{\kern0pt}{\isadigit{1}}{\isacharparenright}{\kern0pt}\isanewline
\ \ \ \ \ \ \ \ \ \ \ \ \ \ \ \ \ \ \ \ \ \ carrier{\isacharunderscore}{\kern0pt}matD{\isacharparenleft}{\kern0pt}{\isadigit{2}}{\isacharparenright}{\kern0pt}\ control{\isadigit{2}}{\isacharunderscore}{\kern0pt}carrier{\isacharunderscore}{\kern0pt}mat\ i{\isadigit{1}}\ i{\isadigit{4}}\ index{\isacharunderscore}{\kern0pt}mat{\isacharparenleft}{\kern0pt}{\isadigit{1}}{\isacharparenright}{\kern0pt}\ lessI\ numeral{\isacharunderscore}{\kern0pt}{\isadigit{3}}{\isacharunderscore}{\kern0pt}eq{\isacharunderscore}{\kern0pt}{\isadigit{3}}\ \isanewline
\ \ \ \ \ \ \ \ \ \ \ \ \ \ \ \ \ \ \ \ \ \ numeral{\isacharunderscore}{\kern0pt}Bit{\isadigit{0}}\ plus{\isacharunderscore}{\kern0pt}{\isadigit{1}}{\isacharunderscore}{\kern0pt}eq{\isacharunderscore}{\kern0pt}Suc\ prod{\isachardot}{\kern0pt}simps{\isacharparenleft}{\kern0pt}{\isadigit{2}}{\isacharparenright}{\kern0pt}{\isacharparenright}{\kern0pt}\isanewline
\ \ \ \ \ \ \ \ \ \ \ \ \ \ \isacommand{also}\isamarkupfalse%
\ \isacommand{have}\isamarkupfalse%
\ {\isachardoublequoteopen}{\isasymdots}\ {\isacharequal}{\kern0pt}\ U\ {\isachardollar}{\kern0pt}{\isachardollar}{\kern0pt}\ {\isacharparenleft}{\kern0pt}{\isadigit{0}}{\isacharcomma}{\kern0pt}{\isadigit{0}}{\isacharparenright}{\kern0pt}\ {\isacharasterisk}{\kern0pt}\ {\isacharparenleft}{\kern0pt}cnj\ {\isacharparenleft}{\kern0pt}U\ {\isachardollar}{\kern0pt}{\isachardollar}{\kern0pt}\ {\isacharparenleft}{\kern0pt}{\isadigit{1}}{\isacharcomma}{\kern0pt}{\isadigit{0}}{\isacharparenright}{\kern0pt}{\isacharparenright}{\kern0pt}{\isacharparenright}{\kern0pt}\ {\isacharplus}{\kern0pt}\isanewline
\ \ \ \ \ \ \ \ \ \ \ \ \ \ \ \ \ \ \ \ \ \ \ \ \ \ \ \ \ \ U\ {\isachardollar}{\kern0pt}{\isachardollar}{\kern0pt}\ {\isacharparenleft}{\kern0pt}{\isadigit{0}}{\isacharcomma}{\kern0pt}{\isadigit{1}}{\isacharparenright}{\kern0pt}\ {\isacharasterisk}{\kern0pt}\ {\isacharparenleft}{\kern0pt}cnj\ {\isacharparenleft}{\kern0pt}U\ {\isachardollar}{\kern0pt}{\isachardollar}{\kern0pt}\ {\isacharparenleft}{\kern0pt}{\isadigit{1}}{\isacharcomma}{\kern0pt}{\isadigit{1}}{\isacharparenright}{\kern0pt}{\isacharparenright}{\kern0pt}{\isacharparenright}{\kern0pt}{\isachardoublequoteclose}\isanewline
\ \ \ \ \ \ \ \ \ \ \ \ \ \ \ \ \isacommand{using}\isamarkupfalse%
\ control{\isadigit{2}}{\isacharunderscore}{\kern0pt}def\ index{\isacharunderscore}{\kern0pt}mat{\isacharunderscore}{\kern0pt}of{\isacharunderscore}{\kern0pt}cols{\isacharunderscore}{\kern0pt}list\ \isacommand{by}\isamarkupfalse%
\ simp\isanewline
\ \ \ \ \ \ \ \ \ \ \ \ \ \ \isacommand{also}\isamarkupfalse%
\ \isacommand{have}\isamarkupfalse%
\ {\isachardoublequoteopen}{\isasymdots}\ {\isacharequal}{\kern0pt}\ {\isacharparenleft}{\kern0pt}U\ {\isachardollar}{\kern0pt}{\isachardollar}{\kern0pt}\ {\isacharparenleft}{\kern0pt}{\isadigit{0}}{\isacharcomma}{\kern0pt}{\isadigit{0}}{\isacharparenright}{\kern0pt}{\isacharparenright}{\kern0pt}\ {\isacharasterisk}{\kern0pt}\ {\isacharparenleft}{\kern0pt}{\isacharparenleft}{\kern0pt}U\isactrlsup {\isasymdagger}{\isacharparenright}{\kern0pt}\ {\isachardollar}{\kern0pt}{\isachardollar}{\kern0pt}\ {\isacharparenleft}{\kern0pt}{\isadigit{0}}{\isacharcomma}{\kern0pt}{\isadigit{1}}{\isacharparenright}{\kern0pt}{\isacharparenright}{\kern0pt}\ {\isacharplus}{\kern0pt}\isanewline
\ \ \ \ \ \ \ \ \ \ \ \ \ \ \ \ \ \ \ \ \ \ \ \ \ \ \ \ \ \ {\isacharparenleft}{\kern0pt}U\ {\isachardollar}{\kern0pt}{\isachardollar}{\kern0pt}\ {\isacharparenleft}{\kern0pt}{\isadigit{0}}{\isacharcomma}{\kern0pt}{\isadigit{1}}{\isacharparenright}{\kern0pt}{\isacharparenright}{\kern0pt}\ {\isacharasterisk}{\kern0pt}\ {\isacharparenleft}{\kern0pt}{\isacharparenleft}{\kern0pt}U\isactrlsup {\isasymdagger}{\isacharparenright}{\kern0pt}\ {\isachardollar}{\kern0pt}{\isachardollar}{\kern0pt}\ {\isacharparenleft}{\kern0pt}{\isadigit{1}}{\isacharcomma}{\kern0pt}{\isadigit{1}}{\isacharparenright}{\kern0pt}{\isacharparenright}{\kern0pt}{\isachardoublequoteclose}\isanewline
\ \ \ \ \ \ \ \ \ \ \ \ \ \ \ \ \isacommand{using}\isamarkupfalse%
\ dagger{\isacharunderscore}{\kern0pt}def\ assms{\isacharparenleft}{\kern0pt}{\isadigit{1}}{\isacharparenright}{\kern0pt}\ gate{\isacharunderscore}{\kern0pt}def\ \isacommand{by}\isamarkupfalse%
\ force\isanewline
\ \ \ \ \ \ \ \ \ \ \ \ \ \ \isacommand{also}\isamarkupfalse%
\ \isacommand{have}\isamarkupfalse%
\ {\isachardoublequoteopen}{\isasymdots}\ {\isacharequal}{\kern0pt}\ {\isacharparenleft}{\kern0pt}U\ {\isacharasterisk}{\kern0pt}\ {\isacharparenleft}{\kern0pt}U\isactrlsup {\isasymdagger}{\isacharparenright}{\kern0pt}{\isacharparenright}{\kern0pt}\ {\isachardollar}{\kern0pt}{\isachardollar}{\kern0pt}\ {\isacharparenleft}{\kern0pt}{\isadigit{0}}{\isacharcomma}{\kern0pt}{\isadigit{1}}{\isacharparenright}{\kern0pt}{\isachardoublequoteclose}\ \isanewline
\ \ \ \ \ \ \ \ \ \ \ \ \ \ \ \ \isacommand{using}\isamarkupfalse%
\ times{\isacharunderscore}{\kern0pt}mat{\isacharunderscore}{\kern0pt}def\ assms{\isacharparenleft}{\kern0pt}{\isadigit{1}}{\isacharparenright}{\kern0pt}\ gate{\isacharunderscore}{\kern0pt}carrier{\isacharunderscore}{\kern0pt}mat\ sumof{\isadigit{2}}\isanewline
\ \ \ \ \ \ \ \ \ \ \ \ \ \ \ \ \isacommand{by}\isamarkupfalse%
\ {\isacharparenleft}{\kern0pt}smt\ {\isacharparenleft}{\kern0pt}z{\isadigit{3}}{\isacharparenright}{\kern0pt}\ Suc{\isacharunderscore}{\kern0pt}{\isadigit{1}}\ carrier{\isacharunderscore}{\kern0pt}matD{\isacharparenleft}{\kern0pt}{\isadigit{2}}{\isacharparenright}{\kern0pt}\ dagger{\isacharunderscore}{\kern0pt}def\ dim{\isacharunderscore}{\kern0pt}col{\isacharunderscore}{\kern0pt}mat{\isacharparenleft}{\kern0pt}{\isadigit{1}}{\isacharparenright}{\kern0pt}\ dim{\isacharunderscore}{\kern0pt}row{\isacharunderscore}{\kern0pt}of{\isacharunderscore}{\kern0pt}dagger\ \isanewline
\ \ \ \ \ \ \ \ \ \ \ \ \ \ \ \ \ \ \ \ gate{\isachardot}{\kern0pt}dim{\isacharunderscore}{\kern0pt}row\ index{\isacharunderscore}{\kern0pt}matrix{\isacharunderscore}{\kern0pt}prod\ lessI\ pos{\isadigit{2}}\ power{\isacharunderscore}{\kern0pt}one{\isacharunderscore}{\kern0pt}right{\isacharparenright}{\kern0pt}\isanewline
\ \ \ \ \ \ \ \ \ \ \ \ \ \ \isacommand{also}\isamarkupfalse%
\ \isacommand{have}\isamarkupfalse%
\ {\isachardoublequoteopen}{\isasymdots}\ {\isacharequal}{\kern0pt}\ {\isacharparenleft}{\kern0pt}{\isadigit{1}}\isactrlsub m\ {\isadigit{2}}{\isacharparenright}{\kern0pt}\ {\isachardollar}{\kern0pt}{\isachardollar}{\kern0pt}\ {\isacharparenleft}{\kern0pt}{\isadigit{0}}{\isacharcomma}{\kern0pt}{\isadigit{1}}{\isacharparenright}{\kern0pt}{\isachardoublequoteclose}\ \isacommand{using}\isamarkupfalse%
\ assms{\isacharparenleft}{\kern0pt}{\isadigit{1}}{\isacharparenright}{\kern0pt}\ gate{\isacharunderscore}{\kern0pt}def\ unitary{\isacharunderscore}{\kern0pt}def\ \isacommand{by}\isamarkupfalse%
\ auto\isanewline
\ \ \ \ \ \ \ \ \ \ \ \ \ \ \isacommand{also}\isamarkupfalse%
\ \isacommand{have}\isamarkupfalse%
\ {\isachardoublequoteopen}{\isasymdots}\ {\isacharequal}{\kern0pt}\ {\isadigit{0}}{\isachardoublequoteclose}\ \isacommand{by}\isamarkupfalse%
\ auto\isanewline
\ \ \ \ \ \ \ \ \ \ \ \ \ \ \isacommand{also}\isamarkupfalse%
\ \isacommand{have}\isamarkupfalse%
\ {\isachardoublequoteopen}{\isasymdots}\ {\isacharequal}{\kern0pt}\ {\isadigit{1}}\isactrlsub m\ {\isadigit{4}}\ {\isachardollar}{\kern0pt}{\isachardollar}{\kern0pt}\ {\isacharparenleft}{\kern0pt}{\isadigit{1}}{\isacharcomma}{\kern0pt}{\isadigit{3}}{\isacharparenright}{\kern0pt}{\isachardoublequoteclose}\ \isacommand{by}\isamarkupfalse%
\ simp\isanewline
\ \ \ \ \ \ \ \ \ \ \ \ \ \ \isacommand{finally}\isamarkupfalse%
\ \isacommand{show}\isamarkupfalse%
\ {\isacharquery}{\kern0pt}thesis\ \isacommand{using}\isamarkupfalse%
\ i{\isadigit{1}}\ j{\isadigit{3}}\ \isacommand{by}\isamarkupfalse%
\ simp\isanewline
\ \ \ \ \ \ \ \ \ \ \ \ \isacommand{qed}\isamarkupfalse%
\isanewline
\ \ \ \ \ \ \ \ \ \ \isacommand{qed}\isamarkupfalse%
\isanewline
\ \ \ \ \ \ \ \ \isacommand{qed}\isamarkupfalse%
\isanewline
\ \ \ \ \ \ \isacommand{qed}\isamarkupfalse%
\isanewline
\ \ \ \ \isacommand{next}\isamarkupfalse%
\isanewline
\ \ \ \ \ \ \isacommand{assume}\isamarkupfalse%
\ il{\isadigit{2}}{\isacharcolon}{\kern0pt}{\isachardoublequoteopen}i\ {\isacharequal}{\kern0pt}\ {\isadigit{2}}\ {\isasymor}\ i\ {\isacharequal}{\kern0pt}\ {\isadigit{3}}{\isachardoublequoteclose}\isanewline
\ \ \ \ \ \ \isacommand{show}\isamarkupfalse%
\ {\isachardoublequoteopen}{\isacharparenleft}{\kern0pt}control{\isadigit{2}}\ U\ {\isacharasterisk}{\kern0pt}\ {\isacharparenleft}{\kern0pt}{\isacharparenleft}{\kern0pt}control{\isadigit{2}}\ U{\isacharparenright}{\kern0pt}\isactrlsup {\isasymdagger}{\isacharparenright}{\kern0pt}{\isacharparenright}{\kern0pt}\ {\isachardollar}{\kern0pt}{\isachardollar}{\kern0pt}\ {\isacharparenleft}{\kern0pt}i{\isacharcomma}{\kern0pt}\ j{\isacharparenright}{\kern0pt}\ {\isacharequal}{\kern0pt}\ {\isadigit{1}}\isactrlsub m\ {\isadigit{4}}\ {\isachardollar}{\kern0pt}{\isachardollar}{\kern0pt}\ {\isacharparenleft}{\kern0pt}i{\isacharcomma}{\kern0pt}\ j{\isacharparenright}{\kern0pt}{\isachardoublequoteclose}\isanewline
\ \ \ \ \ \ \isacommand{proof}\isamarkupfalse%
\ {\isacharparenleft}{\kern0pt}rule\ disjE{\isacharparenright}{\kern0pt}\isanewline
\ \ \ \ \ \ \ \ \isacommand{show}\isamarkupfalse%
\ {\isachardoublequoteopen}i\ {\isacharequal}{\kern0pt}\ {\isadigit{2}}\ {\isasymor}\ i\ {\isacharequal}{\kern0pt}\ {\isadigit{3}}{\isachardoublequoteclose}\ \isacommand{using}\isamarkupfalse%
\ il{\isadigit{2}}\ \isacommand{by}\isamarkupfalse%
\ this\isanewline
\ \ \ \ \ \ \isacommand{next}\isamarkupfalse%
\isanewline
\ \ \ \ \ \ \ \ \isacommand{assume}\isamarkupfalse%
\ i{\isadigit{2}}{\isacharcolon}{\kern0pt}{\isachardoublequoteopen}i\ {\isacharequal}{\kern0pt}\ {\isadigit{2}}{\isachardoublequoteclose}\isanewline
\ \ \ \ \ \ \ \ \isacommand{show}\isamarkupfalse%
\ {\isachardoublequoteopen}{\isacharparenleft}{\kern0pt}control{\isadigit{2}}\ U\ {\isacharasterisk}{\kern0pt}\ {\isacharparenleft}{\kern0pt}{\isacharparenleft}{\kern0pt}control{\isadigit{2}}\ U{\isacharparenright}{\kern0pt}\isactrlsup {\isasymdagger}{\isacharparenright}{\kern0pt}{\isacharparenright}{\kern0pt}\ {\isachardollar}{\kern0pt}{\isachardollar}{\kern0pt}\ {\isacharparenleft}{\kern0pt}i{\isacharcomma}{\kern0pt}\ j{\isacharparenright}{\kern0pt}\ {\isacharequal}{\kern0pt}\ {\isadigit{1}}\isactrlsub m\ {\isadigit{4}}\ {\isachardollar}{\kern0pt}{\isachardollar}{\kern0pt}\ {\isacharparenleft}{\kern0pt}i{\isacharcomma}{\kern0pt}\ j{\isacharparenright}{\kern0pt}{\isachardoublequoteclose}\isanewline
\ \ \ \ \ \ \ \ \isacommand{proof}\isamarkupfalse%
\ {\isacharparenleft}{\kern0pt}rule\ disjE{\isacharparenright}{\kern0pt}\isanewline
\ \ \ \ \ \ \ \ \ \ \isacommand{show}\isamarkupfalse%
\ {\isachardoublequoteopen}j\ {\isacharequal}{\kern0pt}\ {\isadigit{0}}\ {\isasymor}\ j\ {\isacharequal}{\kern0pt}\ {\isadigit{1}}\ {\isasymor}\ j\ {\isacharequal}{\kern0pt}\ {\isadigit{2}}\ {\isasymor}\ j\ {\isacharequal}{\kern0pt}\ {\isadigit{3}}{\isachardoublequoteclose}\ \isacommand{using}\isamarkupfalse%
\ j{\isadigit{4}}\ \isacommand{by}\isamarkupfalse%
\ auto\isanewline
\ \ \ \ \ \ \ \ \isacommand{next}\isamarkupfalse%
\isanewline
\ \ \ \ \ \ \ \ \ \ \isacommand{assume}\isamarkupfalse%
\ j{\isadigit{0}}{\isacharcolon}{\kern0pt}{\isachardoublequoteopen}j\ {\isacharequal}{\kern0pt}\ {\isadigit{0}}{\isachardoublequoteclose}\isanewline
\ \ \ \ \ \ \ \ \ \ \isacommand{show}\isamarkupfalse%
\ {\isachardoublequoteopen}{\isacharparenleft}{\kern0pt}control{\isadigit{2}}\ U\ {\isacharasterisk}{\kern0pt}\ {\isacharparenleft}{\kern0pt}{\isacharparenleft}{\kern0pt}control{\isadigit{2}}\ U{\isacharparenright}{\kern0pt}\isactrlsup {\isasymdagger}{\isacharparenright}{\kern0pt}{\isacharparenright}{\kern0pt}\ {\isachardollar}{\kern0pt}{\isachardollar}{\kern0pt}\ {\isacharparenleft}{\kern0pt}i{\isacharcomma}{\kern0pt}\ j{\isacharparenright}{\kern0pt}\ {\isacharequal}{\kern0pt}\ {\isadigit{1}}\isactrlsub m\ {\isadigit{4}}\ {\isachardollar}{\kern0pt}{\isachardollar}{\kern0pt}\ {\isacharparenleft}{\kern0pt}i{\isacharcomma}{\kern0pt}\ j{\isacharparenright}{\kern0pt}{\isachardoublequoteclose}\isanewline
\ \ \ \ \ \ \ \ \ \ \isacommand{proof}\isamarkupfalse%
\ {\isacharminus}{\kern0pt}\isanewline
\ \ \ \ \ \ \ \ \ \ \ \ \isacommand{have}\isamarkupfalse%
\ {\isachardoublequoteopen}{\isacharparenleft}{\kern0pt}control{\isadigit{2}}\ U\ {\isacharasterisk}{\kern0pt}\ {\isacharparenleft}{\kern0pt}{\isacharparenleft}{\kern0pt}control{\isadigit{2}}\ U{\isacharparenright}{\kern0pt}\isactrlsup {\isasymdagger}{\isacharparenright}{\kern0pt}{\isacharparenright}{\kern0pt}\ {\isachardollar}{\kern0pt}{\isachardollar}{\kern0pt}\ {\isacharparenleft}{\kern0pt}{\isadigit{2}}{\isacharcomma}{\kern0pt}{\isadigit{0}}{\isacharparenright}{\kern0pt}\ {\isacharequal}{\kern0pt}\ \isanewline
\ \ \ \ \ \ \ \ \ \ \ \ \ \ \ \ \ \ {\isacharparenleft}{\kern0pt}control{\isadigit{2}}\ U{\isacharparenright}{\kern0pt}\ {\isachardollar}{\kern0pt}{\isachardollar}{\kern0pt}\ {\isacharparenleft}{\kern0pt}{\isadigit{2}}{\isacharcomma}{\kern0pt}{\isadigit{0}}{\isacharparenright}{\kern0pt}\ {\isacharasterisk}{\kern0pt}\ {\isacharparenleft}{\kern0pt}{\isacharparenleft}{\kern0pt}control{\isadigit{2}}\ U{\isacharparenright}{\kern0pt}\isactrlsup {\isasymdagger}{\isacharparenright}{\kern0pt}\ {\isachardollar}{\kern0pt}{\isachardollar}{\kern0pt}\ {\isacharparenleft}{\kern0pt}{\isadigit{0}}{\isacharcomma}{\kern0pt}{\isadigit{0}}{\isacharparenright}{\kern0pt}\ {\isacharplus}{\kern0pt}\isanewline
\ \ \ \ \ \ \ \ \ \ \ \ \ \ \ \ \ \ {\isacharparenleft}{\kern0pt}control{\isadigit{2}}\ U{\isacharparenright}{\kern0pt}\ {\isachardollar}{\kern0pt}{\isachardollar}{\kern0pt}\ {\isacharparenleft}{\kern0pt}{\isadigit{2}}{\isacharcomma}{\kern0pt}{\isadigit{1}}{\isacharparenright}{\kern0pt}\ {\isacharasterisk}{\kern0pt}\ {\isacharparenleft}{\kern0pt}{\isacharparenleft}{\kern0pt}control{\isadigit{2}}\ U{\isacharparenright}{\kern0pt}\isactrlsup {\isasymdagger}{\isacharparenright}{\kern0pt}\ {\isachardollar}{\kern0pt}{\isachardollar}{\kern0pt}\ {\isacharparenleft}{\kern0pt}{\isadigit{1}}{\isacharcomma}{\kern0pt}{\isadigit{0}}{\isacharparenright}{\kern0pt}\ {\isacharplus}{\kern0pt}\isanewline
\ \ \ \ \ \ \ \ \ \ \ \ \ \ \ \ \ \ {\isacharparenleft}{\kern0pt}control{\isadigit{2}}\ U{\isacharparenright}{\kern0pt}\ {\isachardollar}{\kern0pt}{\isachardollar}{\kern0pt}\ {\isacharparenleft}{\kern0pt}{\isadigit{2}}{\isacharcomma}{\kern0pt}{\isadigit{2}}{\isacharparenright}{\kern0pt}\ {\isacharasterisk}{\kern0pt}\ {\isacharparenleft}{\kern0pt}{\isacharparenleft}{\kern0pt}control{\isadigit{2}}\ U{\isacharparenright}{\kern0pt}\isactrlsup {\isasymdagger}{\isacharparenright}{\kern0pt}\ {\isachardollar}{\kern0pt}{\isachardollar}{\kern0pt}\ {\isacharparenleft}{\kern0pt}{\isadigit{2}}{\isacharcomma}{\kern0pt}{\isadigit{0}}{\isacharparenright}{\kern0pt}\ {\isacharplus}{\kern0pt}\isanewline
\ \ \ \ \ \ \ \ \ \ \ \ \ \ \ \ \ \ {\isacharparenleft}{\kern0pt}control{\isadigit{2}}\ U{\isacharparenright}{\kern0pt}\ {\isachardollar}{\kern0pt}{\isachardollar}{\kern0pt}\ {\isacharparenleft}{\kern0pt}{\isadigit{2}}{\isacharcomma}{\kern0pt}{\isadigit{3}}{\isacharparenright}{\kern0pt}\ {\isacharasterisk}{\kern0pt}\ {\isacharparenleft}{\kern0pt}{\isacharparenleft}{\kern0pt}control{\isadigit{2}}\ U{\isacharparenright}{\kern0pt}\isactrlsup {\isasymdagger}{\isacharparenright}{\kern0pt}\ {\isachardollar}{\kern0pt}{\isachardollar}{\kern0pt}\ {\isacharparenleft}{\kern0pt}{\isadigit{3}}{\isacharcomma}{\kern0pt}{\isadigit{0}}{\isacharparenright}{\kern0pt}{\isachardoublequoteclose}\isanewline
\ \ \ \ \ \ \ \ \ \ \ \ \ \ \isacommand{using}\isamarkupfalse%
\ times{\isacharunderscore}{\kern0pt}mat{\isacharunderscore}{\kern0pt}def\ sumof{\isadigit{4}}\isanewline
\ \ \ \ \ \ \ \ \ \ \ \ \ \ \isacommand{by}\isamarkupfalse%
\ {\isacharparenleft}{\kern0pt}smt\ {\isacharparenleft}{\kern0pt}z{\isadigit{3}}{\isacharparenright}{\kern0pt}\ carrier{\isacharunderscore}{\kern0pt}matD{\isacharparenleft}{\kern0pt}{\isadigit{1}}{\isacharparenright}{\kern0pt}\ carrier{\isacharunderscore}{\kern0pt}matD{\isacharparenleft}{\kern0pt}{\isadigit{2}}{\isacharparenright}{\kern0pt}\ control{\isadigit{2}}{\isacharunderscore}{\kern0pt}carrier{\isacharunderscore}{\kern0pt}mat\ dim{\isacharunderscore}{\kern0pt}col{\isacharunderscore}{\kern0pt}of{\isacharunderscore}{\kern0pt}dagger\ \isanewline
\ \ \ \ \ \ \ \ \ \ \ \ \ \ \ \ \ \ \ \ dim{\isacharunderscore}{\kern0pt}row{\isacharunderscore}{\kern0pt}of{\isacharunderscore}{\kern0pt}dagger\ i{\isadigit{2}}\ i{\isadigit{4}}\ index{\isacharunderscore}{\kern0pt}matrix{\isacharunderscore}{\kern0pt}prod\ j{\isadigit{0}}\ j{\isadigit{4}}{\isacharparenright}{\kern0pt}\isanewline
\ \ \ \ \ \ \ \ \ \ \ \ \isacommand{also}\isamarkupfalse%
\ \isacommand{have}\isamarkupfalse%
\ {\isachardoublequoteopen}{\isasymdots}\ {\isacharequal}{\kern0pt}\ {\isacharparenleft}{\kern0pt}{\isacharparenleft}{\kern0pt}control{\isadigit{2}}\ U{\isacharparenright}{\kern0pt}\isactrlsup {\isasymdagger}{\isacharparenright}{\kern0pt}\ {\isachardollar}{\kern0pt}{\isachardollar}{\kern0pt}\ {\isacharparenleft}{\kern0pt}{\isadigit{2}}{\isacharcomma}{\kern0pt}{\isadigit{0}}{\isacharparenright}{\kern0pt}{\isachardoublequoteclose}\isanewline
\ \ \ \ \ \ \ \ \ \ \ \ \ \ \isacommand{using}\isamarkupfalse%
\ control{\isadigit{2}}{\isacharunderscore}{\kern0pt}def\ index{\isacharunderscore}{\kern0pt}mat{\isacharunderscore}{\kern0pt}of{\isacharunderscore}{\kern0pt}cols{\isacharunderscore}{\kern0pt}list\ \isacommand{by}\isamarkupfalse%
\ force\isanewline
\ \ \ \ \ \ \ \ \ \ \ \ \isacommand{also}\isamarkupfalse%
\ \isacommand{have}\isamarkupfalse%
\ {\isachardoublequoteopen}{\isasymdots}\ {\isacharequal}{\kern0pt}\ cnj\ {\isacharparenleft}{\kern0pt}{\isacharparenleft}{\kern0pt}control{\isadigit{2}}\ U{\isacharparenright}{\kern0pt}\ {\isachardollar}{\kern0pt}{\isachardollar}{\kern0pt}\ {\isacharparenleft}{\kern0pt}{\isadigit{0}}{\isacharcomma}{\kern0pt}{\isadigit{2}}{\isacharparenright}{\kern0pt}{\isacharparenright}{\kern0pt}{\isachardoublequoteclose}\isanewline
\ \ \ \ \ \ \ \ \ \ \ \ \ \ \isacommand{using}\isamarkupfalse%
\ dagger{\isacharunderscore}{\kern0pt}def\ \isanewline
\ \ \ \ \ \ \ \ \ \ \ \ \ \ \isacommand{by}\isamarkupfalse%
\ {\isacharparenleft}{\kern0pt}metis\ carrier{\isacharunderscore}{\kern0pt}matD{\isacharparenleft}{\kern0pt}{\isadigit{1}}{\isacharparenright}{\kern0pt}\ carrier{\isacharunderscore}{\kern0pt}matD{\isacharparenleft}{\kern0pt}{\isadigit{2}}{\isacharparenright}{\kern0pt}\ control{\isadigit{2}}{\isacharunderscore}{\kern0pt}carrier{\isacharunderscore}{\kern0pt}mat\ i{\isadigit{2}}\ i{\isadigit{4}}\ index{\isacharunderscore}{\kern0pt}mat{\isacharparenleft}{\kern0pt}{\isadigit{1}}{\isacharparenright}{\kern0pt}\ \isanewline
\ \ \ \ \ \ \ \ \ \ \ \ \ \ \ \ \ \ j{\isadigit{0}}\ j{\isadigit{4}}\ prod{\isachardot}{\kern0pt}simps{\isacharparenleft}{\kern0pt}{\isadigit{2}}{\isacharparenright}{\kern0pt}{\isacharparenright}{\kern0pt}\isanewline
\ \ \ \ \ \ \ \ \ \ \ \ \isacommand{also}\isamarkupfalse%
\ \isacommand{have}\isamarkupfalse%
\ {\isachardoublequoteopen}{\isasymdots}\ {\isacharequal}{\kern0pt}\ {\isadigit{0}}{\isachardoublequoteclose}\ \isacommand{using}\isamarkupfalse%
\ control{\isadigit{2}}{\isacharunderscore}{\kern0pt}def\ index{\isacharunderscore}{\kern0pt}mat{\isacharunderscore}{\kern0pt}of{\isacharunderscore}{\kern0pt}cols{\isacharunderscore}{\kern0pt}list\ \isacommand{by}\isamarkupfalse%
\ auto\isanewline
\ \ \ \ \ \ \ \ \ \ \ \ \isacommand{also}\isamarkupfalse%
\ \isacommand{have}\isamarkupfalse%
\ {\isachardoublequoteopen}{\isasymdots}\ {\isacharequal}{\kern0pt}\ {\isadigit{1}}\isactrlsub m\ {\isadigit{4}}\ {\isachardollar}{\kern0pt}{\isachardollar}{\kern0pt}\ {\isacharparenleft}{\kern0pt}{\isadigit{2}}{\isacharcomma}{\kern0pt}{\isadigit{0}}{\isacharparenright}{\kern0pt}{\isachardoublequoteclose}\ \isacommand{by}\isamarkupfalse%
\ simp\isanewline
\ \ \ \ \ \ \ \ \ \ \ \ \isacommand{finally}\isamarkupfalse%
\ \isacommand{show}\isamarkupfalse%
\ {\isacharquery}{\kern0pt}thesis\ \isacommand{using}\isamarkupfalse%
\ i{\isadigit{2}}\ j{\isadigit{0}}\ \isacommand{by}\isamarkupfalse%
\ simp\isanewline
\ \ \ \ \ \ \ \ \ \ \isacommand{qed}\isamarkupfalse%
\isanewline
\ \ \ \ \ \ \ \ \isacommand{next}\isamarkupfalse%
\isanewline
\ \ \ \ \ \ \ \ \ \ \isacommand{assume}\isamarkupfalse%
\ jl{\isadigit{3}}{\isacharcolon}{\kern0pt}{\isachardoublequoteopen}j\ {\isacharequal}{\kern0pt}\ {\isadigit{1}}\ {\isasymor}\ j\ {\isacharequal}{\kern0pt}\ {\isadigit{2}}\ {\isasymor}\ j\ {\isacharequal}{\kern0pt}\ {\isadigit{3}}{\isachardoublequoteclose}\isanewline
\ \ \ \ \ \ \ \ \ \ \isacommand{show}\isamarkupfalse%
\ {\isachardoublequoteopen}{\isacharparenleft}{\kern0pt}control{\isadigit{2}}\ U\ {\isacharasterisk}{\kern0pt}\ {\isacharparenleft}{\kern0pt}{\isacharparenleft}{\kern0pt}control{\isadigit{2}}\ U{\isacharparenright}{\kern0pt}\isactrlsup {\isasymdagger}{\isacharparenright}{\kern0pt}{\isacharparenright}{\kern0pt}\ {\isachardollar}{\kern0pt}{\isachardollar}{\kern0pt}\ {\isacharparenleft}{\kern0pt}i{\isacharcomma}{\kern0pt}\ j{\isacharparenright}{\kern0pt}\ {\isacharequal}{\kern0pt}\ {\isadigit{1}}\isactrlsub m\ {\isadigit{4}}\ {\isachardollar}{\kern0pt}{\isachardollar}{\kern0pt}\ {\isacharparenleft}{\kern0pt}i{\isacharcomma}{\kern0pt}\ j{\isacharparenright}{\kern0pt}{\isachardoublequoteclose}\isanewline
\ \ \ \ \ \ \ \ \ \ \isacommand{proof}\isamarkupfalse%
\ {\isacharparenleft}{\kern0pt}rule\ disjE{\isacharparenright}{\kern0pt}\isanewline
\ \ \ \ \ \ \ \ \ \ \ \ \isacommand{show}\isamarkupfalse%
\ {\isachardoublequoteopen}j\ {\isacharequal}{\kern0pt}\ {\isadigit{1}}\ {\isasymor}\ j\ {\isacharequal}{\kern0pt}\ {\isadigit{2}}\ {\isasymor}\ j\ {\isacharequal}{\kern0pt}\ {\isadigit{3}}{\isachardoublequoteclose}\ \isacommand{using}\isamarkupfalse%
\ jl{\isadigit{3}}\ \isacommand{by}\isamarkupfalse%
\ this\isanewline
\ \ \ \ \ \ \ \ \ \ \isacommand{next}\isamarkupfalse%
\isanewline
\ \ \ \ \ \ \ \ \ \ \ \ \isacommand{assume}\isamarkupfalse%
\ j{\isadigit{1}}{\isacharcolon}{\kern0pt}{\isachardoublequoteopen}j\ {\isacharequal}{\kern0pt}\ {\isadigit{1}}{\isachardoublequoteclose}\isanewline
\ \ \ \ \ \ \ \ \ \ \ \ \isacommand{show}\isamarkupfalse%
\ {\isachardoublequoteopen}{\isacharparenleft}{\kern0pt}control{\isadigit{2}}\ U\ {\isacharasterisk}{\kern0pt}\ {\isacharparenleft}{\kern0pt}{\isacharparenleft}{\kern0pt}control{\isadigit{2}}\ U{\isacharparenright}{\kern0pt}\isactrlsup {\isasymdagger}{\isacharparenright}{\kern0pt}{\isacharparenright}{\kern0pt}\ {\isachardollar}{\kern0pt}{\isachardollar}{\kern0pt}\ {\isacharparenleft}{\kern0pt}i{\isacharcomma}{\kern0pt}\ j{\isacharparenright}{\kern0pt}\ {\isacharequal}{\kern0pt}\ {\isadigit{1}}\isactrlsub m\ {\isadigit{4}}\ {\isachardollar}{\kern0pt}{\isachardollar}{\kern0pt}\ {\isacharparenleft}{\kern0pt}i{\isacharcomma}{\kern0pt}\ j{\isacharparenright}{\kern0pt}{\isachardoublequoteclose}\isanewline
\ \ \ \ \ \ \ \ \ \ \ \ \isacommand{proof}\isamarkupfalse%
\ {\isacharminus}{\kern0pt}\isanewline
\ \ \ \ \ \ \ \ \ \ \ \ \ \ \isacommand{have}\isamarkupfalse%
\ {\isachardoublequoteopen}{\isacharparenleft}{\kern0pt}control{\isadigit{2}}\ U\ {\isacharasterisk}{\kern0pt}\ {\isacharparenleft}{\kern0pt}{\isacharparenleft}{\kern0pt}control{\isadigit{2}}\ U{\isacharparenright}{\kern0pt}\isactrlsup {\isasymdagger}{\isacharparenright}{\kern0pt}{\isacharparenright}{\kern0pt}\ {\isachardollar}{\kern0pt}{\isachardollar}{\kern0pt}\ {\isacharparenleft}{\kern0pt}{\isadigit{2}}{\isacharcomma}{\kern0pt}{\isadigit{1}}{\isacharparenright}{\kern0pt}\ {\isacharequal}{\kern0pt}\ \isanewline
\ \ \ \ \ \ \ \ \ \ \ \ \ \ \ \ \ \ {\isacharparenleft}{\kern0pt}control{\isadigit{2}}\ U{\isacharparenright}{\kern0pt}\ {\isachardollar}{\kern0pt}{\isachardollar}{\kern0pt}\ {\isacharparenleft}{\kern0pt}{\isadigit{2}}{\isacharcomma}{\kern0pt}{\isadigit{0}}{\isacharparenright}{\kern0pt}\ {\isacharasterisk}{\kern0pt}\ {\isacharparenleft}{\kern0pt}{\isacharparenleft}{\kern0pt}control{\isadigit{2}}\ U{\isacharparenright}{\kern0pt}\isactrlsup {\isasymdagger}{\isacharparenright}{\kern0pt}\ {\isachardollar}{\kern0pt}{\isachardollar}{\kern0pt}\ {\isacharparenleft}{\kern0pt}{\isadigit{0}}{\isacharcomma}{\kern0pt}{\isadigit{1}}{\isacharparenright}{\kern0pt}\ {\isacharplus}{\kern0pt}\isanewline
\ \ \ \ \ \ \ \ \ \ \ \ \ \ \ \ \ \ {\isacharparenleft}{\kern0pt}control{\isadigit{2}}\ U{\isacharparenright}{\kern0pt}\ {\isachardollar}{\kern0pt}{\isachardollar}{\kern0pt}\ {\isacharparenleft}{\kern0pt}{\isadigit{2}}{\isacharcomma}{\kern0pt}{\isadigit{1}}{\isacharparenright}{\kern0pt}\ {\isacharasterisk}{\kern0pt}\ {\isacharparenleft}{\kern0pt}{\isacharparenleft}{\kern0pt}control{\isadigit{2}}\ U{\isacharparenright}{\kern0pt}\isactrlsup {\isasymdagger}{\isacharparenright}{\kern0pt}\ {\isachardollar}{\kern0pt}{\isachardollar}{\kern0pt}\ {\isacharparenleft}{\kern0pt}{\isadigit{1}}{\isacharcomma}{\kern0pt}{\isadigit{1}}{\isacharparenright}{\kern0pt}\ {\isacharplus}{\kern0pt}\isanewline
\ \ \ \ \ \ \ \ \ \ \ \ \ \ \ \ \ \ {\isacharparenleft}{\kern0pt}control{\isadigit{2}}\ U{\isacharparenright}{\kern0pt}\ {\isachardollar}{\kern0pt}{\isachardollar}{\kern0pt}\ {\isacharparenleft}{\kern0pt}{\isadigit{2}}{\isacharcomma}{\kern0pt}{\isadigit{2}}{\isacharparenright}{\kern0pt}\ {\isacharasterisk}{\kern0pt}\ {\isacharparenleft}{\kern0pt}{\isacharparenleft}{\kern0pt}control{\isadigit{2}}\ U{\isacharparenright}{\kern0pt}\isactrlsup {\isasymdagger}{\isacharparenright}{\kern0pt}\ {\isachardollar}{\kern0pt}{\isachardollar}{\kern0pt}\ {\isacharparenleft}{\kern0pt}{\isadigit{2}}{\isacharcomma}{\kern0pt}{\isadigit{1}}{\isacharparenright}{\kern0pt}\ {\isacharplus}{\kern0pt}\isanewline
\ \ \ \ \ \ \ \ \ \ \ \ \ \ \ \ \ \ {\isacharparenleft}{\kern0pt}control{\isadigit{2}}\ U{\isacharparenright}{\kern0pt}\ {\isachardollar}{\kern0pt}{\isachardollar}{\kern0pt}\ {\isacharparenleft}{\kern0pt}{\isadigit{2}}{\isacharcomma}{\kern0pt}{\isadigit{3}}{\isacharparenright}{\kern0pt}\ {\isacharasterisk}{\kern0pt}\ {\isacharparenleft}{\kern0pt}{\isacharparenleft}{\kern0pt}control{\isadigit{2}}\ U{\isacharparenright}{\kern0pt}\isactrlsup {\isasymdagger}{\isacharparenright}{\kern0pt}\ {\isachardollar}{\kern0pt}{\isachardollar}{\kern0pt}\ {\isacharparenleft}{\kern0pt}{\isadigit{3}}{\isacharcomma}{\kern0pt}{\isadigit{1}}{\isacharparenright}{\kern0pt}{\isachardoublequoteclose}\isanewline
\ \ \ \ \ \ \ \ \ \ \ \ \ \ \ \ \isacommand{using}\isamarkupfalse%
\ times{\isacharunderscore}{\kern0pt}mat{\isacharunderscore}{\kern0pt}def\ sumof{\isadigit{4}}\isanewline
\ \ \ \ \ \ \ \ \ \ \ \ \ \ \ \ \isacommand{by}\isamarkupfalse%
\ {\isacharparenleft}{\kern0pt}smt\ {\isacharparenleft}{\kern0pt}z{\isadigit{3}}{\isacharparenright}{\kern0pt}\ carrier{\isacharunderscore}{\kern0pt}matD{\isacharparenleft}{\kern0pt}{\isadigit{1}}{\isacharparenright}{\kern0pt}\ carrier{\isacharunderscore}{\kern0pt}matD{\isacharparenleft}{\kern0pt}{\isadigit{2}}{\isacharparenright}{\kern0pt}\ control{\isadigit{2}}{\isacharunderscore}{\kern0pt}carrier{\isacharunderscore}{\kern0pt}mat\ dim{\isacharunderscore}{\kern0pt}col{\isacharunderscore}{\kern0pt}of{\isacharunderscore}{\kern0pt}dagger\ \isanewline
\ \ \ \ \ \ \ \ \ \ \ \ \ \ \ \ \ \ \ \ \ \ dim{\isacharunderscore}{\kern0pt}row{\isacharunderscore}{\kern0pt}of{\isacharunderscore}{\kern0pt}dagger\ i{\isadigit{2}}\ i{\isadigit{4}}\ index{\isacharunderscore}{\kern0pt}matrix{\isacharunderscore}{\kern0pt}prod\ j{\isadigit{1}}\ j{\isadigit{4}}{\isacharparenright}{\kern0pt}\isanewline
\ \ \ \ \ \ \ \ \ \ \ \ \ \ \isacommand{also}\isamarkupfalse%
\ \isacommand{have}\isamarkupfalse%
\ {\isachardoublequoteopen}{\isasymdots}\ {\isacharequal}{\kern0pt}\ {\isacharparenleft}{\kern0pt}{\isacharparenleft}{\kern0pt}control{\isadigit{2}}\ U{\isacharparenright}{\kern0pt}\isactrlsup {\isasymdagger}{\isacharparenright}{\kern0pt}\ {\isachardollar}{\kern0pt}{\isachardollar}{\kern0pt}\ {\isacharparenleft}{\kern0pt}{\isadigit{2}}{\isacharcomma}{\kern0pt}{\isadigit{1}}{\isacharparenright}{\kern0pt}{\isachardoublequoteclose}\isanewline
\ \ \ \ \ \ \ \ \ \ \ \ \ \ \ \ \isacommand{using}\isamarkupfalse%
\ control{\isadigit{2}}{\isacharunderscore}{\kern0pt}def\ index{\isacharunderscore}{\kern0pt}mat{\isacharunderscore}{\kern0pt}of{\isacharunderscore}{\kern0pt}cols{\isacharunderscore}{\kern0pt}list\ \isacommand{by}\isamarkupfalse%
\ force\isanewline
\ \ \ \ \ \ \ \ \ \ \ \ \ \ \isacommand{also}\isamarkupfalse%
\ \isacommand{have}\isamarkupfalse%
\ {\isachardoublequoteopen}{\isasymdots}\ {\isacharequal}{\kern0pt}\ cnj\ {\isacharparenleft}{\kern0pt}{\isacharparenleft}{\kern0pt}control{\isadigit{2}}\ U{\isacharparenright}{\kern0pt}\ {\isachardollar}{\kern0pt}{\isachardollar}{\kern0pt}\ {\isacharparenleft}{\kern0pt}{\isadigit{1}}{\isacharcomma}{\kern0pt}{\isadigit{2}}{\isacharparenright}{\kern0pt}{\isacharparenright}{\kern0pt}{\isachardoublequoteclose}\isanewline
\ \ \ \ \ \ \ \ \ \ \ \ \ \ \ \ \isacommand{using}\isamarkupfalse%
\ dagger{\isacharunderscore}{\kern0pt}def\ \isanewline
\ \ \ \ \ \ \ \ \ \ \ \ \ \ \ \ \isacommand{by}\isamarkupfalse%
\ {\isacharparenleft}{\kern0pt}metis\ carrier{\isacharunderscore}{\kern0pt}matD{\isacharparenleft}{\kern0pt}{\isadigit{1}}{\isacharparenright}{\kern0pt}\ carrier{\isacharunderscore}{\kern0pt}matD{\isacharparenleft}{\kern0pt}{\isadigit{2}}{\isacharparenright}{\kern0pt}\ control{\isadigit{2}}{\isacharunderscore}{\kern0pt}carrier{\isacharunderscore}{\kern0pt}mat\ i{\isadigit{2}}\ i{\isadigit{4}}\ index{\isacharunderscore}{\kern0pt}mat{\isacharparenleft}{\kern0pt}{\isadigit{1}}{\isacharparenright}{\kern0pt}\ \isanewline
\ \ \ \ \ \ \ \ \ \ \ \ \ \ \ \ \ \ \ \ j{\isadigit{1}}\ j{\isadigit{4}}\ prod{\isachardot}{\kern0pt}simps{\isacharparenleft}{\kern0pt}{\isadigit{2}}{\isacharparenright}{\kern0pt}{\isacharparenright}{\kern0pt}\isanewline
\ \ \ \ \ \ \ \ \ \ \ \ \ \ \isacommand{also}\isamarkupfalse%
\ \isacommand{have}\isamarkupfalse%
\ {\isachardoublequoteopen}{\isasymdots}\ {\isacharequal}{\kern0pt}\ {\isadigit{0}}{\isachardoublequoteclose}\ \isacommand{using}\isamarkupfalse%
\ control{\isadigit{2}}{\isacharunderscore}{\kern0pt}def\ index{\isacharunderscore}{\kern0pt}mat{\isacharunderscore}{\kern0pt}of{\isacharunderscore}{\kern0pt}cols{\isacharunderscore}{\kern0pt}list\ \isacommand{by}\isamarkupfalse%
\ auto\isanewline
\ \ \ \ \ \ \ \ \ \ \ \ \ \ \isacommand{also}\isamarkupfalse%
\ \isacommand{have}\isamarkupfalse%
\ {\isachardoublequoteopen}{\isasymdots}\ {\isacharequal}{\kern0pt}\ {\isadigit{1}}\isactrlsub m\ {\isadigit{4}}\ {\isachardollar}{\kern0pt}{\isachardollar}{\kern0pt}\ {\isacharparenleft}{\kern0pt}{\isadigit{2}}{\isacharcomma}{\kern0pt}{\isadigit{1}}{\isacharparenright}{\kern0pt}{\isachardoublequoteclose}\ \isacommand{by}\isamarkupfalse%
\ simp\isanewline
\ \ \ \ \ \ \ \ \ \ \ \ \ \ \isacommand{finally}\isamarkupfalse%
\ \isacommand{show}\isamarkupfalse%
\ {\isacharquery}{\kern0pt}thesis\ \isacommand{using}\isamarkupfalse%
\ i{\isadigit{2}}\ j{\isadigit{1}}\ \isacommand{by}\isamarkupfalse%
\ simp\isanewline
\ \ \ \ \ \ \ \ \ \ \ \ \isacommand{qed}\isamarkupfalse%
\isanewline
\ \ \ \ \ \ \ \ \ \ \isacommand{next}\isamarkupfalse%
\isanewline
\ \ \ \ \ \ \ \ \ \ \ \ \isacommand{assume}\isamarkupfalse%
\ jl{\isadigit{2}}{\isacharcolon}{\kern0pt}{\isachardoublequoteopen}j\ {\isacharequal}{\kern0pt}\ {\isadigit{2}}\ {\isasymor}\ j\ {\isacharequal}{\kern0pt}\ {\isadigit{3}}{\isachardoublequoteclose}\isanewline
\ \ \ \ \ \ \ \ \ \ \ \ \isacommand{show}\isamarkupfalse%
\ {\isachardoublequoteopen}{\isacharparenleft}{\kern0pt}control{\isadigit{2}}\ U\ {\isacharasterisk}{\kern0pt}\ {\isacharparenleft}{\kern0pt}{\isacharparenleft}{\kern0pt}control{\isadigit{2}}\ U{\isacharparenright}{\kern0pt}\isactrlsup {\isasymdagger}{\isacharparenright}{\kern0pt}{\isacharparenright}{\kern0pt}\ {\isachardollar}{\kern0pt}{\isachardollar}{\kern0pt}\ {\isacharparenleft}{\kern0pt}i{\isacharcomma}{\kern0pt}\ j{\isacharparenright}{\kern0pt}\ {\isacharequal}{\kern0pt}\ {\isadigit{1}}\isactrlsub m\ {\isadigit{4}}\ {\isachardollar}{\kern0pt}{\isachardollar}{\kern0pt}\ {\isacharparenleft}{\kern0pt}i{\isacharcomma}{\kern0pt}\ j{\isacharparenright}{\kern0pt}{\isachardoublequoteclose}\isanewline
\ \ \ \ \ \ \ \ \ \ \ \ \isacommand{proof}\isamarkupfalse%
\ {\isacharparenleft}{\kern0pt}rule\ disjE{\isacharparenright}{\kern0pt}\isanewline
\ \ \ \ \ \ \ \ \ \ \ \ \ \ \isacommand{show}\isamarkupfalse%
\ {\isachardoublequoteopen}j\ {\isacharequal}{\kern0pt}\ {\isadigit{2}}\ {\isasymor}\ j\ {\isacharequal}{\kern0pt}\ {\isadigit{3}}{\isachardoublequoteclose}\ \isacommand{using}\isamarkupfalse%
\ jl{\isadigit{2}}\ \isacommand{by}\isamarkupfalse%
\ this\isanewline
\ \ \ \ \ \ \ \ \ \ \ \ \isacommand{next}\isamarkupfalse%
\isanewline
\ \ \ \ \ \ \ \ \ \ \ \ \ \ \isacommand{assume}\isamarkupfalse%
\ j{\isadigit{2}}{\isacharcolon}{\kern0pt}{\isachardoublequoteopen}j\ {\isacharequal}{\kern0pt}\ {\isadigit{2}}{\isachardoublequoteclose}\isanewline
\ \ \ \ \ \ \ \ \ \ \ \ \ \ \isacommand{show}\isamarkupfalse%
\ {\isachardoublequoteopen}{\isacharparenleft}{\kern0pt}control{\isadigit{2}}\ U\ {\isacharasterisk}{\kern0pt}\ {\isacharparenleft}{\kern0pt}{\isacharparenleft}{\kern0pt}control{\isadigit{2}}\ U{\isacharparenright}{\kern0pt}\isactrlsup {\isasymdagger}{\isacharparenright}{\kern0pt}{\isacharparenright}{\kern0pt}\ {\isachardollar}{\kern0pt}{\isachardollar}{\kern0pt}\ {\isacharparenleft}{\kern0pt}i{\isacharcomma}{\kern0pt}\ j{\isacharparenright}{\kern0pt}\ {\isacharequal}{\kern0pt}\ {\isadigit{1}}\isactrlsub m\ {\isadigit{4}}\ {\isachardollar}{\kern0pt}{\isachardollar}{\kern0pt}\ {\isacharparenleft}{\kern0pt}i{\isacharcomma}{\kern0pt}\ j{\isacharparenright}{\kern0pt}{\isachardoublequoteclose}\isanewline
\ \ \ \ \ \ \ \ \ \ \ \ \ \ \isacommand{proof}\isamarkupfalse%
\ {\isacharminus}{\kern0pt}\isanewline
\ \ \ \ \ \ \ \ \ \ \ \ \ \ \ \ \isacommand{have}\isamarkupfalse%
\ {\isachardoublequoteopen}{\isacharparenleft}{\kern0pt}control{\isadigit{2}}\ U\ {\isacharasterisk}{\kern0pt}\ {\isacharparenleft}{\kern0pt}{\isacharparenleft}{\kern0pt}control{\isadigit{2}}\ U{\isacharparenright}{\kern0pt}\isactrlsup {\isasymdagger}{\isacharparenright}{\kern0pt}{\isacharparenright}{\kern0pt}\ {\isachardollar}{\kern0pt}{\isachardollar}{\kern0pt}\ {\isacharparenleft}{\kern0pt}{\isadigit{2}}{\isacharcomma}{\kern0pt}{\isadigit{2}}{\isacharparenright}{\kern0pt}\ {\isacharequal}{\kern0pt}\ \isanewline
\ \ \ \ \ \ \ \ \ \ \ \ \ \ \ \ \ \ {\isacharparenleft}{\kern0pt}control{\isadigit{2}}\ U{\isacharparenright}{\kern0pt}\ {\isachardollar}{\kern0pt}{\isachardollar}{\kern0pt}\ {\isacharparenleft}{\kern0pt}{\isadigit{2}}{\isacharcomma}{\kern0pt}{\isadigit{0}}{\isacharparenright}{\kern0pt}\ {\isacharasterisk}{\kern0pt}\ {\isacharparenleft}{\kern0pt}{\isacharparenleft}{\kern0pt}control{\isadigit{2}}\ U{\isacharparenright}{\kern0pt}\isactrlsup {\isasymdagger}{\isacharparenright}{\kern0pt}\ {\isachardollar}{\kern0pt}{\isachardollar}{\kern0pt}\ {\isacharparenleft}{\kern0pt}{\isadigit{0}}{\isacharcomma}{\kern0pt}{\isadigit{2}}{\isacharparenright}{\kern0pt}\ {\isacharplus}{\kern0pt}\isanewline
\ \ \ \ \ \ \ \ \ \ \ \ \ \ \ \ \ \ {\isacharparenleft}{\kern0pt}control{\isadigit{2}}\ U{\isacharparenright}{\kern0pt}\ {\isachardollar}{\kern0pt}{\isachardollar}{\kern0pt}\ {\isacharparenleft}{\kern0pt}{\isadigit{2}}{\isacharcomma}{\kern0pt}{\isadigit{1}}{\isacharparenright}{\kern0pt}\ {\isacharasterisk}{\kern0pt}\ {\isacharparenleft}{\kern0pt}{\isacharparenleft}{\kern0pt}control{\isadigit{2}}\ U{\isacharparenright}{\kern0pt}\isactrlsup {\isasymdagger}{\isacharparenright}{\kern0pt}\ {\isachardollar}{\kern0pt}{\isachardollar}{\kern0pt}\ {\isacharparenleft}{\kern0pt}{\isadigit{1}}{\isacharcomma}{\kern0pt}{\isadigit{2}}{\isacharparenright}{\kern0pt}\ {\isacharplus}{\kern0pt}\isanewline
\ \ \ \ \ \ \ \ \ \ \ \ \ \ \ \ \ \ {\isacharparenleft}{\kern0pt}control{\isadigit{2}}\ U{\isacharparenright}{\kern0pt}\ {\isachardollar}{\kern0pt}{\isachardollar}{\kern0pt}\ {\isacharparenleft}{\kern0pt}{\isadigit{2}}{\isacharcomma}{\kern0pt}{\isadigit{2}}{\isacharparenright}{\kern0pt}\ {\isacharasterisk}{\kern0pt}\ {\isacharparenleft}{\kern0pt}{\isacharparenleft}{\kern0pt}control{\isadigit{2}}\ U{\isacharparenright}{\kern0pt}\isactrlsup {\isasymdagger}{\isacharparenright}{\kern0pt}\ {\isachardollar}{\kern0pt}{\isachardollar}{\kern0pt}\ {\isacharparenleft}{\kern0pt}{\isadigit{2}}{\isacharcomma}{\kern0pt}{\isadigit{2}}{\isacharparenright}{\kern0pt}\ {\isacharplus}{\kern0pt}\isanewline
\ \ \ \ \ \ \ \ \ \ \ \ \ \ \ \ \ \ {\isacharparenleft}{\kern0pt}control{\isadigit{2}}\ U{\isacharparenright}{\kern0pt}\ {\isachardollar}{\kern0pt}{\isachardollar}{\kern0pt}\ {\isacharparenleft}{\kern0pt}{\isadigit{2}}{\isacharcomma}{\kern0pt}{\isadigit{3}}{\isacharparenright}{\kern0pt}\ {\isacharasterisk}{\kern0pt}\ {\isacharparenleft}{\kern0pt}{\isacharparenleft}{\kern0pt}control{\isadigit{2}}\ U{\isacharparenright}{\kern0pt}\isactrlsup {\isasymdagger}{\isacharparenright}{\kern0pt}\ {\isachardollar}{\kern0pt}{\isachardollar}{\kern0pt}\ {\isacharparenleft}{\kern0pt}{\isadigit{3}}{\isacharcomma}{\kern0pt}{\isadigit{2}}{\isacharparenright}{\kern0pt}{\isachardoublequoteclose}\isanewline
\ \ \ \ \ \ \ \ \ \ \ \ \ \ \ \ \isacommand{using}\isamarkupfalse%
\ times{\isacharunderscore}{\kern0pt}mat{\isacharunderscore}{\kern0pt}def\ sumof{\isadigit{4}}\isanewline
\ \ \ \ \ \ \ \ \ \ \ \ \ \ \ \ \isacommand{by}\isamarkupfalse%
\ {\isacharparenleft}{\kern0pt}smt\ {\isacharparenleft}{\kern0pt}z{\isadigit{3}}{\isacharparenright}{\kern0pt}\ carrier{\isacharunderscore}{\kern0pt}matD{\isacharparenleft}{\kern0pt}{\isadigit{1}}{\isacharparenright}{\kern0pt}\ carrier{\isacharunderscore}{\kern0pt}matD{\isacharparenleft}{\kern0pt}{\isadigit{2}}{\isacharparenright}{\kern0pt}\ control{\isadigit{2}}{\isacharunderscore}{\kern0pt}carrier{\isacharunderscore}{\kern0pt}mat\ dim{\isacharunderscore}{\kern0pt}col{\isacharunderscore}{\kern0pt}of{\isacharunderscore}{\kern0pt}dagger\ \isanewline
\ \ \ \ \ \ \ \ \ \ \ \ \ \ \ \ \ \ \ \ \ \ dim{\isacharunderscore}{\kern0pt}row{\isacharunderscore}{\kern0pt}of{\isacharunderscore}{\kern0pt}dagger\ i{\isadigit{2}}\ i{\isadigit{4}}\ index{\isacharunderscore}{\kern0pt}matrix{\isacharunderscore}{\kern0pt}prod\ j{\isadigit{2}}\ j{\isadigit{4}}{\isacharparenright}{\kern0pt}\isanewline
\ \ \ \ \ \ \ \ \ \ \ \ \ \ \isacommand{also}\isamarkupfalse%
\ \isacommand{have}\isamarkupfalse%
\ {\isachardoublequoteopen}{\isasymdots}\ {\isacharequal}{\kern0pt}\ {\isacharparenleft}{\kern0pt}{\isacharparenleft}{\kern0pt}control{\isadigit{2}}\ U{\isacharparenright}{\kern0pt}\isactrlsup {\isasymdagger}{\isacharparenright}{\kern0pt}\ {\isachardollar}{\kern0pt}{\isachardollar}{\kern0pt}\ {\isacharparenleft}{\kern0pt}{\isadigit{2}}{\isacharcomma}{\kern0pt}{\isadigit{2}}{\isacharparenright}{\kern0pt}{\isachardoublequoteclose}\isanewline
\ \ \ \ \ \ \ \ \ \ \ \ \ \ \ \ \isacommand{using}\isamarkupfalse%
\ control{\isadigit{2}}{\isacharunderscore}{\kern0pt}def\ index{\isacharunderscore}{\kern0pt}mat{\isacharunderscore}{\kern0pt}of{\isacharunderscore}{\kern0pt}cols{\isacharunderscore}{\kern0pt}list\ \isacommand{by}\isamarkupfalse%
\ force\isanewline
\ \ \ \ \ \ \ \ \ \ \ \ \ \ \isacommand{also}\isamarkupfalse%
\ \isacommand{have}\isamarkupfalse%
\ {\isachardoublequoteopen}{\isasymdots}\ {\isacharequal}{\kern0pt}\ cnj\ {\isacharparenleft}{\kern0pt}{\isacharparenleft}{\kern0pt}control{\isadigit{2}}\ U{\isacharparenright}{\kern0pt}\ {\isachardollar}{\kern0pt}{\isachardollar}{\kern0pt}\ {\isacharparenleft}{\kern0pt}{\isadigit{2}}{\isacharcomma}{\kern0pt}{\isadigit{2}}{\isacharparenright}{\kern0pt}{\isacharparenright}{\kern0pt}{\isachardoublequoteclose}\isanewline
\ \ \ \ \ \ \ \ \ \ \ \ \ \ \ \ \isacommand{using}\isamarkupfalse%
\ dagger{\isacharunderscore}{\kern0pt}def\ \isanewline
\ \ \ \ \ \ \ \ \ \ \ \ \ \ \ \ \isacommand{by}\isamarkupfalse%
\ {\isacharparenleft}{\kern0pt}metis\ carrier{\isacharunderscore}{\kern0pt}matD{\isacharparenleft}{\kern0pt}{\isadigit{1}}{\isacharparenright}{\kern0pt}\ carrier{\isacharunderscore}{\kern0pt}matD{\isacharparenleft}{\kern0pt}{\isadigit{2}}{\isacharparenright}{\kern0pt}\ control{\isadigit{2}}{\isacharunderscore}{\kern0pt}carrier{\isacharunderscore}{\kern0pt}mat\ i{\isadigit{2}}\ index{\isacharunderscore}{\kern0pt}mat{\isacharparenleft}{\kern0pt}{\isadigit{1}}{\isacharparenright}{\kern0pt}\ \isanewline
\ \ \ \ \ \ \ \ \ \ \ \ \ \ \ \ \ \ \ \ j{\isadigit{2}}\ j{\isadigit{4}}\ prod{\isachardot}{\kern0pt}simps{\isacharparenleft}{\kern0pt}{\isadigit{2}}{\isacharparenright}{\kern0pt}{\isacharparenright}{\kern0pt}\isanewline
\ \ \ \ \ \ \ \ \ \ \ \ \ \ \isacommand{also}\isamarkupfalse%
\ \isacommand{have}\isamarkupfalse%
\ {\isachardoublequoteopen}{\isasymdots}\ {\isacharequal}{\kern0pt}\ {\isadigit{1}}{\isachardoublequoteclose}\ \isacommand{using}\isamarkupfalse%
\ control{\isadigit{2}}{\isacharunderscore}{\kern0pt}def\ index{\isacharunderscore}{\kern0pt}mat{\isacharunderscore}{\kern0pt}of{\isacharunderscore}{\kern0pt}cols{\isacharunderscore}{\kern0pt}list\ \isacommand{by}\isamarkupfalse%
\ auto\isanewline
\ \ \ \ \ \ \ \ \ \ \ \ \ \ \isacommand{also}\isamarkupfalse%
\ \isacommand{have}\isamarkupfalse%
\ {\isachardoublequoteopen}{\isasymdots}\ {\isacharequal}{\kern0pt}\ {\isadigit{1}}\isactrlsub m\ {\isadigit{4}}\ {\isachardollar}{\kern0pt}{\isachardollar}{\kern0pt}\ {\isacharparenleft}{\kern0pt}{\isadigit{2}}{\isacharcomma}{\kern0pt}{\isadigit{2}}{\isacharparenright}{\kern0pt}{\isachardoublequoteclose}\ \isacommand{by}\isamarkupfalse%
\ simp\isanewline
\ \ \ \ \ \ \ \ \ \ \ \ \ \ \isacommand{finally}\isamarkupfalse%
\ \isacommand{show}\isamarkupfalse%
\ {\isacharquery}{\kern0pt}thesis\ \isacommand{using}\isamarkupfalse%
\ i{\isadigit{2}}\ j{\isadigit{2}}\ \isacommand{by}\isamarkupfalse%
\ simp\isanewline
\ \ \ \ \ \ \ \ \ \ \ \ \isacommand{qed}\isamarkupfalse%
\isanewline
\ \ \ \ \ \ \ \ \ \ \isacommand{next}\isamarkupfalse%
\isanewline
\ \ \ \ \ \ \ \ \ \ \ \ \isacommand{assume}\isamarkupfalse%
\ j{\isadigit{3}}{\isacharcolon}{\kern0pt}{\isachardoublequoteopen}j\ {\isacharequal}{\kern0pt}\ {\isadigit{3}}{\isachardoublequoteclose}\isanewline
\ \ \ \ \ \ \ \ \ \ \ \ \isacommand{show}\isamarkupfalse%
\ {\isachardoublequoteopen}{\isacharparenleft}{\kern0pt}control{\isadigit{2}}\ U\ {\isacharasterisk}{\kern0pt}\ {\isacharparenleft}{\kern0pt}{\isacharparenleft}{\kern0pt}control{\isadigit{2}}\ U{\isacharparenright}{\kern0pt}\isactrlsup {\isasymdagger}{\isacharparenright}{\kern0pt}{\isacharparenright}{\kern0pt}\ {\isachardollar}{\kern0pt}{\isachardollar}{\kern0pt}\ {\isacharparenleft}{\kern0pt}i{\isacharcomma}{\kern0pt}\ j{\isacharparenright}{\kern0pt}\ {\isacharequal}{\kern0pt}\ {\isadigit{1}}\isactrlsub m\ {\isadigit{4}}\ {\isachardollar}{\kern0pt}{\isachardollar}{\kern0pt}\ {\isacharparenleft}{\kern0pt}i{\isacharcomma}{\kern0pt}\ j{\isacharparenright}{\kern0pt}{\isachardoublequoteclose}\isanewline
\ \ \ \ \ \ \ \ \ \ \ \ \isacommand{proof}\isamarkupfalse%
\ {\isacharminus}{\kern0pt}\isanewline
\ \ \ \ \ \ \ \ \ \ \ \ \ \ \isacommand{have}\isamarkupfalse%
\ {\isachardoublequoteopen}{\isacharparenleft}{\kern0pt}control{\isadigit{2}}\ U\ {\isacharasterisk}{\kern0pt}\ {\isacharparenleft}{\kern0pt}{\isacharparenleft}{\kern0pt}control{\isadigit{2}}\ U{\isacharparenright}{\kern0pt}\isactrlsup {\isasymdagger}{\isacharparenright}{\kern0pt}{\isacharparenright}{\kern0pt}\ {\isachardollar}{\kern0pt}{\isachardollar}{\kern0pt}\ {\isacharparenleft}{\kern0pt}{\isadigit{2}}{\isacharcomma}{\kern0pt}{\isadigit{3}}{\isacharparenright}{\kern0pt}\ {\isacharequal}{\kern0pt}\ \isanewline
\ \ \ \ \ \ \ \ \ \ \ \ \ \ \ \ \ \ {\isacharparenleft}{\kern0pt}control{\isadigit{2}}\ U{\isacharparenright}{\kern0pt}\ {\isachardollar}{\kern0pt}{\isachardollar}{\kern0pt}\ {\isacharparenleft}{\kern0pt}{\isadigit{2}}{\isacharcomma}{\kern0pt}{\isadigit{0}}{\isacharparenright}{\kern0pt}\ {\isacharasterisk}{\kern0pt}\ {\isacharparenleft}{\kern0pt}{\isacharparenleft}{\kern0pt}control{\isadigit{2}}\ U{\isacharparenright}{\kern0pt}\isactrlsup {\isasymdagger}{\isacharparenright}{\kern0pt}\ {\isachardollar}{\kern0pt}{\isachardollar}{\kern0pt}\ {\isacharparenleft}{\kern0pt}{\isadigit{0}}{\isacharcomma}{\kern0pt}{\isadigit{3}}{\isacharparenright}{\kern0pt}\ {\isacharplus}{\kern0pt}\isanewline
\ \ \ \ \ \ \ \ \ \ \ \ \ \ \ \ \ \ {\isacharparenleft}{\kern0pt}control{\isadigit{2}}\ U{\isacharparenright}{\kern0pt}\ {\isachardollar}{\kern0pt}{\isachardollar}{\kern0pt}\ {\isacharparenleft}{\kern0pt}{\isadigit{2}}{\isacharcomma}{\kern0pt}{\isadigit{1}}{\isacharparenright}{\kern0pt}\ {\isacharasterisk}{\kern0pt}\ {\isacharparenleft}{\kern0pt}{\isacharparenleft}{\kern0pt}control{\isadigit{2}}\ U{\isacharparenright}{\kern0pt}\isactrlsup {\isasymdagger}{\isacharparenright}{\kern0pt}\ {\isachardollar}{\kern0pt}{\isachardollar}{\kern0pt}\ {\isacharparenleft}{\kern0pt}{\isadigit{1}}{\isacharcomma}{\kern0pt}{\isadigit{3}}{\isacharparenright}{\kern0pt}\ {\isacharplus}{\kern0pt}\isanewline
\ \ \ \ \ \ \ \ \ \ \ \ \ \ \ \ \ \ {\isacharparenleft}{\kern0pt}control{\isadigit{2}}\ U{\isacharparenright}{\kern0pt}\ {\isachardollar}{\kern0pt}{\isachardollar}{\kern0pt}\ {\isacharparenleft}{\kern0pt}{\isadigit{2}}{\isacharcomma}{\kern0pt}{\isadigit{2}}{\isacharparenright}{\kern0pt}\ {\isacharasterisk}{\kern0pt}\ {\isacharparenleft}{\kern0pt}{\isacharparenleft}{\kern0pt}control{\isadigit{2}}\ U{\isacharparenright}{\kern0pt}\isactrlsup {\isasymdagger}{\isacharparenright}{\kern0pt}\ {\isachardollar}{\kern0pt}{\isachardollar}{\kern0pt}\ {\isacharparenleft}{\kern0pt}{\isadigit{2}}{\isacharcomma}{\kern0pt}{\isadigit{3}}{\isacharparenright}{\kern0pt}\ {\isacharplus}{\kern0pt}\isanewline
\ \ \ \ \ \ \ \ \ \ \ \ \ \ \ \ \ \ {\isacharparenleft}{\kern0pt}control{\isadigit{2}}\ U{\isacharparenright}{\kern0pt}\ {\isachardollar}{\kern0pt}{\isachardollar}{\kern0pt}\ {\isacharparenleft}{\kern0pt}{\isadigit{2}}{\isacharcomma}{\kern0pt}{\isadigit{3}}{\isacharparenright}{\kern0pt}\ {\isacharasterisk}{\kern0pt}\ {\isacharparenleft}{\kern0pt}{\isacharparenleft}{\kern0pt}control{\isadigit{2}}\ U{\isacharparenright}{\kern0pt}\isactrlsup {\isasymdagger}{\isacharparenright}{\kern0pt}\ {\isachardollar}{\kern0pt}{\isachardollar}{\kern0pt}\ {\isacharparenleft}{\kern0pt}{\isadigit{3}}{\isacharcomma}{\kern0pt}{\isadigit{3}}{\isacharparenright}{\kern0pt}{\isachardoublequoteclose}\isanewline
\ \ \ \ \ \ \ \ \ \ \ \ \ \ \ \ \isacommand{using}\isamarkupfalse%
\ times{\isacharunderscore}{\kern0pt}mat{\isacharunderscore}{\kern0pt}def\ sumof{\isadigit{4}}\isanewline
\ \ \ \ \ \ \ \ \ \ \ \ \ \ \ \ \isacommand{by}\isamarkupfalse%
\ {\isacharparenleft}{\kern0pt}smt\ {\isacharparenleft}{\kern0pt}z{\isadigit{3}}{\isacharparenright}{\kern0pt}\ carrier{\isacharunderscore}{\kern0pt}matD{\isacharparenleft}{\kern0pt}{\isadigit{1}}{\isacharparenright}{\kern0pt}\ carrier{\isacharunderscore}{\kern0pt}matD{\isacharparenleft}{\kern0pt}{\isadigit{2}}{\isacharparenright}{\kern0pt}\ control{\isadigit{2}}{\isacharunderscore}{\kern0pt}carrier{\isacharunderscore}{\kern0pt}mat\ dim{\isacharunderscore}{\kern0pt}col{\isacharunderscore}{\kern0pt}of{\isacharunderscore}{\kern0pt}dagger\ \isanewline
\ \ \ \ \ \ \ \ \ \ \ \ \ \ \ \ \ \ \ \ \ \ dim{\isacharunderscore}{\kern0pt}row{\isacharunderscore}{\kern0pt}of{\isacharunderscore}{\kern0pt}dagger\ i{\isadigit{2}}\ i{\isadigit{4}}\ index{\isacharunderscore}{\kern0pt}matrix{\isacharunderscore}{\kern0pt}prod\ j{\isadigit{3}}\ j{\isadigit{4}}{\isacharparenright}{\kern0pt}\isanewline
\ \ \ \ \ \ \ \ \ \ \ \ \ \ \isacommand{also}\isamarkupfalse%
\ \isacommand{have}\isamarkupfalse%
\ {\isachardoublequoteopen}{\isasymdots}\ {\isacharequal}{\kern0pt}\ {\isacharparenleft}{\kern0pt}{\isacharparenleft}{\kern0pt}control{\isadigit{2}}\ U{\isacharparenright}{\kern0pt}\isactrlsup {\isasymdagger}{\isacharparenright}{\kern0pt}\ {\isachardollar}{\kern0pt}{\isachardollar}{\kern0pt}\ {\isacharparenleft}{\kern0pt}{\isadigit{2}}{\isacharcomma}{\kern0pt}{\isadigit{3}}{\isacharparenright}{\kern0pt}{\isachardoublequoteclose}\isanewline
\ \ \ \ \ \ \ \ \ \ \ \ \ \ \ \ \isacommand{using}\isamarkupfalse%
\ control{\isadigit{2}}{\isacharunderscore}{\kern0pt}def\ index{\isacharunderscore}{\kern0pt}mat{\isacharunderscore}{\kern0pt}of{\isacharunderscore}{\kern0pt}cols{\isacharunderscore}{\kern0pt}list\ \isacommand{by}\isamarkupfalse%
\ force\isanewline
\ \ \ \ \ \ \ \ \ \ \ \ \ \ \isacommand{also}\isamarkupfalse%
\ \isacommand{have}\isamarkupfalse%
\ {\isachardoublequoteopen}{\isasymdots}\ {\isacharequal}{\kern0pt}\ cnj\ {\isacharparenleft}{\kern0pt}{\isacharparenleft}{\kern0pt}control{\isadigit{2}}\ U{\isacharparenright}{\kern0pt}\ {\isachardollar}{\kern0pt}{\isachardollar}{\kern0pt}\ {\isacharparenleft}{\kern0pt}{\isadigit{3}}{\isacharcomma}{\kern0pt}{\isadigit{2}}{\isacharparenright}{\kern0pt}{\isacharparenright}{\kern0pt}{\isachardoublequoteclose}\isanewline
\ \ \ \ \ \ \ \ \ \ \ \ \ \ \ \ \isacommand{using}\isamarkupfalse%
\ dagger{\isacharunderscore}{\kern0pt}def\ \isanewline
\ \ \ \ \ \ \ \ \ \ \ \ \ \ \ \ \isacommand{by}\isamarkupfalse%
\ {\isacharparenleft}{\kern0pt}metis\ carrier{\isacharunderscore}{\kern0pt}matD{\isacharparenleft}{\kern0pt}{\isadigit{1}}{\isacharparenright}{\kern0pt}\ carrier{\isacharunderscore}{\kern0pt}matD{\isacharparenleft}{\kern0pt}{\isadigit{2}}{\isacharparenright}{\kern0pt}\ control{\isadigit{2}}{\isacharunderscore}{\kern0pt}carrier{\isacharunderscore}{\kern0pt}mat\ i{\isadigit{2}}\ i{\isadigit{4}}\ index{\isacharunderscore}{\kern0pt}mat{\isacharparenleft}{\kern0pt}{\isadigit{1}}{\isacharparenright}{\kern0pt}\ \isanewline
\ \ \ \ \ \ \ \ \ \ \ \ \ \ \ \ \ \ \ \ j{\isadigit{3}}\ j{\isadigit{4}}\ prod{\isachardot}{\kern0pt}simps{\isacharparenleft}{\kern0pt}{\isadigit{2}}{\isacharparenright}{\kern0pt}{\isacharparenright}{\kern0pt}\isanewline
\ \ \ \ \ \ \ \ \ \ \ \ \ \ \isacommand{also}\isamarkupfalse%
\ \isacommand{have}\isamarkupfalse%
\ {\isachardoublequoteopen}{\isasymdots}\ {\isacharequal}{\kern0pt}\ {\isadigit{0}}{\isachardoublequoteclose}\ \isacommand{using}\isamarkupfalse%
\ control{\isadigit{2}}{\isacharunderscore}{\kern0pt}def\ index{\isacharunderscore}{\kern0pt}mat{\isacharunderscore}{\kern0pt}of{\isacharunderscore}{\kern0pt}cols{\isacharunderscore}{\kern0pt}list\ \isacommand{by}\isamarkupfalse%
\ auto\isanewline
\ \ \ \ \ \ \ \ \ \ \ \ \ \ \isacommand{also}\isamarkupfalse%
\ \isacommand{have}\isamarkupfalse%
\ {\isachardoublequoteopen}{\isasymdots}\ {\isacharequal}{\kern0pt}\ {\isadigit{1}}\isactrlsub m\ {\isadigit{4}}\ {\isachardollar}{\kern0pt}{\isachardollar}{\kern0pt}\ {\isacharparenleft}{\kern0pt}{\isadigit{2}}{\isacharcomma}{\kern0pt}{\isadigit{3}}{\isacharparenright}{\kern0pt}{\isachardoublequoteclose}\ \isacommand{by}\isamarkupfalse%
\ simp\isanewline
\ \ \ \ \ \ \ \ \ \ \ \ \ \ \isacommand{finally}\isamarkupfalse%
\ \isacommand{show}\isamarkupfalse%
\ {\isacharquery}{\kern0pt}thesis\ \isacommand{using}\isamarkupfalse%
\ i{\isadigit{2}}\ j{\isadigit{3}}\ \isacommand{by}\isamarkupfalse%
\ simp\isanewline
\ \ \ \ \ \ \ \ \ \ \ \ \isacommand{qed}\isamarkupfalse%
\isanewline
\ \ \ \ \ \ \ \ \ \ \isacommand{qed}\isamarkupfalse%
\isanewline
\ \ \ \ \ \ \ \ \isacommand{qed}\isamarkupfalse%
\isanewline
\ \ \ \ \ \ \isacommand{qed}\isamarkupfalse%
\isanewline
\ \ \ \ \isacommand{next}\isamarkupfalse%
\isanewline
\ \ \ \ \ \ \isacommand{assume}\isamarkupfalse%
\ i{\isadigit{3}}{\isacharcolon}{\kern0pt}{\isachardoublequoteopen}i\ {\isacharequal}{\kern0pt}\ {\isadigit{3}}{\isachardoublequoteclose}\isanewline
\ \ \ \ \ \ \isacommand{show}\isamarkupfalse%
\ {\isachardoublequoteopen}{\isacharparenleft}{\kern0pt}control{\isadigit{2}}\ U\ {\isacharasterisk}{\kern0pt}\ {\isacharparenleft}{\kern0pt}{\isacharparenleft}{\kern0pt}control{\isadigit{2}}\ U{\isacharparenright}{\kern0pt}\isactrlsup {\isasymdagger}{\isacharparenright}{\kern0pt}{\isacharparenright}{\kern0pt}\ {\isachardollar}{\kern0pt}{\isachardollar}{\kern0pt}\ {\isacharparenleft}{\kern0pt}i{\isacharcomma}{\kern0pt}\ j{\isacharparenright}{\kern0pt}\ {\isacharequal}{\kern0pt}\ {\isadigit{1}}\isactrlsub m\ {\isadigit{4}}\ {\isachardollar}{\kern0pt}{\isachardollar}{\kern0pt}\ {\isacharparenleft}{\kern0pt}i{\isacharcomma}{\kern0pt}\ j{\isacharparenright}{\kern0pt}{\isachardoublequoteclose}\isanewline
\ \ \ \ \ \ \isacommand{proof}\isamarkupfalse%
\ {\isacharparenleft}{\kern0pt}rule\ disjE{\isacharparenright}{\kern0pt}\isanewline
\ \ \ \ \ \ \ \ \isacommand{show}\isamarkupfalse%
\ {\isachardoublequoteopen}j\ {\isacharequal}{\kern0pt}\ {\isadigit{0}}\ {\isasymor}\ j\ {\isacharequal}{\kern0pt}\ {\isadigit{1}}\ {\isasymor}\ j\ {\isacharequal}{\kern0pt}\ {\isadigit{2}}\ {\isasymor}\ j\ {\isacharequal}{\kern0pt}\ {\isadigit{3}}{\isachardoublequoteclose}\ \isacommand{using}\isamarkupfalse%
\ j{\isadigit{4}}\ \isacommand{by}\isamarkupfalse%
\ auto\isanewline
\ \ \ \ \ \ \isacommand{next}\isamarkupfalse%
\isanewline
\ \ \ \ \ \ \ \ \isacommand{assume}\isamarkupfalse%
\ j{\isadigit{0}}{\isacharcolon}{\kern0pt}{\isachardoublequoteopen}j\ {\isacharequal}{\kern0pt}\ {\isadigit{0}}{\isachardoublequoteclose}\isanewline
\ \ \ \ \ \ \ \ \isacommand{show}\isamarkupfalse%
\ {\isachardoublequoteopen}{\isacharparenleft}{\kern0pt}control{\isadigit{2}}\ U\ {\isacharasterisk}{\kern0pt}\ {\isacharparenleft}{\kern0pt}{\isacharparenleft}{\kern0pt}control{\isadigit{2}}\ U{\isacharparenright}{\kern0pt}\isactrlsup {\isasymdagger}{\isacharparenright}{\kern0pt}{\isacharparenright}{\kern0pt}\ {\isachardollar}{\kern0pt}{\isachardollar}{\kern0pt}\ {\isacharparenleft}{\kern0pt}i{\isacharcomma}{\kern0pt}\ j{\isacharparenright}{\kern0pt}\ {\isacharequal}{\kern0pt}\ {\isadigit{1}}\isactrlsub m\ {\isadigit{4}}\ {\isachardollar}{\kern0pt}{\isachardollar}{\kern0pt}\ {\isacharparenleft}{\kern0pt}i{\isacharcomma}{\kern0pt}\ j{\isacharparenright}{\kern0pt}{\isachardoublequoteclose}\isanewline
\ \ \ \ \ \ \ \ \isacommand{proof}\isamarkupfalse%
\ {\isacharminus}{\kern0pt}\isanewline
\ \ \ \ \ \ \ \ \ \ \isacommand{have}\isamarkupfalse%
\ {\isachardoublequoteopen}{\isacharparenleft}{\kern0pt}control{\isadigit{2}}\ U\ {\isacharasterisk}{\kern0pt}\ {\isacharparenleft}{\kern0pt}{\isacharparenleft}{\kern0pt}control{\isadigit{2}}\ U{\isacharparenright}{\kern0pt}\isactrlsup {\isasymdagger}{\isacharparenright}{\kern0pt}{\isacharparenright}{\kern0pt}\ {\isachardollar}{\kern0pt}{\isachardollar}{\kern0pt}\ {\isacharparenleft}{\kern0pt}{\isadigit{3}}{\isacharcomma}{\kern0pt}{\isadigit{0}}{\isacharparenright}{\kern0pt}\ {\isacharequal}{\kern0pt}\ \isanewline
\ \ \ \ \ \ \ \ \ \ \ \ \ \ \ \ \ \ \ \ {\isacharparenleft}{\kern0pt}control{\isadigit{2}}\ U{\isacharparenright}{\kern0pt}\ {\isachardollar}{\kern0pt}{\isachardollar}{\kern0pt}\ {\isacharparenleft}{\kern0pt}{\isadigit{3}}{\isacharcomma}{\kern0pt}{\isadigit{0}}{\isacharparenright}{\kern0pt}\ {\isacharasterisk}{\kern0pt}\ {\isacharparenleft}{\kern0pt}{\isacharparenleft}{\kern0pt}control{\isadigit{2}}\ U{\isacharparenright}{\kern0pt}\isactrlsup {\isasymdagger}{\isacharparenright}{\kern0pt}\ {\isachardollar}{\kern0pt}{\isachardollar}{\kern0pt}\ {\isacharparenleft}{\kern0pt}{\isadigit{0}}{\isacharcomma}{\kern0pt}{\isadigit{0}}{\isacharparenright}{\kern0pt}\ {\isacharplus}{\kern0pt}\isanewline
\ \ \ \ \ \ \ \ \ \ \ \ \ \ \ \ \ \ \ \ {\isacharparenleft}{\kern0pt}control{\isadigit{2}}\ U{\isacharparenright}{\kern0pt}\ {\isachardollar}{\kern0pt}{\isachardollar}{\kern0pt}\ {\isacharparenleft}{\kern0pt}{\isadigit{3}}{\isacharcomma}{\kern0pt}{\isadigit{1}}{\isacharparenright}{\kern0pt}\ {\isacharasterisk}{\kern0pt}\ {\isacharparenleft}{\kern0pt}{\isacharparenleft}{\kern0pt}control{\isadigit{2}}\ U{\isacharparenright}{\kern0pt}\isactrlsup {\isasymdagger}{\isacharparenright}{\kern0pt}\ {\isachardollar}{\kern0pt}{\isachardollar}{\kern0pt}\ {\isacharparenleft}{\kern0pt}{\isadigit{1}}{\isacharcomma}{\kern0pt}{\isadigit{0}}{\isacharparenright}{\kern0pt}\ {\isacharplus}{\kern0pt}\isanewline
\ \ \ \ \ \ \ \ \ \ \ \ \ \ \ \ \ \ \ \ {\isacharparenleft}{\kern0pt}control{\isadigit{2}}\ U{\isacharparenright}{\kern0pt}\ {\isachardollar}{\kern0pt}{\isachardollar}{\kern0pt}\ {\isacharparenleft}{\kern0pt}{\isadigit{3}}{\isacharcomma}{\kern0pt}{\isadigit{2}}{\isacharparenright}{\kern0pt}\ {\isacharasterisk}{\kern0pt}\ {\isacharparenleft}{\kern0pt}{\isacharparenleft}{\kern0pt}control{\isadigit{2}}\ U{\isacharparenright}{\kern0pt}\isactrlsup {\isasymdagger}{\isacharparenright}{\kern0pt}\ {\isachardollar}{\kern0pt}{\isachardollar}{\kern0pt}\ {\isacharparenleft}{\kern0pt}{\isadigit{2}}{\isacharcomma}{\kern0pt}{\isadigit{0}}{\isacharparenright}{\kern0pt}\ {\isacharplus}{\kern0pt}\isanewline
\ \ \ \ \ \ \ \ \ \ \ \ \ \ \ \ \ \ \ \ {\isacharparenleft}{\kern0pt}control{\isadigit{2}}\ U{\isacharparenright}{\kern0pt}\ {\isachardollar}{\kern0pt}{\isachardollar}{\kern0pt}\ {\isacharparenleft}{\kern0pt}{\isadigit{3}}{\isacharcomma}{\kern0pt}{\isadigit{3}}{\isacharparenright}{\kern0pt}\ {\isacharasterisk}{\kern0pt}\ {\isacharparenleft}{\kern0pt}{\isacharparenleft}{\kern0pt}control{\isadigit{2}}\ U{\isacharparenright}{\kern0pt}\isactrlsup {\isasymdagger}{\isacharparenright}{\kern0pt}\ {\isachardollar}{\kern0pt}{\isachardollar}{\kern0pt}\ {\isacharparenleft}{\kern0pt}{\isadigit{3}}{\isacharcomma}{\kern0pt}{\isadigit{0}}{\isacharparenright}{\kern0pt}{\isachardoublequoteclose}\isanewline
\ \ \ \ \ \ \ \ \ \ \ \ \isacommand{using}\isamarkupfalse%
\ times{\isacharunderscore}{\kern0pt}mat{\isacharunderscore}{\kern0pt}def\ sumof{\isadigit{4}}\isanewline
\ \ \ \ \ \ \ \ \ \ \ \ \ \ \isacommand{by}\isamarkupfalse%
\ {\isacharparenleft}{\kern0pt}smt\ {\isacharparenleft}{\kern0pt}z{\isadigit{3}}{\isacharparenright}{\kern0pt}\ carrier{\isacharunderscore}{\kern0pt}matD{\isacharparenleft}{\kern0pt}{\isadigit{1}}{\isacharparenright}{\kern0pt}\ carrier{\isacharunderscore}{\kern0pt}matD{\isacharparenleft}{\kern0pt}{\isadigit{2}}{\isacharparenright}{\kern0pt}\ control{\isadigit{2}}{\isacharunderscore}{\kern0pt}carrier{\isacharunderscore}{\kern0pt}mat\ dim{\isacharunderscore}{\kern0pt}col{\isacharunderscore}{\kern0pt}of{\isacharunderscore}{\kern0pt}dagger\ \isanewline
\ \ \ \ \ \ \ \ \ \ \ \ \ \ \ \ \ \ \ \ dim{\isacharunderscore}{\kern0pt}row{\isacharunderscore}{\kern0pt}of{\isacharunderscore}{\kern0pt}dagger\ i{\isadigit{3}}\ i{\isadigit{4}}\ index{\isacharunderscore}{\kern0pt}matrix{\isacharunderscore}{\kern0pt}prod\ j{\isadigit{0}}\ j{\isadigit{4}}{\isacharparenright}{\kern0pt}\isanewline
\ \ \ \ \ \ \ \ \ \ \isacommand{also}\isamarkupfalse%
\ \isacommand{have}\isamarkupfalse%
\ {\isachardoublequoteopen}{\isasymdots}\ {\isacharequal}{\kern0pt}\ {\isacharparenleft}{\kern0pt}control{\isadigit{2}}\ U{\isacharparenright}{\kern0pt}\ {\isachardollar}{\kern0pt}{\isachardollar}{\kern0pt}\ {\isacharparenleft}{\kern0pt}{\isadigit{3}}{\isacharcomma}{\kern0pt}{\isadigit{1}}{\isacharparenright}{\kern0pt}\ {\isacharasterisk}{\kern0pt}\ {\isacharparenleft}{\kern0pt}{\isacharparenleft}{\kern0pt}control{\isadigit{2}}\ U{\isacharparenright}{\kern0pt}\isactrlsup {\isasymdagger}{\isacharparenright}{\kern0pt}\ {\isachardollar}{\kern0pt}{\isachardollar}{\kern0pt}\ {\isacharparenleft}{\kern0pt}{\isadigit{1}}{\isacharcomma}{\kern0pt}{\isadigit{0}}{\isacharparenright}{\kern0pt}\ {\isacharplus}{\kern0pt}\ \isanewline
\ \ \ \ \ \ \ \ \ \ \ \ \ \ \ \ \ \ \ \ \ \ \ \ \ \ \ \ {\isacharparenleft}{\kern0pt}control{\isadigit{2}}\ U{\isacharparenright}{\kern0pt}\ {\isachardollar}{\kern0pt}{\isachardollar}{\kern0pt}\ {\isacharparenleft}{\kern0pt}{\isadigit{3}}{\isacharcomma}{\kern0pt}{\isadigit{3}}{\isacharparenright}{\kern0pt}\ {\isacharasterisk}{\kern0pt}\ {\isacharparenleft}{\kern0pt}{\isacharparenleft}{\kern0pt}control{\isadigit{2}}\ U{\isacharparenright}{\kern0pt}\isactrlsup {\isasymdagger}{\isacharparenright}{\kern0pt}\ {\isachardollar}{\kern0pt}{\isachardollar}{\kern0pt}\ {\isacharparenleft}{\kern0pt}{\isadigit{3}}{\isacharcomma}{\kern0pt}{\isadigit{0}}{\isacharparenright}{\kern0pt}{\isachardoublequoteclose}\isanewline
\ \ \ \ \ \ \ \ \ \ \ \ \ \ \isacommand{using}\isamarkupfalse%
\ control{\isadigit{2}}{\isacharunderscore}{\kern0pt}def\ index{\isacharunderscore}{\kern0pt}mat{\isacharunderscore}{\kern0pt}of{\isacharunderscore}{\kern0pt}cols{\isacharunderscore}{\kern0pt}list\ \isacommand{by}\isamarkupfalse%
\ force\isanewline
\ \ \ \ \ \ \ \ \ \ \isacommand{also}\isamarkupfalse%
\ \isacommand{have}\isamarkupfalse%
\ {\isachardoublequoteopen}{\isasymdots}\ {\isacharequal}{\kern0pt}\ {\isacharparenleft}{\kern0pt}control{\isadigit{2}}\ U{\isacharparenright}{\kern0pt}\ {\isachardollar}{\kern0pt}{\isachardollar}{\kern0pt}\ {\isacharparenleft}{\kern0pt}{\isadigit{3}}{\isacharcomma}{\kern0pt}{\isadigit{1}}{\isacharparenright}{\kern0pt}\ {\isacharasterisk}{\kern0pt}\ {\isacharparenleft}{\kern0pt}cnj\ {\isacharparenleft}{\kern0pt}{\isacharparenleft}{\kern0pt}control{\isadigit{2}}\ U{\isacharparenright}{\kern0pt}\ {\isachardollar}{\kern0pt}{\isachardollar}{\kern0pt}\ {\isacharparenleft}{\kern0pt}{\isadigit{0}}{\isacharcomma}{\kern0pt}{\isadigit{1}}{\isacharparenright}{\kern0pt}{\isacharparenright}{\kern0pt}{\isacharparenright}{\kern0pt}\ {\isacharplus}{\kern0pt}\ \isanewline
\ \ \ \ \ \ \ \ \ \ \ \ \ \ \ \ \ \ \ \ \ \ \ \ \ \ \ \ {\isacharparenleft}{\kern0pt}control{\isadigit{2}}\ U{\isacharparenright}{\kern0pt}\ {\isachardollar}{\kern0pt}{\isachardollar}{\kern0pt}\ {\isacharparenleft}{\kern0pt}{\isadigit{3}}{\isacharcomma}{\kern0pt}{\isadigit{3}}{\isacharparenright}{\kern0pt}\ {\isacharasterisk}{\kern0pt}\ {\isacharparenleft}{\kern0pt}cnj\ {\isacharparenleft}{\kern0pt}{\isacharparenleft}{\kern0pt}control{\isadigit{2}}\ U{\isacharparenright}{\kern0pt}\ {\isachardollar}{\kern0pt}{\isachardollar}{\kern0pt}\ {\isacharparenleft}{\kern0pt}{\isadigit{0}}{\isacharcomma}{\kern0pt}{\isadigit{3}}{\isacharparenright}{\kern0pt}{\isacharparenright}{\kern0pt}{\isacharparenright}{\kern0pt}{\isachardoublequoteclose}\isanewline
\ \ \ \ \ \ \ \ \ \ \ \ \ \ \isacommand{using}\isamarkupfalse%
\ dagger{\isacharunderscore}{\kern0pt}def\ Tensor{\isachardot}{\kern0pt}mat{\isacharunderscore}{\kern0pt}of{\isacharunderscore}{\kern0pt}cols{\isacharunderscore}{\kern0pt}list{\isacharunderscore}{\kern0pt}def\ control{\isadigit{2}}{\isacharunderscore}{\kern0pt}def\ \isacommand{by}\isamarkupfalse%
\ auto\isanewline
\ \ \ \ \ \ \ \ \ \ \isacommand{also}\isamarkupfalse%
\ \isacommand{have}\isamarkupfalse%
\ {\isachardoublequoteopen}{\isasymdots}\ {\isacharequal}{\kern0pt}\ {\isacharparenleft}{\kern0pt}control{\isadigit{2}}\ U{\isacharparenright}{\kern0pt}\ {\isachardollar}{\kern0pt}{\isachardollar}{\kern0pt}\ {\isacharparenleft}{\kern0pt}{\isadigit{3}}{\isacharcomma}{\kern0pt}{\isadigit{1}}{\isacharparenright}{\kern0pt}\ {\isacharasterisk}{\kern0pt}\ {\isacharparenleft}{\kern0pt}cnj\ {\isadigit{0}}{\isacharparenright}{\kern0pt}\ {\isacharplus}{\kern0pt}\isanewline
\ \ \ \ \ \ \ \ \ \ \ \ \ \ \ \ \ \ \ \ \ \ \ \ \ \ \ \ {\isacharparenleft}{\kern0pt}control{\isadigit{2}}\ U{\isacharparenright}{\kern0pt}\ {\isachardollar}{\kern0pt}{\isachardollar}{\kern0pt}\ {\isacharparenleft}{\kern0pt}{\isadigit{3}}{\isacharcomma}{\kern0pt}{\isadigit{3}}{\isacharparenright}{\kern0pt}\ {\isacharasterisk}{\kern0pt}\ {\isacharparenleft}{\kern0pt}cnj\ {\isadigit{0}}{\isacharparenright}{\kern0pt}{\isachardoublequoteclose}\isanewline
\ \ \ \ \ \ \ \ \ \ \ \ \ \ \isacommand{using}\isamarkupfalse%
\ control{\isadigit{2}}{\isacharunderscore}{\kern0pt}def\ index{\isacharunderscore}{\kern0pt}mat{\isacharunderscore}{\kern0pt}of{\isacharunderscore}{\kern0pt}cols{\isacharunderscore}{\kern0pt}list\ \isacommand{by}\isamarkupfalse%
\ simp\isanewline
\ \ \ \ \ \ \ \ \ \ \isacommand{also}\isamarkupfalse%
\ \isacommand{have}\isamarkupfalse%
\ {\isachardoublequoteopen}{\isasymdots}\ {\isacharequal}{\kern0pt}\ {\isadigit{0}}{\isachardoublequoteclose}\ \isacommand{by}\isamarkupfalse%
\ auto\isanewline
\ \ \ \ \ \ \ \ \ \ \isacommand{also}\isamarkupfalse%
\ \isacommand{have}\isamarkupfalse%
\ {\isachardoublequoteopen}{\isasymdots}\ {\isacharequal}{\kern0pt}\ {\isadigit{1}}\isactrlsub m\ {\isadigit{4}}\ {\isachardollar}{\kern0pt}{\isachardollar}{\kern0pt}\ {\isacharparenleft}{\kern0pt}{\isadigit{3}}{\isacharcomma}{\kern0pt}{\isadigit{0}}{\isacharparenright}{\kern0pt}{\isachardoublequoteclose}\ \isacommand{by}\isamarkupfalse%
\ simp\isanewline
\ \ \ \ \ \ \ \ \ \ \isacommand{finally}\isamarkupfalse%
\ \isacommand{show}\isamarkupfalse%
\ {\isacharquery}{\kern0pt}thesis\ \isacommand{using}\isamarkupfalse%
\ i{\isadigit{3}}\ j{\isadigit{0}}\ \isacommand{by}\isamarkupfalse%
\ simp\isanewline
\ \ \ \ \ \ \ \ \isacommand{qed}\isamarkupfalse%
\isanewline
\ \ \ \ \ \ \isacommand{next}\isamarkupfalse%
\isanewline
\ \ \ \ \ \ \ \ \isacommand{assume}\isamarkupfalse%
\ jl{\isadigit{3}}{\isacharcolon}{\kern0pt}{\isachardoublequoteopen}j\ {\isacharequal}{\kern0pt}\ {\isadigit{1}}\ {\isasymor}\ j\ {\isacharequal}{\kern0pt}\ {\isadigit{2}}\ {\isasymor}\ j\ {\isacharequal}{\kern0pt}\ {\isadigit{3}}{\isachardoublequoteclose}\isanewline
\ \ \ \ \ \ \ \ \isacommand{show}\isamarkupfalse%
\ {\isachardoublequoteopen}{\isacharparenleft}{\kern0pt}control{\isadigit{2}}\ U\ {\isacharasterisk}{\kern0pt}\ {\isacharparenleft}{\kern0pt}{\isacharparenleft}{\kern0pt}control{\isadigit{2}}\ U{\isacharparenright}{\kern0pt}\isactrlsup {\isasymdagger}{\isacharparenright}{\kern0pt}{\isacharparenright}{\kern0pt}\ {\isachardollar}{\kern0pt}{\isachardollar}{\kern0pt}\ {\isacharparenleft}{\kern0pt}i{\isacharcomma}{\kern0pt}\ j{\isacharparenright}{\kern0pt}\ {\isacharequal}{\kern0pt}\ {\isadigit{1}}\isactrlsub m\ {\isadigit{4}}\ {\isachardollar}{\kern0pt}{\isachardollar}{\kern0pt}\ {\isacharparenleft}{\kern0pt}i{\isacharcomma}{\kern0pt}\ j{\isacharparenright}{\kern0pt}{\isachardoublequoteclose}\isanewline
\ \ \ \ \ \ \ \ \isacommand{proof}\isamarkupfalse%
\ {\isacharparenleft}{\kern0pt}rule\ disjE{\isacharparenright}{\kern0pt}\isanewline
\ \ \ \ \ \ \ \ \ \ \isacommand{show}\isamarkupfalse%
\ {\isachardoublequoteopen}j\ {\isacharequal}{\kern0pt}\ {\isadigit{1}}\ {\isasymor}\ j\ {\isacharequal}{\kern0pt}\ {\isadigit{2}}\ {\isasymor}\ j\ {\isacharequal}{\kern0pt}\ {\isadigit{3}}{\isachardoublequoteclose}\ \isacommand{using}\isamarkupfalse%
\ jl{\isadigit{3}}\ \isacommand{by}\isamarkupfalse%
\ this\isanewline
\ \ \ \ \ \ \ \ \isacommand{next}\isamarkupfalse%
\isanewline
\ \ \ \ \ \ \ \ \ \ \isacommand{assume}\isamarkupfalse%
\ j{\isadigit{1}}{\isacharcolon}{\kern0pt}{\isachardoublequoteopen}j\ {\isacharequal}{\kern0pt}\ {\isadigit{1}}{\isachardoublequoteclose}\isanewline
\ \ \ \ \ \ \ \ \ \ \isacommand{show}\isamarkupfalse%
\ {\isachardoublequoteopen}{\isacharparenleft}{\kern0pt}control{\isadigit{2}}\ U\ {\isacharasterisk}{\kern0pt}\ {\isacharparenleft}{\kern0pt}{\isacharparenleft}{\kern0pt}control{\isadigit{2}}\ U{\isacharparenright}{\kern0pt}\isactrlsup {\isasymdagger}{\isacharparenright}{\kern0pt}{\isacharparenright}{\kern0pt}\ {\isachardollar}{\kern0pt}{\isachardollar}{\kern0pt}\ {\isacharparenleft}{\kern0pt}i{\isacharcomma}{\kern0pt}\ j{\isacharparenright}{\kern0pt}\ {\isacharequal}{\kern0pt}\ {\isadigit{1}}\isactrlsub m\ {\isadigit{4}}\ {\isachardollar}{\kern0pt}{\isachardollar}{\kern0pt}\ {\isacharparenleft}{\kern0pt}i{\isacharcomma}{\kern0pt}\ j{\isacharparenright}{\kern0pt}{\isachardoublequoteclose}\isanewline
\ \ \ \ \ \ \ \ \ \ \isacommand{proof}\isamarkupfalse%
\ {\isacharminus}{\kern0pt}\isanewline
\ \ \ \ \ \ \ \ \ \ \ \ \isacommand{have}\isamarkupfalse%
\ {\isachardoublequoteopen}{\isacharparenleft}{\kern0pt}control{\isadigit{2}}\ U\ {\isacharasterisk}{\kern0pt}\ {\isacharparenleft}{\kern0pt}{\isacharparenleft}{\kern0pt}control{\isadigit{2}}\ U{\isacharparenright}{\kern0pt}\isactrlsup {\isasymdagger}{\isacharparenright}{\kern0pt}{\isacharparenright}{\kern0pt}\ {\isachardollar}{\kern0pt}{\isachardollar}{\kern0pt}\ {\isacharparenleft}{\kern0pt}{\isadigit{3}}{\isacharcomma}{\kern0pt}{\isadigit{1}}{\isacharparenright}{\kern0pt}\ {\isacharequal}{\kern0pt}\ \isanewline
\ \ \ \ \ \ \ \ \ \ \ \ \ \ \ \ \ \ \ \ {\isacharparenleft}{\kern0pt}control{\isadigit{2}}\ U{\isacharparenright}{\kern0pt}\ {\isachardollar}{\kern0pt}{\isachardollar}{\kern0pt}\ {\isacharparenleft}{\kern0pt}{\isadigit{3}}{\isacharcomma}{\kern0pt}{\isadigit{0}}{\isacharparenright}{\kern0pt}\ {\isacharasterisk}{\kern0pt}\ {\isacharparenleft}{\kern0pt}{\isacharparenleft}{\kern0pt}control{\isadigit{2}}\ U{\isacharparenright}{\kern0pt}\isactrlsup {\isasymdagger}{\isacharparenright}{\kern0pt}\ {\isachardollar}{\kern0pt}{\isachardollar}{\kern0pt}\ {\isacharparenleft}{\kern0pt}{\isadigit{0}}{\isacharcomma}{\kern0pt}{\isadigit{1}}{\isacharparenright}{\kern0pt}\ {\isacharplus}{\kern0pt}\isanewline
\ \ \ \ \ \ \ \ \ \ \ \ \ \ \ \ \ \ \ \ {\isacharparenleft}{\kern0pt}control{\isadigit{2}}\ U{\isacharparenright}{\kern0pt}\ {\isachardollar}{\kern0pt}{\isachardollar}{\kern0pt}\ {\isacharparenleft}{\kern0pt}{\isadigit{3}}{\isacharcomma}{\kern0pt}{\isadigit{1}}{\isacharparenright}{\kern0pt}\ {\isacharasterisk}{\kern0pt}\ {\isacharparenleft}{\kern0pt}{\isacharparenleft}{\kern0pt}control{\isadigit{2}}\ U{\isacharparenright}{\kern0pt}\isactrlsup {\isasymdagger}{\isacharparenright}{\kern0pt}\ {\isachardollar}{\kern0pt}{\isachardollar}{\kern0pt}\ {\isacharparenleft}{\kern0pt}{\isadigit{1}}{\isacharcomma}{\kern0pt}{\isadigit{1}}{\isacharparenright}{\kern0pt}\ {\isacharplus}{\kern0pt}\isanewline
\ \ \ \ \ \ \ \ \ \ \ \ \ \ \ \ \ \ \ \ {\isacharparenleft}{\kern0pt}control{\isadigit{2}}\ U{\isacharparenright}{\kern0pt}\ {\isachardollar}{\kern0pt}{\isachardollar}{\kern0pt}\ {\isacharparenleft}{\kern0pt}{\isadigit{3}}{\isacharcomma}{\kern0pt}{\isadigit{2}}{\isacharparenright}{\kern0pt}\ {\isacharasterisk}{\kern0pt}\ {\isacharparenleft}{\kern0pt}{\isacharparenleft}{\kern0pt}control{\isadigit{2}}\ U{\isacharparenright}{\kern0pt}\isactrlsup {\isasymdagger}{\isacharparenright}{\kern0pt}\ {\isachardollar}{\kern0pt}{\isachardollar}{\kern0pt}\ {\isacharparenleft}{\kern0pt}{\isadigit{2}}{\isacharcomma}{\kern0pt}{\isadigit{1}}{\isacharparenright}{\kern0pt}\ {\isacharplus}{\kern0pt}\isanewline
\ \ \ \ \ \ \ \ \ \ \ \ \ \ \ \ \ \ \ \ {\isacharparenleft}{\kern0pt}control{\isadigit{2}}\ U{\isacharparenright}{\kern0pt}\ {\isachardollar}{\kern0pt}{\isachardollar}{\kern0pt}\ {\isacharparenleft}{\kern0pt}{\isadigit{3}}{\isacharcomma}{\kern0pt}{\isadigit{3}}{\isacharparenright}{\kern0pt}\ {\isacharasterisk}{\kern0pt}\ {\isacharparenleft}{\kern0pt}{\isacharparenleft}{\kern0pt}control{\isadigit{2}}\ U{\isacharparenright}{\kern0pt}\isactrlsup {\isasymdagger}{\isacharparenright}{\kern0pt}\ {\isachardollar}{\kern0pt}{\isachardollar}{\kern0pt}\ {\isacharparenleft}{\kern0pt}{\isadigit{3}}{\isacharcomma}{\kern0pt}{\isadigit{1}}{\isacharparenright}{\kern0pt}{\isachardoublequoteclose}\isanewline
\ \ \ \ \ \ \ \ \ \ \ \ \ \ \isacommand{using}\isamarkupfalse%
\ times{\isacharunderscore}{\kern0pt}mat{\isacharunderscore}{\kern0pt}def\ sumof{\isadigit{4}}\isanewline
\ \ \ \ \ \ \ \ \ \ \ \ \ \ \isacommand{by}\isamarkupfalse%
\ {\isacharparenleft}{\kern0pt}smt\ {\isacharparenleft}{\kern0pt}z{\isadigit{3}}{\isacharparenright}{\kern0pt}\ carrier{\isacharunderscore}{\kern0pt}matD{\isacharparenleft}{\kern0pt}{\isadigit{1}}{\isacharparenright}{\kern0pt}\ carrier{\isacharunderscore}{\kern0pt}matD{\isacharparenleft}{\kern0pt}{\isadigit{2}}{\isacharparenright}{\kern0pt}\ control{\isadigit{2}}{\isacharunderscore}{\kern0pt}carrier{\isacharunderscore}{\kern0pt}mat\ dim{\isacharunderscore}{\kern0pt}col{\isacharunderscore}{\kern0pt}of{\isacharunderscore}{\kern0pt}dagger\ \isanewline
\ \ \ \ \ \ \ \ \ \ \ \ \ \ \ \ \ \ \ \ \ \ dim{\isacharunderscore}{\kern0pt}row{\isacharunderscore}{\kern0pt}of{\isacharunderscore}{\kern0pt}dagger\ i{\isadigit{3}}\ i{\isadigit{4}}\ index{\isacharunderscore}{\kern0pt}matrix{\isacharunderscore}{\kern0pt}prod\ j{\isadigit{1}}\ j{\isadigit{4}}{\isacharparenright}{\kern0pt}\isanewline
\ \ \ \ \ \ \ \ \ \ \ \ \isacommand{also}\isamarkupfalse%
\ \isacommand{have}\isamarkupfalse%
\ {\isachardoublequoteopen}{\isasymdots}\ {\isacharequal}{\kern0pt}\ {\isacharparenleft}{\kern0pt}control{\isadigit{2}}\ U{\isacharparenright}{\kern0pt}\ {\isachardollar}{\kern0pt}{\isachardollar}{\kern0pt}\ {\isacharparenleft}{\kern0pt}{\isadigit{3}}{\isacharcomma}{\kern0pt}{\isadigit{1}}{\isacharparenright}{\kern0pt}\ {\isacharasterisk}{\kern0pt}\ {\isacharparenleft}{\kern0pt}{\isacharparenleft}{\kern0pt}control{\isadigit{2}}\ U{\isacharparenright}{\kern0pt}\isactrlsup {\isasymdagger}{\isacharparenright}{\kern0pt}\ {\isachardollar}{\kern0pt}{\isachardollar}{\kern0pt}\ {\isacharparenleft}{\kern0pt}{\isadigit{1}}{\isacharcomma}{\kern0pt}{\isadigit{1}}{\isacharparenright}{\kern0pt}\ {\isacharplus}{\kern0pt}\ \isanewline
\ \ \ \ \ \ \ \ \ \ \ \ \ \ \ \ \ \ \ \ \ \ \ \ \ \ \ \ \ \ {\isacharparenleft}{\kern0pt}control{\isadigit{2}}\ U{\isacharparenright}{\kern0pt}\ {\isachardollar}{\kern0pt}{\isachardollar}{\kern0pt}\ {\isacharparenleft}{\kern0pt}{\isadigit{3}}{\isacharcomma}{\kern0pt}{\isadigit{3}}{\isacharparenright}{\kern0pt}\ {\isacharasterisk}{\kern0pt}\ {\isacharparenleft}{\kern0pt}{\isacharparenleft}{\kern0pt}control{\isadigit{2}}\ U{\isacharparenright}{\kern0pt}\isactrlsup {\isasymdagger}{\isacharparenright}{\kern0pt}\ {\isachardollar}{\kern0pt}{\isachardollar}{\kern0pt}\ {\isacharparenleft}{\kern0pt}{\isadigit{3}}{\isacharcomma}{\kern0pt}{\isadigit{1}}{\isacharparenright}{\kern0pt}{\isachardoublequoteclose}\isanewline
\ \ \ \ \ \ \ \ \ \ \ \ \ \ \ \ \isacommand{using}\isamarkupfalse%
\ control{\isadigit{2}}{\isacharunderscore}{\kern0pt}def\ index{\isacharunderscore}{\kern0pt}mat{\isacharunderscore}{\kern0pt}of{\isacharunderscore}{\kern0pt}cols{\isacharunderscore}{\kern0pt}list\ \isacommand{by}\isamarkupfalse%
\ force\isanewline
\ \ \ \ \ \ \ \ \ \ \ \ \isacommand{also}\isamarkupfalse%
\ \isacommand{have}\isamarkupfalse%
\ {\isachardoublequoteopen}{\isasymdots}\ {\isacharequal}{\kern0pt}\ {\isacharparenleft}{\kern0pt}control{\isadigit{2}}\ U{\isacharparenright}{\kern0pt}\ {\isachardollar}{\kern0pt}{\isachardollar}{\kern0pt}\ {\isacharparenleft}{\kern0pt}{\isadigit{3}}{\isacharcomma}{\kern0pt}{\isadigit{1}}{\isacharparenright}{\kern0pt}\ {\isacharasterisk}{\kern0pt}\ {\isacharparenleft}{\kern0pt}cnj\ {\isacharparenleft}{\kern0pt}{\isacharparenleft}{\kern0pt}control{\isadigit{2}}\ U{\isacharparenright}{\kern0pt}\ {\isachardollar}{\kern0pt}{\isachardollar}{\kern0pt}\ {\isacharparenleft}{\kern0pt}{\isadigit{1}}{\isacharcomma}{\kern0pt}{\isadigit{1}}{\isacharparenright}{\kern0pt}{\isacharparenright}{\kern0pt}{\isacharparenright}{\kern0pt}\ {\isacharplus}{\kern0pt}\ \isanewline
\ \ \ \ \ \ \ \ \ \ \ \ \ \ \ \ \ \ \ \ \ \ \ \ \ \ \ \ \ \ {\isacharparenleft}{\kern0pt}control{\isadigit{2}}\ U{\isacharparenright}{\kern0pt}\ {\isachardollar}{\kern0pt}{\isachardollar}{\kern0pt}\ {\isacharparenleft}{\kern0pt}{\isadigit{3}}{\isacharcomma}{\kern0pt}{\isadigit{3}}{\isacharparenright}{\kern0pt}\ {\isacharasterisk}{\kern0pt}\ {\isacharparenleft}{\kern0pt}cnj\ {\isacharparenleft}{\kern0pt}{\isacharparenleft}{\kern0pt}control{\isadigit{2}}\ U{\isacharparenright}{\kern0pt}\ {\isachardollar}{\kern0pt}{\isachardollar}{\kern0pt}\ {\isacharparenleft}{\kern0pt}{\isadigit{1}}{\isacharcomma}{\kern0pt}{\isadigit{3}}{\isacharparenright}{\kern0pt}{\isacharparenright}{\kern0pt}{\isacharparenright}{\kern0pt}{\isachardoublequoteclose}\isanewline
\ \ \ \ \ \ \ \ \ \ \ \ \ \ \ \ \isacommand{using}\isamarkupfalse%
\ dagger{\isacharunderscore}{\kern0pt}def\ Tensor{\isachardot}{\kern0pt}mat{\isacharunderscore}{\kern0pt}of{\isacharunderscore}{\kern0pt}cols{\isacharunderscore}{\kern0pt}list{\isacharunderscore}{\kern0pt}def\ control{\isadigit{2}}{\isacharunderscore}{\kern0pt}def\ \isacommand{by}\isamarkupfalse%
\ auto\isanewline
\ \ \ \ \ \ \ \ \ \ \ \ \isacommand{also}\isamarkupfalse%
\ \isacommand{have}\isamarkupfalse%
\ {\isachardoublequoteopen}{\isasymdots}\ {\isacharequal}{\kern0pt}\ U\ {\isachardollar}{\kern0pt}{\isachardollar}{\kern0pt}\ {\isacharparenleft}{\kern0pt}{\isadigit{1}}{\isacharcomma}{\kern0pt}{\isadigit{0}}{\isacharparenright}{\kern0pt}\ {\isacharasterisk}{\kern0pt}\ {\isacharparenleft}{\kern0pt}cnj\ {\isacharparenleft}{\kern0pt}U\ {\isachardollar}{\kern0pt}{\isachardollar}{\kern0pt}\ {\isacharparenleft}{\kern0pt}{\isadigit{0}}{\isacharcomma}{\kern0pt}{\isadigit{0}}{\isacharparenright}{\kern0pt}{\isacharparenright}{\kern0pt}{\isacharparenright}{\kern0pt}\ {\isacharplus}{\kern0pt}\isanewline
\ \ \ \ \ \ \ \ \ \ \ \ \ \ \ \ \ \ \ \ \ \ \ \ \ \ \ \ U\ {\isachardollar}{\kern0pt}{\isachardollar}{\kern0pt}\ {\isacharparenleft}{\kern0pt}{\isadigit{1}}{\isacharcomma}{\kern0pt}{\isadigit{1}}{\isacharparenright}{\kern0pt}\ {\isacharasterisk}{\kern0pt}\ {\isacharparenleft}{\kern0pt}cnj\ {\isacharparenleft}{\kern0pt}U\ {\isachardollar}{\kern0pt}{\isachardollar}{\kern0pt}\ {\isacharparenleft}{\kern0pt}{\isadigit{0}}{\isacharcomma}{\kern0pt}{\isadigit{1}}{\isacharparenright}{\kern0pt}{\isacharparenright}{\kern0pt}{\isacharparenright}{\kern0pt}{\isachardoublequoteclose}\isanewline
\ \ \ \ \ \ \ \ \ \ \ \ \ \ \isacommand{using}\isamarkupfalse%
\ control{\isadigit{2}}{\isacharunderscore}{\kern0pt}def\ index{\isacharunderscore}{\kern0pt}mat{\isacharunderscore}{\kern0pt}of{\isacharunderscore}{\kern0pt}cols{\isacharunderscore}{\kern0pt}list\ \isacommand{by}\isamarkupfalse%
\ simp\isanewline
\ \ \ \ \ \ \ \ \ \ \ \ \isacommand{also}\isamarkupfalse%
\ \isacommand{have}\isamarkupfalse%
\ {\isachardoublequoteopen}{\isasymdots}\ {\isacharequal}{\kern0pt}\ {\isacharparenleft}{\kern0pt}U\ {\isachardollar}{\kern0pt}{\isachardollar}{\kern0pt}\ {\isacharparenleft}{\kern0pt}{\isadigit{1}}{\isacharcomma}{\kern0pt}{\isadigit{0}}{\isacharparenright}{\kern0pt}{\isacharparenright}{\kern0pt}\ {\isacharasterisk}{\kern0pt}\ {\isacharparenleft}{\kern0pt}{\isacharparenleft}{\kern0pt}U\isactrlsup {\isasymdagger}{\isacharparenright}{\kern0pt}\ {\isachardollar}{\kern0pt}{\isachardollar}{\kern0pt}\ {\isacharparenleft}{\kern0pt}{\isadigit{0}}{\isacharcomma}{\kern0pt}{\isadigit{0}}{\isacharparenright}{\kern0pt}{\isacharparenright}{\kern0pt}\ {\isacharplus}{\kern0pt}\isanewline
\ \ \ \ \ \ \ \ \ \ \ \ \ \ \ \ \ \ \ \ \ \ \ \ \ \ \ \ {\isacharparenleft}{\kern0pt}U\ {\isachardollar}{\kern0pt}{\isachardollar}{\kern0pt}\ {\isacharparenleft}{\kern0pt}{\isadigit{1}}{\isacharcomma}{\kern0pt}{\isadigit{1}}{\isacharparenright}{\kern0pt}{\isacharparenright}{\kern0pt}\ {\isacharasterisk}{\kern0pt}\ {\isacharparenleft}{\kern0pt}{\isacharparenleft}{\kern0pt}U\isactrlsup {\isasymdagger}{\isacharparenright}{\kern0pt}\ {\isachardollar}{\kern0pt}{\isachardollar}{\kern0pt}\ {\isacharparenleft}{\kern0pt}{\isadigit{1}}{\isacharcomma}{\kern0pt}{\isadigit{0}}{\isacharparenright}{\kern0pt}{\isacharparenright}{\kern0pt}{\isachardoublequoteclose}\isanewline
\ \ \ \ \ \ \ \ \ \ \ \ \ \ \isacommand{using}\isamarkupfalse%
\ dagger{\isacharunderscore}{\kern0pt}def\ assms{\isacharparenleft}{\kern0pt}{\isadigit{1}}{\isacharparenright}{\kern0pt}\ gate{\isacharunderscore}{\kern0pt}def\ \isacommand{by}\isamarkupfalse%
\ force\isanewline
\ \ \ \ \ \ \ \ \ \ \ \ \isacommand{also}\isamarkupfalse%
\ \isacommand{have}\isamarkupfalse%
\ {\isachardoublequoteopen}{\isasymdots}\ {\isacharequal}{\kern0pt}\ {\isacharparenleft}{\kern0pt}U\ {\isacharasterisk}{\kern0pt}\ {\isacharparenleft}{\kern0pt}U\isactrlsup {\isasymdagger}{\isacharparenright}{\kern0pt}{\isacharparenright}{\kern0pt}\ {\isachardollar}{\kern0pt}{\isachardollar}{\kern0pt}\ {\isacharparenleft}{\kern0pt}{\isadigit{1}}{\isacharcomma}{\kern0pt}{\isadigit{0}}{\isacharparenright}{\kern0pt}{\isachardoublequoteclose}\ \isanewline
\ \ \ \ \ \ \ \ \ \ \ \ \ \ \isacommand{using}\isamarkupfalse%
\ times{\isacharunderscore}{\kern0pt}mat{\isacharunderscore}{\kern0pt}def\ assms{\isacharparenleft}{\kern0pt}{\isadigit{1}}{\isacharparenright}{\kern0pt}\ gate{\isacharunderscore}{\kern0pt}carrier{\isacharunderscore}{\kern0pt}mat\ sumof{\isadigit{2}}\isanewline
\ \ \ \ \ \ \ \ \ \ \ \ \ \ \isacommand{by}\isamarkupfalse%
\ {\isacharparenleft}{\kern0pt}smt\ {\isacharparenleft}{\kern0pt}z{\isadigit{3}}{\isacharparenright}{\kern0pt}\ Suc{\isacharunderscore}{\kern0pt}{\isadigit{1}}\ carrier{\isacharunderscore}{\kern0pt}matD{\isacharparenleft}{\kern0pt}{\isadigit{2}}{\isacharparenright}{\kern0pt}\ dagger{\isacharunderscore}{\kern0pt}def\ dim{\isacharunderscore}{\kern0pt}col{\isacharunderscore}{\kern0pt}mat{\isacharparenleft}{\kern0pt}{\isadigit{1}}{\isacharparenright}{\kern0pt}\ dim{\isacharunderscore}{\kern0pt}row{\isacharunderscore}{\kern0pt}of{\isacharunderscore}{\kern0pt}dagger\ \isanewline
\ \ \ \ \ \ \ \ \ \ \ \ \ \ \ \ \ \ gate{\isachardot}{\kern0pt}dim{\isacharunderscore}{\kern0pt}row\ index{\isacharunderscore}{\kern0pt}matrix{\isacharunderscore}{\kern0pt}prod\ lessI\ pos{\isadigit{2}}\ power{\isacharunderscore}{\kern0pt}one{\isacharunderscore}{\kern0pt}right{\isacharparenright}{\kern0pt}\isanewline
\ \ \ \ \ \ \ \ \ \ \ \ \isacommand{also}\isamarkupfalse%
\ \isacommand{have}\isamarkupfalse%
\ {\isachardoublequoteopen}{\isasymdots}\ {\isacharequal}{\kern0pt}\ {\isacharparenleft}{\kern0pt}{\isadigit{1}}\isactrlsub m\ {\isadigit{2}}{\isacharparenright}{\kern0pt}\ {\isachardollar}{\kern0pt}{\isachardollar}{\kern0pt}\ {\isacharparenleft}{\kern0pt}{\isadigit{1}}{\isacharcomma}{\kern0pt}{\isadigit{0}}{\isacharparenright}{\kern0pt}{\isachardoublequoteclose}\ \isacommand{using}\isamarkupfalse%
\ assms{\isacharparenleft}{\kern0pt}{\isadigit{1}}{\isacharparenright}{\kern0pt}\ gate{\isacharunderscore}{\kern0pt}def\ unitary{\isacharunderscore}{\kern0pt}def\ \isacommand{by}\isamarkupfalse%
\ auto\isanewline
\ \ \ \ \ \ \ \ \ \ \ \ \isacommand{also}\isamarkupfalse%
\ \isacommand{have}\isamarkupfalse%
\ {\isachardoublequoteopen}{\isasymdots}\ {\isacharequal}{\kern0pt}\ {\isadigit{0}}{\isachardoublequoteclose}\ \isacommand{by}\isamarkupfalse%
\ auto\isanewline
\ \ \ \ \ \ \ \ \ \ \ \ \isacommand{also}\isamarkupfalse%
\ \isacommand{have}\isamarkupfalse%
\ {\isachardoublequoteopen}{\isasymdots}\ {\isacharequal}{\kern0pt}\ {\isadigit{1}}\isactrlsub m\ {\isadigit{4}}\ {\isachardollar}{\kern0pt}{\isachardollar}{\kern0pt}\ {\isacharparenleft}{\kern0pt}{\isadigit{3}}{\isacharcomma}{\kern0pt}{\isadigit{1}}{\isacharparenright}{\kern0pt}{\isachardoublequoteclose}\ \isacommand{by}\isamarkupfalse%
\ simp\isanewline
\ \ \ \ \ \ \ \ \ \ \ \ \isacommand{finally}\isamarkupfalse%
\ \isacommand{show}\isamarkupfalse%
\ {\isacharquery}{\kern0pt}thesis\ \isacommand{using}\isamarkupfalse%
\ i{\isadigit{3}}\ j{\isadigit{1}}\ \isacommand{by}\isamarkupfalse%
\ simp\isanewline
\ \ \ \ \ \ \ \ \ \ \isacommand{qed}\isamarkupfalse%
\isanewline
\ \ \ \ \ \ \ \ \isacommand{next}\isamarkupfalse%
\isanewline
\ \ \ \ \ \ \ \ \ \ \isacommand{assume}\isamarkupfalse%
\ jl{\isadigit{2}}{\isacharcolon}{\kern0pt}{\isachardoublequoteopen}j\ {\isacharequal}{\kern0pt}\ {\isadigit{2}}\ {\isasymor}\ j\ {\isacharequal}{\kern0pt}\ {\isadigit{3}}{\isachardoublequoteclose}\isanewline
\ \ \ \ \ \ \ \ \ \ \isacommand{show}\isamarkupfalse%
\ {\isachardoublequoteopen}{\isacharparenleft}{\kern0pt}control{\isadigit{2}}\ U\ {\isacharasterisk}{\kern0pt}\ {\isacharparenleft}{\kern0pt}{\isacharparenleft}{\kern0pt}control{\isadigit{2}}\ U{\isacharparenright}{\kern0pt}\isactrlsup {\isasymdagger}{\isacharparenright}{\kern0pt}{\isacharparenright}{\kern0pt}\ {\isachardollar}{\kern0pt}{\isachardollar}{\kern0pt}\ {\isacharparenleft}{\kern0pt}i{\isacharcomma}{\kern0pt}\ j{\isacharparenright}{\kern0pt}\ {\isacharequal}{\kern0pt}\ {\isadigit{1}}\isactrlsub m\ {\isadigit{4}}\ {\isachardollar}{\kern0pt}{\isachardollar}{\kern0pt}\ {\isacharparenleft}{\kern0pt}i{\isacharcomma}{\kern0pt}\ j{\isacharparenright}{\kern0pt}{\isachardoublequoteclose}\isanewline
\ \ \ \ \ \ \ \ \ \ \isacommand{proof}\isamarkupfalse%
\ {\isacharparenleft}{\kern0pt}rule\ disjE{\isacharparenright}{\kern0pt}\isanewline
\ \ \ \ \ \ \ \ \ \ \ \ \isacommand{show}\isamarkupfalse%
\ {\isachardoublequoteopen}j\ {\isacharequal}{\kern0pt}\ {\isadigit{2}}\ {\isasymor}\ j\ {\isacharequal}{\kern0pt}\ {\isadigit{3}}{\isachardoublequoteclose}\ \isacommand{using}\isamarkupfalse%
\ jl{\isadigit{2}}\ \isacommand{by}\isamarkupfalse%
\ this\isanewline
\ \ \ \ \ \ \ \ \ \ \isacommand{next}\isamarkupfalse%
\isanewline
\ \ \ \ \ \ \ \ \ \ \ \ \isacommand{assume}\isamarkupfalse%
\ j{\isadigit{2}}{\isacharcolon}{\kern0pt}{\isachardoublequoteopen}j\ {\isacharequal}{\kern0pt}\ {\isadigit{2}}{\isachardoublequoteclose}\isanewline
\ \ \ \ \ \ \ \ \ \ \ \ \isacommand{show}\isamarkupfalse%
\ {\isachardoublequoteopen}{\isacharparenleft}{\kern0pt}control{\isadigit{2}}\ U\ {\isacharasterisk}{\kern0pt}\ {\isacharparenleft}{\kern0pt}{\isacharparenleft}{\kern0pt}control{\isadigit{2}}\ U{\isacharparenright}{\kern0pt}\isactrlsup {\isasymdagger}{\isacharparenright}{\kern0pt}{\isacharparenright}{\kern0pt}\ {\isachardollar}{\kern0pt}{\isachardollar}{\kern0pt}\ {\isacharparenleft}{\kern0pt}i{\isacharcomma}{\kern0pt}\ j{\isacharparenright}{\kern0pt}\ {\isacharequal}{\kern0pt}\ {\isadigit{1}}\isactrlsub m\ {\isadigit{4}}\ {\isachardollar}{\kern0pt}{\isachardollar}{\kern0pt}\ {\isacharparenleft}{\kern0pt}i{\isacharcomma}{\kern0pt}\ j{\isacharparenright}{\kern0pt}{\isachardoublequoteclose}\isanewline
\ \ \ \ \ \ \ \ \ \ \ \ \isacommand{proof}\isamarkupfalse%
\ {\isacharminus}{\kern0pt}\isanewline
\ \ \ \ \ \ \ \ \ \ \ \ \ \ \isacommand{have}\isamarkupfalse%
\ {\isachardoublequoteopen}{\isacharparenleft}{\kern0pt}control{\isadigit{2}}\ U\ {\isacharasterisk}{\kern0pt}\ {\isacharparenleft}{\kern0pt}{\isacharparenleft}{\kern0pt}control{\isadigit{2}}\ U{\isacharparenright}{\kern0pt}\isactrlsup {\isasymdagger}{\isacharparenright}{\kern0pt}{\isacharparenright}{\kern0pt}\ {\isachardollar}{\kern0pt}{\isachardollar}{\kern0pt}\ {\isacharparenleft}{\kern0pt}{\isadigit{3}}{\isacharcomma}{\kern0pt}{\isadigit{2}}{\isacharparenright}{\kern0pt}\ {\isacharequal}{\kern0pt}\ \isanewline
\ \ \ \ \ \ \ \ \ \ \ \ \ \ \ \ \ \ \ \ {\isacharparenleft}{\kern0pt}control{\isadigit{2}}\ U{\isacharparenright}{\kern0pt}\ {\isachardollar}{\kern0pt}{\isachardollar}{\kern0pt}\ {\isacharparenleft}{\kern0pt}{\isadigit{3}}{\isacharcomma}{\kern0pt}{\isadigit{0}}{\isacharparenright}{\kern0pt}\ {\isacharasterisk}{\kern0pt}\ {\isacharparenleft}{\kern0pt}{\isacharparenleft}{\kern0pt}control{\isadigit{2}}\ U{\isacharparenright}{\kern0pt}\isactrlsup {\isasymdagger}{\isacharparenright}{\kern0pt}\ {\isachardollar}{\kern0pt}{\isachardollar}{\kern0pt}\ {\isacharparenleft}{\kern0pt}{\isadigit{0}}{\isacharcomma}{\kern0pt}{\isadigit{2}}{\isacharparenright}{\kern0pt}\ {\isacharplus}{\kern0pt}\isanewline
\ \ \ \ \ \ \ \ \ \ \ \ \ \ \ \ \ \ \ \ {\isacharparenleft}{\kern0pt}control{\isadigit{2}}\ U{\isacharparenright}{\kern0pt}\ {\isachardollar}{\kern0pt}{\isachardollar}{\kern0pt}\ {\isacharparenleft}{\kern0pt}{\isadigit{3}}{\isacharcomma}{\kern0pt}{\isadigit{1}}{\isacharparenright}{\kern0pt}\ {\isacharasterisk}{\kern0pt}\ {\isacharparenleft}{\kern0pt}{\isacharparenleft}{\kern0pt}control{\isadigit{2}}\ U{\isacharparenright}{\kern0pt}\isactrlsup {\isasymdagger}{\isacharparenright}{\kern0pt}\ {\isachardollar}{\kern0pt}{\isachardollar}{\kern0pt}\ {\isacharparenleft}{\kern0pt}{\isadigit{1}}{\isacharcomma}{\kern0pt}{\isadigit{2}}{\isacharparenright}{\kern0pt}\ {\isacharplus}{\kern0pt}\isanewline
\ \ \ \ \ \ \ \ \ \ \ \ \ \ \ \ \ \ \ \ {\isacharparenleft}{\kern0pt}control{\isadigit{2}}\ U{\isacharparenright}{\kern0pt}\ {\isachardollar}{\kern0pt}{\isachardollar}{\kern0pt}\ {\isacharparenleft}{\kern0pt}{\isadigit{3}}{\isacharcomma}{\kern0pt}{\isadigit{2}}{\isacharparenright}{\kern0pt}\ {\isacharasterisk}{\kern0pt}\ {\isacharparenleft}{\kern0pt}{\isacharparenleft}{\kern0pt}control{\isadigit{2}}\ U{\isacharparenright}{\kern0pt}\isactrlsup {\isasymdagger}{\isacharparenright}{\kern0pt}\ {\isachardollar}{\kern0pt}{\isachardollar}{\kern0pt}\ {\isacharparenleft}{\kern0pt}{\isadigit{2}}{\isacharcomma}{\kern0pt}{\isadigit{2}}{\isacharparenright}{\kern0pt}\ {\isacharplus}{\kern0pt}\isanewline
\ \ \ \ \ \ \ \ \ \ \ \ \ \ \ \ \ \ \ \ {\isacharparenleft}{\kern0pt}control{\isadigit{2}}\ U{\isacharparenright}{\kern0pt}\ {\isachardollar}{\kern0pt}{\isachardollar}{\kern0pt}\ {\isacharparenleft}{\kern0pt}{\isadigit{3}}{\isacharcomma}{\kern0pt}{\isadigit{3}}{\isacharparenright}{\kern0pt}\ {\isacharasterisk}{\kern0pt}\ {\isacharparenleft}{\kern0pt}{\isacharparenleft}{\kern0pt}control{\isadigit{2}}\ U{\isacharparenright}{\kern0pt}\isactrlsup {\isasymdagger}{\isacharparenright}{\kern0pt}\ {\isachardollar}{\kern0pt}{\isachardollar}{\kern0pt}\ {\isacharparenleft}{\kern0pt}{\isadigit{3}}{\isacharcomma}{\kern0pt}{\isadigit{2}}{\isacharparenright}{\kern0pt}{\isachardoublequoteclose}\isanewline
\ \ \ \ \ \ \ \ \ \ \ \ \ \ \ \ \isacommand{using}\isamarkupfalse%
\ times{\isacharunderscore}{\kern0pt}mat{\isacharunderscore}{\kern0pt}def\ sumof{\isadigit{4}}\isanewline
\ \ \ \ \ \ \ \ \ \ \ \ \ \ \ \ \ \ \isacommand{by}\isamarkupfalse%
\ {\isacharparenleft}{\kern0pt}smt\ {\isacharparenleft}{\kern0pt}z{\isadigit{3}}{\isacharparenright}{\kern0pt}\ carrier{\isacharunderscore}{\kern0pt}matD{\isacharparenleft}{\kern0pt}{\isadigit{1}}{\isacharparenright}{\kern0pt}\ carrier{\isacharunderscore}{\kern0pt}matD{\isacharparenleft}{\kern0pt}{\isadigit{2}}{\isacharparenright}{\kern0pt}\ control{\isadigit{2}}{\isacharunderscore}{\kern0pt}carrier{\isacharunderscore}{\kern0pt}mat\ dim{\isacharunderscore}{\kern0pt}col{\isacharunderscore}{\kern0pt}of{\isacharunderscore}{\kern0pt}dagger\ \isanewline
\ \ \ \ \ \ \ \ \ \ \ \ \ \ \ \ \ \ \ \ \ \ \ \ dim{\isacharunderscore}{\kern0pt}row{\isacharunderscore}{\kern0pt}of{\isacharunderscore}{\kern0pt}dagger\ i{\isadigit{3}}\ i{\isadigit{4}}\ index{\isacharunderscore}{\kern0pt}matrix{\isacharunderscore}{\kern0pt}prod\ j{\isadigit{2}}\ j{\isadigit{4}}{\isacharparenright}{\kern0pt}\isanewline
\ \ \ \ \ \ \ \ \ \ \ \ \ \ \isacommand{also}\isamarkupfalse%
\ \isacommand{have}\isamarkupfalse%
\ {\isachardoublequoteopen}{\isasymdots}\ {\isacharequal}{\kern0pt}\ {\isacharparenleft}{\kern0pt}control{\isadigit{2}}\ U{\isacharparenright}{\kern0pt}\ {\isachardollar}{\kern0pt}{\isachardollar}{\kern0pt}\ {\isacharparenleft}{\kern0pt}{\isadigit{3}}{\isacharcomma}{\kern0pt}{\isadigit{1}}{\isacharparenright}{\kern0pt}\ {\isacharasterisk}{\kern0pt}\ {\isacharparenleft}{\kern0pt}{\isacharparenleft}{\kern0pt}control{\isadigit{2}}\ U{\isacharparenright}{\kern0pt}\isactrlsup {\isasymdagger}{\isacharparenright}{\kern0pt}\ {\isachardollar}{\kern0pt}{\isachardollar}{\kern0pt}\ {\isacharparenleft}{\kern0pt}{\isadigit{1}}{\isacharcomma}{\kern0pt}{\isadigit{2}}{\isacharparenright}{\kern0pt}\ {\isacharplus}{\kern0pt}\ \isanewline
\ \ \ \ \ \ \ \ \ \ \ \ \ \ \ \ \ \ \ \ \ \ \ \ \ \ \ \ \ \ \ \ {\isacharparenleft}{\kern0pt}control{\isadigit{2}}\ U{\isacharparenright}{\kern0pt}\ {\isachardollar}{\kern0pt}{\isachardollar}{\kern0pt}\ {\isacharparenleft}{\kern0pt}{\isadigit{3}}{\isacharcomma}{\kern0pt}{\isadigit{3}}{\isacharparenright}{\kern0pt}\ {\isacharasterisk}{\kern0pt}\ {\isacharparenleft}{\kern0pt}{\isacharparenleft}{\kern0pt}control{\isadigit{2}}\ U{\isacharparenright}{\kern0pt}\isactrlsup {\isasymdagger}{\isacharparenright}{\kern0pt}\ {\isachardollar}{\kern0pt}{\isachardollar}{\kern0pt}\ {\isacharparenleft}{\kern0pt}{\isadigit{3}}{\isacharcomma}{\kern0pt}{\isadigit{2}}{\isacharparenright}{\kern0pt}{\isachardoublequoteclose}\isanewline
\ \ \ \ \ \ \ \ \ \ \ \ \ \ \ \ \ \ \isacommand{using}\isamarkupfalse%
\ control{\isadigit{2}}{\isacharunderscore}{\kern0pt}def\ index{\isacharunderscore}{\kern0pt}mat{\isacharunderscore}{\kern0pt}of{\isacharunderscore}{\kern0pt}cols{\isacharunderscore}{\kern0pt}list\ \isacommand{by}\isamarkupfalse%
\ force\isanewline
\ \ \ \ \ \ \ \ \ \ \ \ \ \ \isacommand{also}\isamarkupfalse%
\ \isacommand{have}\isamarkupfalse%
\ {\isachardoublequoteopen}{\isasymdots}\ {\isacharequal}{\kern0pt}\ {\isacharparenleft}{\kern0pt}control{\isadigit{2}}\ U{\isacharparenright}{\kern0pt}\ {\isachardollar}{\kern0pt}{\isachardollar}{\kern0pt}\ {\isacharparenleft}{\kern0pt}{\isadigit{3}}{\isacharcomma}{\kern0pt}{\isadigit{1}}{\isacharparenright}{\kern0pt}\ {\isacharasterisk}{\kern0pt}\ {\isacharparenleft}{\kern0pt}cnj\ {\isacharparenleft}{\kern0pt}{\isacharparenleft}{\kern0pt}control{\isadigit{2}}\ U{\isacharparenright}{\kern0pt}\ {\isachardollar}{\kern0pt}{\isachardollar}{\kern0pt}\ {\isacharparenleft}{\kern0pt}{\isadigit{2}}{\isacharcomma}{\kern0pt}{\isadigit{1}}{\isacharparenright}{\kern0pt}{\isacharparenright}{\kern0pt}{\isacharparenright}{\kern0pt}\ {\isacharplus}{\kern0pt}\ \isanewline
\ \ \ \ \ \ \ \ \ \ \ \ \ \ \ \ \ \ \ \ \ \ \ \ \ \ \ \ \ \ \ \ {\isacharparenleft}{\kern0pt}control{\isadigit{2}}\ U{\isacharparenright}{\kern0pt}\ {\isachardollar}{\kern0pt}{\isachardollar}{\kern0pt}\ {\isacharparenleft}{\kern0pt}{\isadigit{3}}{\isacharcomma}{\kern0pt}{\isadigit{3}}{\isacharparenright}{\kern0pt}\ {\isacharasterisk}{\kern0pt}\ {\isacharparenleft}{\kern0pt}cnj\ {\isacharparenleft}{\kern0pt}{\isacharparenleft}{\kern0pt}control{\isadigit{2}}\ U{\isacharparenright}{\kern0pt}\ {\isachardollar}{\kern0pt}{\isachardollar}{\kern0pt}\ {\isacharparenleft}{\kern0pt}{\isadigit{2}}{\isacharcomma}{\kern0pt}{\isadigit{3}}{\isacharparenright}{\kern0pt}{\isacharparenright}{\kern0pt}{\isacharparenright}{\kern0pt}{\isachardoublequoteclose}\isanewline
\ \ \ \ \ \ \ \ \ \ \ \ \ \ \ \ \ \ \isacommand{using}\isamarkupfalse%
\ dagger{\isacharunderscore}{\kern0pt}def\ Tensor{\isachardot}{\kern0pt}mat{\isacharunderscore}{\kern0pt}of{\isacharunderscore}{\kern0pt}cols{\isacharunderscore}{\kern0pt}list{\isacharunderscore}{\kern0pt}def\ control{\isadigit{2}}{\isacharunderscore}{\kern0pt}def\ \isacommand{by}\isamarkupfalse%
\ auto\isanewline
\ \ \ \ \ \ \ \ \ \ \ \ \ \ \isacommand{also}\isamarkupfalse%
\ \isacommand{have}\isamarkupfalse%
\ {\isachardoublequoteopen}{\isasymdots}\ {\isacharequal}{\kern0pt}\ {\isacharparenleft}{\kern0pt}control{\isadigit{2}}\ U{\isacharparenright}{\kern0pt}\ {\isachardollar}{\kern0pt}{\isachardollar}{\kern0pt}\ {\isacharparenleft}{\kern0pt}{\isadigit{3}}{\isacharcomma}{\kern0pt}{\isadigit{1}}{\isacharparenright}{\kern0pt}\ {\isacharasterisk}{\kern0pt}\ {\isacharparenleft}{\kern0pt}cnj\ {\isadigit{0}}{\isacharparenright}{\kern0pt}\ {\isacharplus}{\kern0pt}\isanewline
\ \ \ \ \ \ \ \ \ \ \ \ \ \ \ \ \ \ \ \ \ \ \ \ \ \ \ \ \ \ \ \ {\isacharparenleft}{\kern0pt}control{\isadigit{2}}\ U{\isacharparenright}{\kern0pt}\ {\isachardollar}{\kern0pt}{\isachardollar}{\kern0pt}\ {\isacharparenleft}{\kern0pt}{\isadigit{3}}{\isacharcomma}{\kern0pt}{\isadigit{3}}{\isacharparenright}{\kern0pt}\ {\isacharasterisk}{\kern0pt}\ {\isacharparenleft}{\kern0pt}cnj\ {\isadigit{0}}{\isacharparenright}{\kern0pt}{\isachardoublequoteclose}\isanewline
\ \ \ \ \ \ \ \ \ \ \ \ \ \ \ \ \ \ \isacommand{using}\isamarkupfalse%
\ control{\isadigit{2}}{\isacharunderscore}{\kern0pt}def\ index{\isacharunderscore}{\kern0pt}mat{\isacharunderscore}{\kern0pt}of{\isacharunderscore}{\kern0pt}cols{\isacharunderscore}{\kern0pt}list\ \isacommand{by}\isamarkupfalse%
\ simp\isanewline
\ \ \ \ \ \ \ \ \ \ \ \ \ \ \isacommand{also}\isamarkupfalse%
\ \isacommand{have}\isamarkupfalse%
\ {\isachardoublequoteopen}{\isasymdots}\ {\isacharequal}{\kern0pt}\ {\isadigit{0}}{\isachardoublequoteclose}\ \isacommand{by}\isamarkupfalse%
\ auto\isanewline
\ \ \ \ \ \ \ \ \ \ \ \ \ \ \isacommand{also}\isamarkupfalse%
\ \isacommand{have}\isamarkupfalse%
\ {\isachardoublequoteopen}{\isasymdots}\ {\isacharequal}{\kern0pt}\ {\isadigit{1}}\isactrlsub m\ {\isadigit{4}}\ {\isachardollar}{\kern0pt}{\isachardollar}{\kern0pt}\ {\isacharparenleft}{\kern0pt}{\isadigit{3}}{\isacharcomma}{\kern0pt}{\isadigit{2}}{\isacharparenright}{\kern0pt}{\isachardoublequoteclose}\ \isacommand{by}\isamarkupfalse%
\ simp\isanewline
\ \ \ \ \ \ \ \ \ \ \ \ \ \ \isacommand{finally}\isamarkupfalse%
\ \isacommand{show}\isamarkupfalse%
\ {\isacharquery}{\kern0pt}thesis\ \isacommand{using}\isamarkupfalse%
\ i{\isadigit{3}}\ j{\isadigit{2}}\ \isacommand{by}\isamarkupfalse%
\ simp\isanewline
\ \ \ \ \ \ \ \ \ \ \ \ \isacommand{qed}\isamarkupfalse%
\isanewline
\ \ \ \ \ \ \ \ \ \ \isacommand{next}\isamarkupfalse%
\isanewline
\ \ \ \ \ \ \ \ \ \ \ \ \isacommand{assume}\isamarkupfalse%
\ j{\isadigit{3}}{\isacharcolon}{\kern0pt}{\isachardoublequoteopen}j\ {\isacharequal}{\kern0pt}\ {\isadigit{3}}{\isachardoublequoteclose}\isanewline
\ \ \ \ \ \ \ \ \ \ \ \ \isacommand{show}\isamarkupfalse%
\ {\isachardoublequoteopen}{\isacharparenleft}{\kern0pt}control{\isadigit{2}}\ U\ {\isacharasterisk}{\kern0pt}\ {\isacharparenleft}{\kern0pt}{\isacharparenleft}{\kern0pt}control{\isadigit{2}}\ U{\isacharparenright}{\kern0pt}\isactrlsup {\isasymdagger}{\isacharparenright}{\kern0pt}{\isacharparenright}{\kern0pt}\ {\isachardollar}{\kern0pt}{\isachardollar}{\kern0pt}\ {\isacharparenleft}{\kern0pt}i{\isacharcomma}{\kern0pt}\ j{\isacharparenright}{\kern0pt}\ {\isacharequal}{\kern0pt}\ {\isadigit{1}}\isactrlsub m\ {\isadigit{4}}\ {\isachardollar}{\kern0pt}{\isachardollar}{\kern0pt}\ {\isacharparenleft}{\kern0pt}i{\isacharcomma}{\kern0pt}\ j{\isacharparenright}{\kern0pt}{\isachardoublequoteclose}\isanewline
\ \ \ \ \ \ \ \ \ \ \ \ \isacommand{proof}\isamarkupfalse%
\ {\isacharminus}{\kern0pt}\isanewline
\ \ \ \ \ \ \ \ \ \ \ \ \ \ \isacommand{have}\isamarkupfalse%
\ {\isachardoublequoteopen}{\isacharparenleft}{\kern0pt}control{\isadigit{2}}\ U\ {\isacharasterisk}{\kern0pt}\ {\isacharparenleft}{\kern0pt}{\isacharparenleft}{\kern0pt}control{\isadigit{2}}\ U{\isacharparenright}{\kern0pt}\isactrlsup {\isasymdagger}{\isacharparenright}{\kern0pt}{\isacharparenright}{\kern0pt}\ {\isachardollar}{\kern0pt}{\isachardollar}{\kern0pt}\ {\isacharparenleft}{\kern0pt}{\isadigit{3}}{\isacharcomma}{\kern0pt}{\isadigit{3}}{\isacharparenright}{\kern0pt}\ {\isacharequal}{\kern0pt}\ \isanewline
\ \ \ \ \ \ \ \ \ \ \ \ \ \ \ \ \ \ \ \ {\isacharparenleft}{\kern0pt}control{\isadigit{2}}\ U{\isacharparenright}{\kern0pt}\ {\isachardollar}{\kern0pt}{\isachardollar}{\kern0pt}\ {\isacharparenleft}{\kern0pt}{\isadigit{3}}{\isacharcomma}{\kern0pt}{\isadigit{0}}{\isacharparenright}{\kern0pt}\ {\isacharasterisk}{\kern0pt}\ {\isacharparenleft}{\kern0pt}{\isacharparenleft}{\kern0pt}control{\isadigit{2}}\ U{\isacharparenright}{\kern0pt}\isactrlsup {\isasymdagger}{\isacharparenright}{\kern0pt}\ {\isachardollar}{\kern0pt}{\isachardollar}{\kern0pt}\ {\isacharparenleft}{\kern0pt}{\isadigit{0}}{\isacharcomma}{\kern0pt}{\isadigit{3}}{\isacharparenright}{\kern0pt}\ {\isacharplus}{\kern0pt}\isanewline
\ \ \ \ \ \ \ \ \ \ \ \ \ \ \ \ \ \ \ \ {\isacharparenleft}{\kern0pt}control{\isadigit{2}}\ U{\isacharparenright}{\kern0pt}\ {\isachardollar}{\kern0pt}{\isachardollar}{\kern0pt}\ {\isacharparenleft}{\kern0pt}{\isadigit{3}}{\isacharcomma}{\kern0pt}{\isadigit{1}}{\isacharparenright}{\kern0pt}\ {\isacharasterisk}{\kern0pt}\ {\isacharparenleft}{\kern0pt}{\isacharparenleft}{\kern0pt}control{\isadigit{2}}\ U{\isacharparenright}{\kern0pt}\isactrlsup {\isasymdagger}{\isacharparenright}{\kern0pt}\ {\isachardollar}{\kern0pt}{\isachardollar}{\kern0pt}\ {\isacharparenleft}{\kern0pt}{\isadigit{1}}{\isacharcomma}{\kern0pt}{\isadigit{3}}{\isacharparenright}{\kern0pt}\ {\isacharplus}{\kern0pt}\isanewline
\ \ \ \ \ \ \ \ \ \ \ \ \ \ \ \ \ \ \ \ {\isacharparenleft}{\kern0pt}control{\isadigit{2}}\ U{\isacharparenright}{\kern0pt}\ {\isachardollar}{\kern0pt}{\isachardollar}{\kern0pt}\ {\isacharparenleft}{\kern0pt}{\isadigit{3}}{\isacharcomma}{\kern0pt}{\isadigit{2}}{\isacharparenright}{\kern0pt}\ {\isacharasterisk}{\kern0pt}\ {\isacharparenleft}{\kern0pt}{\isacharparenleft}{\kern0pt}control{\isadigit{2}}\ U{\isacharparenright}{\kern0pt}\isactrlsup {\isasymdagger}{\isacharparenright}{\kern0pt}\ {\isachardollar}{\kern0pt}{\isachardollar}{\kern0pt}\ {\isacharparenleft}{\kern0pt}{\isadigit{2}}{\isacharcomma}{\kern0pt}{\isadigit{3}}{\isacharparenright}{\kern0pt}\ {\isacharplus}{\kern0pt}\isanewline
\ \ \ \ \ \ \ \ \ \ \ \ \ \ \ \ \ \ \ \ {\isacharparenleft}{\kern0pt}control{\isadigit{2}}\ U{\isacharparenright}{\kern0pt}\ {\isachardollar}{\kern0pt}{\isachardollar}{\kern0pt}\ {\isacharparenleft}{\kern0pt}{\isadigit{3}}{\isacharcomma}{\kern0pt}{\isadigit{3}}{\isacharparenright}{\kern0pt}\ {\isacharasterisk}{\kern0pt}\ {\isacharparenleft}{\kern0pt}{\isacharparenleft}{\kern0pt}control{\isadigit{2}}\ U{\isacharparenright}{\kern0pt}\isactrlsup {\isasymdagger}{\isacharparenright}{\kern0pt}\ {\isachardollar}{\kern0pt}{\isachardollar}{\kern0pt}\ {\isacharparenleft}{\kern0pt}{\isadigit{3}}{\isacharcomma}{\kern0pt}{\isadigit{3}}{\isacharparenright}{\kern0pt}{\isachardoublequoteclose}\isanewline
\ \ \ \ \ \ \ \ \ \ \ \ \ \ \ \ \isacommand{using}\isamarkupfalse%
\ times{\isacharunderscore}{\kern0pt}mat{\isacharunderscore}{\kern0pt}def\ sumof{\isadigit{4}}\isanewline
\ \ \ \ \ \ \ \ \ \ \ \ \ \ \ \ \isacommand{by}\isamarkupfalse%
\ {\isacharparenleft}{\kern0pt}smt\ {\isacharparenleft}{\kern0pt}z{\isadigit{3}}{\isacharparenright}{\kern0pt}\ carrier{\isacharunderscore}{\kern0pt}matD{\isacharparenleft}{\kern0pt}{\isadigit{1}}{\isacharparenright}{\kern0pt}\ carrier{\isacharunderscore}{\kern0pt}matD{\isacharparenleft}{\kern0pt}{\isadigit{2}}{\isacharparenright}{\kern0pt}\ control{\isadigit{2}}{\isacharunderscore}{\kern0pt}carrier{\isacharunderscore}{\kern0pt}mat\ dim{\isacharunderscore}{\kern0pt}col{\isacharunderscore}{\kern0pt}of{\isacharunderscore}{\kern0pt}dagger\ \isanewline
\ \ \ \ \ \ \ \ \ \ \ \ \ \ \ \ \ \ \ \ \ \ \ \ dim{\isacharunderscore}{\kern0pt}row{\isacharunderscore}{\kern0pt}of{\isacharunderscore}{\kern0pt}dagger\ i{\isadigit{3}}\ i{\isadigit{4}}\ index{\isacharunderscore}{\kern0pt}matrix{\isacharunderscore}{\kern0pt}prod\ j{\isadigit{3}}\ j{\isadigit{4}}{\isacharparenright}{\kern0pt}\isanewline
\ \ \ \ \ \ \ \ \ \ \ \ \ \ \isacommand{also}\isamarkupfalse%
\ \isacommand{have}\isamarkupfalse%
\ {\isachardoublequoteopen}{\isasymdots}\ {\isacharequal}{\kern0pt}\ {\isacharparenleft}{\kern0pt}control{\isadigit{2}}\ U{\isacharparenright}{\kern0pt}\ {\isachardollar}{\kern0pt}{\isachardollar}{\kern0pt}\ {\isacharparenleft}{\kern0pt}{\isadigit{3}}{\isacharcomma}{\kern0pt}{\isadigit{1}}{\isacharparenright}{\kern0pt}\ {\isacharasterisk}{\kern0pt}\ {\isacharparenleft}{\kern0pt}{\isacharparenleft}{\kern0pt}control{\isadigit{2}}\ U{\isacharparenright}{\kern0pt}\isactrlsup {\isasymdagger}{\isacharparenright}{\kern0pt}\ {\isachardollar}{\kern0pt}{\isachardollar}{\kern0pt}\ {\isacharparenleft}{\kern0pt}{\isadigit{1}}{\isacharcomma}{\kern0pt}{\isadigit{3}}{\isacharparenright}{\kern0pt}\ {\isacharplus}{\kern0pt}\ \isanewline
\ \ \ \ \ \ \ \ \ \ \ \ \ \ \ \ \ \ \ \ \ \ \ \ \ \ \ \ \ \ \ \ {\isacharparenleft}{\kern0pt}control{\isadigit{2}}\ U{\isacharparenright}{\kern0pt}\ {\isachardollar}{\kern0pt}{\isachardollar}{\kern0pt}\ {\isacharparenleft}{\kern0pt}{\isadigit{3}}{\isacharcomma}{\kern0pt}{\isadigit{3}}{\isacharparenright}{\kern0pt}\ {\isacharasterisk}{\kern0pt}\ {\isacharparenleft}{\kern0pt}{\isacharparenleft}{\kern0pt}control{\isadigit{2}}\ U{\isacharparenright}{\kern0pt}\isactrlsup {\isasymdagger}{\isacharparenright}{\kern0pt}\ {\isachardollar}{\kern0pt}{\isachardollar}{\kern0pt}\ {\isacharparenleft}{\kern0pt}{\isadigit{3}}{\isacharcomma}{\kern0pt}{\isadigit{3}}{\isacharparenright}{\kern0pt}{\isachardoublequoteclose}\isanewline
\ \ \ \ \ \ \ \ \ \ \ \ \ \ \ \ \ \ \isacommand{using}\isamarkupfalse%
\ control{\isadigit{2}}{\isacharunderscore}{\kern0pt}def\ index{\isacharunderscore}{\kern0pt}mat{\isacharunderscore}{\kern0pt}of{\isacharunderscore}{\kern0pt}cols{\isacharunderscore}{\kern0pt}list\ \isacommand{by}\isamarkupfalse%
\ force\isanewline
\ \ \ \ \ \ \ \ \ \ \ \ \ \ \isacommand{also}\isamarkupfalse%
\ \isacommand{have}\isamarkupfalse%
\ {\isachardoublequoteopen}{\isasymdots}\ {\isacharequal}{\kern0pt}\ {\isacharparenleft}{\kern0pt}control{\isadigit{2}}\ U{\isacharparenright}{\kern0pt}\ {\isachardollar}{\kern0pt}{\isachardollar}{\kern0pt}\ {\isacharparenleft}{\kern0pt}{\isadigit{3}}{\isacharcomma}{\kern0pt}{\isadigit{1}}{\isacharparenright}{\kern0pt}\ {\isacharasterisk}{\kern0pt}\ {\isacharparenleft}{\kern0pt}cnj\ {\isacharparenleft}{\kern0pt}{\isacharparenleft}{\kern0pt}control{\isadigit{2}}\ U{\isacharparenright}{\kern0pt}\ {\isachardollar}{\kern0pt}{\isachardollar}{\kern0pt}\ {\isacharparenleft}{\kern0pt}{\isadigit{3}}{\isacharcomma}{\kern0pt}{\isadigit{1}}{\isacharparenright}{\kern0pt}{\isacharparenright}{\kern0pt}{\isacharparenright}{\kern0pt}\ {\isacharplus}{\kern0pt}\ \isanewline
\ \ \ \ \ \ \ \ \ \ \ \ \ \ \ \ \ \ \ \ \ \ \ \ \ \ \ \ \ \ \ \ {\isacharparenleft}{\kern0pt}control{\isadigit{2}}\ U{\isacharparenright}{\kern0pt}\ {\isachardollar}{\kern0pt}{\isachardollar}{\kern0pt}\ {\isacharparenleft}{\kern0pt}{\isadigit{3}}{\isacharcomma}{\kern0pt}{\isadigit{3}}{\isacharparenright}{\kern0pt}\ {\isacharasterisk}{\kern0pt}\ {\isacharparenleft}{\kern0pt}cnj\ {\isacharparenleft}{\kern0pt}{\isacharparenleft}{\kern0pt}control{\isadigit{2}}\ U{\isacharparenright}{\kern0pt}\ {\isachardollar}{\kern0pt}{\isachardollar}{\kern0pt}\ {\isacharparenleft}{\kern0pt}{\isadigit{3}}{\isacharcomma}{\kern0pt}{\isadigit{3}}{\isacharparenright}{\kern0pt}{\isacharparenright}{\kern0pt}{\isacharparenright}{\kern0pt}{\isachardoublequoteclose}\isanewline
\ \ \ \ \ \ \ \ \ \ \ \ \ \ \ \ \ \ \isacommand{using}\isamarkupfalse%
\ dagger{\isacharunderscore}{\kern0pt}def\ Tensor{\isachardot}{\kern0pt}mat{\isacharunderscore}{\kern0pt}of{\isacharunderscore}{\kern0pt}cols{\isacharunderscore}{\kern0pt}list{\isacharunderscore}{\kern0pt}def\ control{\isadigit{2}}{\isacharunderscore}{\kern0pt}def\ \isacommand{by}\isamarkupfalse%
\ auto\isanewline
\ \ \ \ \ \ \ \ \ \ \ \ \ \ \isacommand{also}\isamarkupfalse%
\ \isacommand{have}\isamarkupfalse%
\ {\isachardoublequoteopen}{\isasymdots}\ {\isacharequal}{\kern0pt}\ U\ {\isachardollar}{\kern0pt}{\isachardollar}{\kern0pt}\ {\isacharparenleft}{\kern0pt}{\isadigit{1}}{\isacharcomma}{\kern0pt}{\isadigit{0}}{\isacharparenright}{\kern0pt}\ {\isacharasterisk}{\kern0pt}\ {\isacharparenleft}{\kern0pt}cnj\ {\isacharparenleft}{\kern0pt}U\ {\isachardollar}{\kern0pt}{\isachardollar}{\kern0pt}\ {\isacharparenleft}{\kern0pt}{\isadigit{1}}{\isacharcomma}{\kern0pt}{\isadigit{0}}{\isacharparenright}{\kern0pt}{\isacharparenright}{\kern0pt}{\isacharparenright}{\kern0pt}\ {\isacharplus}{\kern0pt}\isanewline
\ \ \ \ \ \ \ \ \ \ \ \ \ \ \ \ \ \ \ \ \ \ \ \ \ \ \ \ \ \ U\ {\isachardollar}{\kern0pt}{\isachardollar}{\kern0pt}\ {\isacharparenleft}{\kern0pt}{\isadigit{1}}{\isacharcomma}{\kern0pt}{\isadigit{1}}{\isacharparenright}{\kern0pt}\ {\isacharasterisk}{\kern0pt}\ {\isacharparenleft}{\kern0pt}cnj\ {\isacharparenleft}{\kern0pt}U\ {\isachardollar}{\kern0pt}{\isachardollar}{\kern0pt}\ {\isacharparenleft}{\kern0pt}{\isadigit{1}}{\isacharcomma}{\kern0pt}{\isadigit{1}}{\isacharparenright}{\kern0pt}{\isacharparenright}{\kern0pt}{\isacharparenright}{\kern0pt}{\isachardoublequoteclose}\isanewline
\ \ \ \ \ \ \ \ \ \ \ \ \ \ \ \ \isacommand{using}\isamarkupfalse%
\ control{\isadigit{2}}{\isacharunderscore}{\kern0pt}def\ index{\isacharunderscore}{\kern0pt}mat{\isacharunderscore}{\kern0pt}of{\isacharunderscore}{\kern0pt}cols{\isacharunderscore}{\kern0pt}list\ \isacommand{by}\isamarkupfalse%
\ simp\isanewline
\ \ \ \ \ \ \ \ \ \ \ \ \ \ \isacommand{also}\isamarkupfalse%
\ \isacommand{have}\isamarkupfalse%
\ {\isachardoublequoteopen}{\isasymdots}\ {\isacharequal}{\kern0pt}\ {\isacharparenleft}{\kern0pt}U\ {\isachardollar}{\kern0pt}{\isachardollar}{\kern0pt}\ {\isacharparenleft}{\kern0pt}{\isadigit{1}}{\isacharcomma}{\kern0pt}{\isadigit{0}}{\isacharparenright}{\kern0pt}{\isacharparenright}{\kern0pt}\ {\isacharasterisk}{\kern0pt}\ {\isacharparenleft}{\kern0pt}{\isacharparenleft}{\kern0pt}U\isactrlsup {\isasymdagger}{\isacharparenright}{\kern0pt}\ {\isachardollar}{\kern0pt}{\isachardollar}{\kern0pt}\ {\isacharparenleft}{\kern0pt}{\isadigit{0}}{\isacharcomma}{\kern0pt}{\isadigit{1}}{\isacharparenright}{\kern0pt}{\isacharparenright}{\kern0pt}\ {\isacharplus}{\kern0pt}\isanewline
\ \ \ \ \ \ \ \ \ \ \ \ \ \ \ \ \ \ \ \ \ \ \ \ \ \ \ \ \ \ {\isacharparenleft}{\kern0pt}U\ {\isachardollar}{\kern0pt}{\isachardollar}{\kern0pt}\ {\isacharparenleft}{\kern0pt}{\isadigit{1}}{\isacharcomma}{\kern0pt}{\isadigit{1}}{\isacharparenright}{\kern0pt}{\isacharparenright}{\kern0pt}\ {\isacharasterisk}{\kern0pt}\ {\isacharparenleft}{\kern0pt}{\isacharparenleft}{\kern0pt}U\isactrlsup {\isasymdagger}{\isacharparenright}{\kern0pt}\ {\isachardollar}{\kern0pt}{\isachardollar}{\kern0pt}\ {\isacharparenleft}{\kern0pt}{\isadigit{1}}{\isacharcomma}{\kern0pt}{\isadigit{1}}{\isacharparenright}{\kern0pt}{\isacharparenright}{\kern0pt}{\isachardoublequoteclose}\isanewline
\ \ \ \ \ \ \ \ \ \ \ \ \ \ \ \ \isacommand{using}\isamarkupfalse%
\ dagger{\isacharunderscore}{\kern0pt}def\ assms{\isacharparenleft}{\kern0pt}{\isadigit{1}}{\isacharparenright}{\kern0pt}\ gate{\isacharunderscore}{\kern0pt}def\ \isacommand{by}\isamarkupfalse%
\ force\isanewline
\ \ \ \ \ \ \ \ \ \ \ \ \ \ \isacommand{also}\isamarkupfalse%
\ \isacommand{have}\isamarkupfalse%
\ {\isachardoublequoteopen}{\isasymdots}\ {\isacharequal}{\kern0pt}\ {\isacharparenleft}{\kern0pt}U\ {\isacharasterisk}{\kern0pt}\ {\isacharparenleft}{\kern0pt}U\isactrlsup {\isasymdagger}{\isacharparenright}{\kern0pt}{\isacharparenright}{\kern0pt}\ {\isachardollar}{\kern0pt}{\isachardollar}{\kern0pt}\ {\isacharparenleft}{\kern0pt}{\isadigit{1}}{\isacharcomma}{\kern0pt}{\isadigit{1}}{\isacharparenright}{\kern0pt}{\isachardoublequoteclose}\ \isanewline
\ \ \ \ \ \ \ \ \ \ \ \ \ \ \ \ \isacommand{using}\isamarkupfalse%
\ times{\isacharunderscore}{\kern0pt}mat{\isacharunderscore}{\kern0pt}def\ assms{\isacharparenleft}{\kern0pt}{\isadigit{1}}{\isacharparenright}{\kern0pt}\ gate{\isacharunderscore}{\kern0pt}carrier{\isacharunderscore}{\kern0pt}mat\ sumof{\isadigit{2}}\isanewline
\ \ \ \ \ \ \ \ \ \ \ \ \ \ \ \ \isacommand{by}\isamarkupfalse%
\ {\isacharparenleft}{\kern0pt}smt\ {\isacharparenleft}{\kern0pt}z{\isadigit{3}}{\isacharparenright}{\kern0pt}\ Suc{\isacharunderscore}{\kern0pt}{\isadigit{1}}\ carrier{\isacharunderscore}{\kern0pt}matD{\isacharparenleft}{\kern0pt}{\isadigit{2}}{\isacharparenright}{\kern0pt}\ dagger{\isacharunderscore}{\kern0pt}def\ dim{\isacharunderscore}{\kern0pt}col{\isacharunderscore}{\kern0pt}mat{\isacharparenleft}{\kern0pt}{\isadigit{1}}{\isacharparenright}{\kern0pt}\ dim{\isacharunderscore}{\kern0pt}row{\isacharunderscore}{\kern0pt}of{\isacharunderscore}{\kern0pt}dagger\ \isanewline
\ \ \ \ \ \ \ \ \ \ \ \ \ \ \ \ \ \ \ \ gate{\isachardot}{\kern0pt}dim{\isacharunderscore}{\kern0pt}row\ index{\isacharunderscore}{\kern0pt}matrix{\isacharunderscore}{\kern0pt}prod\ lessI\ pos{\isadigit{2}}\ power{\isacharunderscore}{\kern0pt}one{\isacharunderscore}{\kern0pt}right{\isacharparenright}{\kern0pt}\isanewline
\ \ \ \ \ \ \ \ \ \ \ \ \ \ \isacommand{also}\isamarkupfalse%
\ \isacommand{have}\isamarkupfalse%
\ {\isachardoublequoteopen}{\isasymdots}\ {\isacharequal}{\kern0pt}\ {\isacharparenleft}{\kern0pt}{\isadigit{1}}\isactrlsub m\ {\isadigit{2}}{\isacharparenright}{\kern0pt}\ {\isachardollar}{\kern0pt}{\isachardollar}{\kern0pt}\ {\isacharparenleft}{\kern0pt}{\isadigit{1}}{\isacharcomma}{\kern0pt}{\isadigit{1}}{\isacharparenright}{\kern0pt}{\isachardoublequoteclose}\ \isacommand{using}\isamarkupfalse%
\ assms{\isacharparenleft}{\kern0pt}{\isadigit{1}}{\isacharparenright}{\kern0pt}\ gate{\isacharunderscore}{\kern0pt}def\ unitary{\isacharunderscore}{\kern0pt}def\ \isacommand{by}\isamarkupfalse%
\ auto\isanewline
\ \ \ \ \ \ \ \ \ \ \ \ \ \ \isacommand{also}\isamarkupfalse%
\ \isacommand{have}\isamarkupfalse%
\ {\isachardoublequoteopen}{\isasymdots}\ {\isacharequal}{\kern0pt}\ {\isadigit{1}}{\isachardoublequoteclose}\ \isacommand{by}\isamarkupfalse%
\ auto\isanewline
\ \ \ \ \ \ \ \ \ \ \ \ \ \ \isacommand{also}\isamarkupfalse%
\ \isacommand{have}\isamarkupfalse%
\ {\isachardoublequoteopen}{\isasymdots}\ {\isacharequal}{\kern0pt}\ {\isadigit{1}}\isactrlsub m\ {\isadigit{4}}\ {\isachardollar}{\kern0pt}{\isachardollar}{\kern0pt}\ {\isacharparenleft}{\kern0pt}{\isadigit{3}}{\isacharcomma}{\kern0pt}{\isadigit{3}}{\isacharparenright}{\kern0pt}{\isachardoublequoteclose}\ \isacommand{by}\isamarkupfalse%
\ simp\isanewline
\ \ \ \ \ \ \ \ \ \ \ \ \ \ \isacommand{finally}\isamarkupfalse%
\ \isacommand{show}\isamarkupfalse%
\ {\isacharquery}{\kern0pt}thesis\ \isacommand{using}\isamarkupfalse%
\ i{\isadigit{3}}\ j{\isadigit{3}}\ \isacommand{by}\isamarkupfalse%
\ simp\isanewline
\ \ \ \ \ \ \ \ \ \ \ \ \isacommand{qed}\isamarkupfalse%
\isanewline
\ \ \ \ \ \ \ \ \ \ \isacommand{qed}\isamarkupfalse%
\isanewline
\ \ \ \ \ \ \ \ \isacommand{qed}\isamarkupfalse%
\isanewline
\ \ \ \ \ \ \isacommand{qed}\isamarkupfalse%
\isanewline
\ \ \ \ \isacommand{qed}\isamarkupfalse%
\isanewline
\ \ \isacommand{qed}\isamarkupfalse%
\isanewline
\isacommand{qed}\isamarkupfalse%
\isanewline
\isacommand{qed}\isamarkupfalse%
\isanewline
\isacommand{next}\isamarkupfalse%
\isanewline
\ \ \isacommand{show}\isamarkupfalse%
\ {\isachardoublequoteopen}dim{\isacharunderscore}{\kern0pt}row\ {\isacharparenleft}{\kern0pt}control{\isadigit{2}}\ U\ {\isacharasterisk}{\kern0pt}\ {\isacharparenleft}{\kern0pt}{\isacharparenleft}{\kern0pt}control{\isadigit{2}}\ U{\isacharparenright}{\kern0pt}\isactrlsup {\isasymdagger}{\isacharparenright}{\kern0pt}{\isacharparenright}{\kern0pt}\ {\isacharequal}{\kern0pt}\ dim{\isacharunderscore}{\kern0pt}row\ {\isacharparenleft}{\kern0pt}{\isadigit{1}}\isactrlsub m\ {\isadigit{4}}{\isacharparenright}{\kern0pt}{\isachardoublequoteclose}\isanewline
\ \ \ \ \isacommand{by}\isamarkupfalse%
\ {\isacharparenleft}{\kern0pt}metis\ carrier{\isacharunderscore}{\kern0pt}matD{\isacharparenleft}{\kern0pt}{\isadigit{1}}{\isacharparenright}{\kern0pt}\ control{\isadigit{2}}{\isacharunderscore}{\kern0pt}carrier{\isacharunderscore}{\kern0pt}mat\ index{\isacharunderscore}{\kern0pt}mult{\isacharunderscore}{\kern0pt}mat{\isacharparenleft}{\kern0pt}{\isadigit{2}}{\isacharparenright}{\kern0pt}\ index{\isacharunderscore}{\kern0pt}one{\isacharunderscore}{\kern0pt}mat{\isacharparenleft}{\kern0pt}{\isadigit{2}}{\isacharparenright}{\kern0pt}{\isacharparenright}{\kern0pt}\isanewline
\isacommand{next}\isamarkupfalse%
\isanewline
\ \ \isacommand{show}\isamarkupfalse%
\ {\isachardoublequoteopen}dim{\isacharunderscore}{\kern0pt}col\ {\isacharparenleft}{\kern0pt}control{\isadigit{2}}\ U\ {\isacharasterisk}{\kern0pt}\ {\isacharparenleft}{\kern0pt}{\isacharparenleft}{\kern0pt}control{\isadigit{2}}\ U{\isacharparenright}{\kern0pt}\isactrlsup {\isasymdagger}{\isacharparenright}{\kern0pt}{\isacharparenright}{\kern0pt}\ {\isacharequal}{\kern0pt}\ dim{\isacharunderscore}{\kern0pt}col\ {\isacharparenleft}{\kern0pt}{\isadigit{1}}\isactrlsub m\ {\isadigit{4}}{\isacharparenright}{\kern0pt}{\isachardoublequoteclose}\isanewline
\ \ \ \ \isacommand{by}\isamarkupfalse%
\ {\isacharparenleft}{\kern0pt}metis\ carrier{\isacharunderscore}{\kern0pt}matD{\isacharparenleft}{\kern0pt}{\isadigit{1}}{\isacharparenright}{\kern0pt}\ control{\isadigit{2}}{\isacharunderscore}{\kern0pt}carrier{\isacharunderscore}{\kern0pt}mat\ dim{\isacharunderscore}{\kern0pt}col{\isacharunderscore}{\kern0pt}of{\isacharunderscore}{\kern0pt}dagger\ index{\isacharunderscore}{\kern0pt}mult{\isacharunderscore}{\kern0pt}mat{\isacharparenleft}{\kern0pt}{\isadigit{3}}{\isacharparenright}{\kern0pt}\ \isanewline
\ \ \ \ \ \ \ \ index{\isacharunderscore}{\kern0pt}one{\isacharunderscore}{\kern0pt}mat{\isacharparenleft}{\kern0pt}{\isadigit{3}}{\isacharparenright}{\kern0pt}{\isacharparenright}{\kern0pt}\isanewline
\isacommand{qed}\isamarkupfalse%
%
\endisatagproof
{\isafoldproof}%
%
\isadelimproof
\isanewline
%
\endisadelimproof
\isanewline
\isacommand{lemma}\isamarkupfalse%
\ control{\isadigit{2}}{\isacharunderscore}{\kern0pt}inv{\isacharprime}{\kern0pt}{\isacharcolon}{\kern0pt}\isanewline
\ \ \isakeyword{assumes}\ {\isachardoublequoteopen}gate\ {\isadigit{1}}\ U{\isachardoublequoteclose}\isanewline
\ \ \isakeyword{shows}\ {\isachardoublequoteopen}{\isacharparenleft}{\kern0pt}control{\isadigit{2}}\ U{\isacharparenright}{\kern0pt}\isactrlsup {\isasymdagger}\ {\isacharasterisk}{\kern0pt}\ {\isacharparenleft}{\kern0pt}control{\isadigit{2}}\ U{\isacharparenright}{\kern0pt}\ {\isacharequal}{\kern0pt}\ {\isadigit{1}}\isactrlsub m\ {\isadigit{4}}{\isachardoublequoteclose}\isanewline
%
\isadelimproof
%
\endisadelimproof
%
\isatagproof
\isacommand{proof}\isamarkupfalse%
\isanewline
\ \ \isacommand{show}\isamarkupfalse%
\ {\isachardoublequoteopen}{\isasymAnd}i\ j{\isachardot}{\kern0pt}\ i\ {\isacharless}{\kern0pt}\ dim{\isacharunderscore}{\kern0pt}row\ {\isacharparenleft}{\kern0pt}{\isadigit{1}}\isactrlsub m\ {\isadigit{4}}{\isacharparenright}{\kern0pt}\ {\isasymLongrightarrow}\ j\ {\isacharless}{\kern0pt}\ dim{\isacharunderscore}{\kern0pt}col\ {\isacharparenleft}{\kern0pt}{\isadigit{1}}\isactrlsub m\ {\isadigit{4}}{\isacharparenright}{\kern0pt}\ {\isasymLongrightarrow}\isanewline
\ \ \ \ \ \ \ \ \ \ \ {\isacharparenleft}{\kern0pt}{\isacharparenleft}{\kern0pt}control{\isadigit{2}}\ U{\isacharparenright}{\kern0pt}\isactrlsup {\isasymdagger}\ {\isacharasterisk}{\kern0pt}\ control{\isadigit{2}}\ U{\isacharparenright}{\kern0pt}\ {\isachardollar}{\kern0pt}{\isachardollar}{\kern0pt}\ {\isacharparenleft}{\kern0pt}i{\isacharcomma}{\kern0pt}\ j{\isacharparenright}{\kern0pt}\ {\isacharequal}{\kern0pt}\ {\isadigit{1}}\isactrlsub m\ {\isadigit{4}}\ {\isachardollar}{\kern0pt}{\isachardollar}{\kern0pt}\ {\isacharparenleft}{\kern0pt}i{\isacharcomma}{\kern0pt}\ j{\isacharparenright}{\kern0pt}{\isachardoublequoteclose}\isanewline
\ \ \isacommand{proof}\isamarkupfalse%
\ {\isacharminus}{\kern0pt}\isanewline
\ \ \ \ \isacommand{fix}\isamarkupfalse%
\ i\ j\isanewline
\ \ \ \ \isacommand{assume}\isamarkupfalse%
\ {\isachardoublequoteopen}i\ {\isacharless}{\kern0pt}\ dim{\isacharunderscore}{\kern0pt}row\ {\isacharparenleft}{\kern0pt}{\isadigit{1}}\isactrlsub m\ {\isadigit{4}}{\isacharparenright}{\kern0pt}{\isachardoublequoteclose}\isanewline
\ \ \ \ \isacommand{hence}\isamarkupfalse%
\ i{\isadigit{4}}{\isacharcolon}{\kern0pt}{\isachardoublequoteopen}i\ {\isacharless}{\kern0pt}\ {\isadigit{4}}{\isachardoublequoteclose}\ \isacommand{by}\isamarkupfalse%
\ auto\isanewline
\ \ \ \ \isacommand{assume}\isamarkupfalse%
\ {\isachardoublequoteopen}j\ {\isacharless}{\kern0pt}\ dim{\isacharunderscore}{\kern0pt}col\ {\isacharparenleft}{\kern0pt}{\isadigit{1}}\isactrlsub m\ {\isadigit{4}}{\isacharparenright}{\kern0pt}{\isachardoublequoteclose}\isanewline
\ \ \ \ \isacommand{hence}\isamarkupfalse%
\ j{\isadigit{4}}{\isacharcolon}{\kern0pt}{\isachardoublequoteopen}j\ {\isacharless}{\kern0pt}\ {\isadigit{4}}{\isachardoublequoteclose}\ \isacommand{by}\isamarkupfalse%
\ auto\isanewline
\ \ \ \ \isacommand{show}\isamarkupfalse%
\ {\isachardoublequoteopen}{\isacharparenleft}{\kern0pt}{\isacharparenleft}{\kern0pt}control{\isadigit{2}}\ U{\isacharparenright}{\kern0pt}\isactrlsup {\isasymdagger}\ {\isacharasterisk}{\kern0pt}\ control{\isadigit{2}}\ U{\isacharparenright}{\kern0pt}\ {\isachardollar}{\kern0pt}{\isachardollar}{\kern0pt}\ {\isacharparenleft}{\kern0pt}i{\isacharcomma}{\kern0pt}\ j{\isacharparenright}{\kern0pt}\ {\isacharequal}{\kern0pt}\ {\isadigit{1}}\isactrlsub m\ {\isadigit{4}}\ {\isachardollar}{\kern0pt}{\isachardollar}{\kern0pt}\ {\isacharparenleft}{\kern0pt}i{\isacharcomma}{\kern0pt}\ j{\isacharparenright}{\kern0pt}{\isachardoublequoteclose}\isanewline
\ \ \ \ \isacommand{proof}\isamarkupfalse%
\ {\isacharparenleft}{\kern0pt}rule\ disjE{\isacharparenright}{\kern0pt}\isanewline
\ \ \ \ \ \ \isacommand{show}\isamarkupfalse%
\ {\isachardoublequoteopen}i\ {\isacharequal}{\kern0pt}\ {\isadigit{0}}\ {\isasymor}\ i\ {\isacharequal}{\kern0pt}\ {\isadigit{1}}\ {\isasymor}\ i\ {\isacharequal}{\kern0pt}\ {\isadigit{2}}\ {\isasymor}\ i\ {\isacharequal}{\kern0pt}\ {\isadigit{3}}{\isachardoublequoteclose}\ \isacommand{using}\isamarkupfalse%
\ i{\isadigit{4}}\ \isacommand{by}\isamarkupfalse%
\ auto\isanewline
\ \ \ \ \isacommand{next}\isamarkupfalse%
\isanewline
\ \ \ \ \ \ \isacommand{assume}\isamarkupfalse%
\ i{\isadigit{0}}{\isacharcolon}{\kern0pt}{\isachardoublequoteopen}i\ {\isacharequal}{\kern0pt}\ {\isadigit{0}}{\isachardoublequoteclose}\isanewline
\ \ \ \ \ \ \isacommand{show}\isamarkupfalse%
\ {\isachardoublequoteopen}{\isacharparenleft}{\kern0pt}{\isacharparenleft}{\kern0pt}control{\isadigit{2}}\ U{\isacharparenright}{\kern0pt}\isactrlsup {\isasymdagger}\ {\isacharasterisk}{\kern0pt}\ control{\isadigit{2}}\ U{\isacharparenright}{\kern0pt}\ {\isachardollar}{\kern0pt}{\isachardollar}{\kern0pt}\ {\isacharparenleft}{\kern0pt}i{\isacharcomma}{\kern0pt}\ j{\isacharparenright}{\kern0pt}\ {\isacharequal}{\kern0pt}\ {\isadigit{1}}\isactrlsub m\ {\isadigit{4}}\ {\isachardollar}{\kern0pt}{\isachardollar}{\kern0pt}\ {\isacharparenleft}{\kern0pt}i{\isacharcomma}{\kern0pt}\ j{\isacharparenright}{\kern0pt}{\isachardoublequoteclose}\isanewline
\ \ \ \ \ \ \isacommand{proof}\isamarkupfalse%
\ {\isacharparenleft}{\kern0pt}rule\ disjE{\isacharparenright}{\kern0pt}\isanewline
\ \ \ \ \ \ \ \ \isacommand{show}\isamarkupfalse%
\ {\isachardoublequoteopen}j\ {\isacharequal}{\kern0pt}\ {\isadigit{0}}\ {\isasymor}\ j\ {\isacharequal}{\kern0pt}\ {\isadigit{1}}\ {\isasymor}\ j\ {\isacharequal}{\kern0pt}\ {\isadigit{2}}\ {\isasymor}\ j\ {\isacharequal}{\kern0pt}\ {\isadigit{3}}{\isachardoublequoteclose}\ \isacommand{using}\isamarkupfalse%
\ j{\isadigit{4}}\ \isacommand{by}\isamarkupfalse%
\ auto\isanewline
\ \ \ \ \ \ \isacommand{next}\isamarkupfalse%
\isanewline
\ \ \ \ \ \ \ \ \isacommand{assume}\isamarkupfalse%
\ j{\isadigit{0}}{\isacharcolon}{\kern0pt}{\isachardoublequoteopen}j\ {\isacharequal}{\kern0pt}\ {\isadigit{0}}{\isachardoublequoteclose}\isanewline
\ \ \ \ \ \ \ \ \isacommand{show}\isamarkupfalse%
\ {\isachardoublequoteopen}{\isacharparenleft}{\kern0pt}{\isacharparenleft}{\kern0pt}control{\isadigit{2}}\ U{\isacharparenright}{\kern0pt}\isactrlsup {\isasymdagger}\ {\isacharasterisk}{\kern0pt}\ control{\isadigit{2}}\ U{\isacharparenright}{\kern0pt}\ {\isachardollar}{\kern0pt}{\isachardollar}{\kern0pt}\ {\isacharparenleft}{\kern0pt}i{\isacharcomma}{\kern0pt}\ j{\isacharparenright}{\kern0pt}\ {\isacharequal}{\kern0pt}\ {\isadigit{1}}\isactrlsub m\ {\isadigit{4}}\ {\isachardollar}{\kern0pt}{\isachardollar}{\kern0pt}\ {\isacharparenleft}{\kern0pt}i{\isacharcomma}{\kern0pt}\ j{\isacharparenright}{\kern0pt}{\isachardoublequoteclose}\isanewline
\ \ \ \ \ \ \ \ \isacommand{proof}\isamarkupfalse%
\ {\isacharminus}{\kern0pt}\isanewline
\ \ \ \ \ \ \ \ \ \ \isacommand{have}\isamarkupfalse%
\ {\isachardoublequoteopen}{\isacharparenleft}{\kern0pt}{\isacharparenleft}{\kern0pt}control{\isadigit{2}}\ U{\isacharparenright}{\kern0pt}\isactrlsup {\isasymdagger}\ {\isacharasterisk}{\kern0pt}\ control{\isadigit{2}}\ U{\isacharparenright}{\kern0pt}\ {\isachardollar}{\kern0pt}{\isachardollar}{\kern0pt}\ {\isacharparenleft}{\kern0pt}{\isadigit{0}}{\isacharcomma}{\kern0pt}{\isadigit{0}}{\isacharparenright}{\kern0pt}\ {\isacharequal}{\kern0pt}\isanewline
\ \ \ \ \ \ \ \ \ \ \ \ \ \ \ \ {\isacharparenleft}{\kern0pt}{\isacharparenleft}{\kern0pt}control{\isadigit{2}}\ U{\isacharparenright}{\kern0pt}\isactrlsup {\isasymdagger}{\isacharparenright}{\kern0pt}\ {\isachardollar}{\kern0pt}{\isachardollar}{\kern0pt}\ {\isacharparenleft}{\kern0pt}{\isadigit{0}}{\isacharcomma}{\kern0pt}{\isadigit{0}}{\isacharparenright}{\kern0pt}\ {\isacharasterisk}{\kern0pt}\ {\isacharparenleft}{\kern0pt}control{\isadigit{2}}\ U{\isacharparenright}{\kern0pt}\ {\isachardollar}{\kern0pt}{\isachardollar}{\kern0pt}\ {\isacharparenleft}{\kern0pt}{\isadigit{0}}{\isacharcomma}{\kern0pt}{\isadigit{0}}{\isacharparenright}{\kern0pt}\ {\isacharplus}{\kern0pt}\isanewline
\ \ \ \ \ \ \ \ \ \ \ \ \ \ \ \ {\isacharparenleft}{\kern0pt}{\isacharparenleft}{\kern0pt}control{\isadigit{2}}\ U{\isacharparenright}{\kern0pt}\isactrlsup {\isasymdagger}{\isacharparenright}{\kern0pt}\ {\isachardollar}{\kern0pt}{\isachardollar}{\kern0pt}\ {\isacharparenleft}{\kern0pt}{\isadigit{0}}{\isacharcomma}{\kern0pt}{\isadigit{1}}{\isacharparenright}{\kern0pt}\ {\isacharasterisk}{\kern0pt}\ {\isacharparenleft}{\kern0pt}control{\isadigit{2}}\ U{\isacharparenright}{\kern0pt}\ {\isachardollar}{\kern0pt}{\isachardollar}{\kern0pt}\ {\isacharparenleft}{\kern0pt}{\isadigit{1}}{\isacharcomma}{\kern0pt}{\isadigit{0}}{\isacharparenright}{\kern0pt}\ {\isacharplus}{\kern0pt}\isanewline
\ \ \ \ \ \ \ \ \ \ \ \ \ \ \ \ {\isacharparenleft}{\kern0pt}{\isacharparenleft}{\kern0pt}control{\isadigit{2}}\ U{\isacharparenright}{\kern0pt}\isactrlsup {\isasymdagger}{\isacharparenright}{\kern0pt}\ {\isachardollar}{\kern0pt}{\isachardollar}{\kern0pt}\ {\isacharparenleft}{\kern0pt}{\isadigit{0}}{\isacharcomma}{\kern0pt}{\isadigit{2}}{\isacharparenright}{\kern0pt}\ {\isacharasterisk}{\kern0pt}\ {\isacharparenleft}{\kern0pt}control{\isadigit{2}}\ U{\isacharparenright}{\kern0pt}\ {\isachardollar}{\kern0pt}{\isachardollar}{\kern0pt}\ {\isacharparenleft}{\kern0pt}{\isadigit{2}}{\isacharcomma}{\kern0pt}{\isadigit{0}}{\isacharparenright}{\kern0pt}\ {\isacharplus}{\kern0pt}\isanewline
\ \ \ \ \ \ \ \ \ \ \ \ \ \ \ \ {\isacharparenleft}{\kern0pt}{\isacharparenleft}{\kern0pt}control{\isadigit{2}}\ U{\isacharparenright}{\kern0pt}\isactrlsup {\isasymdagger}{\isacharparenright}{\kern0pt}\ {\isachardollar}{\kern0pt}{\isachardollar}{\kern0pt}\ {\isacharparenleft}{\kern0pt}{\isadigit{0}}{\isacharcomma}{\kern0pt}{\isadigit{3}}{\isacharparenright}{\kern0pt}\ {\isacharasterisk}{\kern0pt}\ {\isacharparenleft}{\kern0pt}control{\isadigit{2}}\ U{\isacharparenright}{\kern0pt}\ {\isachardollar}{\kern0pt}{\isachardollar}{\kern0pt}\ {\isacharparenleft}{\kern0pt}{\isadigit{3}}{\isacharcomma}{\kern0pt}{\isadigit{0}}{\isacharparenright}{\kern0pt}{\isachardoublequoteclose}\isanewline
\ \ \ \ \ \ \ \ \ \ \ \ \isacommand{using}\isamarkupfalse%
\ sumof{\isadigit{4}}\isanewline
\ \ \ \ \ \ \ \ \ \ \ \ \isacommand{by}\isamarkupfalse%
\ {\isacharparenleft}{\kern0pt}metis\ {\isacharparenleft}{\kern0pt}no{\isacharunderscore}{\kern0pt}types{\isacharcomma}{\kern0pt}\ lifting{\isacharparenright}{\kern0pt}\ carrier{\isacharunderscore}{\kern0pt}matD{\isacharparenleft}{\kern0pt}{\isadigit{1}}{\isacharparenright}{\kern0pt}\ carrier{\isacharunderscore}{\kern0pt}matD{\isacharparenleft}{\kern0pt}{\isadigit{2}}{\isacharparenright}{\kern0pt}\ control{\isadigit{2}}{\isacharunderscore}{\kern0pt}carrier{\isacharunderscore}{\kern0pt}mat\ \isanewline
\ \ \ \ \ \ \ \ \ \ \ \ \ \ \ \ dim{\isacharunderscore}{\kern0pt}col{\isacharunderscore}{\kern0pt}of{\isacharunderscore}{\kern0pt}dagger\ dim{\isacharunderscore}{\kern0pt}row{\isacharunderscore}{\kern0pt}of{\isacharunderscore}{\kern0pt}dagger\ i{\isadigit{0}}\ i{\isadigit{4}}\ index{\isacharunderscore}{\kern0pt}matrix{\isacharunderscore}{\kern0pt}prod{\isacharparenright}{\kern0pt}\isanewline
\ \ \ \ \ \ \ \ \ \ \isacommand{also}\isamarkupfalse%
\ \isacommand{have}\isamarkupfalse%
\ {\isachardoublequoteopen}{\isasymdots}\ {\isacharequal}{\kern0pt}\ {\isacharparenleft}{\kern0pt}{\isacharparenleft}{\kern0pt}control{\isadigit{2}}\ U{\isacharparenright}{\kern0pt}\isactrlsup {\isasymdagger}{\isacharparenright}{\kern0pt}\ {\isachardollar}{\kern0pt}{\isachardollar}{\kern0pt}\ {\isacharparenleft}{\kern0pt}{\isadigit{0}}{\isacharcomma}{\kern0pt}{\isadigit{0}}{\isacharparenright}{\kern0pt}{\isachardoublequoteclose}\isanewline
\ \ \ \ \ \ \ \ \ \ \ \ \isacommand{using}\isamarkupfalse%
\ control{\isadigit{2}}{\isacharunderscore}{\kern0pt}def\ index{\isacharunderscore}{\kern0pt}mat{\isacharunderscore}{\kern0pt}of{\isacharunderscore}{\kern0pt}cols{\isacharunderscore}{\kern0pt}list\ \isacommand{by}\isamarkupfalse%
\ force\isanewline
\ \ \ \ \ \ \ \ \ \ \isacommand{also}\isamarkupfalse%
\ \isacommand{have}\isamarkupfalse%
\ {\isachardoublequoteopen}{\isasymdots}\ {\isacharequal}{\kern0pt}\ cnj\ {\isacharparenleft}{\kern0pt}{\isacharparenleft}{\kern0pt}control{\isadigit{2}}\ U{\isacharparenright}{\kern0pt}\ {\isachardollar}{\kern0pt}{\isachardollar}{\kern0pt}\ {\isacharparenleft}{\kern0pt}{\isadigit{0}}{\isacharcomma}{\kern0pt}{\isadigit{0}}{\isacharparenright}{\kern0pt}{\isacharparenright}{\kern0pt}{\isachardoublequoteclose}\isanewline
\ \ \ \ \ \ \ \ \ \ \ \ \isacommand{using}\isamarkupfalse%
\ dagger{\isacharunderscore}{\kern0pt}def\isanewline
\ \ \ \ \ \ \ \ \ \ \ \ \isacommand{by}\isamarkupfalse%
\ {\isacharparenleft}{\kern0pt}simp\ add{\isacharcolon}{\kern0pt}\ Tensor{\isachardot}{\kern0pt}mat{\isacharunderscore}{\kern0pt}of{\isacharunderscore}{\kern0pt}cols{\isacharunderscore}{\kern0pt}list{\isacharunderscore}{\kern0pt}def\ control{\isadigit{2}}{\isacharunderscore}{\kern0pt}def{\isacharparenright}{\kern0pt}\isanewline
\ \ \ \ \ \ \ \ \ \ \isacommand{also}\isamarkupfalse%
\ \isacommand{have}\isamarkupfalse%
\ {\isachardoublequoteopen}{\isasymdots}\ {\isacharequal}{\kern0pt}\ {\isadigit{1}}{\isachardoublequoteclose}\ \isacommand{using}\isamarkupfalse%
\ control{\isadigit{2}}{\isacharunderscore}{\kern0pt}def\ index{\isacharunderscore}{\kern0pt}mat{\isacharunderscore}{\kern0pt}of{\isacharunderscore}{\kern0pt}cols{\isacharunderscore}{\kern0pt}list\ \isacommand{by}\isamarkupfalse%
\ auto\isanewline
\ \ \ \ \ \ \ \ \ \ \isacommand{also}\isamarkupfalse%
\ \isacommand{have}\isamarkupfalse%
\ {\isachardoublequoteopen}{\isasymdots}\ {\isacharequal}{\kern0pt}\ {\isadigit{1}}\isactrlsub m\ {\isadigit{4}}\ {\isachardollar}{\kern0pt}{\isachardollar}{\kern0pt}\ {\isacharparenleft}{\kern0pt}{\isadigit{0}}{\isacharcomma}{\kern0pt}{\isadigit{0}}{\isacharparenright}{\kern0pt}{\isachardoublequoteclose}\ \isacommand{by}\isamarkupfalse%
\ simp\isanewline
\ \ \ \ \ \ \ \ \ \ \isacommand{finally}\isamarkupfalse%
\ \isacommand{show}\isamarkupfalse%
\ {\isacharquery}{\kern0pt}thesis\ \isacommand{using}\isamarkupfalse%
\ i{\isadigit{0}}\ j{\isadigit{0}}\ \isacommand{by}\isamarkupfalse%
\ simp\isanewline
\ \ \ \ \ \ \ \ \isacommand{qed}\isamarkupfalse%
\isanewline
\ \ \ \ \ \ \isacommand{next}\isamarkupfalse%
\isanewline
\ \ \ \ \ \ \ \ \isacommand{assume}\isamarkupfalse%
\ jl{\isadigit{3}}{\isacharcolon}{\kern0pt}{\isachardoublequoteopen}j\ {\isacharequal}{\kern0pt}\ {\isadigit{1}}\ {\isasymor}\ j\ {\isacharequal}{\kern0pt}\ {\isadigit{2}}\ {\isasymor}\ j\ {\isacharequal}{\kern0pt}\ {\isadigit{3}}{\isachardoublequoteclose}\isanewline
\ \ \ \ \ \ \ \ \isacommand{show}\isamarkupfalse%
\ {\isachardoublequoteopen}{\isacharparenleft}{\kern0pt}{\isacharparenleft}{\kern0pt}control{\isadigit{2}}\ U{\isacharparenright}{\kern0pt}\isactrlsup {\isasymdagger}\ {\isacharasterisk}{\kern0pt}\ control{\isadigit{2}}\ U{\isacharparenright}{\kern0pt}\ {\isachardollar}{\kern0pt}{\isachardollar}{\kern0pt}\ {\isacharparenleft}{\kern0pt}i{\isacharcomma}{\kern0pt}\ j{\isacharparenright}{\kern0pt}\ {\isacharequal}{\kern0pt}\ {\isadigit{1}}\isactrlsub m\ {\isadigit{4}}\ {\isachardollar}{\kern0pt}{\isachardollar}{\kern0pt}\ {\isacharparenleft}{\kern0pt}i{\isacharcomma}{\kern0pt}\ j{\isacharparenright}{\kern0pt}{\isachardoublequoteclose}\isanewline
\ \ \ \ \ \ \ \ \isacommand{proof}\isamarkupfalse%
\ {\isacharparenleft}{\kern0pt}rule\ disjE{\isacharparenright}{\kern0pt}\isanewline
\ \ \ \ \ \ \ \ \ \ \isacommand{show}\isamarkupfalse%
\ {\isachardoublequoteopen}j\ {\isacharequal}{\kern0pt}\ {\isadigit{1}}\ {\isasymor}\ j\ {\isacharequal}{\kern0pt}\ {\isadigit{2}}\ {\isasymor}\ j\ {\isacharequal}{\kern0pt}\ {\isadigit{3}}{\isachardoublequoteclose}\ \isacommand{using}\isamarkupfalse%
\ jl{\isadigit{3}}\ \isacommand{by}\isamarkupfalse%
\ this\isanewline
\ \ \ \ \ \ \ \ \isacommand{next}\isamarkupfalse%
\isanewline
\ \ \ \ \ \ \ \ \ \ \isacommand{assume}\isamarkupfalse%
\ j{\isadigit{1}}{\isacharcolon}{\kern0pt}{\isachardoublequoteopen}j\ {\isacharequal}{\kern0pt}\ {\isadigit{1}}{\isachardoublequoteclose}\isanewline
\ \ \ \ \ \ \ \ \ \ \isacommand{show}\isamarkupfalse%
\ {\isachardoublequoteopen}{\isacharparenleft}{\kern0pt}{\isacharparenleft}{\kern0pt}control{\isadigit{2}}\ U{\isacharparenright}{\kern0pt}\isactrlsup {\isasymdagger}\ {\isacharasterisk}{\kern0pt}\ control{\isadigit{2}}\ U{\isacharparenright}{\kern0pt}\ {\isachardollar}{\kern0pt}{\isachardollar}{\kern0pt}\ {\isacharparenleft}{\kern0pt}i{\isacharcomma}{\kern0pt}\ j{\isacharparenright}{\kern0pt}\ {\isacharequal}{\kern0pt}\ {\isadigit{1}}\isactrlsub m\ {\isadigit{4}}\ {\isachardollar}{\kern0pt}{\isachardollar}{\kern0pt}\ {\isacharparenleft}{\kern0pt}i{\isacharcomma}{\kern0pt}\ j{\isacharparenright}{\kern0pt}{\isachardoublequoteclose}\isanewline
\ \ \ \ \ \ \ \ \ \ \isacommand{proof}\isamarkupfalse%
\ {\isacharminus}{\kern0pt}\isanewline
\ \ \ \ \ \ \ \ \ \ \ \ \isacommand{have}\isamarkupfalse%
\ {\isachardoublequoteopen}{\isacharparenleft}{\kern0pt}{\isacharparenleft}{\kern0pt}control{\isadigit{2}}\ U{\isacharparenright}{\kern0pt}\isactrlsup {\isasymdagger}\ {\isacharasterisk}{\kern0pt}\ control{\isadigit{2}}\ U{\isacharparenright}{\kern0pt}\ {\isachardollar}{\kern0pt}{\isachardollar}{\kern0pt}\ {\isacharparenleft}{\kern0pt}{\isadigit{0}}{\isacharcomma}{\kern0pt}{\isadigit{1}}{\isacharparenright}{\kern0pt}\ {\isacharequal}{\kern0pt}\isanewline
\ \ \ \ \ \ \ \ \ \ \ \ \ \ \ \ {\isacharparenleft}{\kern0pt}{\isacharparenleft}{\kern0pt}control{\isadigit{2}}\ U{\isacharparenright}{\kern0pt}\isactrlsup {\isasymdagger}{\isacharparenright}{\kern0pt}\ {\isachardollar}{\kern0pt}{\isachardollar}{\kern0pt}\ {\isacharparenleft}{\kern0pt}{\isadigit{0}}{\isacharcomma}{\kern0pt}{\isadigit{0}}{\isacharparenright}{\kern0pt}\ {\isacharasterisk}{\kern0pt}\ {\isacharparenleft}{\kern0pt}control{\isadigit{2}}\ U{\isacharparenright}{\kern0pt}\ {\isachardollar}{\kern0pt}{\isachardollar}{\kern0pt}\ {\isacharparenleft}{\kern0pt}{\isadigit{0}}{\isacharcomma}{\kern0pt}{\isadigit{1}}{\isacharparenright}{\kern0pt}\ {\isacharplus}{\kern0pt}\isanewline
\ \ \ \ \ \ \ \ \ \ \ \ \ \ \ \ {\isacharparenleft}{\kern0pt}{\isacharparenleft}{\kern0pt}control{\isadigit{2}}\ U{\isacharparenright}{\kern0pt}\isactrlsup {\isasymdagger}{\isacharparenright}{\kern0pt}\ {\isachardollar}{\kern0pt}{\isachardollar}{\kern0pt}\ {\isacharparenleft}{\kern0pt}{\isadigit{0}}{\isacharcomma}{\kern0pt}{\isadigit{1}}{\isacharparenright}{\kern0pt}\ {\isacharasterisk}{\kern0pt}\ {\isacharparenleft}{\kern0pt}control{\isadigit{2}}\ U{\isacharparenright}{\kern0pt}\ {\isachardollar}{\kern0pt}{\isachardollar}{\kern0pt}\ {\isacharparenleft}{\kern0pt}{\isadigit{1}}{\isacharcomma}{\kern0pt}{\isadigit{1}}{\isacharparenright}{\kern0pt}\ {\isacharplus}{\kern0pt}\isanewline
\ \ \ \ \ \ \ \ \ \ \ \ \ \ \ \ {\isacharparenleft}{\kern0pt}{\isacharparenleft}{\kern0pt}control{\isadigit{2}}\ U{\isacharparenright}{\kern0pt}\isactrlsup {\isasymdagger}{\isacharparenright}{\kern0pt}\ {\isachardollar}{\kern0pt}{\isachardollar}{\kern0pt}\ {\isacharparenleft}{\kern0pt}{\isadigit{0}}{\isacharcomma}{\kern0pt}{\isadigit{2}}{\isacharparenright}{\kern0pt}\ {\isacharasterisk}{\kern0pt}\ {\isacharparenleft}{\kern0pt}control{\isadigit{2}}\ U{\isacharparenright}{\kern0pt}\ {\isachardollar}{\kern0pt}{\isachardollar}{\kern0pt}\ {\isacharparenleft}{\kern0pt}{\isadigit{2}}{\isacharcomma}{\kern0pt}{\isadigit{1}}{\isacharparenright}{\kern0pt}\ {\isacharplus}{\kern0pt}\isanewline
\ \ \ \ \ \ \ \ \ \ \ \ \ \ \ \ {\isacharparenleft}{\kern0pt}{\isacharparenleft}{\kern0pt}control{\isadigit{2}}\ U{\isacharparenright}{\kern0pt}\isactrlsup {\isasymdagger}{\isacharparenright}{\kern0pt}\ {\isachardollar}{\kern0pt}{\isachardollar}{\kern0pt}\ {\isacharparenleft}{\kern0pt}{\isadigit{0}}{\isacharcomma}{\kern0pt}{\isadigit{3}}{\isacharparenright}{\kern0pt}\ {\isacharasterisk}{\kern0pt}\ {\isacharparenleft}{\kern0pt}control{\isadigit{2}}\ U{\isacharparenright}{\kern0pt}\ {\isachardollar}{\kern0pt}{\isachardollar}{\kern0pt}\ {\isacharparenleft}{\kern0pt}{\isadigit{3}}{\isacharcomma}{\kern0pt}{\isadigit{1}}{\isacharparenright}{\kern0pt}{\isachardoublequoteclose}\isanewline
\ \ \ \ \ \ \ \ \ \ \ \ \ \ \isacommand{using}\isamarkupfalse%
\ sumof{\isadigit{4}}\isanewline
\ \ \ \ \ \ \ \ \ \ \ \ \ \ \isacommand{by}\isamarkupfalse%
\ {\isacharparenleft}{\kern0pt}smt\ {\isacharparenleft}{\kern0pt}z{\isadigit{3}}{\isacharparenright}{\kern0pt}\ carrier{\isacharunderscore}{\kern0pt}matD{\isacharparenleft}{\kern0pt}{\isadigit{1}}{\isacharparenright}{\kern0pt}\ carrier{\isacharunderscore}{\kern0pt}matD{\isacharparenleft}{\kern0pt}{\isadigit{2}}{\isacharparenright}{\kern0pt}\ control{\isadigit{2}}{\isacharunderscore}{\kern0pt}carrier{\isacharunderscore}{\kern0pt}mat\ dim{\isacharunderscore}{\kern0pt}col{\isacharunderscore}{\kern0pt}of{\isacharunderscore}{\kern0pt}dagger\ \isanewline
\ \ \ \ \ \ \ \ \ \ \ \ \ \ \ \ \ \ dim{\isacharunderscore}{\kern0pt}row{\isacharunderscore}{\kern0pt}of{\isacharunderscore}{\kern0pt}dagger\ index{\isacharunderscore}{\kern0pt}matrix{\isacharunderscore}{\kern0pt}prod\ one{\isacharunderscore}{\kern0pt}less{\isacharunderscore}{\kern0pt}numeral{\isacharunderscore}{\kern0pt}iff\ semiring{\isacharunderscore}{\kern0pt}norm{\isacharparenleft}{\kern0pt}{\isadigit{7}}{\isadigit{6}}{\isacharparenright}{\kern0pt}\ \isanewline
\ \ \ \ \ \ \ \ \ \ \ \ \ \ \ \ \ \ zero{\isacharunderscore}{\kern0pt}less{\isacharunderscore}{\kern0pt}numeral{\isacharparenright}{\kern0pt}\isanewline
\ \ \ \ \ \ \ \ \ \ \ \ \isacommand{also}\isamarkupfalse%
\ \isacommand{have}\isamarkupfalse%
\ {\isachardoublequoteopen}{\isasymdots}\ {\isacharequal}{\kern0pt}\ {\isacharparenleft}{\kern0pt}{\isacharparenleft}{\kern0pt}control{\isadigit{2}}\ U{\isacharparenright}{\kern0pt}\isactrlsup {\isasymdagger}{\isacharparenright}{\kern0pt}\ {\isachardollar}{\kern0pt}{\isachardollar}{\kern0pt}\ {\isacharparenleft}{\kern0pt}{\isadigit{0}}{\isacharcomma}{\kern0pt}{\isadigit{1}}{\isacharparenright}{\kern0pt}\ {\isacharasterisk}{\kern0pt}\ {\isacharparenleft}{\kern0pt}control{\isadigit{2}}\ U{\isacharparenright}{\kern0pt}\ {\isachardollar}{\kern0pt}{\isachardollar}{\kern0pt}\ {\isacharparenleft}{\kern0pt}{\isadigit{1}}{\isacharcomma}{\kern0pt}{\isadigit{1}}{\isacharparenright}{\kern0pt}\ {\isacharplus}{\kern0pt}\isanewline
\ \ \ \ \ \ \ \ \ \ \ \ \ \ \ \ \ \ \ \ \ \ \ \ \ \ \ \ {\isacharparenleft}{\kern0pt}{\isacharparenleft}{\kern0pt}control{\isadigit{2}}\ U{\isacharparenright}{\kern0pt}\isactrlsup {\isasymdagger}{\isacharparenright}{\kern0pt}\ {\isachardollar}{\kern0pt}{\isachardollar}{\kern0pt}\ {\isacharparenleft}{\kern0pt}{\isadigit{0}}{\isacharcomma}{\kern0pt}{\isadigit{3}}{\isacharparenright}{\kern0pt}\ {\isacharasterisk}{\kern0pt}\ {\isacharparenleft}{\kern0pt}control{\isadigit{2}}\ U{\isacharparenright}{\kern0pt}\ {\isachardollar}{\kern0pt}{\isachardollar}{\kern0pt}\ {\isacharparenleft}{\kern0pt}{\isadigit{3}}{\isacharcomma}{\kern0pt}{\isadigit{1}}{\isacharparenright}{\kern0pt}{\isachardoublequoteclose}\isanewline
\ \ \ \ \ \ \ \ \ \ \ \ \ \ \isacommand{using}\isamarkupfalse%
\ control{\isadigit{2}}{\isacharunderscore}{\kern0pt}def\ index{\isacharunderscore}{\kern0pt}mat{\isacharunderscore}{\kern0pt}of{\isacharunderscore}{\kern0pt}cols{\isacharunderscore}{\kern0pt}list\ \isacommand{by}\isamarkupfalse%
\ force\isanewline
\ \ \ \ \ \ \ \ \ \ \ \ \isacommand{also}\isamarkupfalse%
\ \isacommand{have}\isamarkupfalse%
\ {\isachardoublequoteopen}{\isasymdots}\ {\isacharequal}{\kern0pt}\ cnj\ {\isacharparenleft}{\kern0pt}{\isacharparenleft}{\kern0pt}control{\isadigit{2}}\ U{\isacharparenright}{\kern0pt}\ {\isachardollar}{\kern0pt}{\isachardollar}{\kern0pt}\ {\isacharparenleft}{\kern0pt}{\isadigit{1}}{\isacharcomma}{\kern0pt}{\isadigit{0}}{\isacharparenright}{\kern0pt}{\isacharparenright}{\kern0pt}\ {\isacharasterisk}{\kern0pt}\ {\isacharparenleft}{\kern0pt}control{\isadigit{2}}\ U{\isacharparenright}{\kern0pt}\ {\isachardollar}{\kern0pt}{\isachardollar}{\kern0pt}\ {\isacharparenleft}{\kern0pt}{\isadigit{1}}{\isacharcomma}{\kern0pt}{\isadigit{1}}{\isacharparenright}{\kern0pt}\ {\isacharplus}{\kern0pt}\isanewline
\ \ \ \ \ \ \ \ \ \ \ \ \ \ \ \ \ \ \ \ \ \ \ \ \ \ \ \ cnj\ {\isacharparenleft}{\kern0pt}{\isacharparenleft}{\kern0pt}control{\isadigit{2}}\ U{\isacharparenright}{\kern0pt}\ {\isachardollar}{\kern0pt}{\isachardollar}{\kern0pt}\ {\isacharparenleft}{\kern0pt}{\isadigit{3}}{\isacharcomma}{\kern0pt}{\isadigit{0}}{\isacharparenright}{\kern0pt}{\isacharparenright}{\kern0pt}\ {\isacharasterisk}{\kern0pt}\ {\isacharparenleft}{\kern0pt}control{\isadigit{2}}\ U{\isacharparenright}{\kern0pt}\ {\isachardollar}{\kern0pt}{\isachardollar}{\kern0pt}\ {\isacharparenleft}{\kern0pt}{\isadigit{3}}{\isacharcomma}{\kern0pt}{\isadigit{1}}{\isacharparenright}{\kern0pt}{\isachardoublequoteclose}\isanewline
\ \ \ \ \ \ \ \ \ \ \ \ \ \ \isacommand{using}\isamarkupfalse%
\ dagger{\isacharunderscore}{\kern0pt}def\isanewline
\ \ \ \ \ \ \ \ \ \ \ \ \ \ \isacommand{by}\isamarkupfalse%
\ {\isacharparenleft}{\kern0pt}simp\ add{\isacharcolon}{\kern0pt}\ Tensor{\isachardot}{\kern0pt}mat{\isacharunderscore}{\kern0pt}of{\isacharunderscore}{\kern0pt}cols{\isacharunderscore}{\kern0pt}list{\isacharunderscore}{\kern0pt}def\ control{\isadigit{2}}{\isacharunderscore}{\kern0pt}def{\isacharparenright}{\kern0pt}\isanewline
\ \ \ \ \ \ \ \ \ \ \ \ \isacommand{also}\isamarkupfalse%
\ \isacommand{have}\isamarkupfalse%
\ {\isachardoublequoteopen}{\isasymdots}\ {\isacharequal}{\kern0pt}\ {\isadigit{0}}{\isachardoublequoteclose}\ \isacommand{using}\isamarkupfalse%
\ control{\isadigit{2}}{\isacharunderscore}{\kern0pt}def\ index{\isacharunderscore}{\kern0pt}mat{\isacharunderscore}{\kern0pt}of{\isacharunderscore}{\kern0pt}cols{\isacharunderscore}{\kern0pt}list\ \isacommand{by}\isamarkupfalse%
\ auto\isanewline
\ \ \ \ \ \ \ \ \ \ \ \ \isacommand{also}\isamarkupfalse%
\ \isacommand{have}\isamarkupfalse%
\ {\isachardoublequoteopen}{\isasymdots}\ {\isacharequal}{\kern0pt}\ {\isadigit{1}}\isactrlsub m\ {\isadigit{4}}\ {\isachardollar}{\kern0pt}{\isachardollar}{\kern0pt}\ {\isacharparenleft}{\kern0pt}{\isadigit{0}}{\isacharcomma}{\kern0pt}{\isadigit{1}}{\isacharparenright}{\kern0pt}{\isachardoublequoteclose}\ \isacommand{by}\isamarkupfalse%
\ simp\isanewline
\ \ \ \ \ \ \ \ \ \ \ \ \isacommand{finally}\isamarkupfalse%
\ \isacommand{show}\isamarkupfalse%
\ {\isacharquery}{\kern0pt}thesis\ \isacommand{using}\isamarkupfalse%
\ i{\isadigit{0}}\ j{\isadigit{1}}\ \isacommand{by}\isamarkupfalse%
\ simp\isanewline
\ \ \ \ \ \ \ \ \ \ \isacommand{qed}\isamarkupfalse%
\isanewline
\ \ \ \ \ \ \ \ \isacommand{next}\isamarkupfalse%
\isanewline
\ \ \ \ \ \ \ \ \ \ \isacommand{assume}\isamarkupfalse%
\ jl{\isadigit{2}}{\isacharcolon}{\kern0pt}{\isachardoublequoteopen}j\ {\isacharequal}{\kern0pt}\ {\isadigit{2}}\ {\isasymor}\ j\ {\isacharequal}{\kern0pt}\ {\isadigit{3}}{\isachardoublequoteclose}\isanewline
\ \ \ \ \ \ \ \ \ \ \isacommand{show}\isamarkupfalse%
\ {\isachardoublequoteopen}{\isacharparenleft}{\kern0pt}{\isacharparenleft}{\kern0pt}control{\isadigit{2}}\ U{\isacharparenright}{\kern0pt}\isactrlsup {\isasymdagger}\ {\isacharasterisk}{\kern0pt}\ control{\isadigit{2}}\ U{\isacharparenright}{\kern0pt}\ {\isachardollar}{\kern0pt}{\isachardollar}{\kern0pt}\ {\isacharparenleft}{\kern0pt}i{\isacharcomma}{\kern0pt}\ j{\isacharparenright}{\kern0pt}\ {\isacharequal}{\kern0pt}\ {\isadigit{1}}\isactrlsub m\ {\isadigit{4}}\ {\isachardollar}{\kern0pt}{\isachardollar}{\kern0pt}\ {\isacharparenleft}{\kern0pt}i{\isacharcomma}{\kern0pt}\ j{\isacharparenright}{\kern0pt}{\isachardoublequoteclose}\isanewline
\ \ \ \ \ \ \ \ \ \ \isacommand{proof}\isamarkupfalse%
\ {\isacharparenleft}{\kern0pt}rule\ disjE{\isacharparenright}{\kern0pt}\isanewline
\ \ \ \ \ \ \ \ \ \ \ \ \isacommand{show}\isamarkupfalse%
\ {\isachardoublequoteopen}j\ {\isacharequal}{\kern0pt}\ {\isadigit{2}}\ {\isasymor}\ j\ {\isacharequal}{\kern0pt}\ {\isadigit{3}}{\isachardoublequoteclose}\ \isacommand{using}\isamarkupfalse%
\ jl{\isadigit{2}}\ \isacommand{by}\isamarkupfalse%
\ this\isanewline
\ \ \ \ \ \ \ \ \ \ \isacommand{next}\isamarkupfalse%
\isanewline
\ \ \ \ \ \ \ \ \ \ \ \ \isacommand{assume}\isamarkupfalse%
\ j{\isadigit{2}}{\isacharcolon}{\kern0pt}{\isachardoublequoteopen}j\ {\isacharequal}{\kern0pt}\ {\isadigit{2}}{\isachardoublequoteclose}\isanewline
\ \ \ \ \ \ \ \ \ \ \ \ \isacommand{show}\isamarkupfalse%
\ {\isachardoublequoteopen}{\isacharparenleft}{\kern0pt}{\isacharparenleft}{\kern0pt}control{\isadigit{2}}\ U{\isacharparenright}{\kern0pt}\isactrlsup {\isasymdagger}\ {\isacharasterisk}{\kern0pt}\ control{\isadigit{2}}\ U{\isacharparenright}{\kern0pt}\ {\isachardollar}{\kern0pt}{\isachardollar}{\kern0pt}\ {\isacharparenleft}{\kern0pt}i{\isacharcomma}{\kern0pt}\ j{\isacharparenright}{\kern0pt}\ {\isacharequal}{\kern0pt}\ {\isadigit{1}}\isactrlsub m\ {\isadigit{4}}\ {\isachardollar}{\kern0pt}{\isachardollar}{\kern0pt}\ {\isacharparenleft}{\kern0pt}i{\isacharcomma}{\kern0pt}\ j{\isacharparenright}{\kern0pt}{\isachardoublequoteclose}\isanewline
\ \ \ \ \ \ \ \ \ \ \ \ \isacommand{proof}\isamarkupfalse%
\ {\isacharminus}{\kern0pt}\isanewline
\ \ \ \ \ \ \ \ \ \ \ \ \ \ \isacommand{have}\isamarkupfalse%
\ {\isachardoublequoteopen}{\isacharparenleft}{\kern0pt}{\isacharparenleft}{\kern0pt}control{\isadigit{2}}\ U{\isacharparenright}{\kern0pt}\isactrlsup {\isasymdagger}\ {\isacharasterisk}{\kern0pt}\ control{\isadigit{2}}\ U{\isacharparenright}{\kern0pt}\ {\isachardollar}{\kern0pt}{\isachardollar}{\kern0pt}\ {\isacharparenleft}{\kern0pt}{\isadigit{0}}{\isacharcomma}{\kern0pt}{\isadigit{2}}{\isacharparenright}{\kern0pt}\ {\isacharequal}{\kern0pt}\isanewline
\ \ \ \ \ \ \ \ \ \ \ \ \ \ \ \ {\isacharparenleft}{\kern0pt}{\isacharparenleft}{\kern0pt}control{\isadigit{2}}\ U{\isacharparenright}{\kern0pt}\isactrlsup {\isasymdagger}{\isacharparenright}{\kern0pt}\ {\isachardollar}{\kern0pt}{\isachardollar}{\kern0pt}\ {\isacharparenleft}{\kern0pt}{\isadigit{0}}{\isacharcomma}{\kern0pt}{\isadigit{0}}{\isacharparenright}{\kern0pt}\ {\isacharasterisk}{\kern0pt}\ {\isacharparenleft}{\kern0pt}control{\isadigit{2}}\ U{\isacharparenright}{\kern0pt}\ {\isachardollar}{\kern0pt}{\isachardollar}{\kern0pt}\ {\isacharparenleft}{\kern0pt}{\isadigit{0}}{\isacharcomma}{\kern0pt}{\isadigit{2}}{\isacharparenright}{\kern0pt}\ {\isacharplus}{\kern0pt}\isanewline
\ \ \ \ \ \ \ \ \ \ \ \ \ \ \ \ {\isacharparenleft}{\kern0pt}{\isacharparenleft}{\kern0pt}control{\isadigit{2}}\ U{\isacharparenright}{\kern0pt}\isactrlsup {\isasymdagger}{\isacharparenright}{\kern0pt}\ {\isachardollar}{\kern0pt}{\isachardollar}{\kern0pt}\ {\isacharparenleft}{\kern0pt}{\isadigit{0}}{\isacharcomma}{\kern0pt}{\isadigit{1}}{\isacharparenright}{\kern0pt}\ {\isacharasterisk}{\kern0pt}\ {\isacharparenleft}{\kern0pt}control{\isadigit{2}}\ U{\isacharparenright}{\kern0pt}\ {\isachardollar}{\kern0pt}{\isachardollar}{\kern0pt}\ {\isacharparenleft}{\kern0pt}{\isadigit{1}}{\isacharcomma}{\kern0pt}{\isadigit{2}}{\isacharparenright}{\kern0pt}\ {\isacharplus}{\kern0pt}\isanewline
\ \ \ \ \ \ \ \ \ \ \ \ \ \ \ \ {\isacharparenleft}{\kern0pt}{\isacharparenleft}{\kern0pt}control{\isadigit{2}}\ U{\isacharparenright}{\kern0pt}\isactrlsup {\isasymdagger}{\isacharparenright}{\kern0pt}\ {\isachardollar}{\kern0pt}{\isachardollar}{\kern0pt}\ {\isacharparenleft}{\kern0pt}{\isadigit{0}}{\isacharcomma}{\kern0pt}{\isadigit{2}}{\isacharparenright}{\kern0pt}\ {\isacharasterisk}{\kern0pt}\ {\isacharparenleft}{\kern0pt}control{\isadigit{2}}\ U{\isacharparenright}{\kern0pt}\ {\isachardollar}{\kern0pt}{\isachardollar}{\kern0pt}\ {\isacharparenleft}{\kern0pt}{\isadigit{2}}{\isacharcomma}{\kern0pt}{\isadigit{2}}{\isacharparenright}{\kern0pt}\ {\isacharplus}{\kern0pt}\isanewline
\ \ \ \ \ \ \ \ \ \ \ \ \ \ \ \ {\isacharparenleft}{\kern0pt}{\isacharparenleft}{\kern0pt}control{\isadigit{2}}\ U{\isacharparenright}{\kern0pt}\isactrlsup {\isasymdagger}{\isacharparenright}{\kern0pt}\ {\isachardollar}{\kern0pt}{\isachardollar}{\kern0pt}\ {\isacharparenleft}{\kern0pt}{\isadigit{0}}{\isacharcomma}{\kern0pt}{\isadigit{3}}{\isacharparenright}{\kern0pt}\ {\isacharasterisk}{\kern0pt}\ {\isacharparenleft}{\kern0pt}control{\isadigit{2}}\ U{\isacharparenright}{\kern0pt}\ {\isachardollar}{\kern0pt}{\isachardollar}{\kern0pt}\ {\isacharparenleft}{\kern0pt}{\isadigit{3}}{\isacharcomma}{\kern0pt}{\isadigit{2}}{\isacharparenright}{\kern0pt}{\isachardoublequoteclose}\isanewline
\ \ \ \ \ \ \ \ \ \ \ \ \ \ \ \ \isacommand{using}\isamarkupfalse%
\ sumof{\isadigit{4}}\isanewline
\ \ \ \ \ \ \ \ \ \ \ \ \ \ \ \ \isacommand{by}\isamarkupfalse%
\ {\isacharparenleft}{\kern0pt}smt\ {\isacharparenleft}{\kern0pt}z{\isadigit{3}}{\isacharparenright}{\kern0pt}\ carrier{\isacharunderscore}{\kern0pt}matD{\isacharparenleft}{\kern0pt}{\isadigit{1}}{\isacharparenright}{\kern0pt}\ carrier{\isacharunderscore}{\kern0pt}matD{\isacharparenleft}{\kern0pt}{\isadigit{2}}{\isacharparenright}{\kern0pt}\ control{\isadigit{2}}{\isacharunderscore}{\kern0pt}carrier{\isacharunderscore}{\kern0pt}mat\ dim{\isacharunderscore}{\kern0pt}col{\isacharunderscore}{\kern0pt}of{\isacharunderscore}{\kern0pt}dagger\ \isanewline
\ \ \ \ \ \ \ \ \ \ \ \ \ \ \ \ \ \ \ \ dim{\isacharunderscore}{\kern0pt}row{\isacharunderscore}{\kern0pt}of{\isacharunderscore}{\kern0pt}dagger\ index{\isacharunderscore}{\kern0pt}matrix{\isacharunderscore}{\kern0pt}prod\ j{\isadigit{2}}\ j{\isadigit{4}}\ zero{\isacharunderscore}{\kern0pt}less{\isacharunderscore}{\kern0pt}numeral{\isacharparenright}{\kern0pt}\isanewline
\ \ \ \ \ \ \ \ \ \ \ \ \ \ \isacommand{also}\isamarkupfalse%
\ \isacommand{have}\isamarkupfalse%
\ {\isachardoublequoteopen}{\isasymdots}\ {\isacharequal}{\kern0pt}\ {\isacharparenleft}{\kern0pt}{\isacharparenleft}{\kern0pt}control{\isadigit{2}}\ U{\isacharparenright}{\kern0pt}\isactrlsup {\isasymdagger}{\isacharparenright}{\kern0pt}\ {\isachardollar}{\kern0pt}{\isachardollar}{\kern0pt}\ {\isacharparenleft}{\kern0pt}{\isadigit{0}}{\isacharcomma}{\kern0pt}{\isadigit{2}}{\isacharparenright}{\kern0pt}{\isachardoublequoteclose}\isanewline
\ \ \ \ \ \ \ \ \ \ \ \ \ \ \ \ \isacommand{using}\isamarkupfalse%
\ control{\isadigit{2}}{\isacharunderscore}{\kern0pt}def\ index{\isacharunderscore}{\kern0pt}mat{\isacharunderscore}{\kern0pt}of{\isacharunderscore}{\kern0pt}cols{\isacharunderscore}{\kern0pt}list\ \isacommand{by}\isamarkupfalse%
\ force\isanewline
\ \ \ \ \ \ \ \ \ \ \ \ \ \ \isacommand{also}\isamarkupfalse%
\ \isacommand{have}\isamarkupfalse%
\ {\isachardoublequoteopen}{\isasymdots}\ {\isacharequal}{\kern0pt}\ cnj\ {\isacharparenleft}{\kern0pt}{\isacharparenleft}{\kern0pt}control{\isadigit{2}}\ U{\isacharparenright}{\kern0pt}\ {\isachardollar}{\kern0pt}{\isachardollar}{\kern0pt}\ {\isacharparenleft}{\kern0pt}{\isadigit{2}}{\isacharcomma}{\kern0pt}{\isadigit{0}}{\isacharparenright}{\kern0pt}{\isacharparenright}{\kern0pt}{\isachardoublequoteclose}\isanewline
\ \ \ \ \ \ \ \ \ \ \ \ \ \ \ \ \isacommand{using}\isamarkupfalse%
\ dagger{\isacharunderscore}{\kern0pt}def\isanewline
\ \ \ \ \ \ \ \ \ \ \ \ \ \ \ \ \isacommand{by}\isamarkupfalse%
\ {\isacharparenleft}{\kern0pt}simp\ add{\isacharcolon}{\kern0pt}\ Tensor{\isachardot}{\kern0pt}mat{\isacharunderscore}{\kern0pt}of{\isacharunderscore}{\kern0pt}cols{\isacharunderscore}{\kern0pt}list{\isacharunderscore}{\kern0pt}def\ control{\isadigit{2}}{\isacharunderscore}{\kern0pt}def{\isacharparenright}{\kern0pt}\isanewline
\ \ \ \ \ \ \ \ \ \ \ \ \ \ \isacommand{also}\isamarkupfalse%
\ \isacommand{have}\isamarkupfalse%
\ {\isachardoublequoteopen}{\isasymdots}\ {\isacharequal}{\kern0pt}\ {\isadigit{0}}{\isachardoublequoteclose}\ \isacommand{using}\isamarkupfalse%
\ control{\isadigit{2}}{\isacharunderscore}{\kern0pt}def\ index{\isacharunderscore}{\kern0pt}mat{\isacharunderscore}{\kern0pt}of{\isacharunderscore}{\kern0pt}cols{\isacharunderscore}{\kern0pt}list\ \isacommand{by}\isamarkupfalse%
\ auto\isanewline
\ \ \ \ \ \ \ \ \ \ \ \ \ \ \isacommand{also}\isamarkupfalse%
\ \isacommand{have}\isamarkupfalse%
\ {\isachardoublequoteopen}{\isasymdots}\ {\isacharequal}{\kern0pt}\ {\isadigit{1}}\isactrlsub m\ {\isadigit{4}}\ {\isachardollar}{\kern0pt}{\isachardollar}{\kern0pt}\ {\isacharparenleft}{\kern0pt}{\isadigit{0}}{\isacharcomma}{\kern0pt}{\isadigit{2}}{\isacharparenright}{\kern0pt}{\isachardoublequoteclose}\ \isacommand{by}\isamarkupfalse%
\ simp\isanewline
\ \ \ \ \ \ \ \ \ \ \ \ \ \ \isacommand{finally}\isamarkupfalse%
\ \isacommand{show}\isamarkupfalse%
\ {\isacharquery}{\kern0pt}thesis\ \isacommand{using}\isamarkupfalse%
\ i{\isadigit{0}}\ j{\isadigit{2}}\ \isacommand{by}\isamarkupfalse%
\ simp\isanewline
\ \ \ \ \ \ \ \ \ \ \ \ \isacommand{qed}\isamarkupfalse%
\isanewline
\ \ \ \ \ \ \ \ \ \ \isacommand{next}\isamarkupfalse%
\isanewline
\ \ \ \ \ \ \ \ \ \ \ \ \isacommand{assume}\isamarkupfalse%
\ j{\isadigit{3}}{\isacharcolon}{\kern0pt}{\isachardoublequoteopen}j\ {\isacharequal}{\kern0pt}\ {\isadigit{3}}{\isachardoublequoteclose}\isanewline
\ \ \ \ \ \ \ \ \ \ \ \ \isacommand{show}\isamarkupfalse%
\ {\isachardoublequoteopen}{\isacharparenleft}{\kern0pt}{\isacharparenleft}{\kern0pt}control{\isadigit{2}}\ U{\isacharparenright}{\kern0pt}\isactrlsup {\isasymdagger}\ {\isacharasterisk}{\kern0pt}\ control{\isadigit{2}}\ U{\isacharparenright}{\kern0pt}\ {\isachardollar}{\kern0pt}{\isachardollar}{\kern0pt}\ {\isacharparenleft}{\kern0pt}i{\isacharcomma}{\kern0pt}\ j{\isacharparenright}{\kern0pt}\ {\isacharequal}{\kern0pt}\ {\isadigit{1}}\isactrlsub m\ {\isadigit{4}}\ {\isachardollar}{\kern0pt}{\isachardollar}{\kern0pt}\ {\isacharparenleft}{\kern0pt}i{\isacharcomma}{\kern0pt}\ j{\isacharparenright}{\kern0pt}{\isachardoublequoteclose}\isanewline
\ \ \ \ \ \ \ \ \ \ \ \ \isacommand{proof}\isamarkupfalse%
\ {\isacharminus}{\kern0pt}\isanewline
\ \ \ \ \ \ \ \ \ \ \ \ \ \ \isacommand{have}\isamarkupfalse%
\ {\isachardoublequoteopen}{\isacharparenleft}{\kern0pt}{\isacharparenleft}{\kern0pt}control{\isadigit{2}}\ U{\isacharparenright}{\kern0pt}\isactrlsup {\isasymdagger}\ {\isacharasterisk}{\kern0pt}\ control{\isadigit{2}}\ U{\isacharparenright}{\kern0pt}\ {\isachardollar}{\kern0pt}{\isachardollar}{\kern0pt}\ {\isacharparenleft}{\kern0pt}{\isadigit{0}}{\isacharcomma}{\kern0pt}{\isadigit{3}}{\isacharparenright}{\kern0pt}\ {\isacharequal}{\kern0pt}\isanewline
\ \ \ \ \ \ \ \ \ \ \ \ \ \ \ \ {\isacharparenleft}{\kern0pt}{\isacharparenleft}{\kern0pt}control{\isadigit{2}}\ U{\isacharparenright}{\kern0pt}\isactrlsup {\isasymdagger}{\isacharparenright}{\kern0pt}\ {\isachardollar}{\kern0pt}{\isachardollar}{\kern0pt}\ {\isacharparenleft}{\kern0pt}{\isadigit{0}}{\isacharcomma}{\kern0pt}{\isadigit{0}}{\isacharparenright}{\kern0pt}\ {\isacharasterisk}{\kern0pt}\ {\isacharparenleft}{\kern0pt}control{\isadigit{2}}\ U{\isacharparenright}{\kern0pt}\ {\isachardollar}{\kern0pt}{\isachardollar}{\kern0pt}\ {\isacharparenleft}{\kern0pt}{\isadigit{0}}{\isacharcomma}{\kern0pt}{\isadigit{3}}{\isacharparenright}{\kern0pt}\ {\isacharplus}{\kern0pt}\isanewline
\ \ \ \ \ \ \ \ \ \ \ \ \ \ \ \ {\isacharparenleft}{\kern0pt}{\isacharparenleft}{\kern0pt}control{\isadigit{2}}\ U{\isacharparenright}{\kern0pt}\isactrlsup {\isasymdagger}{\isacharparenright}{\kern0pt}\ {\isachardollar}{\kern0pt}{\isachardollar}{\kern0pt}\ {\isacharparenleft}{\kern0pt}{\isadigit{0}}{\isacharcomma}{\kern0pt}{\isadigit{1}}{\isacharparenright}{\kern0pt}\ {\isacharasterisk}{\kern0pt}\ {\isacharparenleft}{\kern0pt}control{\isadigit{2}}\ U{\isacharparenright}{\kern0pt}\ {\isachardollar}{\kern0pt}{\isachardollar}{\kern0pt}\ {\isacharparenleft}{\kern0pt}{\isadigit{1}}{\isacharcomma}{\kern0pt}{\isadigit{3}}{\isacharparenright}{\kern0pt}\ {\isacharplus}{\kern0pt}\isanewline
\ \ \ \ \ \ \ \ \ \ \ \ \ \ \ \ {\isacharparenleft}{\kern0pt}{\isacharparenleft}{\kern0pt}control{\isadigit{2}}\ U{\isacharparenright}{\kern0pt}\isactrlsup {\isasymdagger}{\isacharparenright}{\kern0pt}\ {\isachardollar}{\kern0pt}{\isachardollar}{\kern0pt}\ {\isacharparenleft}{\kern0pt}{\isadigit{0}}{\isacharcomma}{\kern0pt}{\isadigit{2}}{\isacharparenright}{\kern0pt}\ {\isacharasterisk}{\kern0pt}\ {\isacharparenleft}{\kern0pt}control{\isadigit{2}}\ U{\isacharparenright}{\kern0pt}\ {\isachardollar}{\kern0pt}{\isachardollar}{\kern0pt}\ {\isacharparenleft}{\kern0pt}{\isadigit{2}}{\isacharcomma}{\kern0pt}{\isadigit{3}}{\isacharparenright}{\kern0pt}\ {\isacharplus}{\kern0pt}\isanewline
\ \ \ \ \ \ \ \ \ \ \ \ \ \ \ \ {\isacharparenleft}{\kern0pt}{\isacharparenleft}{\kern0pt}control{\isadigit{2}}\ U{\isacharparenright}{\kern0pt}\isactrlsup {\isasymdagger}{\isacharparenright}{\kern0pt}\ {\isachardollar}{\kern0pt}{\isachardollar}{\kern0pt}\ {\isacharparenleft}{\kern0pt}{\isadigit{0}}{\isacharcomma}{\kern0pt}{\isadigit{3}}{\isacharparenright}{\kern0pt}\ {\isacharasterisk}{\kern0pt}\ {\isacharparenleft}{\kern0pt}control{\isadigit{2}}\ U{\isacharparenright}{\kern0pt}\ {\isachardollar}{\kern0pt}{\isachardollar}{\kern0pt}\ {\isacharparenleft}{\kern0pt}{\isadigit{3}}{\isacharcomma}{\kern0pt}{\isadigit{3}}{\isacharparenright}{\kern0pt}{\isachardoublequoteclose}\isanewline
\ \ \ \ \ \ \ \ \ \ \ \ \ \ \ \ \isacommand{using}\isamarkupfalse%
\ sumof{\isadigit{4}}\isanewline
\ \ \ \ \ \ \ \ \ \ \ \ \ \ \ \ \isacommand{by}\isamarkupfalse%
\ {\isacharparenleft}{\kern0pt}smt\ {\isacharparenleft}{\kern0pt}z{\isadigit{3}}{\isacharparenright}{\kern0pt}\ carrier{\isacharunderscore}{\kern0pt}matD{\isacharparenleft}{\kern0pt}{\isadigit{1}}{\isacharparenright}{\kern0pt}\ carrier{\isacharunderscore}{\kern0pt}matD{\isacharparenleft}{\kern0pt}{\isadigit{2}}{\isacharparenright}{\kern0pt}\ control{\isadigit{2}}{\isacharunderscore}{\kern0pt}carrier{\isacharunderscore}{\kern0pt}mat\ dim{\isacharunderscore}{\kern0pt}col{\isacharunderscore}{\kern0pt}of{\isacharunderscore}{\kern0pt}dagger\ \isanewline
\ \ \ \ \ \ \ \ \ \ \ \ \ \ \ \ \ \ \ \ dim{\isacharunderscore}{\kern0pt}row{\isacharunderscore}{\kern0pt}of{\isacharunderscore}{\kern0pt}dagger\ index{\isacharunderscore}{\kern0pt}matrix{\isacharunderscore}{\kern0pt}prod\ j{\isadigit{3}}\ j{\isadigit{4}}\ zero{\isacharunderscore}{\kern0pt}less{\isacharunderscore}{\kern0pt}numeral{\isacharparenright}{\kern0pt}\isanewline
\ \ \ \ \ \ \ \ \ \ \ \ \ \ \isacommand{also}\isamarkupfalse%
\ \isacommand{have}\isamarkupfalse%
\ {\isachardoublequoteopen}{\isasymdots}\ {\isacharequal}{\kern0pt}\ {\isacharparenleft}{\kern0pt}{\isacharparenleft}{\kern0pt}control{\isadigit{2}}\ U{\isacharparenright}{\kern0pt}\isactrlsup {\isasymdagger}{\isacharparenright}{\kern0pt}\ {\isachardollar}{\kern0pt}{\isachardollar}{\kern0pt}\ {\isacharparenleft}{\kern0pt}{\isadigit{0}}{\isacharcomma}{\kern0pt}{\isadigit{1}}{\isacharparenright}{\kern0pt}\ {\isacharasterisk}{\kern0pt}\ {\isacharparenleft}{\kern0pt}control{\isadigit{2}}\ U{\isacharparenright}{\kern0pt}\ {\isachardollar}{\kern0pt}{\isachardollar}{\kern0pt}\ {\isacharparenleft}{\kern0pt}{\isadigit{1}}{\isacharcomma}{\kern0pt}{\isadigit{3}}{\isacharparenright}{\kern0pt}\ {\isacharplus}{\kern0pt}\isanewline
\ \ \ \ \ \ \ \ \ \ \ \ \ \ \ \ \ \ \ \ \ \ \ \ \ \ \ \ \ \ {\isacharparenleft}{\kern0pt}{\isacharparenleft}{\kern0pt}control{\isadigit{2}}\ U{\isacharparenright}{\kern0pt}\isactrlsup {\isasymdagger}{\isacharparenright}{\kern0pt}\ {\isachardollar}{\kern0pt}{\isachardollar}{\kern0pt}\ {\isacharparenleft}{\kern0pt}{\isadigit{0}}{\isacharcomma}{\kern0pt}{\isadigit{3}}{\isacharparenright}{\kern0pt}\ {\isacharasterisk}{\kern0pt}\ {\isacharparenleft}{\kern0pt}control{\isadigit{2}}\ U{\isacharparenright}{\kern0pt}\ {\isachardollar}{\kern0pt}{\isachardollar}{\kern0pt}\ {\isacharparenleft}{\kern0pt}{\isadigit{3}}{\isacharcomma}{\kern0pt}{\isadigit{3}}{\isacharparenright}{\kern0pt}{\isachardoublequoteclose}\isanewline
\ \ \ \ \ \ \ \ \ \ \ \ \ \ \ \ \isacommand{using}\isamarkupfalse%
\ control{\isadigit{2}}{\isacharunderscore}{\kern0pt}def\ index{\isacharunderscore}{\kern0pt}mat{\isacharunderscore}{\kern0pt}of{\isacharunderscore}{\kern0pt}cols{\isacharunderscore}{\kern0pt}list\ \isacommand{by}\isamarkupfalse%
\ force\isanewline
\ \ \ \ \ \ \ \ \ \ \ \ \ \ \isacommand{also}\isamarkupfalse%
\ \isacommand{have}\isamarkupfalse%
\ {\isachardoublequoteopen}{\isasymdots}\ {\isacharequal}{\kern0pt}\ cnj\ {\isacharparenleft}{\kern0pt}{\isacharparenleft}{\kern0pt}control{\isadigit{2}}\ U{\isacharparenright}{\kern0pt}\ {\isachardollar}{\kern0pt}{\isachardollar}{\kern0pt}\ {\isacharparenleft}{\kern0pt}{\isadigit{1}}{\isacharcomma}{\kern0pt}{\isadigit{0}}{\isacharparenright}{\kern0pt}{\isacharparenright}{\kern0pt}\ {\isacharasterisk}{\kern0pt}\ {\isacharparenleft}{\kern0pt}control{\isadigit{2}}\ U{\isacharparenright}{\kern0pt}\ {\isachardollar}{\kern0pt}{\isachardollar}{\kern0pt}\ {\isacharparenleft}{\kern0pt}{\isadigit{1}}{\isacharcomma}{\kern0pt}{\isadigit{3}}{\isacharparenright}{\kern0pt}\ {\isacharplus}{\kern0pt}\isanewline
\ \ \ \ \ \ \ \ \ \ \ \ \ \ \ \ \ \ \ \ \ \ \ \ \ \ \ \ \ \ cnj\ {\isacharparenleft}{\kern0pt}{\isacharparenleft}{\kern0pt}control{\isadigit{2}}\ U{\isacharparenright}{\kern0pt}\ {\isachardollar}{\kern0pt}{\isachardollar}{\kern0pt}\ {\isacharparenleft}{\kern0pt}{\isadigit{3}}{\isacharcomma}{\kern0pt}{\isadigit{0}}{\isacharparenright}{\kern0pt}{\isacharparenright}{\kern0pt}\ {\isacharasterisk}{\kern0pt}\ {\isacharparenleft}{\kern0pt}control{\isadigit{2}}\ U{\isacharparenright}{\kern0pt}\ {\isachardollar}{\kern0pt}{\isachardollar}{\kern0pt}\ {\isacharparenleft}{\kern0pt}{\isadigit{3}}{\isacharcomma}{\kern0pt}{\isadigit{3}}{\isacharparenright}{\kern0pt}{\isachardoublequoteclose}\isanewline
\ \ \ \ \ \ \ \ \ \ \ \ \ \ \ \ \isacommand{using}\isamarkupfalse%
\ dagger{\isacharunderscore}{\kern0pt}def\isanewline
\ \ \ \ \ \ \ \ \ \ \ \ \ \ \ \ \isacommand{by}\isamarkupfalse%
\ {\isacharparenleft}{\kern0pt}simp\ add{\isacharcolon}{\kern0pt}\ Tensor{\isachardot}{\kern0pt}mat{\isacharunderscore}{\kern0pt}of{\isacharunderscore}{\kern0pt}cols{\isacharunderscore}{\kern0pt}list{\isacharunderscore}{\kern0pt}def\ control{\isadigit{2}}{\isacharunderscore}{\kern0pt}def{\isacharparenright}{\kern0pt}\isanewline
\ \ \ \ \ \ \ \ \ \ \ \ \ \ \isacommand{also}\isamarkupfalse%
\ \isacommand{have}\isamarkupfalse%
\ {\isachardoublequoteopen}{\isasymdots}\ {\isacharequal}{\kern0pt}\ {\isadigit{0}}{\isachardoublequoteclose}\ \isacommand{using}\isamarkupfalse%
\ control{\isadigit{2}}{\isacharunderscore}{\kern0pt}def\ index{\isacharunderscore}{\kern0pt}mat{\isacharunderscore}{\kern0pt}of{\isacharunderscore}{\kern0pt}cols{\isacharunderscore}{\kern0pt}list\ \isacommand{by}\isamarkupfalse%
\ auto\isanewline
\ \ \ \ \ \ \ \ \ \ \ \ \ \ \isacommand{also}\isamarkupfalse%
\ \isacommand{have}\isamarkupfalse%
\ {\isachardoublequoteopen}{\isasymdots}\ {\isacharequal}{\kern0pt}\ {\isadigit{1}}\isactrlsub m\ {\isadigit{4}}\ {\isachardollar}{\kern0pt}{\isachardollar}{\kern0pt}\ {\isacharparenleft}{\kern0pt}{\isadigit{0}}{\isacharcomma}{\kern0pt}{\isadigit{3}}{\isacharparenright}{\kern0pt}{\isachardoublequoteclose}\ \isacommand{by}\isamarkupfalse%
\ simp\isanewline
\ \ \ \ \ \ \ \ \ \ \ \ \ \ \isacommand{finally}\isamarkupfalse%
\ \isacommand{show}\isamarkupfalse%
\ {\isacharquery}{\kern0pt}thesis\ \isacommand{using}\isamarkupfalse%
\ i{\isadigit{0}}\ j{\isadigit{3}}\ \isacommand{by}\isamarkupfalse%
\ simp\isanewline
\ \ \ \ \ \ \ \ \ \ \ \ \isacommand{qed}\isamarkupfalse%
\isanewline
\ \ \ \ \ \ \ \ \ \ \isacommand{qed}\isamarkupfalse%
\isanewline
\ \ \ \ \ \ \ \ \isacommand{qed}\isamarkupfalse%
\isanewline
\ \ \ \ \ \ \isacommand{qed}\isamarkupfalse%
\isanewline
\ \ \ \ \isacommand{next}\isamarkupfalse%
\isanewline
\ \ \ \ \ \ \isacommand{assume}\isamarkupfalse%
\ il{\isadigit{3}}{\isacharcolon}{\kern0pt}{\isachardoublequoteopen}i\ {\isacharequal}{\kern0pt}\ {\isadigit{1}}\ {\isasymor}\ i\ {\isacharequal}{\kern0pt}\ {\isadigit{2}}\ {\isasymor}\ i\ {\isacharequal}{\kern0pt}\ {\isadigit{3}}{\isachardoublequoteclose}\isanewline
\ \ \ \ \ \ \isacommand{show}\isamarkupfalse%
\ {\isachardoublequoteopen}{\isacharparenleft}{\kern0pt}{\isacharparenleft}{\kern0pt}control{\isadigit{2}}\ U{\isacharparenright}{\kern0pt}\isactrlsup {\isasymdagger}\ {\isacharasterisk}{\kern0pt}\ control{\isadigit{2}}\ U{\isacharparenright}{\kern0pt}\ {\isachardollar}{\kern0pt}{\isachardollar}{\kern0pt}\ {\isacharparenleft}{\kern0pt}i{\isacharcomma}{\kern0pt}\ j{\isacharparenright}{\kern0pt}\ {\isacharequal}{\kern0pt}\ {\isadigit{1}}\isactrlsub m\ {\isadigit{4}}\ {\isachardollar}{\kern0pt}{\isachardollar}{\kern0pt}\ {\isacharparenleft}{\kern0pt}i{\isacharcomma}{\kern0pt}\ j{\isacharparenright}{\kern0pt}{\isachardoublequoteclose}\isanewline
\ \ \ \ \ \ \isacommand{proof}\isamarkupfalse%
\ {\isacharparenleft}{\kern0pt}rule\ disjE{\isacharparenright}{\kern0pt}\isanewline
\ \ \ \ \ \ \ \ \isacommand{show}\isamarkupfalse%
\ {\isachardoublequoteopen}i\ {\isacharequal}{\kern0pt}\ {\isadigit{1}}\ {\isasymor}\ i\ {\isacharequal}{\kern0pt}\ {\isadigit{2}}\ {\isasymor}\ i\ {\isacharequal}{\kern0pt}\ {\isadigit{3}}{\isachardoublequoteclose}\ \isacommand{using}\isamarkupfalse%
\ il{\isadigit{3}}\ \isacommand{by}\isamarkupfalse%
\ this\isanewline
\ \ \ \ \ \ \isacommand{next}\isamarkupfalse%
\isanewline
\ \ \ \ \ \ \ \ \isacommand{assume}\isamarkupfalse%
\ i{\isadigit{1}}{\isacharcolon}{\kern0pt}{\isachardoublequoteopen}i\ {\isacharequal}{\kern0pt}\ {\isadigit{1}}{\isachardoublequoteclose}\isanewline
\ \ \ \ \ \ \ \ \isacommand{show}\isamarkupfalse%
\ {\isachardoublequoteopen}{\isacharparenleft}{\kern0pt}{\isacharparenleft}{\kern0pt}control{\isadigit{2}}\ U{\isacharparenright}{\kern0pt}\isactrlsup {\isasymdagger}\ {\isacharasterisk}{\kern0pt}\ control{\isadigit{2}}\ U{\isacharparenright}{\kern0pt}\ {\isachardollar}{\kern0pt}{\isachardollar}{\kern0pt}\ {\isacharparenleft}{\kern0pt}i{\isacharcomma}{\kern0pt}\ j{\isacharparenright}{\kern0pt}\ {\isacharequal}{\kern0pt}\ {\isadigit{1}}\isactrlsub m\ {\isadigit{4}}\ {\isachardollar}{\kern0pt}{\isachardollar}{\kern0pt}\ {\isacharparenleft}{\kern0pt}i{\isacharcomma}{\kern0pt}\ j{\isacharparenright}{\kern0pt}{\isachardoublequoteclose}\isanewline
\ \ \ \ \ \ \ \ \isacommand{proof}\isamarkupfalse%
\ {\isacharparenleft}{\kern0pt}rule\ disjE{\isacharparenright}{\kern0pt}\isanewline
\ \ \ \ \ \ \ \ \ \ \isacommand{show}\isamarkupfalse%
\ {\isachardoublequoteopen}j\ {\isacharequal}{\kern0pt}\ {\isadigit{0}}\ {\isasymor}\ j\ {\isacharequal}{\kern0pt}\ {\isadigit{1}}\ {\isasymor}\ j\ {\isacharequal}{\kern0pt}\ {\isadigit{2}}\ {\isasymor}\ j\ {\isacharequal}{\kern0pt}\ {\isadigit{3}}{\isachardoublequoteclose}\ \isacommand{using}\isamarkupfalse%
\ j{\isadigit{4}}\ \isacommand{by}\isamarkupfalse%
\ auto\isanewline
\ \ \ \ \ \ \ \ \isacommand{next}\isamarkupfalse%
\isanewline
\ \ \ \ \ \ \ \ \ \ \isacommand{assume}\isamarkupfalse%
\ j{\isadigit{0}}{\isacharcolon}{\kern0pt}{\isachardoublequoteopen}j\ {\isacharequal}{\kern0pt}\ {\isadigit{0}}{\isachardoublequoteclose}\isanewline
\ \ \ \ \ \ \ \ \ \ \isacommand{show}\isamarkupfalse%
\ {\isachardoublequoteopen}{\isacharparenleft}{\kern0pt}{\isacharparenleft}{\kern0pt}control{\isadigit{2}}\ U{\isacharparenright}{\kern0pt}\isactrlsup {\isasymdagger}\ {\isacharasterisk}{\kern0pt}\ control{\isadigit{2}}\ U{\isacharparenright}{\kern0pt}\ {\isachardollar}{\kern0pt}{\isachardollar}{\kern0pt}\ {\isacharparenleft}{\kern0pt}i{\isacharcomma}{\kern0pt}\ j{\isacharparenright}{\kern0pt}\ {\isacharequal}{\kern0pt}\ {\isadigit{1}}\isactrlsub m\ {\isadigit{4}}\ {\isachardollar}{\kern0pt}{\isachardollar}{\kern0pt}\ {\isacharparenleft}{\kern0pt}i{\isacharcomma}{\kern0pt}\ j{\isacharparenright}{\kern0pt}{\isachardoublequoteclose}\isanewline
\ \ \ \ \ \ \ \ \ \ \isacommand{proof}\isamarkupfalse%
\ {\isacharminus}{\kern0pt}\isanewline
\ \ \ \ \ \ \ \ \ \ \ \ \isacommand{have}\isamarkupfalse%
\ {\isachardoublequoteopen}{\isacharparenleft}{\kern0pt}{\isacharparenleft}{\kern0pt}control{\isadigit{2}}\ U{\isacharparenright}{\kern0pt}\isactrlsup {\isasymdagger}\ {\isacharasterisk}{\kern0pt}\ control{\isadigit{2}}\ U{\isacharparenright}{\kern0pt}\ {\isachardollar}{\kern0pt}{\isachardollar}{\kern0pt}\ {\isacharparenleft}{\kern0pt}{\isadigit{1}}{\isacharcomma}{\kern0pt}{\isadigit{0}}{\isacharparenright}{\kern0pt}\ {\isacharequal}{\kern0pt}\isanewline
\ \ \ \ \ \ \ \ \ \ \ \ \ \ \ \ {\isacharparenleft}{\kern0pt}{\isacharparenleft}{\kern0pt}control{\isadigit{2}}\ U{\isacharparenright}{\kern0pt}\isactrlsup {\isasymdagger}{\isacharparenright}{\kern0pt}\ {\isachardollar}{\kern0pt}{\isachardollar}{\kern0pt}\ {\isacharparenleft}{\kern0pt}{\isadigit{1}}{\isacharcomma}{\kern0pt}{\isadigit{0}}{\isacharparenright}{\kern0pt}\ {\isacharasterisk}{\kern0pt}\ {\isacharparenleft}{\kern0pt}control{\isadigit{2}}\ U{\isacharparenright}{\kern0pt}\ {\isachardollar}{\kern0pt}{\isachardollar}{\kern0pt}\ {\isacharparenleft}{\kern0pt}{\isadigit{0}}{\isacharcomma}{\kern0pt}{\isadigit{0}}{\isacharparenright}{\kern0pt}\ {\isacharplus}{\kern0pt}\isanewline
\ \ \ \ \ \ \ \ \ \ \ \ \ \ \ \ {\isacharparenleft}{\kern0pt}{\isacharparenleft}{\kern0pt}control{\isadigit{2}}\ U{\isacharparenright}{\kern0pt}\isactrlsup {\isasymdagger}{\isacharparenright}{\kern0pt}\ {\isachardollar}{\kern0pt}{\isachardollar}{\kern0pt}\ {\isacharparenleft}{\kern0pt}{\isadigit{1}}{\isacharcomma}{\kern0pt}{\isadigit{1}}{\isacharparenright}{\kern0pt}\ {\isacharasterisk}{\kern0pt}\ {\isacharparenleft}{\kern0pt}control{\isadigit{2}}\ U{\isacharparenright}{\kern0pt}\ {\isachardollar}{\kern0pt}{\isachardollar}{\kern0pt}\ {\isacharparenleft}{\kern0pt}{\isadigit{1}}{\isacharcomma}{\kern0pt}{\isadigit{0}}{\isacharparenright}{\kern0pt}\ {\isacharplus}{\kern0pt}\isanewline
\ \ \ \ \ \ \ \ \ \ \ \ \ \ \ \ {\isacharparenleft}{\kern0pt}{\isacharparenleft}{\kern0pt}control{\isadigit{2}}\ U{\isacharparenright}{\kern0pt}\isactrlsup {\isasymdagger}{\isacharparenright}{\kern0pt}\ {\isachardollar}{\kern0pt}{\isachardollar}{\kern0pt}\ {\isacharparenleft}{\kern0pt}{\isadigit{1}}{\isacharcomma}{\kern0pt}{\isadigit{2}}{\isacharparenright}{\kern0pt}\ {\isacharasterisk}{\kern0pt}\ {\isacharparenleft}{\kern0pt}control{\isadigit{2}}\ U{\isacharparenright}{\kern0pt}\ {\isachardollar}{\kern0pt}{\isachardollar}{\kern0pt}\ {\isacharparenleft}{\kern0pt}{\isadigit{2}}{\isacharcomma}{\kern0pt}{\isadigit{0}}{\isacharparenright}{\kern0pt}\ {\isacharplus}{\kern0pt}\isanewline
\ \ \ \ \ \ \ \ \ \ \ \ \ \ \ \ {\isacharparenleft}{\kern0pt}{\isacharparenleft}{\kern0pt}control{\isadigit{2}}\ U{\isacharparenright}{\kern0pt}\isactrlsup {\isasymdagger}{\isacharparenright}{\kern0pt}\ {\isachardollar}{\kern0pt}{\isachardollar}{\kern0pt}\ {\isacharparenleft}{\kern0pt}{\isadigit{1}}{\isacharcomma}{\kern0pt}{\isadigit{3}}{\isacharparenright}{\kern0pt}\ {\isacharasterisk}{\kern0pt}\ {\isacharparenleft}{\kern0pt}control{\isadigit{2}}\ U{\isacharparenright}{\kern0pt}\ {\isachardollar}{\kern0pt}{\isachardollar}{\kern0pt}\ {\isacharparenleft}{\kern0pt}{\isadigit{3}}{\isacharcomma}{\kern0pt}{\isadigit{0}}{\isacharparenright}{\kern0pt}{\isachardoublequoteclose}\isanewline
\ \ \ \ \ \ \ \ \ \ \ \ \ \ \isacommand{using}\isamarkupfalse%
\ sumof{\isadigit{4}}\isanewline
\ \ \ \ \ \ \ \ \ \ \ \ \ \ \isacommand{by}\isamarkupfalse%
\ {\isacharparenleft}{\kern0pt}smt\ {\isacharparenleft}{\kern0pt}z{\isadigit{3}}{\isacharparenright}{\kern0pt}\ carrier{\isacharunderscore}{\kern0pt}matD{\isacharparenleft}{\kern0pt}{\isadigit{1}}{\isacharparenright}{\kern0pt}\ carrier{\isacharunderscore}{\kern0pt}matD{\isacharparenleft}{\kern0pt}{\isadigit{2}}{\isacharparenright}{\kern0pt}\ control{\isadigit{2}}{\isacharunderscore}{\kern0pt}carrier{\isacharunderscore}{\kern0pt}mat\ dim{\isacharunderscore}{\kern0pt}col{\isacharunderscore}{\kern0pt}of{\isacharunderscore}{\kern0pt}dagger\ \isanewline
\ \ \ \ \ \ \ \ \ \ \ \ \ \ \ \ \ \ dim{\isacharunderscore}{\kern0pt}row{\isacharunderscore}{\kern0pt}of{\isacharunderscore}{\kern0pt}dagger\ index{\isacharunderscore}{\kern0pt}matrix{\isacharunderscore}{\kern0pt}prod\ one{\isacharunderscore}{\kern0pt}less{\isacharunderscore}{\kern0pt}numeral{\isacharunderscore}{\kern0pt}iff\ semiring{\isacharunderscore}{\kern0pt}norm{\isacharparenleft}{\kern0pt}{\isadigit{7}}{\isadigit{6}}{\isacharparenright}{\kern0pt}\ \isanewline
\ \ \ \ \ \ \ \ \ \ \ \ \ \ \ \ \ \ zero{\isacharunderscore}{\kern0pt}less{\isacharunderscore}{\kern0pt}numeral{\isacharparenright}{\kern0pt}\isanewline
\ \ \ \ \ \ \ \ \ \ \ \ \isacommand{also}\isamarkupfalse%
\ \isacommand{have}\isamarkupfalse%
\ {\isachardoublequoteopen}{\isasymdots}\ {\isacharequal}{\kern0pt}\ {\isacharparenleft}{\kern0pt}{\isacharparenleft}{\kern0pt}control{\isadigit{2}}\ U{\isacharparenright}{\kern0pt}\isactrlsup {\isasymdagger}{\isacharparenright}{\kern0pt}\ {\isachardollar}{\kern0pt}{\isachardollar}{\kern0pt}\ {\isacharparenleft}{\kern0pt}{\isadigit{1}}{\isacharcomma}{\kern0pt}{\isadigit{0}}{\isacharparenright}{\kern0pt}{\isachardoublequoteclose}\isanewline
\ \ \ \ \ \ \ \ \ \ \ \ \ \ \isacommand{using}\isamarkupfalse%
\ control{\isadigit{2}}{\isacharunderscore}{\kern0pt}def\ index{\isacharunderscore}{\kern0pt}mat{\isacharunderscore}{\kern0pt}of{\isacharunderscore}{\kern0pt}cols{\isacharunderscore}{\kern0pt}list\ \isacommand{by}\isamarkupfalse%
\ force\isanewline
\ \ \ \ \ \ \ \ \ \ \ \ \isacommand{also}\isamarkupfalse%
\ \isacommand{have}\isamarkupfalse%
\ {\isachardoublequoteopen}{\isasymdots}\ {\isacharequal}{\kern0pt}\ cnj\ {\isacharparenleft}{\kern0pt}{\isacharparenleft}{\kern0pt}control{\isadigit{2}}\ U{\isacharparenright}{\kern0pt}\ {\isachardollar}{\kern0pt}{\isachardollar}{\kern0pt}\ {\isacharparenleft}{\kern0pt}{\isadigit{0}}{\isacharcomma}{\kern0pt}{\isadigit{1}}{\isacharparenright}{\kern0pt}{\isacharparenright}{\kern0pt}{\isachardoublequoteclose}\isanewline
\ \ \ \ \ \ \ \ \ \ \ \ \ \ \isacommand{using}\isamarkupfalse%
\ dagger{\isacharunderscore}{\kern0pt}def\isanewline
\ \ \ \ \ \ \ \ \ \ \ \ \ \ \isacommand{by}\isamarkupfalse%
\ {\isacharparenleft}{\kern0pt}simp\ add{\isacharcolon}{\kern0pt}\ Tensor{\isachardot}{\kern0pt}mat{\isacharunderscore}{\kern0pt}of{\isacharunderscore}{\kern0pt}cols{\isacharunderscore}{\kern0pt}list{\isacharunderscore}{\kern0pt}def\ control{\isadigit{2}}{\isacharunderscore}{\kern0pt}def{\isacharparenright}{\kern0pt}\isanewline
\ \ \ \ \ \ \ \ \ \ \ \ \isacommand{also}\isamarkupfalse%
\ \isacommand{have}\isamarkupfalse%
\ {\isachardoublequoteopen}{\isasymdots}\ {\isacharequal}{\kern0pt}\ {\isadigit{0}}{\isachardoublequoteclose}\ \isacommand{using}\isamarkupfalse%
\ control{\isadigit{2}}{\isacharunderscore}{\kern0pt}def\ index{\isacharunderscore}{\kern0pt}mat{\isacharunderscore}{\kern0pt}of{\isacharunderscore}{\kern0pt}cols{\isacharunderscore}{\kern0pt}list\ \isacommand{by}\isamarkupfalse%
\ auto\isanewline
\ \ \ \ \ \ \ \ \ \ \ \ \isacommand{also}\isamarkupfalse%
\ \isacommand{have}\isamarkupfalse%
\ {\isachardoublequoteopen}{\isasymdots}\ {\isacharequal}{\kern0pt}\ {\isadigit{1}}\isactrlsub m\ {\isadigit{4}}\ {\isachardollar}{\kern0pt}{\isachardollar}{\kern0pt}\ {\isacharparenleft}{\kern0pt}{\isadigit{1}}{\isacharcomma}{\kern0pt}{\isadigit{0}}{\isacharparenright}{\kern0pt}{\isachardoublequoteclose}\ \isacommand{by}\isamarkupfalse%
\ simp\isanewline
\ \ \ \ \ \ \ \ \ \ \ \ \isacommand{finally}\isamarkupfalse%
\ \isacommand{show}\isamarkupfalse%
\ {\isacharquery}{\kern0pt}thesis\ \isacommand{using}\isamarkupfalse%
\ i{\isadigit{1}}\ j{\isadigit{0}}\ \isacommand{by}\isamarkupfalse%
\ simp\isanewline
\ \ \ \ \ \ \ \ \ \ \isacommand{qed}\isamarkupfalse%
\isanewline
\ \ \ \ \ \ \ \ \isacommand{next}\isamarkupfalse%
\isanewline
\ \ \ \ \ \ \ \ \ \ \isacommand{assume}\isamarkupfalse%
\ jl{\isadigit{3}}{\isacharcolon}{\kern0pt}{\isachardoublequoteopen}j\ {\isacharequal}{\kern0pt}\ {\isadigit{1}}\ {\isasymor}\ j\ {\isacharequal}{\kern0pt}\ {\isadigit{2}}\ {\isasymor}\ j\ {\isacharequal}{\kern0pt}\ {\isadigit{3}}{\isachardoublequoteclose}\isanewline
\ \ \ \ \ \ \ \ \ \ \isacommand{show}\isamarkupfalse%
\ {\isachardoublequoteopen}{\isacharparenleft}{\kern0pt}{\isacharparenleft}{\kern0pt}control{\isadigit{2}}\ U{\isacharparenright}{\kern0pt}\isactrlsup {\isasymdagger}\ {\isacharasterisk}{\kern0pt}\ control{\isadigit{2}}\ U{\isacharparenright}{\kern0pt}\ {\isachardollar}{\kern0pt}{\isachardollar}{\kern0pt}\ {\isacharparenleft}{\kern0pt}i{\isacharcomma}{\kern0pt}\ j{\isacharparenright}{\kern0pt}\ {\isacharequal}{\kern0pt}\ {\isadigit{1}}\isactrlsub m\ {\isadigit{4}}\ {\isachardollar}{\kern0pt}{\isachardollar}{\kern0pt}\ {\isacharparenleft}{\kern0pt}i{\isacharcomma}{\kern0pt}\ j{\isacharparenright}{\kern0pt}{\isachardoublequoteclose}\isanewline
\ \ \ \ \ \ \ \ \ \ \isacommand{proof}\isamarkupfalse%
\ {\isacharparenleft}{\kern0pt}rule\ disjE{\isacharparenright}{\kern0pt}\isanewline
\ \ \ \ \ \ \ \ \ \ \ \ \isacommand{show}\isamarkupfalse%
\ {\isachardoublequoteopen}j\ {\isacharequal}{\kern0pt}\ {\isadigit{1}}\ {\isasymor}\ j\ {\isacharequal}{\kern0pt}\ {\isadigit{2}}\ {\isasymor}\ j\ {\isacharequal}{\kern0pt}\ {\isadigit{3}}{\isachardoublequoteclose}\ \isacommand{using}\isamarkupfalse%
\ jl{\isadigit{3}}\ \isacommand{by}\isamarkupfalse%
\ this\isanewline
\ \ \ \ \ \ \ \ \ \ \isacommand{next}\isamarkupfalse%
\isanewline
\ \ \ \ \ \ \ \ \ \ \ \ \isacommand{assume}\isamarkupfalse%
\ j{\isadigit{1}}{\isacharcolon}{\kern0pt}{\isachardoublequoteopen}j\ {\isacharequal}{\kern0pt}\ {\isadigit{1}}{\isachardoublequoteclose}\isanewline
\ \ \ \ \ \ \ \ \ \ \ \ \isacommand{show}\isamarkupfalse%
\ {\isachardoublequoteopen}{\isacharparenleft}{\kern0pt}{\isacharparenleft}{\kern0pt}control{\isadigit{2}}\ U{\isacharparenright}{\kern0pt}\isactrlsup {\isasymdagger}\ {\isacharasterisk}{\kern0pt}\ control{\isadigit{2}}\ U{\isacharparenright}{\kern0pt}\ {\isachardollar}{\kern0pt}{\isachardollar}{\kern0pt}\ {\isacharparenleft}{\kern0pt}i{\isacharcomma}{\kern0pt}\ j{\isacharparenright}{\kern0pt}\ {\isacharequal}{\kern0pt}\ {\isadigit{1}}\isactrlsub m\ {\isadigit{4}}\ {\isachardollar}{\kern0pt}{\isachardollar}{\kern0pt}\ {\isacharparenleft}{\kern0pt}i{\isacharcomma}{\kern0pt}\ j{\isacharparenright}{\kern0pt}{\isachardoublequoteclose}\isanewline
\ \ \ \ \ \ \ \ \ \ \ \ \isacommand{proof}\isamarkupfalse%
\ {\isacharminus}{\kern0pt}\isanewline
\ \ \ \ \ \ \ \ \ \ \ \ \ \ \isacommand{have}\isamarkupfalse%
\ {\isachardoublequoteopen}{\isacharparenleft}{\kern0pt}{\isacharparenleft}{\kern0pt}control{\isadigit{2}}\ U{\isacharparenright}{\kern0pt}\isactrlsup {\isasymdagger}\ {\isacharasterisk}{\kern0pt}\ control{\isadigit{2}}\ U{\isacharparenright}{\kern0pt}\ {\isachardollar}{\kern0pt}{\isachardollar}{\kern0pt}\ {\isacharparenleft}{\kern0pt}{\isadigit{1}}{\isacharcomma}{\kern0pt}{\isadigit{1}}{\isacharparenright}{\kern0pt}\ {\isacharequal}{\kern0pt}\isanewline
\ \ \ \ \ \ \ \ \ \ \ \ \ \ \ \ {\isacharparenleft}{\kern0pt}{\isacharparenleft}{\kern0pt}control{\isadigit{2}}\ U{\isacharparenright}{\kern0pt}\isactrlsup {\isasymdagger}{\isacharparenright}{\kern0pt}\ {\isachardollar}{\kern0pt}{\isachardollar}{\kern0pt}\ {\isacharparenleft}{\kern0pt}{\isadigit{1}}{\isacharcomma}{\kern0pt}{\isadigit{0}}{\isacharparenright}{\kern0pt}\ {\isacharasterisk}{\kern0pt}\ {\isacharparenleft}{\kern0pt}control{\isadigit{2}}\ U{\isacharparenright}{\kern0pt}\ {\isachardollar}{\kern0pt}{\isachardollar}{\kern0pt}\ {\isacharparenleft}{\kern0pt}{\isadigit{0}}{\isacharcomma}{\kern0pt}{\isadigit{1}}{\isacharparenright}{\kern0pt}\ {\isacharplus}{\kern0pt}\isanewline
\ \ \ \ \ \ \ \ \ \ \ \ \ \ \ \ {\isacharparenleft}{\kern0pt}{\isacharparenleft}{\kern0pt}control{\isadigit{2}}\ U{\isacharparenright}{\kern0pt}\isactrlsup {\isasymdagger}{\isacharparenright}{\kern0pt}\ {\isachardollar}{\kern0pt}{\isachardollar}{\kern0pt}\ {\isacharparenleft}{\kern0pt}{\isadigit{1}}{\isacharcomma}{\kern0pt}{\isadigit{1}}{\isacharparenright}{\kern0pt}\ {\isacharasterisk}{\kern0pt}\ {\isacharparenleft}{\kern0pt}control{\isadigit{2}}\ U{\isacharparenright}{\kern0pt}\ {\isachardollar}{\kern0pt}{\isachardollar}{\kern0pt}\ {\isacharparenleft}{\kern0pt}{\isadigit{1}}{\isacharcomma}{\kern0pt}{\isadigit{1}}{\isacharparenright}{\kern0pt}\ {\isacharplus}{\kern0pt}\isanewline
\ \ \ \ \ \ \ \ \ \ \ \ \ \ \ \ {\isacharparenleft}{\kern0pt}{\isacharparenleft}{\kern0pt}control{\isadigit{2}}\ U{\isacharparenright}{\kern0pt}\isactrlsup {\isasymdagger}{\isacharparenright}{\kern0pt}\ {\isachardollar}{\kern0pt}{\isachardollar}{\kern0pt}\ {\isacharparenleft}{\kern0pt}{\isadigit{1}}{\isacharcomma}{\kern0pt}{\isadigit{2}}{\isacharparenright}{\kern0pt}\ {\isacharasterisk}{\kern0pt}\ {\isacharparenleft}{\kern0pt}control{\isadigit{2}}\ U{\isacharparenright}{\kern0pt}\ {\isachardollar}{\kern0pt}{\isachardollar}{\kern0pt}\ {\isacharparenleft}{\kern0pt}{\isadigit{2}}{\isacharcomma}{\kern0pt}{\isadigit{1}}{\isacharparenright}{\kern0pt}\ {\isacharplus}{\kern0pt}\isanewline
\ \ \ \ \ \ \ \ \ \ \ \ \ \ \ \ {\isacharparenleft}{\kern0pt}{\isacharparenleft}{\kern0pt}control{\isadigit{2}}\ U{\isacharparenright}{\kern0pt}\isactrlsup {\isasymdagger}{\isacharparenright}{\kern0pt}\ {\isachardollar}{\kern0pt}{\isachardollar}{\kern0pt}\ {\isacharparenleft}{\kern0pt}{\isadigit{1}}{\isacharcomma}{\kern0pt}{\isadigit{3}}{\isacharparenright}{\kern0pt}\ {\isacharasterisk}{\kern0pt}\ {\isacharparenleft}{\kern0pt}control{\isadigit{2}}\ U{\isacharparenright}{\kern0pt}\ {\isachardollar}{\kern0pt}{\isachardollar}{\kern0pt}\ {\isacharparenleft}{\kern0pt}{\isadigit{3}}{\isacharcomma}{\kern0pt}{\isadigit{1}}{\isacharparenright}{\kern0pt}{\isachardoublequoteclose}\isanewline
\ \ \ \ \ \ \ \ \ \ \ \ \ \ \ \ \isacommand{using}\isamarkupfalse%
\ sumof{\isadigit{4}}\isanewline
\ \ \ \ \ \ \ \ \ \ \ \ \ \ \ \ \isacommand{by}\isamarkupfalse%
\ {\isacharparenleft}{\kern0pt}smt\ {\isacharparenleft}{\kern0pt}z{\isadigit{3}}{\isacharparenright}{\kern0pt}\ carrier{\isacharunderscore}{\kern0pt}matD{\isacharparenleft}{\kern0pt}{\isadigit{1}}{\isacharparenright}{\kern0pt}\ carrier{\isacharunderscore}{\kern0pt}matD{\isacharparenleft}{\kern0pt}{\isadigit{2}}{\isacharparenright}{\kern0pt}\ control{\isadigit{2}}{\isacharunderscore}{\kern0pt}carrier{\isacharunderscore}{\kern0pt}mat\ dim{\isacharunderscore}{\kern0pt}col{\isacharunderscore}{\kern0pt}of{\isacharunderscore}{\kern0pt}dagger\ \isanewline
\ \ \ \ \ \ \ \ \ \ \ \ \ \ \ \ \ \ \ \ dim{\isacharunderscore}{\kern0pt}row{\isacharunderscore}{\kern0pt}of{\isacharunderscore}{\kern0pt}dagger\ index{\isacharunderscore}{\kern0pt}matrix{\isacharunderscore}{\kern0pt}prod\ one{\isacharunderscore}{\kern0pt}less{\isacharunderscore}{\kern0pt}numeral{\isacharunderscore}{\kern0pt}iff\ semiring{\isacharunderscore}{\kern0pt}norm{\isacharparenleft}{\kern0pt}{\isadigit{7}}{\isadigit{6}}{\isacharparenright}{\kern0pt}\ \isanewline
\ \ \ \ \ \ \ \ \ \ \ \ \ \ \ \ \ \ \ \ zero{\isacharunderscore}{\kern0pt}less{\isacharunderscore}{\kern0pt}numeral{\isacharparenright}{\kern0pt}\isanewline
\ \ \ \ \ \ \ \ \ \ \ \ \ \ \isacommand{also}\isamarkupfalse%
\ \isacommand{have}\isamarkupfalse%
\ {\isachardoublequoteopen}{\isasymdots}\ {\isacharequal}{\kern0pt}\ {\isacharparenleft}{\kern0pt}{\isacharparenleft}{\kern0pt}control{\isadigit{2}}\ U{\isacharparenright}{\kern0pt}\isactrlsup {\isasymdagger}{\isacharparenright}{\kern0pt}\ {\isachardollar}{\kern0pt}{\isachardollar}{\kern0pt}\ {\isacharparenleft}{\kern0pt}{\isadigit{1}}{\isacharcomma}{\kern0pt}{\isadigit{1}}{\isacharparenright}{\kern0pt}\ {\isacharasterisk}{\kern0pt}\ {\isacharparenleft}{\kern0pt}control{\isadigit{2}}\ U{\isacharparenright}{\kern0pt}\ {\isachardollar}{\kern0pt}{\isachardollar}{\kern0pt}\ {\isacharparenleft}{\kern0pt}{\isadigit{1}}{\isacharcomma}{\kern0pt}{\isadigit{1}}{\isacharparenright}{\kern0pt}\ {\isacharplus}{\kern0pt}\isanewline
\ \ \ \ \ \ \ \ \ \ \ \ \ \ \ \ \ \ \ \ \ \ \ \ \ \ \ \ \ \ {\isacharparenleft}{\kern0pt}{\isacharparenleft}{\kern0pt}control{\isadigit{2}}\ U{\isacharparenright}{\kern0pt}\isactrlsup {\isasymdagger}{\isacharparenright}{\kern0pt}\ {\isachardollar}{\kern0pt}{\isachardollar}{\kern0pt}\ {\isacharparenleft}{\kern0pt}{\isadigit{1}}{\isacharcomma}{\kern0pt}{\isadigit{3}}{\isacharparenright}{\kern0pt}\ {\isacharasterisk}{\kern0pt}\ {\isacharparenleft}{\kern0pt}control{\isadigit{2}}\ U{\isacharparenright}{\kern0pt}\ {\isachardollar}{\kern0pt}{\isachardollar}{\kern0pt}\ {\isacharparenleft}{\kern0pt}{\isadigit{3}}{\isacharcomma}{\kern0pt}{\isadigit{1}}{\isacharparenright}{\kern0pt}{\isachardoublequoteclose}\isanewline
\ \ \ \ \ \ \ \ \ \ \ \ \ \ \ \ \isacommand{using}\isamarkupfalse%
\ control{\isadigit{2}}{\isacharunderscore}{\kern0pt}def\ index{\isacharunderscore}{\kern0pt}mat{\isacharunderscore}{\kern0pt}of{\isacharunderscore}{\kern0pt}cols{\isacharunderscore}{\kern0pt}list\ \isacommand{by}\isamarkupfalse%
\ force\isanewline
\ \ \ \ \ \ \ \ \ \ \ \ \ \ \isacommand{also}\isamarkupfalse%
\ \isacommand{have}\isamarkupfalse%
\ {\isachardoublequoteopen}{\isasymdots}\ {\isacharequal}{\kern0pt}\ cnj\ {\isacharparenleft}{\kern0pt}{\isacharparenleft}{\kern0pt}control{\isadigit{2}}\ U{\isacharparenright}{\kern0pt}\ {\isachardollar}{\kern0pt}{\isachardollar}{\kern0pt}\ {\isacharparenleft}{\kern0pt}{\isadigit{1}}{\isacharcomma}{\kern0pt}{\isadigit{1}}{\isacharparenright}{\kern0pt}{\isacharparenright}{\kern0pt}\ {\isacharasterisk}{\kern0pt}\ {\isacharparenleft}{\kern0pt}control{\isadigit{2}}\ U{\isacharparenright}{\kern0pt}\ {\isachardollar}{\kern0pt}{\isachardollar}{\kern0pt}\ {\isacharparenleft}{\kern0pt}{\isadigit{1}}{\isacharcomma}{\kern0pt}{\isadigit{1}}{\isacharparenright}{\kern0pt}\ {\isacharplus}{\kern0pt}\isanewline
\ \ \ \ \ \ \ \ \ \ \ \ \ \ \ \ \ \ \ \ \ \ \ \ \ \ \ \ \ \ cnj\ {\isacharparenleft}{\kern0pt}{\isacharparenleft}{\kern0pt}control{\isadigit{2}}\ U{\isacharparenright}{\kern0pt}\ {\isachardollar}{\kern0pt}{\isachardollar}{\kern0pt}\ {\isacharparenleft}{\kern0pt}{\isadigit{3}}{\isacharcomma}{\kern0pt}{\isadigit{1}}{\isacharparenright}{\kern0pt}{\isacharparenright}{\kern0pt}\ {\isacharasterisk}{\kern0pt}\ {\isacharparenleft}{\kern0pt}control{\isadigit{2}}\ U{\isacharparenright}{\kern0pt}\ {\isachardollar}{\kern0pt}{\isachardollar}{\kern0pt}\ {\isacharparenleft}{\kern0pt}{\isadigit{3}}{\isacharcomma}{\kern0pt}{\isadigit{1}}{\isacharparenright}{\kern0pt}{\isachardoublequoteclose}\isanewline
\ \ \ \ \ \ \ \ \ \ \ \ \ \ \ \ \isacommand{using}\isamarkupfalse%
\ dagger{\isacharunderscore}{\kern0pt}def\isanewline
\ \ \ \ \ \ \ \ \ \ \ \ \ \ \ \ \isacommand{by}\isamarkupfalse%
\ {\isacharparenleft}{\kern0pt}simp\ add{\isacharcolon}{\kern0pt}\ Tensor{\isachardot}{\kern0pt}mat{\isacharunderscore}{\kern0pt}of{\isacharunderscore}{\kern0pt}cols{\isacharunderscore}{\kern0pt}list{\isacharunderscore}{\kern0pt}def\ control{\isadigit{2}}{\isacharunderscore}{\kern0pt}def{\isacharparenright}{\kern0pt}\isanewline
\ \ \ \ \ \ \ \ \ \ \ \ \ \ \isacommand{also}\isamarkupfalse%
\ \isacommand{have}\isamarkupfalse%
\ {\isachardoublequoteopen}{\isasymdots}\ {\isacharequal}{\kern0pt}\ cnj\ {\isacharparenleft}{\kern0pt}U\ {\isachardollar}{\kern0pt}{\isachardollar}{\kern0pt}\ {\isacharparenleft}{\kern0pt}{\isadigit{0}}{\isacharcomma}{\kern0pt}{\isadigit{0}}{\isacharparenright}{\kern0pt}{\isacharparenright}{\kern0pt}\ {\isacharasterisk}{\kern0pt}\ {\isacharparenleft}{\kern0pt}U\ {\isachardollar}{\kern0pt}{\isachardollar}{\kern0pt}\ {\isacharparenleft}{\kern0pt}{\isadigit{0}}{\isacharcomma}{\kern0pt}{\isadigit{0}}{\isacharparenright}{\kern0pt}{\isacharparenright}{\kern0pt}\ {\isacharplus}{\kern0pt}\isanewline
\ \ \ \ \ \ \ \ \ \ \ \ \ \ \ \ \ \ \ \ \ \ \ \ \ \ \ \ \ \ cnj\ {\isacharparenleft}{\kern0pt}U\ {\isachardollar}{\kern0pt}{\isachardollar}{\kern0pt}\ {\isacharparenleft}{\kern0pt}{\isadigit{1}}{\isacharcomma}{\kern0pt}{\isadigit{0}}{\isacharparenright}{\kern0pt}{\isacharparenright}{\kern0pt}\ {\isacharasterisk}{\kern0pt}\ {\isacharparenleft}{\kern0pt}U\ {\isachardollar}{\kern0pt}{\isachardollar}{\kern0pt}\ {\isacharparenleft}{\kern0pt}{\isadigit{1}}{\isacharcomma}{\kern0pt}{\isadigit{0}}{\isacharparenright}{\kern0pt}{\isacharparenright}{\kern0pt}{\isachardoublequoteclose}\isanewline
\ \ \ \ \ \ \ \ \ \ \ \ \ \ \ \ \isacommand{using}\isamarkupfalse%
\ control{\isadigit{2}}{\isacharunderscore}{\kern0pt}def\ index{\isacharunderscore}{\kern0pt}mat{\isacharunderscore}{\kern0pt}of{\isacharunderscore}{\kern0pt}cols{\isacharunderscore}{\kern0pt}list\ \isacommand{by}\isamarkupfalse%
\ simp\isanewline
\ \ \ \ \ \ \ \ \ \ \ \ \ \ \isacommand{also}\isamarkupfalse%
\ \isacommand{have}\isamarkupfalse%
\ {\isachardoublequoteopen}{\isasymdots}\ {\isacharequal}{\kern0pt}\ {\isacharparenleft}{\kern0pt}{\isacharparenleft}{\kern0pt}U\isactrlsup {\isasymdagger}{\isacharparenright}{\kern0pt}\ {\isacharasterisk}{\kern0pt}\ U{\isacharparenright}{\kern0pt}\ {\isachardollar}{\kern0pt}{\isachardollar}{\kern0pt}\ {\isacharparenleft}{\kern0pt}{\isadigit{0}}{\isacharcomma}{\kern0pt}{\isadigit{0}}{\isacharparenright}{\kern0pt}{\isachardoublequoteclose}\isanewline
\ \ \ \ \ \ \ \ \ \ \ \ \ \ \ \ \isacommand{using}\isamarkupfalse%
\ times{\isacharunderscore}{\kern0pt}mat{\isacharunderscore}{\kern0pt}def\ sumof{\isadigit{2}}\ assms{\isacharparenleft}{\kern0pt}{\isadigit{1}}{\isacharparenright}{\kern0pt}\ gate{\isacharunderscore}{\kern0pt}carrier{\isacharunderscore}{\kern0pt}mat\isanewline
\ \ \ \ \ \ \ \ \ \ \ \ \ \ \ \ \isacommand{by}\isamarkupfalse%
\ {\isacharparenleft}{\kern0pt}smt\ {\isacharparenleft}{\kern0pt}verit{\isacharcomma}{\kern0pt}\ del{\isacharunderscore}{\kern0pt}insts{\isacharparenright}{\kern0pt}\ Suc{\isacharunderscore}{\kern0pt}{\isadigit{1}}\ carrier{\isacharunderscore}{\kern0pt}matD{\isacharparenleft}{\kern0pt}{\isadigit{2}}{\isacharparenright}{\kern0pt}\ dagger{\isacharunderscore}{\kern0pt}def\ dim{\isacharunderscore}{\kern0pt}col{\isacharunderscore}{\kern0pt}mat{\isacharparenleft}{\kern0pt}{\isadigit{1}}{\isacharparenright}{\kern0pt}\ \isanewline
\ \ \ \ \ \ \ \ \ \ \ \ \ \ \ \ \ \ \ \ dim{\isacharunderscore}{\kern0pt}row{\isacharunderscore}{\kern0pt}of{\isacharunderscore}{\kern0pt}dagger\ gate{\isachardot}{\kern0pt}dim{\isacharunderscore}{\kern0pt}row\ index{\isacharunderscore}{\kern0pt}mat{\isacharparenleft}{\kern0pt}{\isadigit{1}}{\isacharparenright}{\kern0pt}\ index{\isacharunderscore}{\kern0pt}matrix{\isacharunderscore}{\kern0pt}prod\ lessI\ \isanewline
\ \ \ \ \ \ \ \ \ \ \ \ \ \ \ \ \ \ \ \ old{\isachardot}{\kern0pt}prod{\isachardot}{\kern0pt}case\ pos{\isadigit{2}}\ power{\isacharunderscore}{\kern0pt}one{\isacharunderscore}{\kern0pt}right{\isacharparenright}{\kern0pt}\isanewline
\ \ \ \ \ \ \ \ \ \ \ \ \ \ \isacommand{also}\isamarkupfalse%
\ \isacommand{have}\isamarkupfalse%
\ {\isachardoublequoteopen}{\isasymdots}\ {\isacharequal}{\kern0pt}\ {\isacharparenleft}{\kern0pt}{\isadigit{1}}\isactrlsub m\ {\isadigit{2}}{\isacharparenright}{\kern0pt}\ {\isachardollar}{\kern0pt}{\isachardollar}{\kern0pt}\ {\isacharparenleft}{\kern0pt}{\isadigit{0}}{\isacharcomma}{\kern0pt}{\isadigit{0}}{\isacharparenright}{\kern0pt}{\isachardoublequoteclose}\ \isacommand{using}\isamarkupfalse%
\ assms{\isacharparenleft}{\kern0pt}{\isadigit{1}}{\isacharparenright}{\kern0pt}\ gate{\isacharunderscore}{\kern0pt}def\ unitary{\isacharunderscore}{\kern0pt}def\ \isacommand{by}\isamarkupfalse%
\ auto\isanewline
\ \ \ \ \ \ \ \ \ \ \ \ \ \ \isacommand{also}\isamarkupfalse%
\ \isacommand{have}\isamarkupfalse%
\ {\isachardoublequoteopen}{\isasymdots}\ {\isacharequal}{\kern0pt}\ {\isadigit{1}}{\isachardoublequoteclose}\ \isacommand{using}\isamarkupfalse%
\ control{\isadigit{2}}{\isacharunderscore}{\kern0pt}def\ index{\isacharunderscore}{\kern0pt}mat{\isacharunderscore}{\kern0pt}of{\isacharunderscore}{\kern0pt}cols{\isacharunderscore}{\kern0pt}list\ \isacommand{by}\isamarkupfalse%
\ auto\isanewline
\ \ \ \ \ \ \ \ \ \ \ \ \ \ \isacommand{also}\isamarkupfalse%
\ \isacommand{have}\isamarkupfalse%
\ {\isachardoublequoteopen}{\isasymdots}\ {\isacharequal}{\kern0pt}\ {\isadigit{1}}\isactrlsub m\ {\isadigit{4}}\ {\isachardollar}{\kern0pt}{\isachardollar}{\kern0pt}\ {\isacharparenleft}{\kern0pt}{\isadigit{1}}{\isacharcomma}{\kern0pt}{\isadigit{1}}{\isacharparenright}{\kern0pt}{\isachardoublequoteclose}\ \isacommand{by}\isamarkupfalse%
\ simp\isanewline
\ \ \ \ \ \ \ \ \ \ \ \ \ \ \isacommand{finally}\isamarkupfalse%
\ \isacommand{show}\isamarkupfalse%
\ {\isacharquery}{\kern0pt}thesis\ \isacommand{using}\isamarkupfalse%
\ i{\isadigit{1}}\ j{\isadigit{1}}\ \isacommand{by}\isamarkupfalse%
\ simp\isanewline
\ \ \ \ \ \ \ \ \ \ \ \ \isacommand{qed}\isamarkupfalse%
\isanewline
\ \ \ \ \ \ \ \ \ \ \isacommand{next}\isamarkupfalse%
\isanewline
\ \ \ \ \ \ \ \ \ \ \ \ \isacommand{assume}\isamarkupfalse%
\ jl{\isadigit{2}}{\isacharcolon}{\kern0pt}{\isachardoublequoteopen}j\ {\isacharequal}{\kern0pt}\ {\isadigit{2}}\ {\isasymor}\ j\ {\isacharequal}{\kern0pt}\ {\isadigit{3}}{\isachardoublequoteclose}\isanewline
\ \ \ \ \ \ \ \ \ \ \ \ \isacommand{show}\isamarkupfalse%
\ {\isachardoublequoteopen}{\isacharparenleft}{\kern0pt}{\isacharparenleft}{\kern0pt}control{\isadigit{2}}\ U{\isacharparenright}{\kern0pt}\isactrlsup {\isasymdagger}\ {\isacharasterisk}{\kern0pt}\ control{\isadigit{2}}\ U{\isacharparenright}{\kern0pt}\ {\isachardollar}{\kern0pt}{\isachardollar}{\kern0pt}\ {\isacharparenleft}{\kern0pt}i{\isacharcomma}{\kern0pt}\ j{\isacharparenright}{\kern0pt}\ {\isacharequal}{\kern0pt}\ {\isadigit{1}}\isactrlsub m\ {\isadigit{4}}\ {\isachardollar}{\kern0pt}{\isachardollar}{\kern0pt}\ {\isacharparenleft}{\kern0pt}i{\isacharcomma}{\kern0pt}\ j{\isacharparenright}{\kern0pt}{\isachardoublequoteclose}\isanewline
\ \ \ \ \ \ \ \ \ \ \ \ \isacommand{proof}\isamarkupfalse%
\ {\isacharparenleft}{\kern0pt}rule\ disjE{\isacharparenright}{\kern0pt}\isanewline
\ \ \ \ \ \ \ \ \ \ \ \ \ \ \isacommand{show}\isamarkupfalse%
\ {\isachardoublequoteopen}j\ {\isacharequal}{\kern0pt}\ {\isadigit{2}}\ {\isasymor}\ j\ {\isacharequal}{\kern0pt}\ {\isadigit{3}}{\isachardoublequoteclose}\ \isacommand{using}\isamarkupfalse%
\ jl{\isadigit{2}}\ \isacommand{by}\isamarkupfalse%
\ this\isanewline
\ \ \ \ \ \ \ \ \ \ \ \ \isacommand{next}\isamarkupfalse%
\isanewline
\ \ \ \ \ \ \ \ \ \ \ \ \ \ \isacommand{assume}\isamarkupfalse%
\ j{\isadigit{2}}{\isacharcolon}{\kern0pt}{\isachardoublequoteopen}j\ {\isacharequal}{\kern0pt}\ {\isadigit{2}}{\isachardoublequoteclose}\isanewline
\ \ \ \ \ \ \ \ \ \ \ \ \ \ \isacommand{show}\isamarkupfalse%
\ {\isachardoublequoteopen}{\isacharparenleft}{\kern0pt}{\isacharparenleft}{\kern0pt}control{\isadigit{2}}\ U{\isacharparenright}{\kern0pt}\isactrlsup {\isasymdagger}\ {\isacharasterisk}{\kern0pt}\ control{\isadigit{2}}\ U{\isacharparenright}{\kern0pt}\ {\isachardollar}{\kern0pt}{\isachardollar}{\kern0pt}\ {\isacharparenleft}{\kern0pt}i{\isacharcomma}{\kern0pt}\ j{\isacharparenright}{\kern0pt}\ {\isacharequal}{\kern0pt}\ {\isadigit{1}}\isactrlsub m\ {\isadigit{4}}\ {\isachardollar}{\kern0pt}{\isachardollar}{\kern0pt}\ {\isacharparenleft}{\kern0pt}i{\isacharcomma}{\kern0pt}\ j{\isacharparenright}{\kern0pt}{\isachardoublequoteclose}\isanewline
\ \ \ \ \ \ \ \ \ \ \ \ \ \ \isacommand{proof}\isamarkupfalse%
\ {\isacharminus}{\kern0pt}\isanewline
\ \ \ \ \ \ \ \ \ \ \ \ \ \ \ \ \isacommand{have}\isamarkupfalse%
\ {\isachardoublequoteopen}{\isacharparenleft}{\kern0pt}{\isacharparenleft}{\kern0pt}control{\isadigit{2}}\ U{\isacharparenright}{\kern0pt}\isactrlsup {\isasymdagger}\ {\isacharasterisk}{\kern0pt}\ control{\isadigit{2}}\ U{\isacharparenright}{\kern0pt}\ {\isachardollar}{\kern0pt}{\isachardollar}{\kern0pt}\ {\isacharparenleft}{\kern0pt}{\isadigit{1}}{\isacharcomma}{\kern0pt}{\isadigit{2}}{\isacharparenright}{\kern0pt}\ {\isacharequal}{\kern0pt}\isanewline
\ \ \ \ \ \ \ \ \ \ \ \ \ \ \ \ {\isacharparenleft}{\kern0pt}{\isacharparenleft}{\kern0pt}control{\isadigit{2}}\ U{\isacharparenright}{\kern0pt}\isactrlsup {\isasymdagger}{\isacharparenright}{\kern0pt}\ {\isachardollar}{\kern0pt}{\isachardollar}{\kern0pt}\ {\isacharparenleft}{\kern0pt}{\isadigit{1}}{\isacharcomma}{\kern0pt}{\isadigit{0}}{\isacharparenright}{\kern0pt}\ {\isacharasterisk}{\kern0pt}\ {\isacharparenleft}{\kern0pt}control{\isadigit{2}}\ U{\isacharparenright}{\kern0pt}\ {\isachardollar}{\kern0pt}{\isachardollar}{\kern0pt}\ {\isacharparenleft}{\kern0pt}{\isadigit{0}}{\isacharcomma}{\kern0pt}{\isadigit{2}}{\isacharparenright}{\kern0pt}\ {\isacharplus}{\kern0pt}\isanewline
\ \ \ \ \ \ \ \ \ \ \ \ \ \ \ \ {\isacharparenleft}{\kern0pt}{\isacharparenleft}{\kern0pt}control{\isadigit{2}}\ U{\isacharparenright}{\kern0pt}\isactrlsup {\isasymdagger}{\isacharparenright}{\kern0pt}\ {\isachardollar}{\kern0pt}{\isachardollar}{\kern0pt}\ {\isacharparenleft}{\kern0pt}{\isadigit{1}}{\isacharcomma}{\kern0pt}{\isadigit{1}}{\isacharparenright}{\kern0pt}\ {\isacharasterisk}{\kern0pt}\ {\isacharparenleft}{\kern0pt}control{\isadigit{2}}\ U{\isacharparenright}{\kern0pt}\ {\isachardollar}{\kern0pt}{\isachardollar}{\kern0pt}\ {\isacharparenleft}{\kern0pt}{\isadigit{1}}{\isacharcomma}{\kern0pt}{\isadigit{2}}{\isacharparenright}{\kern0pt}\ {\isacharplus}{\kern0pt}\isanewline
\ \ \ \ \ \ \ \ \ \ \ \ \ \ \ \ {\isacharparenleft}{\kern0pt}{\isacharparenleft}{\kern0pt}control{\isadigit{2}}\ U{\isacharparenright}{\kern0pt}\isactrlsup {\isasymdagger}{\isacharparenright}{\kern0pt}\ {\isachardollar}{\kern0pt}{\isachardollar}{\kern0pt}\ {\isacharparenleft}{\kern0pt}{\isadigit{1}}{\isacharcomma}{\kern0pt}{\isadigit{2}}{\isacharparenright}{\kern0pt}\ {\isacharasterisk}{\kern0pt}\ {\isacharparenleft}{\kern0pt}control{\isadigit{2}}\ U{\isacharparenright}{\kern0pt}\ {\isachardollar}{\kern0pt}{\isachardollar}{\kern0pt}\ {\isacharparenleft}{\kern0pt}{\isadigit{2}}{\isacharcomma}{\kern0pt}{\isadigit{2}}{\isacharparenright}{\kern0pt}\ {\isacharplus}{\kern0pt}\isanewline
\ \ \ \ \ \ \ \ \ \ \ \ \ \ \ \ {\isacharparenleft}{\kern0pt}{\isacharparenleft}{\kern0pt}control{\isadigit{2}}\ U{\isacharparenright}{\kern0pt}\isactrlsup {\isasymdagger}{\isacharparenright}{\kern0pt}\ {\isachardollar}{\kern0pt}{\isachardollar}{\kern0pt}\ {\isacharparenleft}{\kern0pt}{\isadigit{1}}{\isacharcomma}{\kern0pt}{\isadigit{3}}{\isacharparenright}{\kern0pt}\ {\isacharasterisk}{\kern0pt}\ {\isacharparenleft}{\kern0pt}control{\isadigit{2}}\ U{\isacharparenright}{\kern0pt}\ {\isachardollar}{\kern0pt}{\isachardollar}{\kern0pt}\ {\isacharparenleft}{\kern0pt}{\isadigit{3}}{\isacharcomma}{\kern0pt}{\isadigit{2}}{\isacharparenright}{\kern0pt}{\isachardoublequoteclose}\isanewline
\ \ \ \ \ \ \ \ \ \ \ \ \ \ \ \ \ \ \isacommand{using}\isamarkupfalse%
\ sumof{\isadigit{4}}\isanewline
\ \ \ \ \ \ \ \ \ \ \ \ \ \ \ \ \ \ \isacommand{by}\isamarkupfalse%
\ {\isacharparenleft}{\kern0pt}smt\ {\isacharparenleft}{\kern0pt}z{\isadigit{3}}{\isacharparenright}{\kern0pt}\ carrier{\isacharunderscore}{\kern0pt}matD{\isacharparenleft}{\kern0pt}{\isadigit{1}}{\isacharparenright}{\kern0pt}\ carrier{\isacharunderscore}{\kern0pt}matD{\isacharparenleft}{\kern0pt}{\isadigit{2}}{\isacharparenright}{\kern0pt}\ control{\isadigit{2}}{\isacharunderscore}{\kern0pt}carrier{\isacharunderscore}{\kern0pt}mat\ \isanewline
\ \ \ \ \ \ \ \ \ \ \ \ \ \ \ \ \ \ \ \ \ \ dim{\isacharunderscore}{\kern0pt}col{\isacharunderscore}{\kern0pt}of{\isacharunderscore}{\kern0pt}dagger\ dim{\isacharunderscore}{\kern0pt}row{\isacharunderscore}{\kern0pt}of{\isacharunderscore}{\kern0pt}dagger\ index{\isacharunderscore}{\kern0pt}matrix{\isacharunderscore}{\kern0pt}prod\ j{\isadigit{2}}\ j{\isadigit{4}}\ \isanewline
\ \ \ \ \ \ \ \ \ \ \ \ \ \ \ \ \ \ \ \ \ \ one{\isacharunderscore}{\kern0pt}less{\isacharunderscore}{\kern0pt}numeral{\isacharunderscore}{\kern0pt}iff\ semiring{\isacharunderscore}{\kern0pt}norm{\isacharparenleft}{\kern0pt}{\isadigit{7}}{\isadigit{6}}{\isacharparenright}{\kern0pt}{\isacharparenright}{\kern0pt}\isanewline
\ \ \ \ \ \ \ \ \ \ \ \ \ \ \ \ \isacommand{also}\isamarkupfalse%
\ \isacommand{have}\isamarkupfalse%
\ {\isachardoublequoteopen}{\isasymdots}\ {\isacharequal}{\kern0pt}\ {\isacharparenleft}{\kern0pt}{\isacharparenleft}{\kern0pt}control{\isadigit{2}}\ U{\isacharparenright}{\kern0pt}\isactrlsup {\isasymdagger}{\isacharparenright}{\kern0pt}\ {\isachardollar}{\kern0pt}{\isachardollar}{\kern0pt}\ {\isacharparenleft}{\kern0pt}{\isadigit{1}}{\isacharcomma}{\kern0pt}{\isadigit{2}}{\isacharparenright}{\kern0pt}{\isachardoublequoteclose}\isanewline
\ \ \ \ \ \ \ \ \ \ \ \ \ \ \ \ \ \ \isacommand{using}\isamarkupfalse%
\ control{\isadigit{2}}{\isacharunderscore}{\kern0pt}def\ index{\isacharunderscore}{\kern0pt}mat{\isacharunderscore}{\kern0pt}of{\isacharunderscore}{\kern0pt}cols{\isacharunderscore}{\kern0pt}list\ \isacommand{by}\isamarkupfalse%
\ force\isanewline
\ \ \ \ \ \ \ \ \ \ \ \ \ \ \ \ \isacommand{also}\isamarkupfalse%
\ \isacommand{have}\isamarkupfalse%
\ {\isachardoublequoteopen}{\isasymdots}\ {\isacharequal}{\kern0pt}\ cnj\ {\isacharparenleft}{\kern0pt}{\isacharparenleft}{\kern0pt}control{\isadigit{2}}\ U{\isacharparenright}{\kern0pt}\ {\isachardollar}{\kern0pt}{\isachardollar}{\kern0pt}\ {\isacharparenleft}{\kern0pt}{\isadigit{2}}{\isacharcomma}{\kern0pt}{\isadigit{1}}{\isacharparenright}{\kern0pt}{\isacharparenright}{\kern0pt}{\isachardoublequoteclose}\isanewline
\ \ \ \ \ \ \ \ \ \ \ \ \ \ \ \ \ \ \isacommand{using}\isamarkupfalse%
\ dagger{\isacharunderscore}{\kern0pt}def\isanewline
\ \ \ \ \ \ \ \ \ \ \ \ \ \ \ \ \ \ \isacommand{by}\isamarkupfalse%
\ {\isacharparenleft}{\kern0pt}simp\ add{\isacharcolon}{\kern0pt}\ Tensor{\isachardot}{\kern0pt}mat{\isacharunderscore}{\kern0pt}of{\isacharunderscore}{\kern0pt}cols{\isacharunderscore}{\kern0pt}list{\isacharunderscore}{\kern0pt}def\ control{\isadigit{2}}{\isacharunderscore}{\kern0pt}def{\isacharparenright}{\kern0pt}\isanewline
\ \ \ \ \ \ \ \ \ \ \ \ \ \ \ \ \isacommand{also}\isamarkupfalse%
\ \isacommand{have}\isamarkupfalse%
\ {\isachardoublequoteopen}{\isasymdots}\ {\isacharequal}{\kern0pt}\ {\isadigit{0}}{\isachardoublequoteclose}\ \isacommand{using}\isamarkupfalse%
\ control{\isadigit{2}}{\isacharunderscore}{\kern0pt}def\ index{\isacharunderscore}{\kern0pt}mat{\isacharunderscore}{\kern0pt}of{\isacharunderscore}{\kern0pt}cols{\isacharunderscore}{\kern0pt}list\ \isacommand{by}\isamarkupfalse%
\ auto\isanewline
\ \ \ \ \ \ \ \ \ \ \ \ \ \ \ \ \isacommand{also}\isamarkupfalse%
\ \isacommand{have}\isamarkupfalse%
\ {\isachardoublequoteopen}{\isasymdots}\ {\isacharequal}{\kern0pt}\ {\isadigit{1}}\isactrlsub m\ {\isadigit{4}}\ {\isachardollar}{\kern0pt}{\isachardollar}{\kern0pt}\ {\isacharparenleft}{\kern0pt}{\isadigit{1}}{\isacharcomma}{\kern0pt}{\isadigit{2}}{\isacharparenright}{\kern0pt}{\isachardoublequoteclose}\ \isacommand{by}\isamarkupfalse%
\ simp\isanewline
\ \ \ \ \ \ \ \ \ \ \ \ \ \ \ \ \isacommand{finally}\isamarkupfalse%
\ \isacommand{show}\isamarkupfalse%
\ {\isacharquery}{\kern0pt}thesis\ \isacommand{using}\isamarkupfalse%
\ i{\isadigit{1}}\ j{\isadigit{2}}\ \isacommand{by}\isamarkupfalse%
\ simp\isanewline
\ \ \ \ \ \ \ \ \ \ \ \ \ \ \isacommand{qed}\isamarkupfalse%
\isanewline
\ \ \ \ \ \ \ \ \ \ \ \ \isacommand{next}\isamarkupfalse%
\isanewline
\ \ \ \ \ \ \ \ \ \ \ \ \ \ \isacommand{assume}\isamarkupfalse%
\ j{\isadigit{3}}{\isacharcolon}{\kern0pt}{\isachardoublequoteopen}j\ {\isacharequal}{\kern0pt}\ {\isadigit{3}}{\isachardoublequoteclose}\isanewline
\ \ \ \ \ \ \ \ \ \ \ \ \ \ \isacommand{show}\isamarkupfalse%
\ {\isachardoublequoteopen}{\isacharparenleft}{\kern0pt}{\isacharparenleft}{\kern0pt}control{\isadigit{2}}\ U{\isacharparenright}{\kern0pt}\isactrlsup {\isasymdagger}\ {\isacharasterisk}{\kern0pt}\ control{\isadigit{2}}\ U{\isacharparenright}{\kern0pt}\ {\isachardollar}{\kern0pt}{\isachardollar}{\kern0pt}\ {\isacharparenleft}{\kern0pt}i{\isacharcomma}{\kern0pt}\ j{\isacharparenright}{\kern0pt}\ {\isacharequal}{\kern0pt}\ {\isadigit{1}}\isactrlsub m\ {\isadigit{4}}\ {\isachardollar}{\kern0pt}{\isachardollar}{\kern0pt}\ {\isacharparenleft}{\kern0pt}i{\isacharcomma}{\kern0pt}\ j{\isacharparenright}{\kern0pt}{\isachardoublequoteclose}\isanewline
\ \ \ \ \ \ \ \ \ \ \ \ \ \ \isacommand{proof}\isamarkupfalse%
\ {\isacharminus}{\kern0pt}\isanewline
\ \ \ \ \ \ \ \ \ \ \ \ \ \ \ \ \isacommand{have}\isamarkupfalse%
\ {\isachardoublequoteopen}{\isacharparenleft}{\kern0pt}{\isacharparenleft}{\kern0pt}control{\isadigit{2}}\ U{\isacharparenright}{\kern0pt}\isactrlsup {\isasymdagger}\ {\isacharasterisk}{\kern0pt}\ control{\isadigit{2}}\ U{\isacharparenright}{\kern0pt}\ {\isachardollar}{\kern0pt}{\isachardollar}{\kern0pt}\ {\isacharparenleft}{\kern0pt}{\isadigit{1}}{\isacharcomma}{\kern0pt}{\isadigit{3}}{\isacharparenright}{\kern0pt}\ {\isacharequal}{\kern0pt}\isanewline
\ \ \ \ \ \ \ \ \ \ \ \ \ \ \ \ {\isacharparenleft}{\kern0pt}{\isacharparenleft}{\kern0pt}control{\isadigit{2}}\ U{\isacharparenright}{\kern0pt}\isactrlsup {\isasymdagger}{\isacharparenright}{\kern0pt}\ {\isachardollar}{\kern0pt}{\isachardollar}{\kern0pt}\ {\isacharparenleft}{\kern0pt}{\isadigit{1}}{\isacharcomma}{\kern0pt}{\isadigit{0}}{\isacharparenright}{\kern0pt}\ {\isacharasterisk}{\kern0pt}\ {\isacharparenleft}{\kern0pt}control{\isadigit{2}}\ U{\isacharparenright}{\kern0pt}\ {\isachardollar}{\kern0pt}{\isachardollar}{\kern0pt}\ {\isacharparenleft}{\kern0pt}{\isadigit{0}}{\isacharcomma}{\kern0pt}{\isadigit{3}}{\isacharparenright}{\kern0pt}\ {\isacharplus}{\kern0pt}\isanewline
\ \ \ \ \ \ \ \ \ \ \ \ \ \ \ \ {\isacharparenleft}{\kern0pt}{\isacharparenleft}{\kern0pt}control{\isadigit{2}}\ U{\isacharparenright}{\kern0pt}\isactrlsup {\isasymdagger}{\isacharparenright}{\kern0pt}\ {\isachardollar}{\kern0pt}{\isachardollar}{\kern0pt}\ {\isacharparenleft}{\kern0pt}{\isadigit{1}}{\isacharcomma}{\kern0pt}{\isadigit{1}}{\isacharparenright}{\kern0pt}\ {\isacharasterisk}{\kern0pt}\ {\isacharparenleft}{\kern0pt}control{\isadigit{2}}\ U{\isacharparenright}{\kern0pt}\ {\isachardollar}{\kern0pt}{\isachardollar}{\kern0pt}\ {\isacharparenleft}{\kern0pt}{\isadigit{1}}{\isacharcomma}{\kern0pt}{\isadigit{3}}{\isacharparenright}{\kern0pt}\ {\isacharplus}{\kern0pt}\isanewline
\ \ \ \ \ \ \ \ \ \ \ \ \ \ \ \ {\isacharparenleft}{\kern0pt}{\isacharparenleft}{\kern0pt}control{\isadigit{2}}\ U{\isacharparenright}{\kern0pt}\isactrlsup {\isasymdagger}{\isacharparenright}{\kern0pt}\ {\isachardollar}{\kern0pt}{\isachardollar}{\kern0pt}\ {\isacharparenleft}{\kern0pt}{\isadigit{1}}{\isacharcomma}{\kern0pt}{\isadigit{2}}{\isacharparenright}{\kern0pt}\ {\isacharasterisk}{\kern0pt}\ {\isacharparenleft}{\kern0pt}control{\isadigit{2}}\ U{\isacharparenright}{\kern0pt}\ {\isachardollar}{\kern0pt}{\isachardollar}{\kern0pt}\ {\isacharparenleft}{\kern0pt}{\isadigit{2}}{\isacharcomma}{\kern0pt}{\isadigit{3}}{\isacharparenright}{\kern0pt}\ {\isacharplus}{\kern0pt}\isanewline
\ \ \ \ \ \ \ \ \ \ \ \ \ \ \ \ {\isacharparenleft}{\kern0pt}{\isacharparenleft}{\kern0pt}control{\isadigit{2}}\ U{\isacharparenright}{\kern0pt}\isactrlsup {\isasymdagger}{\isacharparenright}{\kern0pt}\ {\isachardollar}{\kern0pt}{\isachardollar}{\kern0pt}\ {\isacharparenleft}{\kern0pt}{\isadigit{1}}{\isacharcomma}{\kern0pt}{\isadigit{3}}{\isacharparenright}{\kern0pt}\ {\isacharasterisk}{\kern0pt}\ {\isacharparenleft}{\kern0pt}control{\isadigit{2}}\ U{\isacharparenright}{\kern0pt}\ {\isachardollar}{\kern0pt}{\isachardollar}{\kern0pt}\ {\isacharparenleft}{\kern0pt}{\isadigit{3}}{\isacharcomma}{\kern0pt}{\isadigit{3}}{\isacharparenright}{\kern0pt}{\isachardoublequoteclose}\isanewline
\ \ \ \ \ \ \ \ \ \ \ \ \ \ \ \ \ \ \isacommand{using}\isamarkupfalse%
\ sumof{\isadigit{4}}\isanewline
\ \ \ \ \ \ \ \ \ \ \ \ \ \ \ \ \ \ \isacommand{by}\isamarkupfalse%
\ {\isacharparenleft}{\kern0pt}metis\ {\isacharparenleft}{\kern0pt}no{\isacharunderscore}{\kern0pt}types{\isacharcomma}{\kern0pt}\ lifting{\isacharparenright}{\kern0pt}\ carrier{\isacharunderscore}{\kern0pt}matD{\isacharparenleft}{\kern0pt}{\isadigit{1}}{\isacharparenright}{\kern0pt}\ carrier{\isacharunderscore}{\kern0pt}matD{\isacharparenleft}{\kern0pt}{\isadigit{2}}{\isacharparenright}{\kern0pt}\ \isanewline
\ \ \ \ \ \ \ \ \ \ \ \ \ \ \ \ \ \ \ \ \ \ control{\isadigit{2}}{\isacharunderscore}{\kern0pt}carrier{\isacharunderscore}{\kern0pt}mat\ dim{\isacharunderscore}{\kern0pt}col{\isacharunderscore}{\kern0pt}of{\isacharunderscore}{\kern0pt}dagger\ dim{\isacharunderscore}{\kern0pt}row{\isacharunderscore}{\kern0pt}of{\isacharunderscore}{\kern0pt}dagger\ i{\isadigit{1}}\ i{\isadigit{4}}\ \isanewline
\ \ \ \ \ \ \ \ \ \ \ \ \ \ \ \ \ \ \ \ \ \ index{\isacharunderscore}{\kern0pt}matrix{\isacharunderscore}{\kern0pt}prod\ j{\isadigit{3}}\ j{\isadigit{4}}{\isacharparenright}{\kern0pt}\isanewline
\ \ \ \ \ \ \ \ \ \ \ \ \ \ \ \ \isacommand{also}\isamarkupfalse%
\ \isacommand{have}\isamarkupfalse%
\ {\isachardoublequoteopen}{\isasymdots}\ {\isacharequal}{\kern0pt}\ {\isacharparenleft}{\kern0pt}{\isacharparenleft}{\kern0pt}control{\isadigit{2}}\ U{\isacharparenright}{\kern0pt}\isactrlsup {\isasymdagger}{\isacharparenright}{\kern0pt}\ {\isachardollar}{\kern0pt}{\isachardollar}{\kern0pt}\ {\isacharparenleft}{\kern0pt}{\isadigit{1}}{\isacharcomma}{\kern0pt}{\isadigit{1}}{\isacharparenright}{\kern0pt}\ {\isacharasterisk}{\kern0pt}\ {\isacharparenleft}{\kern0pt}control{\isadigit{2}}\ U{\isacharparenright}{\kern0pt}\ {\isachardollar}{\kern0pt}{\isachardollar}{\kern0pt}\ {\isacharparenleft}{\kern0pt}{\isadigit{1}}{\isacharcomma}{\kern0pt}{\isadigit{3}}{\isacharparenright}{\kern0pt}\ {\isacharplus}{\kern0pt}\isanewline
\ \ \ \ \ \ \ \ \ \ \ \ \ \ \ \ \ \ \ \ \ \ \ \ \ \ \ \ \ \ \ \ {\isacharparenleft}{\kern0pt}{\isacharparenleft}{\kern0pt}control{\isadigit{2}}\ U{\isacharparenright}{\kern0pt}\isactrlsup {\isasymdagger}{\isacharparenright}{\kern0pt}\ {\isachardollar}{\kern0pt}{\isachardollar}{\kern0pt}\ {\isacharparenleft}{\kern0pt}{\isadigit{1}}{\isacharcomma}{\kern0pt}{\isadigit{3}}{\isacharparenright}{\kern0pt}\ {\isacharasterisk}{\kern0pt}\ {\isacharparenleft}{\kern0pt}control{\isadigit{2}}\ U{\isacharparenright}{\kern0pt}\ {\isachardollar}{\kern0pt}{\isachardollar}{\kern0pt}\ {\isacharparenleft}{\kern0pt}{\isadigit{3}}{\isacharcomma}{\kern0pt}{\isadigit{3}}{\isacharparenright}{\kern0pt}{\isachardoublequoteclose}\isanewline
\ \ \ \ \ \ \ \ \ \ \ \ \ \ \ \ \ \ \isacommand{using}\isamarkupfalse%
\ control{\isadigit{2}}{\isacharunderscore}{\kern0pt}def\ index{\isacharunderscore}{\kern0pt}mat{\isacharunderscore}{\kern0pt}of{\isacharunderscore}{\kern0pt}cols{\isacharunderscore}{\kern0pt}list\ \isacommand{by}\isamarkupfalse%
\ force\isanewline
\ \ \ \ \ \ \ \ \ \ \ \ \ \ \ \ \isacommand{also}\isamarkupfalse%
\ \isacommand{have}\isamarkupfalse%
\ {\isachardoublequoteopen}{\isasymdots}\ {\isacharequal}{\kern0pt}\ cnj\ {\isacharparenleft}{\kern0pt}{\isacharparenleft}{\kern0pt}control{\isadigit{2}}\ U{\isacharparenright}{\kern0pt}\ {\isachardollar}{\kern0pt}{\isachardollar}{\kern0pt}\ {\isacharparenleft}{\kern0pt}{\isadigit{1}}{\isacharcomma}{\kern0pt}{\isadigit{1}}{\isacharparenright}{\kern0pt}{\isacharparenright}{\kern0pt}\ {\isacharasterisk}{\kern0pt}\ {\isacharparenleft}{\kern0pt}control{\isadigit{2}}\ U{\isacharparenright}{\kern0pt}\ {\isachardollar}{\kern0pt}{\isachardollar}{\kern0pt}\ {\isacharparenleft}{\kern0pt}{\isadigit{1}}{\isacharcomma}{\kern0pt}{\isadigit{3}}{\isacharparenright}{\kern0pt}\ {\isacharplus}{\kern0pt}\isanewline
\ \ \ \ \ \ \ \ \ \ \ \ \ \ \ \ \ \ \ \ \ \ \ \ \ \ \ \ \ \ \ \ cnj\ {\isacharparenleft}{\kern0pt}{\isacharparenleft}{\kern0pt}control{\isadigit{2}}\ U{\isacharparenright}{\kern0pt}\ {\isachardollar}{\kern0pt}{\isachardollar}{\kern0pt}\ {\isacharparenleft}{\kern0pt}{\isadigit{3}}{\isacharcomma}{\kern0pt}{\isadigit{1}}{\isacharparenright}{\kern0pt}{\isacharparenright}{\kern0pt}\ {\isacharasterisk}{\kern0pt}\ {\isacharparenleft}{\kern0pt}control{\isadigit{2}}\ U{\isacharparenright}{\kern0pt}\ {\isachardollar}{\kern0pt}{\isachardollar}{\kern0pt}\ {\isacharparenleft}{\kern0pt}{\isadigit{3}}{\isacharcomma}{\kern0pt}{\isadigit{3}}{\isacharparenright}{\kern0pt}{\isachardoublequoteclose}\isanewline
\ \ \ \ \ \ \ \ \ \ \ \ \ \ \ \ \ \ \isacommand{using}\isamarkupfalse%
\ dagger{\isacharunderscore}{\kern0pt}def\isanewline
\ \ \ \ \ \ \ \ \ \ \ \ \ \ \ \ \ \ \isacommand{by}\isamarkupfalse%
\ {\isacharparenleft}{\kern0pt}simp\ add{\isacharcolon}{\kern0pt}\ Tensor{\isachardot}{\kern0pt}mat{\isacharunderscore}{\kern0pt}of{\isacharunderscore}{\kern0pt}cols{\isacharunderscore}{\kern0pt}list{\isacharunderscore}{\kern0pt}def\ control{\isadigit{2}}{\isacharunderscore}{\kern0pt}def{\isacharparenright}{\kern0pt}\isanewline
\ \ \ \ \ \ \ \ \ \ \ \ \ \ \ \ \isacommand{also}\isamarkupfalse%
\ \isacommand{have}\isamarkupfalse%
\ {\isachardoublequoteopen}{\isasymdots}\ {\isacharequal}{\kern0pt}\ cnj\ {\isacharparenleft}{\kern0pt}U\ {\isachardollar}{\kern0pt}{\isachardollar}{\kern0pt}\ {\isacharparenleft}{\kern0pt}{\isadigit{0}}{\isacharcomma}{\kern0pt}{\isadigit{0}}{\isacharparenright}{\kern0pt}{\isacharparenright}{\kern0pt}\ {\isacharasterisk}{\kern0pt}\ {\isacharparenleft}{\kern0pt}U\ {\isachardollar}{\kern0pt}{\isachardollar}{\kern0pt}\ {\isacharparenleft}{\kern0pt}{\isadigit{0}}{\isacharcomma}{\kern0pt}{\isadigit{1}}{\isacharparenright}{\kern0pt}{\isacharparenright}{\kern0pt}\ {\isacharplus}{\kern0pt}\isanewline
\ \ \ \ \ \ \ \ \ \ \ \ \ \ \ \ \ \ \ \ \ \ \ \ \ \ \ \ \ \ \ \ cnj\ {\isacharparenleft}{\kern0pt}U\ {\isachardollar}{\kern0pt}{\isachardollar}{\kern0pt}\ {\isacharparenleft}{\kern0pt}{\isadigit{1}}{\isacharcomma}{\kern0pt}{\isadigit{0}}{\isacharparenright}{\kern0pt}{\isacharparenright}{\kern0pt}\ {\isacharasterisk}{\kern0pt}\ {\isacharparenleft}{\kern0pt}U\ {\isachardollar}{\kern0pt}{\isachardollar}{\kern0pt}\ {\isacharparenleft}{\kern0pt}{\isadigit{1}}{\isacharcomma}{\kern0pt}{\isadigit{1}}{\isacharparenright}{\kern0pt}{\isacharparenright}{\kern0pt}{\isachardoublequoteclose}\isanewline
\ \ \ \ \ \ \ \ \ \ \ \ \ \ \ \ \ \ \isacommand{using}\isamarkupfalse%
\ control{\isadigit{2}}{\isacharunderscore}{\kern0pt}def\ index{\isacharunderscore}{\kern0pt}mat{\isacharunderscore}{\kern0pt}of{\isacharunderscore}{\kern0pt}cols{\isacharunderscore}{\kern0pt}list\ \isacommand{by}\isamarkupfalse%
\ simp\isanewline
\ \ \ \ \ \ \ \ \ \ \ \ \ \ \ \ \isacommand{also}\isamarkupfalse%
\ \isacommand{have}\isamarkupfalse%
\ {\isachardoublequoteopen}{\isasymdots}\ {\isacharequal}{\kern0pt}\ {\isacharparenleft}{\kern0pt}{\isacharparenleft}{\kern0pt}U\isactrlsup {\isasymdagger}{\isacharparenright}{\kern0pt}\ {\isacharasterisk}{\kern0pt}\ U{\isacharparenright}{\kern0pt}\ {\isachardollar}{\kern0pt}{\isachardollar}{\kern0pt}\ {\isacharparenleft}{\kern0pt}{\isadigit{0}}{\isacharcomma}{\kern0pt}{\isadigit{1}}{\isacharparenright}{\kern0pt}{\isachardoublequoteclose}\isanewline
\ \ \ \ \ \ \ \ \ \ \ \ \ \ \ \ \ \ \isacommand{using}\isamarkupfalse%
\ times{\isacharunderscore}{\kern0pt}mat{\isacharunderscore}{\kern0pt}def\ sumof{\isadigit{2}}\ assms{\isacharparenleft}{\kern0pt}{\isadigit{1}}{\isacharparenright}{\kern0pt}\ gate{\isacharunderscore}{\kern0pt}carrier{\isacharunderscore}{\kern0pt}mat\isanewline
\ \ \ \ \ \ \ \ \ \ \ \ \ \ \ \ \ \ \isacommand{by}\isamarkupfalse%
\ {\isacharparenleft}{\kern0pt}smt\ {\isacharparenleft}{\kern0pt}verit{\isacharcomma}{\kern0pt}\ del{\isacharunderscore}{\kern0pt}insts{\isacharparenright}{\kern0pt}\ Suc{\isacharunderscore}{\kern0pt}{\isadigit{1}}\ carrier{\isacharunderscore}{\kern0pt}matD{\isacharparenleft}{\kern0pt}{\isadigit{2}}{\isacharparenright}{\kern0pt}\ dagger{\isacharunderscore}{\kern0pt}def\ dim{\isacharunderscore}{\kern0pt}col{\isacharunderscore}{\kern0pt}mat{\isacharparenleft}{\kern0pt}{\isadigit{1}}{\isacharparenright}{\kern0pt}\ \isanewline
\ \ \ \ \ \ \ \ \ \ \ \ \ \ \ \ \ \ \ \ \ \ dim{\isacharunderscore}{\kern0pt}row{\isacharunderscore}{\kern0pt}of{\isacharunderscore}{\kern0pt}dagger\ gate{\isachardot}{\kern0pt}dim{\isacharunderscore}{\kern0pt}row\ index{\isacharunderscore}{\kern0pt}mat{\isacharparenleft}{\kern0pt}{\isadigit{1}}{\isacharparenright}{\kern0pt}\ index{\isacharunderscore}{\kern0pt}matrix{\isacharunderscore}{\kern0pt}prod\ lessI\ \isanewline
\ \ \ \ \ \ \ \ \ \ \ \ \ \ \ \ \ \ \ \ \ \ old{\isachardot}{\kern0pt}prod{\isachardot}{\kern0pt}case\ pos{\isadigit{2}}\ power{\isacharunderscore}{\kern0pt}one{\isacharunderscore}{\kern0pt}right{\isacharparenright}{\kern0pt}\isanewline
\ \ \ \ \ \ \ \ \ \ \ \ \ \ \ \ \isacommand{also}\isamarkupfalse%
\ \isacommand{have}\isamarkupfalse%
\ {\isachardoublequoteopen}{\isasymdots}\ {\isacharequal}{\kern0pt}\ {\isacharparenleft}{\kern0pt}{\isadigit{1}}\isactrlsub m\ {\isadigit{2}}{\isacharparenright}{\kern0pt}\ {\isachardollar}{\kern0pt}{\isachardollar}{\kern0pt}\ {\isacharparenleft}{\kern0pt}{\isadigit{0}}{\isacharcomma}{\kern0pt}{\isadigit{1}}{\isacharparenright}{\kern0pt}{\isachardoublequoteclose}\ \isacommand{using}\isamarkupfalse%
\ assms{\isacharparenleft}{\kern0pt}{\isadigit{1}}{\isacharparenright}{\kern0pt}\ gate{\isacharunderscore}{\kern0pt}def\ unitary{\isacharunderscore}{\kern0pt}def\ \isacommand{by}\isamarkupfalse%
\ auto\isanewline
\ \ \ \ \ \ \ \ \ \ \ \ \ \ \ \ \isacommand{also}\isamarkupfalse%
\ \isacommand{have}\isamarkupfalse%
\ {\isachardoublequoteopen}{\isasymdots}\ {\isacharequal}{\kern0pt}\ {\isadigit{0}}{\isachardoublequoteclose}\ \isacommand{using}\isamarkupfalse%
\ control{\isadigit{2}}{\isacharunderscore}{\kern0pt}def\ index{\isacharunderscore}{\kern0pt}mat{\isacharunderscore}{\kern0pt}of{\isacharunderscore}{\kern0pt}cols{\isacharunderscore}{\kern0pt}list\ \isacommand{by}\isamarkupfalse%
\ auto\isanewline
\ \ \ \ \ \ \ \ \ \ \ \ \ \ \ \ \isacommand{also}\isamarkupfalse%
\ \isacommand{have}\isamarkupfalse%
\ {\isachardoublequoteopen}{\isasymdots}\ {\isacharequal}{\kern0pt}\ {\isadigit{1}}\isactrlsub m\ {\isadigit{4}}\ {\isachardollar}{\kern0pt}{\isachardollar}{\kern0pt}\ {\isacharparenleft}{\kern0pt}{\isadigit{1}}{\isacharcomma}{\kern0pt}{\isadigit{3}}{\isacharparenright}{\kern0pt}{\isachardoublequoteclose}\ \isacommand{by}\isamarkupfalse%
\ simp\isanewline
\ \ \ \ \ \ \ \ \ \ \ \ \ \ \ \ \isacommand{finally}\isamarkupfalse%
\ \isacommand{show}\isamarkupfalse%
\ {\isacharquery}{\kern0pt}thesis\ \isacommand{using}\isamarkupfalse%
\ i{\isadigit{1}}\ j{\isadigit{3}}\ \isacommand{by}\isamarkupfalse%
\ simp\isanewline
\ \ \ \ \ \ \ \ \ \ \ \ \ \ \isacommand{qed}\isamarkupfalse%
\isanewline
\ \ \ \ \ \ \ \ \ \ \ \ \isacommand{qed}\isamarkupfalse%
\isanewline
\ \ \ \ \ \ \ \ \ \ \isacommand{qed}\isamarkupfalse%
\isanewline
\ \ \ \ \ \ \ \ \isacommand{qed}\isamarkupfalse%
\isanewline
\ \ \ \ \ \ \isacommand{next}\isamarkupfalse%
\isanewline
\ \ \ \ \ \ \ \ \isacommand{assume}\isamarkupfalse%
\ il{\isadigit{2}}{\isacharcolon}{\kern0pt}{\isachardoublequoteopen}i\ {\isacharequal}{\kern0pt}\ {\isadigit{2}}\ {\isasymor}\ i\ {\isacharequal}{\kern0pt}\ {\isadigit{3}}{\isachardoublequoteclose}\isanewline
\ \ \ \ \ \ \ \ \isacommand{show}\isamarkupfalse%
\ {\isachardoublequoteopen}{\isacharparenleft}{\kern0pt}{\isacharparenleft}{\kern0pt}control{\isadigit{2}}\ U{\isacharparenright}{\kern0pt}\isactrlsup {\isasymdagger}\ {\isacharasterisk}{\kern0pt}\ control{\isadigit{2}}\ U{\isacharparenright}{\kern0pt}\ {\isachardollar}{\kern0pt}{\isachardollar}{\kern0pt}\ {\isacharparenleft}{\kern0pt}i{\isacharcomma}{\kern0pt}\ j{\isacharparenright}{\kern0pt}\ {\isacharequal}{\kern0pt}\ {\isadigit{1}}\isactrlsub m\ {\isadigit{4}}\ {\isachardollar}{\kern0pt}{\isachardollar}{\kern0pt}\ {\isacharparenleft}{\kern0pt}i{\isacharcomma}{\kern0pt}\ j{\isacharparenright}{\kern0pt}{\isachardoublequoteclose}\isanewline
\ \ \ \ \ \ \ \ \isacommand{proof}\isamarkupfalse%
\ {\isacharparenleft}{\kern0pt}rule\ disjE{\isacharparenright}{\kern0pt}\isanewline
\ \ \ \ \ \ \ \ \ \ \isacommand{show}\isamarkupfalse%
\ {\isachardoublequoteopen}i\ {\isacharequal}{\kern0pt}\ {\isadigit{2}}\ {\isasymor}\ i\ {\isacharequal}{\kern0pt}\ {\isadigit{3}}{\isachardoublequoteclose}\ \isacommand{using}\isamarkupfalse%
\ il{\isadigit{2}}\ \isacommand{by}\isamarkupfalse%
\ this\isanewline
\ \ \ \ \ \ \ \ \isacommand{next}\isamarkupfalse%
\isanewline
\ \ \ \ \ \ \ \ \ \ \isacommand{assume}\isamarkupfalse%
\ i{\isadigit{2}}{\isacharcolon}{\kern0pt}{\isachardoublequoteopen}i\ {\isacharequal}{\kern0pt}\ {\isadigit{2}}{\isachardoublequoteclose}\isanewline
\ \ \ \ \ \ \ \ \ \ \isacommand{show}\isamarkupfalse%
\ {\isachardoublequoteopen}{\isacharparenleft}{\kern0pt}{\isacharparenleft}{\kern0pt}control{\isadigit{2}}\ U{\isacharparenright}{\kern0pt}\isactrlsup {\isasymdagger}\ {\isacharasterisk}{\kern0pt}\ control{\isadigit{2}}\ U{\isacharparenright}{\kern0pt}\ {\isachardollar}{\kern0pt}{\isachardollar}{\kern0pt}\ {\isacharparenleft}{\kern0pt}i{\isacharcomma}{\kern0pt}\ j{\isacharparenright}{\kern0pt}\ {\isacharequal}{\kern0pt}\ {\isadigit{1}}\isactrlsub m\ {\isadigit{4}}\ {\isachardollar}{\kern0pt}{\isachardollar}{\kern0pt}\ {\isacharparenleft}{\kern0pt}i{\isacharcomma}{\kern0pt}\ j{\isacharparenright}{\kern0pt}{\isachardoublequoteclose}\isanewline
\ \ \ \ \ \ \ \ \ \ \isacommand{proof}\isamarkupfalse%
\ {\isacharparenleft}{\kern0pt}rule\ disjE{\isacharparenright}{\kern0pt}\isanewline
\ \ \ \ \ \ \ \ \ \ \ \ \isacommand{show}\isamarkupfalse%
\ {\isachardoublequoteopen}j\ {\isacharequal}{\kern0pt}\ {\isadigit{0}}\ {\isasymor}\ j\ {\isacharequal}{\kern0pt}\ {\isadigit{1}}\ {\isasymor}\ j\ {\isacharequal}{\kern0pt}\ {\isadigit{2}}\ {\isasymor}\ j\ {\isacharequal}{\kern0pt}\ {\isadigit{3}}{\isachardoublequoteclose}\ \isacommand{using}\isamarkupfalse%
\ j{\isadigit{4}}\ \isacommand{by}\isamarkupfalse%
\ auto\isanewline
\ \ \ \ \ \ \ \ \ \ \isacommand{next}\isamarkupfalse%
\isanewline
\ \ \ \ \ \ \ \ \ \ \ \ \isacommand{assume}\isamarkupfalse%
\ j{\isadigit{0}}{\isacharcolon}{\kern0pt}{\isachardoublequoteopen}j\ {\isacharequal}{\kern0pt}\ {\isadigit{0}}{\isachardoublequoteclose}\isanewline
\ \ \ \ \ \ \ \ \ \ \ \ \isacommand{show}\isamarkupfalse%
\ {\isachardoublequoteopen}{\isacharparenleft}{\kern0pt}{\isacharparenleft}{\kern0pt}control{\isadigit{2}}\ U{\isacharparenright}{\kern0pt}\isactrlsup {\isasymdagger}\ {\isacharasterisk}{\kern0pt}\ control{\isadigit{2}}\ U{\isacharparenright}{\kern0pt}\ {\isachardollar}{\kern0pt}{\isachardollar}{\kern0pt}\ {\isacharparenleft}{\kern0pt}i{\isacharcomma}{\kern0pt}\ j{\isacharparenright}{\kern0pt}\ {\isacharequal}{\kern0pt}\ {\isadigit{1}}\isactrlsub m\ {\isadigit{4}}\ {\isachardollar}{\kern0pt}{\isachardollar}{\kern0pt}\ {\isacharparenleft}{\kern0pt}i{\isacharcomma}{\kern0pt}\ j{\isacharparenright}{\kern0pt}{\isachardoublequoteclose}\isanewline
\ \ \ \ \ \ \ \ \ \ \ \ \isacommand{proof}\isamarkupfalse%
\ {\isacharminus}{\kern0pt}\isanewline
\ \ \ \ \ \ \ \ \ \ \ \ \ \ \isacommand{have}\isamarkupfalse%
\ {\isachardoublequoteopen}{\isacharparenleft}{\kern0pt}{\isacharparenleft}{\kern0pt}control{\isadigit{2}}\ U{\isacharparenright}{\kern0pt}\isactrlsup {\isasymdagger}\ {\isacharasterisk}{\kern0pt}\ control{\isadigit{2}}\ U{\isacharparenright}{\kern0pt}\ {\isachardollar}{\kern0pt}{\isachardollar}{\kern0pt}\ {\isacharparenleft}{\kern0pt}{\isadigit{2}}{\isacharcomma}{\kern0pt}{\isadigit{0}}{\isacharparenright}{\kern0pt}\ {\isacharequal}{\kern0pt}\isanewline
\ \ \ \ \ \ \ \ \ \ \ \ \ \ \ \ {\isacharparenleft}{\kern0pt}{\isacharparenleft}{\kern0pt}control{\isadigit{2}}\ U{\isacharparenright}{\kern0pt}\isactrlsup {\isasymdagger}{\isacharparenright}{\kern0pt}\ {\isachardollar}{\kern0pt}{\isachardollar}{\kern0pt}\ {\isacharparenleft}{\kern0pt}{\isadigit{2}}{\isacharcomma}{\kern0pt}{\isadigit{0}}{\isacharparenright}{\kern0pt}\ {\isacharasterisk}{\kern0pt}\ {\isacharparenleft}{\kern0pt}control{\isadigit{2}}\ U{\isacharparenright}{\kern0pt}\ {\isachardollar}{\kern0pt}{\isachardollar}{\kern0pt}\ {\isacharparenleft}{\kern0pt}{\isadigit{0}}{\isacharcomma}{\kern0pt}{\isadigit{0}}{\isacharparenright}{\kern0pt}\ {\isacharplus}{\kern0pt}\isanewline
\ \ \ \ \ \ \ \ \ \ \ \ \ \ \ \ {\isacharparenleft}{\kern0pt}{\isacharparenleft}{\kern0pt}control{\isadigit{2}}\ U{\isacharparenright}{\kern0pt}\isactrlsup {\isasymdagger}{\isacharparenright}{\kern0pt}\ {\isachardollar}{\kern0pt}{\isachardollar}{\kern0pt}\ {\isacharparenleft}{\kern0pt}{\isadigit{2}}{\isacharcomma}{\kern0pt}{\isadigit{1}}{\isacharparenright}{\kern0pt}\ {\isacharasterisk}{\kern0pt}\ {\isacharparenleft}{\kern0pt}control{\isadigit{2}}\ U{\isacharparenright}{\kern0pt}\ {\isachardollar}{\kern0pt}{\isachardollar}{\kern0pt}\ {\isacharparenleft}{\kern0pt}{\isadigit{1}}{\isacharcomma}{\kern0pt}{\isadigit{0}}{\isacharparenright}{\kern0pt}\ {\isacharplus}{\kern0pt}\isanewline
\ \ \ \ \ \ \ \ \ \ \ \ \ \ \ \ {\isacharparenleft}{\kern0pt}{\isacharparenleft}{\kern0pt}control{\isadigit{2}}\ U{\isacharparenright}{\kern0pt}\isactrlsup {\isasymdagger}{\isacharparenright}{\kern0pt}\ {\isachardollar}{\kern0pt}{\isachardollar}{\kern0pt}\ {\isacharparenleft}{\kern0pt}{\isadigit{2}}{\isacharcomma}{\kern0pt}{\isadigit{2}}{\isacharparenright}{\kern0pt}\ {\isacharasterisk}{\kern0pt}\ {\isacharparenleft}{\kern0pt}control{\isadigit{2}}\ U{\isacharparenright}{\kern0pt}\ {\isachardollar}{\kern0pt}{\isachardollar}{\kern0pt}\ {\isacharparenleft}{\kern0pt}{\isadigit{2}}{\isacharcomma}{\kern0pt}{\isadigit{0}}{\isacharparenright}{\kern0pt}\ {\isacharplus}{\kern0pt}\isanewline
\ \ \ \ \ \ \ \ \ \ \ \ \ \ \ \ {\isacharparenleft}{\kern0pt}{\isacharparenleft}{\kern0pt}control{\isadigit{2}}\ U{\isacharparenright}{\kern0pt}\isactrlsup {\isasymdagger}{\isacharparenright}{\kern0pt}\ {\isachardollar}{\kern0pt}{\isachardollar}{\kern0pt}\ {\isacharparenleft}{\kern0pt}{\isadigit{2}}{\isacharcomma}{\kern0pt}{\isadigit{3}}{\isacharparenright}{\kern0pt}\ {\isacharasterisk}{\kern0pt}\ {\isacharparenleft}{\kern0pt}control{\isadigit{2}}\ U{\isacharparenright}{\kern0pt}\ {\isachardollar}{\kern0pt}{\isachardollar}{\kern0pt}\ {\isacharparenleft}{\kern0pt}{\isadigit{3}}{\isacharcomma}{\kern0pt}{\isadigit{0}}{\isacharparenright}{\kern0pt}{\isachardoublequoteclose}\isanewline
\ \ \ \ \ \ \ \ \ \ \ \ \ \ \ \ \isacommand{using}\isamarkupfalse%
\ sumof{\isadigit{4}}\isanewline
\ \ \ \ \ \ \ \ \ \ \ \ \ \ \ \ \isacommand{by}\isamarkupfalse%
\ {\isacharparenleft}{\kern0pt}smt\ {\isacharparenleft}{\kern0pt}z{\isadigit{3}}{\isacharparenright}{\kern0pt}\ carrier{\isacharunderscore}{\kern0pt}matD{\isacharparenleft}{\kern0pt}{\isadigit{1}}{\isacharparenright}{\kern0pt}\ carrier{\isacharunderscore}{\kern0pt}matD{\isacharparenleft}{\kern0pt}{\isadigit{2}}{\isacharparenright}{\kern0pt}\ control{\isadigit{2}}{\isacharunderscore}{\kern0pt}carrier{\isacharunderscore}{\kern0pt}mat\ dim{\isacharunderscore}{\kern0pt}col{\isacharunderscore}{\kern0pt}of{\isacharunderscore}{\kern0pt}dagger\isanewline
\ \ \ \ \ \ \ \ \ \ \ \ \ \ \ \ \ \ \ \ dim{\isacharunderscore}{\kern0pt}row{\isacharunderscore}{\kern0pt}of{\isacharunderscore}{\kern0pt}dagger\ i{\isadigit{2}}\ i{\isadigit{4}}\ index{\isacharunderscore}{\kern0pt}matrix{\isacharunderscore}{\kern0pt}prod\ zero{\isacharunderscore}{\kern0pt}less{\isacharunderscore}{\kern0pt}numeral{\isacharparenright}{\kern0pt}\isanewline
\ \ \ \ \ \ \ \ \ \ \ \ \ \ \isacommand{also}\isamarkupfalse%
\ \isacommand{have}\isamarkupfalse%
\ {\isachardoublequoteopen}{\isasymdots}\ {\isacharequal}{\kern0pt}\ {\isacharparenleft}{\kern0pt}{\isacharparenleft}{\kern0pt}control{\isadigit{2}}\ U{\isacharparenright}{\kern0pt}\isactrlsup {\isasymdagger}{\isacharparenright}{\kern0pt}\ {\isachardollar}{\kern0pt}{\isachardollar}{\kern0pt}\ {\isacharparenleft}{\kern0pt}{\isadigit{2}}{\isacharcomma}{\kern0pt}{\isadigit{0}}{\isacharparenright}{\kern0pt}{\isachardoublequoteclose}\isanewline
\ \ \ \ \ \ \ \ \ \ \ \ \ \ \ \ \isacommand{using}\isamarkupfalse%
\ control{\isadigit{2}}{\isacharunderscore}{\kern0pt}def\ index{\isacharunderscore}{\kern0pt}mat{\isacharunderscore}{\kern0pt}of{\isacharunderscore}{\kern0pt}cols{\isacharunderscore}{\kern0pt}list\ \isacommand{by}\isamarkupfalse%
\ force\isanewline
\ \ \ \ \ \ \ \ \ \ \ \ \ \ \isacommand{also}\isamarkupfalse%
\ \isacommand{have}\isamarkupfalse%
\ {\isachardoublequoteopen}{\isasymdots}\ {\isacharequal}{\kern0pt}\ cnj\ {\isacharparenleft}{\kern0pt}{\isacharparenleft}{\kern0pt}control{\isadigit{2}}\ U{\isacharparenright}{\kern0pt}\ {\isachardollar}{\kern0pt}{\isachardollar}{\kern0pt}\ {\isacharparenleft}{\kern0pt}{\isadigit{0}}{\isacharcomma}{\kern0pt}{\isadigit{2}}{\isacharparenright}{\kern0pt}{\isacharparenright}{\kern0pt}{\isachardoublequoteclose}\isanewline
\ \ \ \ \ \ \ \ \ \ \ \ \ \ \ \ \isacommand{using}\isamarkupfalse%
\ dagger{\isacharunderscore}{\kern0pt}def\isanewline
\ \ \ \ \ \ \ \ \ \ \ \ \ \ \ \ \isacommand{by}\isamarkupfalse%
\ {\isacharparenleft}{\kern0pt}simp\ add{\isacharcolon}{\kern0pt}\ Tensor{\isachardot}{\kern0pt}mat{\isacharunderscore}{\kern0pt}of{\isacharunderscore}{\kern0pt}cols{\isacharunderscore}{\kern0pt}list{\isacharunderscore}{\kern0pt}def\ control{\isadigit{2}}{\isacharunderscore}{\kern0pt}def{\isacharparenright}{\kern0pt}\isanewline
\ \ \ \ \ \ \ \ \ \ \ \ \ \ \isacommand{also}\isamarkupfalse%
\ \isacommand{have}\isamarkupfalse%
\ {\isachardoublequoteopen}{\isasymdots}\ {\isacharequal}{\kern0pt}\ {\isadigit{0}}{\isachardoublequoteclose}\ \isacommand{using}\isamarkupfalse%
\ control{\isadigit{2}}{\isacharunderscore}{\kern0pt}def\ index{\isacharunderscore}{\kern0pt}mat{\isacharunderscore}{\kern0pt}of{\isacharunderscore}{\kern0pt}cols{\isacharunderscore}{\kern0pt}list\ \isacommand{by}\isamarkupfalse%
\ auto\isanewline
\ \ \ \ \ \ \ \ \ \ \ \ \ \ \isacommand{also}\isamarkupfalse%
\ \isacommand{have}\isamarkupfalse%
\ {\isachardoublequoteopen}{\isasymdots}\ {\isacharequal}{\kern0pt}\ {\isadigit{1}}\isactrlsub m\ {\isadigit{4}}\ {\isachardollar}{\kern0pt}{\isachardollar}{\kern0pt}\ {\isacharparenleft}{\kern0pt}{\isadigit{2}}{\isacharcomma}{\kern0pt}{\isadigit{0}}{\isacharparenright}{\kern0pt}{\isachardoublequoteclose}\ \isacommand{by}\isamarkupfalse%
\ simp\isanewline
\ \ \ \ \ \ \ \ \ \ \ \ \ \ \isacommand{finally}\isamarkupfalse%
\ \isacommand{show}\isamarkupfalse%
\ {\isacharquery}{\kern0pt}thesis\ \isacommand{using}\isamarkupfalse%
\ i{\isadigit{2}}\ j{\isadigit{0}}\ \isacommand{by}\isamarkupfalse%
\ simp\isanewline
\ \ \ \ \ \ \ \ \ \ \ \ \isacommand{qed}\isamarkupfalse%
\isanewline
\ \ \ \ \ \ \ \ \ \ \isacommand{next}\isamarkupfalse%
\isanewline
\ \ \ \ \ \ \ \ \ \ \ \ \isacommand{assume}\isamarkupfalse%
\ jl{\isadigit{3}}{\isacharcolon}{\kern0pt}{\isachardoublequoteopen}j\ {\isacharequal}{\kern0pt}\ {\isadigit{1}}\ {\isasymor}\ j\ {\isacharequal}{\kern0pt}\ {\isadigit{2}}\ {\isasymor}\ j\ {\isacharequal}{\kern0pt}\ {\isadigit{3}}{\isachardoublequoteclose}\isanewline
\ \ \ \ \ \ \ \ \ \ \ \ \isacommand{show}\isamarkupfalse%
\ {\isachardoublequoteopen}{\isacharparenleft}{\kern0pt}{\isacharparenleft}{\kern0pt}control{\isadigit{2}}\ U{\isacharparenright}{\kern0pt}\isactrlsup {\isasymdagger}\ {\isacharasterisk}{\kern0pt}\ control{\isadigit{2}}\ U{\isacharparenright}{\kern0pt}\ {\isachardollar}{\kern0pt}{\isachardollar}{\kern0pt}\ {\isacharparenleft}{\kern0pt}i{\isacharcomma}{\kern0pt}\ j{\isacharparenright}{\kern0pt}\ {\isacharequal}{\kern0pt}\ {\isadigit{1}}\isactrlsub m\ {\isadigit{4}}\ {\isachardollar}{\kern0pt}{\isachardollar}{\kern0pt}\ {\isacharparenleft}{\kern0pt}i{\isacharcomma}{\kern0pt}\ j{\isacharparenright}{\kern0pt}{\isachardoublequoteclose}\isanewline
\ \ \ \ \ \ \ \ \ \ \ \ \isacommand{proof}\isamarkupfalse%
\ {\isacharparenleft}{\kern0pt}rule\ disjE{\isacharparenright}{\kern0pt}\isanewline
\ \ \ \ \ \ \ \ \ \ \ \ \ \ \isacommand{show}\isamarkupfalse%
\ {\isachardoublequoteopen}j\ {\isacharequal}{\kern0pt}\ {\isadigit{1}}\ {\isasymor}\ j\ {\isacharequal}{\kern0pt}\ {\isadigit{2}}\ {\isasymor}\ j\ {\isacharequal}{\kern0pt}\ {\isadigit{3}}{\isachardoublequoteclose}\ \isacommand{using}\isamarkupfalse%
\ jl{\isadigit{3}}\ \isacommand{by}\isamarkupfalse%
\ this\isanewline
\ \ \ \ \ \ \ \ \ \ \ \ \isacommand{next}\isamarkupfalse%
\isanewline
\ \ \ \ \ \ \ \ \ \ \ \ \ \ \isacommand{assume}\isamarkupfalse%
\ j{\isadigit{1}}{\isacharcolon}{\kern0pt}{\isachardoublequoteopen}j\ {\isacharequal}{\kern0pt}\ {\isadigit{1}}{\isachardoublequoteclose}\isanewline
\ \ \ \ \ \ \ \ \ \ \ \ \ \ \isacommand{show}\isamarkupfalse%
\ {\isachardoublequoteopen}{\isacharparenleft}{\kern0pt}{\isacharparenleft}{\kern0pt}control{\isadigit{2}}\ U{\isacharparenright}{\kern0pt}\isactrlsup {\isasymdagger}\ {\isacharasterisk}{\kern0pt}\ control{\isadigit{2}}\ U{\isacharparenright}{\kern0pt}\ {\isachardollar}{\kern0pt}{\isachardollar}{\kern0pt}\ {\isacharparenleft}{\kern0pt}i{\isacharcomma}{\kern0pt}\ j{\isacharparenright}{\kern0pt}\ {\isacharequal}{\kern0pt}\ {\isadigit{1}}\isactrlsub m\ {\isadigit{4}}\ {\isachardollar}{\kern0pt}{\isachardollar}{\kern0pt}\ {\isacharparenleft}{\kern0pt}i{\isacharcomma}{\kern0pt}\ j{\isacharparenright}{\kern0pt}{\isachardoublequoteclose}\isanewline
\ \ \ \ \ \ \ \ \ \ \ \ \ \ \isacommand{proof}\isamarkupfalse%
\ {\isacharminus}{\kern0pt}\isanewline
\ \ \ \ \ \ \ \ \ \ \ \ \ \ \ \ \isacommand{have}\isamarkupfalse%
\ {\isachardoublequoteopen}{\isacharparenleft}{\kern0pt}{\isacharparenleft}{\kern0pt}control{\isadigit{2}}\ U{\isacharparenright}{\kern0pt}\isactrlsup {\isasymdagger}\ {\isacharasterisk}{\kern0pt}\ control{\isadigit{2}}\ U{\isacharparenright}{\kern0pt}\ {\isachardollar}{\kern0pt}{\isachardollar}{\kern0pt}\ {\isacharparenleft}{\kern0pt}{\isadigit{2}}{\isacharcomma}{\kern0pt}{\isadigit{1}}{\isacharparenright}{\kern0pt}\ {\isacharequal}{\kern0pt}\isanewline
\ \ \ \ \ \ \ \ \ \ \ \ \ \ \ \ {\isacharparenleft}{\kern0pt}{\isacharparenleft}{\kern0pt}control{\isadigit{2}}\ U{\isacharparenright}{\kern0pt}\isactrlsup {\isasymdagger}{\isacharparenright}{\kern0pt}\ {\isachardollar}{\kern0pt}{\isachardollar}{\kern0pt}\ {\isacharparenleft}{\kern0pt}{\isadigit{2}}{\isacharcomma}{\kern0pt}{\isadigit{0}}{\isacharparenright}{\kern0pt}\ {\isacharasterisk}{\kern0pt}\ {\isacharparenleft}{\kern0pt}control{\isadigit{2}}\ U{\isacharparenright}{\kern0pt}\ {\isachardollar}{\kern0pt}{\isachardollar}{\kern0pt}\ {\isacharparenleft}{\kern0pt}{\isadigit{0}}{\isacharcomma}{\kern0pt}{\isadigit{1}}{\isacharparenright}{\kern0pt}\ {\isacharplus}{\kern0pt}\isanewline
\ \ \ \ \ \ \ \ \ \ \ \ \ \ \ \ {\isacharparenleft}{\kern0pt}{\isacharparenleft}{\kern0pt}control{\isadigit{2}}\ U{\isacharparenright}{\kern0pt}\isactrlsup {\isasymdagger}{\isacharparenright}{\kern0pt}\ {\isachardollar}{\kern0pt}{\isachardollar}{\kern0pt}\ {\isacharparenleft}{\kern0pt}{\isadigit{2}}{\isacharcomma}{\kern0pt}{\isadigit{1}}{\isacharparenright}{\kern0pt}\ {\isacharasterisk}{\kern0pt}\ {\isacharparenleft}{\kern0pt}control{\isadigit{2}}\ U{\isacharparenright}{\kern0pt}\ {\isachardollar}{\kern0pt}{\isachardollar}{\kern0pt}\ {\isacharparenleft}{\kern0pt}{\isadigit{1}}{\isacharcomma}{\kern0pt}{\isadigit{1}}{\isacharparenright}{\kern0pt}\ {\isacharplus}{\kern0pt}\isanewline
\ \ \ \ \ \ \ \ \ \ \ \ \ \ \ \ {\isacharparenleft}{\kern0pt}{\isacharparenleft}{\kern0pt}control{\isadigit{2}}\ U{\isacharparenright}{\kern0pt}\isactrlsup {\isasymdagger}{\isacharparenright}{\kern0pt}\ {\isachardollar}{\kern0pt}{\isachardollar}{\kern0pt}\ {\isacharparenleft}{\kern0pt}{\isadigit{2}}{\isacharcomma}{\kern0pt}{\isadigit{2}}{\isacharparenright}{\kern0pt}\ {\isacharasterisk}{\kern0pt}\ {\isacharparenleft}{\kern0pt}control{\isadigit{2}}\ U{\isacharparenright}{\kern0pt}\ {\isachardollar}{\kern0pt}{\isachardollar}{\kern0pt}\ {\isacharparenleft}{\kern0pt}{\isadigit{2}}{\isacharcomma}{\kern0pt}{\isadigit{1}}{\isacharparenright}{\kern0pt}\ {\isacharplus}{\kern0pt}\isanewline
\ \ \ \ \ \ \ \ \ \ \ \ \ \ \ \ {\isacharparenleft}{\kern0pt}{\isacharparenleft}{\kern0pt}control{\isadigit{2}}\ U{\isacharparenright}{\kern0pt}\isactrlsup {\isasymdagger}{\isacharparenright}{\kern0pt}\ {\isachardollar}{\kern0pt}{\isachardollar}{\kern0pt}\ {\isacharparenleft}{\kern0pt}{\isadigit{2}}{\isacharcomma}{\kern0pt}{\isadigit{3}}{\isacharparenright}{\kern0pt}\ {\isacharasterisk}{\kern0pt}\ {\isacharparenleft}{\kern0pt}control{\isadigit{2}}\ U{\isacharparenright}{\kern0pt}\ {\isachardollar}{\kern0pt}{\isachardollar}{\kern0pt}\ {\isacharparenleft}{\kern0pt}{\isadigit{3}}{\isacharcomma}{\kern0pt}{\isadigit{1}}{\isacharparenright}{\kern0pt}{\isachardoublequoteclose}\isanewline
\ \ \ \ \ \ \ \ \ \ \ \ \ \ \ \ \ \ \isacommand{using}\isamarkupfalse%
\ sumof{\isadigit{4}}\isanewline
\ \ \ \ \ \ \ \ \ \ \ \ \ \ \ \ \ \ \isacommand{by}\isamarkupfalse%
\ {\isacharparenleft}{\kern0pt}smt\ {\isacharparenleft}{\kern0pt}z{\isadigit{3}}{\isacharparenright}{\kern0pt}\ carrier{\isacharunderscore}{\kern0pt}matD{\isacharparenleft}{\kern0pt}{\isadigit{1}}{\isacharparenright}{\kern0pt}\ carrier{\isacharunderscore}{\kern0pt}matD{\isacharparenleft}{\kern0pt}{\isadigit{2}}{\isacharparenright}{\kern0pt}\ control{\isadigit{2}}{\isacharunderscore}{\kern0pt}carrier{\isacharunderscore}{\kern0pt}mat\ \isanewline
\ \ \ \ \ \ \ \ \ \ \ \ \ \ \ \ \ \ \ \ \ \ dim{\isacharunderscore}{\kern0pt}col{\isacharunderscore}{\kern0pt}of{\isacharunderscore}{\kern0pt}dagger\ dim{\isacharunderscore}{\kern0pt}row{\isacharunderscore}{\kern0pt}of{\isacharunderscore}{\kern0pt}dagger\ i{\isadigit{2}}\ i{\isadigit{4}}\ index{\isacharunderscore}{\kern0pt}matrix{\isacharunderscore}{\kern0pt}prod\ \isanewline
\ \ \ \ \ \ \ \ \ \ \ \ \ \ \ \ \ \ \ \ \ \ one{\isacharunderscore}{\kern0pt}less{\isacharunderscore}{\kern0pt}numeral{\isacharunderscore}{\kern0pt}iff\ semiring{\isacharunderscore}{\kern0pt}norm{\isacharparenleft}{\kern0pt}{\isadigit{7}}{\isadigit{6}}{\isacharparenright}{\kern0pt}{\isacharparenright}{\kern0pt}\isanewline
\ \ \ \ \ \ \ \ \ \ \ \ \ \ \ \ \isacommand{also}\isamarkupfalse%
\ \isacommand{have}\isamarkupfalse%
\ {\isachardoublequoteopen}{\isasymdots}\ {\isacharequal}{\kern0pt}\ {\isacharparenleft}{\kern0pt}{\isacharparenleft}{\kern0pt}control{\isadigit{2}}\ U{\isacharparenright}{\kern0pt}\isactrlsup {\isasymdagger}{\isacharparenright}{\kern0pt}\ {\isachardollar}{\kern0pt}{\isachardollar}{\kern0pt}\ {\isacharparenleft}{\kern0pt}{\isadigit{2}}{\isacharcomma}{\kern0pt}{\isadigit{1}}{\isacharparenright}{\kern0pt}\ {\isacharasterisk}{\kern0pt}\ {\isacharparenleft}{\kern0pt}control{\isadigit{2}}\ U{\isacharparenright}{\kern0pt}\ {\isachardollar}{\kern0pt}{\isachardollar}{\kern0pt}\ {\isacharparenleft}{\kern0pt}{\isadigit{1}}{\isacharcomma}{\kern0pt}{\isadigit{1}}{\isacharparenright}{\kern0pt}\ {\isacharplus}{\kern0pt}\isanewline
\ \ \ \ \ \ \ \ \ \ \ \ \ \ \ \ \ \ \ \ \ \ \ \ \ \ \ \ \ \ \ \ {\isacharparenleft}{\kern0pt}{\isacharparenleft}{\kern0pt}control{\isadigit{2}}\ U{\isacharparenright}{\kern0pt}\isactrlsup {\isasymdagger}{\isacharparenright}{\kern0pt}\ {\isachardollar}{\kern0pt}{\isachardollar}{\kern0pt}\ {\isacharparenleft}{\kern0pt}{\isadigit{2}}{\isacharcomma}{\kern0pt}{\isadigit{3}}{\isacharparenright}{\kern0pt}\ {\isacharasterisk}{\kern0pt}\ {\isacharparenleft}{\kern0pt}control{\isadigit{2}}\ U{\isacharparenright}{\kern0pt}\ {\isachardollar}{\kern0pt}{\isachardollar}{\kern0pt}\ {\isacharparenleft}{\kern0pt}{\isadigit{3}}{\isacharcomma}{\kern0pt}{\isadigit{1}}{\isacharparenright}{\kern0pt}{\isachardoublequoteclose}\isanewline
\ \ \ \ \ \ \ \ \ \ \ \ \ \ \ \ \ \ \isacommand{using}\isamarkupfalse%
\ control{\isadigit{2}}{\isacharunderscore}{\kern0pt}def\ index{\isacharunderscore}{\kern0pt}mat{\isacharunderscore}{\kern0pt}of{\isacharunderscore}{\kern0pt}cols{\isacharunderscore}{\kern0pt}list\ \isacommand{by}\isamarkupfalse%
\ force\isanewline
\ \ \ \ \ \ \ \ \ \ \ \ \ \ \ \ \isacommand{also}\isamarkupfalse%
\ \isacommand{have}\isamarkupfalse%
\ {\isachardoublequoteopen}{\isasymdots}\ {\isacharequal}{\kern0pt}\ cnj\ {\isacharparenleft}{\kern0pt}{\isacharparenleft}{\kern0pt}control{\isadigit{2}}\ U{\isacharparenright}{\kern0pt}\ {\isachardollar}{\kern0pt}{\isachardollar}{\kern0pt}\ {\isacharparenleft}{\kern0pt}{\isadigit{1}}{\isacharcomma}{\kern0pt}{\isadigit{2}}{\isacharparenright}{\kern0pt}{\isacharparenright}{\kern0pt}\ {\isacharasterisk}{\kern0pt}\ {\isacharparenleft}{\kern0pt}control{\isadigit{2}}\ U{\isacharparenright}{\kern0pt}\ {\isachardollar}{\kern0pt}{\isachardollar}{\kern0pt}\ {\isacharparenleft}{\kern0pt}{\isadigit{1}}{\isacharcomma}{\kern0pt}{\isadigit{1}}{\isacharparenright}{\kern0pt}\ {\isacharplus}{\kern0pt}\isanewline
\ \ \ \ \ \ \ \ \ \ \ \ \ \ \ \ \ \ \ \ \ \ \ \ \ \ \ \ \ \ \ \ cnj\ {\isacharparenleft}{\kern0pt}{\isacharparenleft}{\kern0pt}control{\isadigit{2}}\ U{\isacharparenright}{\kern0pt}\ {\isachardollar}{\kern0pt}{\isachardollar}{\kern0pt}\ {\isacharparenleft}{\kern0pt}{\isadigit{3}}{\isacharcomma}{\kern0pt}{\isadigit{2}}{\isacharparenright}{\kern0pt}{\isacharparenright}{\kern0pt}\ {\isacharasterisk}{\kern0pt}\ {\isacharparenleft}{\kern0pt}control{\isadigit{2}}\ U{\isacharparenright}{\kern0pt}\ {\isachardollar}{\kern0pt}{\isachardollar}{\kern0pt}\ {\isacharparenleft}{\kern0pt}{\isadigit{3}}{\isacharcomma}{\kern0pt}{\isadigit{1}}{\isacharparenright}{\kern0pt}{\isachardoublequoteclose}\isanewline
\ \ \ \ \ \ \ \ \ \ \ \ \ \ \ \ \ \ \isacommand{using}\isamarkupfalse%
\ dagger{\isacharunderscore}{\kern0pt}def\isanewline
\ \ \ \ \ \ \ \ \ \ \ \ \ \ \ \ \ \ \isacommand{by}\isamarkupfalse%
\ {\isacharparenleft}{\kern0pt}simp\ add{\isacharcolon}{\kern0pt}\ Tensor{\isachardot}{\kern0pt}mat{\isacharunderscore}{\kern0pt}of{\isacharunderscore}{\kern0pt}cols{\isacharunderscore}{\kern0pt}list{\isacharunderscore}{\kern0pt}def\ control{\isadigit{2}}{\isacharunderscore}{\kern0pt}def{\isacharparenright}{\kern0pt}\isanewline
\ \ \ \ \ \ \ \ \ \ \ \ \ \ \ \ \isacommand{also}\isamarkupfalse%
\ \isacommand{have}\isamarkupfalse%
\ {\isachardoublequoteopen}{\isasymdots}\ {\isacharequal}{\kern0pt}\ {\isadigit{0}}{\isachardoublequoteclose}\ \isacommand{using}\isamarkupfalse%
\ control{\isadigit{2}}{\isacharunderscore}{\kern0pt}def\ index{\isacharunderscore}{\kern0pt}mat{\isacharunderscore}{\kern0pt}of{\isacharunderscore}{\kern0pt}cols{\isacharunderscore}{\kern0pt}list\ \isacommand{by}\isamarkupfalse%
\ auto\isanewline
\ \ \ \ \ \ \ \ \ \ \ \ \ \ \ \ \isacommand{also}\isamarkupfalse%
\ \isacommand{have}\isamarkupfalse%
\ {\isachardoublequoteopen}{\isasymdots}\ {\isacharequal}{\kern0pt}\ {\isadigit{1}}\isactrlsub m\ {\isadigit{4}}\ {\isachardollar}{\kern0pt}{\isachardollar}{\kern0pt}\ {\isacharparenleft}{\kern0pt}{\isadigit{2}}{\isacharcomma}{\kern0pt}{\isadigit{1}}{\isacharparenright}{\kern0pt}{\isachardoublequoteclose}\ \isacommand{by}\isamarkupfalse%
\ simp\isanewline
\ \ \ \ \ \ \ \ \ \ \ \ \ \ \ \ \isacommand{finally}\isamarkupfalse%
\ \isacommand{show}\isamarkupfalse%
\ {\isacharquery}{\kern0pt}thesis\ \isacommand{using}\isamarkupfalse%
\ i{\isadigit{2}}\ j{\isadigit{1}}\ \isacommand{by}\isamarkupfalse%
\ simp\isanewline
\ \ \ \ \ \ \ \ \ \ \ \ \ \ \isacommand{qed}\isamarkupfalse%
\isanewline
\ \ \ \ \ \ \ \ \ \ \ \ \isacommand{next}\isamarkupfalse%
\isanewline
\ \ \ \ \ \ \ \ \ \ \ \ \ \ \isacommand{assume}\isamarkupfalse%
\ jl{\isadigit{2}}{\isacharcolon}{\kern0pt}{\isachardoublequoteopen}j\ {\isacharequal}{\kern0pt}\ {\isadigit{2}}\ {\isasymor}\ j\ {\isacharequal}{\kern0pt}\ {\isadigit{3}}{\isachardoublequoteclose}\isanewline
\ \ \ \ \ \ \ \ \ \ \ \ \ \ \isacommand{show}\isamarkupfalse%
\ {\isachardoublequoteopen}{\isacharparenleft}{\kern0pt}{\isacharparenleft}{\kern0pt}control{\isadigit{2}}\ U{\isacharparenright}{\kern0pt}\isactrlsup {\isasymdagger}\ {\isacharasterisk}{\kern0pt}\ control{\isadigit{2}}\ U{\isacharparenright}{\kern0pt}\ {\isachardollar}{\kern0pt}{\isachardollar}{\kern0pt}\ {\isacharparenleft}{\kern0pt}i{\isacharcomma}{\kern0pt}\ j{\isacharparenright}{\kern0pt}\ {\isacharequal}{\kern0pt}\ {\isadigit{1}}\isactrlsub m\ {\isadigit{4}}\ {\isachardollar}{\kern0pt}{\isachardollar}{\kern0pt}\ {\isacharparenleft}{\kern0pt}i{\isacharcomma}{\kern0pt}\ j{\isacharparenright}{\kern0pt}{\isachardoublequoteclose}\isanewline
\ \ \ \ \ \ \ \ \ \ \ \ \ \ \isacommand{proof}\isamarkupfalse%
\ {\isacharparenleft}{\kern0pt}rule\ disjE{\isacharparenright}{\kern0pt}\isanewline
\ \ \ \ \ \ \ \ \ \ \ \ \ \ \ \ \isacommand{show}\isamarkupfalse%
\ {\isachardoublequoteopen}j\ {\isacharequal}{\kern0pt}\ {\isadigit{2}}\ {\isasymor}\ j\ {\isacharequal}{\kern0pt}\ {\isadigit{3}}{\isachardoublequoteclose}\ \isacommand{using}\isamarkupfalse%
\ jl{\isadigit{2}}\ \isacommand{by}\isamarkupfalse%
\ this\isanewline
\ \ \ \ \ \ \ \ \ \ \ \ \ \ \isacommand{next}\isamarkupfalse%
\isanewline
\ \ \ \ \ \ \ \ \ \ \ \ \ \ \ \ \isacommand{assume}\isamarkupfalse%
\ j{\isadigit{2}}{\isacharcolon}{\kern0pt}{\isachardoublequoteopen}j\ {\isacharequal}{\kern0pt}\ {\isadigit{2}}{\isachardoublequoteclose}\isanewline
\ \ \ \ \ \ \ \ \ \ \ \ \ \ \ \ \isacommand{show}\isamarkupfalse%
\ {\isachardoublequoteopen}{\isacharparenleft}{\kern0pt}{\isacharparenleft}{\kern0pt}control{\isadigit{2}}\ U{\isacharparenright}{\kern0pt}\isactrlsup {\isasymdagger}\ {\isacharasterisk}{\kern0pt}\ control{\isadigit{2}}\ U{\isacharparenright}{\kern0pt}\ {\isachardollar}{\kern0pt}{\isachardollar}{\kern0pt}\ {\isacharparenleft}{\kern0pt}i{\isacharcomma}{\kern0pt}\ j{\isacharparenright}{\kern0pt}\ {\isacharequal}{\kern0pt}\ {\isadigit{1}}\isactrlsub m\ {\isadigit{4}}\ {\isachardollar}{\kern0pt}{\isachardollar}{\kern0pt}\ {\isacharparenleft}{\kern0pt}i{\isacharcomma}{\kern0pt}\ j{\isacharparenright}{\kern0pt}{\isachardoublequoteclose}\isanewline
\ \ \ \ \ \ \ \ \ \ \ \ \ \ \ \ \isacommand{proof}\isamarkupfalse%
\ {\isacharminus}{\kern0pt}\isanewline
\ \ \ \ \ \ \ \ \ \ \ \ \ \ \ \ \ \ \isacommand{have}\isamarkupfalse%
\ {\isachardoublequoteopen}{\isacharparenleft}{\kern0pt}{\isacharparenleft}{\kern0pt}control{\isadigit{2}}\ U{\isacharparenright}{\kern0pt}\isactrlsup {\isasymdagger}\ {\isacharasterisk}{\kern0pt}\ control{\isadigit{2}}\ U{\isacharparenright}{\kern0pt}\ {\isachardollar}{\kern0pt}{\isachardollar}{\kern0pt}\ {\isacharparenleft}{\kern0pt}{\isadigit{2}}{\isacharcomma}{\kern0pt}{\isadigit{2}}{\isacharparenright}{\kern0pt}\ {\isacharequal}{\kern0pt}\isanewline
\ \ \ \ \ \ \ \ \ \ \ \ \ \ \ \ \ \ \ \ \ \ \ \ {\isacharparenleft}{\kern0pt}{\isacharparenleft}{\kern0pt}control{\isadigit{2}}\ U{\isacharparenright}{\kern0pt}\isactrlsup {\isasymdagger}{\isacharparenright}{\kern0pt}\ {\isachardollar}{\kern0pt}{\isachardollar}{\kern0pt}\ {\isacharparenleft}{\kern0pt}{\isadigit{2}}{\isacharcomma}{\kern0pt}{\isadigit{0}}{\isacharparenright}{\kern0pt}\ {\isacharasterisk}{\kern0pt}\ {\isacharparenleft}{\kern0pt}control{\isadigit{2}}\ U{\isacharparenright}{\kern0pt}\ {\isachardollar}{\kern0pt}{\isachardollar}{\kern0pt}\ {\isacharparenleft}{\kern0pt}{\isadigit{0}}{\isacharcomma}{\kern0pt}{\isadigit{2}}{\isacharparenright}{\kern0pt}\ {\isacharplus}{\kern0pt}\isanewline
\ \ \ \ \ \ \ \ \ \ \ \ \ \ \ \ \ \ \ \ \ \ \ \ {\isacharparenleft}{\kern0pt}{\isacharparenleft}{\kern0pt}control{\isadigit{2}}\ U{\isacharparenright}{\kern0pt}\isactrlsup {\isasymdagger}{\isacharparenright}{\kern0pt}\ {\isachardollar}{\kern0pt}{\isachardollar}{\kern0pt}\ {\isacharparenleft}{\kern0pt}{\isadigit{2}}{\isacharcomma}{\kern0pt}{\isadigit{1}}{\isacharparenright}{\kern0pt}\ {\isacharasterisk}{\kern0pt}\ {\isacharparenleft}{\kern0pt}control{\isadigit{2}}\ U{\isacharparenright}{\kern0pt}\ {\isachardollar}{\kern0pt}{\isachardollar}{\kern0pt}\ {\isacharparenleft}{\kern0pt}{\isadigit{1}}{\isacharcomma}{\kern0pt}{\isadigit{2}}{\isacharparenright}{\kern0pt}\ {\isacharplus}{\kern0pt}\isanewline
\ \ \ \ \ \ \ \ \ \ \ \ \ \ \ \ \ \ \ \ \ \ \ \ {\isacharparenleft}{\kern0pt}{\isacharparenleft}{\kern0pt}control{\isadigit{2}}\ U{\isacharparenright}{\kern0pt}\isactrlsup {\isasymdagger}{\isacharparenright}{\kern0pt}\ {\isachardollar}{\kern0pt}{\isachardollar}{\kern0pt}\ {\isacharparenleft}{\kern0pt}{\isadigit{2}}{\isacharcomma}{\kern0pt}{\isadigit{2}}{\isacharparenright}{\kern0pt}\ {\isacharasterisk}{\kern0pt}\ {\isacharparenleft}{\kern0pt}control{\isadigit{2}}\ U{\isacharparenright}{\kern0pt}\ {\isachardollar}{\kern0pt}{\isachardollar}{\kern0pt}\ {\isacharparenleft}{\kern0pt}{\isadigit{2}}{\isacharcomma}{\kern0pt}{\isadigit{2}}{\isacharparenright}{\kern0pt}\ {\isacharplus}{\kern0pt}\isanewline
\ \ \ \ \ \ \ \ \ \ \ \ \ \ \ \ \ \ \ \ \ \ \ \ {\isacharparenleft}{\kern0pt}{\isacharparenleft}{\kern0pt}control{\isadigit{2}}\ U{\isacharparenright}{\kern0pt}\isactrlsup {\isasymdagger}{\isacharparenright}{\kern0pt}\ {\isachardollar}{\kern0pt}{\isachardollar}{\kern0pt}\ {\isacharparenleft}{\kern0pt}{\isadigit{2}}{\isacharcomma}{\kern0pt}{\isadigit{3}}{\isacharparenright}{\kern0pt}\ {\isacharasterisk}{\kern0pt}\ {\isacharparenleft}{\kern0pt}control{\isadigit{2}}\ U{\isacharparenright}{\kern0pt}\ {\isachardollar}{\kern0pt}{\isachardollar}{\kern0pt}\ {\isacharparenleft}{\kern0pt}{\isadigit{3}}{\isacharcomma}{\kern0pt}{\isadigit{2}}{\isacharparenright}{\kern0pt}{\isachardoublequoteclose}\isanewline
\ \ \ \ \ \ \ \ \ \ \ \ \ \ \ \ \ \ \ \ \isacommand{using}\isamarkupfalse%
\ sumof{\isadigit{4}}\isanewline
\ \ \ \ \ \ \ \ \ \ \ \ \ \ \ \ \ \ \ \ \isacommand{by}\isamarkupfalse%
\ {\isacharparenleft}{\kern0pt}smt\ {\isacharparenleft}{\kern0pt}z{\isadigit{3}}{\isacharparenright}{\kern0pt}\ carrier{\isacharunderscore}{\kern0pt}matD{\isacharparenleft}{\kern0pt}{\isadigit{1}}{\isacharparenright}{\kern0pt}\ carrier{\isacharunderscore}{\kern0pt}matD{\isacharparenleft}{\kern0pt}{\isadigit{2}}{\isacharparenright}{\kern0pt}\ control{\isadigit{2}}{\isacharunderscore}{\kern0pt}carrier{\isacharunderscore}{\kern0pt}mat\ dim{\isacharunderscore}{\kern0pt}col{\isacharunderscore}{\kern0pt}of{\isacharunderscore}{\kern0pt}dagger\isanewline
\ \ \ \ \ \ \ \ \ \ \ \ \ \ \ \ \ \ \ \ \ \ \ \ dim{\isacharunderscore}{\kern0pt}row{\isacharunderscore}{\kern0pt}of{\isacharunderscore}{\kern0pt}dagger\ i{\isadigit{2}}\ i{\isadigit{4}}\ index{\isacharunderscore}{\kern0pt}matrix{\isacharunderscore}{\kern0pt}prod\ zero{\isacharunderscore}{\kern0pt}less{\isacharunderscore}{\kern0pt}numeral{\isacharparenright}{\kern0pt}\isanewline
\ \ \ \ \ \ \ \ \ \ \ \ \ \ \ \ \ \ \isacommand{also}\isamarkupfalse%
\ \isacommand{have}\isamarkupfalse%
\ {\isachardoublequoteopen}{\isasymdots}\ {\isacharequal}{\kern0pt}\ {\isacharparenleft}{\kern0pt}{\isacharparenleft}{\kern0pt}control{\isadigit{2}}\ U{\isacharparenright}{\kern0pt}\isactrlsup {\isasymdagger}{\isacharparenright}{\kern0pt}\ {\isachardollar}{\kern0pt}{\isachardollar}{\kern0pt}\ {\isacharparenleft}{\kern0pt}{\isadigit{2}}{\isacharcomma}{\kern0pt}{\isadigit{2}}{\isacharparenright}{\kern0pt}{\isachardoublequoteclose}\isanewline
\ \ \ \ \ \ \ \ \ \ \ \ \ \ \ \ \ \ \ \ \isacommand{using}\isamarkupfalse%
\ control{\isadigit{2}}{\isacharunderscore}{\kern0pt}def\ index{\isacharunderscore}{\kern0pt}mat{\isacharunderscore}{\kern0pt}of{\isacharunderscore}{\kern0pt}cols{\isacharunderscore}{\kern0pt}list\ \isacommand{by}\isamarkupfalse%
\ force\isanewline
\ \ \ \ \ \ \ \ \ \ \ \ \ \ \ \ \ \ \isacommand{also}\isamarkupfalse%
\ \isacommand{have}\isamarkupfalse%
\ {\isachardoublequoteopen}{\isasymdots}\ {\isacharequal}{\kern0pt}\ cnj\ {\isacharparenleft}{\kern0pt}{\isacharparenleft}{\kern0pt}control{\isadigit{2}}\ U{\isacharparenright}{\kern0pt}\ {\isachardollar}{\kern0pt}{\isachardollar}{\kern0pt}\ {\isacharparenleft}{\kern0pt}{\isadigit{2}}{\isacharcomma}{\kern0pt}{\isadigit{2}}{\isacharparenright}{\kern0pt}{\isacharparenright}{\kern0pt}{\isachardoublequoteclose}\isanewline
\ \ \ \ \ \ \ \ \ \ \ \ \ \ \ \ \ \ \ \ \isacommand{using}\isamarkupfalse%
\ dagger{\isacharunderscore}{\kern0pt}def\isanewline
\ \ \ \ \ \ \ \ \ \ \ \ \ \ \ \ \ \ \ \ \isacommand{by}\isamarkupfalse%
\ {\isacharparenleft}{\kern0pt}simp\ add{\isacharcolon}{\kern0pt}\ Tensor{\isachardot}{\kern0pt}mat{\isacharunderscore}{\kern0pt}of{\isacharunderscore}{\kern0pt}cols{\isacharunderscore}{\kern0pt}list{\isacharunderscore}{\kern0pt}def\ control{\isadigit{2}}{\isacharunderscore}{\kern0pt}def{\isacharparenright}{\kern0pt}\isanewline
\ \ \ \ \ \ \ \ \ \ \ \ \ \ \ \ \ \ \isacommand{also}\isamarkupfalse%
\ \isacommand{have}\isamarkupfalse%
\ {\isachardoublequoteopen}{\isasymdots}\ {\isacharequal}{\kern0pt}\ {\isadigit{1}}{\isachardoublequoteclose}\ \isacommand{using}\isamarkupfalse%
\ control{\isadigit{2}}{\isacharunderscore}{\kern0pt}def\ index{\isacharunderscore}{\kern0pt}mat{\isacharunderscore}{\kern0pt}of{\isacharunderscore}{\kern0pt}cols{\isacharunderscore}{\kern0pt}list\ \isacommand{by}\isamarkupfalse%
\ auto\isanewline
\ \ \ \ \ \ \ \ \ \ \ \ \ \ \ \ \ \ \isacommand{also}\isamarkupfalse%
\ \isacommand{have}\isamarkupfalse%
\ {\isachardoublequoteopen}{\isasymdots}\ {\isacharequal}{\kern0pt}\ {\isadigit{1}}\isactrlsub m\ {\isadigit{4}}\ {\isachardollar}{\kern0pt}{\isachardollar}{\kern0pt}\ {\isacharparenleft}{\kern0pt}{\isadigit{2}}{\isacharcomma}{\kern0pt}{\isadigit{2}}{\isacharparenright}{\kern0pt}{\isachardoublequoteclose}\ \isacommand{by}\isamarkupfalse%
\ simp\isanewline
\ \ \ \ \ \ \ \ \ \ \ \ \ \ \ \ \ \ \isacommand{finally}\isamarkupfalse%
\ \isacommand{show}\isamarkupfalse%
\ {\isacharquery}{\kern0pt}thesis\ \isacommand{using}\isamarkupfalse%
\ i{\isadigit{2}}\ j{\isadigit{2}}\ \isacommand{by}\isamarkupfalse%
\ simp\isanewline
\ \ \ \ \ \ \ \ \ \ \ \ \ \ \ \ \isacommand{qed}\isamarkupfalse%
\isanewline
\ \ \ \ \ \ \ \ \ \ \ \ \ \ \isacommand{next}\isamarkupfalse%
\isanewline
\ \ \ \ \ \ \ \ \ \ \ \ \ \ \ \ \isacommand{assume}\isamarkupfalse%
\ j{\isadigit{3}}{\isacharcolon}{\kern0pt}{\isachardoublequoteopen}j\ {\isacharequal}{\kern0pt}\ {\isadigit{3}}{\isachardoublequoteclose}\isanewline
\ \ \ \ \ \ \ \ \ \ \ \ \ \ \ \ \isacommand{show}\isamarkupfalse%
\ {\isachardoublequoteopen}{\isacharparenleft}{\kern0pt}{\isacharparenleft}{\kern0pt}control{\isadigit{2}}\ U{\isacharparenright}{\kern0pt}\isactrlsup {\isasymdagger}\ {\isacharasterisk}{\kern0pt}\ control{\isadigit{2}}\ U{\isacharparenright}{\kern0pt}\ {\isachardollar}{\kern0pt}{\isachardollar}{\kern0pt}\ {\isacharparenleft}{\kern0pt}i{\isacharcomma}{\kern0pt}\ j{\isacharparenright}{\kern0pt}\ {\isacharequal}{\kern0pt}\ {\isadigit{1}}\isactrlsub m\ {\isadigit{4}}\ {\isachardollar}{\kern0pt}{\isachardollar}{\kern0pt}\ {\isacharparenleft}{\kern0pt}i{\isacharcomma}{\kern0pt}\ j{\isacharparenright}{\kern0pt}{\isachardoublequoteclose}\isanewline
\ \ \ \ \ \ \ \ \ \ \ \ \ \ \ \ \isacommand{proof}\isamarkupfalse%
\ {\isacharminus}{\kern0pt}\isanewline
\ \ \ \ \ \ \ \ \ \ \ \ \ \ \ \ \ \ \isacommand{have}\isamarkupfalse%
\ {\isachardoublequoteopen}{\isacharparenleft}{\kern0pt}{\isacharparenleft}{\kern0pt}control{\isadigit{2}}\ U{\isacharparenright}{\kern0pt}\isactrlsup {\isasymdagger}\ {\isacharasterisk}{\kern0pt}\ control{\isadigit{2}}\ U{\isacharparenright}{\kern0pt}\ {\isachardollar}{\kern0pt}{\isachardollar}{\kern0pt}\ {\isacharparenleft}{\kern0pt}{\isadigit{2}}{\isacharcomma}{\kern0pt}{\isadigit{3}}{\isacharparenright}{\kern0pt}\ {\isacharequal}{\kern0pt}\isanewline
\ \ \ \ \ \ \ \ \ \ \ \ \ \ \ \ \ \ \ \ \ \ \ \ {\isacharparenleft}{\kern0pt}{\isacharparenleft}{\kern0pt}control{\isadigit{2}}\ U{\isacharparenright}{\kern0pt}\isactrlsup {\isasymdagger}{\isacharparenright}{\kern0pt}\ {\isachardollar}{\kern0pt}{\isachardollar}{\kern0pt}\ {\isacharparenleft}{\kern0pt}{\isadigit{2}}{\isacharcomma}{\kern0pt}{\isadigit{0}}{\isacharparenright}{\kern0pt}\ {\isacharasterisk}{\kern0pt}\ {\isacharparenleft}{\kern0pt}control{\isadigit{2}}\ U{\isacharparenright}{\kern0pt}\ {\isachardollar}{\kern0pt}{\isachardollar}{\kern0pt}\ {\isacharparenleft}{\kern0pt}{\isadigit{0}}{\isacharcomma}{\kern0pt}{\isadigit{3}}{\isacharparenright}{\kern0pt}\ {\isacharplus}{\kern0pt}\isanewline
\ \ \ \ \ \ \ \ \ \ \ \ \ \ \ \ \ \ \ \ \ \ \ \ {\isacharparenleft}{\kern0pt}{\isacharparenleft}{\kern0pt}control{\isadigit{2}}\ U{\isacharparenright}{\kern0pt}\isactrlsup {\isasymdagger}{\isacharparenright}{\kern0pt}\ {\isachardollar}{\kern0pt}{\isachardollar}{\kern0pt}\ {\isacharparenleft}{\kern0pt}{\isadigit{2}}{\isacharcomma}{\kern0pt}{\isadigit{1}}{\isacharparenright}{\kern0pt}\ {\isacharasterisk}{\kern0pt}\ {\isacharparenleft}{\kern0pt}control{\isadigit{2}}\ U{\isacharparenright}{\kern0pt}\ {\isachardollar}{\kern0pt}{\isachardollar}{\kern0pt}\ {\isacharparenleft}{\kern0pt}{\isadigit{1}}{\isacharcomma}{\kern0pt}{\isadigit{3}}{\isacharparenright}{\kern0pt}\ {\isacharplus}{\kern0pt}\isanewline
\ \ \ \ \ \ \ \ \ \ \ \ \ \ \ \ \ \ \ \ \ \ \ \ {\isacharparenleft}{\kern0pt}{\isacharparenleft}{\kern0pt}control{\isadigit{2}}\ U{\isacharparenright}{\kern0pt}\isactrlsup {\isasymdagger}{\isacharparenright}{\kern0pt}\ {\isachardollar}{\kern0pt}{\isachardollar}{\kern0pt}\ {\isacharparenleft}{\kern0pt}{\isadigit{2}}{\isacharcomma}{\kern0pt}{\isadigit{2}}{\isacharparenright}{\kern0pt}\ {\isacharasterisk}{\kern0pt}\ {\isacharparenleft}{\kern0pt}control{\isadigit{2}}\ U{\isacharparenright}{\kern0pt}\ {\isachardollar}{\kern0pt}{\isachardollar}{\kern0pt}\ {\isacharparenleft}{\kern0pt}{\isadigit{2}}{\isacharcomma}{\kern0pt}{\isadigit{3}}{\isacharparenright}{\kern0pt}\ {\isacharplus}{\kern0pt}\isanewline
\ \ \ \ \ \ \ \ \ \ \ \ \ \ \ \ \ \ \ \ \ \ \ \ {\isacharparenleft}{\kern0pt}{\isacharparenleft}{\kern0pt}control{\isadigit{2}}\ U{\isacharparenright}{\kern0pt}\isactrlsup {\isasymdagger}{\isacharparenright}{\kern0pt}\ {\isachardollar}{\kern0pt}{\isachardollar}{\kern0pt}\ {\isacharparenleft}{\kern0pt}{\isadigit{2}}{\isacharcomma}{\kern0pt}{\isadigit{3}}{\isacharparenright}{\kern0pt}\ {\isacharasterisk}{\kern0pt}\ {\isacharparenleft}{\kern0pt}control{\isadigit{2}}\ U{\isacharparenright}{\kern0pt}\ {\isachardollar}{\kern0pt}{\isachardollar}{\kern0pt}\ {\isacharparenleft}{\kern0pt}{\isadigit{3}}{\isacharcomma}{\kern0pt}{\isadigit{3}}{\isacharparenright}{\kern0pt}{\isachardoublequoteclose}\isanewline
\ \ \ \ \ \ \ \ \ \ \ \ \ \ \ \ \ \ \ \ \isacommand{using}\isamarkupfalse%
\ sumof{\isadigit{4}}\isanewline
\ \ \ \ \ \ \ \ \ \ \ \ \ \ \ \ \ \ \ \ \isacommand{by}\isamarkupfalse%
\ {\isacharparenleft}{\kern0pt}metis\ {\isacharparenleft}{\kern0pt}no{\isacharunderscore}{\kern0pt}types{\isacharcomma}{\kern0pt}\ lifting{\isacharparenright}{\kern0pt}\ carrier{\isacharunderscore}{\kern0pt}matD{\isacharparenleft}{\kern0pt}{\isadigit{1}}{\isacharparenright}{\kern0pt}\ carrier{\isacharunderscore}{\kern0pt}matD{\isacharparenleft}{\kern0pt}{\isadigit{2}}{\isacharparenright}{\kern0pt}\ \isanewline
\ \ \ \ \ \ \ \ \ \ \ \ \ \ \ \ \ \ \ \ \ \ \ \ control{\isadigit{2}}{\isacharunderscore}{\kern0pt}carrier{\isacharunderscore}{\kern0pt}mat\ dim{\isacharunderscore}{\kern0pt}col{\isacharunderscore}{\kern0pt}of{\isacharunderscore}{\kern0pt}dagger\ dim{\isacharunderscore}{\kern0pt}row{\isacharunderscore}{\kern0pt}of{\isacharunderscore}{\kern0pt}dagger\ i{\isadigit{2}}\ i{\isadigit{4}}\ \isanewline
\ \ \ \ \ \ \ \ \ \ \ \ \ \ \ \ \ \ \ \ \ \ \ \ index{\isacharunderscore}{\kern0pt}matrix{\isacharunderscore}{\kern0pt}prod\ j{\isadigit{3}}\ j{\isadigit{4}}{\isacharparenright}{\kern0pt}\isanewline
\ \ \ \ \ \ \ \ \ \ \ \ \ \ \ \ \ \ \isacommand{also}\isamarkupfalse%
\ \isacommand{have}\isamarkupfalse%
\ {\isachardoublequoteopen}{\isasymdots}\ {\isacharequal}{\kern0pt}\ {\isacharparenleft}{\kern0pt}{\isacharparenleft}{\kern0pt}control{\isadigit{2}}\ U{\isacharparenright}{\kern0pt}\isactrlsup {\isasymdagger}{\isacharparenright}{\kern0pt}\ {\isachardollar}{\kern0pt}{\isachardollar}{\kern0pt}\ {\isacharparenleft}{\kern0pt}{\isadigit{2}}{\isacharcomma}{\kern0pt}{\isadigit{1}}{\isacharparenright}{\kern0pt}\ {\isacharasterisk}{\kern0pt}\ {\isacharparenleft}{\kern0pt}control{\isadigit{2}}\ U{\isacharparenright}{\kern0pt}\ {\isachardollar}{\kern0pt}{\isachardollar}{\kern0pt}\ {\isacharparenleft}{\kern0pt}{\isadigit{1}}{\isacharcomma}{\kern0pt}{\isadigit{3}}{\isacharparenright}{\kern0pt}\ {\isacharplus}{\kern0pt}\isanewline
\ \ \ \ \ \ \ \ \ \ \ \ \ \ \ \ \ \ \ \ \ \ \ \ \ \ \ \ \ \ \ \ \ \ {\isacharparenleft}{\kern0pt}{\isacharparenleft}{\kern0pt}control{\isadigit{2}}\ U{\isacharparenright}{\kern0pt}\isactrlsup {\isasymdagger}{\isacharparenright}{\kern0pt}\ {\isachardollar}{\kern0pt}{\isachardollar}{\kern0pt}\ {\isacharparenleft}{\kern0pt}{\isadigit{2}}{\isacharcomma}{\kern0pt}{\isadigit{3}}{\isacharparenright}{\kern0pt}\ {\isacharasterisk}{\kern0pt}\ {\isacharparenleft}{\kern0pt}control{\isadigit{2}}\ U{\isacharparenright}{\kern0pt}\ {\isachardollar}{\kern0pt}{\isachardollar}{\kern0pt}\ {\isacharparenleft}{\kern0pt}{\isadigit{3}}{\isacharcomma}{\kern0pt}{\isadigit{3}}{\isacharparenright}{\kern0pt}{\isachardoublequoteclose}\isanewline
\ \ \ \ \ \ \ \ \ \ \ \ \ \ \ \ \ \ \ \ \isacommand{using}\isamarkupfalse%
\ control{\isadigit{2}}{\isacharunderscore}{\kern0pt}def\ index{\isacharunderscore}{\kern0pt}mat{\isacharunderscore}{\kern0pt}of{\isacharunderscore}{\kern0pt}cols{\isacharunderscore}{\kern0pt}list\ \isacommand{by}\isamarkupfalse%
\ force\isanewline
\ \ \ \ \ \ \ \ \ \ \ \ \ \ \ \ \ \ \isacommand{also}\isamarkupfalse%
\ \isacommand{have}\isamarkupfalse%
\ {\isachardoublequoteopen}{\isasymdots}\ {\isacharequal}{\kern0pt}\ cnj\ {\isacharparenleft}{\kern0pt}{\isacharparenleft}{\kern0pt}control{\isadigit{2}}\ U{\isacharparenright}{\kern0pt}\ {\isachardollar}{\kern0pt}{\isachardollar}{\kern0pt}\ {\isacharparenleft}{\kern0pt}{\isadigit{1}}{\isacharcomma}{\kern0pt}{\isadigit{2}}{\isacharparenright}{\kern0pt}{\isacharparenright}{\kern0pt}\ {\isacharasterisk}{\kern0pt}\ {\isacharparenleft}{\kern0pt}control{\isadigit{2}}\ U{\isacharparenright}{\kern0pt}\ {\isachardollar}{\kern0pt}{\isachardollar}{\kern0pt}\ {\isacharparenleft}{\kern0pt}{\isadigit{1}}{\isacharcomma}{\kern0pt}{\isadigit{3}}{\isacharparenright}{\kern0pt}\ {\isacharplus}{\kern0pt}\isanewline
\ \ \ \ \ \ \ \ \ \ \ \ \ \ \ \ \ \ \ \ \ \ \ \ \ \ \ \ \ \ \ \ \ \ cnj\ {\isacharparenleft}{\kern0pt}{\isacharparenleft}{\kern0pt}control{\isadigit{2}}\ U{\isacharparenright}{\kern0pt}\ {\isachardollar}{\kern0pt}{\isachardollar}{\kern0pt}\ {\isacharparenleft}{\kern0pt}{\isadigit{3}}{\isacharcomma}{\kern0pt}{\isadigit{2}}{\isacharparenright}{\kern0pt}{\isacharparenright}{\kern0pt}\ {\isacharasterisk}{\kern0pt}\ {\isacharparenleft}{\kern0pt}control{\isadigit{2}}\ U{\isacharparenright}{\kern0pt}\ {\isachardollar}{\kern0pt}{\isachardollar}{\kern0pt}\ {\isacharparenleft}{\kern0pt}{\isadigit{3}}{\isacharcomma}{\kern0pt}{\isadigit{3}}{\isacharparenright}{\kern0pt}{\isachardoublequoteclose}\isanewline
\ \ \ \ \ \ \ \ \ \ \ \ \ \ \ \ \ \ \ \ \isacommand{using}\isamarkupfalse%
\ dagger{\isacharunderscore}{\kern0pt}def\isanewline
\ \ \ \ \ \ \ \ \ \ \ \ \ \ \ \ \ \ \ \ \isacommand{by}\isamarkupfalse%
\ {\isacharparenleft}{\kern0pt}simp\ add{\isacharcolon}{\kern0pt}\ Tensor{\isachardot}{\kern0pt}mat{\isacharunderscore}{\kern0pt}of{\isacharunderscore}{\kern0pt}cols{\isacharunderscore}{\kern0pt}list{\isacharunderscore}{\kern0pt}def\ control{\isadigit{2}}{\isacharunderscore}{\kern0pt}def{\isacharparenright}{\kern0pt}\isanewline
\ \ \ \ \ \ \ \ \ \ \ \ \ \ \ \ \ \ \isacommand{also}\isamarkupfalse%
\ \isacommand{have}\isamarkupfalse%
\ {\isachardoublequoteopen}{\isasymdots}\ {\isacharequal}{\kern0pt}\ {\isadigit{0}}{\isachardoublequoteclose}\ \isacommand{using}\isamarkupfalse%
\ control{\isadigit{2}}{\isacharunderscore}{\kern0pt}def\ index{\isacharunderscore}{\kern0pt}mat{\isacharunderscore}{\kern0pt}of{\isacharunderscore}{\kern0pt}cols{\isacharunderscore}{\kern0pt}list\ \isacommand{by}\isamarkupfalse%
\ auto\isanewline
\ \ \ \ \ \ \ \ \ \ \ \ \ \ \ \ \ \ \isacommand{also}\isamarkupfalse%
\ \isacommand{have}\isamarkupfalse%
\ {\isachardoublequoteopen}{\isasymdots}\ {\isacharequal}{\kern0pt}\ {\isadigit{1}}\isactrlsub m\ {\isadigit{4}}\ {\isachardollar}{\kern0pt}{\isachardollar}{\kern0pt}\ {\isacharparenleft}{\kern0pt}{\isadigit{2}}{\isacharcomma}{\kern0pt}{\isadigit{3}}{\isacharparenright}{\kern0pt}{\isachardoublequoteclose}\ \isacommand{by}\isamarkupfalse%
\ simp\isanewline
\ \ \ \ \ \ \ \ \ \ \ \ \ \ \ \ \ \ \isacommand{finally}\isamarkupfalse%
\ \isacommand{show}\isamarkupfalse%
\ {\isacharquery}{\kern0pt}thesis\ \isacommand{using}\isamarkupfalse%
\ i{\isadigit{2}}\ j{\isadigit{3}}\ \isacommand{by}\isamarkupfalse%
\ simp\isanewline
\ \ \ \ \ \ \ \ \ \ \ \ \ \ \ \ \isacommand{qed}\isamarkupfalse%
\isanewline
\ \ \ \ \ \ \ \ \ \ \ \ \ \ \isacommand{qed}\isamarkupfalse%
\isanewline
\ \ \ \ \ \ \ \ \ \ \ \ \isacommand{qed}\isamarkupfalse%
\isanewline
\ \ \ \ \ \ \ \ \ \ \isacommand{qed}\isamarkupfalse%
\isanewline
\ \ \ \ \ \ \ \ \isacommand{next}\isamarkupfalse%
\isanewline
\ \ \ \ \ \ \ \ \ \ \isacommand{assume}\isamarkupfalse%
\ i{\isadigit{3}}{\isacharcolon}{\kern0pt}{\isachardoublequoteopen}i\ {\isacharequal}{\kern0pt}\ {\isadigit{3}}{\isachardoublequoteclose}\isanewline
\ \ \ \ \ \ \ \ \ \ \isacommand{show}\isamarkupfalse%
\ {\isachardoublequoteopen}{\isacharparenleft}{\kern0pt}{\isacharparenleft}{\kern0pt}control{\isadigit{2}}\ U{\isacharparenright}{\kern0pt}\isactrlsup {\isasymdagger}\ {\isacharasterisk}{\kern0pt}\ control{\isadigit{2}}\ U{\isacharparenright}{\kern0pt}\ {\isachardollar}{\kern0pt}{\isachardollar}{\kern0pt}\ {\isacharparenleft}{\kern0pt}i{\isacharcomma}{\kern0pt}\ j{\isacharparenright}{\kern0pt}\ {\isacharequal}{\kern0pt}\ {\isadigit{1}}\isactrlsub m\ {\isadigit{4}}\ {\isachardollar}{\kern0pt}{\isachardollar}{\kern0pt}\ {\isacharparenleft}{\kern0pt}i{\isacharcomma}{\kern0pt}\ j{\isacharparenright}{\kern0pt}{\isachardoublequoteclose}\isanewline
\ \ \ \ \ \ \ \ \ \ \isacommand{proof}\isamarkupfalse%
\ {\isacharparenleft}{\kern0pt}rule\ disjE{\isacharparenright}{\kern0pt}\isanewline
\ \ \ \ \ \ \ \ \ \ \ \ \isacommand{show}\isamarkupfalse%
\ {\isachardoublequoteopen}j\ {\isacharequal}{\kern0pt}\ {\isadigit{0}}\ {\isasymor}\ j\ {\isacharequal}{\kern0pt}\ {\isadigit{1}}\ {\isasymor}\ j\ {\isacharequal}{\kern0pt}\ {\isadigit{2}}\ {\isasymor}\ j\ {\isacharequal}{\kern0pt}\ {\isadigit{3}}{\isachardoublequoteclose}\ \isacommand{using}\isamarkupfalse%
\ j{\isadigit{4}}\ \isacommand{by}\isamarkupfalse%
\ auto\isanewline
\ \ \ \ \ \ \ \ \ \ \isacommand{next}\isamarkupfalse%
\isanewline
\ \ \ \ \ \ \ \ \ \ \ \ \isacommand{assume}\isamarkupfalse%
\ j{\isadigit{0}}{\isacharcolon}{\kern0pt}{\isachardoublequoteopen}j\ {\isacharequal}{\kern0pt}\ {\isadigit{0}}{\isachardoublequoteclose}\isanewline
\ \ \ \ \ \ \ \ \ \ \ \ \isacommand{show}\isamarkupfalse%
\ {\isachardoublequoteopen}{\isacharparenleft}{\kern0pt}{\isacharparenleft}{\kern0pt}control{\isadigit{2}}\ U{\isacharparenright}{\kern0pt}\isactrlsup {\isasymdagger}\ {\isacharasterisk}{\kern0pt}\ control{\isadigit{2}}\ U{\isacharparenright}{\kern0pt}\ {\isachardollar}{\kern0pt}{\isachardollar}{\kern0pt}\ {\isacharparenleft}{\kern0pt}i{\isacharcomma}{\kern0pt}\ j{\isacharparenright}{\kern0pt}\ {\isacharequal}{\kern0pt}\ {\isadigit{1}}\isactrlsub m\ {\isadigit{4}}\ {\isachardollar}{\kern0pt}{\isachardollar}{\kern0pt}\ {\isacharparenleft}{\kern0pt}i{\isacharcomma}{\kern0pt}\ j{\isacharparenright}{\kern0pt}{\isachardoublequoteclose}\isanewline
\ \ \ \ \ \ \ \ \ \ \ \ \isacommand{proof}\isamarkupfalse%
\ {\isacharminus}{\kern0pt}\isanewline
\ \ \ \ \ \ \ \ \ \ \ \ \ \ \isacommand{have}\isamarkupfalse%
\ {\isachardoublequoteopen}{\isacharparenleft}{\kern0pt}{\isacharparenleft}{\kern0pt}control{\isadigit{2}}\ U{\isacharparenright}{\kern0pt}\isactrlsup {\isasymdagger}\ {\isacharasterisk}{\kern0pt}\ control{\isadigit{2}}\ U{\isacharparenright}{\kern0pt}\ {\isachardollar}{\kern0pt}{\isachardollar}{\kern0pt}\ {\isacharparenleft}{\kern0pt}{\isadigit{3}}{\isacharcomma}{\kern0pt}{\isadigit{0}}{\isacharparenright}{\kern0pt}\ {\isacharequal}{\kern0pt}\isanewline
\ \ \ \ \ \ \ \ \ \ \ \ \ \ \ \ {\isacharparenleft}{\kern0pt}{\isacharparenleft}{\kern0pt}control{\isadigit{2}}\ U{\isacharparenright}{\kern0pt}\isactrlsup {\isasymdagger}{\isacharparenright}{\kern0pt}\ {\isachardollar}{\kern0pt}{\isachardollar}{\kern0pt}\ {\isacharparenleft}{\kern0pt}{\isadigit{3}}{\isacharcomma}{\kern0pt}{\isadigit{0}}{\isacharparenright}{\kern0pt}\ {\isacharasterisk}{\kern0pt}\ {\isacharparenleft}{\kern0pt}control{\isadigit{2}}\ U{\isacharparenright}{\kern0pt}\ {\isachardollar}{\kern0pt}{\isachardollar}{\kern0pt}\ {\isacharparenleft}{\kern0pt}{\isadigit{0}}{\isacharcomma}{\kern0pt}{\isadigit{0}}{\isacharparenright}{\kern0pt}\ {\isacharplus}{\kern0pt}\isanewline
\ \ \ \ \ \ \ \ \ \ \ \ \ \ \ \ {\isacharparenleft}{\kern0pt}{\isacharparenleft}{\kern0pt}control{\isadigit{2}}\ U{\isacharparenright}{\kern0pt}\isactrlsup {\isasymdagger}{\isacharparenright}{\kern0pt}\ {\isachardollar}{\kern0pt}{\isachardollar}{\kern0pt}\ {\isacharparenleft}{\kern0pt}{\isadigit{3}}{\isacharcomma}{\kern0pt}{\isadigit{1}}{\isacharparenright}{\kern0pt}\ {\isacharasterisk}{\kern0pt}\ {\isacharparenleft}{\kern0pt}control{\isadigit{2}}\ U{\isacharparenright}{\kern0pt}\ {\isachardollar}{\kern0pt}{\isachardollar}{\kern0pt}\ {\isacharparenleft}{\kern0pt}{\isadigit{1}}{\isacharcomma}{\kern0pt}{\isadigit{0}}{\isacharparenright}{\kern0pt}\ {\isacharplus}{\kern0pt}\isanewline
\ \ \ \ \ \ \ \ \ \ \ \ \ \ \ \ {\isacharparenleft}{\kern0pt}{\isacharparenleft}{\kern0pt}control{\isadigit{2}}\ U{\isacharparenright}{\kern0pt}\isactrlsup {\isasymdagger}{\isacharparenright}{\kern0pt}\ {\isachardollar}{\kern0pt}{\isachardollar}{\kern0pt}\ {\isacharparenleft}{\kern0pt}{\isadigit{3}}{\isacharcomma}{\kern0pt}{\isadigit{2}}{\isacharparenright}{\kern0pt}\ {\isacharasterisk}{\kern0pt}\ {\isacharparenleft}{\kern0pt}control{\isadigit{2}}\ U{\isacharparenright}{\kern0pt}\ {\isachardollar}{\kern0pt}{\isachardollar}{\kern0pt}\ {\isacharparenleft}{\kern0pt}{\isadigit{2}}{\isacharcomma}{\kern0pt}{\isadigit{0}}{\isacharparenright}{\kern0pt}\ {\isacharplus}{\kern0pt}\isanewline
\ \ \ \ \ \ \ \ \ \ \ \ \ \ \ \ {\isacharparenleft}{\kern0pt}{\isacharparenleft}{\kern0pt}control{\isadigit{2}}\ U{\isacharparenright}{\kern0pt}\isactrlsup {\isasymdagger}{\isacharparenright}{\kern0pt}\ {\isachardollar}{\kern0pt}{\isachardollar}{\kern0pt}\ {\isacharparenleft}{\kern0pt}{\isadigit{3}}{\isacharcomma}{\kern0pt}{\isadigit{3}}{\isacharparenright}{\kern0pt}\ {\isacharasterisk}{\kern0pt}\ {\isacharparenleft}{\kern0pt}control{\isadigit{2}}\ U{\isacharparenright}{\kern0pt}\ {\isachardollar}{\kern0pt}{\isachardollar}{\kern0pt}\ {\isacharparenleft}{\kern0pt}{\isadigit{3}}{\isacharcomma}{\kern0pt}{\isadigit{0}}{\isacharparenright}{\kern0pt}{\isachardoublequoteclose}\isanewline
\ \ \ \ \ \ \ \ \ \ \ \ \ \ \ \ \isacommand{using}\isamarkupfalse%
\ sumof{\isadigit{4}}\isanewline
\ \ \ \ \ \ \ \ \ \ \ \ \ \ \ \ \isacommand{by}\isamarkupfalse%
\ {\isacharparenleft}{\kern0pt}metis\ {\isacharparenleft}{\kern0pt}no{\isacharunderscore}{\kern0pt}types{\isacharcomma}{\kern0pt}\ lifting{\isacharparenright}{\kern0pt}\ carrier{\isacharunderscore}{\kern0pt}matD{\isacharparenleft}{\kern0pt}{\isadigit{1}}{\isacharparenright}{\kern0pt}\ carrier{\isacharunderscore}{\kern0pt}matD{\isacharparenleft}{\kern0pt}{\isadigit{2}}{\isacharparenright}{\kern0pt}\ control{\isadigit{2}}{\isacharunderscore}{\kern0pt}carrier{\isacharunderscore}{\kern0pt}mat\ \isanewline
\ \ \ \ \ \ \ \ \ \ \ \ \ \ \ \ \ \ \ \ dim{\isacharunderscore}{\kern0pt}col{\isacharunderscore}{\kern0pt}of{\isacharunderscore}{\kern0pt}dagger\ dim{\isacharunderscore}{\kern0pt}row{\isacharunderscore}{\kern0pt}of{\isacharunderscore}{\kern0pt}dagger\ i{\isadigit{3}}\ i{\isadigit{4}}\ index{\isacharunderscore}{\kern0pt}matrix{\isacharunderscore}{\kern0pt}prod\ j{\isadigit{0}}\ j{\isadigit{4}}{\isacharparenright}{\kern0pt}\isanewline
\ \ \ \ \ \ \ \ \ \ \ \ \ \ \isacommand{also}\isamarkupfalse%
\ \isacommand{have}\isamarkupfalse%
\ {\isachardoublequoteopen}{\isasymdots}\ {\isacharequal}{\kern0pt}\ {\isacharparenleft}{\kern0pt}{\isacharparenleft}{\kern0pt}control{\isadigit{2}}\ U{\isacharparenright}{\kern0pt}\isactrlsup {\isasymdagger}{\isacharparenright}{\kern0pt}\ {\isachardollar}{\kern0pt}{\isachardollar}{\kern0pt}\ {\isacharparenleft}{\kern0pt}{\isadigit{3}}{\isacharcomma}{\kern0pt}{\isadigit{0}}{\isacharparenright}{\kern0pt}{\isachardoublequoteclose}\isanewline
\ \ \ \ \ \ \ \ \ \ \ \ \ \ \ \ \isacommand{using}\isamarkupfalse%
\ control{\isadigit{2}}{\isacharunderscore}{\kern0pt}def\ index{\isacharunderscore}{\kern0pt}mat{\isacharunderscore}{\kern0pt}of{\isacharunderscore}{\kern0pt}cols{\isacharunderscore}{\kern0pt}list\ \isacommand{by}\isamarkupfalse%
\ force\isanewline
\ \ \ \ \ \ \ \ \ \ \ \ \ \ \isacommand{also}\isamarkupfalse%
\ \isacommand{have}\isamarkupfalse%
\ {\isachardoublequoteopen}{\isasymdots}\ {\isacharequal}{\kern0pt}\ cnj\ {\isacharparenleft}{\kern0pt}{\isacharparenleft}{\kern0pt}control{\isadigit{2}}\ U{\isacharparenright}{\kern0pt}\ {\isachardollar}{\kern0pt}{\isachardollar}{\kern0pt}\ {\isacharparenleft}{\kern0pt}{\isadigit{0}}{\isacharcomma}{\kern0pt}{\isadigit{3}}{\isacharparenright}{\kern0pt}{\isacharparenright}{\kern0pt}{\isachardoublequoteclose}\isanewline
\ \ \ \ \ \ \ \ \ \ \ \ \ \ \ \ \isacommand{using}\isamarkupfalse%
\ dagger{\isacharunderscore}{\kern0pt}def\isanewline
\ \ \ \ \ \ \ \ \ \ \ \ \ \ \ \ \isacommand{by}\isamarkupfalse%
\ {\isacharparenleft}{\kern0pt}simp\ add{\isacharcolon}{\kern0pt}\ Tensor{\isachardot}{\kern0pt}mat{\isacharunderscore}{\kern0pt}of{\isacharunderscore}{\kern0pt}cols{\isacharunderscore}{\kern0pt}list{\isacharunderscore}{\kern0pt}def\ control{\isadigit{2}}{\isacharunderscore}{\kern0pt}def{\isacharparenright}{\kern0pt}\isanewline
\ \ \ \ \ \ \ \ \ \ \ \ \ \ \isacommand{also}\isamarkupfalse%
\ \isacommand{have}\isamarkupfalse%
\ {\isachardoublequoteopen}{\isasymdots}\ {\isacharequal}{\kern0pt}\ {\isadigit{0}}{\isachardoublequoteclose}\ \isacommand{using}\isamarkupfalse%
\ control{\isadigit{2}}{\isacharunderscore}{\kern0pt}def\ index{\isacharunderscore}{\kern0pt}mat{\isacharunderscore}{\kern0pt}of{\isacharunderscore}{\kern0pt}cols{\isacharunderscore}{\kern0pt}list\ \isacommand{by}\isamarkupfalse%
\ auto\isanewline
\ \ \ \ \ \ \ \ \ \ \ \ \ \ \isacommand{also}\isamarkupfalse%
\ \isacommand{have}\isamarkupfalse%
\ {\isachardoublequoteopen}{\isasymdots}\ {\isacharequal}{\kern0pt}\ {\isadigit{1}}\isactrlsub m\ {\isadigit{4}}\ {\isachardollar}{\kern0pt}{\isachardollar}{\kern0pt}\ {\isacharparenleft}{\kern0pt}{\isadigit{3}}{\isacharcomma}{\kern0pt}{\isadigit{0}}{\isacharparenright}{\kern0pt}{\isachardoublequoteclose}\ \isacommand{by}\isamarkupfalse%
\ simp\isanewline
\ \ \ \ \ \ \ \ \ \ \ \ \ \ \isacommand{finally}\isamarkupfalse%
\ \isacommand{show}\isamarkupfalse%
\ {\isacharquery}{\kern0pt}thesis\ \isacommand{using}\isamarkupfalse%
\ i{\isadigit{3}}\ j{\isadigit{0}}\ \isacommand{by}\isamarkupfalse%
\ simp\isanewline
\ \ \ \ \ \ \ \ \ \ \ \ \isacommand{qed}\isamarkupfalse%
\isanewline
\ \ \ \ \ \ \ \ \ \ \isacommand{next}\isamarkupfalse%
\isanewline
\ \ \ \ \ \ \ \ \ \ \ \ \isacommand{assume}\isamarkupfalse%
\ jl{\isadigit{3}}{\isacharcolon}{\kern0pt}{\isachardoublequoteopen}j\ {\isacharequal}{\kern0pt}\ {\isadigit{1}}\ {\isasymor}\ j\ {\isacharequal}{\kern0pt}\ {\isadigit{2}}\ {\isasymor}\ j\ {\isacharequal}{\kern0pt}\ {\isadigit{3}}{\isachardoublequoteclose}\isanewline
\ \ \ \ \ \ \ \ \ \ \ \ \isacommand{show}\isamarkupfalse%
\ {\isachardoublequoteopen}{\isacharparenleft}{\kern0pt}{\isacharparenleft}{\kern0pt}control{\isadigit{2}}\ U{\isacharparenright}{\kern0pt}\isactrlsup {\isasymdagger}\ {\isacharasterisk}{\kern0pt}\ control{\isadigit{2}}\ U{\isacharparenright}{\kern0pt}\ {\isachardollar}{\kern0pt}{\isachardollar}{\kern0pt}\ {\isacharparenleft}{\kern0pt}i{\isacharcomma}{\kern0pt}\ j{\isacharparenright}{\kern0pt}\ {\isacharequal}{\kern0pt}\ {\isadigit{1}}\isactrlsub m\ {\isadigit{4}}\ {\isachardollar}{\kern0pt}{\isachardollar}{\kern0pt}\ {\isacharparenleft}{\kern0pt}i{\isacharcomma}{\kern0pt}\ j{\isacharparenright}{\kern0pt}{\isachardoublequoteclose}\isanewline
\ \ \ \ \ \ \ \ \ \ \ \ \isacommand{proof}\isamarkupfalse%
\ {\isacharparenleft}{\kern0pt}rule\ disjE{\isacharparenright}{\kern0pt}\isanewline
\ \ \ \ \ \ \ \ \ \ \ \ \ \ \isacommand{show}\isamarkupfalse%
\ {\isachardoublequoteopen}j\ {\isacharequal}{\kern0pt}\ {\isadigit{1}}\ {\isasymor}\ j\ {\isacharequal}{\kern0pt}\ {\isadigit{2}}\ {\isasymor}\ j\ {\isacharequal}{\kern0pt}\ {\isadigit{3}}{\isachardoublequoteclose}\ \isacommand{using}\isamarkupfalse%
\ jl{\isadigit{3}}\ \isacommand{by}\isamarkupfalse%
\ this\isanewline
\ \ \ \ \ \ \ \ \ \ \ \ \isacommand{next}\isamarkupfalse%
\isanewline
\ \ \ \ \ \ \ \ \ \ \ \ \ \ \isacommand{assume}\isamarkupfalse%
\ j{\isadigit{1}}{\isacharcolon}{\kern0pt}{\isachardoublequoteopen}j\ {\isacharequal}{\kern0pt}\ {\isadigit{1}}{\isachardoublequoteclose}\isanewline
\ \ \ \ \ \ \ \ \ \ \ \ \ \ \isacommand{show}\isamarkupfalse%
\ {\isachardoublequoteopen}{\isacharparenleft}{\kern0pt}{\isacharparenleft}{\kern0pt}control{\isadigit{2}}\ U{\isacharparenright}{\kern0pt}\isactrlsup {\isasymdagger}\ {\isacharasterisk}{\kern0pt}\ control{\isadigit{2}}\ U{\isacharparenright}{\kern0pt}\ {\isachardollar}{\kern0pt}{\isachardollar}{\kern0pt}\ {\isacharparenleft}{\kern0pt}i{\isacharcomma}{\kern0pt}\ j{\isacharparenright}{\kern0pt}\ {\isacharequal}{\kern0pt}\ {\isadigit{1}}\isactrlsub m\ {\isadigit{4}}\ {\isachardollar}{\kern0pt}{\isachardollar}{\kern0pt}\ {\isacharparenleft}{\kern0pt}i{\isacharcomma}{\kern0pt}\ j{\isacharparenright}{\kern0pt}{\isachardoublequoteclose}\isanewline
\ \ \ \ \ \ \ \ \ \ \ \ \ \ \isacommand{proof}\isamarkupfalse%
\ {\isacharminus}{\kern0pt}\isanewline
\ \ \ \ \ \ \ \ \ \ \ \ \ \ \ \ \isacommand{have}\isamarkupfalse%
\ {\isachardoublequoteopen}{\isacharparenleft}{\kern0pt}{\isacharparenleft}{\kern0pt}control{\isadigit{2}}\ U{\isacharparenright}{\kern0pt}\isactrlsup {\isasymdagger}\ {\isacharasterisk}{\kern0pt}\ control{\isadigit{2}}\ U{\isacharparenright}{\kern0pt}\ {\isachardollar}{\kern0pt}{\isachardollar}{\kern0pt}\ {\isacharparenleft}{\kern0pt}{\isadigit{3}}{\isacharcomma}{\kern0pt}{\isadigit{1}}{\isacharparenright}{\kern0pt}\ {\isacharequal}{\kern0pt}\isanewline
\ \ \ \ \ \ \ \ \ \ \ \ \ \ \ \ {\isacharparenleft}{\kern0pt}{\isacharparenleft}{\kern0pt}control{\isadigit{2}}\ U{\isacharparenright}{\kern0pt}\isactrlsup {\isasymdagger}{\isacharparenright}{\kern0pt}\ {\isachardollar}{\kern0pt}{\isachardollar}{\kern0pt}\ {\isacharparenleft}{\kern0pt}{\isadigit{3}}{\isacharcomma}{\kern0pt}{\isadigit{0}}{\isacharparenright}{\kern0pt}\ {\isacharasterisk}{\kern0pt}\ {\isacharparenleft}{\kern0pt}control{\isadigit{2}}\ U{\isacharparenright}{\kern0pt}\ {\isachardollar}{\kern0pt}{\isachardollar}{\kern0pt}\ {\isacharparenleft}{\kern0pt}{\isadigit{0}}{\isacharcomma}{\kern0pt}{\isadigit{1}}{\isacharparenright}{\kern0pt}\ {\isacharplus}{\kern0pt}\isanewline
\ \ \ \ \ \ \ \ \ \ \ \ \ \ \ \ {\isacharparenleft}{\kern0pt}{\isacharparenleft}{\kern0pt}control{\isadigit{2}}\ U{\isacharparenright}{\kern0pt}\isactrlsup {\isasymdagger}{\isacharparenright}{\kern0pt}\ {\isachardollar}{\kern0pt}{\isachardollar}{\kern0pt}\ {\isacharparenleft}{\kern0pt}{\isadigit{3}}{\isacharcomma}{\kern0pt}{\isadigit{1}}{\isacharparenright}{\kern0pt}\ {\isacharasterisk}{\kern0pt}\ {\isacharparenleft}{\kern0pt}control{\isadigit{2}}\ U{\isacharparenright}{\kern0pt}\ {\isachardollar}{\kern0pt}{\isachardollar}{\kern0pt}\ {\isacharparenleft}{\kern0pt}{\isadigit{1}}{\isacharcomma}{\kern0pt}{\isadigit{1}}{\isacharparenright}{\kern0pt}\ {\isacharplus}{\kern0pt}\isanewline
\ \ \ \ \ \ \ \ \ \ \ \ \ \ \ \ {\isacharparenleft}{\kern0pt}{\isacharparenleft}{\kern0pt}control{\isadigit{2}}\ U{\isacharparenright}{\kern0pt}\isactrlsup {\isasymdagger}{\isacharparenright}{\kern0pt}\ {\isachardollar}{\kern0pt}{\isachardollar}{\kern0pt}\ {\isacharparenleft}{\kern0pt}{\isadigit{3}}{\isacharcomma}{\kern0pt}{\isadigit{2}}{\isacharparenright}{\kern0pt}\ {\isacharasterisk}{\kern0pt}\ {\isacharparenleft}{\kern0pt}control{\isadigit{2}}\ U{\isacharparenright}{\kern0pt}\ {\isachardollar}{\kern0pt}{\isachardollar}{\kern0pt}\ {\isacharparenleft}{\kern0pt}{\isadigit{2}}{\isacharcomma}{\kern0pt}{\isadigit{1}}{\isacharparenright}{\kern0pt}\ {\isacharplus}{\kern0pt}\isanewline
\ \ \ \ \ \ \ \ \ \ \ \ \ \ \ \ {\isacharparenleft}{\kern0pt}{\isacharparenleft}{\kern0pt}control{\isadigit{2}}\ U{\isacharparenright}{\kern0pt}\isactrlsup {\isasymdagger}{\isacharparenright}{\kern0pt}\ {\isachardollar}{\kern0pt}{\isachardollar}{\kern0pt}\ {\isacharparenleft}{\kern0pt}{\isadigit{3}}{\isacharcomma}{\kern0pt}{\isadigit{3}}{\isacharparenright}{\kern0pt}\ {\isacharasterisk}{\kern0pt}\ {\isacharparenleft}{\kern0pt}control{\isadigit{2}}\ U{\isacharparenright}{\kern0pt}\ {\isachardollar}{\kern0pt}{\isachardollar}{\kern0pt}\ {\isacharparenleft}{\kern0pt}{\isadigit{3}}{\isacharcomma}{\kern0pt}{\isadigit{1}}{\isacharparenright}{\kern0pt}{\isachardoublequoteclose}\isanewline
\ \ \ \ \ \ \ \ \ \ \ \ \ \ \ \ \ \ \isacommand{using}\isamarkupfalse%
\ sumof{\isadigit{4}}\isanewline
\ \ \ \ \ \ \ \ \ \ \ \ \ \ \ \ \ \ \isacommand{by}\isamarkupfalse%
\ {\isacharparenleft}{\kern0pt}metis\ {\isacharparenleft}{\kern0pt}no{\isacharunderscore}{\kern0pt}types{\isacharcomma}{\kern0pt}\ lifting{\isacharparenright}{\kern0pt}\ carrier{\isacharunderscore}{\kern0pt}matD{\isacharparenleft}{\kern0pt}{\isadigit{1}}{\isacharparenright}{\kern0pt}\ carrier{\isacharunderscore}{\kern0pt}matD{\isacharparenleft}{\kern0pt}{\isadigit{2}}{\isacharparenright}{\kern0pt}\ \isanewline
\ \ \ \ \ \ \ \ \ \ \ \ \ \ \ \ \ \ \ \ \ \ control{\isadigit{2}}{\isacharunderscore}{\kern0pt}carrier{\isacharunderscore}{\kern0pt}mat\ dim{\isacharunderscore}{\kern0pt}col{\isacharunderscore}{\kern0pt}of{\isacharunderscore}{\kern0pt}dagger\ dim{\isacharunderscore}{\kern0pt}row{\isacharunderscore}{\kern0pt}of{\isacharunderscore}{\kern0pt}dagger\ i{\isadigit{3}}\ i{\isadigit{4}}\ \isanewline
\ \ \ \ \ \ \ \ \ \ \ \ \ \ \ \ \ \ \ \ \ \ index{\isacharunderscore}{\kern0pt}matrix{\isacharunderscore}{\kern0pt}prod\ j{\isadigit{1}}\ j{\isadigit{4}}{\isacharparenright}{\kern0pt}\isanewline
\ \ \ \ \ \ \ \ \ \ \ \ \ \ \ \ \isacommand{also}\isamarkupfalse%
\ \isacommand{have}\isamarkupfalse%
\ {\isachardoublequoteopen}{\isasymdots}\ {\isacharequal}{\kern0pt}\ {\isacharparenleft}{\kern0pt}{\isacharparenleft}{\kern0pt}control{\isadigit{2}}\ U{\isacharparenright}{\kern0pt}\isactrlsup {\isasymdagger}{\isacharparenright}{\kern0pt}\ {\isachardollar}{\kern0pt}{\isachardollar}{\kern0pt}\ {\isacharparenleft}{\kern0pt}{\isadigit{3}}{\isacharcomma}{\kern0pt}{\isadigit{1}}{\isacharparenright}{\kern0pt}\ {\isacharasterisk}{\kern0pt}\ {\isacharparenleft}{\kern0pt}control{\isadigit{2}}\ U{\isacharparenright}{\kern0pt}\ {\isachardollar}{\kern0pt}{\isachardollar}{\kern0pt}\ {\isacharparenleft}{\kern0pt}{\isadigit{1}}{\isacharcomma}{\kern0pt}{\isadigit{1}}{\isacharparenright}{\kern0pt}\ {\isacharplus}{\kern0pt}\isanewline
\ \ \ \ \ \ \ \ \ \ \ \ \ \ \ \ \ \ \ \ \ \ \ \ \ \ \ \ \ \ \ \ {\isacharparenleft}{\kern0pt}{\isacharparenleft}{\kern0pt}control{\isadigit{2}}\ U{\isacharparenright}{\kern0pt}\isactrlsup {\isasymdagger}{\isacharparenright}{\kern0pt}\ {\isachardollar}{\kern0pt}{\isachardollar}{\kern0pt}\ {\isacharparenleft}{\kern0pt}{\isadigit{3}}{\isacharcomma}{\kern0pt}{\isadigit{3}}{\isacharparenright}{\kern0pt}\ {\isacharasterisk}{\kern0pt}\ {\isacharparenleft}{\kern0pt}control{\isadigit{2}}\ U{\isacharparenright}{\kern0pt}\ {\isachardollar}{\kern0pt}{\isachardollar}{\kern0pt}\ {\isacharparenleft}{\kern0pt}{\isadigit{3}}{\isacharcomma}{\kern0pt}{\isadigit{1}}{\isacharparenright}{\kern0pt}{\isachardoublequoteclose}\isanewline
\ \ \ \ \ \ \ \ \ \ \ \ \ \ \ \ \ \ \isacommand{using}\isamarkupfalse%
\ control{\isadigit{2}}{\isacharunderscore}{\kern0pt}def\ index{\isacharunderscore}{\kern0pt}mat{\isacharunderscore}{\kern0pt}of{\isacharunderscore}{\kern0pt}cols{\isacharunderscore}{\kern0pt}list\ \isacommand{by}\isamarkupfalse%
\ force\isanewline
\ \ \ \ \ \ \ \ \ \ \ \ \ \ \ \ \isacommand{also}\isamarkupfalse%
\ \isacommand{have}\isamarkupfalse%
\ {\isachardoublequoteopen}{\isasymdots}\ {\isacharequal}{\kern0pt}\ cnj\ {\isacharparenleft}{\kern0pt}{\isacharparenleft}{\kern0pt}control{\isadigit{2}}\ U{\isacharparenright}{\kern0pt}\ {\isachardollar}{\kern0pt}{\isachardollar}{\kern0pt}\ {\isacharparenleft}{\kern0pt}{\isadigit{1}}{\isacharcomma}{\kern0pt}{\isadigit{3}}{\isacharparenright}{\kern0pt}{\isacharparenright}{\kern0pt}\ {\isacharasterisk}{\kern0pt}\ {\isacharparenleft}{\kern0pt}control{\isadigit{2}}\ U{\isacharparenright}{\kern0pt}\ {\isachardollar}{\kern0pt}{\isachardollar}{\kern0pt}\ {\isacharparenleft}{\kern0pt}{\isadigit{1}}{\isacharcomma}{\kern0pt}{\isadigit{1}}{\isacharparenright}{\kern0pt}\ {\isacharplus}{\kern0pt}\isanewline
\ \ \ \ \ \ \ \ \ \ \ \ \ \ \ \ \ \ \ \ \ \ \ \ \ \ \ \ \ \ \ \ cnj\ {\isacharparenleft}{\kern0pt}{\isacharparenleft}{\kern0pt}control{\isadigit{2}}\ U{\isacharparenright}{\kern0pt}\ {\isachardollar}{\kern0pt}{\isachardollar}{\kern0pt}\ {\isacharparenleft}{\kern0pt}{\isadigit{3}}{\isacharcomma}{\kern0pt}{\isadigit{3}}{\isacharparenright}{\kern0pt}{\isacharparenright}{\kern0pt}\ {\isacharasterisk}{\kern0pt}\ {\isacharparenleft}{\kern0pt}control{\isadigit{2}}\ U{\isacharparenright}{\kern0pt}\ {\isachardollar}{\kern0pt}{\isachardollar}{\kern0pt}\ {\isacharparenleft}{\kern0pt}{\isadigit{3}}{\isacharcomma}{\kern0pt}{\isadigit{1}}{\isacharparenright}{\kern0pt}{\isachardoublequoteclose}\isanewline
\ \ \ \ \ \ \ \ \ \ \ \ \ \ \ \ \ \ \isacommand{using}\isamarkupfalse%
\ dagger{\isacharunderscore}{\kern0pt}def\isanewline
\ \ \ \ \ \ \ \ \ \ \ \ \ \ \ \ \ \ \isacommand{by}\isamarkupfalse%
\ {\isacharparenleft}{\kern0pt}simp\ add{\isacharcolon}{\kern0pt}\ Tensor{\isachardot}{\kern0pt}mat{\isacharunderscore}{\kern0pt}of{\isacharunderscore}{\kern0pt}cols{\isacharunderscore}{\kern0pt}list{\isacharunderscore}{\kern0pt}def\ control{\isadigit{2}}{\isacharunderscore}{\kern0pt}def{\isacharparenright}{\kern0pt}\isanewline
\ \ \ \ \ \ \ \ \ \ \ \ \ \ \ \ \isacommand{also}\isamarkupfalse%
\ \isacommand{have}\isamarkupfalse%
\ {\isachardoublequoteopen}{\isasymdots}\ {\isacharequal}{\kern0pt}\ cnj\ {\isacharparenleft}{\kern0pt}U\ {\isachardollar}{\kern0pt}{\isachardollar}{\kern0pt}\ {\isacharparenleft}{\kern0pt}{\isadigit{0}}{\isacharcomma}{\kern0pt}{\isadigit{1}}{\isacharparenright}{\kern0pt}{\isacharparenright}{\kern0pt}\ {\isacharasterisk}{\kern0pt}\ {\isacharparenleft}{\kern0pt}U\ {\isachardollar}{\kern0pt}{\isachardollar}{\kern0pt}\ {\isacharparenleft}{\kern0pt}{\isadigit{0}}{\isacharcomma}{\kern0pt}{\isadigit{0}}{\isacharparenright}{\kern0pt}{\isacharparenright}{\kern0pt}\ {\isacharplus}{\kern0pt}\isanewline
\ \ \ \ \ \ \ \ \ \ \ \ \ \ \ \ \ \ \ \ \ \ \ \ \ \ \ \ \ \ \ \ cnj\ {\isacharparenleft}{\kern0pt}U\ {\isachardollar}{\kern0pt}{\isachardollar}{\kern0pt}\ {\isacharparenleft}{\kern0pt}{\isadigit{1}}{\isacharcomma}{\kern0pt}{\isadigit{1}}{\isacharparenright}{\kern0pt}{\isacharparenright}{\kern0pt}\ {\isacharasterisk}{\kern0pt}\ {\isacharparenleft}{\kern0pt}U\ {\isachardollar}{\kern0pt}{\isachardollar}{\kern0pt}\ {\isacharparenleft}{\kern0pt}{\isadigit{1}}{\isacharcomma}{\kern0pt}{\isadigit{0}}{\isacharparenright}{\kern0pt}{\isacharparenright}{\kern0pt}{\isachardoublequoteclose}\isanewline
\ \ \ \ \ \ \ \ \ \ \ \ \ \ \ \ \ \ \isacommand{using}\isamarkupfalse%
\ control{\isadigit{2}}{\isacharunderscore}{\kern0pt}def\ index{\isacharunderscore}{\kern0pt}mat{\isacharunderscore}{\kern0pt}of{\isacharunderscore}{\kern0pt}cols{\isacharunderscore}{\kern0pt}list\ \isacommand{by}\isamarkupfalse%
\ simp\isanewline
\ \ \ \ \ \ \ \ \ \ \ \ \ \ \ \ \isacommand{also}\isamarkupfalse%
\ \isacommand{have}\isamarkupfalse%
\ {\isachardoublequoteopen}{\isasymdots}\ {\isacharequal}{\kern0pt}\ {\isacharparenleft}{\kern0pt}{\isacharparenleft}{\kern0pt}U\isactrlsup {\isasymdagger}{\isacharparenright}{\kern0pt}\ {\isacharasterisk}{\kern0pt}\ U{\isacharparenright}{\kern0pt}\ {\isachardollar}{\kern0pt}{\isachardollar}{\kern0pt}\ {\isacharparenleft}{\kern0pt}{\isadigit{1}}{\isacharcomma}{\kern0pt}{\isadigit{0}}{\isacharparenright}{\kern0pt}{\isachardoublequoteclose}\isanewline
\ \ \ \ \ \ \ \ \ \ \ \ \ \ \ \ \ \ \isacommand{using}\isamarkupfalse%
\ times{\isacharunderscore}{\kern0pt}mat{\isacharunderscore}{\kern0pt}def\ sumof{\isadigit{2}}\ assms{\isacharparenleft}{\kern0pt}{\isadigit{1}}{\isacharparenright}{\kern0pt}\ gate{\isacharunderscore}{\kern0pt}carrier{\isacharunderscore}{\kern0pt}mat\isanewline
\ \ \ \ \ \ \ \ \ \ \ \ \ \ \ \ \ \ \isacommand{by}\isamarkupfalse%
\ {\isacharparenleft}{\kern0pt}smt\ {\isacharparenleft}{\kern0pt}verit{\isacharcomma}{\kern0pt}\ del{\isacharunderscore}{\kern0pt}insts{\isacharparenright}{\kern0pt}\ Suc{\isacharunderscore}{\kern0pt}{\isadigit{1}}\ carrier{\isacharunderscore}{\kern0pt}matD{\isacharparenleft}{\kern0pt}{\isadigit{2}}{\isacharparenright}{\kern0pt}\ dagger{\isacharunderscore}{\kern0pt}def\ dim{\isacharunderscore}{\kern0pt}col{\isacharunderscore}{\kern0pt}mat{\isacharparenleft}{\kern0pt}{\isadigit{1}}{\isacharparenright}{\kern0pt}\ \isanewline
\ \ \ \ \ \ \ \ \ \ \ \ \ \ \ \ \ \ \ \ \ \ dim{\isacharunderscore}{\kern0pt}row{\isacharunderscore}{\kern0pt}of{\isacharunderscore}{\kern0pt}dagger\ gate{\isachardot}{\kern0pt}dim{\isacharunderscore}{\kern0pt}row\ index{\isacharunderscore}{\kern0pt}mat{\isacharparenleft}{\kern0pt}{\isadigit{1}}{\isacharparenright}{\kern0pt}\ index{\isacharunderscore}{\kern0pt}matrix{\isacharunderscore}{\kern0pt}prod\ lessI\ \isanewline
\ \ \ \ \ \ \ \ \ \ \ \ \ \ \ \ \ \ \ \ \ \ old{\isachardot}{\kern0pt}prod{\isachardot}{\kern0pt}case\ pos{\isadigit{2}}\ power{\isacharunderscore}{\kern0pt}one{\isacharunderscore}{\kern0pt}right{\isacharparenright}{\kern0pt}\isanewline
\ \ \ \ \ \ \ \ \ \ \ \ \ \ \ \ \isacommand{also}\isamarkupfalse%
\ \isacommand{have}\isamarkupfalse%
\ {\isachardoublequoteopen}{\isasymdots}\ {\isacharequal}{\kern0pt}\ {\isacharparenleft}{\kern0pt}{\isadigit{1}}\isactrlsub m\ {\isadigit{2}}{\isacharparenright}{\kern0pt}\ {\isachardollar}{\kern0pt}{\isachardollar}{\kern0pt}\ {\isacharparenleft}{\kern0pt}{\isadigit{1}}{\isacharcomma}{\kern0pt}{\isadigit{0}}{\isacharparenright}{\kern0pt}{\isachardoublequoteclose}\ \isacommand{using}\isamarkupfalse%
\ assms{\isacharparenleft}{\kern0pt}{\isadigit{1}}{\isacharparenright}{\kern0pt}\ gate{\isacharunderscore}{\kern0pt}def\ unitary{\isacharunderscore}{\kern0pt}def\ \isacommand{by}\isamarkupfalse%
\ auto\isanewline
\ \ \ \ \ \ \ \ \ \ \ \ \ \ \ \ \isacommand{also}\isamarkupfalse%
\ \isacommand{have}\isamarkupfalse%
\ {\isachardoublequoteopen}{\isasymdots}\ {\isacharequal}{\kern0pt}\ {\isadigit{0}}{\isachardoublequoteclose}\ \isacommand{using}\isamarkupfalse%
\ control{\isadigit{2}}{\isacharunderscore}{\kern0pt}def\ index{\isacharunderscore}{\kern0pt}mat{\isacharunderscore}{\kern0pt}of{\isacharunderscore}{\kern0pt}cols{\isacharunderscore}{\kern0pt}list\ \isacommand{by}\isamarkupfalse%
\ auto\isanewline
\ \ \ \ \ \ \ \ \ \ \ \ \ \ \ \ \isacommand{also}\isamarkupfalse%
\ \isacommand{have}\isamarkupfalse%
\ {\isachardoublequoteopen}{\isasymdots}\ {\isacharequal}{\kern0pt}\ {\isadigit{1}}\isactrlsub m\ {\isadigit{4}}\ {\isachardollar}{\kern0pt}{\isachardollar}{\kern0pt}\ {\isacharparenleft}{\kern0pt}{\isadigit{3}}{\isacharcomma}{\kern0pt}{\isadigit{1}}{\isacharparenright}{\kern0pt}{\isachardoublequoteclose}\ \isacommand{by}\isamarkupfalse%
\ simp\isanewline
\ \ \ \ \ \ \ \ \ \ \ \ \ \ \ \ \isacommand{finally}\isamarkupfalse%
\ \isacommand{show}\isamarkupfalse%
\ {\isacharquery}{\kern0pt}thesis\ \isacommand{using}\isamarkupfalse%
\ i{\isadigit{3}}\ j{\isadigit{1}}\ \isacommand{by}\isamarkupfalse%
\ simp\isanewline
\ \ \ \ \ \ \ \ \ \ \ \ \ \ \isacommand{qed}\isamarkupfalse%
\isanewline
\ \ \ \ \ \ \ \ \ \ \ \ \isacommand{next}\isamarkupfalse%
\isanewline
\ \ \ \ \ \ \ \ \ \ \ \ \ \ \isacommand{assume}\isamarkupfalse%
\ jl{\isadigit{2}}{\isacharcolon}{\kern0pt}{\isachardoublequoteopen}j\ {\isacharequal}{\kern0pt}\ {\isadigit{2}}\ {\isasymor}\ j\ {\isacharequal}{\kern0pt}\ {\isadigit{3}}{\isachardoublequoteclose}\isanewline
\ \ \ \ \ \ \ \ \ \ \ \ \ \ \isacommand{show}\isamarkupfalse%
\ {\isachardoublequoteopen}{\isacharparenleft}{\kern0pt}{\isacharparenleft}{\kern0pt}control{\isadigit{2}}\ U{\isacharparenright}{\kern0pt}\isactrlsup {\isasymdagger}\ {\isacharasterisk}{\kern0pt}\ control{\isadigit{2}}\ U{\isacharparenright}{\kern0pt}\ {\isachardollar}{\kern0pt}{\isachardollar}{\kern0pt}\ {\isacharparenleft}{\kern0pt}i{\isacharcomma}{\kern0pt}\ j{\isacharparenright}{\kern0pt}\ {\isacharequal}{\kern0pt}\ {\isadigit{1}}\isactrlsub m\ {\isadigit{4}}\ {\isachardollar}{\kern0pt}{\isachardollar}{\kern0pt}\ {\isacharparenleft}{\kern0pt}i{\isacharcomma}{\kern0pt}\ j{\isacharparenright}{\kern0pt}{\isachardoublequoteclose}\isanewline
\ \ \ \ \ \ \ \ \ \ \ \ \ \ \isacommand{proof}\isamarkupfalse%
\ {\isacharparenleft}{\kern0pt}rule\ disjE{\isacharparenright}{\kern0pt}\isanewline
\ \ \ \ \ \ \ \ \ \ \ \ \ \ \ \ \isacommand{show}\isamarkupfalse%
\ {\isachardoublequoteopen}j\ {\isacharequal}{\kern0pt}\ {\isadigit{2}}\ {\isasymor}\ j\ {\isacharequal}{\kern0pt}\ {\isadigit{3}}{\isachardoublequoteclose}\ \isacommand{using}\isamarkupfalse%
\ jl{\isadigit{2}}\ \isacommand{by}\isamarkupfalse%
\ this\isanewline
\ \ \ \ \ \ \ \ \ \ \ \ \ \ \isacommand{next}\isamarkupfalse%
\isanewline
\ \ \ \ \ \ \ \ \ \ \ \ \ \ \ \ \isacommand{assume}\isamarkupfalse%
\ j{\isadigit{2}}{\isacharcolon}{\kern0pt}{\isachardoublequoteopen}j\ {\isacharequal}{\kern0pt}\ {\isadigit{2}}{\isachardoublequoteclose}\isanewline
\ \ \ \ \ \ \ \ \ \ \ \ \ \ \ \ \isacommand{show}\isamarkupfalse%
\ {\isachardoublequoteopen}{\isacharparenleft}{\kern0pt}{\isacharparenleft}{\kern0pt}control{\isadigit{2}}\ U{\isacharparenright}{\kern0pt}\isactrlsup {\isasymdagger}\ {\isacharasterisk}{\kern0pt}\ control{\isadigit{2}}\ U{\isacharparenright}{\kern0pt}\ {\isachardollar}{\kern0pt}{\isachardollar}{\kern0pt}\ {\isacharparenleft}{\kern0pt}i{\isacharcomma}{\kern0pt}\ j{\isacharparenright}{\kern0pt}\ {\isacharequal}{\kern0pt}\ {\isadigit{1}}\isactrlsub m\ {\isadigit{4}}\ {\isachardollar}{\kern0pt}{\isachardollar}{\kern0pt}\ {\isacharparenleft}{\kern0pt}i{\isacharcomma}{\kern0pt}\ j{\isacharparenright}{\kern0pt}{\isachardoublequoteclose}\isanewline
\ \ \ \ \ \ \ \ \ \ \ \ \ \ \ \ \isacommand{proof}\isamarkupfalse%
\ {\isacharminus}{\kern0pt}\isanewline
\ \ \ \ \ \ \ \ \ \ \ \ \ \ \ \ \ \ \isacommand{have}\isamarkupfalse%
\ {\isachardoublequoteopen}{\isacharparenleft}{\kern0pt}{\isacharparenleft}{\kern0pt}control{\isadigit{2}}\ U{\isacharparenright}{\kern0pt}\isactrlsup {\isasymdagger}\ {\isacharasterisk}{\kern0pt}\ control{\isadigit{2}}\ U{\isacharparenright}{\kern0pt}\ {\isachardollar}{\kern0pt}{\isachardollar}{\kern0pt}\ {\isacharparenleft}{\kern0pt}{\isadigit{3}}{\isacharcomma}{\kern0pt}{\isadigit{2}}{\isacharparenright}{\kern0pt}\ {\isacharequal}{\kern0pt}\isanewline
\ \ \ \ \ \ \ \ \ \ \ \ \ \ \ \ \ \ \ \ \ \ \ \ {\isacharparenleft}{\kern0pt}{\isacharparenleft}{\kern0pt}control{\isadigit{2}}\ U{\isacharparenright}{\kern0pt}\isactrlsup {\isasymdagger}{\isacharparenright}{\kern0pt}\ {\isachardollar}{\kern0pt}{\isachardollar}{\kern0pt}\ {\isacharparenleft}{\kern0pt}{\isadigit{3}}{\isacharcomma}{\kern0pt}{\isadigit{0}}{\isacharparenright}{\kern0pt}\ {\isacharasterisk}{\kern0pt}\ {\isacharparenleft}{\kern0pt}control{\isadigit{2}}\ U{\isacharparenright}{\kern0pt}\ {\isachardollar}{\kern0pt}{\isachardollar}{\kern0pt}\ {\isacharparenleft}{\kern0pt}{\isadigit{0}}{\isacharcomma}{\kern0pt}{\isadigit{2}}{\isacharparenright}{\kern0pt}\ {\isacharplus}{\kern0pt}\isanewline
\ \ \ \ \ \ \ \ \ \ \ \ \ \ \ \ \ \ \ \ \ \ \ \ {\isacharparenleft}{\kern0pt}{\isacharparenleft}{\kern0pt}control{\isadigit{2}}\ U{\isacharparenright}{\kern0pt}\isactrlsup {\isasymdagger}{\isacharparenright}{\kern0pt}\ {\isachardollar}{\kern0pt}{\isachardollar}{\kern0pt}\ {\isacharparenleft}{\kern0pt}{\isadigit{3}}{\isacharcomma}{\kern0pt}{\isadigit{1}}{\isacharparenright}{\kern0pt}\ {\isacharasterisk}{\kern0pt}\ {\isacharparenleft}{\kern0pt}control{\isadigit{2}}\ U{\isacharparenright}{\kern0pt}\ {\isachardollar}{\kern0pt}{\isachardollar}{\kern0pt}\ {\isacharparenleft}{\kern0pt}{\isadigit{1}}{\isacharcomma}{\kern0pt}{\isadigit{2}}{\isacharparenright}{\kern0pt}\ {\isacharplus}{\kern0pt}\isanewline
\ \ \ \ \ \ \ \ \ \ \ \ \ \ \ \ \ \ \ \ \ \ \ \ {\isacharparenleft}{\kern0pt}{\isacharparenleft}{\kern0pt}control{\isadigit{2}}\ U{\isacharparenright}{\kern0pt}\isactrlsup {\isasymdagger}{\isacharparenright}{\kern0pt}\ {\isachardollar}{\kern0pt}{\isachardollar}{\kern0pt}\ {\isacharparenleft}{\kern0pt}{\isadigit{3}}{\isacharcomma}{\kern0pt}{\isadigit{2}}{\isacharparenright}{\kern0pt}\ {\isacharasterisk}{\kern0pt}\ {\isacharparenleft}{\kern0pt}control{\isadigit{2}}\ U{\isacharparenright}{\kern0pt}\ {\isachardollar}{\kern0pt}{\isachardollar}{\kern0pt}\ {\isacharparenleft}{\kern0pt}{\isadigit{2}}{\isacharcomma}{\kern0pt}{\isadigit{2}}{\isacharparenright}{\kern0pt}\ {\isacharplus}{\kern0pt}\isanewline
\ \ \ \ \ \ \ \ \ \ \ \ \ \ \ \ \ \ \ \ \ \ \ \ {\isacharparenleft}{\kern0pt}{\isacharparenleft}{\kern0pt}control{\isadigit{2}}\ U{\isacharparenright}{\kern0pt}\isactrlsup {\isasymdagger}{\isacharparenright}{\kern0pt}\ {\isachardollar}{\kern0pt}{\isachardollar}{\kern0pt}\ {\isacharparenleft}{\kern0pt}{\isadigit{3}}{\isacharcomma}{\kern0pt}{\isadigit{3}}{\isacharparenright}{\kern0pt}\ {\isacharasterisk}{\kern0pt}\ {\isacharparenleft}{\kern0pt}control{\isadigit{2}}\ U{\isacharparenright}{\kern0pt}\ {\isachardollar}{\kern0pt}{\isachardollar}{\kern0pt}\ {\isacharparenleft}{\kern0pt}{\isadigit{3}}{\isacharcomma}{\kern0pt}{\isadigit{2}}{\isacharparenright}{\kern0pt}{\isachardoublequoteclose}\isanewline
\ \ \ \ \ \ \ \ \ \ \ \ \ \ \ \ \ \ \ \ \isacommand{using}\isamarkupfalse%
\ sumof{\isadigit{4}}\isanewline
\ \ \ \ \ \ \ \ \ \ \ \ \ \ \ \ \ \ \ \ \isacommand{by}\isamarkupfalse%
\ {\isacharparenleft}{\kern0pt}metis\ {\isacharparenleft}{\kern0pt}no{\isacharunderscore}{\kern0pt}types{\isacharcomma}{\kern0pt}\ lifting{\isacharparenright}{\kern0pt}\ carrier{\isacharunderscore}{\kern0pt}matD{\isacharparenleft}{\kern0pt}{\isadigit{1}}{\isacharparenright}{\kern0pt}\ carrier{\isacharunderscore}{\kern0pt}matD{\isacharparenleft}{\kern0pt}{\isadigit{2}}{\isacharparenright}{\kern0pt}\ control{\isadigit{2}}{\isacharunderscore}{\kern0pt}carrier{\isacharunderscore}{\kern0pt}mat\ \isanewline
\ \ \ \ \ \ \ \ \ \ \ \ \ \ \ \ \ \ \ \ \ \ \ \ dim{\isacharunderscore}{\kern0pt}col{\isacharunderscore}{\kern0pt}of{\isacharunderscore}{\kern0pt}dagger\ dim{\isacharunderscore}{\kern0pt}row{\isacharunderscore}{\kern0pt}of{\isacharunderscore}{\kern0pt}dagger\ i{\isadigit{3}}\ i{\isadigit{4}}\ index{\isacharunderscore}{\kern0pt}matrix{\isacharunderscore}{\kern0pt}prod\ j{\isadigit{2}}\ j{\isadigit{4}}{\isacharparenright}{\kern0pt}\isanewline
\ \ \ \ \ \ \ \ \ \ \ \ \ \ \ \ \ \ \isacommand{also}\isamarkupfalse%
\ \isacommand{have}\isamarkupfalse%
\ {\isachardoublequoteopen}{\isasymdots}\ {\isacharequal}{\kern0pt}\ {\isacharparenleft}{\kern0pt}{\isacharparenleft}{\kern0pt}control{\isadigit{2}}\ U{\isacharparenright}{\kern0pt}\isactrlsup {\isasymdagger}{\isacharparenright}{\kern0pt}\ {\isachardollar}{\kern0pt}{\isachardollar}{\kern0pt}\ {\isacharparenleft}{\kern0pt}{\isadigit{3}}{\isacharcomma}{\kern0pt}{\isadigit{2}}{\isacharparenright}{\kern0pt}{\isachardoublequoteclose}\isanewline
\ \ \ \ \ \ \ \ \ \ \ \ \ \ \ \ \ \ \ \ \isacommand{using}\isamarkupfalse%
\ control{\isadigit{2}}{\isacharunderscore}{\kern0pt}def\ index{\isacharunderscore}{\kern0pt}mat{\isacharunderscore}{\kern0pt}of{\isacharunderscore}{\kern0pt}cols{\isacharunderscore}{\kern0pt}list\ \isacommand{by}\isamarkupfalse%
\ force\isanewline
\ \ \ \ \ \ \ \ \ \ \ \ \ \ \ \ \ \ \isacommand{also}\isamarkupfalse%
\ \isacommand{have}\isamarkupfalse%
\ {\isachardoublequoteopen}{\isasymdots}\ {\isacharequal}{\kern0pt}\ cnj\ {\isacharparenleft}{\kern0pt}{\isacharparenleft}{\kern0pt}control{\isadigit{2}}\ U{\isacharparenright}{\kern0pt}\ {\isachardollar}{\kern0pt}{\isachardollar}{\kern0pt}\ {\isacharparenleft}{\kern0pt}{\isadigit{2}}{\isacharcomma}{\kern0pt}{\isadigit{3}}{\isacharparenright}{\kern0pt}{\isacharparenright}{\kern0pt}{\isachardoublequoteclose}\isanewline
\ \ \ \ \ \ \ \ \ \ \ \ \ \ \ \ \ \ \ \ \isacommand{using}\isamarkupfalse%
\ dagger{\isacharunderscore}{\kern0pt}def\isanewline
\ \ \ \ \ \ \ \ \ \ \ \ \ \ \ \ \ \ \ \ \isacommand{by}\isamarkupfalse%
\ {\isacharparenleft}{\kern0pt}simp\ add{\isacharcolon}{\kern0pt}\ Tensor{\isachardot}{\kern0pt}mat{\isacharunderscore}{\kern0pt}of{\isacharunderscore}{\kern0pt}cols{\isacharunderscore}{\kern0pt}list{\isacharunderscore}{\kern0pt}def\ control{\isadigit{2}}{\isacharunderscore}{\kern0pt}def{\isacharparenright}{\kern0pt}\isanewline
\ \ \ \ \ \ \ \ \ \ \ \ \ \ \ \ \ \ \isacommand{also}\isamarkupfalse%
\ \isacommand{have}\isamarkupfalse%
\ {\isachardoublequoteopen}{\isasymdots}\ {\isacharequal}{\kern0pt}\ {\isadigit{0}}{\isachardoublequoteclose}\ \isacommand{using}\isamarkupfalse%
\ control{\isadigit{2}}{\isacharunderscore}{\kern0pt}def\ index{\isacharunderscore}{\kern0pt}mat{\isacharunderscore}{\kern0pt}of{\isacharunderscore}{\kern0pt}cols{\isacharunderscore}{\kern0pt}list\ \isacommand{by}\isamarkupfalse%
\ auto\isanewline
\ \ \ \ \ \ \ \ \ \ \ \ \ \ \ \ \ \ \isacommand{also}\isamarkupfalse%
\ \isacommand{have}\isamarkupfalse%
\ {\isachardoublequoteopen}{\isasymdots}\ {\isacharequal}{\kern0pt}\ {\isadigit{1}}\isactrlsub m\ {\isadigit{4}}\ {\isachardollar}{\kern0pt}{\isachardollar}{\kern0pt}\ {\isacharparenleft}{\kern0pt}{\isadigit{3}}{\isacharcomma}{\kern0pt}{\isadigit{2}}{\isacharparenright}{\kern0pt}{\isachardoublequoteclose}\ \isacommand{by}\isamarkupfalse%
\ simp\isanewline
\ \ \ \ \ \ \ \ \ \ \ \ \ \ \ \ \ \ \isacommand{finally}\isamarkupfalse%
\ \isacommand{show}\isamarkupfalse%
\ {\isacharquery}{\kern0pt}thesis\ \isacommand{using}\isamarkupfalse%
\ i{\isadigit{3}}\ j{\isadigit{2}}\ \isacommand{by}\isamarkupfalse%
\ simp\isanewline
\ \ \ \ \ \ \ \ \ \ \ \ \ \ \ \ \isacommand{qed}\isamarkupfalse%
\isanewline
\ \ \ \ \ \ \ \ \ \ \ \ \ \ \isacommand{next}\isamarkupfalse%
\isanewline
\ \ \ \ \ \ \ \ \ \ \ \ \ \ \ \ \isacommand{assume}\isamarkupfalse%
\ j{\isadigit{3}}{\isacharcolon}{\kern0pt}{\isachardoublequoteopen}j\ {\isacharequal}{\kern0pt}\ {\isadigit{3}}{\isachardoublequoteclose}\isanewline
\ \ \ \ \ \ \ \ \ \ \ \ \ \ \ \ \isacommand{show}\isamarkupfalse%
\ {\isachardoublequoteopen}{\isacharparenleft}{\kern0pt}{\isacharparenleft}{\kern0pt}control{\isadigit{2}}\ U{\isacharparenright}{\kern0pt}\isactrlsup {\isasymdagger}\ {\isacharasterisk}{\kern0pt}\ control{\isadigit{2}}\ U{\isacharparenright}{\kern0pt}\ {\isachardollar}{\kern0pt}{\isachardollar}{\kern0pt}\ {\isacharparenleft}{\kern0pt}i{\isacharcomma}{\kern0pt}\ j{\isacharparenright}{\kern0pt}\ {\isacharequal}{\kern0pt}\ {\isadigit{1}}\isactrlsub m\ {\isadigit{4}}\ {\isachardollar}{\kern0pt}{\isachardollar}{\kern0pt}\ {\isacharparenleft}{\kern0pt}i{\isacharcomma}{\kern0pt}\ j{\isacharparenright}{\kern0pt}{\isachardoublequoteclose}\isanewline
\ \ \ \ \ \ \ \ \ \ \ \ \ \ \ \ \isacommand{proof}\isamarkupfalse%
\ {\isacharminus}{\kern0pt}\isanewline
\ \ \ \ \ \ \ \ \ \ \ \ \ \ \ \ \ \ \isacommand{have}\isamarkupfalse%
\ {\isachardoublequoteopen}{\isacharparenleft}{\kern0pt}{\isacharparenleft}{\kern0pt}control{\isadigit{2}}\ U{\isacharparenright}{\kern0pt}\isactrlsup {\isasymdagger}\ {\isacharasterisk}{\kern0pt}\ control{\isadigit{2}}\ U{\isacharparenright}{\kern0pt}\ {\isachardollar}{\kern0pt}{\isachardollar}{\kern0pt}\ {\isacharparenleft}{\kern0pt}{\isadigit{3}}{\isacharcomma}{\kern0pt}{\isadigit{3}}{\isacharparenright}{\kern0pt}\ {\isacharequal}{\kern0pt}\isanewline
\ \ \ \ \ \ \ \ \ \ \ \ \ \ \ \ \ \ \ \ \ \ \ \ {\isacharparenleft}{\kern0pt}{\isacharparenleft}{\kern0pt}control{\isadigit{2}}\ U{\isacharparenright}{\kern0pt}\isactrlsup {\isasymdagger}{\isacharparenright}{\kern0pt}\ {\isachardollar}{\kern0pt}{\isachardollar}{\kern0pt}\ {\isacharparenleft}{\kern0pt}{\isadigit{3}}{\isacharcomma}{\kern0pt}{\isadigit{0}}{\isacharparenright}{\kern0pt}\ {\isacharasterisk}{\kern0pt}\ {\isacharparenleft}{\kern0pt}control{\isadigit{2}}\ U{\isacharparenright}{\kern0pt}\ {\isachardollar}{\kern0pt}{\isachardollar}{\kern0pt}\ {\isacharparenleft}{\kern0pt}{\isadigit{0}}{\isacharcomma}{\kern0pt}{\isadigit{3}}{\isacharparenright}{\kern0pt}\ {\isacharplus}{\kern0pt}\isanewline
\ \ \ \ \ \ \ \ \ \ \ \ \ \ \ \ \ \ \ \ \ \ \ \ {\isacharparenleft}{\kern0pt}{\isacharparenleft}{\kern0pt}control{\isadigit{2}}\ U{\isacharparenright}{\kern0pt}\isactrlsup {\isasymdagger}{\isacharparenright}{\kern0pt}\ {\isachardollar}{\kern0pt}{\isachardollar}{\kern0pt}\ {\isacharparenleft}{\kern0pt}{\isadigit{3}}{\isacharcomma}{\kern0pt}{\isadigit{1}}{\isacharparenright}{\kern0pt}\ {\isacharasterisk}{\kern0pt}\ {\isacharparenleft}{\kern0pt}control{\isadigit{2}}\ U{\isacharparenright}{\kern0pt}\ {\isachardollar}{\kern0pt}{\isachardollar}{\kern0pt}\ {\isacharparenleft}{\kern0pt}{\isadigit{1}}{\isacharcomma}{\kern0pt}{\isadigit{3}}{\isacharparenright}{\kern0pt}\ {\isacharplus}{\kern0pt}\isanewline
\ \ \ \ \ \ \ \ \ \ \ \ \ \ \ \ \ \ \ \ \ \ \ \ {\isacharparenleft}{\kern0pt}{\isacharparenleft}{\kern0pt}control{\isadigit{2}}\ U{\isacharparenright}{\kern0pt}\isactrlsup {\isasymdagger}{\isacharparenright}{\kern0pt}\ {\isachardollar}{\kern0pt}{\isachardollar}{\kern0pt}\ {\isacharparenleft}{\kern0pt}{\isadigit{3}}{\isacharcomma}{\kern0pt}{\isadigit{2}}{\isacharparenright}{\kern0pt}\ {\isacharasterisk}{\kern0pt}\ {\isacharparenleft}{\kern0pt}control{\isadigit{2}}\ U{\isacharparenright}{\kern0pt}\ {\isachardollar}{\kern0pt}{\isachardollar}{\kern0pt}\ {\isacharparenleft}{\kern0pt}{\isadigit{2}}{\isacharcomma}{\kern0pt}{\isadigit{3}}{\isacharparenright}{\kern0pt}\ {\isacharplus}{\kern0pt}\isanewline
\ \ \ \ \ \ \ \ \ \ \ \ \ \ \ \ \ \ \ \ \ \ \ \ {\isacharparenleft}{\kern0pt}{\isacharparenleft}{\kern0pt}control{\isadigit{2}}\ U{\isacharparenright}{\kern0pt}\isactrlsup {\isasymdagger}{\isacharparenright}{\kern0pt}\ {\isachardollar}{\kern0pt}{\isachardollar}{\kern0pt}\ {\isacharparenleft}{\kern0pt}{\isadigit{3}}{\isacharcomma}{\kern0pt}{\isadigit{3}}{\isacharparenright}{\kern0pt}\ {\isacharasterisk}{\kern0pt}\ {\isacharparenleft}{\kern0pt}control{\isadigit{2}}\ U{\isacharparenright}{\kern0pt}\ {\isachardollar}{\kern0pt}{\isachardollar}{\kern0pt}\ {\isacharparenleft}{\kern0pt}{\isadigit{3}}{\isacharcomma}{\kern0pt}{\isadigit{3}}{\isacharparenright}{\kern0pt}{\isachardoublequoteclose}\isanewline
\ \ \ \ \ \ \ \ \ \ \ \ \ \ \ \ \ \ \ \ \isacommand{using}\isamarkupfalse%
\ sumof{\isadigit{4}}\isanewline
\ \ \ \ \ \ \ \ \ \ \ \ \ \ \ \ \ \ \ \ \isacommand{by}\isamarkupfalse%
\ {\isacharparenleft}{\kern0pt}metis\ {\isacharparenleft}{\kern0pt}no{\isacharunderscore}{\kern0pt}types{\isacharcomma}{\kern0pt}\ lifting{\isacharparenright}{\kern0pt}\ carrier{\isacharunderscore}{\kern0pt}matD{\isacharparenleft}{\kern0pt}{\isadigit{1}}{\isacharparenright}{\kern0pt}\ carrier{\isacharunderscore}{\kern0pt}matD{\isacharparenleft}{\kern0pt}{\isadigit{2}}{\isacharparenright}{\kern0pt}\ \isanewline
\ \ \ \ \ \ \ \ \ \ \ \ \ \ \ \ \ \ \ \ \ \ \ \ control{\isadigit{2}}{\isacharunderscore}{\kern0pt}carrier{\isacharunderscore}{\kern0pt}mat\ dim{\isacharunderscore}{\kern0pt}col{\isacharunderscore}{\kern0pt}of{\isacharunderscore}{\kern0pt}dagger\ dim{\isacharunderscore}{\kern0pt}row{\isacharunderscore}{\kern0pt}of{\isacharunderscore}{\kern0pt}dagger\ i{\isadigit{3}}\ \isanewline
\ \ \ \ \ \ \ \ \ \ \ \ \ \ \ \ \ \ \ \ \ \ \ \ index{\isacharunderscore}{\kern0pt}matrix{\isacharunderscore}{\kern0pt}prod\ j{\isadigit{3}}\ j{\isadigit{4}}{\isacharparenright}{\kern0pt}\isanewline
\ \ \ \ \ \ \ \ \ \ \ \ \ \ \ \ \ \ \isacommand{also}\isamarkupfalse%
\ \isacommand{have}\isamarkupfalse%
\ {\isachardoublequoteopen}{\isasymdots}\ {\isacharequal}{\kern0pt}\ {\isacharparenleft}{\kern0pt}{\isacharparenleft}{\kern0pt}control{\isadigit{2}}\ U{\isacharparenright}{\kern0pt}\isactrlsup {\isasymdagger}{\isacharparenright}{\kern0pt}\ {\isachardollar}{\kern0pt}{\isachardollar}{\kern0pt}\ {\isacharparenleft}{\kern0pt}{\isadigit{3}}{\isacharcomma}{\kern0pt}{\isadigit{1}}{\isacharparenright}{\kern0pt}\ {\isacharasterisk}{\kern0pt}\ {\isacharparenleft}{\kern0pt}control{\isadigit{2}}\ U{\isacharparenright}{\kern0pt}\ {\isachardollar}{\kern0pt}{\isachardollar}{\kern0pt}\ {\isacharparenleft}{\kern0pt}{\isadigit{1}}{\isacharcomma}{\kern0pt}{\isadigit{3}}{\isacharparenright}{\kern0pt}\ {\isacharplus}{\kern0pt}\isanewline
\ \ \ \ \ \ \ \ \ \ \ \ \ \ \ \ \ \ \ \ \ \ \ \ \ \ \ \ \ \ \ \ \ \ {\isacharparenleft}{\kern0pt}{\isacharparenleft}{\kern0pt}control{\isadigit{2}}\ U{\isacharparenright}{\kern0pt}\isactrlsup {\isasymdagger}{\isacharparenright}{\kern0pt}\ {\isachardollar}{\kern0pt}{\isachardollar}{\kern0pt}\ {\isacharparenleft}{\kern0pt}{\isadigit{3}}{\isacharcomma}{\kern0pt}{\isadigit{3}}{\isacharparenright}{\kern0pt}\ {\isacharasterisk}{\kern0pt}\ {\isacharparenleft}{\kern0pt}control{\isadigit{2}}\ U{\isacharparenright}{\kern0pt}\ {\isachardollar}{\kern0pt}{\isachardollar}{\kern0pt}\ {\isacharparenleft}{\kern0pt}{\isadigit{3}}{\isacharcomma}{\kern0pt}{\isadigit{3}}{\isacharparenright}{\kern0pt}{\isachardoublequoteclose}\isanewline
\ \ \ \ \ \ \ \ \ \ \ \ \ \ \ \ \ \ \ \ \isacommand{using}\isamarkupfalse%
\ control{\isadigit{2}}{\isacharunderscore}{\kern0pt}def\ index{\isacharunderscore}{\kern0pt}mat{\isacharunderscore}{\kern0pt}of{\isacharunderscore}{\kern0pt}cols{\isacharunderscore}{\kern0pt}list\ \isacommand{by}\isamarkupfalse%
\ force\isanewline
\ \ \ \ \ \ \ \ \ \ \ \ \ \ \ \ \ \ \isacommand{also}\isamarkupfalse%
\ \isacommand{have}\isamarkupfalse%
\ {\isachardoublequoteopen}{\isasymdots}\ {\isacharequal}{\kern0pt}\ cnj\ {\isacharparenleft}{\kern0pt}{\isacharparenleft}{\kern0pt}control{\isadigit{2}}\ U{\isacharparenright}{\kern0pt}\ {\isachardollar}{\kern0pt}{\isachardollar}{\kern0pt}\ {\isacharparenleft}{\kern0pt}{\isadigit{1}}{\isacharcomma}{\kern0pt}{\isadigit{3}}{\isacharparenright}{\kern0pt}{\isacharparenright}{\kern0pt}\ {\isacharasterisk}{\kern0pt}\ {\isacharparenleft}{\kern0pt}control{\isadigit{2}}\ U{\isacharparenright}{\kern0pt}\ {\isachardollar}{\kern0pt}{\isachardollar}{\kern0pt}\ {\isacharparenleft}{\kern0pt}{\isadigit{1}}{\isacharcomma}{\kern0pt}{\isadigit{3}}{\isacharparenright}{\kern0pt}\ {\isacharplus}{\kern0pt}\isanewline
\ \ \ \ \ \ \ \ \ \ \ \ \ \ \ \ \ \ \ \ \ \ \ \ \ \ \ \ \ \ \ \ \ \ cnj\ {\isacharparenleft}{\kern0pt}{\isacharparenleft}{\kern0pt}control{\isadigit{2}}\ U{\isacharparenright}{\kern0pt}\ {\isachardollar}{\kern0pt}{\isachardollar}{\kern0pt}\ {\isacharparenleft}{\kern0pt}{\isadigit{3}}{\isacharcomma}{\kern0pt}{\isadigit{3}}{\isacharparenright}{\kern0pt}{\isacharparenright}{\kern0pt}\ {\isacharasterisk}{\kern0pt}\ {\isacharparenleft}{\kern0pt}control{\isadigit{2}}\ U{\isacharparenright}{\kern0pt}\ {\isachardollar}{\kern0pt}{\isachardollar}{\kern0pt}\ {\isacharparenleft}{\kern0pt}{\isadigit{3}}{\isacharcomma}{\kern0pt}{\isadigit{3}}{\isacharparenright}{\kern0pt}{\isachardoublequoteclose}\isanewline
\ \ \ \ \ \ \ \ \ \ \ \ \ \ \ \ \ \ \ \ \isacommand{using}\isamarkupfalse%
\ dagger{\isacharunderscore}{\kern0pt}def\isanewline
\ \ \ \ \ \ \ \ \ \ \ \ \ \ \ \ \ \ \ \ \isacommand{by}\isamarkupfalse%
\ {\isacharparenleft}{\kern0pt}simp\ add{\isacharcolon}{\kern0pt}\ Tensor{\isachardot}{\kern0pt}mat{\isacharunderscore}{\kern0pt}of{\isacharunderscore}{\kern0pt}cols{\isacharunderscore}{\kern0pt}list{\isacharunderscore}{\kern0pt}def\ control{\isadigit{2}}{\isacharunderscore}{\kern0pt}def{\isacharparenright}{\kern0pt}\isanewline
\ \ \ \ \ \ \ \ \ \ \ \ \ \ \ \ \ \ \isacommand{also}\isamarkupfalse%
\ \isacommand{have}\isamarkupfalse%
\ {\isachardoublequoteopen}{\isasymdots}\ {\isacharequal}{\kern0pt}\ cnj\ {\isacharparenleft}{\kern0pt}U\ {\isachardollar}{\kern0pt}{\isachardollar}{\kern0pt}\ {\isacharparenleft}{\kern0pt}{\isadigit{0}}{\isacharcomma}{\kern0pt}{\isadigit{1}}{\isacharparenright}{\kern0pt}{\isacharparenright}{\kern0pt}\ {\isacharasterisk}{\kern0pt}\ {\isacharparenleft}{\kern0pt}U\ {\isachardollar}{\kern0pt}{\isachardollar}{\kern0pt}\ {\isacharparenleft}{\kern0pt}{\isadigit{0}}{\isacharcomma}{\kern0pt}{\isadigit{1}}{\isacharparenright}{\kern0pt}{\isacharparenright}{\kern0pt}\ {\isacharplus}{\kern0pt}\isanewline
\ \ \ \ \ \ \ \ \ \ \ \ \ \ \ \ \ \ \ \ \ \ \ \ \ \ \ \ \ \ \ \ \ \ cnj\ {\isacharparenleft}{\kern0pt}U\ {\isachardollar}{\kern0pt}{\isachardollar}{\kern0pt}\ {\isacharparenleft}{\kern0pt}{\isadigit{1}}{\isacharcomma}{\kern0pt}{\isadigit{1}}{\isacharparenright}{\kern0pt}{\isacharparenright}{\kern0pt}\ {\isacharasterisk}{\kern0pt}\ {\isacharparenleft}{\kern0pt}U\ {\isachardollar}{\kern0pt}{\isachardollar}{\kern0pt}\ {\isacharparenleft}{\kern0pt}{\isadigit{1}}{\isacharcomma}{\kern0pt}{\isadigit{1}}{\isacharparenright}{\kern0pt}{\isacharparenright}{\kern0pt}{\isachardoublequoteclose}\isanewline
\ \ \ \ \ \ \ \ \ \ \ \ \ \ \ \ \ \ \ \ \isacommand{using}\isamarkupfalse%
\ control{\isadigit{2}}{\isacharunderscore}{\kern0pt}def\ index{\isacharunderscore}{\kern0pt}mat{\isacharunderscore}{\kern0pt}of{\isacharunderscore}{\kern0pt}cols{\isacharunderscore}{\kern0pt}list\ \isacommand{by}\isamarkupfalse%
\ simp\isanewline
\ \ \ \ \ \ \ \ \ \ \ \ \ \ \ \ \ \ \isacommand{also}\isamarkupfalse%
\ \isacommand{have}\isamarkupfalse%
\ {\isachardoublequoteopen}{\isasymdots}\ {\isacharequal}{\kern0pt}\ {\isacharparenleft}{\kern0pt}{\isacharparenleft}{\kern0pt}U\isactrlsup {\isasymdagger}{\isacharparenright}{\kern0pt}\ {\isacharasterisk}{\kern0pt}\ U{\isacharparenright}{\kern0pt}\ {\isachardollar}{\kern0pt}{\isachardollar}{\kern0pt}\ {\isacharparenleft}{\kern0pt}{\isadigit{1}}{\isacharcomma}{\kern0pt}{\isadigit{1}}{\isacharparenright}{\kern0pt}{\isachardoublequoteclose}\isanewline
\ \ \ \ \ \ \ \ \ \ \ \ \ \ \ \ \ \ \ \ \isacommand{using}\isamarkupfalse%
\ times{\isacharunderscore}{\kern0pt}mat{\isacharunderscore}{\kern0pt}def\ sumof{\isadigit{2}}\ assms{\isacharparenleft}{\kern0pt}{\isadigit{1}}{\isacharparenright}{\kern0pt}\ gate{\isacharunderscore}{\kern0pt}carrier{\isacharunderscore}{\kern0pt}mat\isanewline
\ \ \ \ \ \ \ \ \ \ \ \ \ \ \ \ \ \ \ \ \isacommand{by}\isamarkupfalse%
\ {\isacharparenleft}{\kern0pt}smt\ {\isacharparenleft}{\kern0pt}verit{\isacharcomma}{\kern0pt}\ del{\isacharunderscore}{\kern0pt}insts{\isacharparenright}{\kern0pt}\ Suc{\isacharunderscore}{\kern0pt}{\isadigit{1}}\ carrier{\isacharunderscore}{\kern0pt}matD{\isacharparenleft}{\kern0pt}{\isadigit{2}}{\isacharparenright}{\kern0pt}\ dagger{\isacharunderscore}{\kern0pt}def\ dim{\isacharunderscore}{\kern0pt}col{\isacharunderscore}{\kern0pt}mat{\isacharparenleft}{\kern0pt}{\isadigit{1}}{\isacharparenright}{\kern0pt}\ \isanewline
\ \ \ \ \ \ \ \ \ \ \ \ \ \ \ \ \ \ \ \ \ \ \ \ dim{\isacharunderscore}{\kern0pt}row{\isacharunderscore}{\kern0pt}of{\isacharunderscore}{\kern0pt}dagger\ gate{\isachardot}{\kern0pt}dim{\isacharunderscore}{\kern0pt}row\ index{\isacharunderscore}{\kern0pt}mat{\isacharparenleft}{\kern0pt}{\isadigit{1}}{\isacharparenright}{\kern0pt}\ index{\isacharunderscore}{\kern0pt}matrix{\isacharunderscore}{\kern0pt}prod\ lessI\ \isanewline
\ \ \ \ \ \ \ \ \ \ \ \ \ \ \ \ \ \ \ \ \ \ \ \ old{\isachardot}{\kern0pt}prod{\isachardot}{\kern0pt}case\ pos{\isadigit{2}}\ power{\isacharunderscore}{\kern0pt}one{\isacharunderscore}{\kern0pt}right{\isacharparenright}{\kern0pt}\isanewline
\ \ \ \ \ \ \ \ \ \ \ \ \ \ \ \ \ \ \isacommand{also}\isamarkupfalse%
\ \isacommand{have}\isamarkupfalse%
\ {\isachardoublequoteopen}{\isasymdots}\ {\isacharequal}{\kern0pt}\ {\isacharparenleft}{\kern0pt}{\isadigit{1}}\isactrlsub m\ {\isadigit{2}}{\isacharparenright}{\kern0pt}\ {\isachardollar}{\kern0pt}{\isachardollar}{\kern0pt}\ {\isacharparenleft}{\kern0pt}{\isadigit{1}}{\isacharcomma}{\kern0pt}{\isadigit{1}}{\isacharparenright}{\kern0pt}{\isachardoublequoteclose}\ \isacommand{using}\isamarkupfalse%
\ assms{\isacharparenleft}{\kern0pt}{\isadigit{1}}{\isacharparenright}{\kern0pt}\ gate{\isacharunderscore}{\kern0pt}def\ unitary{\isacharunderscore}{\kern0pt}def\ \isacommand{by}\isamarkupfalse%
\ auto\isanewline
\ \ \ \ \ \ \ \ \ \ \ \ \ \ \ \ \ \ \isacommand{also}\isamarkupfalse%
\ \isacommand{have}\isamarkupfalse%
\ {\isachardoublequoteopen}{\isasymdots}\ {\isacharequal}{\kern0pt}\ {\isadigit{1}}{\isachardoublequoteclose}\ \isacommand{using}\isamarkupfalse%
\ control{\isadigit{2}}{\isacharunderscore}{\kern0pt}def\ index{\isacharunderscore}{\kern0pt}mat{\isacharunderscore}{\kern0pt}of{\isacharunderscore}{\kern0pt}cols{\isacharunderscore}{\kern0pt}list\ \isacommand{by}\isamarkupfalse%
\ auto\isanewline
\ \ \ \ \ \ \ \ \ \ \ \ \ \ \ \ \ \ \isacommand{also}\isamarkupfalse%
\ \isacommand{have}\isamarkupfalse%
\ {\isachardoublequoteopen}{\isasymdots}\ {\isacharequal}{\kern0pt}\ {\isadigit{1}}\isactrlsub m\ {\isadigit{4}}\ {\isachardollar}{\kern0pt}{\isachardollar}{\kern0pt}\ {\isacharparenleft}{\kern0pt}{\isadigit{3}}{\isacharcomma}{\kern0pt}{\isadigit{3}}{\isacharparenright}{\kern0pt}{\isachardoublequoteclose}\ \isacommand{by}\isamarkupfalse%
\ simp\isanewline
\ \ \ \ \ \ \ \ \ \ \ \ \ \ \ \ \ \ \isacommand{finally}\isamarkupfalse%
\ \isacommand{show}\isamarkupfalse%
\ {\isacharquery}{\kern0pt}thesis\ \isacommand{using}\isamarkupfalse%
\ i{\isadigit{3}}\ j{\isadigit{3}}\ \isacommand{by}\isamarkupfalse%
\ simp\isanewline
\ \ \ \ \ \ \ \ \ \ \ \ \ \ \ \ \isacommand{qed}\isamarkupfalse%
\isanewline
\ \ \ \ \ \ \ \ \ \ \ \ \ \ \isacommand{qed}\isamarkupfalse%
\isanewline
\ \ \ \ \ \ \ \ \ \ \ \ \isacommand{qed}\isamarkupfalse%
\isanewline
\ \ \ \ \ \ \ \ \ \ \isacommand{qed}\isamarkupfalse%
\isanewline
\ \ \ \ \ \ \ \ \isacommand{qed}\isamarkupfalse%
\isanewline
\ \ \ \ \ \ \isacommand{qed}\isamarkupfalse%
\isanewline
\ \ \ \ \isacommand{qed}\isamarkupfalse%
\isanewline
\ \ \isacommand{qed}\isamarkupfalse%
\isanewline
\isacommand{next}\isamarkupfalse%
\isanewline
\ \ \isacommand{show}\isamarkupfalse%
\ {\isachardoublequoteopen}dim{\isacharunderscore}{\kern0pt}row\ {\isacharparenleft}{\kern0pt}{\isacharparenleft}{\kern0pt}control{\isadigit{2}}\ U{\isacharparenright}{\kern0pt}\isactrlsup {\isasymdagger}\ {\isacharasterisk}{\kern0pt}\ control{\isadigit{2}}\ U{\isacharparenright}{\kern0pt}\ {\isacharequal}{\kern0pt}\ dim{\isacharunderscore}{\kern0pt}row\ {\isacharparenleft}{\kern0pt}{\isadigit{1}}\isactrlsub m\ {\isadigit{4}}{\isacharparenright}{\kern0pt}{\isachardoublequoteclose}\isanewline
\ \ \ \ \isacommand{by}\isamarkupfalse%
\ {\isacharparenleft}{\kern0pt}metis\ carrier{\isacharunderscore}{\kern0pt}matD{\isacharparenleft}{\kern0pt}{\isadigit{2}}{\isacharparenright}{\kern0pt}\ control{\isadigit{2}}{\isacharunderscore}{\kern0pt}carrier{\isacharunderscore}{\kern0pt}mat\ dim{\isacharunderscore}{\kern0pt}row{\isacharunderscore}{\kern0pt}of{\isacharunderscore}{\kern0pt}dagger\ \isanewline
\ \ \ \ \ \ \ \ index{\isacharunderscore}{\kern0pt}mult{\isacharunderscore}{\kern0pt}mat{\isacharparenleft}{\kern0pt}{\isadigit{2}}{\isacharparenright}{\kern0pt}\ index{\isacharunderscore}{\kern0pt}one{\isacharunderscore}{\kern0pt}mat{\isacharparenleft}{\kern0pt}{\isadigit{2}}{\isacharparenright}{\kern0pt}{\isacharparenright}{\kern0pt}\isanewline
\isacommand{next}\isamarkupfalse%
\isanewline
\ \ \isacommand{show}\isamarkupfalse%
\ {\isachardoublequoteopen}dim{\isacharunderscore}{\kern0pt}col\ {\isacharparenleft}{\kern0pt}{\isacharparenleft}{\kern0pt}control{\isadigit{2}}\ U{\isacharparenright}{\kern0pt}\isactrlsup {\isasymdagger}\ {\isacharasterisk}{\kern0pt}\ control{\isadigit{2}}\ U{\isacharparenright}{\kern0pt}\ {\isacharequal}{\kern0pt}\ dim{\isacharunderscore}{\kern0pt}col\ {\isacharparenleft}{\kern0pt}{\isadigit{1}}\isactrlsub m\ {\isadigit{4}}{\isacharparenright}{\kern0pt}{\isachardoublequoteclose}\isanewline
\ \ \ \ \isacommand{by}\isamarkupfalse%
\ {\isacharparenleft}{\kern0pt}metis\ carrier{\isacharunderscore}{\kern0pt}matD{\isacharparenleft}{\kern0pt}{\isadigit{2}}{\isacharparenright}{\kern0pt}\ control{\isadigit{2}}{\isacharunderscore}{\kern0pt}carrier{\isacharunderscore}{\kern0pt}mat\ index{\isacharunderscore}{\kern0pt}mult{\isacharunderscore}{\kern0pt}mat{\isacharparenleft}{\kern0pt}{\isadigit{3}}{\isacharparenright}{\kern0pt}\ \isanewline
\ \ \ \ \ \ \ \ index{\isacharunderscore}{\kern0pt}one{\isacharunderscore}{\kern0pt}mat{\isacharparenleft}{\kern0pt}{\isadigit{3}}{\isacharparenright}{\kern0pt}{\isacharparenright}{\kern0pt}\isanewline
\isacommand{qed}\isamarkupfalse%
%
\endisatagproof
{\isafoldproof}%
%
\isadelimproof
\isanewline
%
\endisadelimproof
\isanewline
\isacommand{lemma}\isamarkupfalse%
\ control{\isadigit{2}}{\isacharunderscore}{\kern0pt}is{\isacharunderscore}{\kern0pt}gate{\isacharcolon}{\kern0pt}\isanewline
\ \ \isakeyword{assumes}\ {\isachardoublequoteopen}gate\ {\isadigit{1}}\ U{\isachardoublequoteclose}\isanewline
\ \ \isakeyword{shows}\ {\isachardoublequoteopen}gate\ {\isadigit{2}}\ {\isacharparenleft}{\kern0pt}control{\isadigit{2}}\ U{\isacharparenright}{\kern0pt}{\isachardoublequoteclose}\isanewline
%
\isadelimproof
%
\endisadelimproof
%
\isatagproof
\isacommand{proof}\isamarkupfalse%
\isanewline
\ \ \isacommand{show}\isamarkupfalse%
\ {\isachardoublequoteopen}dim{\isacharunderscore}{\kern0pt}row\ {\isacharparenleft}{\kern0pt}control{\isadigit{2}}\ U{\isacharparenright}{\kern0pt}\ {\isacharequal}{\kern0pt}\ {\isadigit{2}}{\isacharcircum}{\kern0pt}{\isadigit{2}}{\isachardoublequoteclose}\ \isacommand{using}\isamarkupfalse%
\ control{\isadigit{2}}{\isacharunderscore}{\kern0pt}carrier{\isacharunderscore}{\kern0pt}mat\ \isanewline
\ \ \ \ \isacommand{by}\isamarkupfalse%
\ {\isacharparenleft}{\kern0pt}simp\ add{\isacharcolon}{\kern0pt}\ Tensor{\isachardot}{\kern0pt}mat{\isacharunderscore}{\kern0pt}of{\isacharunderscore}{\kern0pt}cols{\isacharunderscore}{\kern0pt}list{\isacharunderscore}{\kern0pt}def\ control{\isadigit{2}}{\isacharunderscore}{\kern0pt}def{\isacharparenright}{\kern0pt}\isanewline
\isacommand{next}\isamarkupfalse%
\isanewline
\ \ \isacommand{show}\isamarkupfalse%
\ {\isachardoublequoteopen}square{\isacharunderscore}{\kern0pt}mat\ {\isacharparenleft}{\kern0pt}control{\isadigit{2}}\ U{\isacharparenright}{\kern0pt}{\isachardoublequoteclose}\isanewline
\ \ \ \ \isacommand{by}\isamarkupfalse%
\ {\isacharparenleft}{\kern0pt}metis\ carrier{\isacharunderscore}{\kern0pt}matD{\isacharparenleft}{\kern0pt}{\isadigit{1}}{\isacharparenright}{\kern0pt}\ carrier{\isacharunderscore}{\kern0pt}matD{\isacharparenleft}{\kern0pt}{\isadigit{2}}{\isacharparenright}{\kern0pt}\ control{\isadigit{2}}{\isacharunderscore}{\kern0pt}carrier{\isacharunderscore}{\kern0pt}mat\ square{\isacharunderscore}{\kern0pt}mat{\isachardot}{\kern0pt}elims{\isacharparenleft}{\kern0pt}{\isadigit{3}}{\isacharparenright}{\kern0pt}{\isacharparenright}{\kern0pt}\isanewline
\isacommand{next}\isamarkupfalse%
\isanewline
\ \ \isacommand{show}\isamarkupfalse%
\ {\isachardoublequoteopen}unitary\ {\isacharparenleft}{\kern0pt}control{\isadigit{2}}\ U{\isacharparenright}{\kern0pt}{\isachardoublequoteclose}\ \isanewline
\ \ \ \ \isacommand{using}\isamarkupfalse%
\ control{\isadigit{2}}{\isacharunderscore}{\kern0pt}inv\ control{\isadigit{2}}{\isacharunderscore}{\kern0pt}inv{\isacharprime}{\kern0pt}\ unitary{\isacharunderscore}{\kern0pt}def\isanewline
\ \ \ \ \isacommand{by}\isamarkupfalse%
\ {\isacharparenleft}{\kern0pt}metis\ assms\ carrier{\isacharunderscore}{\kern0pt}matD{\isacharparenleft}{\kern0pt}{\isadigit{1}}{\isacharparenright}{\kern0pt}\ carrier{\isacharunderscore}{\kern0pt}matD{\isacharparenleft}{\kern0pt}{\isadigit{2}}{\isacharparenright}{\kern0pt}\ control{\isadigit{2}}{\isacharunderscore}{\kern0pt}carrier{\isacharunderscore}{\kern0pt}mat{\isacharparenright}{\kern0pt}\isanewline
\isacommand{qed}\isamarkupfalse%
%
\endisatagproof
{\isafoldproof}%
%
\isadelimproof
\isanewline
%
\endisadelimproof
\isanewline
\isacommand{lemma}\isamarkupfalse%
\ SWAP{\isacharunderscore}{\kern0pt}down{\isacharunderscore}{\kern0pt}is{\isacharunderscore}{\kern0pt}gate{\isacharcolon}{\kern0pt}\isanewline
\ \ \isakeyword{shows}\ {\isachardoublequoteopen}gate\ n\ {\isacharparenleft}{\kern0pt}SWAP{\isacharunderscore}{\kern0pt}down\ n{\isacharparenright}{\kern0pt}{\isachardoublequoteclose}\isanewline
%
\isadelimproof
%
\endisadelimproof
%
\isatagproof
\isacommand{proof}\isamarkupfalse%
\ {\isacharparenleft}{\kern0pt}induct\ n\ rule{\isacharcolon}{\kern0pt}\ SWAP{\isacharunderscore}{\kern0pt}down{\isachardot}{\kern0pt}induct{\isacharparenright}{\kern0pt}\isanewline
\ \ \isacommand{case}\isamarkupfalse%
\ {\isadigit{1}}\isanewline
\ \ \isacommand{then}\isamarkupfalse%
\ \isacommand{show}\isamarkupfalse%
\ {\isacharquery}{\kern0pt}case\isanewline
\ \ \isacommand{by}\isamarkupfalse%
\ {\isacharparenleft}{\kern0pt}metis\ Quantum{\isachardot}{\kern0pt}Id{\isacharunderscore}{\kern0pt}def\ SWAP{\isacharunderscore}{\kern0pt}down{\isachardot}{\kern0pt}simps{\isacharparenleft}{\kern0pt}{\isadigit{1}}{\isacharparenright}{\kern0pt}\ SWAP{\isacharunderscore}{\kern0pt}up{\isachardot}{\kern0pt}simps{\isacharparenleft}{\kern0pt}{\isadigit{1}}{\isacharparenright}{\kern0pt}\ SWAP{\isacharunderscore}{\kern0pt}up{\isacharunderscore}{\kern0pt}carrier{\isacharunderscore}{\kern0pt}mat\ \isanewline
\ \ \ \ \ \ carrier{\isacharunderscore}{\kern0pt}matD{\isacharparenleft}{\kern0pt}{\isadigit{2}}{\isacharparenright}{\kern0pt}\ id{\isacharunderscore}{\kern0pt}is{\isacharunderscore}{\kern0pt}gate\ index{\isacharunderscore}{\kern0pt}one{\isacharunderscore}{\kern0pt}mat{\isacharparenleft}{\kern0pt}{\isadigit{3}}{\isacharparenright}{\kern0pt}{\isacharparenright}{\kern0pt}\isanewline
\isacommand{next}\isamarkupfalse%
\isanewline
\ \ \isacommand{case}\isamarkupfalse%
\ {\isadigit{2}}\isanewline
\ \ \isacommand{then}\isamarkupfalse%
\ \isacommand{show}\isamarkupfalse%
\ {\isacharquery}{\kern0pt}case\isanewline
\ \ \ \ \isacommand{by}\isamarkupfalse%
\ {\isacharparenleft}{\kern0pt}metis\ H{\isacharunderscore}{\kern0pt}inv\ H{\isacharunderscore}{\kern0pt}is{\isacharunderscore}{\kern0pt}gate\ One{\isacharunderscore}{\kern0pt}nat{\isacharunderscore}{\kern0pt}def\ SWAP{\isacharunderscore}{\kern0pt}down{\isachardot}{\kern0pt}simps{\isacharparenleft}{\kern0pt}{\isadigit{2}}{\isacharparenright}{\kern0pt}\ prod{\isacharunderscore}{\kern0pt}of{\isacharunderscore}{\kern0pt}gate{\isacharunderscore}{\kern0pt}is{\isacharunderscore}{\kern0pt}gate{\isacharparenright}{\kern0pt}\isanewline
\isacommand{next}\isamarkupfalse%
\isanewline
\ \ \isacommand{case}\isamarkupfalse%
\ {\isadigit{3}}\isanewline
\ \ \isacommand{then}\isamarkupfalse%
\ \isacommand{show}\isamarkupfalse%
\ {\isacharquery}{\kern0pt}case\isanewline
\ \ \ \ \isacommand{by}\isamarkupfalse%
\ {\isacharparenleft}{\kern0pt}metis\ One{\isacharunderscore}{\kern0pt}nat{\isacharunderscore}{\kern0pt}def\ SWAP{\isacharunderscore}{\kern0pt}down{\isachardot}{\kern0pt}simps{\isacharparenleft}{\kern0pt}{\isadigit{3}}{\isacharparenright}{\kern0pt}\ SWAP{\isacharunderscore}{\kern0pt}is{\isacharunderscore}{\kern0pt}gate\ Suc{\isacharunderscore}{\kern0pt}{\isadigit{1}}{\isacharparenright}{\kern0pt}\isanewline
\isacommand{next}\isamarkupfalse%
\isanewline
\ \ \isacommand{case}\isamarkupfalse%
\ {\isacharparenleft}{\kern0pt}{\isadigit{4}}\ v{\isacharparenright}{\kern0pt}\isanewline
\ \ \isacommand{then}\isamarkupfalse%
\ \isacommand{show}\isamarkupfalse%
\ {\isacharquery}{\kern0pt}case\isanewline
\ \ \isacommand{proof}\isamarkupfalse%
\ {\isacharminus}{\kern0pt}\isanewline
\ \ \ \ \isacommand{assume}\isamarkupfalse%
\ HI{\isacharcolon}{\kern0pt}{\isachardoublequoteopen}gate\ {\isacharparenleft}{\kern0pt}Suc\ {\isacharparenleft}{\kern0pt}Suc\ v{\isacharparenright}{\kern0pt}{\isacharparenright}{\kern0pt}\ {\isacharparenleft}{\kern0pt}SWAP{\isacharunderscore}{\kern0pt}down\ {\isacharparenleft}{\kern0pt}Suc\ {\isacharparenleft}{\kern0pt}Suc\ v{\isacharparenright}{\kern0pt}{\isacharparenright}{\kern0pt}{\isacharparenright}{\kern0pt}{\isachardoublequoteclose}\isanewline
\ \ \ \ \isacommand{show}\isamarkupfalse%
\ {\isachardoublequoteopen}gate\ {\isacharparenleft}{\kern0pt}Suc\ {\isacharparenleft}{\kern0pt}Suc\ {\isacharparenleft}{\kern0pt}Suc\ v{\isacharparenright}{\kern0pt}{\isacharparenright}{\kern0pt}{\isacharparenright}{\kern0pt}\ {\isacharparenleft}{\kern0pt}SWAP{\isacharunderscore}{\kern0pt}down\ {\isacharparenleft}{\kern0pt}Suc\ {\isacharparenleft}{\kern0pt}Suc\ {\isacharparenleft}{\kern0pt}Suc\ v{\isacharparenright}{\kern0pt}{\isacharparenright}{\kern0pt}{\isacharparenright}{\kern0pt}{\isacharparenright}{\kern0pt}{\isachardoublequoteclose}\isanewline
\ \ \ \ \isacommand{proof}\isamarkupfalse%
\ {\isacharminus}{\kern0pt}\isanewline
\ \ \ \ \ \ \isacommand{have}\isamarkupfalse%
\ {\isachardoublequoteopen}gate\ {\isacharparenleft}{\kern0pt}Suc\ {\isacharparenleft}{\kern0pt}Suc\ {\isacharparenleft}{\kern0pt}Suc\ v{\isacharparenright}{\kern0pt}{\isacharparenright}{\kern0pt}{\isacharparenright}{\kern0pt}\ {\isacharparenleft}{\kern0pt}{\isacharparenleft}{\kern0pt}{\isacharparenleft}{\kern0pt}{\isadigit{1}}\isactrlsub m\ {\isacharparenleft}{\kern0pt}{\isadigit{2}}{\isacharcircum}{\kern0pt}Suc\ v{\isacharparenright}{\kern0pt}{\isacharparenright}{\kern0pt}\ {\isasymOtimes}\ SWAP{\isacharparenright}{\kern0pt}\ {\isacharasterisk}{\kern0pt}\ \isanewline
\ \ \ \ \ \ \ \ \ \ \ \ \ \ \ \ \ \ \ \ \ \ \ \ \ \ \ \ \ \ \ \ \ \ \ \ \ \ {\isacharparenleft}{\kern0pt}{\isacharparenleft}{\kern0pt}SWAP{\isacharunderscore}{\kern0pt}down\ {\isacharparenleft}{\kern0pt}Suc\ {\isacharparenleft}{\kern0pt}Suc\ v{\isacharparenright}{\kern0pt}{\isacharparenright}{\kern0pt}{\isacharparenright}{\kern0pt}\ {\isasymOtimes}\ {\isacharparenleft}{\kern0pt}{\isadigit{1}}\isactrlsub m\ {\isadigit{2}}{\isacharparenright}{\kern0pt}{\isacharparenright}{\kern0pt}{\isacharparenright}{\kern0pt}{\isachardoublequoteclose}\isanewline
\ \ \ \ \ \ \isacommand{proof}\isamarkupfalse%
\ {\isacharparenleft}{\kern0pt}rule\ prod{\isacharunderscore}{\kern0pt}of{\isacharunderscore}{\kern0pt}gate{\isacharunderscore}{\kern0pt}is{\isacharunderscore}{\kern0pt}gate{\isacharparenright}{\kern0pt}\isanewline
\ \ \ \ \ \ \ \ \isacommand{show}\isamarkupfalse%
\ {\isachardoublequoteopen}gate\ {\isacharparenleft}{\kern0pt}Suc\ {\isacharparenleft}{\kern0pt}Suc\ {\isacharparenleft}{\kern0pt}Suc\ v{\isacharparenright}{\kern0pt}{\isacharparenright}{\kern0pt}{\isacharparenright}{\kern0pt}\ {\isacharparenleft}{\kern0pt}{\isadigit{1}}\isactrlsub m\ {\isacharparenleft}{\kern0pt}{\isadigit{2}}\ {\isacharcircum}{\kern0pt}\ Suc\ v{\isacharparenright}{\kern0pt}\ {\isasymOtimes}\ SWAP{\isacharparenright}{\kern0pt}{\isachardoublequoteclose}\isanewline
\ \ \ \ \ \ \ \ \ \ \isacommand{using}\isamarkupfalse%
\ SWAP{\isacharunderscore}{\kern0pt}is{\isacharunderscore}{\kern0pt}gate\ tensor{\isacharunderscore}{\kern0pt}gate\isanewline
\ \ \ \ \ \ \ \ \ \ \isacommand{by}\isamarkupfalse%
\ {\isacharparenleft}{\kern0pt}metis\ Quantum{\isachardot}{\kern0pt}Id{\isacharunderscore}{\kern0pt}def\ add{\isacharunderscore}{\kern0pt}{\isadigit{2}}{\isacharunderscore}{\kern0pt}eq{\isacharunderscore}{\kern0pt}Suc{\isacharprime}{\kern0pt}\ id{\isacharunderscore}{\kern0pt}is{\isacharunderscore}{\kern0pt}gate{\isacharparenright}{\kern0pt}\isanewline
\ \ \ \ \ \ \isacommand{next}\isamarkupfalse%
\isanewline
\ \ \ \ \ \ \ \ \isacommand{show}\isamarkupfalse%
\ {\isachardoublequoteopen}gate\ {\isacharparenleft}{\kern0pt}Suc\ {\isacharparenleft}{\kern0pt}Suc\ {\isacharparenleft}{\kern0pt}Suc\ v{\isacharparenright}{\kern0pt}{\isacharparenright}{\kern0pt}{\isacharparenright}{\kern0pt}\ {\isacharparenleft}{\kern0pt}SWAP{\isacharunderscore}{\kern0pt}down\ {\isacharparenleft}{\kern0pt}Suc\ {\isacharparenleft}{\kern0pt}Suc\ v{\isacharparenright}{\kern0pt}{\isacharparenright}{\kern0pt}\ {\isasymOtimes}\ {\isadigit{1}}\isactrlsub m\ {\isadigit{2}}{\isacharparenright}{\kern0pt}{\isachardoublequoteclose}\isanewline
\ \ \ \ \ \ \ \ \ \ \isacommand{using}\isamarkupfalse%
\ HI\ tensor{\isacharunderscore}{\kern0pt}gate\isanewline
\ \ \ \ \ \ \ \ \ \ \isacommand{by}\isamarkupfalse%
\ {\isacharparenleft}{\kern0pt}metis\ Suc{\isacharunderscore}{\kern0pt}eq{\isacharunderscore}{\kern0pt}plus{\isadigit{1}}\ Y{\isacharunderscore}{\kern0pt}inv\ Y{\isacharunderscore}{\kern0pt}is{\isacharunderscore}{\kern0pt}gate\ prod{\isacharunderscore}{\kern0pt}of{\isacharunderscore}{\kern0pt}gate{\isacharunderscore}{\kern0pt}is{\isacharunderscore}{\kern0pt}gate{\isacharparenright}{\kern0pt}\isanewline
\ \ \ \ \ \ \isacommand{qed}\isamarkupfalse%
\isanewline
\ \ \ \ \ \ \isacommand{thus}\isamarkupfalse%
\ {\isacharquery}{\kern0pt}thesis\ \isacommand{using}\isamarkupfalse%
\ SWAP{\isacharunderscore}{\kern0pt}down{\isachardot}{\kern0pt}simps\ \isacommand{by}\isamarkupfalse%
\ auto\isanewline
\ \ \ \ \isacommand{qed}\isamarkupfalse%
\isanewline
\ \ \isacommand{qed}\isamarkupfalse%
\isanewline
\isacommand{qed}\isamarkupfalse%
%
\endisatagproof
{\isafoldproof}%
%
\isadelimproof
\isanewline
%
\endisadelimproof
\isanewline
\isacommand{lemma}\isamarkupfalse%
\ SWAP{\isacharunderscore}{\kern0pt}up{\isacharunderscore}{\kern0pt}is{\isacharunderscore}{\kern0pt}gate{\isacharcolon}{\kern0pt}\isanewline
\ \ \isakeyword{shows}\ {\isachardoublequoteopen}gate\ n\ {\isacharparenleft}{\kern0pt}SWAP{\isacharunderscore}{\kern0pt}up\ n{\isacharparenright}{\kern0pt}{\isachardoublequoteclose}\isanewline
%
\isadelimproof
%
\endisadelimproof
%
\isatagproof
\isacommand{proof}\isamarkupfalse%
\ {\isacharparenleft}{\kern0pt}induct\ n\ rule{\isacharcolon}{\kern0pt}\ SWAP{\isacharunderscore}{\kern0pt}up{\isachardot}{\kern0pt}induct{\isacharparenright}{\kern0pt}\isanewline
\ \ \isacommand{case}\isamarkupfalse%
\ {\isadigit{1}}\isanewline
\ \ \isacommand{then}\isamarkupfalse%
\ \isacommand{show}\isamarkupfalse%
\ {\isacharquery}{\kern0pt}case\ \isacommand{using}\isamarkupfalse%
\ id{\isacharunderscore}{\kern0pt}is{\isacharunderscore}{\kern0pt}gate\ SWAP{\isacharunderscore}{\kern0pt}up{\isachardot}{\kern0pt}simps\isanewline
\ \ \ \ \isacommand{by}\isamarkupfalse%
\ {\isacharparenleft}{\kern0pt}metis\ SWAP{\isacharunderscore}{\kern0pt}down{\isachardot}{\kern0pt}simps{\isacharparenleft}{\kern0pt}{\isadigit{1}}{\isacharparenright}{\kern0pt}\ SWAP{\isacharunderscore}{\kern0pt}down{\isacharunderscore}{\kern0pt}is{\isacharunderscore}{\kern0pt}gate{\isacharparenright}{\kern0pt}\isanewline
\isacommand{next}\isamarkupfalse%
\isanewline
\ \ \isacommand{case}\isamarkupfalse%
\ {\isadigit{2}}\isanewline
\ \ \isacommand{then}\isamarkupfalse%
\ \isacommand{show}\isamarkupfalse%
\ {\isacharquery}{\kern0pt}case\isanewline
\ \ \ \ \isacommand{by}\isamarkupfalse%
\ {\isacharparenleft}{\kern0pt}metis\ SWAP{\isacharunderscore}{\kern0pt}down{\isachardot}{\kern0pt}simps{\isacharparenleft}{\kern0pt}{\isadigit{2}}{\isacharparenright}{\kern0pt}\ SWAP{\isacharunderscore}{\kern0pt}down{\isacharunderscore}{\kern0pt}is{\isacharunderscore}{\kern0pt}gate\ SWAP{\isacharunderscore}{\kern0pt}up{\isachardot}{\kern0pt}simps{\isacharparenleft}{\kern0pt}{\isadigit{2}}{\isacharparenright}{\kern0pt}{\isacharparenright}{\kern0pt}\isanewline
\isacommand{next}\isamarkupfalse%
\isanewline
\ \ \isacommand{case}\isamarkupfalse%
\ {\isadigit{3}}\isanewline
\ \ \isacommand{then}\isamarkupfalse%
\ \isacommand{show}\isamarkupfalse%
\ {\isacharquery}{\kern0pt}case\ \isanewline
\ \ \ \ \isacommand{by}\isamarkupfalse%
\ {\isacharparenleft}{\kern0pt}metis\ One{\isacharunderscore}{\kern0pt}nat{\isacharunderscore}{\kern0pt}def\ SWAP{\isacharunderscore}{\kern0pt}is{\isacharunderscore}{\kern0pt}gate\ SWAP{\isacharunderscore}{\kern0pt}up{\isachardot}{\kern0pt}simps{\isacharparenleft}{\kern0pt}{\isadigit{3}}{\isacharparenright}{\kern0pt}\ Suc{\isacharunderscore}{\kern0pt}{\isadigit{1}}{\isacharparenright}{\kern0pt}\isanewline
\isacommand{next}\isamarkupfalse%
\isanewline
\ \ \isacommand{case}\isamarkupfalse%
\ {\isacharparenleft}{\kern0pt}{\isadigit{4}}\ v{\isacharparenright}{\kern0pt}\isanewline
\ \ \isacommand{then}\isamarkupfalse%
\ \isacommand{show}\isamarkupfalse%
\ {\isacharquery}{\kern0pt}case\isanewline
\ \ \isacommand{proof}\isamarkupfalse%
\ {\isacharminus}{\kern0pt}\isanewline
\ \ \ \ \isacommand{assume}\isamarkupfalse%
\ HI{\isacharcolon}{\kern0pt}{\isachardoublequoteopen}gate\ {\isacharparenleft}{\kern0pt}Suc\ {\isacharparenleft}{\kern0pt}Suc\ v{\isacharparenright}{\kern0pt}{\isacharparenright}{\kern0pt}\ {\isacharparenleft}{\kern0pt}SWAP{\isacharunderscore}{\kern0pt}up\ {\isacharparenleft}{\kern0pt}Suc\ {\isacharparenleft}{\kern0pt}Suc\ v{\isacharparenright}{\kern0pt}{\isacharparenright}{\kern0pt}{\isacharparenright}{\kern0pt}{\isachardoublequoteclose}\isanewline
\ \ \ \ \isacommand{show}\isamarkupfalse%
\ {\isachardoublequoteopen}gate\ {\isacharparenleft}{\kern0pt}Suc\ {\isacharparenleft}{\kern0pt}Suc\ {\isacharparenleft}{\kern0pt}Suc\ v{\isacharparenright}{\kern0pt}{\isacharparenright}{\kern0pt}{\isacharparenright}{\kern0pt}\ {\isacharparenleft}{\kern0pt}SWAP{\isacharunderscore}{\kern0pt}up\ {\isacharparenleft}{\kern0pt}Suc\ {\isacharparenleft}{\kern0pt}Suc\ {\isacharparenleft}{\kern0pt}Suc\ v{\isacharparenright}{\kern0pt}{\isacharparenright}{\kern0pt}{\isacharparenright}{\kern0pt}{\isacharparenright}{\kern0pt}{\isachardoublequoteclose}\isanewline
\ \ \ \ \isacommand{proof}\isamarkupfalse%
\ {\isacharminus}{\kern0pt}\isanewline
\ \ \ \ \ \ \isacommand{have}\isamarkupfalse%
\ {\isachardoublequoteopen}gate\ {\isacharparenleft}{\kern0pt}Suc\ {\isacharparenleft}{\kern0pt}Suc\ {\isacharparenleft}{\kern0pt}Suc\ v{\isacharparenright}{\kern0pt}{\isacharparenright}{\kern0pt}{\isacharparenright}{\kern0pt}\ {\isacharparenleft}{\kern0pt}{\isacharparenleft}{\kern0pt}SWAP\ {\isasymOtimes}\ {\isacharparenleft}{\kern0pt}{\isadigit{1}}\isactrlsub m\ {\isacharparenleft}{\kern0pt}{\isadigit{2}}{\isacharcircum}{\kern0pt}{\isacharparenleft}{\kern0pt}Suc\ v{\isacharparenright}{\kern0pt}{\isacharparenright}{\kern0pt}{\isacharparenright}{\kern0pt}{\isacharparenright}{\kern0pt}\ {\isacharasterisk}{\kern0pt}\ {\isacharparenleft}{\kern0pt}{\isacharparenleft}{\kern0pt}{\isadigit{1}}\isactrlsub m\ {\isadigit{2}}{\isacharparenright}{\kern0pt}\ {\isasymOtimes}\ \isanewline
\ \ \ \ \ \ \ \ \ \ \ \ \ \ \ \ \ \ \ \ \ \ \ \ \ \ \ \ \ \ \ \ \ \ \ \ \ \ {\isacharparenleft}{\kern0pt}SWAP{\isacharunderscore}{\kern0pt}up\ {\isacharparenleft}{\kern0pt}Suc\ {\isacharparenleft}{\kern0pt}Suc\ v{\isacharparenright}{\kern0pt}{\isacharparenright}{\kern0pt}{\isacharparenright}{\kern0pt}{\isacharparenright}{\kern0pt}{\isacharparenright}{\kern0pt}{\isachardoublequoteclose}\isanewline
\ \ \ \ \ \ \isacommand{proof}\isamarkupfalse%
\ {\isacharparenleft}{\kern0pt}rule\ prod{\isacharunderscore}{\kern0pt}of{\isacharunderscore}{\kern0pt}gate{\isacharunderscore}{\kern0pt}is{\isacharunderscore}{\kern0pt}gate{\isacharparenright}{\kern0pt}\isanewline
\ \ \ \ \ \ \ \ \isacommand{show}\isamarkupfalse%
\ {\isachardoublequoteopen}gate\ {\isacharparenleft}{\kern0pt}Suc\ {\isacharparenleft}{\kern0pt}Suc\ {\isacharparenleft}{\kern0pt}Suc\ v{\isacharparenright}{\kern0pt}{\isacharparenright}{\kern0pt}{\isacharparenright}{\kern0pt}\ {\isacharparenleft}{\kern0pt}SWAP\ {\isasymOtimes}\ {\isadigit{1}}\isactrlsub m\ {\isacharparenleft}{\kern0pt}{\isadigit{2}}\ {\isacharcircum}{\kern0pt}\ Suc\ v{\isacharparenright}{\kern0pt}{\isacharparenright}{\kern0pt}{\isachardoublequoteclose}\isanewline
\ \ \ \ \ \ \ \ \ \ \isacommand{using}\isamarkupfalse%
\ tensor{\isacharunderscore}{\kern0pt}gate\ SWAP{\isacharunderscore}{\kern0pt}is{\isacharunderscore}{\kern0pt}gate\isanewline
\ \ \ \ \ \ \ \ \ \ \isacommand{by}\isamarkupfalse%
\ {\isacharparenleft}{\kern0pt}metis\ Quantum{\isachardot}{\kern0pt}Id{\isacharunderscore}{\kern0pt}def\ add{\isacharunderscore}{\kern0pt}{\isadigit{2}}{\isacharunderscore}{\kern0pt}eq{\isacharunderscore}{\kern0pt}Suc\ id{\isacharunderscore}{\kern0pt}is{\isacharunderscore}{\kern0pt}gate{\isacharparenright}{\kern0pt}\isanewline
\ \ \ \ \ \ \isacommand{next}\isamarkupfalse%
\isanewline
\ \ \ \ \ \ \ \ \isacommand{show}\isamarkupfalse%
\ {\isachardoublequoteopen}gate\ {\isacharparenleft}{\kern0pt}Suc\ {\isacharparenleft}{\kern0pt}Suc\ {\isacharparenleft}{\kern0pt}Suc\ v{\isacharparenright}{\kern0pt}{\isacharparenright}{\kern0pt}{\isacharparenright}{\kern0pt}\ {\isacharparenleft}{\kern0pt}{\isadigit{1}}\isactrlsub m\ {\isadigit{2}}\ {\isasymOtimes}\ SWAP{\isacharunderscore}{\kern0pt}up\ {\isacharparenleft}{\kern0pt}Suc\ {\isacharparenleft}{\kern0pt}Suc\ v{\isacharparenright}{\kern0pt}{\isacharparenright}{\kern0pt}{\isacharparenright}{\kern0pt}{\isachardoublequoteclose}\isanewline
\ \ \ \ \ \ \ \ \ \ \isacommand{using}\isamarkupfalse%
\ tensor{\isacharunderscore}{\kern0pt}gate\ HI\ \isanewline
\ \ \ \ \ \ \ \ \ \ \isacommand{by}\isamarkupfalse%
\ {\isacharparenleft}{\kern0pt}metis\ One{\isacharunderscore}{\kern0pt}nat{\isacharunderscore}{\kern0pt}def\ SWAP{\isacharunderscore}{\kern0pt}down{\isachardot}{\kern0pt}simps{\isacharparenleft}{\kern0pt}{\isadigit{2}}{\isacharparenright}{\kern0pt}\ SWAP{\isacharunderscore}{\kern0pt}down{\isacharunderscore}{\kern0pt}is{\isacharunderscore}{\kern0pt}gate\ plus{\isacharunderscore}{\kern0pt}{\isadigit{1}}{\isacharunderscore}{\kern0pt}eq{\isacharunderscore}{\kern0pt}Suc{\isacharparenright}{\kern0pt}\isanewline
\ \ \ \ \ \ \isacommand{qed}\isamarkupfalse%
\isanewline
\ \ \ \ \ \ \isacommand{thus}\isamarkupfalse%
\ {\isacharquery}{\kern0pt}thesis\ \isacommand{using}\isamarkupfalse%
\ SWAP{\isacharunderscore}{\kern0pt}up{\isachardot}{\kern0pt}simps{\isacharparenleft}{\kern0pt}{\isadigit{3}}{\isacharparenright}{\kern0pt}\ \isacommand{by}\isamarkupfalse%
\ simp\isanewline
\ \ \ \ \isacommand{qed}\isamarkupfalse%
\isanewline
\ \ \isacommand{qed}\isamarkupfalse%
\isanewline
\isacommand{qed}\isamarkupfalse%
%
\endisatagproof
{\isafoldproof}%
%
\isadelimproof
\isanewline
%
\endisadelimproof
\isanewline
\isacommand{lemma}\isamarkupfalse%
\ control{\isacharunderscore}{\kern0pt}is{\isacharunderscore}{\kern0pt}gate{\isacharcolon}{\kern0pt}\isanewline
\ \ \isakeyword{assumes}\ {\isachardoublequoteopen}gate\ {\isadigit{1}}\ U{\isachardoublequoteclose}\isanewline
\ \ \isakeyword{shows}\ {\isachardoublequoteopen}gate\ n\ {\isacharparenleft}{\kern0pt}control\ n\ U{\isacharparenright}{\kern0pt}{\isachardoublequoteclose}\isanewline
%
\isadelimproof
%
\endisadelimproof
%
\isatagproof
\isacommand{proof}\isamarkupfalse%
\ {\isacharparenleft}{\kern0pt}cases\ n{\isacharparenright}{\kern0pt}\isanewline
\ \ \isacommand{case}\isamarkupfalse%
\ {\isadigit{0}}\isanewline
\ \ \isacommand{then}\isamarkupfalse%
\ \isacommand{show}\isamarkupfalse%
\ {\isacharquery}{\kern0pt}thesis\isanewline
\ \ \ \ \isacommand{by}\isamarkupfalse%
\ {\isacharparenleft}{\kern0pt}metis\ SWAP{\isacharunderscore}{\kern0pt}up{\isachardot}{\kern0pt}simps{\isacharparenleft}{\kern0pt}{\isadigit{1}}{\isacharparenright}{\kern0pt}\ SWAP{\isacharunderscore}{\kern0pt}up{\isacharunderscore}{\kern0pt}is{\isacharunderscore}{\kern0pt}gate\ control{\isachardot}{\kern0pt}simps{\isacharparenleft}{\kern0pt}{\isadigit{1}}{\isacharparenright}{\kern0pt}{\isacharparenright}{\kern0pt}\isanewline
\isacommand{next}\isamarkupfalse%
\isanewline
\ \ \isacommand{case}\isamarkupfalse%
\ {\isacharparenleft}{\kern0pt}Suc\ nat{\isacharparenright}{\kern0pt}\isanewline
\ \ \isacommand{then}\isamarkupfalse%
\ \isacommand{show}\isamarkupfalse%
\ {\isacharquery}{\kern0pt}thesis\isanewline
\ \ \isacommand{proof}\isamarkupfalse%
\ {\isacharminus}{\kern0pt}\isanewline
\ \ \ \ \isacommand{assume}\isamarkupfalse%
\ nnat{\isacharcolon}{\kern0pt}{\isachardoublequoteopen}n\ {\isacharequal}{\kern0pt}\ Suc\ nat{\isachardoublequoteclose}\isanewline
\ \ \ \ \isacommand{show}\isamarkupfalse%
\ {\isachardoublequoteopen}gate\ n\ {\isacharparenleft}{\kern0pt}control\ n\ U{\isacharparenright}{\kern0pt}{\isachardoublequoteclose}\isanewline
\ \ \ \ \isacommand{proof}\isamarkupfalse%
\ {\isacharminus}{\kern0pt}\isanewline
\ \ \ \ \ \ \isacommand{have}\isamarkupfalse%
\ {\isachardoublequoteopen}gate\ {\isacharparenleft}{\kern0pt}Suc\ nat{\isacharparenright}{\kern0pt}\ {\isacharparenleft}{\kern0pt}control\ {\isacharparenleft}{\kern0pt}Suc\ nat{\isacharparenright}{\kern0pt}\ U{\isacharparenright}{\kern0pt}{\isachardoublequoteclose}\isanewline
\ \ \ \ \ \ \isacommand{proof}\isamarkupfalse%
\ {\isacharparenleft}{\kern0pt}cases\ nat{\isacharparenright}{\kern0pt}\isanewline
\ \ \ \ \ \ \ \ \isacommand{case}\isamarkupfalse%
\ {\isadigit{0}}\isanewline
\ \ \ \ \ \ \ \ \isacommand{then}\isamarkupfalse%
\ \isacommand{show}\isamarkupfalse%
\ {\isacharquery}{\kern0pt}thesis\ \isanewline
\ \ \ \ \ \ \ \ \ \ \isacommand{by}\isamarkupfalse%
\ {\isacharparenleft}{\kern0pt}simp\ add{\isacharcolon}{\kern0pt}\ gate{\isacharunderscore}{\kern0pt}def{\isacharparenright}{\kern0pt}\isanewline
\ \ \ \ \ \ \isacommand{next}\isamarkupfalse%
\isanewline
\ \ \ \ \ \ \ \ \isacommand{case}\isamarkupfalse%
\ {\isacharparenleft}{\kern0pt}Suc\ nata{\isacharparenright}{\kern0pt}\isanewline
\ \ \ \ \ \ \ \ \isacommand{then}\isamarkupfalse%
\ \isacommand{show}\isamarkupfalse%
\ {\isacharquery}{\kern0pt}thesis\isanewline
\ \ \ \ \ \ \ \ \isacommand{proof}\isamarkupfalse%
\ {\isacharminus}{\kern0pt}\isanewline
\ \ \ \ \ \ \ \ \ \ \isacommand{assume}\isamarkupfalse%
\ nnat{\isacharunderscore}{\kern0pt}{\isacharcolon}{\kern0pt}{\isachardoublequoteopen}nat\ {\isacharequal}{\kern0pt}\ Suc\ nata{\isachardoublequoteclose}\isanewline
\ \ \ \ \ \ \ \ \ \ \isacommand{show}\isamarkupfalse%
\ {\isachardoublequoteopen}gate\ {\isacharparenleft}{\kern0pt}Suc\ nat{\isacharparenright}{\kern0pt}\ {\isacharparenleft}{\kern0pt}control\ {\isacharparenleft}{\kern0pt}Suc\ nat{\isacharparenright}{\kern0pt}\ U{\isacharparenright}{\kern0pt}{\isachardoublequoteclose}\isanewline
\ \ \ \ \ \ \ \ \ \ \isacommand{proof}\isamarkupfalse%
\ {\isacharminus}{\kern0pt}\isanewline
\ \ \ \ \ \ \ \ \ \ \ \ \isacommand{have}\isamarkupfalse%
\ {\isachardoublequoteopen}gate\ {\isacharparenleft}{\kern0pt}Suc\ {\isacharparenleft}{\kern0pt}Suc\ nata{\isacharparenright}{\kern0pt}{\isacharparenright}{\kern0pt}\ {\isacharparenleft}{\kern0pt}control\ {\isacharparenleft}{\kern0pt}Suc\ {\isacharparenleft}{\kern0pt}Suc\ nata{\isacharparenright}{\kern0pt}{\isacharparenright}{\kern0pt}\ U{\isacharparenright}{\kern0pt}{\isachardoublequoteclose}\isanewline
\ \ \ \ \ \ \ \ \ \ \ \ \isacommand{proof}\isamarkupfalse%
\ {\isacharparenleft}{\kern0pt}cases\ nata{\isacharparenright}{\kern0pt}\isanewline
\ \ \ \ \ \ \ \ \ \ \ \ \ \ \isacommand{case}\isamarkupfalse%
\ {\isadigit{0}}\isanewline
\ \ \ \ \ \ \ \ \ \ \ \ \ \ \isacommand{then}\isamarkupfalse%
\ \isacommand{show}\isamarkupfalse%
\ {\isacharquery}{\kern0pt}thesis\isanewline
\ \ \ \ \ \ \ \ \ \ \ \ \ \ \ \ \isacommand{using}\isamarkupfalse%
\ One{\isacharunderscore}{\kern0pt}nat{\isacharunderscore}{\kern0pt}def\ Suc{\isacharunderscore}{\kern0pt}{\isadigit{1}}\ assms\ control{\isachardot}{\kern0pt}simps{\isacharparenleft}{\kern0pt}{\isadigit{3}}{\isacharparenright}{\kern0pt}\ control{\isadigit{2}}{\isacharunderscore}{\kern0pt}is{\isacharunderscore}{\kern0pt}gate\ \isacommand{by}\isamarkupfalse%
\ presburger\isanewline
\ \ \ \ \ \ \ \ \ \ \ \ \isacommand{next}\isamarkupfalse%
\isanewline
\ \ \ \ \ \ \ \ \ \ \ \ \ \ \isacommand{case}\isamarkupfalse%
\ {\isacharparenleft}{\kern0pt}Suc\ natb{\isacharparenright}{\kern0pt}\isanewline
\ \ \ \ \ \ \ \ \ \ \ \ \ \ \isacommand{then}\isamarkupfalse%
\ \isacommand{show}\isamarkupfalse%
\ {\isacharquery}{\kern0pt}thesis\isanewline
\ \ \ \ \ \ \ \ \ \ \ \ \ \ \isacommand{proof}\isamarkupfalse%
\ {\isacharminus}{\kern0pt}\isanewline
\ \ \ \ \ \ \ \ \ \ \ \ \ \ \ \ \isacommand{assume}\isamarkupfalse%
\ nnatb{\isacharcolon}{\kern0pt}{\isachardoublequoteopen}nata\ {\isacharequal}{\kern0pt}\ Suc\ natb{\isachardoublequoteclose}\isanewline
\ \ \ \ \ \ \ \ \ \ \ \ \ \ \ \ \isacommand{show}\isamarkupfalse%
\ {\isachardoublequoteopen}gate\ {\isacharparenleft}{\kern0pt}Suc\ {\isacharparenleft}{\kern0pt}Suc\ nata{\isacharparenright}{\kern0pt}{\isacharparenright}{\kern0pt}\ {\isacharparenleft}{\kern0pt}control\ {\isacharparenleft}{\kern0pt}Suc\ {\isacharparenleft}{\kern0pt}Suc\ nata{\isacharparenright}{\kern0pt}{\isacharparenright}{\kern0pt}\ U{\isacharparenright}{\kern0pt}{\isachardoublequoteclose}\isanewline
\ \ \ \ \ \ \ \ \ \ \ \ \ \ \ \ \isacommand{proof}\isamarkupfalse%
\ {\isacharminus}{\kern0pt}\isanewline
\ \ \ \ \ \ \ \ \ \ \ \ \ \ \ \ \ \ \isacommand{have}\isamarkupfalse%
\ {\isachardoublequoteopen}gate\ {\isacharparenleft}{\kern0pt}Suc\ {\isacharparenleft}{\kern0pt}Suc\ {\isacharparenleft}{\kern0pt}Suc\ natb{\isacharparenright}{\kern0pt}{\isacharparenright}{\kern0pt}{\isacharparenright}{\kern0pt}\ {\isacharparenleft}{\kern0pt}control\ {\isacharparenleft}{\kern0pt}Suc\ {\isacharparenleft}{\kern0pt}Suc\ {\isacharparenleft}{\kern0pt}Suc\ natb{\isacharparenright}{\kern0pt}{\isacharparenright}{\kern0pt}{\isacharparenright}{\kern0pt}\ U{\isacharparenright}{\kern0pt}{\isachardoublequoteclose}\isanewline
\ \ \ \ \ \ \ \ \ \ \ \ \ \ \ \ \ \ \isacommand{proof}\isamarkupfalse%
\ {\isacharminus}{\kern0pt}\isanewline
\ \ \ \ \ \ \ \ \ \ \ \ \ \ \ \ \ \ \ \ \isacommand{have}\isamarkupfalse%
\ {\isachardoublequoteopen}gate\ {\isacharparenleft}{\kern0pt}Suc\ {\isacharparenleft}{\kern0pt}Suc\ {\isacharparenleft}{\kern0pt}Suc\ natb{\isacharparenright}{\kern0pt}{\isacharparenright}{\kern0pt}{\isacharparenright}{\kern0pt}\ {\isacharparenleft}{\kern0pt}{\isacharparenleft}{\kern0pt}{\isacharparenleft}{\kern0pt}{\isadigit{1}}\isactrlsub m\ {\isadigit{2}}{\isacharparenright}{\kern0pt}\ {\isasymOtimes}\ SWAP{\isacharunderscore}{\kern0pt}down\ {\isacharparenleft}{\kern0pt}Suc\ {\isacharparenleft}{\kern0pt}Suc\ natb{\isacharparenright}{\kern0pt}{\isacharparenright}{\kern0pt}{\isacharparenright}{\kern0pt}\ {\isacharasterisk}{\kern0pt}\ \isanewline
\ \ \ \ \ \ \ \ \ \ \ \ \ \ \ \ \ \ \ \ \ \ \ \ {\isacharparenleft}{\kern0pt}control{\isadigit{2}}\ U\ {\isasymOtimes}\ {\isacharparenleft}{\kern0pt}{\isadigit{1}}\isactrlsub m\ {\isacharparenleft}{\kern0pt}{\isadigit{2}}{\isacharcircum}{\kern0pt}{\isacharparenleft}{\kern0pt}Suc\ natb{\isacharparenright}{\kern0pt}{\isacharparenright}{\kern0pt}{\isacharparenright}{\kern0pt}{\isacharparenright}{\kern0pt}\ {\isacharasterisk}{\kern0pt}\ {\isacharparenleft}{\kern0pt}{\isacharparenleft}{\kern0pt}{\isadigit{1}}\isactrlsub m\ {\isadigit{2}}{\isacharparenright}{\kern0pt}\ {\isasymOtimes}\ SWAP{\isacharunderscore}{\kern0pt}up\ {\isacharparenleft}{\kern0pt}Suc\ {\isacharparenleft}{\kern0pt}Suc\ natb{\isacharparenright}{\kern0pt}{\isacharparenright}{\kern0pt}{\isacharparenright}{\kern0pt}{\isacharparenright}{\kern0pt}{\isachardoublequoteclose}\isanewline
\ \ \ \ \ \ \ \ \ \ \ \ \ \ \ \ \ \ \ \ \isacommand{proof}\isamarkupfalse%
\ {\isacharparenleft}{\kern0pt}rule\ prod{\isacharunderscore}{\kern0pt}of{\isacharunderscore}{\kern0pt}gate{\isacharunderscore}{\kern0pt}is{\isacharunderscore}{\kern0pt}gate{\isacharparenright}{\kern0pt}{\isacharplus}{\kern0pt}\isanewline
\ \ \ \ \ \ \ \ \ \ \ \ \ \ \ \ \ \ \ \ \ \ \isacommand{show}\isamarkupfalse%
\ {\isachardoublequoteopen}gate\ {\isacharparenleft}{\kern0pt}Suc\ {\isacharparenleft}{\kern0pt}Suc\ {\isacharparenleft}{\kern0pt}Suc\ natb{\isacharparenright}{\kern0pt}{\isacharparenright}{\kern0pt}{\isacharparenright}{\kern0pt}\ {\isacharparenleft}{\kern0pt}{\isadigit{1}}\isactrlsub m\ {\isadigit{2}}\ {\isasymOtimes}\ SWAP{\isacharunderscore}{\kern0pt}down\ {\isacharparenleft}{\kern0pt}Suc\ {\isacharparenleft}{\kern0pt}Suc\ natb{\isacharparenright}{\kern0pt}{\isacharparenright}{\kern0pt}{\isacharparenright}{\kern0pt}{\isachardoublequoteclose}\isanewline
\ \ \ \ \ \ \ \ \ \ \ \ \ \ \ \ \ \ \ \ \ \ \ \ \isacommand{using}\isamarkupfalse%
\ SWAP{\isacharunderscore}{\kern0pt}down{\isacharunderscore}{\kern0pt}is{\isacharunderscore}{\kern0pt}gate\ id{\isacharunderscore}{\kern0pt}is{\isacharunderscore}{\kern0pt}gate\ tensor{\isacharunderscore}{\kern0pt}gate\isanewline
\ \ \ \ \ \ \ \ \ \ \ \ \ \ \ \ \ \ \ \ \ \ \ \ \isacommand{by}\isamarkupfalse%
\ {\isacharparenleft}{\kern0pt}metis\ One{\isacharunderscore}{\kern0pt}nat{\isacharunderscore}{\kern0pt}def\ SWAP{\isacharunderscore}{\kern0pt}up{\isachardot}{\kern0pt}simps{\isacharparenleft}{\kern0pt}{\isadigit{2}}{\isacharparenright}{\kern0pt}\ SWAP{\isacharunderscore}{\kern0pt}up{\isacharunderscore}{\kern0pt}is{\isacharunderscore}{\kern0pt}gate\ plus{\isacharunderscore}{\kern0pt}{\isadigit{1}}{\isacharunderscore}{\kern0pt}eq{\isacharunderscore}{\kern0pt}Suc{\isacharparenright}{\kern0pt}\isanewline
\ \ \ \ \ \ \ \ \ \ \ \ \ \ \ \ \ \ \ \ \isacommand{next}\isamarkupfalse%
\isanewline
\ \ \ \ \ \ \ \ \ \ \ \ \ \ \ \ \ \ \ \ \ \ \isacommand{show}\isamarkupfalse%
\ {\isachardoublequoteopen}gate\ {\isacharparenleft}{\kern0pt}Suc\ {\isacharparenleft}{\kern0pt}Suc\ {\isacharparenleft}{\kern0pt}Suc\ natb{\isacharparenright}{\kern0pt}{\isacharparenright}{\kern0pt}{\isacharparenright}{\kern0pt}\ {\isacharparenleft}{\kern0pt}control{\isadigit{2}}\ U\ {\isasymOtimes}\ {\isadigit{1}}\isactrlsub m\ {\isacharparenleft}{\kern0pt}{\isadigit{2}}\ {\isacharcircum}{\kern0pt}\ Suc\ natb{\isacharparenright}{\kern0pt}{\isacharparenright}{\kern0pt}{\isachardoublequoteclose}\isanewline
\ \ \ \ \ \ \ \ \ \ \ \ \ \ \ \ \ \ \ \ \ \ \ \ \isacommand{using}\isamarkupfalse%
\ control{\isadigit{2}}{\isacharunderscore}{\kern0pt}is{\isacharunderscore}{\kern0pt}gate\ id{\isacharunderscore}{\kern0pt}is{\isacharunderscore}{\kern0pt}gate\ tensor{\isacharunderscore}{\kern0pt}gate\isanewline
\ \ \ \ \ \ \ \ \ \ \ \ \ \ \ \ \ \ \ \ \ \ \ \ \isacommand{by}\isamarkupfalse%
\ {\isacharparenleft}{\kern0pt}metis\ Quantum{\isachardot}{\kern0pt}Id{\isacharunderscore}{\kern0pt}def\ add{\isacharunderscore}{\kern0pt}{\isadigit{2}}{\isacharunderscore}{\kern0pt}eq{\isacharunderscore}{\kern0pt}Suc\ assms{\isacharparenright}{\kern0pt}\isanewline
\ \ \ \ \ \ \ \ \ \ \ \ \ \ \ \ \ \ \ \ \isacommand{next}\isamarkupfalse%
\isanewline
\ \ \ \ \ \ \ \ \ \ \ \ \ \ \ \ \ \ \ \ \ \ \isacommand{show}\isamarkupfalse%
\ {\isachardoublequoteopen}gate\ {\isacharparenleft}{\kern0pt}Suc\ {\isacharparenleft}{\kern0pt}Suc\ {\isacharparenleft}{\kern0pt}Suc\ natb{\isacharparenright}{\kern0pt}{\isacharparenright}{\kern0pt}{\isacharparenright}{\kern0pt}\ {\isacharparenleft}{\kern0pt}{\isadigit{1}}\isactrlsub m\ {\isadigit{2}}\ {\isasymOtimes}\ SWAP{\isacharunderscore}{\kern0pt}up\ {\isacharparenleft}{\kern0pt}Suc\ {\isacharparenleft}{\kern0pt}Suc\ natb{\isacharparenright}{\kern0pt}{\isacharparenright}{\kern0pt}{\isacharparenright}{\kern0pt}{\isachardoublequoteclose}\isanewline
\ \ \ \ \ \ \ \ \ \ \ \ \ \ \ \ \ \ \ \ \ \ \ \ \isacommand{using}\isamarkupfalse%
\ SWAP{\isacharunderscore}{\kern0pt}up{\isacharunderscore}{\kern0pt}is{\isacharunderscore}{\kern0pt}gate\ id{\isacharunderscore}{\kern0pt}is{\isacharunderscore}{\kern0pt}gate\ tensor{\isacharunderscore}{\kern0pt}gate\isanewline
\ \ \ \ \ \ \ \ \ \ \ \ \ \ \ \ \ \ \ \ \ \ \ \ \isacommand{by}\isamarkupfalse%
\ {\isacharparenleft}{\kern0pt}metis\ Y{\isacharunderscore}{\kern0pt}inv\ Y{\isacharunderscore}{\kern0pt}is{\isacharunderscore}{\kern0pt}gate\ plus{\isacharunderscore}{\kern0pt}{\isadigit{1}}{\isacharunderscore}{\kern0pt}eq{\isacharunderscore}{\kern0pt}Suc\ prod{\isacharunderscore}{\kern0pt}of{\isacharunderscore}{\kern0pt}gate{\isacharunderscore}{\kern0pt}is{\isacharunderscore}{\kern0pt}gate{\isacharparenright}{\kern0pt}\isanewline
\ \ \ \ \ \ \ \ \ \ \ \ \ \ \ \ \ \ \ \ \isacommand{qed}\isamarkupfalse%
\isanewline
\ \ \ \ \ \ \ \ \ \ \ \ \ \ \ \ \ \ \ \ \isacommand{thus}\isamarkupfalse%
\ {\isacharquery}{\kern0pt}thesis\ \isacommand{using}\isamarkupfalse%
\ control{\isachardot}{\kern0pt}simps\ \isacommand{by}\isamarkupfalse%
\ simp\isanewline
\ \ \ \ \ \ \ \ \ \ \ \ \ \ \ \ \ \ \isacommand{qed}\isamarkupfalse%
\isanewline
\ \ \ \ \ \ \ \ \ \ \ \ \ \ \ \ \ \ \isacommand{thus}\isamarkupfalse%
\ {\isacharquery}{\kern0pt}thesis\ \isacommand{using}\isamarkupfalse%
\ nnatb\ \isacommand{by}\isamarkupfalse%
\ simp\isanewline
\ \ \ \ \ \ \ \ \ \ \ \ \ \ \ \ \isacommand{qed}\isamarkupfalse%
\isanewline
\ \ \ \ \ \ \ \ \ \ \ \ \ \ \isacommand{qed}\isamarkupfalse%
\isanewline
\ \ \ \ \ \ \ \ \ \ \ \ \isacommand{qed}\isamarkupfalse%
\isanewline
\ \ \ \ \ \ \ \ \ \ \ \ \isacommand{thus}\isamarkupfalse%
\ {\isacharquery}{\kern0pt}thesis\ \isacommand{using}\isamarkupfalse%
\ nnat{\isacharunderscore}{\kern0pt}\ \isacommand{by}\isamarkupfalse%
\ simp\isanewline
\ \ \ \ \ \ \ \ \ \ \isacommand{qed}\isamarkupfalse%
\isanewline
\ \ \ \ \ \ \ \ \isacommand{qed}\isamarkupfalse%
\isanewline
\ \ \ \ \ \ \isacommand{qed}\isamarkupfalse%
\isanewline
\ \ \ \ \ \ \isacommand{thus}\isamarkupfalse%
\ {\isacharquery}{\kern0pt}thesis\ \isacommand{using}\isamarkupfalse%
\ nnat\ \isacommand{by}\isamarkupfalse%
\ simp\isanewline
\ \ \ \ \isacommand{qed}\isamarkupfalse%
\isanewline
\ \ \isacommand{qed}\isamarkupfalse%
\isanewline
\isacommand{qed}\isamarkupfalse%
%
\endisatagproof
{\isafoldproof}%
%
\isadelimproof
\isanewline
%
\endisadelimproof
\isanewline
\isacommand{lemma}\isamarkupfalse%
\ controlled{\isacharunderscore}{\kern0pt}rotations{\isacharunderscore}{\kern0pt}is{\isacharunderscore}{\kern0pt}gate{\isacharcolon}{\kern0pt}\isanewline
\ \ \isakeyword{shows}\ {\isachardoublequoteopen}gate\ n\ {\isacharparenleft}{\kern0pt}controlled{\isacharunderscore}{\kern0pt}rotations\ n{\isacharparenright}{\kern0pt}{\isachardoublequoteclose}\isanewline
%
\isadelimproof
%
\endisadelimproof
%
\isatagproof
\isacommand{proof}\isamarkupfalse%
\ {\isacharparenleft}{\kern0pt}induct\ n\ rule{\isacharcolon}{\kern0pt}\ controlled{\isacharunderscore}{\kern0pt}rotations{\isachardot}{\kern0pt}induct{\isacharparenright}{\kern0pt}\isanewline
\ \ \isacommand{case}\isamarkupfalse%
\ {\isadigit{1}}\isanewline
\ \ \isacommand{then}\isamarkupfalse%
\ \isacommand{show}\isamarkupfalse%
\ {\isacharquery}{\kern0pt}case\ \isanewline
\ \ \ \ \isacommand{by}\isamarkupfalse%
\ {\isacharparenleft}{\kern0pt}metis\ SWAP{\isacharunderscore}{\kern0pt}down{\isachardot}{\kern0pt}simps{\isacharparenleft}{\kern0pt}{\isadigit{1}}{\isacharparenright}{\kern0pt}\ SWAP{\isacharunderscore}{\kern0pt}down{\isacharunderscore}{\kern0pt}is{\isacharunderscore}{\kern0pt}gate\ controlled{\isacharunderscore}{\kern0pt}rotations{\isachardot}{\kern0pt}simps{\isacharparenleft}{\kern0pt}{\isadigit{1}}{\isacharparenright}{\kern0pt}{\isacharparenright}{\kern0pt}\isanewline
\isacommand{next}\isamarkupfalse%
\isanewline
\ \ \isacommand{case}\isamarkupfalse%
\ {\isadigit{2}}\isanewline
\ \ \isacommand{then}\isamarkupfalse%
\ \isacommand{show}\isamarkupfalse%
\ {\isacharquery}{\kern0pt}case\ \isanewline
\ \ \ \ \isacommand{by}\isamarkupfalse%
\ {\isacharparenleft}{\kern0pt}metis\ SWAP{\isacharunderscore}{\kern0pt}down{\isachardot}{\kern0pt}simps{\isacharparenleft}{\kern0pt}{\isadigit{2}}{\isacharparenright}{\kern0pt}\ SWAP{\isacharunderscore}{\kern0pt}down{\isacharunderscore}{\kern0pt}is{\isacharunderscore}{\kern0pt}gate\ controlled{\isacharunderscore}{\kern0pt}rotations{\isachardot}{\kern0pt}simps{\isacharparenleft}{\kern0pt}{\isadigit{2}}{\isacharparenright}{\kern0pt}{\isacharparenright}{\kern0pt}\isanewline
\isacommand{next}\isamarkupfalse%
\isanewline
\ \ \isacommand{case}\isamarkupfalse%
\ {\isacharparenleft}{\kern0pt}{\isadigit{3}}\ v{\isacharparenright}{\kern0pt}\isanewline
\ \ \isacommand{then}\isamarkupfalse%
\ \isacommand{show}\isamarkupfalse%
\ {\isacharquery}{\kern0pt}case\isanewline
\ \ \isacommand{proof}\isamarkupfalse%
\ {\isacharminus}{\kern0pt}\isanewline
\ \ \ \ \isacommand{assume}\isamarkupfalse%
\ HI{\isacharcolon}{\kern0pt}{\isachardoublequoteopen}gate\ {\isacharparenleft}{\kern0pt}Suc\ v{\isacharparenright}{\kern0pt}\ {\isacharparenleft}{\kern0pt}controlled{\isacharunderscore}{\kern0pt}rotations\ {\isacharparenleft}{\kern0pt}Suc\ v{\isacharparenright}{\kern0pt}{\isacharparenright}{\kern0pt}{\isachardoublequoteclose}\isanewline
\ \ \ \ \isacommand{show}\isamarkupfalse%
\ {\isachardoublequoteopen}gate\ {\isacharparenleft}{\kern0pt}Suc\ {\isacharparenleft}{\kern0pt}Suc\ v{\isacharparenright}{\kern0pt}{\isacharparenright}{\kern0pt}\ {\isacharparenleft}{\kern0pt}controlled{\isacharunderscore}{\kern0pt}rotations\ {\isacharparenleft}{\kern0pt}Suc\ {\isacharparenleft}{\kern0pt}Suc\ v{\isacharparenright}{\kern0pt}{\isacharparenright}{\kern0pt}{\isacharparenright}{\kern0pt}{\isachardoublequoteclose}\isanewline
\ \ \ \ \isacommand{proof}\isamarkupfalse%
\ {\isacharminus}{\kern0pt}\isanewline
\ \ \ \ \ \ \isacommand{have}\isamarkupfalse%
\ {\isachardoublequoteopen}gate\ {\isacharparenleft}{\kern0pt}Suc\ {\isacharparenleft}{\kern0pt}Suc\ v{\isacharparenright}{\kern0pt}{\isacharparenright}{\kern0pt}\ {\isacharparenleft}{\kern0pt}{\isacharparenleft}{\kern0pt}control\ {\isacharparenleft}{\kern0pt}Suc\ {\isacharparenleft}{\kern0pt}Suc\ v{\isacharparenright}{\kern0pt}{\isacharparenright}{\kern0pt}\ {\isacharparenleft}{\kern0pt}R\ {\isacharparenleft}{\kern0pt}Suc\ {\isacharparenleft}{\kern0pt}Suc\ v{\isacharparenright}{\kern0pt}{\isacharparenright}{\kern0pt}{\isacharparenright}{\kern0pt}{\isacharparenright}{\kern0pt}\ {\isacharasterisk}{\kern0pt}\ \isanewline
\ \ \ \ \ \ \ \ \ \ \ \ \ \ \ \ \ \ \ \ \ \ \ \ \ \ \ \ \ \ \ {\isacharparenleft}{\kern0pt}{\isacharparenleft}{\kern0pt}controlled{\isacharunderscore}{\kern0pt}rotations\ {\isacharparenleft}{\kern0pt}Suc\ v{\isacharparenright}{\kern0pt}{\isacharparenright}{\kern0pt}\ {\isasymOtimes}\ {\isacharparenleft}{\kern0pt}{\isadigit{1}}\isactrlsub m\ {\isadigit{2}}{\isacharparenright}{\kern0pt}{\isacharparenright}{\kern0pt}{\isacharparenright}{\kern0pt}{\isachardoublequoteclose}\isanewline
\ \ \ \ \ \ \isacommand{proof}\isamarkupfalse%
\ {\isacharparenleft}{\kern0pt}rule\ prod{\isacharunderscore}{\kern0pt}of{\isacharunderscore}{\kern0pt}gate{\isacharunderscore}{\kern0pt}is{\isacharunderscore}{\kern0pt}gate{\isacharparenright}{\kern0pt}\isanewline
\ \ \ \ \ \ \ \ \isacommand{show}\isamarkupfalse%
\ {\isachardoublequoteopen}gate\ {\isacharparenleft}{\kern0pt}Suc\ {\isacharparenleft}{\kern0pt}Suc\ v{\isacharparenright}{\kern0pt}{\isacharparenright}{\kern0pt}\ {\isacharparenleft}{\kern0pt}control\ {\isacharparenleft}{\kern0pt}Suc\ {\isacharparenleft}{\kern0pt}Suc\ v{\isacharparenright}{\kern0pt}{\isacharparenright}{\kern0pt}\ {\isacharparenleft}{\kern0pt}R\ {\isacharparenleft}{\kern0pt}Suc\ {\isacharparenleft}{\kern0pt}Suc\ v{\isacharparenright}{\kern0pt}{\isacharparenright}{\kern0pt}{\isacharparenright}{\kern0pt}{\isacharparenright}{\kern0pt}{\isachardoublequoteclose}\isanewline
\ \ \ \ \ \ \ \ \ \ \isacommand{using}\isamarkupfalse%
\ control{\isacharunderscore}{\kern0pt}is{\isacharunderscore}{\kern0pt}gate\ R{\isacharunderscore}{\kern0pt}is{\isacharunderscore}{\kern0pt}gate\ \isacommand{by}\isamarkupfalse%
\ blast\isanewline
\ \ \ \ \ \ \isacommand{next}\isamarkupfalse%
\isanewline
\ \ \ \ \ \ \ \ \isacommand{show}\isamarkupfalse%
\ {\isachardoublequoteopen}gate\ {\isacharparenleft}{\kern0pt}Suc\ {\isacharparenleft}{\kern0pt}Suc\ v{\isacharparenright}{\kern0pt}{\isacharparenright}{\kern0pt}\ {\isacharparenleft}{\kern0pt}controlled{\isacharunderscore}{\kern0pt}rotations\ {\isacharparenleft}{\kern0pt}Suc\ v{\isacharparenright}{\kern0pt}\ {\isasymOtimes}\ {\isadigit{1}}\isactrlsub m\ {\isadigit{2}}{\isacharparenright}{\kern0pt}{\isachardoublequoteclose}\isanewline
\ \ \ \ \ \ \ \ \ \ \isacommand{using}\isamarkupfalse%
\ tensor{\isacharunderscore}{\kern0pt}gate\ HI\ id{\isacharunderscore}{\kern0pt}is{\isacharunderscore}{\kern0pt}gate\ \isanewline
\ \ \ \ \ \ \ \ \ \ \isacommand{by}\isamarkupfalse%
\ {\isacharparenleft}{\kern0pt}metis\ One{\isacharunderscore}{\kern0pt}nat{\isacharunderscore}{\kern0pt}def\ SWAP{\isacharunderscore}{\kern0pt}up{\isachardot}{\kern0pt}simps{\isacharparenleft}{\kern0pt}{\isadigit{2}}{\isacharparenright}{\kern0pt}\ SWAP{\isacharunderscore}{\kern0pt}up{\isacharunderscore}{\kern0pt}is{\isacharunderscore}{\kern0pt}gate\ Suc{\isacharunderscore}{\kern0pt}eq{\isacharunderscore}{\kern0pt}plus{\isadigit{1}}{\isacharparenright}{\kern0pt}\isanewline
\ \ \ \ \ \ \isacommand{qed}\isamarkupfalse%
\isanewline
\ \ \ \ \ \ \isacommand{thus}\isamarkupfalse%
\ {\isacharquery}{\kern0pt}thesis\ \isacommand{using}\isamarkupfalse%
\ controlled{\isacharunderscore}{\kern0pt}rotations{\isachardot}{\kern0pt}simps\ \isacommand{by}\isamarkupfalse%
\ simp\isanewline
\ \ \ \ \isacommand{qed}\isamarkupfalse%
\isanewline
\ \ \isacommand{qed}\isamarkupfalse%
\isanewline
\isacommand{qed}\isamarkupfalse%
%
\endisatagproof
{\isafoldproof}%
%
\isadelimproof
\isanewline
%
\endisadelimproof
\isanewline
\isacommand{theorem}\isamarkupfalse%
\ QFT{\isacharunderscore}{\kern0pt}is{\isacharunderscore}{\kern0pt}gate{\isacharcolon}{\kern0pt}\isanewline
\ \ \isakeyword{shows}\ {\isachardoublequoteopen}gate\ n\ {\isacharparenleft}{\kern0pt}QFT\ n{\isacharparenright}{\kern0pt}{\isachardoublequoteclose}\isanewline
%
\isadelimproof
%
\endisadelimproof
%
\isatagproof
\isacommand{proof}\isamarkupfalse%
\ {\isacharparenleft}{\kern0pt}induction\ n\ rule{\isacharcolon}{\kern0pt}\ QFT{\isachardot}{\kern0pt}induct{\isacharparenright}{\kern0pt}\isanewline
\ \ \isacommand{case}\isamarkupfalse%
\ {\isadigit{1}}\isanewline
\ \ \isacommand{then}\isamarkupfalse%
\ \isacommand{show}\isamarkupfalse%
\ {\isacharquery}{\kern0pt}case\isanewline
\ \ \ \ \isacommand{by}\isamarkupfalse%
\ {\isacharparenleft}{\kern0pt}metis\ QFT{\isachardot}{\kern0pt}simps{\isacharparenleft}{\kern0pt}{\isadigit{1}}{\isacharparenright}{\kern0pt}\ controlled{\isacharunderscore}{\kern0pt}rotations{\isachardot}{\kern0pt}simps{\isacharparenleft}{\kern0pt}{\isadigit{1}}{\isacharparenright}{\kern0pt}\ controlled{\isacharunderscore}{\kern0pt}rotations{\isacharunderscore}{\kern0pt}is{\isacharunderscore}{\kern0pt}gate{\isacharparenright}{\kern0pt}\isanewline
\isacommand{next}\isamarkupfalse%
\isanewline
\ \ \isacommand{case}\isamarkupfalse%
\ {\isadigit{2}}\isanewline
\ \ \isacommand{then}\isamarkupfalse%
\ \isacommand{show}\isamarkupfalse%
\ {\isacharquery}{\kern0pt}case\isanewline
\ \ \ \ \isacommand{using}\isamarkupfalse%
\ H{\isacharunderscore}{\kern0pt}is{\isacharunderscore}{\kern0pt}gate\ \isacommand{by}\isamarkupfalse%
\ auto\isanewline
\isacommand{next}\isamarkupfalse%
\isanewline
\ \ \isacommand{case}\isamarkupfalse%
\ {\isacharparenleft}{\kern0pt}{\isadigit{3}}\ v{\isacharparenright}{\kern0pt}\isanewline
\ \ \isacommand{then}\isamarkupfalse%
\ \isacommand{show}\isamarkupfalse%
\ {\isacharquery}{\kern0pt}case\isanewline
\ \ \isacommand{proof}\isamarkupfalse%
\ {\isacharminus}{\kern0pt}\isanewline
\ \ \ \ \isacommand{assume}\isamarkupfalse%
\ HI{\isacharcolon}{\kern0pt}{\isachardoublequoteopen}gate\ {\isacharparenleft}{\kern0pt}Suc\ v{\isacharparenright}{\kern0pt}\ {\isacharparenleft}{\kern0pt}QFT\ {\isacharparenleft}{\kern0pt}Suc\ v{\isacharparenright}{\kern0pt}{\isacharparenright}{\kern0pt}{\isachardoublequoteclose}\isanewline
\ \ \ \ \isacommand{show}\isamarkupfalse%
\ {\isachardoublequoteopen}gate\ {\isacharparenleft}{\kern0pt}Suc\ {\isacharparenleft}{\kern0pt}Suc\ v{\isacharparenright}{\kern0pt}{\isacharparenright}{\kern0pt}\ {\isacharparenleft}{\kern0pt}QFT\ {\isacharparenleft}{\kern0pt}Suc\ {\isacharparenleft}{\kern0pt}Suc\ v{\isacharparenright}{\kern0pt}{\isacharparenright}{\kern0pt}{\isacharparenright}{\kern0pt}{\isachardoublequoteclose}\isanewline
\ \ \ \ \isacommand{proof}\isamarkupfalse%
\ {\isacharminus}{\kern0pt}\isanewline
\ \ \ \ \ \ \isacommand{have}\isamarkupfalse%
\ {\isachardoublequoteopen}gate\ {\isacharparenleft}{\kern0pt}Suc\ {\isacharparenleft}{\kern0pt}Suc\ v{\isacharparenright}{\kern0pt}{\isacharparenright}{\kern0pt}\ {\isacharparenleft}{\kern0pt}{\isacharparenleft}{\kern0pt}{\isacharparenleft}{\kern0pt}{\isadigit{1}}\isactrlsub m\ {\isadigit{2}}{\isacharparenright}{\kern0pt}\ {\isasymOtimes}\ {\isacharparenleft}{\kern0pt}QFT\ {\isacharparenleft}{\kern0pt}Suc\ v{\isacharparenright}{\kern0pt}{\isacharparenright}{\kern0pt}{\isacharparenright}{\kern0pt}\ {\isacharasterisk}{\kern0pt}\ \isanewline
\ \ \ \ \ \ \ \ \ \ \ \ \ \ \ \ \ \ \ \ \ \ \ \ \ \ \ \ \ \ \ \ {\isacharparenleft}{\kern0pt}controlled{\isacharunderscore}{\kern0pt}rotations\ {\isacharparenleft}{\kern0pt}Suc\ {\isacharparenleft}{\kern0pt}Suc\ v{\isacharparenright}{\kern0pt}{\isacharparenright}{\kern0pt}{\isacharparenright}{\kern0pt}\ {\isacharasterisk}{\kern0pt}\ {\isacharparenleft}{\kern0pt}H\ {\isasymOtimes}\ {\isacharparenleft}{\kern0pt}{\isacharparenleft}{\kern0pt}{\isadigit{1}}\isactrlsub m\ {\isacharparenleft}{\kern0pt}{\isadigit{2}}{\isacharcircum}{\kern0pt}Suc\ v{\isacharparenright}{\kern0pt}{\isacharparenright}{\kern0pt}{\isacharparenright}{\kern0pt}{\isacharparenright}{\kern0pt}{\isacharparenright}{\kern0pt}{\isachardoublequoteclose}\isanewline
\ \ \ \ \ \ \isacommand{proof}\isamarkupfalse%
\ {\isacharparenleft}{\kern0pt}rule\ prod{\isacharunderscore}{\kern0pt}of{\isacharunderscore}{\kern0pt}gate{\isacharunderscore}{\kern0pt}is{\isacharunderscore}{\kern0pt}gate{\isacharparenright}{\kern0pt}{\isacharplus}{\kern0pt}\isanewline
\ \ \ \ \ \ \ \ \isacommand{show}\isamarkupfalse%
\ {\isachardoublequoteopen}gate\ {\isacharparenleft}{\kern0pt}Suc\ {\isacharparenleft}{\kern0pt}Suc\ v{\isacharparenright}{\kern0pt}{\isacharparenright}{\kern0pt}\ {\isacharparenleft}{\kern0pt}{\isadigit{1}}\isactrlsub m\ {\isadigit{2}}\ {\isasymOtimes}\ QFT\ {\isacharparenleft}{\kern0pt}Suc\ v{\isacharparenright}{\kern0pt}{\isacharparenright}{\kern0pt}{\isachardoublequoteclose}\isanewline
\ \ \ \ \ \ \ \ \ \ \isacommand{using}\isamarkupfalse%
\ HI\ tensor{\isacharunderscore}{\kern0pt}gate\ id{\isacharunderscore}{\kern0pt}is{\isacharunderscore}{\kern0pt}gate\isanewline
\ \ \ \ \ \ \ \ \ \ \isacommand{by}\isamarkupfalse%
\ {\isacharparenleft}{\kern0pt}metis\ One{\isacharunderscore}{\kern0pt}nat{\isacharunderscore}{\kern0pt}def\ controlled{\isacharunderscore}{\kern0pt}rotations{\isachardot}{\kern0pt}simps{\isacharparenleft}{\kern0pt}{\isadigit{2}}{\isacharparenright}{\kern0pt}\ controlled{\isacharunderscore}{\kern0pt}rotations{\isacharunderscore}{\kern0pt}is{\isacharunderscore}{\kern0pt}gate\ \isanewline
\ \ \ \ \ \ \ \ \ \ \ \ \ \ plus{\isacharunderscore}{\kern0pt}{\isadigit{1}}{\isacharunderscore}{\kern0pt}eq{\isacharunderscore}{\kern0pt}Suc{\isacharparenright}{\kern0pt}\isanewline
\ \ \ \ \ \ \ \ \isacommand{show}\isamarkupfalse%
\ {\isachardoublequoteopen}gate\ {\isacharparenleft}{\kern0pt}Suc\ {\isacharparenleft}{\kern0pt}Suc\ v{\isacharparenright}{\kern0pt}{\isacharparenright}{\kern0pt}\ {\isacharparenleft}{\kern0pt}controlled{\isacharunderscore}{\kern0pt}rotations\ {\isacharparenleft}{\kern0pt}Suc\ {\isacharparenleft}{\kern0pt}Suc\ v{\isacharparenright}{\kern0pt}{\isacharparenright}{\kern0pt}{\isacharparenright}{\kern0pt}{\isachardoublequoteclose}\isanewline
\ \ \ \ \ \ \ \ \ \ \isacommand{using}\isamarkupfalse%
\ controlled{\isacharunderscore}{\kern0pt}rotations{\isacharunderscore}{\kern0pt}is{\isacharunderscore}{\kern0pt}gate\ \isacommand{by}\isamarkupfalse%
\ metis\isanewline
\ \ \ \ \ \ \ \ \isacommand{show}\isamarkupfalse%
\ {\isachardoublequoteopen}gate\ {\isacharparenleft}{\kern0pt}Suc\ {\isacharparenleft}{\kern0pt}Suc\ v{\isacharparenright}{\kern0pt}{\isacharparenright}{\kern0pt}\ {\isacharparenleft}{\kern0pt}H\ {\isasymOtimes}\ {\isadigit{1}}\isactrlsub m\ {\isacharparenleft}{\kern0pt}{\isadigit{2}}\ {\isacharcircum}{\kern0pt}\ Suc\ v{\isacharparenright}{\kern0pt}{\isacharparenright}{\kern0pt}{\isachardoublequoteclose}\isanewline
\ \ \ \ \ \ \ \ \ \ \isacommand{using}\isamarkupfalse%
\ H{\isacharunderscore}{\kern0pt}is{\isacharunderscore}{\kern0pt}gate\ id{\isacharunderscore}{\kern0pt}is{\isacharunderscore}{\kern0pt}gate\ tensor{\isacharunderscore}{\kern0pt}gate\ \isanewline
\ \ \ \ \ \ \ \ \ \ \isacommand{by}\isamarkupfalse%
\ {\isacharparenleft}{\kern0pt}metis\ Quantum{\isachardot}{\kern0pt}Id{\isacharunderscore}{\kern0pt}def\ plus{\isacharunderscore}{\kern0pt}{\isadigit{1}}{\isacharunderscore}{\kern0pt}eq{\isacharunderscore}{\kern0pt}Suc{\isacharparenright}{\kern0pt}\isanewline
\ \ \ \ \ \ \isacommand{qed}\isamarkupfalse%
\isanewline
\ \ \ \ \ \ \isacommand{thus}\isamarkupfalse%
\ {\isacharquery}{\kern0pt}thesis\ \isacommand{using}\isamarkupfalse%
\ QFT{\isachardot}{\kern0pt}simps\ \isacommand{by}\isamarkupfalse%
\ simp\isanewline
\ \ \ \ \isacommand{qed}\isamarkupfalse%
\isanewline
\ \ \isacommand{qed}\isamarkupfalse%
\isanewline
\isacommand{qed}\isamarkupfalse%
%
\endisatagproof
{\isafoldproof}%
%
\isadelimproof
\isanewline
%
\endisadelimproof
\isanewline
\isacommand{corollary}\isamarkupfalse%
\ QFT{\isacharunderscore}{\kern0pt}is{\isacharunderscore}{\kern0pt}unitary{\isacharcolon}{\kern0pt}\isanewline
\ \ \isakeyword{shows}\ {\isachardoublequoteopen}unitary\ {\isacharparenleft}{\kern0pt}QFT\ n{\isacharparenright}{\kern0pt}{\isachardoublequoteclose}\isanewline
%
\isadelimproof
\ \ \ \ %
\endisadelimproof
%
\isatagproof
\isacommand{using}\isamarkupfalse%
\ QFT{\isacharunderscore}{\kern0pt}is{\isacharunderscore}{\kern0pt}gate\ gate{\isacharunderscore}{\kern0pt}def\ \isacommand{by}\isamarkupfalse%
\ simp%
\endisatagproof
{\isafoldproof}%
%
\isadelimproof
\isanewline
%
\endisadelimproof
\isanewline
\isacommand{corollary}\isamarkupfalse%
\ reverse{\isacharunderscore}{\kern0pt}product{\isacharunderscore}{\kern0pt}rep{\isacharunderscore}{\kern0pt}is{\isacharunderscore}{\kern0pt}state{\isacharcolon}{\kern0pt}\isanewline
\ \ \isakeyword{assumes}\ {\isachardoublequoteopen}j\ {\isacharless}{\kern0pt}\ {\isadigit{2}}{\isacharcircum}{\kern0pt}n{\isachardoublequoteclose}\isanewline
\ \ \isakeyword{shows}\ {\isachardoublequoteopen}state\ n\ {\isacharparenleft}{\kern0pt}reverse{\isacharunderscore}{\kern0pt}QFT{\isacharunderscore}{\kern0pt}product{\isacharunderscore}{\kern0pt}representation\ j\ n{\isacharparenright}{\kern0pt}{\isachardoublequoteclose}\isanewline
%
\isadelimproof
\ \ \ \ %
\endisadelimproof
%
\isatagproof
\isacommand{using}\isamarkupfalse%
\ QFT{\isacharunderscore}{\kern0pt}is{\isacharunderscore}{\kern0pt}gate\ QFT{\isacharunderscore}{\kern0pt}is{\isacharunderscore}{\kern0pt}correct\ gate{\isacharunderscore}{\kern0pt}on{\isacharunderscore}{\kern0pt}state{\isacharunderscore}{\kern0pt}is{\isacharunderscore}{\kern0pt}state\ assms\ state{\isacharunderscore}{\kern0pt}basis{\isacharunderscore}{\kern0pt}is{\isacharunderscore}{\kern0pt}state\isanewline
\ \ \ \ \isacommand{by}\isamarkupfalse%
\ {\isacharparenleft}{\kern0pt}metis\ dim{\isacharunderscore}{\kern0pt}col{\isacharunderscore}{\kern0pt}mat{\isacharparenleft}{\kern0pt}{\isadigit{1}}{\isacharparenright}{\kern0pt}\ dim{\isacharunderscore}{\kern0pt}row{\isacharunderscore}{\kern0pt}mat{\isacharparenleft}{\kern0pt}{\isadigit{1}}{\isacharparenright}{\kern0pt}\ index{\isacharunderscore}{\kern0pt}unit{\isacharunderscore}{\kern0pt}vec{\isacharparenleft}{\kern0pt}{\isadigit{3}}{\isacharparenright}{\kern0pt}\ ket{\isacharunderscore}{\kern0pt}vec{\isacharunderscore}{\kern0pt}col\ ket{\isacharunderscore}{\kern0pt}vec{\isacharunderscore}{\kern0pt}def\ \isanewline
\ \ \ \ \ \ \ \ state{\isacharunderscore}{\kern0pt}basis{\isacharunderscore}{\kern0pt}def\ state{\isacharunderscore}{\kern0pt}def\ unit{\isacharunderscore}{\kern0pt}cpx{\isacharunderscore}{\kern0pt}vec{\isacharunderscore}{\kern0pt}length{\isacharparenright}{\kern0pt}%
\endisatagproof
{\isafoldproof}%
%
\isadelimproof
\isanewline
%
\endisadelimproof
\isanewline
\isacommand{lemma}\isamarkupfalse%
\ reverse{\isacharunderscore}{\kern0pt}qubits{\isacharunderscore}{\kern0pt}is{\isacharunderscore}{\kern0pt}gate{\isacharcolon}{\kern0pt}\isanewline
\ \ \isakeyword{shows}\ {\isachardoublequoteopen}gate\ n\ {\isacharparenleft}{\kern0pt}reverse{\isacharunderscore}{\kern0pt}qubits\ n{\isacharparenright}{\kern0pt}{\isachardoublequoteclose}\isanewline
%
\isadelimproof
%
\endisadelimproof
%
\isatagproof
\isacommand{proof}\isamarkupfalse%
\ {\isacharparenleft}{\kern0pt}induct\ n\ rule{\isacharcolon}{\kern0pt}\ reverse{\isacharunderscore}{\kern0pt}qubits{\isachardot}{\kern0pt}induct{\isacharparenright}{\kern0pt}\isanewline
\ \ \isacommand{case}\isamarkupfalse%
\ {\isadigit{1}}\isanewline
\ \ \isacommand{then}\isamarkupfalse%
\ \isacommand{show}\isamarkupfalse%
\ {\isacharquery}{\kern0pt}case\ \isanewline
\ \ \ \ \isacommand{by}\isamarkupfalse%
\ {\isacharparenleft}{\kern0pt}metis\ QFT{\isachardot}{\kern0pt}simps{\isacharparenleft}{\kern0pt}{\isadigit{1}}{\isacharparenright}{\kern0pt}\ QFT{\isacharunderscore}{\kern0pt}is{\isacharunderscore}{\kern0pt}gate\ reverse{\isacharunderscore}{\kern0pt}qubits{\isachardot}{\kern0pt}simps{\isacharparenleft}{\kern0pt}{\isadigit{1}}{\isacharparenright}{\kern0pt}{\isacharparenright}{\kern0pt}\isanewline
\isacommand{next}\isamarkupfalse%
\isanewline
\ \ \isacommand{case}\isamarkupfalse%
\ {\isadigit{2}}\isanewline
\ \ \isacommand{then}\isamarkupfalse%
\ \isacommand{show}\isamarkupfalse%
\ {\isacharquery}{\kern0pt}case\isanewline
\ \ \ \ \isacommand{using}\isamarkupfalse%
\ Y{\isacharunderscore}{\kern0pt}is{\isacharunderscore}{\kern0pt}gate\ prod{\isacharunderscore}{\kern0pt}of{\isacharunderscore}{\kern0pt}gate{\isacharunderscore}{\kern0pt}is{\isacharunderscore}{\kern0pt}gate\ \isacommand{by}\isamarkupfalse%
\ fastforce\isanewline
\isacommand{next}\isamarkupfalse%
\isanewline
\ \ \isacommand{case}\isamarkupfalse%
\ {\isadigit{3}}\isanewline
\ \ \isacommand{then}\isamarkupfalse%
\ \isacommand{show}\isamarkupfalse%
\ {\isacharquery}{\kern0pt}case\isanewline
\ \ \ \ \isacommand{using}\isamarkupfalse%
\ One{\isacharunderscore}{\kern0pt}nat{\isacharunderscore}{\kern0pt}def\ SWAP{\isacharunderscore}{\kern0pt}is{\isacharunderscore}{\kern0pt}gate\ Suc{\isacharunderscore}{\kern0pt}{\isadigit{1}}\ reverse{\isacharunderscore}{\kern0pt}qubits{\isachardot}{\kern0pt}simps{\isacharparenleft}{\kern0pt}{\isadigit{3}}{\isacharparenright}{\kern0pt}\ \isacommand{by}\isamarkupfalse%
\ presburger\isanewline
\isacommand{next}\isamarkupfalse%
\isanewline
\ \ \isacommand{case}\isamarkupfalse%
\ {\isacharparenleft}{\kern0pt}{\isadigit{4}}\ va{\isacharparenright}{\kern0pt}\isanewline
\ \ \isacommand{then}\isamarkupfalse%
\ \isacommand{show}\isamarkupfalse%
\ {\isacharquery}{\kern0pt}case\isanewline
\ \ \isacommand{proof}\isamarkupfalse%
\ {\isacharminus}{\kern0pt}\isanewline
\ \ \ \ \isacommand{assume}\isamarkupfalse%
\ HI{\isacharcolon}{\kern0pt}{\isachardoublequoteopen}gate\ {\isacharparenleft}{\kern0pt}Suc\ {\isacharparenleft}{\kern0pt}Suc\ va{\isacharparenright}{\kern0pt}{\isacharparenright}{\kern0pt}\ {\isacharparenleft}{\kern0pt}reverse{\isacharunderscore}{\kern0pt}qubits\ {\isacharparenleft}{\kern0pt}Suc\ {\isacharparenleft}{\kern0pt}Suc\ va{\isacharparenright}{\kern0pt}{\isacharparenright}{\kern0pt}{\isacharparenright}{\kern0pt}{\isachardoublequoteclose}\isanewline
\ \ \ \ \isacommand{show}\isamarkupfalse%
\ {\isachardoublequoteopen}gate\ {\isacharparenleft}{\kern0pt}Suc\ {\isacharparenleft}{\kern0pt}Suc\ {\isacharparenleft}{\kern0pt}Suc\ va{\isacharparenright}{\kern0pt}{\isacharparenright}{\kern0pt}{\isacharparenright}{\kern0pt}\ {\isacharparenleft}{\kern0pt}reverse{\isacharunderscore}{\kern0pt}qubits\ {\isacharparenleft}{\kern0pt}Suc\ {\isacharparenleft}{\kern0pt}Suc\ {\isacharparenleft}{\kern0pt}Suc\ va{\isacharparenright}{\kern0pt}{\isacharparenright}{\kern0pt}{\isacharparenright}{\kern0pt}{\isacharparenright}{\kern0pt}{\isachardoublequoteclose}\isanewline
\ \ \ \ \isacommand{proof}\isamarkupfalse%
\ {\isacharminus}{\kern0pt}\isanewline
\ \ \ \ \ \ \isacommand{have}\isamarkupfalse%
\ {\isachardoublequoteopen}gate\ {\isacharparenleft}{\kern0pt}Suc\ {\isacharparenleft}{\kern0pt}Suc\ {\isacharparenleft}{\kern0pt}Suc\ va{\isacharparenright}{\kern0pt}{\isacharparenright}{\kern0pt}{\isacharparenright}{\kern0pt}\ {\isacharparenleft}{\kern0pt}{\isacharparenleft}{\kern0pt}{\isacharparenleft}{\kern0pt}reverse{\isacharunderscore}{\kern0pt}qubits\ {\isacharparenleft}{\kern0pt}Suc\ {\isacharparenleft}{\kern0pt}Suc\ va{\isacharparenright}{\kern0pt}{\isacharparenright}{\kern0pt}{\isacharparenright}{\kern0pt}\ {\isasymOtimes}\ {\isacharparenleft}{\kern0pt}{\isadigit{1}}\isactrlsub m\ {\isadigit{2}}{\isacharparenright}{\kern0pt}{\isacharparenright}{\kern0pt}\ {\isacharasterisk}{\kern0pt}\isanewline
\ \ \ \ \ \ \ \ \ \ \ \ \ \ \ \ \ \ \ \ \ \ \ \ \ \ \ \ \ \ \ \ \ \ \ \ \ \ \ \ \ {\isacharparenleft}{\kern0pt}SWAP{\isacharunderscore}{\kern0pt}down\ {\isacharparenleft}{\kern0pt}Suc\ {\isacharparenleft}{\kern0pt}Suc\ {\isacharparenleft}{\kern0pt}Suc\ va{\isacharparenright}{\kern0pt}{\isacharparenright}{\kern0pt}{\isacharparenright}{\kern0pt}{\isacharparenright}{\kern0pt}{\isacharparenright}{\kern0pt}{\isachardoublequoteclose}\isanewline
\ \ \ \ \ \ \isacommand{proof}\isamarkupfalse%
\ {\isacharparenleft}{\kern0pt}rule\ prod{\isacharunderscore}{\kern0pt}of{\isacharunderscore}{\kern0pt}gate{\isacharunderscore}{\kern0pt}is{\isacharunderscore}{\kern0pt}gate{\isacharparenright}{\kern0pt}\isanewline
\ \ \ \ \ \ \ \ \isacommand{show}\isamarkupfalse%
\ {\isachardoublequoteopen}gate\ {\isacharparenleft}{\kern0pt}Suc\ {\isacharparenleft}{\kern0pt}Suc\ {\isacharparenleft}{\kern0pt}Suc\ va{\isacharparenright}{\kern0pt}{\isacharparenright}{\kern0pt}{\isacharparenright}{\kern0pt}\ {\isacharparenleft}{\kern0pt}reverse{\isacharunderscore}{\kern0pt}qubits\ {\isacharparenleft}{\kern0pt}Suc\ {\isacharparenleft}{\kern0pt}Suc\ va{\isacharparenright}{\kern0pt}{\isacharparenright}{\kern0pt}\ {\isasymOtimes}\ {\isadigit{1}}\isactrlsub m\ {\isadigit{2}}{\isacharparenright}{\kern0pt}{\isachardoublequoteclose}\isanewline
\ \ \ \ \ \ \ \ \ \ \isacommand{using}\isamarkupfalse%
\ HI\ id{\isacharunderscore}{\kern0pt}is{\isacharunderscore}{\kern0pt}gate\ tensor{\isacharunderscore}{\kern0pt}gate\ \isanewline
\ \ \ \ \ \ \ \ \ \ \isacommand{by}\isamarkupfalse%
\ {\isacharparenleft}{\kern0pt}metis\ One{\isacharunderscore}{\kern0pt}nat{\isacharunderscore}{\kern0pt}def\ Suc{\isacharunderscore}{\kern0pt}eq{\isacharunderscore}{\kern0pt}plus{\isadigit{1}}\ controlled{\isacharunderscore}{\kern0pt}rotations{\isachardot}{\kern0pt}simps{\isacharparenleft}{\kern0pt}{\isadigit{2}}{\isacharparenright}{\kern0pt}\ \isanewline
\ \ \ \ \ \ \ \ \ \ \ \ \ \ controlled{\isacharunderscore}{\kern0pt}rotations{\isacharunderscore}{\kern0pt}is{\isacharunderscore}{\kern0pt}gate{\isacharparenright}{\kern0pt}\isanewline
\ \ \ \ \ \ \isacommand{next}\isamarkupfalse%
\isanewline
\ \ \ \ \ \ \ \ \isacommand{show}\isamarkupfalse%
\ {\isachardoublequoteopen}gate\ {\isacharparenleft}{\kern0pt}Suc\ {\isacharparenleft}{\kern0pt}Suc\ {\isacharparenleft}{\kern0pt}Suc\ va{\isacharparenright}{\kern0pt}{\isacharparenright}{\kern0pt}{\isacharparenright}{\kern0pt}\ {\isacharparenleft}{\kern0pt}SWAP{\isacharunderscore}{\kern0pt}down\ {\isacharparenleft}{\kern0pt}Suc\ {\isacharparenleft}{\kern0pt}Suc\ {\isacharparenleft}{\kern0pt}Suc\ va{\isacharparenright}{\kern0pt}{\isacharparenright}{\kern0pt}{\isacharparenright}{\kern0pt}{\isacharparenright}{\kern0pt}{\isachardoublequoteclose}\isanewline
\ \ \ \ \ \ \ \ \ \ \isacommand{using}\isamarkupfalse%
\ SWAP{\isacharunderscore}{\kern0pt}down{\isacharunderscore}{\kern0pt}is{\isacharunderscore}{\kern0pt}gate\ \isacommand{by}\isamarkupfalse%
\ metis\isanewline
\ \ \ \ \ \ \isacommand{qed}\isamarkupfalse%
\isanewline
\ \ \ \ \ \ \isacommand{thus}\isamarkupfalse%
\ {\isacharquery}{\kern0pt}thesis\ \isacommand{using}\isamarkupfalse%
\ reverse{\isacharunderscore}{\kern0pt}qubits{\isachardot}{\kern0pt}simps\ \isacommand{by}\isamarkupfalse%
\ simp\isanewline
\ \ \ \ \isacommand{qed}\isamarkupfalse%
\isanewline
\ \ \isacommand{qed}\isamarkupfalse%
\isanewline
\isacommand{qed}\isamarkupfalse%
%
\endisatagproof
{\isafoldproof}%
%
\isadelimproof
\isanewline
%
\endisadelimproof
\isanewline
\isacommand{theorem}\isamarkupfalse%
\ ordered{\isacharunderscore}{\kern0pt}QFT{\isacharunderscore}{\kern0pt}is{\isacharunderscore}{\kern0pt}gate{\isacharcolon}{\kern0pt}\isanewline
\ \ \isakeyword{shows}\ {\isachardoublequoteopen}gate\ n\ {\isacharparenleft}{\kern0pt}ordered{\isacharunderscore}{\kern0pt}QFT\ n{\isacharparenright}{\kern0pt}{\isachardoublequoteclose}\isanewline
%
\isadelimproof
\ \ \ \ %
\endisadelimproof
%
\isatagproof
\isacommand{using}\isamarkupfalse%
\ reverse{\isacharunderscore}{\kern0pt}qubits{\isacharunderscore}{\kern0pt}is{\isacharunderscore}{\kern0pt}gate\ QFT{\isacharunderscore}{\kern0pt}is{\isacharunderscore}{\kern0pt}gate\ ordered{\isacharunderscore}{\kern0pt}QFT{\isacharunderscore}{\kern0pt}def\ prod{\isacharunderscore}{\kern0pt}of{\isacharunderscore}{\kern0pt}gate{\isacharunderscore}{\kern0pt}is{\isacharunderscore}{\kern0pt}gate\ \isacommand{by}\isamarkupfalse%
\ auto%
\endisatagproof
{\isafoldproof}%
%
\isadelimproof
\isanewline
%
\endisadelimproof
\isanewline
\isacommand{corollary}\isamarkupfalse%
\ ordered{\isacharunderscore}{\kern0pt}QFT{\isacharunderscore}{\kern0pt}is{\isacharunderscore}{\kern0pt}unitary{\isacharcolon}{\kern0pt}\isanewline
\ \ \isakeyword{shows}\ {\isachardoublequoteopen}unitary\ {\isacharparenleft}{\kern0pt}ordered{\isacharunderscore}{\kern0pt}QFT\ n{\isacharparenright}{\kern0pt}{\isachardoublequoteclose}\isanewline
%
\isadelimproof
\ \ \ \ %
\endisadelimproof
%
\isatagproof
\isacommand{using}\isamarkupfalse%
\ ordered{\isacharunderscore}{\kern0pt}QFT{\isacharunderscore}{\kern0pt}is{\isacharunderscore}{\kern0pt}gate\ gate{\isacharunderscore}{\kern0pt}def\ \isacommand{by}\isamarkupfalse%
\ simp%
\endisatagproof
{\isafoldproof}%
%
\isadelimproof
\isanewline
%
\endisadelimproof
\isanewline
\isacommand{corollary}\isamarkupfalse%
\ product{\isacharunderscore}{\kern0pt}rep{\isacharunderscore}{\kern0pt}is{\isacharunderscore}{\kern0pt}state{\isacharcolon}{\kern0pt}\isanewline
\ \ \isakeyword{assumes}\ {\isachardoublequoteopen}j\ {\isacharless}{\kern0pt}\ {\isadigit{2}}{\isacharcircum}{\kern0pt}n{\isachardoublequoteclose}\isanewline
\ \ \isakeyword{shows}\ {\isachardoublequoteopen}state\ n\ {\isacharparenleft}{\kern0pt}QFT{\isacharunderscore}{\kern0pt}product{\isacharunderscore}{\kern0pt}representation\ j\ n{\isacharparenright}{\kern0pt}{\isachardoublequoteclose}\isanewline
%
\isadelimproof
\ \ \ \ %
\endisadelimproof
%
\isatagproof
\isacommand{using}\isamarkupfalse%
\ ordered{\isacharunderscore}{\kern0pt}QFT{\isacharunderscore}{\kern0pt}is{\isacharunderscore}{\kern0pt}gate\ ordered{\isacharunderscore}{\kern0pt}QFT{\isacharunderscore}{\kern0pt}is{\isacharunderscore}{\kern0pt}correct\ gate{\isacharunderscore}{\kern0pt}on{\isacharunderscore}{\kern0pt}state{\isacharunderscore}{\kern0pt}is{\isacharunderscore}{\kern0pt}state\ assms\ \isanewline
\ \ \ \ state{\isacharunderscore}{\kern0pt}basis{\isacharunderscore}{\kern0pt}is{\isacharunderscore}{\kern0pt}state\isanewline
\ \ \ \ \isacommand{by}\isamarkupfalse%
\ {\isacharparenleft}{\kern0pt}metis\ reverse{\isacharunderscore}{\kern0pt}product{\isacharunderscore}{\kern0pt}rep{\isacharunderscore}{\kern0pt}is{\isacharunderscore}{\kern0pt}state\ reverse{\isacharunderscore}{\kern0pt}qubits{\isacharunderscore}{\kern0pt}is{\isacharunderscore}{\kern0pt}gate\ \isanewline
\ \ \ \ \ \ \ \ reverse{\isacharunderscore}{\kern0pt}qubits{\isacharunderscore}{\kern0pt}product{\isacharunderscore}{\kern0pt}representation{\isacharparenright}{\kern0pt}%
\endisatagproof
{\isafoldproof}%
%
\isadelimproof
\isanewline
%
\endisadelimproof
%
\isadelimtheory
\isanewline
%
\endisadelimtheory
%
\isatagtheory
\isacommand{end}\isamarkupfalse%
%
\endisatagtheory
{\isafoldtheory}%
%
\isadelimtheory
%
\endisadelimtheory
%
\end{isabellebody}%
\endinput
%:%file=~/Dropbox/Quantum_Fourier_Transform/QFT.thy%:%
%:%6=6%:%
%:%7=7%:%
%:%12=8%:%
%:%13=8%:%
%:%14=9%:%
%:%15=10%:%
%:%16=11%:%
%:%17=12%:%
%:%31=14%:%
%:%41=16%:%
%:%42=16%:%
%:%43=17%:%
%:%44=18%:%
%:%51=19%:%
%:%52=19%:%
%:%53=20%:%
%:%54=20%:%
%:%55=20%:%
%:%56=20%:%
%:%57=21%:%
%:%58=21%:%
%:%59=22%:%
%:%60=22%:%
%:%61=22%:%
%:%62=22%:%
%:%63=23%:%
%:%69=23%:%
%:%72=24%:%
%:%73=25%:%
%:%74=25%:%
%:%75=26%:%
%:%76=27%:%
%:%83=28%:%
%:%84=28%:%
%:%85=29%:%
%:%86=29%:%
%:%87=29%:%
%:%88=29%:%
%:%89=30%:%
%:%90=30%:%
%:%91=31%:%
%:%92=31%:%
%:%93=31%:%
%:%94=31%:%
%:%95=32%:%
%:%101=32%:%
%:%104=33%:%
%:%105=34%:%
%:%106=34%:%
%:%107=35%:%
%:%110=36%:%
%:%114=36%:%
%:%115=36%:%
%:%120=36%:%
%:%123=37%:%
%:%124=38%:%
%:%125=38%:%
%:%126=39%:%
%:%127=40%:%
%:%130=41%:%
%:%134=41%:%
%:%135=41%:%
%:%140=41%:%
%:%143=42%:%
%:%144=43%:%
%:%145=43%:%
%:%146=44%:%
%:%147=45%:%
%:%150=46%:%
%:%154=46%:%
%:%155=46%:%
%:%160=46%:%
%:%163=47%:%
%:%164=48%:%
%:%165=48%:%
%:%166=49%:%
%:%167=50%:%
%:%168=51%:%
%:%175=52%:%
%:%176=52%:%
%:%177=53%:%
%:%178=53%:%
%:%179=53%:%
%:%180=53%:%
%:%181=54%:%
%:%182=54%:%
%:%183=54%:%
%:%184=54%:%
%:%185=55%:%
%:%191=55%:%
%:%194=56%:%
%:%195=57%:%
%:%196=57%:%
%:%197=58%:%
%:%198=59%:%
%:%205=60%:%
%:%206=60%:%
%:%207=61%:%
%:%208=61%:%
%:%209=62%:%
%:%210=62%:%
%:%211=63%:%
%:%212=63%:%
%:%213=64%:%
%:%214=64%:%
%:%215=65%:%
%:%216=65%:%
%:%217=66%:%
%:%218=67%:%
%:%219=67%:%
%:%220=68%:%
%:%221=68%:%
%:%222=69%:%
%:%223=69%:%
%:%224=69%:%
%:%225=69%:%
%:%226=70%:%
%:%227=70%:%
%:%228=70%:%
%:%229=70%:%
%:%230=71%:%
%:%231=71%:%
%:%232=71%:%
%:%233=71%:%
%:%234=72%:%
%:%235=72%:%
%:%236=72%:%
%:%237=72%:%
%:%238=73%:%
%:%239=73%:%
%:%240=73%:%
%:%241=73%:%
%:%242=74%:%
%:%243=74%:%
%:%244=74%:%
%:%245=74%:%
%:%246=75%:%
%:%247=75%:%
%:%248=75%:%
%:%249=75%:%
%:%250=76%:%
%:%251=76%:%
%:%252=76%:%
%:%253=76%:%
%:%254=77%:%
%:%255=77%:%
%:%256=78%:%
%:%257=78%:%
%:%258=78%:%
%:%259=79%:%
%:%260=79%:%
%:%261=79%:%
%:%262=80%:%
%:%263=81%:%
%:%264=82%:%
%:%265=82%:%
%:%266=82%:%
%:%267=82%:%
%:%268=83%:%
%:%269=83%:%
%:%270=83%:%
%:%271=84%:%
%:%272=84%:%
%:%273=85%:%
%:%274=85%:%
%:%275=86%:%
%:%276=86%:%
%:%277=87%:%
%:%278=87%:%
%:%279=88%:%
%:%280=88%:%
%:%281=88%:%
%:%282=89%:%
%:%283=89%:%
%:%284=90%:%
%:%285=90%:%
%:%286=90%:%
%:%287=90%:%
%:%288=91%:%
%:%289=91%:%
%:%290=92%:%
%:%291=92%:%
%:%292=92%:%
%:%293=92%:%
%:%294=93%:%
%:%295=93%:%
%:%296=94%:%
%:%297=94%:%
%:%298=95%:%
%:%299=95%:%
%:%300=95%:%
%:%301=96%:%
%:%302=96%:%
%:%303=97%:%
%:%304=97%:%
%:%305=97%:%
%:%306=98%:%
%:%312=98%:%
%:%315=99%:%
%:%316=100%:%
%:%317=100%:%
%:%318=101%:%
%:%319=102%:%
%:%326=103%:%
%:%327=103%:%
%:%328=104%:%
%:%329=104%:%
%:%330=104%:%
%:%331=105%:%
%:%332=105%:%
%:%333=105%:%
%:%334=105%:%
%:%335=105%:%
%:%336=106%:%
%:%337=106%:%
%:%338=106%:%
%:%339=107%:%
%:%340=107%:%
%:%341=108%:%
%:%342=108%:%
%:%343=108%:%
%:%344=108%:%
%:%345=109%:%
%:%351=109%:%
%:%354=110%:%
%:%355=111%:%
%:%356=111%:%
%:%357=112%:%
%:%360=113%:%
%:%364=113%:%
%:%365=113%:%
%:%370=113%:%
%:%373=114%:%
%:%374=115%:%
%:%375=115%:%
%:%376=116%:%
%:%379=117%:%
%:%383=117%:%
%:%384=117%:%
%:%389=117%:%
%:%392=118%:%
%:%393=119%:%
%:%394=119%:%
%:%395=120%:%
%:%402=121%:%
%:%403=121%:%
%:%404=122%:%
%:%405=122%:%
%:%406=122%:%
%:%407=122%:%
%:%408=123%:%
%:%409=123%:%
%:%410=123%:%
%:%411=123%:%
%:%412=124%:%
%:%413=124%:%
%:%414=124%:%
%:%415=124%:%
%:%416=125%:%
%:%417=125%:%
%:%418=125%:%
%:%419=125%:%
%:%420=126%:%
%:%435=130%:%
%:%445=132%:%
%:%446=132%:%
%:%447=133%:%
%:%455=137%:%
%:%465=139%:%
%:%466=139%:%
%:%467=140%:%
%:%470=143%:%
%:%471=144%:%
%:%472=145%:%
%:%473=145%:%
%:%474=146%:%
%:%489=161%:%
%:%492=162%:%
%:%496=162%:%
%:%497=162%:%
%:%502=162%:%
%:%505=163%:%
%:%506=164%:%
%:%507=164%:%
%:%508=165%:%
%:%511=166%:%
%:%515=166%:%
%:%516=166%:%
%:%521=166%:%
%:%524=167%:%
%:%525=168%:%
%:%526=168%:%
%:%527=169%:%
%:%530=170%:%
%:%534=170%:%
%:%535=170%:%
%:%540=170%:%
%:%543=171%:%
%:%544=172%:%
%:%545=172%:%
%:%546=173%:%
%:%549=174%:%
%:%553=174%:%
%:%554=174%:%
%:%555=174%:%
%:%564=177%:%
%:%566=179%:%
%:%567=179%:%
%:%568=180%:%
%:%569=181%:%
%:%570=182%:%
%:%577=183%:%
%:%578=183%:%
%:%579=184%:%
%:%580=184%:%
%:%581=185%:%
%:%582=185%:%
%:%583=185%:%
%:%584=186%:%
%:%585=186%:%
%:%586=187%:%
%:%587=187%:%
%:%588=188%:%
%:%589=188%:%
%:%590=188%:%
%:%591=189%:%
%:%592=189%:%
%:%593=190%:%
%:%594=190%:%
%:%595=190%:%
%:%596=191%:%
%:%597=191%:%
%:%598=191%:%
%:%599=191%:%
%:%600=192%:%
%:%601=192%:%
%:%602=193%:%
%:%603=193%:%
%:%604=194%:%
%:%605=194%:%
%:%606=195%:%
%:%607=195%:%
%:%608=196%:%
%:%609=196%:%
%:%610=197%:%
%:%611=197%:%
%:%612=198%:%
%:%613=198%:%
%:%614=198%:%
%:%615=198%:%
%:%616=199%:%
%:%617=199%:%
%:%618=199%:%
%:%619=199%:%
%:%620=200%:%
%:%621=200%:%
%:%622=200%:%
%:%623=200%:%
%:%624=201%:%
%:%625=201%:%
%:%626=201%:%
%:%627=201%:%
%:%628=202%:%
%:%629=202%:%
%:%630=202%:%
%:%631=202%:%
%:%632=203%:%
%:%633=203%:%
%:%634=203%:%
%:%635=203%:%
%:%636=204%:%
%:%637=204%:%
%:%638=204%:%
%:%639=204%:%
%:%640=205%:%
%:%641=205%:%
%:%642=205%:%
%:%643=205%:%
%:%644=206%:%
%:%645=207%:%
%:%646=207%:%
%:%647=208%:%
%:%648=208%:%
%:%649=208%:%
%:%650=209%:%
%:%651=209%:%
%:%652=210%:%
%:%653=210%:%
%:%654=210%:%
%:%655=211%:%
%:%656=212%:%
%:%657=212%:%
%:%658=213%:%
%:%659=214%:%
%:%660=214%:%
%:%661=215%:%
%:%662=216%:%
%:%663=216%:%
%:%664=216%:%
%:%665=217%:%
%:%666=217%:%
%:%667=217%:%
%:%668=218%:%
%:%669=218%:%
%:%670=218%:%
%:%671=219%:%
%:%672=220%:%
%:%673=220%:%
%:%674=220%:%
%:%675=221%:%
%:%676=221%:%
%:%677=221%:%
%:%680=224%:%
%:%681=225%:%
%:%682=225%:%
%:%683=225%:%
%:%684=226%:%
%:%685=226%:%
%:%686=226%:%
%:%689=229%:%
%:%690=230%:%
%:%691=230%:%
%:%692=230%:%
%:%693=231%:%
%:%694=231%:%
%:%695=231%:%
%:%696=232%:%
%:%697=232%:%
%:%698=233%:%
%:%699=233%:%
%:%700=233%:%
%:%701=233%:%
%:%702=234%:%
%:%703=234%:%
%:%704=235%:%
%:%705=235%:%
%:%706=236%:%
%:%707=236%:%
%:%714=243%:%
%:%715=243%:%
%:%716=244%:%
%:%717=244%:%
%:%718=244%:%
%:%719=244%:%
%:%720=244%:%
%:%721=245%:%
%:%722=245%:%
%:%723=245%:%
%:%724=245%:%
%:%725=246%:%
%:%726=246%:%
%:%727=247%:%
%:%728=247%:%
%:%729=248%:%
%:%730=248%:%
%:%731=249%:%
%:%732=249%:%
%:%733=250%:%
%:%734=250%:%
%:%735=250%:%
%:%736=250%:%
%:%737=251%:%
%:%738=251%:%
%:%739=252%:%
%:%740=252%:%
%:%741=253%:%
%:%742=253%:%
%:%749=260%:%
%:%750=260%:%
%:%751=261%:%
%:%752=261%:%
%:%753=261%:%
%:%754=261%:%
%:%755=261%:%
%:%756=262%:%
%:%757=262%:%
%:%758=262%:%
%:%759=262%:%
%:%760=263%:%
%:%761=263%:%
%:%762=264%:%
%:%763=264%:%
%:%764=265%:%
%:%765=265%:%
%:%766=266%:%
%:%767=266%:%
%:%768=267%:%
%:%769=267%:%
%:%770=267%:%
%:%771=267%:%
%:%772=268%:%
%:%773=268%:%
%:%774=269%:%
%:%775=269%:%
%:%776=270%:%
%:%777=270%:%
%:%784=277%:%
%:%785=277%:%
%:%786=278%:%
%:%787=278%:%
%:%788=278%:%
%:%789=278%:%
%:%790=278%:%
%:%791=279%:%
%:%792=279%:%
%:%793=279%:%
%:%794=279%:%
%:%795=280%:%
%:%796=280%:%
%:%797=281%:%
%:%798=281%:%
%:%799=282%:%
%:%800=282%:%
%:%807=289%:%
%:%808=289%:%
%:%809=290%:%
%:%810=290%:%
%:%811=290%:%
%:%812=290%:%
%:%813=290%:%
%:%814=291%:%
%:%815=291%:%
%:%816=291%:%
%:%817=291%:%
%:%818=292%:%
%:%819=292%:%
%:%820=293%:%
%:%821=293%:%
%:%822=294%:%
%:%823=294%:%
%:%824=295%:%
%:%825=295%:%
%:%826=295%:%
%:%827=295%:%
%:%828=295%:%
%:%829=296%:%
%:%830=296%:%
%:%831=297%:%
%:%846=299%:%
%:%856=301%:%
%:%857=301%:%
%:%858=302%:%
%:%859=303%:%
%:%860=304%:%
%:%861=305%:%
%:%862=306%:%
%:%863=307%:%
%:%864=307%:%
%:%865=308%:%
%:%872=309%:%
%:%873=309%:%
%:%874=310%:%
%:%875=310%:%
%:%876=310%:%
%:%877=311%:%
%:%878=311%:%
%:%879=312%:%
%:%880=312%:%
%:%881=312%:%
%:%882=313%:%
%:%883=313%:%
%:%884=314%:%
%:%885=314%:%
%:%886=314%:%
%:%887=314%:%
%:%888=315%:%
%:%889=315%:%
%:%890=316%:%
%:%891=316%:%
%:%892=317%:%
%:%893=317%:%
%:%894=318%:%
%:%895=318%:%
%:%896=319%:%
%:%897=319%:%
%:%898=320%:%
%:%899=320%:%
%:%900=321%:%
%:%901=321%:%
%:%902=322%:%
%:%903=323%:%
%:%904=323%:%
%:%905=323%:%
%:%906=324%:%
%:%907=324%:%
%:%908=324%:%
%:%909=324%:%
%:%910=325%:%
%:%911=325%:%
%:%912=325%:%
%:%913=325%:%
%:%914=326%:%
%:%915=326%:%
%:%916=326%:%
%:%917=327%:%
%:%918=327%:%
%:%919=328%:%
%:%920=328%:%
%:%921=329%:%
%:%922=329%:%
%:%923=330%:%
%:%924=331%:%
%:%925=331%:%
%:%926=331%:%
%:%927=332%:%
%:%928=332%:%
%:%929=332%:%
%:%930=332%:%
%:%931=333%:%
%:%932=333%:%
%:%933=333%:%
%:%934=333%:%
%:%935=334%:%
%:%936=334%:%
%:%937=335%:%
%:%938=335%:%
%:%939=335%:%
%:%940=336%:%
%:%941=336%:%
%:%942=337%:%
%:%957=340%:%
%:%967=342%:%
%:%968=342%:%
%:%969=343%:%
%:%970=344%:%
%:%971=345%:%
%:%972=346%:%
%:%973=347%:%
%:%974=348%:%
%:%975=348%:%
%:%976=349%:%
%:%983=350%:%
%:%984=350%:%
%:%985=351%:%
%:%986=351%:%
%:%987=352%:%
%:%988=352%:%
%:%989=352%:%
%:%990=352%:%
%:%991=353%:%
%:%992=353%:%
%:%993=354%:%
%:%994=354%:%
%:%995=355%:%
%:%996=355%:%
%:%997=355%:%
%:%998=355%:%
%:%999=356%:%
%:%1000=356%:%
%:%1001=357%:%
%:%1002=357%:%
%:%1003=358%:%
%:%1004=358%:%
%:%1005=358%:%
%:%1006=358%:%
%:%1007=359%:%
%:%1008=359%:%
%:%1009=360%:%
%:%1010=360%:%
%:%1011=361%:%
%:%1012=361%:%
%:%1013=361%:%
%:%1014=361%:%
%:%1015=361%:%
%:%1016=362%:%
%:%1031=365%:%
%:%1043=367%:%
%:%1044=368%:%
%:%1046=370%:%
%:%1047=370%:%
%:%1048=371%:%
%:%1049=372%:%
%:%1050=373%:%
%:%1051=374%:%
%:%1052=375%:%
%:%1053=376%:%
%:%1054=377%:%
%:%1055=377%:%
%:%1056=378%:%
%:%1063=379%:%
%:%1064=379%:%
%:%1065=380%:%
%:%1066=380%:%
%:%1067=381%:%
%:%1068=381%:%
%:%1069=381%:%
%:%1070=381%:%
%:%1071=382%:%
%:%1072=382%:%
%:%1073=383%:%
%:%1074=383%:%
%:%1075=384%:%
%:%1076=384%:%
%:%1077=384%:%
%:%1078=384%:%
%:%1079=385%:%
%:%1080=385%:%
%:%1081=386%:%
%:%1082=386%:%
%:%1083=387%:%
%:%1084=387%:%
%:%1085=387%:%
%:%1086=387%:%
%:%1087=388%:%
%:%1088=388%:%
%:%1089=389%:%
%:%1090=389%:%
%:%1091=390%:%
%:%1092=390%:%
%:%1093=390%:%
%:%1094=391%:%
%:%1095=391%:%
%:%1096=392%:%
%:%1097=393%:%
%:%1112=397%:%
%:%1124=399%:%
%:%1125=400%:%
%:%1127=402%:%
%:%1128=402%:%
%:%1129=403%:%
%:%1132=406%:%
%:%1133=407%:%
%:%1134=408%:%
%:%1135=408%:%
%:%1136=409%:%
%:%1139=410%:%
%:%1143=410%:%
%:%1144=410%:%
%:%1149=410%:%
%:%1152=411%:%
%:%1153=412%:%
%:%1154=413%:%
%:%1155=413%:%
%:%1156=414%:%
%:%1157=415%:%
%:%1164=416%:%
%:%1165=416%:%
%:%1166=417%:%
%:%1167=417%:%
%:%1168=418%:%
%:%1169=418%:%
%:%1170=419%:%
%:%1171=419%:%
%:%1172=419%:%
%:%1173=419%:%
%:%1174=420%:%
%:%1175=420%:%
%:%1176=421%:%
%:%1177=421%:%
%:%1178=421%:%
%:%1179=421%:%
%:%1180=422%:%
%:%1181=422%:%
%:%1182=423%:%
%:%1183=423%:%
%:%1184=424%:%
%:%1185=424%:%
%:%1186=425%:%
%:%1187=426%:%
%:%1188=426%:%
%:%1189=427%:%
%:%1190=427%:%
%:%1191=428%:%
%:%1192=429%:%
%:%1193=430%:%
%:%1194=431%:%
%:%1195=431%:%
%:%1196=431%:%
%:%1197=432%:%
%:%1198=432%:%
%:%1199=432%:%
%:%1200=433%:%
%:%1201=433%:%
%:%1202=433%:%
%:%1205=436%:%
%:%1206=437%:%
%:%1207=437%:%
%:%1208=437%:%
%:%1209=438%:%
%:%1210=438%:%
%:%1211=438%:%
%:%1212=439%:%
%:%1213=439%:%
%:%1214=440%:%
%:%1215=440%:%
%:%1216=440%:%
%:%1217=440%:%
%:%1218=441%:%
%:%1219=441%:%
%:%1220=442%:%
%:%1221=442%:%
%:%1222=443%:%
%:%1223=443%:%
%:%1224=444%:%
%:%1225=444%:%
%:%1226=445%:%
%:%1227=445%:%
%:%1228=446%:%
%:%1229=446%:%
%:%1230=447%:%
%:%1231=447%:%
%:%1232=448%:%
%:%1233=448%:%
%:%1234=449%:%
%:%1235=449%:%
%:%1236=450%:%
%:%1237=450%:%
%:%1238=451%:%
%:%1239=451%:%
%:%1246=458%:%
%:%1247=459%:%
%:%1248=459%:%
%:%1249=459%:%
%:%1250=460%:%
%:%1251=460%:%
%:%1252=460%:%
%:%1253=460%:%
%:%1254=461%:%
%:%1255=461%:%
%:%1256=461%:%
%:%1260=465%:%
%:%1261=466%:%
%:%1262=466%:%
%:%1263=466%:%
%:%1264=467%:%
%:%1265=467%:%
%:%1266=468%:%
%:%1267=468%:%
%:%1268=469%:%
%:%1269=469%:%
%:%1273=473%:%
%:%1274=474%:%
%:%1275=474%:%
%:%1276=475%:%
%:%1277=475%:%
%:%1278=475%:%
%:%1279=475%:%
%:%1280=476%:%
%:%1281=476%:%
%:%1282=477%:%
%:%1283=477%:%
%:%1284=478%:%
%:%1285=478%:%
%:%1286=479%:%
%:%1287=479%:%
%:%1288=480%:%
%:%1289=480%:%
%:%1290=481%:%
%:%1291=481%:%
%:%1292=482%:%
%:%1293=483%:%
%:%1294=483%:%
%:%1295=484%:%
%:%1296=484%:%
%:%1297=485%:%
%:%1298=485%:%
%:%1299=486%:%
%:%1300=486%:%
%:%1301=487%:%
%:%1302=487%:%
%:%1303=488%:%
%:%1304=488%:%
%:%1305=489%:%
%:%1306=489%:%
%:%1307=490%:%
%:%1308=490%:%
%:%1309=491%:%
%:%1310=492%:%
%:%1311=493%:%
%:%1312=494%:%
%:%1313=495%:%
%:%1314=495%:%
%:%1315=495%:%
%:%1316=495%:%
%:%1317=496%:%
%:%1318=496%:%
%:%1319=496%:%
%:%1320=496%:%
%:%1321=497%:%
%:%1322=497%:%
%:%1323=498%:%
%:%1324=498%:%
%:%1328=502%:%
%:%1329=503%:%
%:%1330=503%:%
%:%1331=503%:%
%:%1332=504%:%
%:%1333=504%:%
%:%1334=505%:%
%:%1335=505%:%
%:%1336=506%:%
%:%1337=506%:%
%:%1341=510%:%
%:%1342=511%:%
%:%1343=511%:%
%:%1344=512%:%
%:%1345=512%:%
%:%1346=513%:%
%:%1347=513%:%
%:%1348=513%:%
%:%1349=514%:%
%:%1350=514%:%
%:%1351=515%:%
%:%1352=515%:%
%:%1353=516%:%
%:%1354=516%:%
%:%1355=517%:%
%:%1356=517%:%
%:%1357=518%:%
%:%1358=518%:%
%:%1359=519%:%
%:%1360=519%:%
%:%1361=520%:%
%:%1362=520%:%
%:%1363=521%:%
%:%1364=521%:%
%:%1365=522%:%
%:%1366=522%:%
%:%1367=523%:%
%:%1368=523%:%
%:%1369=524%:%
%:%1370=524%:%
%:%1374=528%:%
%:%1375=529%:%
%:%1376=529%:%
%:%1377=529%:%
%:%1378=530%:%
%:%1379=530%:%
%:%1380=531%:%
%:%1381=531%:%
%:%1382=532%:%
%:%1383=532%:%
%:%1384=533%:%
%:%1385=533%:%
%:%1386=534%:%
%:%1387=534%:%
%:%1388=535%:%
%:%1389=535%:%
%:%1390=536%:%
%:%1391=537%:%
%:%1392=537%:%
%:%1393=538%:%
%:%1394=538%:%
%:%1395=539%:%
%:%1396=539%:%
%:%1397=540%:%
%:%1398=540%:%
%:%1399=541%:%
%:%1400=541%:%
%:%1401=542%:%
%:%1402=542%:%
%:%1403=543%:%
%:%1404=543%:%
%:%1405=544%:%
%:%1406=544%:%
%:%1407=545%:%
%:%1408=546%:%
%:%1409=547%:%
%:%1410=548%:%
%:%1411=549%:%
%:%1412=549%:%
%:%1413=549%:%
%:%1414=549%:%
%:%1415=550%:%
%:%1416=550%:%
%:%1417=550%:%
%:%1418=550%:%
%:%1419=551%:%
%:%1420=551%:%
%:%1421=552%:%
%:%1422=552%:%
%:%1426=556%:%
%:%1427=557%:%
%:%1428=557%:%
%:%1429=557%:%
%:%1430=558%:%
%:%1431=558%:%
%:%1432=559%:%
%:%1433=559%:%
%:%1434=560%:%
%:%1435=560%:%
%:%1436=561%:%
%:%1437=561%:%
%:%1438=561%:%
%:%1439=561%:%
%:%1440=561%:%
%:%1441=562%:%
%:%1442=562%:%
%:%1443=563%:%
%:%1444=563%:%
%:%1445=564%:%
%:%1446=564%:%
%:%1447=565%:%
%:%1448=565%:%
%:%1449=566%:%
%:%1450=567%:%
%:%1451=567%:%
%:%1452=568%:%
%:%1453=568%:%
%:%1454=569%:%
%:%1455=569%:%
%:%1456=569%:%
%:%1457=570%:%
%:%1463=570%:%
%:%1466=571%:%
%:%1467=572%:%
%:%1468=573%:%
%:%1469=573%:%
%:%1470=574%:%
%:%1471=575%:%
%:%1474=578%:%
%:%1477=579%:%
%:%1481=579%:%
%:%1482=579%:%
%:%1487=579%:%
%:%1490=580%:%
%:%1491=581%:%
%:%1492=581%:%
%:%1493=582%:%
%:%1494=583%:%
%:%1501=584%:%
%:%1502=584%:%
%:%1503=585%:%
%:%1504=585%:%
%:%1505=586%:%
%:%1506=586%:%
%:%1507=587%:%
%:%1508=587%:%
%:%1509=587%:%
%:%1510=588%:%
%:%1511=588%:%
%:%1512=589%:%
%:%1513=589%:%
%:%1514=589%:%
%:%1515=589%:%
%:%1516=590%:%
%:%1517=590%:%
%:%1518=591%:%
%:%1519=591%:%
%:%1520=592%:%
%:%1521=592%:%
%:%1522=593%:%
%:%1523=594%:%
%:%1524=594%:%
%:%1525=595%:%
%:%1526=595%:%
%:%1527=596%:%
%:%1528=596%:%
%:%1529=597%:%
%:%1530=597%:%
%:%1531=598%:%
%:%1532=598%:%
%:%1533=598%:%
%:%1534=599%:%
%:%1535=599%:%
%:%1536=600%:%
%:%1537=600%:%
%:%1538=600%:%
%:%1539=601%:%
%:%1540=601%:%
%:%1541=602%:%
%:%1542=602%:%
%:%1543=603%:%
%:%1544=603%:%
%:%1545=603%:%
%:%1546=604%:%
%:%1547=604%:%
%:%1548=604%:%
%:%1549=605%:%
%:%1550=605%:%
%:%1551=605%:%
%:%1554=608%:%
%:%1555=609%:%
%:%1556=609%:%
%:%1557=609%:%
%:%1558=610%:%
%:%1559=610%:%
%:%1560=610%:%
%:%1561=611%:%
%:%1562=611%:%
%:%1563=612%:%
%:%1564=612%:%
%:%1565=612%:%
%:%1566=612%:%
%:%1567=613%:%
%:%1568=613%:%
%:%1569=614%:%
%:%1570=614%:%
%:%1571=615%:%
%:%1572=615%:%
%:%1576=619%:%
%:%1577=620%:%
%:%1578=620%:%
%:%1579=620%:%
%:%1580=621%:%
%:%1581=621%:%
%:%1582=622%:%
%:%1583=622%:%
%:%1584=623%:%
%:%1585=623%:%
%:%1589=627%:%
%:%1590=628%:%
%:%1591=628%:%
%:%1592=629%:%
%:%1593=629%:%
%:%1594=629%:%
%:%1595=629%:%
%:%1596=630%:%
%:%1597=630%:%
%:%1598=631%:%
%:%1599=631%:%
%:%1600=632%:%
%:%1601=632%:%
%:%1605=636%:%
%:%1606=637%:%
%:%1607=637%:%
%:%1608=638%:%
%:%1609=638%:%
%:%1610=639%:%
%:%1611=639%:%
%:%1612=640%:%
%:%1613=640%:%
%:%1614=641%:%
%:%1615=641%:%
%:%1619=645%:%
%:%1620=646%:%
%:%1621=646%:%
%:%1622=647%:%
%:%1623=647%:%
%:%1624=647%:%
%:%1625=647%:%
%:%1626=648%:%
%:%1627=648%:%
%:%1628=649%:%
%:%1629=649%:%
%:%1630=650%:%
%:%1631=650%:%
%:%1635=654%:%
%:%1636=655%:%
%:%1637=655%:%
%:%1638=655%:%
%:%1639=656%:%
%:%1640=656%:%
%:%1641=657%:%
%:%1642=657%:%
%:%1643=658%:%
%:%1644=658%:%
%:%1648=662%:%
%:%1649=663%:%
%:%1650=663%:%
%:%1651=664%:%
%:%1652=664%:%
%:%1653=665%:%
%:%1654=665%:%
%:%1655=666%:%
%:%1656=666%:%
%:%1657=667%:%
%:%1658=667%:%
%:%1659=668%:%
%:%1660=668%:%
%:%1661=668%:%
%:%1662=668%:%
%:%1663=668%:%
%:%1664=669%:%
%:%1665=669%:%
%:%1666=670%:%
%:%1667=670%:%
%:%1668=671%:%
%:%1669=671%:%
%:%1670=672%:%
%:%1671=672%:%
%:%1672=673%:%
%:%1673=674%:%
%:%1674=674%:%
%:%1675=675%:%
%:%1676=675%:%
%:%1677=676%:%
%:%1678=676%:%
%:%1679=677%:%
%:%1689=680%:%
%:%1690=681%:%
%:%1692=683%:%
%:%1693=683%:%
%:%1694=684%:%
%:%1695=685%:%
%:%1696=686%:%
%:%1697=687%:%
%:%1698=688%:%
%:%1699=689%:%
%:%1700=690%:%
%:%1701=690%:%
%:%1702=691%:%
%:%1709=692%:%
%:%1710=692%:%
%:%1711=693%:%
%:%1712=693%:%
%:%1713=694%:%
%:%1714=694%:%
%:%1715=694%:%
%:%1716=694%:%
%:%1717=695%:%
%:%1718=695%:%
%:%1719=696%:%
%:%1720=696%:%
%:%1721=697%:%
%:%1722=697%:%
%:%1723=697%:%
%:%1724=698%:%
%:%1725=698%:%
%:%1726=699%:%
%:%1727=700%:%
%:%1728=701%:%
%:%1729=702%:%
%:%1744=706%:%
%:%1748=708%:%
%:%1760=710%:%
%:%1762=712%:%
%:%1763=712%:%
%:%1764=713%:%
%:%1765=714%:%
%:%1766=715%:%
%:%1767=716%:%
%:%1768=717%:%
%:%1769=717%:%
%:%1770=718%:%
%:%1771=719%:%
%:%1778=720%:%
%:%1779=720%:%
%:%1780=721%:%
%:%1781=721%:%
%:%1782=722%:%
%:%1783=722%:%
%:%1784=723%:%
%:%1785=723%:%
%:%1786=724%:%
%:%1787=724%:%
%:%1788=724%:%
%:%1789=724%:%
%:%1790=725%:%
%:%1791=725%:%
%:%1792=725%:%
%:%1793=725%:%
%:%1794=726%:%
%:%1795=726%:%
%:%1796=727%:%
%:%1797=727%:%
%:%1798=727%:%
%:%1799=727%:%
%:%1800=728%:%
%:%1801=728%:%
%:%1802=728%:%
%:%1803=728%:%
%:%1804=729%:%
%:%1805=729%:%
%:%1806=730%:%
%:%1807=730%:%
%:%1808=731%:%
%:%1809=731%:%
%:%1810=732%:%
%:%1811=732%:%
%:%1812=733%:%
%:%1813=733%:%
%:%1814=733%:%
%:%1815=733%:%
%:%1816=734%:%
%:%1817=734%:%
%:%1818=734%:%
%:%1819=735%:%
%:%1820=735%:%
%:%1821=735%:%
%:%1822=736%:%
%:%1823=736%:%
%:%1824=736%:%
%:%1825=737%:%
%:%1826=737%:%
%:%1827=737%:%
%:%1828=738%:%
%:%1829=738%:%
%:%1830=738%:%
%:%1831=739%:%
%:%1837=739%:%
%:%1840=740%:%
%:%1841=741%:%
%:%1842=741%:%
%:%1843=742%:%
%:%1850=743%:%
%:%1851=743%:%
%:%1852=744%:%
%:%1853=744%:%
%:%1854=745%:%
%:%1855=745%:%
%:%1856=745%:%
%:%1857=746%:%
%:%1858=746%:%
%:%1859=746%:%
%:%1860=746%:%
%:%1861=746%:%
%:%1862=747%:%
%:%1863=747%:%
%:%1864=747%:%
%:%1865=747%:%
%:%1866=748%:%
%:%1867=748%:%
%:%1868=748%:%
%:%1869=748%:%
%:%1870=749%:%
%:%1871=749%:%
%:%1872=750%:%
%:%1873=750%:%
%:%1874=751%:%
%:%1875=751%:%
%:%1876=752%:%
%:%1877=752%:%
%:%1878=752%:%
%:%1879=752%:%
%:%1880=753%:%
%:%1881=753%:%
%:%1882=753%:%
%:%1883=753%:%
%:%1884=753%:%
%:%1885=754%:%
%:%1886=754%:%
%:%1887=754%:%
%:%1888=754%:%
%:%1889=754%:%
%:%1890=755%:%
%:%1891=755%:%
%:%1892=755%:%
%:%1893=755%:%
%:%1894=756%:%
%:%1904=759%:%
%:%1906=761%:%
%:%1907=761%:%
%:%1908=762%:%
%:%1912=767%:%
%:%1913=768%:%
%:%1915=770%:%
%:%1916=770%:%
%:%1917=771%:%
%:%1926=776%:%
%:%1938=778%:%
%:%1939=779%:%
%:%1941=781%:%
%:%1942=781%:%
%:%1943=782%:%
%:%1944=783%:%
%:%1945=784%:%
%:%1946=785%:%
%:%1947=786%:%
%:%1948=787%:%
%:%1949=788%:%
%:%1950=788%:%
%:%1951=789%:%
%:%1958=790%:%
%:%1959=790%:%
%:%1960=791%:%
%:%1961=791%:%
%:%1962=792%:%
%:%1963=792%:%
%:%1964=792%:%
%:%1965=792%:%
%:%1966=793%:%
%:%1967=793%:%
%:%1968=794%:%
%:%1969=794%:%
%:%1970=795%:%
%:%1971=795%:%
%:%1972=795%:%
%:%1973=795%:%
%:%1974=796%:%
%:%1975=796%:%
%:%1976=797%:%
%:%1977=797%:%
%:%1978=798%:%
%:%1979=798%:%
%:%1980=798%:%
%:%1981=799%:%
%:%1982=799%:%
%:%1983=800%:%
%:%1984=801%:%
%:%1985=802%:%
%:%1995=805%:%
%:%1997=807%:%
%:%1998=807%:%
%:%1999=808%:%
%:%2000=809%:%
%:%2001=810%:%
%:%2002=811%:%
%:%2003=812%:%
%:%2004=813%:%
%:%2005=813%:%
%:%2006=814%:%
%:%2013=815%:%
%:%2014=815%:%
%:%2015=816%:%
%:%2016=816%:%
%:%2017=817%:%
%:%2018=817%:%
%:%2019=817%:%
%:%2020=817%:%
%:%2021=818%:%
%:%2022=818%:%
%:%2023=819%:%
%:%2024=819%:%
%:%2025=820%:%
%:%2026=820%:%
%:%2027=820%:%
%:%2028=821%:%
%:%2029=821%:%
%:%2030=821%:%
%:%2031=822%:%
%:%2032=822%:%
%:%2033=823%:%
%:%2034=823%:%
%:%2035=824%:%
%:%2036=824%:%
%:%2037=824%:%
%:%2038=825%:%
%:%2039=825%:%
%:%2040=826%:%
%:%2041=827%:%
%:%2042=828%:%
%:%2052=831%:%
%:%2054=833%:%
%:%2055=833%:%
%:%2056=834%:%
%:%2063=838%:%
%:%2075=840%:%
%:%2077=842%:%
%:%2078=842%:%
%:%2079=843%:%
%:%2080=844%:%
%:%2087=845%:%
%:%2088=845%:%
%:%2089=846%:%
%:%2090=846%:%
%:%2091=847%:%
%:%2092=847%:%
%:%2093=847%:%
%:%2094=848%:%
%:%2095=848%:%
%:%2096=848%:%
%:%2097=848%:%
%:%2098=849%:%
%:%2099=849%:%
%:%2100=850%:%
%:%2101=850%:%
%:%2102=851%:%
%:%2103=851%:%
%:%2104=851%:%
%:%2105=852%:%
%:%2106=852%:%
%:%2107=853%:%
%:%2108=853%:%
%:%2109=854%:%
%:%2110=854%:%
%:%2111=855%:%
%:%2112=855%:%
%:%2113=855%:%
%:%2114=855%:%
%:%2115=856%:%
%:%2116=856%:%
%:%2117=857%:%
%:%2118=857%:%
%:%2119=858%:%
%:%2120=858%:%
%:%2121=859%:%
%:%2122=859%:%
%:%2123=860%:%
%:%2124=861%:%
%:%2125=861%:%
%:%2126=861%:%
%:%2127=861%:%
%:%2128=862%:%
%:%2129=862%:%
%:%2130=862%:%
%:%2131=862%:%
%:%2132=863%:%
%:%2133=863%:%
%:%2134=863%:%
%:%2135=864%:%
%:%2136=864%:%
%:%2137=865%:%
%:%2138=866%:%
%:%2139=866%:%
%:%2140=867%:%
%:%2141=867%:%
%:%2142=868%:%
%:%2143=869%:%
%:%2144=869%:%
%:%2145=869%:%
%:%2146=870%:%
%:%2147=870%:%
%:%2148=870%:%
%:%2151=873%:%
%:%2152=874%:%
%:%2153=874%:%
%:%2154=875%:%
%:%2155=875%:%
%:%2156=876%:%
%:%2157=876%:%
%:%2158=876%:%
%:%2159=877%:%
%:%2160=877%:%
%:%2161=878%:%
%:%2162=878%:%
%:%2163=878%:%
%:%2164=879%:%
%:%2165=879%:%
%:%2166=879%:%
%:%2167=880%:%
%:%2168=880%:%
%:%2169=880%:%
%:%2170=881%:%
%:%2171=881%:%
%:%2172=882%:%
%:%2173=882%:%
%:%2174=882%:%
%:%2175=883%:%
%:%2176=883%:%
%:%2177=884%:%
%:%2178=884%:%
%:%2179=884%:%
%:%2180=885%:%
%:%2181=885%:%
%:%2182=885%:%
%:%2183=885%:%
%:%2184=886%:%
%:%2185=886%:%
%:%2186=887%:%
%:%2187=887%:%
%:%2188=887%:%
%:%2189=888%:%
%:%2190=888%:%
%:%2191=889%:%
%:%2192=889%:%
%:%2193=889%:%
%:%2194=890%:%
%:%2195=890%:%
%:%2196=890%:%
%:%2197=891%:%
%:%2198=891%:%
%:%2199=891%:%
%:%2200=892%:%
%:%2201=893%:%
%:%2202=893%:%
%:%2203=893%:%
%:%2204=894%:%
%:%2205=894%:%
%:%2206=894%:%
%:%2207=895%:%
%:%2208=895%:%
%:%2209=896%:%
%:%2210=896%:%
%:%2211=897%:%
%:%2212=897%:%
%:%2213=897%:%
%:%2214=897%:%
%:%2215=898%:%
%:%2216=898%:%
%:%2217=899%:%
%:%2218=899%:%
%:%2219=900%:%
%:%2220=900%:%
%:%2221=900%:%
%:%2222=900%:%
%:%2223=901%:%
%:%2224=902%:%
%:%2225=902%:%
%:%2226=903%:%
%:%2227=903%:%
%:%2228=904%:%
%:%2229=904%:%
%:%2230=904%:%
%:%2231=904%:%
%:%2232=905%:%
%:%2233=905%:%
%:%2234=906%:%
%:%2235=907%:%
%:%2236=907%:%
%:%2237=907%:%
%:%2238=908%:%
%:%2239=908%:%
%:%2240=908%:%
%:%2241=908%:%
%:%2242=908%:%
%:%2243=909%:%
%:%2244=909%:%
%:%2245=909%:%
%:%2246=909%:%
%:%2247=910%:%
%:%2248=910%:%
%:%2249=911%:%
%:%2250=912%:%
%:%2251=912%:%
%:%2252=912%:%
%:%2253=912%:%
%:%2254=912%:%
%:%2255=913%:%
%:%2256=913%:%
%:%2257=913%:%
%:%2258=913%:%
%:%2259=914%:%
%:%2260=914%:%
%:%2261=915%:%
%:%2262=915%:%
%:%2263=916%:%
%:%2264=916%:%
%:%2265=916%:%
%:%2266=916%:%
%:%2267=917%:%
%:%2268=917%:%
%:%2269=917%:%
%:%2270=917%:%
%:%2271=918%:%
%:%2272=918%:%
%:%2273=919%:%
%:%2274=920%:%
%:%2275=920%:%
%:%2276=920%:%
%:%2277=921%:%
%:%2278=921%:%
%:%2279=921%:%
%:%2280=921%:%
%:%2281=921%:%
%:%2282=922%:%
%:%2283=922%:%
%:%2284=922%:%
%:%2285=923%:%
%:%2286=923%:%
%:%2287=924%:%
%:%2288=924%:%
%:%2289=925%:%
%:%2290=926%:%
%:%2291=926%:%
%:%2292=926%:%
%:%2293=927%:%
%:%2294=927%:%
%:%2295=927%:%
%:%2296=928%:%
%:%2297=929%:%
%:%2298=929%:%
%:%2299=929%:%
%:%2300=930%:%
%:%2301=930%:%
%:%2302=930%:%
%:%2303=931%:%
%:%2304=932%:%
%:%2305=932%:%
%:%2306=932%:%
%:%2307=933%:%
%:%2308=933%:%
%:%2309=933%:%
%:%2310=933%:%
%:%2311=933%:%
%:%2312=934%:%
%:%2313=934%:%
%:%2314=934%:%
%:%2315=934%:%
%:%2316=935%:%
%:%2317=935%:%
%:%2318=936%:%
%:%2319=936%:%
%:%2320=936%:%
%:%2321=936%:%
%:%2322=936%:%
%:%2323=937%:%
%:%2324=937%:%
%:%2325=937%:%
%:%2326=937%:%
%:%2327=938%:%
%:%2328=938%:%
%:%2329=939%:%
%:%2330=939%:%
%:%2331=940%:%
%:%2332=940%:%
%:%2333=940%:%
%:%2334=940%:%
%:%2335=940%:%
%:%2336=941%:%
%:%2337=941%:%
%:%2338=942%:%
%:%2339=942%:%
%:%2340=943%:%
%:%2341=943%:%
%:%2342=944%:%
%:%2343=945%:%
%:%2344=945%:%
%:%2345=945%:%
%:%2346=946%:%
%:%2347=946%:%
%:%2348=947%:%
%:%2349=947%:%
%:%2350=948%:%
%:%2351=949%:%
%:%2352=949%:%
%:%2353=949%:%
%:%2354=950%:%
%:%2355=950%:%
%:%2356=951%:%
%:%2357=951%:%
%:%2358=952%:%
%:%2359=952%:%
%:%2360=953%:%
%:%2361=953%:%
%:%2362=953%:%
%:%2363=953%:%
%:%2364=954%:%
%:%2365=954%:%
%:%2366=955%:%
%:%2367=955%:%
%:%2368=956%:%
%:%2369=956%:%
%:%2370=957%:%
%:%2371=957%:%
%:%2372=958%:%
%:%2373=958%:%
%:%2374=958%:%
%:%2375=958%:%
%:%2376=959%:%
%:%2377=959%:%
%:%2378=960%:%
%:%2379=960%:%
%:%2380=960%:%
%:%2381=960%:%
%:%2382=961%:%
%:%2383=961%:%
%:%2384=961%:%
%:%2385=962%:%
%:%2386=962%:%
%:%2387=963%:%
%:%2388=964%:%
%:%2389=964%:%
%:%2390=965%:%
%:%2391=965%:%
%:%2392=966%:%
%:%2393=967%:%
%:%2394=967%:%
%:%2395=967%:%
%:%2396=968%:%
%:%2397=968%:%
%:%2398=968%:%
%:%2401=971%:%
%:%2402=972%:%
%:%2403=972%:%
%:%2404=973%:%
%:%2405=973%:%
%:%2406=974%:%
%:%2407=974%:%
%:%2408=974%:%
%:%2409=975%:%
%:%2410=975%:%
%:%2411=976%:%
%:%2412=976%:%
%:%2413=976%:%
%:%2414=977%:%
%:%2415=977%:%
%:%2416=977%:%
%:%2417=978%:%
%:%2418=978%:%
%:%2419=978%:%
%:%2420=979%:%
%:%2421=979%:%
%:%2422=980%:%
%:%2423=980%:%
%:%2424=980%:%
%:%2425=981%:%
%:%2426=981%:%
%:%2427=982%:%
%:%2428=982%:%
%:%2429=982%:%
%:%2430=983%:%
%:%2431=983%:%
%:%2432=983%:%
%:%2433=984%:%
%:%2434=984%:%
%:%2435=985%:%
%:%2436=985%:%
%:%2437=985%:%
%:%2438=986%:%
%:%2439=986%:%
%:%2440=987%:%
%:%2441=987%:%
%:%2442=987%:%
%:%2443=988%:%
%:%2444=989%:%
%:%2445=989%:%
%:%2446=990%:%
%:%2447=991%:%
%:%2448=991%:%
%:%2449=992%:%
%:%2450=992%:%
%:%2451=992%:%
%:%2452=993%:%
%:%2453=993%:%
%:%2454=994%:%
%:%2455=994%:%
%:%2456=995%:%
%:%2457=995%:%
%:%2458=995%:%
%:%2459=995%:%
%:%2460=996%:%
%:%2461=996%:%
%:%2462=997%:%
%:%2463=997%:%
%:%2464=998%:%
%:%2465=998%:%
%:%2466=998%:%
%:%2467=999%:%
%:%2468=999%:%
%:%2469=1000%:%
%:%2470=1001%:%
%:%2471=1001%:%
%:%2472=1002%:%
%:%2473=1002%:%
%:%2474=1003%:%
%:%2475=1003%:%
%:%2476=1003%:%
%:%2477=1003%:%
%:%2478=1004%:%
%:%2479=1004%:%
%:%2480=1005%:%
%:%2481=1006%:%
%:%2482=1006%:%
%:%2483=1006%:%
%:%2484=1007%:%
%:%2485=1007%:%
%:%2486=1007%:%
%:%2487=1007%:%
%:%2488=1007%:%
%:%2489=1008%:%
%:%2490=1008%:%
%:%2491=1008%:%
%:%2492=1009%:%
%:%2493=1009%:%
%:%2494=1009%:%
%:%2495=1010%:%
%:%2496=1010%:%
%:%2497=1010%:%
%:%2498=1010%:%
%:%2499=1010%:%
%:%2500=1011%:%
%:%2501=1011%:%
%:%2502=1011%:%
%:%2503=1011%:%
%:%2504=1012%:%
%:%2505=1012%:%
%:%2506=1013%:%
%:%2507=1013%:%
%:%2508=1014%:%
%:%2509=1014%:%
%:%2510=1014%:%
%:%2511=1014%:%
%:%2512=1015%:%
%:%2513=1015%:%
%:%2514=1016%:%
%:%2515=1017%:%
%:%2516=1017%:%
%:%2517=1017%:%
%:%2518=1018%:%
%:%2519=1018%:%
%:%2520=1018%:%
%:%2521=1018%:%
%:%2522=1018%:%
%:%2523=1019%:%
%:%2524=1019%:%
%:%2525=1019%:%
%:%2526=1020%:%
%:%2527=1020%:%
%:%2528=1020%:%
%:%2529=1021%:%
%:%2530=1021%:%
%:%2531=1021%:%
%:%2532=1021%:%
%:%2533=1022%:%
%:%2534=1022%:%
%:%2535=1023%:%
%:%2536=1023%:%
%:%2537=1024%:%
%:%2538=1024%:%
%:%2539=1024%:%
%:%2540=1025%:%
%:%2541=1026%:%
%:%2542=1026%:%
%:%2543=1026%:%
%:%2544=1027%:%
%:%2545=1027%:%
%:%2546=1028%:%
%:%2547=1028%:%
%:%2548=1029%:%
%:%2549=1029%:%
%:%2550=1030%:%
%:%2551=1031%:%
%:%2552=1031%:%
%:%2553=1031%:%
%:%2554=1032%:%
%:%2555=1032%:%
%:%2556=1033%:%
%:%2557=1033%:%
%:%2558=1034%:%
%:%2559=1035%:%
%:%2560=1035%:%
%:%2561=1035%:%
%:%2562=1036%:%
%:%2563=1036%:%
%:%2564=1037%:%
%:%2565=1037%:%
%:%2566=1038%:%
%:%2572=1038%:%
%:%2575=1039%:%
%:%2576=1040%:%
%:%2577=1040%:%
%:%2578=1041%:%
%:%2579=1042%:%
%:%2586=1043%:%
%:%2587=1043%:%
%:%2588=1044%:%
%:%2589=1044%:%
%:%2590=1045%:%
%:%2591=1045%:%
%:%2592=1046%:%
%:%2593=1046%:%
%:%2594=1047%:%
%:%2595=1047%:%
%:%2596=1048%:%
%:%2597=1048%:%
%:%2598=1049%:%
%:%2599=1050%:%
%:%2600=1050%:%
%:%2601=1051%:%
%:%2602=1051%:%
%:%2603=1052%:%
%:%2604=1052%:%
%:%2605=1052%:%
%:%2606=1053%:%
%:%2607=1053%:%
%:%2608=1053%:%
%:%2609=1054%:%
%:%2610=1054%:%
%:%2611=1054%:%
%:%2612=1054%:%
%:%2613=1055%:%
%:%2614=1055%:%
%:%2615=1056%:%
%:%2616=1057%:%
%:%2617=1057%:%
%:%2618=1057%:%
%:%2619=1058%:%
%:%2620=1058%:%
%:%2621=1058%:%
%:%2622=1059%:%
%:%2623=1059%:%
%:%2624=1059%:%
%:%2625=1060%:%
%:%2626=1060%:%
%:%2627=1060%:%
%:%2628=1060%:%
%:%2629=1060%:%
%:%2630=1061%:%
%:%2631=1061%:%
%:%2632=1061%:%
%:%2633=1062%:%
%:%2634=1062%:%
%:%2635=1063%:%
%:%2636=1063%:%
%:%2637=1064%:%
%:%2638=1064%:%
%:%2639=1065%:%
%:%2640=1065%:%
%:%2641=1066%:%
%:%2642=1066%:%
%:%2643=1067%:%
%:%2644=1067%:%
%:%2645=1068%:%
%:%2646=1069%:%
%:%2647=1069%:%
%:%2648=1070%:%
%:%2649=1070%:%
%:%2650=1071%:%
%:%2651=1071%:%
%:%2652=1072%:%
%:%2653=1072%:%
%:%2654=1073%:%
%:%2655=1073%:%
%:%2656=1074%:%
%:%2657=1075%:%
%:%2658=1075%:%
%:%2659=1076%:%
%:%2660=1076%:%
%:%2661=1077%:%
%:%2662=1077%:%
%:%2663=1078%:%
%:%2664=1078%:%
%:%2665=1079%:%
%:%2666=1079%:%
%:%2667=1080%:%
%:%2668=1080%:%
%:%2669=1081%:%
%:%2670=1081%:%
%:%2671=1082%:%
%:%2672=1082%:%
%:%2673=1083%:%
%:%2674=1084%:%
%:%2675=1084%:%
%:%2676=1085%:%
%:%2677=1085%:%
%:%2678=1086%:%
%:%2679=1086%:%
%:%2680=1086%:%
%:%2681=1087%:%
%:%2682=1087%:%
%:%2683=1087%:%
%:%2684=1088%:%
%:%2685=1088%:%
%:%2686=1088%:%
%:%2687=1089%:%
%:%2688=1089%:%
%:%2689=1090%:%
%:%2690=1090%:%
%:%2691=1091%:%
%:%2692=1092%:%
%:%2693=1093%:%
%:%2694=1093%:%
%:%2695=1093%:%
%:%2696=1094%:%
%:%2697=1094%:%
%:%2698=1094%:%
%:%2699=1095%:%
%:%2700=1095%:%
%:%2701=1095%:%
%:%2702=1096%:%
%:%2703=1097%:%
%:%2704=1097%:%
%:%2705=1097%:%
%:%2706=1098%:%
%:%2707=1098%:%
%:%2708=1099%:%
%:%2709=1099%:%
%:%2710=1100%:%
%:%2720=1103%:%
%:%2722=1105%:%
%:%2723=1105%:%
%:%2724=1106%:%
%:%2725=1107%:%
%:%2727=1109%:%
%:%2734=1110%:%
%:%2735=1110%:%
%:%2736=1111%:%
%:%2737=1111%:%
%:%2738=1112%:%
%:%2739=1112%:%
%:%2740=1113%:%
%:%2741=1114%:%
%:%2742=1114%:%
%:%2743=1114%:%
%:%2744=1115%:%
%:%2745=1115%:%
%:%2746=1115%:%
%:%2747=1116%:%
%:%2748=1116%:%
%:%2749=1117%:%
%:%2750=1117%:%
%:%2751=1118%:%
%:%2752=1119%:%
%:%2753=1119%:%
%:%2754=1119%:%
%:%2755=1120%:%
%:%2756=1121%:%
%:%2757=1121%:%
%:%2758=1122%:%
%:%2759=1122%:%
%:%2760=1123%:%
%:%2761=1123%:%
%:%2762=1123%:%
%:%2763=1124%:%
%:%2764=1124%:%
%:%2765=1125%:%
%:%2766=1126%:%
%:%2767=1126%:%
%:%2768=1127%:%
%:%2769=1127%:%
%:%2770=1127%:%
%:%2771=1127%:%
%:%2772=1128%:%
%:%2773=1129%:%
%:%2774=1129%:%
%:%2775=1130%:%
%:%2776=1130%:%
%:%2777=1131%:%
%:%2778=1131%:%
%:%2779=1132%:%
%:%2780=1133%:%
%:%2781=1133%:%
%:%2782=1133%:%
%:%2783=1134%:%
%:%2784=1134%:%
%:%2785=1134%:%
%:%2786=1135%:%
%:%2787=1135%:%
%:%2788=1136%:%
%:%2789=1136%:%
%:%2790=1137%:%
%:%2791=1137%:%
%:%2792=1138%:%
%:%2793=1138%:%
%:%2794=1138%:%
%:%2795=1138%:%
%:%2796=1139%:%
%:%2797=1139%:%
%:%2798=1139%:%
%:%2799=1140%:%
%:%2800=1140%:%
%:%2801=1141%:%
%:%2802=1141%:%
%:%2803=1141%:%
%:%2804=1141%:%
%:%2805=1142%:%
%:%2806=1142%:%
%:%2807=1143%:%
%:%2808=1144%:%
%:%2809=1144%:%
%:%2810=1144%:%
%:%2811=1145%:%
%:%2812=1145%:%
%:%2813=1145%:%
%:%2814=1145%:%
%:%2815=1145%:%
%:%2816=1146%:%
%:%2817=1146%:%
%:%2818=1146%:%
%:%2819=1147%:%
%:%2820=1147%:%
%:%2821=1148%:%
%:%2822=1148%:%
%:%2823=1149%:%
%:%2824=1150%:%
%:%2825=1151%:%
%:%2826=1151%:%
%:%2827=1151%:%
%:%2828=1152%:%
%:%2829=1152%:%
%:%2830=1153%:%
%:%2831=1153%:%
%:%2832=1154%:%
%:%2833=1154%:%
%:%2834=1155%:%
%:%2835=1155%:%
%:%2836=1156%:%
%:%2837=1156%:%
%:%2838=1157%:%
%:%2839=1157%:%
%:%2840=1158%:%
%:%2841=1158%:%
%:%2842=1158%:%
%:%2843=1159%:%
%:%2844=1159%:%
%:%2845=1160%:%
%:%2846=1160%:%
%:%2847=1160%:%
%:%2848=1161%:%
%:%2849=1161%:%
%:%2850=1162%:%
%:%2851=1162%:%
%:%2852=1163%:%
%:%2853=1163%:%
%:%2854=1164%:%
%:%2855=1164%:%
%:%2856=1164%:%
%:%2857=1164%:%
%:%2858=1165%:%
%:%2859=1165%:%
%:%2860=1166%:%
%:%2861=1166%:%
%:%2862=1167%:%
%:%2863=1167%:%
%:%2864=1167%:%
%:%2865=1167%:%
%:%2866=1168%:%
%:%2867=1168%:%
%:%2868=1169%:%
%:%2869=1169%:%
%:%2870=1169%:%
%:%2871=1170%:%
%:%2872=1170%:%
%:%2873=1171%:%
%:%2874=1171%:%
%:%2875=1171%:%
%:%2876=1172%:%
%:%2877=1172%:%
%:%2878=1172%:%
%:%2879=1173%:%
%:%2880=1173%:%
%:%2881=1173%:%
%:%2882=1174%:%
%:%2883=1174%:%
%:%2884=1175%:%
%:%2885=1175%:%
%:%2886=1175%:%
%:%2890=1179%:%
%:%2891=1180%:%
%:%2892=1180%:%
%:%2893=1180%:%
%:%2894=1181%:%
%:%2895=1181%:%
%:%2896=1182%:%
%:%2897=1182%:%
%:%2900=1185%:%
%:%2901=1186%:%
%:%2902=1186%:%
%:%2903=1186%:%
%:%2904=1187%:%
%:%2905=1187%:%
%:%2906=1188%:%
%:%2907=1188%:%
%:%2910=1191%:%
%:%2911=1192%:%
%:%2912=1192%:%
%:%2913=1192%:%
%:%2914=1193%:%
%:%2915=1193%:%
%:%2916=1194%:%
%:%2917=1194%:%
%:%2918=1194%:%
%:%2919=1194%:%
%:%2920=1194%:%
%:%2921=1195%:%
%:%2922=1195%:%
%:%2923=1196%:%
%:%2924=1196%:%
%:%2925=1197%:%
%:%2926=1197%:%
%:%2927=1198%:%
%:%2928=1199%:%
%:%2929=1199%:%
%:%2930=1199%:%
%:%2931=1200%:%
%:%2932=1200%:%
%:%2933=1200%:%
%:%2934=1201%:%
%:%2935=1201%:%
%:%2936=1202%:%
%:%2937=1202%:%
%:%2938=1203%:%
%:%2939=1203%:%
%:%2940=1204%:%
%:%2941=1205%:%
%:%2942=1205%:%
%:%2943=1205%:%
%:%2944=1205%:%
%:%2945=1206%:%
%:%2946=1206%:%
%:%2947=1206%:%
%:%2948=1207%:%
%:%2949=1207%:%
%:%2950=1208%:%
%:%2951=1209%:%
%:%2952=1209%:%
%:%2953=1209%:%
%:%2954=1209%:%
%:%2955=1210%:%
%:%2956=1210%:%
%:%2957=1211%:%
%:%2958=1212%:%
%:%2959=1212%:%
%:%2960=1212%:%
%:%2961=1213%:%
%:%2962=1213%:%
%:%2963=1213%:%
%:%2964=1214%:%
%:%2965=1214%:%
%:%2966=1215%:%
%:%2967=1216%:%
%:%2968=1217%:%
%:%2969=1217%:%
%:%2970=1218%:%
%:%2971=1219%:%
%:%2972=1220%:%
%:%2973=1220%:%
%:%2974=1220%:%
%:%2975=1221%:%
%:%2976=1221%:%
%:%2977=1222%:%
%:%2978=1222%:%
%:%2979=1223%:%
%:%2980=1223%:%
%:%2981=1223%:%
%:%2982=1224%:%
%:%2983=1224%:%
%:%2984=1225%:%
%:%2985=1226%:%
%:%2986=1227%:%
%:%2987=1228%:%
%:%2988=1228%:%
%:%2989=1228%:%
%:%2990=1229%:%
%:%2991=1229%:%
%:%2992=1229%:%
%:%2993=1230%:%
%:%2994=1230%:%
%:%2995=1230%:%
%:%2996=1230%:%
%:%2997=1231%:%
%:%2998=1231%:%
%:%2999=1232%:%
%:%3000=1233%:%
%:%3001=1234%:%
%:%3002=1235%:%
%:%3003=1235%:%
%:%3004=1236%:%
%:%3005=1236%:%
%:%3006=1236%:%
%:%3007=1237%:%
%:%3008=1237%:%
%:%3009=1237%:%
%:%3010=1238%:%
%:%3011=1238%:%
%:%3012=1238%:%
%:%3013=1239%:%
%:%3014=1239%:%
%:%3015=1239%:%
%:%3016=1240%:%
%:%3017=1240%:%
%:%3018=1240%:%
%:%3020=1242%:%
%:%3021=1242%:%
%:%3022=1242%:%
%:%3023=1243%:%
%:%3024=1243%:%
%:%3025=1244%:%
%:%3026=1244%:%
%:%3029=1247%:%
%:%3030=1248%:%
%:%3031=1248%:%
%:%3032=1248%:%
%:%3033=1249%:%
%:%3034=1249%:%
%:%3035=1250%:%
%:%3036=1250%:%
%:%3039=1253%:%
%:%3040=1254%:%
%:%3041=1254%:%
%:%3042=1254%:%
%:%3043=1255%:%
%:%3044=1255%:%
%:%3045=1256%:%
%:%3046=1256%:%
%:%3047=1256%:%
%:%3048=1256%:%
%:%3049=1256%:%
%:%3050=1257%:%
%:%3051=1257%:%
%:%3052=1258%:%
%:%3053=1258%:%
%:%3054=1259%:%
%:%3055=1260%:%
%:%3056=1260%:%
%:%3057=1261%:%
%:%3058=1261%:%
%:%3059=1261%:%
%:%3060=1261%:%
%:%3061=1262%:%
%:%3062=1262%:%
%:%3063=1263%:%
%:%3064=1263%:%
%:%3065=1263%:%
%:%3066=1263%:%
%:%3067=1263%:%
%:%3068=1264%:%
%:%3078=1267%:%
%:%3080=1269%:%
%:%3081=1269%:%
%:%3082=1270%:%
%:%3083=1271%:%
%:%3084=1272%:%
%:%3091=1273%:%
%:%3092=1273%:%
%:%3093=1274%:%
%:%3094=1274%:%
%:%3095=1275%:%
%:%3096=1275%:%
%:%3097=1276%:%
%:%3098=1276%:%
%:%3099=1276%:%
%:%3100=1277%:%
%:%3101=1277%:%
%:%3102=1278%:%
%:%3103=1278%:%
%:%3104=1278%:%
%:%3105=1279%:%
%:%3106=1279%:%
%:%3108=1281%:%
%:%3109=1282%:%
%:%3110=1282%:%
%:%3111=1282%:%
%:%3112=1283%:%
%:%3113=1283%:%
%:%3114=1283%:%
%:%3116=1285%:%
%:%3117=1286%:%
%:%3118=1286%:%
%:%3119=1286%:%
%:%3120=1287%:%
%:%3121=1287%:%
%:%3122=1287%:%
%:%3124=1289%:%
%:%3125=1290%:%
%:%3126=1290%:%
%:%3127=1291%:%
%:%3128=1291%:%
%:%3129=1292%:%
%:%3130=1293%:%
%:%3131=1293%:%
%:%3132=1294%:%
%:%3133=1294%:%
%:%3134=1295%:%
%:%3135=1295%:%
%:%3136=1296%:%
%:%3137=1296%:%
%:%3138=1296%:%
%:%3139=1297%:%
%:%3140=1297%:%
%:%3141=1298%:%
%:%3142=1298%:%
%:%3143=1299%:%
%:%3144=1299%:%
%:%3145=1300%:%
%:%3146=1301%:%
%:%3147=1301%:%
%:%3148=1302%:%
%:%3149=1303%:%
%:%3150=1303%:%
%:%3151=1303%:%
%:%3152=1304%:%
%:%3153=1304%:%
%:%3154=1304%:%
%:%3155=1305%:%
%:%3156=1305%:%
%:%3157=1306%:%
%:%3158=1306%:%
%:%3159=1306%:%
%:%3160=1306%:%
%:%3161=1307%:%
%:%3162=1307%:%
%:%3163=1308%:%
%:%3164=1308%:%
%:%3165=1309%:%
%:%3166=1309%:%
%:%3167=1310%:%
%:%3168=1311%:%
%:%3169=1311%:%
%:%3170=1311%:%
%:%3171=1312%:%
%:%3172=1312%:%
%:%3173=1313%:%
%:%3174=1314%:%
%:%3175=1314%:%
%:%3176=1314%:%
%:%3177=1315%:%
%:%3178=1315%:%
%:%3179=1316%:%
%:%3180=1316%:%
%:%3181=1317%:%
%:%3182=1317%:%
%:%3183=1318%:%
%:%3184=1319%:%
%:%3185=1319%:%
%:%3186=1319%:%
%:%3187=1320%:%
%:%3188=1320%:%
%:%3189=1321%:%
%:%3190=1322%:%
%:%3191=1322%:%
%:%3192=1322%:%
%:%3193=1323%:%
%:%3194=1323%:%
%:%3195=1324%:%
%:%3196=1324%:%
%:%3197=1324%:%
%:%3198=1325%:%
%:%3199=1326%:%
%:%3200=1326%:%
%:%3201=1326%:%
%:%3202=1327%:%
%:%3203=1327%:%
%:%3204=1328%:%
%:%3205=1328%:%
%:%3208=1331%:%
%:%3209=1332%:%
%:%3210=1332%:%
%:%3211=1333%:%
%:%3212=1333%:%
%:%3213=1334%:%
%:%3214=1334%:%
%:%3217=1337%:%
%:%3218=1338%:%
%:%3219=1338%:%
%:%3220=1339%:%
%:%3221=1339%:%
%:%3222=1340%:%
%:%3223=1340%:%
%:%3224=1340%:%
%:%3225=1341%:%
%:%3226=1341%:%
%:%3227=1342%:%
%:%3228=1342%:%
%:%3229=1342%:%
%:%3230=1343%:%
%:%3231=1344%:%
%:%3232=1344%:%
%:%3233=1345%:%
%:%3234=1345%:%
%:%3236=1347%:%
%:%3237=1348%:%
%:%3238=1348%:%
%:%3239=1349%:%
%:%3240=1349%:%
%:%3241=1350%:%
%:%3242=1350%:%
%:%3243=1351%:%
%:%3244=1351%:%
%:%3245=1351%:%
%:%3246=1351%:%
%:%3247=1352%:%
%:%3248=1352%:%
%:%3249=1353%:%
%:%3250=1353%:%
%:%3251=1353%:%
%:%3252=1353%:%
%:%3253=1354%:%
%:%3254=1354%:%
%:%3256=1356%:%
%:%3257=1357%:%
%:%3258=1357%:%
%:%3259=1358%:%
%:%3260=1358%:%
%:%3261=1358%:%
%:%3262=1358%:%
%:%3263=1359%:%
%:%3264=1359%:%
%:%3265=1360%:%
%:%3266=1360%:%
%:%3267=1361%:%
%:%3268=1361%:%
%:%3270=1363%:%
%:%3271=1364%:%
%:%3272=1364%:%
%:%3273=1364%:%
%:%3274=1365%:%
%:%3275=1365%:%
%:%3277=1367%:%
%:%3278=1368%:%
%:%3279=1368%:%
%:%3280=1368%:%
%:%3281=1369%:%
%:%3282=1369%:%
%:%3283=1370%:%
%:%3284=1370%:%
%:%3285=1371%:%
%:%3286=1371%:%
%:%3288=1373%:%
%:%3289=1374%:%
%:%3290=1374%:%
%:%3291=1374%:%
%:%3292=1375%:%
%:%3293=1375%:%
%:%3294=1376%:%
%:%3295=1376%:%
%:%3296=1377%:%
%:%3297=1377%:%
%:%3300=1380%:%
%:%3301=1381%:%
%:%3302=1381%:%
%:%3303=1382%:%
%:%3304=1382%:%
%:%3305=1383%:%
%:%3306=1383%:%
%:%3309=1386%:%
%:%3310=1387%:%
%:%3311=1387%:%
%:%3312=1388%:%
%:%3313=1388%:%
%:%3314=1389%:%
%:%3315=1389%:%
%:%3316=1389%:%
%:%3317=1390%:%
%:%3318=1390%:%
%:%3319=1391%:%
%:%3320=1391%:%
%:%3321=1391%:%
%:%3325=1395%:%
%:%3326=1396%:%
%:%3327=1396%:%
%:%3328=1397%:%
%:%3329=1397%:%
%:%3332=1400%:%
%:%3333=1401%:%
%:%3334=1401%:%
%:%3335=1402%:%
%:%3336=1402%:%
%:%3340=1406%:%
%:%3341=1407%:%
%:%3342=1407%:%
%:%3343=1407%:%
%:%3344=1408%:%
%:%3345=1408%:%
%:%3346=1408%:%
%:%3347=1409%:%
%:%3348=1409%:%
%:%3349=1410%:%
%:%3350=1410%:%
%:%3351=1410%:%
%:%3352=1410%:%
%:%3353=1411%:%
%:%3354=1411%:%
%:%3355=1412%:%
%:%3356=1412%:%
%:%3357=1413%:%
%:%3358=1413%:%
%:%3363=1418%:%
%:%3364=1419%:%
%:%3365=1419%:%
%:%3366=1419%:%
%:%3367=1420%:%
%:%3368=1420%:%
%:%3369=1420%:%
%:%3374=1425%:%
%:%3375=1426%:%
%:%3376=1426%:%
%:%3377=1427%:%
%:%3378=1427%:%
%:%3379=1427%:%
%:%3380=1427%:%
%:%3381=1428%:%
%:%3382=1428%:%
%:%3383=1428%:%
%:%3384=1429%:%
%:%3385=1429%:%
%:%3386=1429%:%
%:%3387=1430%:%
%:%3388=1430%:%
%:%3389=1430%:%
%:%3392=1433%:%
%:%3393=1434%:%
%:%3394=1434%:%
%:%3395=1434%:%
%:%3396=1435%:%
%:%3397=1435%:%
%:%3398=1436%:%
%:%3399=1436%:%
%:%3400=1437%:%
%:%3401=1437%:%
%:%3406=1442%:%
%:%3407=1443%:%
%:%3408=1443%:%
%:%3409=1443%:%
%:%3410=1444%:%
%:%3411=1444%:%
%:%3412=1444%:%
%:%3417=1449%:%
%:%3418=1450%:%
%:%3419=1450%:%
%:%3420=1451%:%
%:%3421=1451%:%
%:%3422=1451%:%
%:%3423=1452%:%
%:%3424=1453%:%
%:%3425=1453%:%
%:%3426=1453%:%
%:%3427=1454%:%
%:%3428=1454%:%
%:%3429=1454%:%
%:%3430=1455%:%
%:%3431=1456%:%
%:%3432=1456%:%
%:%3433=1456%:%
%:%3434=1457%:%
%:%3435=1457%:%
%:%3436=1457%:%
%:%3437=1458%:%
%:%3438=1459%:%
%:%3439=1459%:%
%:%3440=1460%:%
%:%3441=1460%:%
%:%3442=1460%:%
%:%3443=1461%:%
%:%3444=1461%:%
%:%3445=1462%:%
%:%3446=1462%:%
%:%3447=1462%:%
%:%3448=1463%:%
%:%3449=1463%:%
%:%3450=1464%:%
%:%3451=1464%:%
%:%3452=1464%:%
%:%3453=1465%:%
%:%3454=1465%:%
%:%3455=1466%:%
%:%3456=1466%:%
%:%3457=1467%:%
%:%3458=1468%:%
%:%3459=1469%:%
%:%3460=1469%:%
%:%3461=1469%:%
%:%3462=1470%:%
%:%3463=1470%:%
%:%3464=1470%:%
%:%3465=1471%:%
%:%3466=1471%:%
%:%3467=1471%:%
%:%3470=1474%:%
%:%3471=1475%:%
%:%3472=1475%:%
%:%3473=1475%:%
%:%3474=1476%:%
%:%3475=1476%:%
%:%3476=1477%:%
%:%3477=1477%:%
%:%3478=1477%:%
%:%3479=1478%:%
%:%3480=1478%:%
%:%3481=1479%:%
%:%3482=1479%:%
%:%3483=1479%:%
%:%3484=1479%:%
%:%3485=1480%:%
%:%3486=1480%:%
%:%3487=1481%:%
%:%3488=1481%:%
%:%3489=1482%:%
%:%3490=1482%:%
%:%3491=1483%:%
%:%3492=1483%:%
%:%3493=1484%:%
%:%3494=1484%:%
%:%3495=1484%:%
%:%3496=1485%:%
%:%3497=1485%:%
%:%3498=1486%:%
%:%3499=1486%:%
%:%3500=1486%:%
%:%3501=1487%:%
%:%3502=1487%:%
%:%3503=1487%:%
%:%3504=1488%:%
%:%3505=1488%:%
%:%3506=1489%:%
%:%3507=1489%:%
%:%3508=1490%:%
%:%3509=1491%:%
%:%3510=1491%:%
%:%3511=1491%:%
%:%3512=1491%:%
%:%3513=1492%:%
%:%3514=1492%:%
%:%3515=1492%:%
%:%3516=1492%:%
%:%3517=1493%:%
%:%3518=1493%:%
%:%3519=1493%:%
%:%3520=1493%:%
%:%3521=1494%:%
%:%3522=1494%:%
%:%3523=1495%:%
%:%3524=1495%:%
%:%3525=1495%:%
%:%3526=1495%:%
%:%3527=1496%:%
%:%3528=1496%:%
%:%3529=1497%:%
%:%3530=1497%:%
%:%3531=1497%:%
%:%3532=1498%:%
%:%3533=1499%:%
%:%3534=1499%:%
%:%3535=1499%:%
%:%3536=1500%:%
%:%3537=1500%:%
%:%3538=1501%:%
%:%3539=1501%:%
%:%3540=1502%:%
%:%3541=1502%:%
%:%3542=1503%:%
%:%3543=1503%:%
%:%3544=1504%:%
%:%3545=1504%:%
%:%3546=1504%:%
%:%3547=1505%:%
%:%3548=1505%:%
%:%3549=1506%:%
%:%3550=1506%:%
%:%3551=1506%:%
%:%3552=1507%:%
%:%3553=1507%:%
%:%3554=1507%:%
%:%3555=1508%:%
%:%3556=1509%:%
%:%3557=1509%:%
%:%3558=1510%:%
%:%3559=1510%:%
%:%3560=1511%:%
%:%3561=1512%:%
%:%3562=1512%:%
%:%3563=1512%:%
%:%3564=1513%:%
%:%3565=1513%:%
%:%3566=1513%:%
%:%3567=1513%:%
%:%3568=1514%:%
%:%3569=1514%:%
%:%3570=1514%:%
%:%3571=1514%:%
%:%3572=1515%:%
%:%3573=1515%:%
%:%3574=1516%:%
%:%3575=1516%:%
%:%3576=1516%:%
%:%3577=1516%:%
%:%3578=1517%:%
%:%3579=1517%:%
%:%3580=1518%:%
%:%3581=1518%:%
%:%3582=1518%:%
%:%3583=1519%:%
%:%3584=1520%:%
%:%3585=1520%:%
%:%3586=1520%:%
%:%3587=1521%:%
%:%3588=1521%:%
%:%3589=1522%:%
%:%3590=1522%:%
%:%3591=1522%:%
%:%3592=1522%:%
%:%3593=1523%:%
%:%3594=1523%:%
%:%3595=1524%:%
%:%3596=1524%:%
%:%3597=1525%:%
%:%3598=1525%:%
%:%3601=1528%:%
%:%3602=1529%:%
%:%3603=1529%:%
%:%3604=1530%:%
%:%3605=1530%:%
%:%3606=1531%:%
%:%3607=1531%:%
%:%3610=1534%:%
%:%3611=1535%:%
%:%3612=1535%:%
%:%3613=1536%:%
%:%3623=1539%:%
%:%3625=1541%:%
%:%3626=1541%:%
%:%3627=1542%:%
%:%3629=1544%:%
%:%3636=1545%:%
%:%3637=1545%:%
%:%3638=1546%:%
%:%3639=1546%:%
%:%3640=1547%:%
%:%3641=1547%:%
%:%3642=1548%:%
%:%3643=1548%:%
%:%3644=1549%:%
%:%3645=1549%:%
%:%3646=1550%:%
%:%3647=1550%:%
%:%3649=1552%:%
%:%3650=1553%:%
%:%3651=1553%:%
%:%3652=1554%:%
%:%3653=1554%:%
%:%3654=1555%:%
%:%3655=1555%:%
%:%3656=1556%:%
%:%3657=1557%:%
%:%3658=1557%:%
%:%3659=1558%:%
%:%3660=1558%:%
%:%3661=1559%:%
%:%3662=1560%:%
%:%3663=1560%:%
%:%3664=1560%:%
%:%3665=1561%:%
%:%3666=1561%:%
%:%3667=1561%:%
%:%3668=1562%:%
%:%3669=1562%:%
%:%3670=1563%:%
%:%3671=1563%:%
%:%3672=1564%:%
%:%3673=1564%:%
%:%3674=1564%:%
%:%3675=1564%:%
%:%3676=1565%:%
%:%3677=1565%:%
%:%3678=1566%:%
%:%3679=1566%:%
%:%3680=1567%:%
%:%3681=1567%:%
%:%3682=1568%:%
%:%3683=1568%:%
%:%3684=1569%:%
%:%3685=1569%:%
%:%3686=1570%:%
%:%3687=1570%:%
%:%3689=1572%:%
%:%3690=1573%:%
%:%3691=1573%:%
%:%3694=1576%:%
%:%3695=1577%:%
%:%3696=1577%:%
%:%3697=1578%:%
%:%3698=1578%:%
%:%3699=1579%:%
%:%3700=1579%:%
%:%3703=1582%:%
%:%3704=1583%:%
%:%3705=1583%:%
%:%3706=1584%:%
%:%3707=1584%:%
%:%3708=1585%:%
%:%3709=1585%:%
%:%3711=1587%:%
%:%3712=1588%:%
%:%3713=1588%:%
%:%3714=1589%:%
%:%3715=1589%:%
%:%3718=1592%:%
%:%3719=1593%:%
%:%3720=1593%:%
%:%3721=1593%:%
%:%3722=1594%:%
%:%3723=1594%:%
%:%3724=1594%:%
%:%3725=1595%:%
%:%3726=1596%:%
%:%3727=1596%:%
%:%3728=1597%:%
%:%3729=1597%:%
%:%3730=1598%:%
%:%3731=1599%:%
%:%3732=1600%:%
%:%3733=1601%:%
%:%3734=1602%:%
%:%3735=1602%:%
%:%3736=1602%:%
%:%3739=1605%:%
%:%3740=1606%:%
%:%3741=1606%:%
%:%3742=1607%:%
%:%3743=1607%:%
%:%3744=1608%:%
%:%3745=1608%:%
%:%3746=1608%:%
%:%3749=1611%:%
%:%3750=1612%:%
%:%3751=1612%:%
%:%3752=1612%:%
%:%3753=1613%:%
%:%3754=1613%:%
%:%3755=1613%:%
%:%3758=1616%:%
%:%3759=1617%:%
%:%3760=1617%:%
%:%3761=1618%:%
%:%3762=1618%:%
%:%3763=1619%:%
%:%3764=1620%:%
%:%3765=1620%:%
%:%3766=1620%:%
%:%3770=1624%:%
%:%3771=1625%:%
%:%3772=1625%:%
%:%3773=1626%:%
%:%3774=1626%:%
%:%3775=1627%:%
%:%3776=1628%:%
%:%3777=1629%:%
%:%3778=1629%:%
%:%3779=1629%:%
%:%3782=1632%:%
%:%3783=1633%:%
%:%3784=1633%:%
%:%3785=1634%:%
%:%3786=1634%:%
%:%3787=1635%:%
%:%3788=1636%:%
%:%3789=1636%:%
%:%3790=1636%:%
%:%3793=1639%:%
%:%3794=1640%:%
%:%3795=1640%:%
%:%3796=1640%:%
%:%3797=1641%:%
%:%3798=1641%:%
%:%3799=1641%:%
%:%3801=1643%:%
%:%3802=1644%:%
%:%3803=1644%:%
%:%3804=1645%:%
%:%3805=1645%:%
%:%3806=1646%:%
%:%3807=1647%:%
%:%3808=1648%:%
%:%3809=1649%:%
%:%3810=1649%:%
%:%3811=1649%:%
%:%3813=1651%:%
%:%3814=1652%:%
%:%3815=1652%:%
%:%3816=1653%:%
%:%3817=1653%:%
%:%3818=1654%:%
%:%3819=1654%:%
%:%3820=1654%:%
%:%3822=1656%:%
%:%3823=1657%:%
%:%3824=1657%:%
%:%3825=1657%:%
%:%3826=1658%:%
%:%3827=1658%:%
%:%3828=1658%:%
%:%3830=1660%:%
%:%3831=1661%:%
%:%3832=1661%:%
%:%3833=1662%:%
%:%3834=1662%:%
%:%3835=1663%:%
%:%3836=1663%:%
%:%3837=1664%:%
%:%3838=1664%:%
%:%3839=1665%:%
%:%3840=1665%:%
%:%3841=1666%:%
%:%3842=1666%:%
%:%3843=1666%:%
%:%3844=1667%:%
%:%3845=1667%:%
%:%3846=1667%:%
%:%3847=1668%:%
%:%3848=1668%:%
%:%3849=1669%:%
%:%3850=1669%:%
%:%3851=1669%:%
%:%3852=1670%:%
%:%3853=1670%:%
%:%3854=1671%:%
%:%3855=1671%:%
%:%3856=1672%:%
%:%3857=1672%:%
%:%3858=1672%:%
%:%3859=1672%:%
%:%3860=1673%:%
%:%3861=1673%:%
%:%3862=1674%:%
%:%3863=1674%:%
%:%3864=1675%:%
%:%3865=1675%:%
%:%3866=1676%:%
%:%3872=1676%:%
%:%3875=1677%:%
%:%3876=1678%:%
%:%3877=1679%:%
%:%3878=1680%:%
%:%3879=1680%:%
%:%3880=1681%:%
%:%3882=1683%:%
%:%3889=1684%:%
%:%3890=1684%:%
%:%3891=1685%:%
%:%3892=1685%:%
%:%3893=1686%:%
%:%3894=1686%:%
%:%3895=1687%:%
%:%3896=1687%:%
%:%3897=1688%:%
%:%3898=1688%:%
%:%3899=1689%:%
%:%3900=1689%:%
%:%3902=1691%:%
%:%3903=1692%:%
%:%3904=1692%:%
%:%3905=1693%:%
%:%3906=1693%:%
%:%3907=1694%:%
%:%3908=1694%:%
%:%3909=1695%:%
%:%3910=1696%:%
%:%3911=1696%:%
%:%3912=1697%:%
%:%3913=1697%:%
%:%3914=1698%:%
%:%3915=1699%:%
%:%3916=1699%:%
%:%3917=1699%:%
%:%3918=1700%:%
%:%3919=1700%:%
%:%3920=1700%:%
%:%3921=1701%:%
%:%3922=1701%:%
%:%3923=1702%:%
%:%3924=1702%:%
%:%3925=1703%:%
%:%3926=1703%:%
%:%3927=1703%:%
%:%3928=1703%:%
%:%3929=1704%:%
%:%3930=1704%:%
%:%3931=1705%:%
%:%3932=1705%:%
%:%3933=1706%:%
%:%3934=1706%:%
%:%3935=1707%:%
%:%3936=1707%:%
%:%3937=1708%:%
%:%3938=1708%:%
%:%3939=1709%:%
%:%3940=1709%:%
%:%3942=1711%:%
%:%3943=1712%:%
%:%3944=1712%:%
%:%3947=1715%:%
%:%3948=1716%:%
%:%3949=1716%:%
%:%3950=1717%:%
%:%3951=1717%:%
%:%3952=1718%:%
%:%3953=1718%:%
%:%3956=1721%:%
%:%3957=1722%:%
%:%3958=1722%:%
%:%3959=1723%:%
%:%3960=1723%:%
%:%3961=1724%:%
%:%3962=1724%:%
%:%3964=1726%:%
%:%3965=1727%:%
%:%3966=1727%:%
%:%3967=1728%:%
%:%3968=1728%:%
%:%3969=1729%:%
%:%3970=1729%:%
%:%3971=1730%:%
%:%3972=1730%:%
%:%3973=1731%:%
%:%3974=1731%:%
%:%3975=1731%:%
%:%3976=1732%:%
%:%3977=1732%:%
%:%3978=1732%:%
%:%3979=1732%:%
%:%3980=1732%:%
%:%3981=1733%:%
%:%3982=1733%:%
%:%3983=1733%:%
%:%3984=1733%:%
%:%3985=1733%:%
%:%3986=1734%:%
%:%3987=1734%:%
%:%3988=1734%:%
%:%3989=1734%:%
%:%3990=1735%:%
%:%3991=1735%:%
%:%3992=1736%:%
%:%3993=1737%:%
%:%3994=1737%:%
%:%3995=1738%:%
%:%3996=1738%:%
%:%3997=1738%:%
%:%3998=1738%:%
%:%3999=1739%:%
%:%4000=1739%:%
%:%4001=1739%:%
%:%4002=1739%:%
%:%4003=1740%:%
%:%4004=1740%:%
%:%4007=1743%:%
%:%4008=1744%:%
%:%4009=1744%:%
%:%4010=1744%:%
%:%4011=1745%:%
%:%4012=1745%:%
%:%4013=1745%:%
%:%4014=1746%:%
%:%4015=1747%:%
%:%4016=1747%:%
%:%4017=1748%:%
%:%4018=1748%:%
%:%4019=1749%:%
%:%4020=1749%:%
%:%4021=1750%:%
%:%4022=1750%:%
%:%4023=1751%:%
%:%4024=1752%:%
%:%4025=1752%:%
%:%4026=1753%:%
%:%4027=1754%:%
%:%4028=1754%:%
%:%4029=1755%:%
%:%4030=1756%:%
%:%4031=1756%:%
%:%4032=1757%:%
%:%4033=1758%:%
%:%4034=1758%:%
%:%4035=1759%:%
%:%4036=1759%:%
%:%4037=1759%:%
%:%4040=1762%:%
%:%4041=1763%:%
%:%4042=1763%:%
%:%4043=1764%:%
%:%4044=1764%:%
%:%4045=1765%:%
%:%4046=1765%:%
%:%4047=1765%:%
%:%4050=1768%:%
%:%4051=1769%:%
%:%4052=1769%:%
%:%4053=1769%:%
%:%4054=1770%:%
%:%4055=1770%:%
%:%4056=1770%:%
%:%4059=1773%:%
%:%4060=1774%:%
%:%4061=1774%:%
%:%4062=1775%:%
%:%4063=1775%:%
%:%4064=1776%:%
%:%4065=1777%:%
%:%4066=1778%:%
%:%4067=1779%:%
%:%4068=1779%:%
%:%4069=1779%:%
%:%4072=1782%:%
%:%4073=1783%:%
%:%4074=1783%:%
%:%4075=1783%:%
%:%4076=1784%:%
%:%4077=1784%:%
%:%4078=1784%:%
%:%4080=1786%:%
%:%4081=1787%:%
%:%4082=1787%:%
%:%4083=1788%:%
%:%4084=1788%:%
%:%4085=1789%:%
%:%4086=1790%:%
%:%4087=1790%:%
%:%4088=1790%:%
%:%4090=1792%:%
%:%4091=1793%:%
%:%4092=1793%:%
%:%4093=1793%:%
%:%4094=1794%:%
%:%4095=1794%:%
%:%4096=1794%:%
%:%4097=1795%:%
%:%4098=1796%:%
%:%4099=1796%:%
%:%4100=1797%:%
%:%4101=1797%:%
%:%4102=1798%:%
%:%4103=1799%:%
%:%4104=1800%:%
%:%4105=1801%:%
%:%4106=1801%:%
%:%4107=1801%:%
%:%4108=1802%:%
%:%4109=1803%:%
%:%4110=1803%:%
%:%4111=1804%:%
%:%4112=1804%:%
%:%4113=1805%:%
%:%4114=1805%:%
%:%4115=1805%:%
%:%4116=1806%:%
%:%4117=1807%:%
%:%4118=1807%:%
%:%4119=1807%:%
%:%4120=1808%:%
%:%4121=1808%:%
%:%4122=1808%:%
%:%4123=1809%:%
%:%4124=1809%:%
%:%4125=1810%:%
%:%4126=1810%:%
%:%4127=1811%:%
%:%4128=1811%:%
%:%4129=1811%:%
%:%4130=1811%:%
%:%4131=1812%:%
%:%4132=1812%:%
%:%4133=1813%:%
%:%4134=1813%:%
%:%4135=1814%:%
%:%4136=1814%:%
%:%4137=1815%:%
%:%4147=1818%:%
%:%4149=1820%:%
%:%4150=1820%:%
%:%4151=1821%:%
%:%4152=1822%:%
%:%4156=1826%:%
%:%4163=1827%:%
%:%4164=1827%:%
%:%4165=1828%:%
%:%4166=1828%:%
%:%4167=1829%:%
%:%4168=1829%:%
%:%4169=1829%:%
%:%4170=1830%:%
%:%4171=1830%:%
%:%4172=1831%:%
%:%4173=1831%:%
%:%4174=1832%:%
%:%4175=1832%:%
%:%4179=1836%:%
%:%4180=1837%:%
%:%4181=1837%:%
%:%4182=1838%:%
%:%4183=1838%:%
%:%4188=1843%:%
%:%4189=1844%:%
%:%4190=1844%:%
%:%4191=1844%:%
%:%4192=1845%:%
%:%4193=1845%:%
%:%4194=1845%:%
%:%4196=1847%:%
%:%4197=1848%:%
%:%4198=1848%:%
%:%4199=1849%:%
%:%4200=1849%:%
%:%4201=1850%:%
%:%4202=1851%:%
%:%4203=1852%:%
%:%4204=1852%:%
%:%4205=1852%:%
%:%4206=1853%:%
%:%4207=1854%:%
%:%4208=1854%:%
%:%4209=1855%:%
%:%4210=1855%:%
%:%4211=1855%:%
%:%4212=1856%:%
%:%4213=1856%:%
%:%4214=1857%:%
%:%4215=1857%:%
%:%4216=1858%:%
%:%4217=1858%:%
%:%4218=1858%:%
%:%4219=1859%:%
%:%4220=1859%:%
%:%4225=1864%:%
%:%4226=1865%:%
%:%4227=1865%:%
%:%4228=1865%:%
%:%4229=1866%:%
%:%4230=1866%:%
%:%4231=1866%:%
%:%4232=1867%:%
%:%4233=1868%:%
%:%4234=1868%:%
%:%4235=1868%:%
%:%4236=1869%:%
%:%4237=1869%:%
%:%4238=1869%:%
%:%4240=1871%:%
%:%4241=1872%:%
%:%4242=1872%:%
%:%4243=1873%:%
%:%4244=1873%:%
%:%4245=1874%:%
%:%4246=1875%:%
%:%4247=1876%:%
%:%4248=1877%:%
%:%4249=1878%:%
%:%4250=1878%:%
%:%4251=1878%:%
%:%4252=1878%:%
%:%4253=1879%:%
%:%4254=1879%:%
%:%4255=1880%:%
%:%4256=1880%:%
%:%4257=1881%:%
%:%4258=1881%:%
%:%4263=1886%:%
%:%4264=1887%:%
%:%4265=1887%:%
%:%4266=1887%:%
%:%4267=1888%:%
%:%4268=1888%:%
%:%4269=1888%:%
%:%4271=1890%:%
%:%4272=1891%:%
%:%4273=1891%:%
%:%4274=1891%:%
%:%4275=1892%:%
%:%4276=1892%:%
%:%4277=1892%:%
%:%4278=1893%:%
%:%4279=1893%:%
%:%4280=1893%:%
%:%4281=1894%:%
%:%4282=1894%:%
%:%4283=1894%:%
%:%4284=1894%:%
%:%4285=1895%:%
%:%4286=1895%:%
%:%4287=1896%:%
%:%4288=1896%:%
%:%4289=1896%:%
%:%4290=1897%:%
%:%4291=1897%:%
%:%4292=1898%:%
%:%4293=1899%:%
%:%4294=1900%:%
%:%4295=1900%:%
%:%4296=1901%:%
%:%4297=1901%:%
%:%4298=1902%:%
%:%4299=1902%:%
%:%4300=1903%:%
%:%4301=1903%:%
%:%4302=1904%:%
%:%4303=1904%:%
%:%4304=1904%:%
%:%4305=1905%:%
%:%4306=1905%:%
%:%4307=1906%:%
%:%4308=1906%:%
%:%4309=1907%:%
%:%4310=1907%:%
%:%4311=1908%:%
%:%4312=1908%:%
%:%4313=1909%:%
%:%4314=1909%:%
%:%4315=1910%:%
%:%4316=1910%:%
%:%4317=1910%:%
%:%4318=1911%:%
%:%4319=1911%:%
%:%4320=1912%:%
%:%4321=1912%:%
%:%4327=1918%:%
%:%4328=1919%:%
%:%4329=1919%:%
%:%4330=1919%:%
%:%4331=1920%:%
%:%4332=1920%:%
%:%4333=1920%:%
%:%4336=1923%:%
%:%4337=1924%:%
%:%4338=1924%:%
%:%4339=1925%:%
%:%4340=1925%:%
%:%4341=1926%:%
%:%4342=1927%:%
%:%4343=1927%:%
%:%4344=1928%:%
%:%4345=1928%:%
%:%4346=1929%:%
%:%4347=1930%:%
%:%4348=1931%:%
%:%4349=1931%:%
%:%4350=1932%:%
%:%4351=1932%:%
%:%4352=1933%:%
%:%4353=1933%:%
%:%4354=1934%:%
%:%4355=1934%:%
%:%4357=1936%:%
%:%4358=1937%:%
%:%4359=1937%:%
%:%4360=1938%:%
%:%4361=1938%:%
%:%4362=1939%:%
%:%4363=1939%:%
%:%4364=1940%:%
%:%4365=1940%:%
%:%4366=1940%:%
%:%4369=1943%:%
%:%4370=1944%:%
%:%4371=1944%:%
%:%4372=1944%:%
%:%4373=1945%:%
%:%4374=1945%:%
%:%4375=1945%:%
%:%4378=1948%:%
%:%4379=1949%:%
%:%4380=1949%:%
%:%4381=1950%:%
%:%4382=1950%:%
%:%4383=1951%:%
%:%4384=1952%:%
%:%4385=1953%:%
%:%4386=1954%:%
%:%4387=1955%:%
%:%4388=1955%:%
%:%4389=1955%:%
%:%4391=1957%:%
%:%4392=1958%:%
%:%4393=1958%:%
%:%4394=1959%:%
%:%4395=1959%:%
%:%4396=1960%:%
%:%4397=1961%:%
%:%4398=1961%:%
%:%4399=1961%:%
%:%4401=1963%:%
%:%4402=1964%:%
%:%4403=1964%:%
%:%4404=1964%:%
%:%4405=1965%:%
%:%4406=1965%:%
%:%4407=1965%:%
%:%4409=1967%:%
%:%4410=1968%:%
%:%4411=1968%:%
%:%4412=1969%:%
%:%4413=1969%:%
%:%4414=1970%:%
%:%4415=1970%:%
%:%4416=1971%:%
%:%4417=1971%:%
%:%4418=1972%:%
%:%4419=1972%:%
%:%4420=1973%:%
%:%4421=1973%:%
%:%4422=1973%:%
%:%4423=1974%:%
%:%4424=1974%:%
%:%4426=1976%:%
%:%4427=1977%:%
%:%4428=1977%:%
%:%4429=1978%:%
%:%4430=1978%:%
%:%4431=1979%:%
%:%4432=1979%:%
%:%4433=1980%:%
%:%4434=1980%:%
%:%4435=1980%:%
%:%4437=1982%:%
%:%4438=1983%:%
%:%4439=1983%:%
%:%4440=1984%:%
%:%4441=1984%:%
%:%4442=1985%:%
%:%4443=1986%:%
%:%4444=1987%:%
%:%4445=1988%:%
%:%4446=1989%:%
%:%4447=1989%:%
%:%4448=1989%:%
%:%4450=1991%:%
%:%4451=1992%:%
%:%4452=1992%:%
%:%4453=1993%:%
%:%4454=1993%:%
%:%4455=1993%:%
%:%4456=1994%:%
%:%4457=1994%:%
%:%4458=1995%:%
%:%4459=1995%:%
%:%4460=1996%:%
%:%4461=1996%:%
%:%4462=1996%:%
%:%4463=1997%:%
%:%4464=1997%:%
%:%4469=2002%:%
%:%4470=2003%:%
%:%4471=2003%:%
%:%4472=2003%:%
%:%4473=2004%:%
%:%4474=2004%:%
%:%4475=2004%:%
%:%4476=2005%:%
%:%4477=2006%:%
%:%4478=2006%:%
%:%4479=2006%:%
%:%4480=2007%:%
%:%4481=2007%:%
%:%4482=2007%:%
%:%4483=2008%:%
%:%4484=2009%:%
%:%4485=2009%:%
%:%4486=2010%:%
%:%4487=2010%:%
%:%4488=2011%:%
%:%4489=2012%:%
%:%4490=2013%:%
%:%4491=2013%:%
%:%4492=2013%:%
%:%4498=2019%:%
%:%4499=2020%:%
%:%4500=2020%:%
%:%4501=2021%:%
%:%4502=2021%:%
%:%4503=2022%:%
%:%4504=2022%:%
%:%4505=2023%:%
%:%4506=2023%:%
%:%4511=2028%:%
%:%4512=2029%:%
%:%4513=2029%:%
%:%4514=2029%:%
%:%4515=2030%:%
%:%4516=2030%:%
%:%4517=2030%:%
%:%4519=2032%:%
%:%4520=2033%:%
%:%4521=2033%:%
%:%4522=2033%:%
%:%4523=2034%:%
%:%4524=2034%:%
%:%4525=2034%:%
%:%4526=2035%:%
%:%4527=2035%:%
%:%4528=2036%:%
%:%4529=2036%:%
%:%4530=2037%:%
%:%4531=2037%:%
%:%4532=2037%:%
%:%4538=2043%:%
%:%4539=2044%:%
%:%4540=2044%:%
%:%4541=2045%:%
%:%4542=2045%:%
%:%4543=2046%:%
%:%4544=2046%:%
%:%4545=2046%:%
%:%4548=2049%:%
%:%4549=2050%:%
%:%4550=2050%:%
%:%4551=2050%:%
%:%4552=2051%:%
%:%4553=2051%:%
%:%4554=2051%:%
%:%4556=2053%:%
%:%4557=2054%:%
%:%4558=2054%:%
%:%4559=2055%:%
%:%4560=2055%:%
%:%4561=2056%:%
%:%4562=2057%:%
%:%4563=2058%:%
%:%4564=2059%:%
%:%4565=2060%:%
%:%4566=2060%:%
%:%4567=2060%:%
%:%4568=2061%:%
%:%4569=2062%:%
%:%4570=2062%:%
%:%4571=2063%:%
%:%4572=2063%:%
%:%4573=2064%:%
%:%4574=2065%:%
%:%4575=2065%:%
%:%4576=2065%:%
%:%4581=2070%:%
%:%4582=2071%:%
%:%4583=2071%:%
%:%4584=2071%:%
%:%4585=2072%:%
%:%4586=2072%:%
%:%4587=2073%:%
%:%4597=2076%:%
%:%4599=2078%:%
%:%4600=2078%:%
%:%4601=2079%:%
%:%4604=2082%:%
%:%4611=2083%:%
%:%4612=2083%:%
%:%4613=2084%:%
%:%4614=2084%:%
%:%4615=2085%:%
%:%4616=2085%:%
%:%4617=2085%:%
%:%4618=2086%:%
%:%4619=2086%:%
%:%4620=2087%:%
%:%4621=2087%:%
%:%4622=2088%:%
%:%4623=2088%:%
%:%4628=2093%:%
%:%4629=2094%:%
%:%4630=2094%:%
%:%4631=2095%:%
%:%4632=2095%:%
%:%4633=2096%:%
%:%4634=2096%:%
%:%4635=2096%:%
%:%4636=2097%:%
%:%4637=2097%:%
%:%4642=2102%:%
%:%4643=2103%:%
%:%4644=2103%:%
%:%4645=2103%:%
%:%4646=2104%:%
%:%4647=2104%:%
%:%4648=2104%:%
%:%4650=2106%:%
%:%4651=2107%:%
%:%4652=2107%:%
%:%4653=2107%:%
%:%4654=2108%:%
%:%4655=2108%:%
%:%4656=2108%:%
%:%4658=2110%:%
%:%4659=2111%:%
%:%4660=2111%:%
%:%4661=2112%:%
%:%4662=2112%:%
%:%4663=2112%:%
%:%4667=2116%:%
%:%4668=2117%:%
%:%4669=2117%:%
%:%4670=2118%:%
%:%4671=2118%:%
%:%4672=2119%:%
%:%4673=2119%:%
%:%4674=2120%:%
%:%4675=2120%:%
%:%4676=2121%:%
%:%4677=2121%:%
%:%4678=2122%:%
%:%4679=2122%:%
%:%4680=2123%:%
%:%4681=2123%:%
%:%4685=2127%:%
%:%4686=2128%:%
%:%4687=2128%:%
%:%4694=2135%:%
%:%4695=2136%:%
%:%4696=2136%:%
%:%4697=2137%:%
%:%4698=2137%:%
%:%4699=2138%:%
%:%4700=2138%:%
%:%4705=2143%:%
%:%4706=2144%:%
%:%4707=2144%:%
%:%4708=2145%:%
%:%4709=2145%:%
%:%4710=2146%:%
%:%4711=2146%:%
%:%4715=2150%:%
%:%4716=2151%:%
%:%4717=2151%:%
%:%4718=2152%:%
%:%4719=2152%:%
%:%4720=2153%:%
%:%4721=2153%:%
%:%4722=2154%:%
%:%4723=2154%:%
%:%4724=2155%:%
%:%4725=2155%:%
%:%4726=2156%:%
%:%4727=2156%:%
%:%4728=2156%:%
%:%4729=2157%:%
%:%4730=2157%:%
%:%4731=2158%:%
%:%4732=2158%:%
%:%4737=2163%:%
%:%4738=2164%:%
%:%4739=2164%:%
%:%4740=2164%:%
%:%4741=2165%:%
%:%4742=2165%:%
%:%4743=2165%:%
%:%4745=2167%:%
%:%4746=2168%:%
%:%4747=2168%:%
%:%4748=2169%:%
%:%4749=2169%:%
%:%4750=2170%:%
%:%4751=2170%:%
%:%4752=2170%:%
%:%4754=2172%:%
%:%4755=2173%:%
%:%4756=2173%:%
%:%4757=2174%:%
%:%4758=2174%:%
%:%4759=2175%:%
%:%4760=2175%:%
%:%4761=2175%:%
%:%4762=2176%:%
%:%4763=2176%:%
%:%4764=2177%:%
%:%4765=2177%:%
%:%4766=2178%:%
%:%4767=2178%:%
%:%4768=2179%:%
%:%4769=2180%:%
%:%4770=2180%:%
%:%4771=2181%:%
%:%4772=2182%:%
%:%4773=2182%:%
%:%4774=2183%:%
%:%4775=2183%:%
%:%4776=2184%:%
%:%4777=2184%:%
%:%4778=2185%:%
%:%4779=2185%:%
%:%4780=2185%:%
%:%4782=2187%:%
%:%4783=2188%:%
%:%4784=2188%:%
%:%4785=2189%:%
%:%4786=2189%:%
%:%4787=2190%:%
%:%4788=2190%:%
%:%4789=2190%:%
%:%4791=2192%:%
%:%4792=2193%:%
%:%4793=2193%:%
%:%4794=2194%:%
%:%4795=2195%:%
%:%4796=2195%:%
%:%4797=2196%:%
%:%4798=2197%:%
%:%4799=2198%:%
%:%4800=2199%:%
%:%4801=2199%:%
%:%4802=2199%:%
%:%4805=2202%:%
%:%4806=2203%:%
%:%4807=2203%:%
%:%4808=2204%:%
%:%4809=2204%:%
%:%4810=2205%:%
%:%4811=2206%:%
%:%4812=2206%:%
%:%4813=2206%:%
%:%4816=2209%:%
%:%4817=2210%:%
%:%4818=2210%:%
%:%4819=2211%:%
%:%4820=2211%:%
%:%4821=2212%:%
%:%4822=2213%:%
%:%4823=2213%:%
%:%4824=2213%:%
%:%4825=2214%:%
%:%4826=2215%:%
%:%4827=2215%:%
%:%4828=2215%:%
%:%4829=2216%:%
%:%4830=2216%:%
%:%4831=2216%:%
%:%4832=2217%:%
%:%4833=2218%:%
%:%4834=2218%:%
%:%4835=2219%:%
%:%4836=2219%:%
%:%4837=2220%:%
%:%4838=2220%:%
%:%4839=2220%:%
%:%4840=2220%:%
%:%4841=2220%:%
%:%4842=2221%:%
%:%4843=2221%:%
%:%4844=2222%:%
%:%4845=2222%:%
%:%4846=2223%:%
%:%4847=2223%:%
%:%4848=2224%:%
%:%4854=2224%:%
%:%4857=2225%:%
%:%4858=2226%:%
%:%4859=2227%:%
%:%4860=2227%:%
%:%4861=2228%:%
%:%4862=2229%:%
%:%4865=2232%:%
%:%4872=2233%:%
%:%4873=2233%:%
%:%4874=2234%:%
%:%4875=2234%:%
%:%4880=2239%:%
%:%4881=2240%:%
%:%4882=2240%:%
%:%4883=2241%:%
%:%4884=2241%:%
%:%4885=2242%:%
%:%4886=2242%:%
%:%4887=2242%:%
%:%4889=2244%:%
%:%4890=2245%:%
%:%4891=2245%:%
%:%4892=2246%:%
%:%4893=2246%:%
%:%4894=2247%:%
%:%4895=2248%:%
%:%4896=2248%:%
%:%4897=2248%:%
%:%4898=2249%:%
%:%4899=2250%:%
%:%4900=2250%:%
%:%4901=2250%:%
%:%4902=2251%:%
%:%4903=2251%:%
%:%4904=2251%:%
%:%4905=2251%:%
%:%4906=2252%:%
%:%4916=2255%:%
%:%4918=2257%:%
%:%4919=2257%:%
%:%4920=2258%:%
%:%4921=2259%:%
%:%4928=2260%:%
%:%4929=2260%:%
%:%4930=2261%:%
%:%4931=2261%:%
%:%4932=2262%:%
%:%4933=2262%:%
%:%4934=2263%:%
%:%4935=2263%:%
%:%4936=2264%:%
%:%4937=2264%:%
%:%4938=2264%:%
%:%4939=2265%:%
%:%4940=2265%:%
%:%4941=2265%:%
%:%4942=2265%:%
%:%4943=2266%:%
%:%4944=2266%:%
%:%4945=2267%:%
%:%4946=2267%:%
%:%4947=2268%:%
%:%4948=2268%:%
%:%4949=2269%:%
%:%4950=2269%:%
%:%4951=2270%:%
%:%4952=2270%:%
%:%4953=2270%:%
%:%4954=2270%:%
%:%4955=2271%:%
%:%4956=2271%:%
%:%4957=2272%:%
%:%4958=2272%:%
%:%4959=2273%:%
%:%4960=2273%:%
%:%4961=2273%:%
%:%4962=2274%:%
%:%4963=2274%:%
%:%4964=2275%:%
%:%4965=2275%:%
%:%4966=2275%:%
%:%4967=2275%:%
%:%4968=2275%:%
%:%4969=2276%:%
%:%4970=2276%:%
%:%4971=2276%:%
%:%4972=2277%:%
%:%4973=2277%:%
%:%4974=2278%:%
%:%4975=2278%:%
%:%4976=2278%:%
%:%4977=2279%:%
%:%4978=2279%:%
%:%4979=2279%:%
%:%4980=2280%:%
%:%4981=2281%:%
%:%4982=2281%:%
%:%4983=2281%:%
%:%4984=2281%:%
%:%4985=2282%:%
%:%4986=2282%:%
%:%4987=2282%:%
%:%4988=2283%:%
%:%4989=2283%:%
%:%4990=2283%:%
%:%4991=2284%:%
%:%4992=2285%:%
%:%4993=2285%:%
%:%4994=2285%:%
%:%4995=2285%:%
%:%4996=2285%:%
%:%4997=2286%:%
%:%4998=2286%:%
%:%4999=2286%:%
%:%5000=2286%:%
%:%5001=2287%:%
%:%5002=2287%:%
%:%5003=2287%:%
%:%5004=2287%:%
%:%5005=2287%:%
%:%5006=2288%:%
%:%5012=2288%:%
%:%5015=2289%:%
%:%5016=2290%:%
%:%5017=2291%:%
%:%5018=2292%:%
%:%5019=2292%:%
%:%5020=2293%:%
%:%5027=2294%:%
%:%5028=2294%:%
%:%5029=2295%:%
%:%5030=2295%:%
%:%5031=2296%:%
%:%5032=2296%:%
%:%5033=2296%:%
%:%5034=2297%:%
%:%5035=2297%:%
%:%5036=2298%:%
%:%5037=2298%:%
%:%5038=2299%:%
%:%5039=2299%:%
%:%5040=2300%:%
%:%5041=2300%:%
%:%5042=2301%:%
%:%5043=2301%:%
%:%5044=2302%:%
%:%5045=2302%:%
%:%5046=2303%:%
%:%5047=2303%:%
%:%5048=2304%:%
%:%5049=2304%:%
%:%5050=2305%:%
%:%5051=2305%:%
%:%5052=2305%:%
%:%5053=2305%:%
%:%5054=2306%:%
%:%5055=2306%:%
%:%5056=2306%:%
%:%5057=2306%:%
%:%5058=2307%:%
%:%5059=2307%:%
%:%5060=2307%:%
%:%5061=2307%:%
%:%5062=2308%:%
%:%5063=2308%:%
%:%5064=2308%:%
%:%5065=2309%:%
%:%5066=2309%:%
%:%5067=2309%:%
%:%5068=2309%:%
%:%5069=2309%:%
%:%5070=2310%:%
%:%5071=2310%:%
%:%5072=2310%:%
%:%5073=2310%:%
%:%5074=2310%:%
%:%5075=2311%:%
%:%5076=2311%:%
%:%5077=2311%:%
%:%5078=2311%:%
%:%5079=2311%:%
%:%5080=2312%:%
%:%5081=2312%:%
%:%5082=2312%:%
%:%5083=2312%:%
%:%5084=2313%:%
%:%5085=2313%:%
%:%5086=2314%:%
%:%5087=2314%:%
%:%5088=2315%:%
%:%5094=2315%:%
%:%5097=2316%:%
%:%5098=2317%:%
%:%5099=2318%:%
%:%5100=2318%:%
%:%5101=2319%:%
%:%5108=2320%:%
%:%5109=2320%:%
%:%5110=2321%:%
%:%5111=2321%:%
%:%5112=2321%:%
%:%5113=2322%:%
%:%5114=2322%:%
%:%5115=2323%:%
%:%5116=2323%:%
%:%5117=2324%:%
%:%5118=2324%:%
%:%5119=2325%:%
%:%5120=2325%:%
%:%5121=2325%:%
%:%5122=2325%:%
%:%5123=2326%:%
%:%5124=2326%:%
%:%5125=2326%:%
%:%5126=2326%:%
%:%5127=2326%:%
%:%5128=2327%:%
%:%5129=2327%:%
%:%5130=2327%:%
%:%5131=2327%:%
%:%5132=2327%:%
%:%5133=2328%:%
%:%5134=2328%:%
%:%5135=2328%:%
%:%5136=2328%:%
%:%5137=2329%:%
%:%5138=2329%:%
%:%5139=2329%:%
%:%5140=2329%:%
%:%5141=2329%:%
%:%5142=2330%:%
%:%5143=2330%:%
%:%5144=2330%:%
%:%5145=2330%:%
%:%5146=2330%:%
%:%5147=2331%:%
%:%5148=2331%:%
%:%5149=2331%:%
%:%5150=2331%:%
%:%5151=2331%:%
%:%5152=2332%:%
%:%5153=2332%:%
%:%5154=2332%:%
%:%5155=2332%:%
%:%5156=2333%:%
%:%5162=2333%:%
%:%5165=2334%:%
%:%5166=2335%:%
%:%5167=2336%:%
%:%5168=2336%:%
%:%5169=2337%:%
%:%5171=2339%:%
%:%5178=2340%:%
%:%5179=2340%:%
%:%5180=2341%:%
%:%5181=2341%:%
%:%5182=2342%:%
%:%5183=2343%:%
%:%5184=2343%:%
%:%5185=2344%:%
%:%5186=2344%:%
%:%5187=2345%:%
%:%5188=2345%:%
%:%5189=2346%:%
%:%5190=2346%:%
%:%5191=2347%:%
%:%5192=2347%:%
%:%5193=2348%:%
%:%5194=2348%:%
%:%5195=2349%:%
%:%5196=2349%:%
%:%5197=2349%:%
%:%5198=2349%:%
%:%5199=2350%:%
%:%5200=2350%:%
%:%5201=2350%:%
%:%5202=2350%:%
%:%5203=2351%:%
%:%5204=2351%:%
%:%5205=2351%:%
%:%5206=2351%:%
%:%5207=2352%:%
%:%5208=2352%:%
%:%5209=2352%:%
%:%5210=2353%:%
%:%5211=2353%:%
%:%5212=2353%:%
%:%5213=2354%:%
%:%5214=2354%:%
%:%5215=2355%:%
%:%5216=2355%:%
%:%5217=2356%:%
%:%5218=2356%:%
%:%5219=2357%:%
%:%5220=2358%:%
%:%5221=2358%:%
%:%5222=2359%:%
%:%5223=2359%:%
%:%5224=2360%:%
%:%5225=2360%:%
%:%5226=2361%:%
%:%5227=2361%:%
%:%5228=2361%:%
%:%5229=2362%:%
%:%5230=2362%:%
%:%5231=2363%:%
%:%5232=2363%:%
%:%5233=2364%:%
%:%5234=2364%:%
%:%5235=2364%:%
%:%5236=2364%:%
%:%5237=2365%:%
%:%5247=2368%:%
%:%5249=2370%:%
%:%5250=2370%:%
%:%5251=2371%:%
%:%5258=2372%:%
%:%5259=2372%:%
%:%5260=2373%:%
%:%5261=2373%:%
%:%5262=2374%:%
%:%5263=2374%:%
%:%5264=2375%:%
%:%5265=2375%:%
%:%5266=2376%:%
%:%5267=2376%:%
%:%5268=2377%:%
%:%5269=2377%:%
%:%5270=2378%:%
%:%5271=2378%:%
%:%5272=2379%:%
%:%5273=2379%:%
%:%5274=2380%:%
%:%5275=2380%:%
%:%5276=2380%:%
%:%5277=2381%:%
%:%5278=2381%:%
%:%5279=2381%:%
%:%5280=2381%:%
%:%5281=2382%:%
%:%5282=2382%:%
%:%5283=2382%:%
%:%5284=2382%:%
%:%5285=2383%:%
%:%5286=2383%:%
%:%5287=2384%:%
%:%5288=2384%:%
%:%5289=2384%:%
%:%5290=2384%:%
%:%5291=2384%:%
%:%5292=2385%:%
%:%5293=2385%:%
%:%5294=2385%:%
%:%5295=2386%:%
%:%5296=2386%:%
%:%5297=2386%:%
%:%5298=2387%:%
%:%5299=2387%:%
%:%5300=2387%:%
%:%5301=2387%:%
%:%5302=2388%:%
%:%5303=2388%:%
%:%5304=2389%:%
%:%5305=2389%:%
%:%5306=2390%:%
%:%5307=2390%:%
%:%5308=2391%:%
%:%5309=2391%:%
%:%5310=2392%:%
%:%5311=2392%:%
%:%5312=2393%:%
%:%5313=2393%:%
%:%5314=2394%:%
%:%5315=2394%:%
%:%5316=2395%:%
%:%5317=2395%:%
%:%5321=2399%:%
%:%5322=2400%:%
%:%5323=2400%:%
%:%5324=2401%:%
%:%5325=2401%:%
%:%5326=2402%:%
%:%5327=2402%:%
%:%5328=2402%:%
%:%5329=2403%:%
%:%5330=2404%:%
%:%5331=2404%:%
%:%5332=2405%:%
%:%5333=2405%:%
%:%5334=2406%:%
%:%5335=2406%:%
%:%5336=2406%:%
%:%5337=2407%:%
%:%5338=2407%:%
%:%5339=2407%:%
%:%5340=2408%:%
%:%5341=2408%:%
%:%5342=2409%:%
%:%5343=2409%:%
%:%5344=2409%:%
%:%5345=2409%:%
%:%5346=2410%:%
%:%5347=2410%:%
%:%5348=2411%:%
%:%5349=2411%:%
%:%5350=2412%:%
%:%5351=2412%:%
%:%5352=2412%:%
%:%5353=2413%:%
%:%5354=2413%:%
%:%5355=2413%:%
%:%5356=2413%:%
%:%5357=2414%:%
%:%5358=2414%:%
%:%5359=2414%:%
%:%5360=2415%:%
%:%5361=2415%:%
%:%5362=2415%:%
%:%5363=2416%:%
%:%5364=2416%:%
%:%5365=2416%:%
%:%5366=2417%:%
%:%5367=2417%:%
%:%5368=2418%:%
%:%5369=2418%:%
%:%5370=2419%:%
%:%5371=2419%:%
%:%5372=2420%:%
%:%5373=2421%:%
%:%5374=2421%:%
%:%5375=2422%:%
%:%5376=2422%:%
%:%5377=2423%:%
%:%5378=2423%:%
%:%5379=2424%:%
%:%5380=2424%:%
%:%5381=2424%:%
%:%5382=2424%:%
%:%5383=2425%:%
%:%5384=2425%:%
%:%5385=2425%:%
%:%5386=2425%:%
%:%5387=2425%:%
%:%5388=2426%:%
%:%5389=2426%:%
%:%5390=2426%:%
%:%5391=2427%:%
%:%5392=2427%:%
%:%5393=2427%:%
%:%5394=2428%:%
%:%5395=2428%:%
%:%5396=2428%:%
%:%5397=2429%:%
%:%5398=2429%:%
%:%5399=2429%:%
%:%5400=2430%:%
%:%5401=2430%:%
%:%5402=2430%:%
%:%5403=2430%:%
%:%5404=2430%:%
%:%5405=2431%:%
%:%5406=2431%:%
%:%5407=2432%:%
%:%5408=2432%:%
%:%5409=2433%:%
%:%5410=2434%:%
%:%5411=2434%:%
%:%5412=2434%:%
%:%5413=2434%:%
%:%5414=2435%:%
%:%5415=2435%:%
%:%5416=2436%:%
%:%5417=2436%:%
%:%5418=2437%:%
%:%5419=2438%:%
%:%5420=2438%:%
%:%5421=2438%:%
%:%5422=2438%:%
%:%5423=2439%:%
%:%5424=2439%:%
%:%5425=2440%:%
%:%5426=2440%:%
%:%5427=2440%:%
%:%5428=2441%:%
%:%5429=2441%:%
%:%5430=2442%:%
%:%5431=2442%:%
%:%5432=2443%:%
%:%5433=2443%:%
%:%5434=2444%:%
%:%5435=2444%:%
%:%5436=2445%:%
%:%5437=2445%:%
%:%5438=2446%:%
%:%5439=2446%:%
%:%5440=2447%:%
%:%5441=2447%:%
%:%5442=2447%:%
%:%5443=2448%:%
%:%5444=2448%:%
%:%5445=2449%:%
%:%5446=2449%:%
%:%5447=2449%:%
%:%5448=2450%:%
%:%5449=2450%:%
%:%5450=2451%:%
%:%5451=2451%:%
%:%5452=2452%:%
%:%5453=2452%:%
%:%5454=2453%:%
%:%5455=2453%:%
%:%5456=2453%:%
%:%5457=2453%:%
%:%5458=2454%:%
%:%5459=2454%:%
%:%5460=2455%:%
%:%5461=2455%:%
%:%5462=2456%:%
%:%5463=2456%:%
%:%5464=2456%:%
%:%5465=2456%:%
%:%5466=2457%:%
%:%5467=2457%:%
%:%5468=2458%:%
%:%5469=2458%:%
%:%5470=2458%:%
%:%5471=2459%:%
%:%5472=2459%:%
%:%5473=2460%:%
%:%5474=2460%:%
%:%5475=2460%:%
%:%5476=2460%:%
%:%5477=2460%:%
%:%5478=2461%:%
%:%5479=2461%:%
%:%5480=2461%:%
%:%5481=2462%:%
%:%5482=2462%:%
%:%5483=2462%:%
%:%5484=2463%:%
%:%5485=2463%:%
%:%5486=2463%:%
%:%5487=2464%:%
%:%5488=2464%:%
%:%5489=2464%:%
%:%5490=2465%:%
%:%5491=2465%:%
%:%5492=2465%:%
%:%5493=2466%:%
%:%5494=2466%:%
%:%5495=2466%:%
%:%5496=2467%:%
%:%5497=2467%:%
%:%5498=2468%:%
%:%5499=2468%:%
%:%5500=2469%:%
%:%5501=2469%:%
%:%5502=2469%:%
%:%5503=2470%:%
%:%5504=2470%:%
%:%5505=2470%:%
%:%5506=2470%:%
%:%5507=2470%:%
%:%5508=2471%:%
%:%5509=2471%:%
%:%5510=2471%:%
%:%5511=2472%:%
%:%5512=2472%:%
%:%5513=2473%:%
%:%5514=2473%:%
%:%5515=2474%:%
%:%5516=2474%:%
%:%5517=2475%:%
%:%5518=2476%:%
%:%5519=2476%:%
%:%5520=2477%:%
%:%5521=2477%:%
%:%5522=2478%:%
%:%5523=2478%:%
%:%5524=2478%:%
%:%5525=2479%:%
%:%5526=2479%:%
%:%5527=2480%:%
%:%5528=2480%:%
%:%5529=2480%:%
%:%5530=2481%:%
%:%5531=2481%:%
%:%5532=2481%:%
%:%5533=2482%:%
%:%5534=2482%:%
%:%5535=2482%:%
%:%5536=2483%:%
%:%5537=2483%:%
%:%5538=2483%:%
%:%5539=2483%:%
%:%5540=2483%:%
%:%5541=2484%:%
%:%5542=2484%:%
%:%5543=2485%:%
%:%5544=2485%:%
%:%5545=2486%:%
%:%5546=2487%:%
%:%5547=2487%:%
%:%5548=2487%:%
%:%5549=2487%:%
%:%5550=2487%:%
%:%5551=2488%:%
%:%5552=2488%:%
%:%5553=2489%:%
%:%5554=2489%:%
%:%5555=2490%:%
%:%5556=2491%:%
%:%5557=2491%:%
%:%5558=2491%:%
%:%5559=2491%:%
%:%5560=2491%:%
%:%5561=2492%:%
%:%5562=2492%:%
%:%5563=2493%:%
%:%5564=2493%:%
%:%5565=2493%:%
%:%5566=2494%:%
%:%5567=2494%:%
%:%5568=2495%:%
%:%5569=2495%:%
%:%5570=2496%:%
%:%5571=2496%:%
%:%5572=2497%:%
%:%5573=2497%:%
%:%5574=2498%:%
%:%5575=2498%:%
%:%5576=2499%:%
%:%5577=2499%:%
%:%5578=2500%:%
%:%5579=2500%:%
%:%5580=2501%:%
%:%5581=2501%:%
%:%5582=2501%:%
%:%5583=2502%:%
%:%5584=2502%:%
%:%5585=2502%:%
%:%5586=2503%:%
%:%5587=2503%:%
%:%5588=2504%:%
%:%5589=2505%:%
%:%5590=2506%:%
%:%5591=2507%:%
%:%5592=2507%:%
%:%5593=2507%:%
%:%5594=2508%:%
%:%5595=2508%:%
%:%5596=2508%:%
%:%5597=2509%:%
%:%5598=2509%:%
%:%5599=2509%:%
%:%5600=2509%:%
%:%5601=2509%:%
%:%5602=2510%:%
%:%5603=2510%:%
%:%5604=2511%:%
%:%5605=2511%:%
%:%5606=2512%:%
%:%5607=2512%:%
%:%5608=2512%:%
%:%5609=2512%:%
%:%5610=2513%:%
%:%5611=2513%:%
%:%5612=2514%:%
%:%5613=2514%:%
%:%5614=2515%:%
%:%5615=2515%:%
%:%5616=2515%:%
%:%5617=2515%:%
%:%5618=2516%:%
%:%5619=2516%:%
%:%5620=2517%:%
%:%5621=2517%:%
%:%5622=2517%:%
%:%5623=2518%:%
%:%5624=2518%:%
%:%5625=2519%:%
%:%5626=2519%:%
%:%5627=2519%:%
%:%5628=2520%:%
%:%5629=2520%:%
%:%5630=2520%:%
%:%5631=2521%:%
%:%5632=2521%:%
%:%5633=2521%:%
%:%5634=2522%:%
%:%5635=2522%:%
%:%5636=2523%:%
%:%5637=2523%:%
%:%5638=2523%:%
%:%5639=2523%:%
%:%5640=2524%:%
%:%5641=2524%:%
%:%5642=2524%:%
%:%5643=2525%:%
%:%5644=2525%:%
%:%5645=2525%:%
%:%5646=2525%:%
%:%5647=2526%:%
%:%5648=2526%:%
%:%5649=2526%:%
%:%5650=2527%:%
%:%5651=2527%:%
%:%5652=2528%:%
%:%5653=2528%:%
%:%5654=2528%:%
%:%5655=2529%:%
%:%5656=2529%:%
%:%5657=2529%:%
%:%5658=2530%:%
%:%5659=2530%:%
%:%5660=2530%:%
%:%5661=2531%:%
%:%5662=2531%:%
%:%5663=2531%:%
%:%5664=2532%:%
%:%5665=2532%:%
%:%5666=2533%:%
%:%5667=2533%:%
%:%5668=2533%:%
%:%5669=2533%:%
%:%5670=2534%:%
%:%5671=2534%:%
%:%5672=2535%:%
%:%5673=2535%:%
%:%5674=2536%:%
%:%5675=2536%:%
%:%5676=2537%:%
%:%5677=2537%:%
%:%5678=2538%:%
%:%5679=2538%:%
%:%5680=2539%:%
%:%5681=2539%:%
%:%5682=2540%:%
%:%5683=2540%:%
%:%5684=2541%:%
%:%5685=2541%:%
%:%5686=2542%:%
%:%5687=2542%:%
%:%5688=2543%:%
%:%5689=2544%:%
%:%5690=2544%:%
%:%5691=2545%:%
%:%5692=2545%:%
%:%5693=2546%:%
%:%5694=2546%:%
%:%5695=2547%:%
%:%5696=2548%:%
%:%5697=2548%:%
%:%5698=2549%:%
%:%5699=2549%:%
%:%5700=2550%:%
%:%5701=2550%:%
%:%5704=2553%:%
%:%5705=2554%:%
%:%5706=2554%:%
%:%5707=2555%:%
%:%5708=2555%:%
%:%5709=2556%:%
%:%5710=2556%:%
%:%5711=2556%:%
%:%5712=2557%:%
%:%5713=2557%:%
%:%5714=2558%:%
%:%5715=2558%:%
%:%5716=2558%:%
%:%5717=2559%:%
%:%5718=2559%:%
%:%5720=2561%:%
%:%5721=2562%:%
%:%5722=2562%:%
%:%5723=2562%:%
%:%5724=2563%:%
%:%5725=2563%:%
%:%5726=2563%:%
%:%5727=2564%:%
%:%5728=2565%:%
%:%5729=2565%:%
%:%5730=2566%:%
%:%5731=2566%:%
%:%5732=2567%:%
%:%5733=2567%:%
%:%5734=2568%:%
%:%5735=2568%:%
%:%5736=2568%:%
%:%5737=2569%:%
%:%5738=2569%:%
%:%5739=2570%:%
%:%5740=2570%:%
%:%5741=2570%:%
%:%5742=2571%:%
%:%5743=2571%:%
%:%5744=2572%:%
%:%5745=2572%:%
%:%5746=2573%:%
%:%5747=2573%:%
%:%5748=2574%:%
%:%5749=2574%:%
%:%5750=2575%:%
%:%5751=2575%:%
%:%5752=2576%:%
%:%5753=2576%:%
%:%5754=2577%:%
%:%5755=2577%:%
%:%5756=2578%:%
%:%5757=2578%:%
%:%5758=2578%:%
%:%5759=2579%:%
%:%5760=2579%:%
%:%5761=2580%:%
%:%5762=2580%:%
%:%5763=2580%:%
%:%5764=2581%:%
%:%5765=2582%:%
%:%5766=2582%:%
%:%5767=2582%:%
%:%5768=2583%:%
%:%5769=2583%:%
%:%5770=2583%:%
%:%5771=2584%:%
%:%5772=2585%:%
%:%5773=2585%:%
%:%5774=2586%:%
%:%5775=2587%:%
%:%5776=2587%:%
%:%5777=2588%:%
%:%5778=2589%:%
%:%5779=2590%:%
%:%5780=2590%:%
%:%5781=2590%:%
%:%5783=2592%:%
%:%5784=2593%:%
%:%5785=2593%:%
%:%5786=2593%:%
%:%5787=2594%:%
%:%5788=2594%:%
%:%5789=2594%:%
%:%5790=2595%:%
%:%5791=2596%:%
%:%5792=2596%:%
%:%5793=2597%:%
%:%5794=2597%:%
%:%5795=2598%:%
%:%5796=2598%:%
%:%5797=2599%:%
%:%5798=2599%:%
%:%5799=2599%:%
%:%5800=2600%:%
%:%5801=2600%:%
%:%5802=2600%:%
%:%5803=2601%:%
%:%5804=2601%:%
%:%5805=2601%:%
%:%5806=2602%:%
%:%5807=2602%:%
%:%5808=2602%:%
%:%5809=2602%:%
%:%5810=2603%:%
%:%5811=2603%:%
%:%5812=2604%:%
%:%5813=2604%:%
%:%5814=2604%:%
%:%5815=2605%:%
%:%5816=2605%:%
%:%5817=2606%:%
%:%5818=2606%:%
%:%5819=2607%:%
%:%5820=2607%:%
%:%5821=2607%:%
%:%5822=2607%:%
%:%5823=2608%:%
%:%5824=2608%:%
%:%5825=2608%:%
%:%5826=2609%:%
%:%5827=2609%:%
%:%5828=2609%:%
%:%5829=2610%:%
%:%5830=2610%:%
%:%5831=2610%:%
%:%5832=2610%:%
%:%5833=2611%:%
%:%5834=2611%:%
%:%5835=2612%:%
%:%5836=2612%:%
%:%5837=2612%:%
%:%5838=2612%:%
%:%5839=2613%:%
%:%5840=2613%:%
%:%5841=2614%:%
%:%5842=2614%:%
%:%5843=2614%:%
%:%5846=2617%:%
%:%5847=2618%:%
%:%5848=2618%:%
%:%5849=2618%:%
%:%5850=2619%:%
%:%5851=2619%:%
%:%5852=2620%:%
%:%5853=2620%:%
%:%5854=2620%:%
%:%5855=2621%:%
%:%5856=2622%:%
%:%5857=2622%:%
%:%5858=2622%:%
%:%5859=2623%:%
%:%5860=2623%:%
%:%5861=2623%:%
%:%5862=2624%:%
%:%5863=2624%:%
%:%5864=2625%:%
%:%5865=2625%:%
%:%5869=2629%:%
%:%5870=2630%:%
%:%5871=2630%:%
%:%5872=2630%:%
%:%5873=2631%:%
%:%5874=2631%:%
%:%5875=2631%:%
%:%5877=2633%:%
%:%5878=2634%:%
%:%5879=2634%:%
%:%5880=2634%:%
%:%5881=2635%:%
%:%5882=2635%:%
%:%5883=2635%:%
%:%5886=2638%:%
%:%5887=2639%:%
%:%5888=2639%:%
%:%5889=2640%:%
%:%5890=2640%:%
%:%5891=2641%:%
%:%5892=2641%:%
%:%5893=2642%:%
%:%5894=2642%:%
%:%5895=2642%:%
%:%5896=2643%:%
%:%5897=2644%:%
%:%5898=2644%:%
%:%5899=2645%:%
%:%5900=2645%:%
%:%5901=2646%:%
%:%5902=2647%:%
%:%5903=2648%:%
%:%5904=2648%:%
%:%5905=2648%:%
%:%5906=2648%:%
%:%5907=2649%:%
%:%5908=2649%:%
%:%5909=2650%:%
%:%5910=2650%:%
%:%5911=2650%:%
%:%5914=2653%:%
%:%5915=2654%:%
%:%5916=2654%:%
%:%5917=2655%:%
%:%5918=2655%:%
%:%5919=2655%:%
%:%5921=2657%:%
%:%5922=2658%:%
%:%5923=2658%:%
%:%5924=2659%:%
%:%5925=2659%:%
%:%5926=2660%:%
%:%5927=2660%:%
%:%5928=2660%:%
%:%5929=2661%:%
%:%5930=2661%:%
%:%5933=2664%:%
%:%5934=2665%:%
%:%5935=2665%:%
%:%5936=2665%:%
%:%5937=2666%:%
%:%5938=2666%:%
%:%5939=2666%:%
%:%5940=2667%:%
%:%5941=2667%:%
%:%5942=2667%:%
%:%5943=2668%:%
%:%5944=2668%:%
%:%5945=2668%:%
%:%5946=2669%:%
%:%5947=2669%:%
%:%5948=2670%:%
%:%5949=2670%:%
%:%5950=2671%:%
%:%5951=2672%:%
%:%5952=2672%:%
%:%5953=2672%:%
%:%5957=2676%:%
%:%5958=2677%:%
%:%5959=2677%:%
%:%5960=2677%:%
%:%5961=2678%:%
%:%5962=2678%:%
%:%5963=2678%:%
%:%5964=2679%:%
%:%5965=2679%:%
%:%5966=2680%:%
%:%5967=2680%:%
%:%5968=2680%:%
%:%5969=2681%:%
%:%5970=2681%:%
%:%5971=2681%:%
%:%5972=2682%:%
%:%5973=2682%:%
%:%5974=2682%:%
%:%5975=2682%:%
%:%5976=2683%:%
%:%5977=2683%:%
%:%5978=2684%:%
%:%5979=2684%:%
%:%5980=2684%:%
%:%5981=2684%:%
%:%5982=2685%:%
%:%5983=2685%:%
%:%5984=2686%:%
%:%5985=2686%:%
%:%5986=2687%:%
%:%5987=2687%:%
%:%5988=2688%:%
%:%6003=2691%:%
%:%6013=2693%:%
%:%6014=2693%:%
%:%6015=2694%:%
%:%6016=2695%:%
%:%6023=2696%:%
%:%6024=2696%:%
%:%6025=2697%:%
%:%6026=2697%:%
%:%6027=2698%:%
%:%6028=2698%:%
%:%6029=2698%:%
%:%6030=2699%:%
%:%6031=2699%:%
%:%6032=2700%:%
%:%6033=2700%:%
%:%6034=2700%:%
%:%6035=2700%:%
%:%6036=2700%:%
%:%6037=2701%:%
%:%6038=2701%:%
%:%6039=2701%:%
%:%6040=2701%:%
%:%6041=2701%:%
%:%6042=2702%:%
%:%6043=2702%:%
%:%6044=2702%:%
%:%6045=2702%:%
%:%6046=2702%:%
%:%6047=2703%:%
%:%6048=2703%:%
%:%6049=2703%:%
%:%6050=2703%:%
%:%6051=2704%:%
%:%6052=2704%:%
%:%6053=2705%:%
%:%6054=2705%:%
%:%6055=2706%:%
%:%6056=2706%:%
%:%6057=2707%:%
%:%6058=2707%:%
%:%6059=2708%:%
%:%6060=2708%:%
%:%6061=2709%:%
%:%6062=2709%:%
%:%6063=2710%:%
%:%6064=2710%:%
%:%6065=2711%:%
%:%6066=2711%:%
%:%6067=2711%:%
%:%6068=2711%:%
%:%6069=2712%:%
%:%6070=2712%:%
%:%6071=2713%:%
%:%6072=2714%:%
%:%6073=2714%:%
%:%6074=2714%:%
%:%6075=2715%:%
%:%6076=2715%:%
%:%6077=2715%:%
%:%6078=2715%:%
%:%6079=2715%:%
%:%6080=2716%:%
%:%6081=2716%:%
%:%6082=2716%:%
%:%6083=2716%:%
%:%6084=2717%:%
%:%6085=2717%:%
%:%6086=2718%:%
%:%6087=2718%:%
%:%6088=2718%:%
%:%6089=2718%:%
%:%6090=2718%:%
%:%6091=2719%:%
%:%6092=2719%:%
%:%6093=2719%:%
%:%6094=2719%:%
%:%6095=2720%:%
%:%6096=2720%:%
%:%6097=2721%:%
%:%6098=2721%:%
%:%6099=2721%:%
%:%6100=2721%:%
%:%6101=2722%:%
%:%6102=2722%:%
%:%6103=2723%:%
%:%6104=2723%:%
%:%6105=2724%:%
%:%6106=2724%:%
%:%6107=2724%:%
%:%6108=2724%:%
%:%6109=2725%:%
%:%6110=2725%:%
%:%6111=2725%:%
%:%6112=2726%:%
%:%6113=2726%:%
%:%6114=2727%:%
%:%6115=2728%:%
%:%6116=2728%:%
%:%6117=2728%:%
%:%6118=2729%:%
%:%6119=2729%:%
%:%6120=2729%:%
%:%6121=2730%:%
%:%6122=2730%:%
%:%6123=2730%:%
%:%6124=2731%:%
%:%6125=2731%:%
%:%6126=2731%:%
%:%6127=2732%:%
%:%6128=2732%:%
%:%6129=2733%:%
%:%6130=2733%:%
%:%6131=2734%:%
%:%6132=2734%:%
%:%6133=2735%:%
%:%6134=2735%:%
%:%6135=2735%:%
%:%6136=2736%:%
%:%6137=2736%:%
%:%6138=2736%:%
%:%6139=2736%:%
%:%6140=2736%:%
%:%6141=2737%:%
%:%6142=2737%:%
%:%6143=2737%:%
%:%6144=2737%:%
%:%6145=2738%:%
%:%6146=2738%:%
%:%6147=2739%:%
%:%6148=2740%:%
%:%6149=2740%:%
%:%6150=2741%:%
%:%6151=2741%:%
%:%6152=2742%:%
%:%6153=2742%:%
%:%6154=2743%:%
%:%6155=2743%:%
%:%6156=2744%:%
%:%6157=2744%:%
%:%6158=2744%:%
%:%6159=2745%:%
%:%6160=2745%:%
%:%6161=2746%:%
%:%6162=2746%:%
%:%6163=2746%:%
%:%6164=2746%:%
%:%6165=2747%:%
%:%6166=2747%:%
%:%6167=2747%:%
%:%6168=2747%:%
%:%6169=2747%:%
%:%6170=2748%:%
%:%6171=2748%:%
%:%6172=2749%:%
%:%6173=2749%:%
%:%6174=2750%:%
%:%6175=2750%:%
%:%6176=2750%:%
%:%6177=2751%:%
%:%6178=2751%:%
%:%6179=2752%:%
%:%6180=2752%:%
%:%6181=2753%:%
%:%6182=2753%:%
%:%6183=2753%:%
%:%6184=2754%:%
%:%6185=2755%:%
%:%6186=2755%:%
%:%6187=2755%:%
%:%6188=2756%:%
%:%6189=2756%:%
%:%6190=2756%:%
%:%6191=2757%:%
%:%6192=2758%:%
%:%6193=2758%:%
%:%6194=2759%:%
%:%6195=2759%:%
%:%6196=2759%:%
%:%6197=2759%:%
%:%6198=2760%:%
%:%6199=2760%:%
%:%6200=2760%:%
%:%6201=2761%:%
%:%6202=2761%:%
%:%6203=2761%:%
%:%6204=2761%:%
%:%6205=2762%:%
%:%6206=2762%:%
%:%6207=2762%:%
%:%6208=2763%:%
%:%6209=2763%:%
%:%6210=2763%:%
%:%6211=2763%:%
%:%6212=2764%:%
%:%6213=2764%:%
%:%6214=2764%:%
%:%6215=2765%:%
%:%6216=2765%:%
%:%6217=2765%:%
%:%6218=2766%:%
%:%6219=2766%:%
%:%6220=2766%:%
%:%6221=2766%:%
%:%6222=2767%:%
%:%6223=2767%:%
%:%6224=2767%:%
%:%6225=2768%:%
%:%6226=2768%:%
%:%6227=2768%:%
%:%6228=2768%:%
%:%6229=2769%:%
%:%6230=2769%:%
%:%6231=2770%:%
%:%6232=2771%:%
%:%6233=2771%:%
%:%6234=2772%:%
%:%6235=2772%:%
%:%6236=2773%:%
%:%6237=2773%:%
%:%6238=2773%:%
%:%6239=2774%:%
%:%6240=2774%:%
%:%6241=2775%:%
%:%6242=2775%:%
%:%6243=2775%:%
%:%6244=2776%:%
%:%6245=2776%:%
%:%6246=2776%:%
%:%6247=2777%:%
%:%6248=2777%:%
%:%6249=2777%:%
%:%6250=2778%:%
%:%6251=2778%:%
%:%6252=2778%:%
%:%6253=2779%:%
%:%6254=2779%:%
%:%6255=2779%:%
%:%6256=2780%:%
%:%6257=2780%:%
%:%6258=2781%:%
%:%6259=2781%:%
%:%6260=2782%:%
%:%6261=2783%:%
%:%6262=2784%:%
%:%6263=2785%:%
%:%6264=2785%:%
%:%6265=2785%:%
%:%6266=2785%:%
%:%6267=2786%:%
%:%6268=2786%:%
%:%6269=2787%:%
%:%6270=2787%:%
%:%6271=2787%:%
%:%6272=2787%:%
%:%6273=2787%:%
%:%6274=2788%:%
%:%6275=2788%:%
%:%6276=2788%:%
%:%6277=2788%:%
%:%6278=2789%:%
%:%6279=2789%:%
%:%6280=2790%:%
%:%6286=2790%:%
%:%6289=2791%:%
%:%6290=2792%:%
%:%6291=2793%:%
%:%6292=2793%:%
%:%6293=2794%:%
%:%6294=2795%:%
%:%6296=2797%:%
%:%6303=2798%:%
%:%6304=2798%:%
%:%6305=2799%:%
%:%6306=2799%:%
%:%6307=2799%:%
%:%6308=2799%:%
%:%6309=2800%:%
%:%6310=2800%:%
%:%6311=2801%:%
%:%6312=2801%:%
%:%6313=2801%:%
%:%6314=2802%:%
%:%6315=2802%:%
%:%6316=2803%:%
%:%6317=2803%:%
%:%6318=2803%:%
%:%6319=2803%:%
%:%6320=2804%:%
%:%6321=2804%:%
%:%6322=2804%:%
%:%6323=2805%:%
%:%6324=2805%:%
%:%6325=2805%:%
%:%6326=2806%:%
%:%6327=2806%:%
%:%6328=2807%:%
%:%6329=2807%:%
%:%6330=2807%:%
%:%6331=2807%:%
%:%6332=2807%:%
%:%6333=2808%:%
%:%6334=2808%:%
%:%6335=2808%:%
%:%6336=2808%:%
%:%6337=2808%:%
%:%6338=2809%:%
%:%6344=2809%:%
%:%6347=2810%:%
%:%6348=2811%:%
%:%6349=2812%:%
%:%6350=2812%:%
%:%6351=2813%:%
%:%6352=2814%:%
%:%6359=2815%:%
%:%6360=2815%:%
%:%6361=2816%:%
%:%6362=2816%:%
%:%6363=2817%:%
%:%6364=2817%:%
%:%6365=2817%:%
%:%6366=2817%:%
%:%6367=2818%:%
%:%6368=2818%:%
%:%6369=2819%:%
%:%6370=2819%:%
%:%6371=2820%:%
%:%6372=2820%:%
%:%6373=2820%:%
%:%6374=2821%:%
%:%6375=2821%:%
%:%6376=2822%:%
%:%6377=2822%:%
%:%6378=2822%:%
%:%6379=2822%:%
%:%6380=2823%:%
%:%6381=2823%:%
%:%6382=2823%:%
%:%6383=2824%:%
%:%6384=2824%:%
%:%6385=2824%:%
%:%6386=2824%:%
%:%6387=2825%:%
%:%6388=2825%:%
%:%6389=2825%:%
%:%6390=2825%:%
%:%6391=2826%:%
%:%6392=2826%:%
%:%6393=2826%:%
%:%6394=2826%:%
%:%6395=2827%:%
%:%6396=2827%:%
%:%6397=2828%:%
%:%6398=2828%:%
%:%6399=2829%:%
%:%6400=2829%:%
%:%6401=2830%:%
%:%6402=2830%:%
%:%6403=2831%:%
%:%6404=2832%:%
%:%6405=2832%:%
%:%6406=2832%:%
%:%6407=2833%:%
%:%6408=2833%:%
%:%6409=2833%:%
%:%6410=2834%:%
%:%6411=2834%:%
%:%6412=2834%:%
%:%6413=2835%:%
%:%6414=2835%:%
%:%6415=2835%:%
%:%6416=2835%:%
%:%6417=2835%:%
%:%6418=2836%:%
%:%6419=2836%:%
%:%6420=2836%:%
%:%6421=2836%:%
%:%6422=2836%:%
%:%6423=2837%:%
%:%6424=2837%:%
%:%6425=2837%:%
%:%6426=2837%:%
%:%6427=2838%:%
%:%6428=2838%:%
%:%6429=2839%:%
%:%6430=2839%:%
%:%6431=2839%:%
%:%6432=2840%:%
%:%6433=2840%:%
%:%6434=2841%:%
%:%6435=2841%:%
%:%6436=2842%:%
%:%6437=2842%:%
%:%6438=2843%:%
%:%6439=2843%:%
%:%6440=2844%:%
%:%6441=2844%:%
%:%6442=2845%:%
%:%6443=2845%:%
%:%6444=2846%:%
%:%6445=2847%:%
%:%6446=2847%:%
%:%6447=2848%:%
%:%6448=2849%:%
%:%6449=2849%:%
%:%6450=2849%:%
%:%6451=2850%:%
%:%6452=2850%:%
%:%6454=2852%:%
%:%6455=2853%:%
%:%6456=2853%:%
%:%6457=2853%:%
%:%6458=2854%:%
%:%6459=2854%:%
%:%6460=2854%:%
%:%6461=2855%:%
%:%6462=2856%:%
%:%6463=2856%:%
%:%6464=2857%:%
%:%6465=2857%:%
%:%6466=2858%:%
%:%6467=2858%:%
%:%6468=2859%:%
%:%6469=2859%:%
%:%6470=2859%:%
%:%6471=2860%:%
%:%6472=2860%:%
%:%6473=2860%:%
%:%6474=2860%:%
%:%6475=2861%:%
%:%6476=2861%:%
%:%6477=2861%:%
%:%6478=2861%:%
%:%6479=2861%:%
%:%6480=2862%:%
%:%6481=2862%:%
%:%6482=2863%:%
%:%6483=2863%:%
%:%6484=2864%:%
%:%6485=2864%:%
%:%6486=2865%:%
%:%6487=2865%:%
%:%6488=2865%:%
%:%6489=2866%:%
%:%6490=2866%:%
%:%6491=2867%:%
%:%6492=2867%:%
%:%6493=2868%:%
%:%6494=2868%:%
%:%6495=2869%:%
%:%6496=2869%:%
%:%6497=2870%:%
%:%6498=2870%:%
%:%6499=2870%:%
%:%6500=2870%:%
%:%6501=2871%:%
%:%6502=2871%:%
%:%6503=2871%:%
%:%6504=2872%:%
%:%6505=2872%:%
%:%6506=2872%:%
%:%6507=2873%:%
%:%6508=2873%:%
%:%6509=2873%:%
%:%6510=2873%:%
%:%6511=2874%:%
%:%6512=2874%:%
%:%6513=2875%:%
%:%6514=2875%:%
%:%6515=2876%:%
%:%6516=2876%:%
%:%6517=2876%:%
%:%6518=2877%:%
%:%6519=2877%:%
%:%6520=2877%:%
%:%6521=2878%:%
%:%6522=2878%:%
%:%6523=2878%:%
%:%6524=2879%:%
%:%6525=2879%:%
%:%6526=2880%:%
%:%6527=2880%:%
%:%6528=2880%:%
%:%6529=2880%:%
%:%6530=2881%:%
%:%6531=2881%:%
%:%6532=2882%:%
%:%6533=2882%:%
%:%6534=2882%:%
%:%6535=2883%:%
%:%6536=2884%:%
%:%6537=2884%:%
%:%6538=2884%:%
%:%6539=2885%:%
%:%6540=2885%:%
%:%6541=2886%:%
%:%6542=2886%:%
%:%6543=2886%:%
%:%6544=2887%:%
%:%6545=2887%:%
%:%6546=2887%:%
%:%6547=2887%:%
%:%6548=2888%:%
%:%6549=2888%:%
%:%6550=2888%:%
%:%6551=2888%:%
%:%6552=2889%:%
%:%6553=2889%:%
%:%6554=2889%:%
%:%6555=2889%:%
%:%6556=2890%:%
%:%6557=2890%:%
%:%6558=2890%:%
%:%6559=2890%:%
%:%6560=2891%:%
%:%6561=2891%:%
%:%6562=2891%:%
%:%6563=2891%:%
%:%6564=2892%:%
%:%6565=2892%:%
%:%6566=2892%:%
%:%6567=2892%:%
%:%6568=2893%:%
%:%6569=2893%:%
%:%6570=2893%:%
%:%6571=2894%:%
%:%6572=2894%:%
%:%6573=2895%:%
%:%6574=2895%:%
%:%6575=2896%:%
%:%6576=2896%:%
%:%6577=2896%:%
%:%6578=2897%:%
%:%6579=2897%:%
%:%6580=2897%:%
%:%6581=2898%:%
%:%6582=2898%:%
%:%6583=2898%:%
%:%6584=2898%:%
%:%6585=2899%:%
%:%6586=2899%:%
%:%6587=2900%:%
%:%6588=2901%:%
%:%6589=2901%:%
%:%6590=2901%:%
%:%6592=2903%:%
%:%6593=2903%:%
%:%6594=2903%:%
%:%6595=2904%:%
%:%6601=2904%:%
%:%6604=2905%:%
%:%6605=2906%:%
%:%6606=2907%:%
%:%6607=2907%:%
%:%6608=2908%:%
%:%6609=2909%:%
%:%6612=2910%:%
%:%6616=2910%:%
%:%6617=2910%:%
%:%6618=2911%:%
%:%6619=2911%:%
%:%6620=2912%:%
%:%6621=2912%:%
%:%6622=2913%:%
%:%6628=2913%:%
%:%6631=2914%:%
%:%6632=2915%:%
%:%6633=2915%:%
%:%6634=2916%:%
%:%6635=2917%:%
%:%6638=2918%:%
%:%6642=2918%:%
%:%6643=2918%:%
%:%6644=2919%:%
%:%6645=2919%:%
%:%6646=2920%:%
%:%6647=2920%:%
%:%6648=2921%:%
%:%6654=2921%:%
%:%6657=2922%:%
%:%6658=2923%:%
%:%6659=2923%:%
%:%6660=2924%:%
%:%6667=2925%:%
%:%6668=2925%:%
%:%6669=2926%:%
%:%6670=2926%:%
%:%6671=2927%:%
%:%6672=2927%:%
%:%6673=2927%:%
%:%6674=2927%:%
%:%6675=2928%:%
%:%6676=2928%:%
%:%6677=2929%:%
%:%6678=2929%:%
%:%6679=2930%:%
%:%6680=2930%:%
%:%6681=2931%:%
%:%6682=2931%:%
%:%6683=2931%:%
%:%6684=2931%:%
%:%6685=2932%:%
%:%6686=2932%:%
%:%6687=2932%:%
%:%6688=2933%:%
%:%6689=2933%:%
%:%6690=2933%:%
%:%6691=2933%:%
%:%6692=2934%:%
%:%6693=2934%:%
%:%6694=2934%:%
%:%6695=2935%:%
%:%6696=2935%:%
%:%6697=2935%:%
%:%6698=2935%:%
%:%6699=2936%:%
%:%6705=2936%:%
%:%6708=2937%:%
%:%6709=2938%:%
%:%6710=2938%:%
%:%6711=2939%:%
%:%6718=2940%:%
%:%6719=2940%:%
%:%6720=2941%:%
%:%6721=2941%:%
%:%6722=2942%:%
%:%6723=2942%:%
%:%6724=2942%:%
%:%6725=2942%:%
%:%6726=2943%:%
%:%6727=2943%:%
%:%6728=2944%:%
%:%6729=2944%:%
%:%6730=2945%:%
%:%6731=2945%:%
%:%6732=2946%:%
%:%6733=2946%:%
%:%6734=2946%:%
%:%6735=2946%:%
%:%6736=2947%:%
%:%6737=2947%:%
%:%6738=2947%:%
%:%6739=2948%:%
%:%6740=2948%:%
%:%6741=2948%:%
%:%6742=2948%:%
%:%6743=2949%:%
%:%6744=2949%:%
%:%6745=2949%:%
%:%6746=2950%:%
%:%6747=2950%:%
%:%6748=2950%:%
%:%6749=2950%:%
%:%6750=2951%:%
%:%6756=2951%:%
%:%6759=2952%:%
%:%6760=2953%:%
%:%6761=2953%:%
%:%6762=2954%:%
%:%6765=2955%:%
%:%6769=2955%:%
%:%6770=2955%:%
%:%6771=2956%:%
%:%6772=2956%:%
%:%6773=2957%:%
%:%6779=2957%:%
%:%6782=2958%:%
%:%6783=2959%:%
%:%6784=2959%:%
%:%6785=2960%:%
%:%6788=2961%:%
%:%6792=2961%:%
%:%6793=2961%:%
%:%6794=2962%:%
%:%6795=2962%:%
%:%6796=2963%:%
%:%6797=2963%:%
%:%6798=2964%:%
%:%6804=2964%:%
%:%6807=2965%:%
%:%6808=2966%:%
%:%6809=2966%:%
%:%6810=2967%:%
%:%6813=2968%:%
%:%6817=2968%:%
%:%6818=2968%:%
%:%6819=2969%:%
%:%6820=2969%:%
%:%6821=2970%:%
%:%6827=2970%:%
%:%6830=2971%:%
%:%6831=2972%:%
%:%6832=2972%:%
%:%6833=2973%:%
%:%6836=2974%:%
%:%6840=2974%:%
%:%6841=2974%:%
%:%6842=2975%:%
%:%6843=2975%:%
%:%6844=2976%:%
%:%6845=2976%:%
%:%6846=2977%:%
%:%6852=2977%:%
%:%6855=2978%:%
%:%6856=2979%:%
%:%6857=2979%:%
%:%6858=2980%:%
%:%6865=2981%:%
%:%6866=2981%:%
%:%6867=2982%:%
%:%6868=2982%:%
%:%6869=2983%:%
%:%6870=2984%:%
%:%6871=2984%:%
%:%6872=2984%:%
%:%6873=2985%:%
%:%6874=2985%:%
%:%6875=2985%:%
%:%6876=2986%:%
%:%6877=2987%:%
%:%6878=2987%:%
%:%6879=2988%:%
%:%6880=2988%:%
%:%6881=2988%:%
%:%6882=2989%:%
%:%6883=2989%:%
%:%6884=2990%:%
%:%6885=2990%:%
%:%6886=2991%:%
%:%6887=2992%:%
%:%6888=2992%:%
%:%6889=2993%:%
%:%6890=2993%:%
%:%6891=2994%:%
%:%6892=2995%:%
%:%6893=2995%:%
%:%6894=2996%:%
%:%6895=2996%:%
%:%6896=2997%:%
%:%6897=2998%:%
%:%6898=2998%:%
%:%6899=2998%:%
%:%6900=2999%:%
%:%6901=2999%:%
%:%6902=2999%:%
%:%6903=2999%:%
%:%6904=3000%:%
%:%6905=3000%:%
%:%6906=3000%:%
%:%6907=3000%:%
%:%6908=3001%:%
%:%6909=3001%:%
%:%6910=3001%:%
%:%6911=3001%:%
%:%6912=3002%:%
%:%6913=3002%:%
%:%6914=3003%:%
%:%6915=3003%:%
%:%6916=3004%:%
%:%6917=3004%:%
%:%6918=3005%:%
%:%6919=3006%:%
%:%6920=3006%:%
%:%6921=3007%:%
%:%6922=3007%:%
%:%6923=3008%:%
%:%6924=3009%:%
%:%6925=3009%:%
%:%6926=3009%:%
%:%6927=3010%:%
%:%6928=3010%:%
%:%6929=3010%:%
%:%6930=3010%:%
%:%6931=3010%:%
%:%6932=3011%:%
%:%6933=3011%:%
%:%6934=3011%:%
%:%6935=3011%:%
%:%6936=3011%:%
%:%6937=3012%:%
%:%6938=3012%:%
%:%6939=3012%:%
%:%6940=3012%:%
%:%6941=3012%:%
%:%6942=3013%:%
%:%6943=3013%:%
%:%6944=3014%:%
%:%6945=3014%:%
%:%6946=3015%:%
%:%6947=3015%:%
%:%6948=3016%:%
%:%6949=3016%:%
%:%6950=3016%:%
%:%6951=3017%:%
%:%6952=3017%:%
%:%6953=3017%:%
%:%6954=3018%:%
%:%6955=3018%:%
%:%6956=3018%:%
%:%6957=3018%:%
%:%6958=3018%:%
%:%6959=3019%:%
%:%6960=3019%:%
%:%6961=3019%:%
%:%6962=3019%:%
%:%6963=3020%:%
%:%6973=3022%:%
%:%6975=3024%:%
%:%6976=3024%:%
%:%6977=3025%:%
%:%6978=3026%:%
%:%6985=3027%:%
%:%6986=3027%:%
%:%6987=3028%:%
%:%6988=3028%:%
%:%6989=3029%:%
%:%6990=3029%:%
%:%6991=3029%:%
%:%6992=3030%:%
%:%6993=3030%:%
%:%6994=3030%:%
%:%6995=3031%:%
%:%6996=3031%:%
%:%6997=3032%:%
%:%6998=3032%:%
%:%6999=3032%:%
%:%7000=3033%:%
%:%7001=3033%:%
%:%7002=3034%:%
%:%7003=3034%:%
%:%7004=3034%:%
%:%7005=3035%:%
%:%7006=3035%:%
%:%7007=3036%:%
%:%7008=3036%:%
%:%7009=3036%:%
%:%7010=3036%:%
%:%7011=3037%:%
%:%7012=3037%:%
%:%7013=3038%:%
%:%7014=3038%:%
%:%7015=3038%:%
%:%7016=3039%:%
%:%7017=3039%:%
%:%7018=3039%:%
%:%7019=3040%:%
%:%7020=3040%:%
%:%7021=3040%:%
%:%7022=3041%:%
%:%7023=3041%:%
%:%7024=3041%:%
%:%7025=3042%:%
%:%7026=3042%:%
%:%7027=3042%:%
%:%7028=3042%:%
%:%7029=3043%:%
%:%7044=3048%:%
%:%7056=3050%:%
%:%7057=3051%:%
%:%7058=3052%:%
%:%7060=3055%:%
%:%7061=3055%:%
%:%7062=3056%:%
%:%7063=3057%:%
%:%7070=3058%:%
%:%7071=3058%:%
%:%7072=3059%:%
%:%7073=3059%:%
%:%7074=3059%:%
%:%7075=3060%:%
%:%7076=3060%:%
%:%7077=3060%:%
%:%7078=3061%:%
%:%7079=3061%:%
%:%7080=3062%:%
%:%7081=3062%:%
%:%7082=3063%:%
%:%7088=3063%:%
%:%7091=3064%:%
%:%7092=3065%:%
%:%7093=3065%:%
%:%7094=3066%:%
%:%7101=3067%:%
%:%7102=3067%:%
%:%7103=3068%:%
%:%7104=3068%:%
%:%7105=3069%:%
%:%7106=3070%:%
%:%7107=3070%:%
%:%7108=3071%:%
%:%7109=3071%:%
%:%7110=3072%:%
%:%7111=3072%:%
%:%7112=3073%:%
%:%7113=3073%:%
%:%7114=3074%:%
%:%7115=3074%:%
%:%7116=3074%:%
%:%7117=3074%:%
%:%7118=3075%:%
%:%7119=3075%:%
%:%7120=3076%:%
%:%7121=3076%:%
%:%7122=3076%:%
%:%7123=3076%:%
%:%7124=3077%:%
%:%7125=3077%:%
%:%7126=3078%:%
%:%7127=3078%:%
%:%7128=3079%:%
%:%7129=3079%:%
%:%7130=3079%:%
%:%7131=3079%:%
%:%7132=3080%:%
%:%7133=3080%:%
%:%7134=3081%:%
%:%7135=3081%:%
%:%7136=3082%:%
%:%7137=3082%:%
%:%7138=3083%:%
%:%7139=3083%:%
%:%7140=3084%:%
%:%7141=3084%:%
%:%7142=3084%:%
%:%7143=3084%:%
%:%7144=3085%:%
%:%7145=3085%:%
%:%7146=3086%:%
%:%7147=3086%:%
%:%7148=3087%:%
%:%7149=3087%:%
%:%7150=3088%:%
%:%7151=3088%:%
%:%7152=3089%:%
%:%7153=3089%:%
%:%7154=3090%:%
%:%7155=3090%:%
%:%7156=3091%:%
%:%7157=3091%:%
%:%7158=3092%:%
%:%7159=3093%:%
%:%7160=3094%:%
%:%7161=3094%:%
%:%7162=3094%:%
%:%7163=3095%:%
%:%7164=3095%:%
%:%7165=3096%:%
%:%7166=3096%:%
%:%7167=3097%:%
%:%7168=3098%:%
%:%7169=3098%:%
%:%7170=3098%:%
%:%7171=3098%:%
%:%7172=3098%:%
%:%7173=3099%:%
%:%7174=3099%:%
%:%7175=3099%:%
%:%7176=3099%:%
%:%7177=3099%:%
%:%7178=3100%:%
%:%7179=3100%:%
%:%7180=3101%:%
%:%7181=3101%:%
%:%7182=3102%:%
%:%7183=3102%:%
%:%7184=3103%:%
%:%7185=3103%:%
%:%7186=3104%:%
%:%7187=3104%:%
%:%7188=3105%:%
%:%7189=3105%:%
%:%7190=3106%:%
%:%7191=3106%:%
%:%7192=3107%:%
%:%7193=3107%:%
%:%7194=3108%:%
%:%7195=3109%:%
%:%7196=3110%:%
%:%7197=3110%:%
%:%7198=3110%:%
%:%7199=3111%:%
%:%7200=3111%:%
%:%7201=3112%:%
%:%7202=3112%:%
%:%7203=3113%:%
%:%7204=3114%:%
%:%7205=3115%:%
%:%7206=3115%:%
%:%7207=3115%:%
%:%7208=3115%:%
%:%7209=3115%:%
%:%7210=3116%:%
%:%7211=3116%:%
%:%7212=3116%:%
%:%7213=3116%:%
%:%7214=3116%:%
%:%7215=3117%:%
%:%7216=3117%:%
%:%7217=3118%:%
%:%7218=3118%:%
%:%7219=3119%:%
%:%7220=3119%:%
%:%7221=3120%:%
%:%7222=3120%:%
%:%7223=3121%:%
%:%7224=3121%:%
%:%7225=3122%:%
%:%7226=3122%:%
%:%7227=3123%:%
%:%7228=3123%:%
%:%7229=3123%:%
%:%7230=3123%:%
%:%7231=3124%:%
%:%7232=3124%:%
%:%7233=3125%:%
%:%7234=3125%:%
%:%7235=3126%:%
%:%7236=3126%:%
%:%7237=3127%:%
%:%7238=3127%:%
%:%7239=3128%:%
%:%7240=3128%:%
%:%7241=3129%:%
%:%7242=3129%:%
%:%7243=3130%:%
%:%7244=3130%:%
%:%7245=3131%:%
%:%7246=3132%:%
%:%7247=3133%:%
%:%7248=3134%:%
%:%7249=3134%:%
%:%7250=3134%:%
%:%7251=3135%:%
%:%7252=3135%:%
%:%7253=3136%:%
%:%7254=3136%:%
%:%7255=3137%:%
%:%7256=3138%:%
%:%7257=3138%:%
%:%7258=3138%:%
%:%7259=3138%:%
%:%7260=3138%:%
%:%7261=3139%:%
%:%7262=3139%:%
%:%7263=3139%:%
%:%7264=3139%:%
%:%7265=3139%:%
%:%7266=3140%:%
%:%7267=3140%:%
%:%7268=3141%:%
%:%7269=3141%:%
%:%7270=3142%:%
%:%7271=3142%:%
%:%7272=3143%:%
%:%7273=3143%:%
%:%7274=3144%:%
%:%7275=3144%:%
%:%7276=3145%:%
%:%7277=3145%:%
%:%7278=3146%:%
%:%7279=3146%:%
%:%7280=3147%:%
%:%7281=3147%:%
%:%7282=3148%:%
%:%7283=3149%:%
%:%7284=3150%:%
%:%7285=3151%:%
%:%7286=3151%:%
%:%7287=3151%:%
%:%7288=3152%:%
%:%7289=3152%:%
%:%7290=3153%:%
%:%7291=3153%:%
%:%7292=3154%:%
%:%7293=3155%:%
%:%7294=3155%:%
%:%7295=3155%:%
%:%7296=3156%:%
%:%7297=3156%:%
%:%7298=3157%:%
%:%7299=3158%:%
%:%7300=3159%:%
%:%7301=3159%:%
%:%7302=3159%:%
%:%7303=3159%:%
%:%7304=3159%:%
%:%7305=3160%:%
%:%7306=3160%:%
%:%7307=3160%:%
%:%7308=3160%:%
%:%7309=3160%:%
%:%7310=3161%:%
%:%7311=3161%:%
%:%7312=3161%:%
%:%7313=3162%:%
%:%7314=3162%:%
%:%7315=3163%:%
%:%7316=3163%:%
%:%7317=3164%:%
%:%7318=3164%:%
%:%7319=3165%:%
%:%7320=3165%:%
%:%7321=3166%:%
%:%7322=3166%:%
%:%7323=3167%:%
%:%7324=3167%:%
%:%7325=3168%:%
%:%7326=3169%:%
%:%7327=3169%:%
%:%7328=3170%:%
%:%7329=3170%:%
%:%7330=3171%:%
%:%7331=3172%:%
%:%7332=3172%:%
%:%7333=3173%:%
%:%7334=3173%:%
%:%7335=3174%:%
%:%7336=3175%:%
%:%7337=3175%:%
%:%7338=3176%:%
%:%7339=3176%:%
%:%7340=3177%:%
%:%7346=3177%:%
%:%7349=3178%:%
%:%7350=3179%:%
%:%7351=3179%:%
%:%7352=3180%:%
%:%7359=3181%:%
%:%7360=3181%:%
%:%7361=3182%:%
%:%7362=3182%:%
%:%7363=3182%:%
%:%7364=3182%:%
%:%7365=3183%:%
%:%7366=3183%:%
%:%7367=3183%:%
%:%7368=3183%:%
%:%7369=3184%:%
%:%7370=3184%:%
%:%7371=3185%:%
%:%7372=3185%:%
%:%7373=3186%:%
%:%7374=3186%:%
%:%7375=3187%:%
%:%7376=3187%:%
%:%7377=3188%:%
%:%7378=3188%:%
%:%7379=3189%:%
%:%7380=3189%:%
%:%7381=3190%:%
%:%7382=3190%:%
%:%7383=3191%:%
%:%7384=3192%:%
%:%7385=3192%:%
%:%7386=3193%:%
%:%7387=3193%:%
%:%7388=3194%:%
%:%7389=3194%:%
%:%7390=3195%:%
%:%7391=3195%:%
%:%7392=3195%:%
%:%7393=3196%:%
%:%7394=3196%:%
%:%7395=3197%:%
%:%7396=3197%:%
%:%7397=3197%:%
%:%7398=3198%:%
%:%7399=3198%:%
%:%7400=3199%:%
%:%7401=3199%:%
%:%7402=3200%:%
%:%7403=3200%:%
%:%7404=3200%:%
%:%7405=3200%:%
%:%7406=3201%:%
%:%7407=3201%:%
%:%7408=3202%:%
%:%7409=3202%:%
%:%7410=3203%:%
%:%7411=3203%:%
%:%7412=3204%:%
%:%7413=3204%:%
%:%7414=3205%:%
%:%7415=3205%:%
%:%7416=3205%:%
%:%7417=3205%:%
%:%7418=3206%:%
%:%7419=3206%:%
%:%7420=3207%:%
%:%7421=3207%:%
%:%7422=3208%:%
%:%7423=3208%:%
%:%7424=3209%:%
%:%7425=3209%:%
%:%7426=3210%:%
%:%7427=3210%:%
%:%7428=3211%:%
%:%7429=3212%:%
%:%7430=3212%:%
%:%7431=3212%:%
%:%7432=3213%:%
%:%7433=3213%:%
%:%7434=3213%:%
%:%7435=3213%:%
%:%7436=3214%:%
%:%7437=3214%:%
%:%7438=3215%:%
%:%7439=3216%:%
%:%7440=3217%:%
%:%7441=3217%:%
%:%7442=3217%:%
%:%7443=3217%:%
%:%7444=3218%:%
%:%7445=3218%:%
%:%7446=3218%:%
%:%7447=3218%:%
%:%7448=3218%:%
%:%7449=3219%:%
%:%7450=3219%:%
%:%7451=3220%:%
%:%7452=3220%:%
%:%7453=3221%:%
%:%7454=3221%:%
%:%7455=3222%:%
%:%7456=3222%:%
%:%7457=3223%:%
%:%7458=3223%:%
%:%7459=3224%:%
%:%7460=3224%:%
%:%7461=3225%:%
%:%7462=3226%:%
%:%7463=3226%:%
%:%7464=3226%:%
%:%7465=3227%:%
%:%7466=3227%:%
%:%7467=3227%:%
%:%7468=3227%:%
%:%7469=3228%:%
%:%7470=3228%:%
%:%7471=3229%:%
%:%7472=3230%:%
%:%7473=3231%:%
%:%7474=3231%:%
%:%7475=3231%:%
%:%7476=3231%:%
%:%7477=3232%:%
%:%7478=3232%:%
%:%7479=3232%:%
%:%7480=3232%:%
%:%7481=3232%:%
%:%7482=3233%:%
%:%7483=3233%:%
%:%7484=3234%:%
%:%7485=3234%:%
%:%7486=3235%:%
%:%7487=3235%:%
%:%7488=3236%:%
%:%7489=3236%:%
%:%7490=3237%:%
%:%7491=3237%:%
%:%7492=3238%:%
%:%7493=3238%:%
%:%7494=3239%:%
%:%7495=3239%:%
%:%7496=3239%:%
%:%7497=3239%:%
%:%7498=3240%:%
%:%7499=3240%:%
%:%7500=3241%:%
%:%7501=3241%:%
%:%7502=3242%:%
%:%7503=3242%:%
%:%7504=3243%:%
%:%7505=3243%:%
%:%7506=3244%:%
%:%7507=3244%:%
%:%7508=3245%:%
%:%7509=3246%:%
%:%7510=3246%:%
%:%7511=3246%:%
%:%7512=3247%:%
%:%7513=3247%:%
%:%7514=3247%:%
%:%7515=3247%:%
%:%7516=3248%:%
%:%7517=3248%:%
%:%7518=3249%:%
%:%7519=3250%:%
%:%7520=3251%:%
%:%7521=3251%:%
%:%7522=3251%:%
%:%7523=3251%:%
%:%7524=3252%:%
%:%7525=3252%:%
%:%7526=3252%:%
%:%7527=3252%:%
%:%7528=3252%:%
%:%7529=3253%:%
%:%7530=3253%:%
%:%7531=3254%:%
%:%7532=3254%:%
%:%7533=3255%:%
%:%7534=3255%:%
%:%7535=3256%:%
%:%7536=3256%:%
%:%7537=3257%:%
%:%7538=3257%:%
%:%7539=3258%:%
%:%7540=3258%:%
%:%7541=3259%:%
%:%7542=3260%:%
%:%7543=3260%:%
%:%7544=3260%:%
%:%7545=3261%:%
%:%7546=3261%:%
%:%7547=3261%:%
%:%7548=3262%:%
%:%7549=3262%:%
%:%7550=3262%:%
%:%7551=3263%:%
%:%7552=3263%:%
%:%7553=3263%:%
%:%7554=3264%:%
%:%7555=3264%:%
%:%7556=3265%:%
%:%7557=3266%:%
%:%7558=3266%:%
%:%7559=3266%:%
%:%7560=3266%:%
%:%7561=3267%:%
%:%7562=3267%:%
%:%7563=3267%:%
%:%7564=3267%:%
%:%7565=3267%:%
%:%7566=3268%:%
%:%7567=3268%:%
%:%7568=3269%:%
%:%7569=3269%:%
%:%7570=3270%:%
%:%7571=3270%:%
%:%7572=3271%:%
%:%7573=3271%:%
%:%7574=3272%:%
%:%7575=3272%:%
%:%7576=3273%:%
%:%7577=3273%:%
%:%7578=3274%:%
%:%7579=3274%:%
%:%7580=3274%:%
%:%7581=3275%:%
%:%7582=3275%:%
%:%7583=3276%:%
%:%7584=3276%:%
%:%7585=3277%:%
%:%7586=3277%:%
%:%7587=3277%:%
%:%7588=3278%:%
%:%7589=3278%:%
%:%7590=3279%:%
%:%7591=3279%:%
%:%7592=3280%:%
%:%7593=3280%:%
%:%7594=3281%:%
%:%7595=3281%:%
%:%7596=3282%:%
%:%7597=3282%:%
%:%7598=3283%:%
%:%7599=3284%:%
%:%7600=3284%:%
%:%7601=3285%:%
%:%7602=3285%:%
%:%7603=3286%:%
%:%7604=3286%:%
%:%7605=3287%:%
%:%7606=3287%:%
%:%7607=3287%:%
%:%7608=3288%:%
%:%7609=3288%:%
%:%7610=3289%:%
%:%7611=3289%:%
%:%7612=3289%:%
%:%7613=3290%:%
%:%7614=3290%:%
%:%7615=3291%:%
%:%7616=3291%:%
%:%7617=3292%:%
%:%7618=3292%:%
%:%7619=3292%:%
%:%7620=3292%:%
%:%7621=3293%:%
%:%7622=3293%:%
%:%7623=3294%:%
%:%7624=3294%:%
%:%7625=3295%:%
%:%7626=3295%:%
%:%7627=3296%:%
%:%7628=3296%:%
%:%7629=3297%:%
%:%7630=3297%:%
%:%7631=3297%:%
%:%7632=3297%:%
%:%7633=3298%:%
%:%7634=3298%:%
%:%7635=3299%:%
%:%7636=3299%:%
%:%7637=3300%:%
%:%7638=3300%:%
%:%7639=3301%:%
%:%7640=3301%:%
%:%7641=3302%:%
%:%7642=3302%:%
%:%7643=3303%:%
%:%7644=3304%:%
%:%7645=3304%:%
%:%7646=3304%:%
%:%7647=3305%:%
%:%7648=3305%:%
%:%7649=3305%:%
%:%7650=3305%:%
%:%7651=3305%:%
%:%7652=3306%:%
%:%7653=3306%:%
%:%7654=3306%:%
%:%7655=3306%:%
%:%7656=3307%:%
%:%7657=3307%:%
%:%7658=3307%:%
%:%7659=3307%:%
%:%7660=3307%:%
%:%7661=3308%:%
%:%7662=3308%:%
%:%7663=3309%:%
%:%7664=3309%:%
%:%7665=3310%:%
%:%7666=3310%:%
%:%7667=3311%:%
%:%7668=3311%:%
%:%7669=3312%:%
%:%7670=3312%:%
%:%7671=3313%:%
%:%7672=3313%:%
%:%7673=3314%:%
%:%7674=3315%:%
%:%7675=3315%:%
%:%7676=3315%:%
%:%7677=3316%:%
%:%7678=3316%:%
%:%7679=3316%:%
%:%7680=3316%:%
%:%7681=3316%:%
%:%7682=3317%:%
%:%7683=3317%:%
%:%7684=3317%:%
%:%7685=3317%:%
%:%7686=3318%:%
%:%7687=3318%:%
%:%7688=3318%:%
%:%7689=3318%:%
%:%7690=3318%:%
%:%7691=3319%:%
%:%7692=3319%:%
%:%7693=3320%:%
%:%7694=3320%:%
%:%7695=3321%:%
%:%7696=3321%:%
%:%7697=3322%:%
%:%7698=3322%:%
%:%7699=3323%:%
%:%7700=3323%:%
%:%7701=3324%:%
%:%7702=3324%:%
%:%7703=3325%:%
%:%7704=3325%:%
%:%7705=3325%:%
%:%7706=3325%:%
%:%7707=3326%:%
%:%7708=3326%:%
%:%7709=3327%:%
%:%7710=3327%:%
%:%7711=3328%:%
%:%7712=3328%:%
%:%7713=3329%:%
%:%7714=3329%:%
%:%7715=3330%:%
%:%7716=3330%:%
%:%7717=3331%:%
%:%7718=3332%:%
%:%7719=3332%:%
%:%7720=3332%:%
%:%7721=3333%:%
%:%7722=3333%:%
%:%7723=3333%:%
%:%7724=3334%:%
%:%7725=3334%:%
%:%7726=3334%:%
%:%7727=3335%:%
%:%7728=3335%:%
%:%7729=3335%:%
%:%7730=3335%:%
%:%7731=3335%:%
%:%7732=3336%:%
%:%7733=3336%:%
%:%7734=3337%:%
%:%7735=3337%:%
%:%7736=3338%:%
%:%7737=3338%:%
%:%7738=3339%:%
%:%7739=3339%:%
%:%7740=3340%:%
%:%7741=3340%:%
%:%7742=3341%:%
%:%7743=3341%:%
%:%7744=3342%:%
%:%7745=3343%:%
%:%7746=3343%:%
%:%7747=3343%:%
%:%7748=3344%:%
%:%7749=3344%:%
%:%7750=3344%:%
%:%7751=3345%:%
%:%7752=3345%:%
%:%7753=3345%:%
%:%7754=3346%:%
%:%7755=3346%:%
%:%7756=3346%:%
%:%7757=3347%:%
%:%7758=3347%:%
%:%7759=3348%:%
%:%7760=3348%:%
%:%7761=3348%:%
%:%7762=3348%:%
%:%7763=3349%:%
%:%7764=3349%:%
%:%7765=3349%:%
%:%7766=3349%:%
%:%7767=3349%:%
%:%7768=3350%:%
%:%7769=3350%:%
%:%7770=3351%:%
%:%7771=3351%:%
%:%7772=3352%:%
%:%7773=3352%:%
%:%7774=3353%:%
%:%7775=3353%:%
%:%7776=3354%:%
%:%7777=3354%:%
%:%7778=3355%:%
%:%7779=3355%:%
%:%7780=3356%:%
%:%7781=3356%:%
%:%7782=3357%:%
%:%7783=3357%:%
%:%7784=3358%:%
%:%7785=3358%:%
%:%7786=3359%:%
%:%7787=3359%:%
%:%7788=3360%:%
%:%7789=3360%:%
%:%7790=3361%:%
%:%7791=3361%:%
%:%7792=3362%:%
%:%7793=3362%:%
%:%7794=3362%:%
%:%7795=3362%:%
%:%7796=3363%:%
%:%7797=3363%:%
%:%7798=3364%:%
%:%7804=3364%:%
%:%7807=3365%:%
%:%7808=3366%:%
%:%7809=3366%:%
%:%7810=3367%:%
%:%7817=3368%:%
%:%7818=3368%:%
%:%7819=3369%:%
%:%7820=3369%:%
%:%7821=3370%:%
%:%7822=3370%:%
%:%7823=3371%:%
%:%7824=3371%:%
%:%7825=3372%:%
%:%7826=3372%:%
%:%7827=3372%:%
%:%7828=3373%:%
%:%7829=3373%:%
%:%7830=3374%:%
%:%7831=3374%:%
%:%7832=3375%:%
%:%7833=3375%:%
%:%7834=3376%:%
%:%7835=3377%:%
%:%7836=3377%:%
%:%7837=3378%:%
%:%7838=3378%:%
%:%7839=3378%:%
%:%7840=3379%:%
%:%7841=3379%:%
%:%7842=3380%:%
%:%7843=3380%:%
%:%7844=3381%:%
%:%7845=3381%:%
%:%7846=3381%:%
%:%7847=3382%:%
%:%7848=3382%:%
%:%7849=3382%:%
%:%7850=3383%:%
%:%7851=3383%:%
%:%7852=3383%:%
%:%7853=3384%:%
%:%7854=3384%:%
%:%7855=3385%:%
%:%7856=3385%:%
%:%7857=3385%:%
%:%7858=3386%:%
%:%7859=3386%:%
%:%7860=3386%:%
%:%7861=3387%:%
%:%7862=3387%:%
%:%7863=3387%:%
%:%7864=3388%:%
%:%7865=3388%:%
%:%7866=3389%:%
%:%7867=3389%:%
%:%7868=3389%:%
%:%7869=3390%:%
%:%7870=3390%:%
%:%7871=3391%:%
%:%7872=3391%:%
%:%7873=3391%:%
%:%7874=3392%:%
%:%7875=3392%:%
%:%7876=3392%:%
%:%7877=3393%:%
%:%7878=3393%:%
%:%7879=3394%:%
%:%7880=3394%:%
%:%7881=3394%:%
%:%7882=3395%:%
%:%7883=3395%:%
%:%7884=3395%:%
%:%7885=3396%:%
%:%7886=3396%:%
%:%7887=3396%:%
%:%7888=3397%:%
%:%7889=3397%:%
%:%7890=3397%:%
%:%7891=3398%:%
%:%7892=3398%:%
%:%7893=3399%:%
%:%7894=3399%:%
%:%7895=3399%:%
%:%7896=3400%:%
%:%7897=3400%:%
%:%7898=3401%:%
%:%7899=3401%:%
%:%7900=3401%:%
%:%7901=3402%:%
%:%7902=3402%:%
%:%7903=3402%:%
%:%7904=3403%:%
%:%7905=3403%:%
%:%7906=3404%:%
%:%7907=3404%:%
%:%7908=3404%:%
%:%7909=3405%:%
%:%7910=3405%:%
%:%7911=3405%:%
%:%7912=3405%:%
%:%7913=3406%:%
%:%7914=3406%:%
%:%7915=3406%:%
%:%7916=3407%:%
%:%7917=3407%:%
%:%7918=3407%:%
%:%7919=3408%:%
%:%7920=3408%:%
%:%7921=3408%:%
%:%7922=3409%:%
%:%7923=3409%:%
%:%7924=3410%:%
%:%7925=3410%:%
%:%7926=3411%:%
%:%7927=3411%:%
%:%7928=3411%:%
%:%7929=3411%:%
%:%7930=3412%:%
%:%7931=3412%:%
%:%7932=3413%:%
%:%7933=3413%:%
%:%7934=3413%:%
%:%7935=3413%:%
%:%7936=3414%:%
%:%7942=3414%:%
%:%7945=3415%:%
%:%7946=3416%:%
%:%7947=3416%:%
%:%7948=3417%:%
%:%7951=3418%:%
%:%7955=3418%:%
%:%7956=3418%:%
%:%7957=3419%:%
%:%7958=3419%:%
%:%7959=3420%:%
%:%7960=3420%:%
%:%7965=3420%:%
%:%7968=3421%:%
%:%7969=3422%:%
%:%7970=3422%:%
%:%7971=3423%:%
%:%7974=3424%:%
%:%7978=3424%:%
%:%7979=3424%:%
%:%7980=3425%:%
%:%7981=3425%:%
%:%7982=3426%:%
%:%7983=3426%:%
%:%7988=3426%:%
%:%7991=3427%:%
%:%7992=3428%:%
%:%7993=3428%:%
%:%7994=3429%:%
%:%8001=3430%:%
%:%8002=3430%:%
%:%8003=3431%:%
%:%8004=3431%:%
%:%8005=3431%:%
%:%8006=3431%:%
%:%8007=3432%:%
%:%8008=3432%:%
%:%8009=3433%:%
%:%8010=3433%:%
%:%8011=3433%:%
%:%8012=3433%:%
%:%8013=3434%:%
%:%8014=3434%:%
%:%8015=3435%:%
%:%8016=3435%:%
%:%8017=3436%:%
%:%8018=3436%:%
%:%8019=3436%:%
%:%8020=3437%:%
%:%8026=3437%:%
%:%8029=3438%:%
%:%8030=3439%:%
%:%8031=3440%:%
%:%8032=3440%:%
%:%8033=3441%:%
%:%8034=3442%:%
%:%8041=3443%:%
%:%8042=3443%:%
%:%8043=3444%:%
%:%8044=3444%:%
%:%8045=3445%:%
%:%8046=3446%:%
%:%8047=3446%:%
%:%8048=3447%:%
%:%8049=3447%:%
%:%8050=3448%:%
%:%8051=3448%:%
%:%8052=3449%:%
%:%8053=3449%:%
%:%8054=3449%:%
%:%8055=3450%:%
%:%8056=3450%:%
%:%8057=3451%:%
%:%8058=3451%:%
%:%8059=3451%:%
%:%8060=3452%:%
%:%8061=3452%:%
%:%8062=3453%:%
%:%8063=3453%:%
%:%8064=3454%:%
%:%8065=3454%:%
%:%8066=3454%:%
%:%8067=3454%:%
%:%8068=3455%:%
%:%8069=3455%:%
%:%8070=3456%:%
%:%8071=3456%:%
%:%8072=3457%:%
%:%8073=3457%:%
%:%8074=3458%:%
%:%8075=3458%:%
%:%8076=3459%:%
%:%8077=3459%:%
%:%8078=3459%:%
%:%8079=3459%:%
%:%8080=3460%:%
%:%8081=3460%:%
%:%8082=3461%:%
%:%8083=3461%:%
%:%8084=3462%:%
%:%8085=3462%:%
%:%8086=3463%:%
%:%8087=3463%:%
%:%8088=3464%:%
%:%8089=3464%:%
%:%8093=3468%:%
%:%8094=3469%:%
%:%8095=3469%:%
%:%8096=3470%:%
%:%8097=3470%:%
%:%8098=3471%:%
%:%8099=3472%:%
%:%8100=3472%:%
%:%8101=3472%:%
%:%8102=3473%:%
%:%8103=3473%:%
%:%8104=3473%:%
%:%8105=3474%:%
%:%8106=3474%:%
%:%8107=3474%:%
%:%8108=3475%:%
%:%8109=3475%:%
%:%8110=3476%:%
%:%8111=3476%:%
%:%8112=3477%:%
%:%8113=3478%:%
%:%8114=3478%:%
%:%8115=3478%:%
%:%8116=3478%:%
%:%8117=3478%:%
%:%8118=3479%:%
%:%8119=3479%:%
%:%8120=3479%:%
%:%8121=3479%:%
%:%8122=3480%:%
%:%8123=3480%:%
%:%8124=3480%:%
%:%8125=3480%:%
%:%8126=3480%:%
%:%8127=3481%:%
%:%8128=3481%:%
%:%8129=3482%:%
%:%8130=3482%:%
%:%8131=3483%:%
%:%8132=3483%:%
%:%8133=3484%:%
%:%8134=3484%:%
%:%8135=3485%:%
%:%8136=3485%:%
%:%8137=3486%:%
%:%8138=3486%:%
%:%8139=3486%:%
%:%8140=3486%:%
%:%8141=3487%:%
%:%8142=3487%:%
%:%8143=3488%:%
%:%8144=3488%:%
%:%8145=3489%:%
%:%8146=3489%:%
%:%8147=3490%:%
%:%8148=3490%:%
%:%8149=3491%:%
%:%8150=3491%:%
%:%8154=3495%:%
%:%8155=3496%:%
%:%8156=3496%:%
%:%8157=3497%:%
%:%8158=3497%:%
%:%8159=3498%:%
%:%8160=3499%:%
%:%8161=3499%:%
%:%8162=3499%:%
%:%8163=3500%:%
%:%8164=3500%:%
%:%8165=3500%:%
%:%8166=3501%:%
%:%8167=3501%:%
%:%8168=3501%:%
%:%8169=3502%:%
%:%8170=3502%:%
%:%8171=3503%:%
%:%8172=3503%:%
%:%8173=3504%:%
%:%8174=3505%:%
%:%8175=3506%:%
%:%8176=3506%:%
%:%8177=3506%:%
%:%8178=3506%:%
%:%8179=3506%:%
%:%8180=3507%:%
%:%8181=3507%:%
%:%8182=3507%:%
%:%8183=3507%:%
%:%8184=3508%:%
%:%8185=3508%:%
%:%8186=3508%:%
%:%8187=3508%:%
%:%8188=3508%:%
%:%8189=3509%:%
%:%8190=3509%:%
%:%8191=3510%:%
%:%8192=3510%:%
%:%8193=3511%:%
%:%8194=3511%:%
%:%8195=3512%:%
%:%8196=3512%:%
%:%8197=3513%:%
%:%8198=3513%:%
%:%8199=3514%:%
%:%8200=3514%:%
%:%8201=3514%:%
%:%8202=3514%:%
%:%8203=3515%:%
%:%8204=3515%:%
%:%8205=3516%:%
%:%8206=3516%:%
%:%8207=3517%:%
%:%8208=3517%:%
%:%8209=3518%:%
%:%8210=3518%:%
%:%8211=3519%:%
%:%8212=3519%:%
%:%8216=3523%:%
%:%8217=3524%:%
%:%8218=3524%:%
%:%8219=3525%:%
%:%8220=3525%:%
%:%8221=3526%:%
%:%8222=3527%:%
%:%8223=3527%:%
%:%8224=3527%:%
%:%8225=3528%:%
%:%8226=3528%:%
%:%8227=3528%:%
%:%8228=3529%:%
%:%8229=3529%:%
%:%8230=3529%:%
%:%8231=3530%:%
%:%8232=3530%:%
%:%8233=3531%:%
%:%8234=3531%:%
%:%8235=3532%:%
%:%8236=3533%:%
%:%8237=3533%:%
%:%8238=3533%:%
%:%8239=3533%:%
%:%8240=3533%:%
%:%8241=3534%:%
%:%8242=3534%:%
%:%8243=3534%:%
%:%8244=3534%:%
%:%8245=3535%:%
%:%8246=3535%:%
%:%8247=3535%:%
%:%8248=3535%:%
%:%8249=3535%:%
%:%8250=3536%:%
%:%8251=3536%:%
%:%8252=3537%:%
%:%8253=3537%:%
%:%8254=3538%:%
%:%8255=3538%:%
%:%8256=3539%:%
%:%8257=3539%:%
%:%8258=3540%:%
%:%8259=3540%:%
%:%8260=3541%:%
%:%8261=3541%:%
%:%8265=3545%:%
%:%8266=3546%:%
%:%8267=3546%:%
%:%8268=3547%:%
%:%8269=3547%:%
%:%8270=3548%:%
%:%8271=3549%:%
%:%8272=3549%:%
%:%8273=3549%:%
%:%8274=3550%:%
%:%8275=3550%:%
%:%8276=3550%:%
%:%8277=3551%:%
%:%8278=3551%:%
%:%8279=3551%:%
%:%8280=3552%:%
%:%8281=3552%:%
%:%8282=3553%:%
%:%8283=3553%:%
%:%8284=3554%:%
%:%8285=3555%:%
%:%8286=3555%:%
%:%8287=3555%:%
%:%8288=3555%:%
%:%8289=3555%:%
%:%8290=3556%:%
%:%8291=3556%:%
%:%8292=3556%:%
%:%8293=3556%:%
%:%8294=3557%:%
%:%8295=3557%:%
%:%8296=3557%:%
%:%8297=3557%:%
%:%8298=3557%:%
%:%8299=3558%:%
%:%8300=3558%:%
%:%8301=3559%:%
%:%8302=3559%:%
%:%8303=3560%:%
%:%8304=3560%:%
%:%8305=3561%:%
%:%8306=3561%:%
%:%8307=3562%:%
%:%8308=3562%:%
%:%8309=3563%:%
%:%8310=3563%:%
%:%8311=3564%:%
%:%8312=3564%:%
%:%8313=3565%:%
%:%8314=3565%:%
%:%8315=3566%:%
%:%8316=3566%:%
%:%8317=3566%:%
%:%8318=3566%:%
%:%8319=3567%:%
%:%8320=3567%:%
%:%8321=3568%:%
%:%8322=3568%:%
%:%8323=3569%:%
%:%8324=3569%:%
%:%8325=3570%:%
%:%8326=3570%:%
%:%8327=3571%:%
%:%8328=3571%:%
%:%8329=3571%:%
%:%8330=3571%:%
%:%8331=3572%:%
%:%8332=3572%:%
%:%8333=3573%:%
%:%8334=3573%:%
%:%8335=3574%:%
%:%8336=3574%:%
%:%8337=3575%:%
%:%8338=3575%:%
%:%8339=3576%:%
%:%8340=3576%:%
%:%8344=3580%:%
%:%8345=3581%:%
%:%8346=3581%:%
%:%8347=3582%:%
%:%8348=3582%:%
%:%8349=3583%:%
%:%8350=3584%:%
%:%8351=3584%:%
%:%8352=3584%:%
%:%8353=3585%:%
%:%8354=3586%:%
%:%8355=3586%:%
%:%8356=3586%:%
%:%8357=3587%:%
%:%8358=3587%:%
%:%8359=3587%:%
%:%8360=3588%:%
%:%8361=3589%:%
%:%8362=3589%:%
%:%8363=3590%:%
%:%8364=3590%:%
%:%8365=3591%:%
%:%8366=3592%:%
%:%8367=3593%:%
%:%8368=3593%:%
%:%8369=3593%:%
%:%8370=3594%:%
%:%8371=3595%:%
%:%8372=3595%:%
%:%8373=3595%:%
%:%8374=3596%:%
%:%8375=3596%:%
%:%8376=3596%:%
%:%8377=3596%:%
%:%8378=3597%:%
%:%8379=3597%:%
%:%8380=3597%:%
%:%8381=3597%:%
%:%8382=3598%:%
%:%8383=3598%:%
%:%8384=3598%:%
%:%8385=3598%:%
%:%8386=3598%:%
%:%8387=3599%:%
%:%8388=3599%:%
%:%8389=3600%:%
%:%8390=3600%:%
%:%8391=3601%:%
%:%8392=3601%:%
%:%8393=3602%:%
%:%8394=3602%:%
%:%8395=3603%:%
%:%8396=3603%:%
%:%8397=3604%:%
%:%8398=3604%:%
%:%8399=3604%:%
%:%8400=3604%:%
%:%8401=3605%:%
%:%8402=3605%:%
%:%8403=3606%:%
%:%8404=3606%:%
%:%8405=3607%:%
%:%8406=3607%:%
%:%8407=3608%:%
%:%8408=3608%:%
%:%8409=3609%:%
%:%8410=3609%:%
%:%8414=3613%:%
%:%8415=3614%:%
%:%8416=3614%:%
%:%8417=3615%:%
%:%8418=3615%:%
%:%8419=3616%:%
%:%8420=3617%:%
%:%8421=3617%:%
%:%8422=3617%:%
%:%8423=3618%:%
%:%8424=3619%:%
%:%8425=3619%:%
%:%8426=3619%:%
%:%8427=3620%:%
%:%8428=3620%:%
%:%8429=3620%:%
%:%8430=3621%:%
%:%8431=3622%:%
%:%8432=3622%:%
%:%8433=3623%:%
%:%8434=3623%:%
%:%8435=3624%:%
%:%8436=3625%:%
%:%8437=3626%:%
%:%8438=3626%:%
%:%8439=3626%:%
%:%8440=3627%:%
%:%8441=3628%:%
%:%8442=3628%:%
%:%8443=3628%:%
%:%8444=3629%:%
%:%8445=3629%:%
%:%8446=3629%:%
%:%8447=3630%:%
%:%8448=3631%:%
%:%8449=3631%:%
%:%8450=3631%:%
%:%8451=3632%:%
%:%8452=3632%:%
%:%8453=3632%:%
%:%8454=3633%:%
%:%8455=3633%:%
%:%8456=3634%:%
%:%8457=3634%:%
%:%8458=3635%:%
%:%8459=3636%:%
%:%8460=3636%:%
%:%8461=3636%:%
%:%8462=3636%:%
%:%8463=3636%:%
%:%8464=3637%:%
%:%8465=3637%:%
%:%8466=3637%:%
%:%8467=3637%:%
%:%8468=3638%:%
%:%8469=3638%:%
%:%8470=3638%:%
%:%8471=3638%:%
%:%8472=3639%:%
%:%8473=3639%:%
%:%8474=3639%:%
%:%8475=3639%:%
%:%8476=3639%:%
%:%8477=3640%:%
%:%8478=3640%:%
%:%8479=3641%:%
%:%8480=3641%:%
%:%8481=3642%:%
%:%8482=3642%:%
%:%8483=3643%:%
%:%8484=3643%:%
%:%8485=3644%:%
%:%8486=3644%:%
%:%8487=3645%:%
%:%8488=3645%:%
%:%8489=3645%:%
%:%8490=3645%:%
%:%8491=3646%:%
%:%8492=3646%:%
%:%8493=3647%:%
%:%8494=3647%:%
%:%8495=3648%:%
%:%8496=3648%:%
%:%8497=3649%:%
%:%8498=3649%:%
%:%8499=3650%:%
%:%8500=3650%:%
%:%8504=3654%:%
%:%8505=3655%:%
%:%8506=3655%:%
%:%8507=3656%:%
%:%8508=3656%:%
%:%8509=3657%:%
%:%8510=3658%:%
%:%8511=3658%:%
%:%8512=3658%:%
%:%8513=3659%:%
%:%8514=3660%:%
%:%8515=3660%:%
%:%8516=3660%:%
%:%8517=3661%:%
%:%8518=3661%:%
%:%8519=3661%:%
%:%8520=3662%:%
%:%8521=3663%:%
%:%8522=3663%:%
%:%8523=3664%:%
%:%8524=3664%:%
%:%8525=3665%:%
%:%8526=3666%:%
%:%8527=3667%:%
%:%8528=3667%:%
%:%8529=3667%:%
%:%8530=3668%:%
%:%8531=3669%:%
%:%8532=3669%:%
%:%8533=3669%:%
%:%8534=3670%:%
%:%8535=3670%:%
%:%8536=3670%:%
%:%8537=3670%:%
%:%8538=3671%:%
%:%8539=3671%:%
%:%8540=3671%:%
%:%8541=3671%:%
%:%8542=3672%:%
%:%8543=3672%:%
%:%8544=3672%:%
%:%8545=3672%:%
%:%8546=3672%:%
%:%8547=3673%:%
%:%8548=3673%:%
%:%8549=3674%:%
%:%8550=3674%:%
%:%8551=3675%:%
%:%8552=3675%:%
%:%8553=3676%:%
%:%8554=3676%:%
%:%8555=3677%:%
%:%8556=3677%:%
%:%8557=3678%:%
%:%8558=3678%:%
%:%8562=3682%:%
%:%8563=3683%:%
%:%8564=3683%:%
%:%8565=3684%:%
%:%8566=3684%:%
%:%8567=3685%:%
%:%8568=3686%:%
%:%8569=3686%:%
%:%8570=3686%:%
%:%8571=3687%:%
%:%8572=3688%:%
%:%8573=3688%:%
%:%8574=3688%:%
%:%8575=3689%:%
%:%8576=3689%:%
%:%8577=3689%:%
%:%8578=3690%:%
%:%8579=3691%:%
%:%8580=3691%:%
%:%8581=3692%:%
%:%8582=3692%:%
%:%8583=3693%:%
%:%8584=3694%:%
%:%8585=3695%:%
%:%8586=3695%:%
%:%8587=3695%:%
%:%8588=3696%:%
%:%8589=3697%:%
%:%8590=3697%:%
%:%8591=3697%:%
%:%8592=3698%:%
%:%8593=3698%:%
%:%8594=3698%:%
%:%8595=3699%:%
%:%8596=3700%:%
%:%8597=3700%:%
%:%8598=3700%:%
%:%8599=3701%:%
%:%8600=3701%:%
%:%8601=3701%:%
%:%8602=3702%:%
%:%8603=3702%:%
%:%8604=3703%:%
%:%8605=3703%:%
%:%8606=3704%:%
%:%8607=3705%:%
%:%8608=3705%:%
%:%8609=3705%:%
%:%8610=3705%:%
%:%8611=3705%:%
%:%8612=3706%:%
%:%8613=3706%:%
%:%8614=3706%:%
%:%8615=3706%:%
%:%8616=3707%:%
%:%8617=3707%:%
%:%8618=3707%:%
%:%8619=3707%:%
%:%8620=3708%:%
%:%8621=3708%:%
%:%8622=3708%:%
%:%8623=3708%:%
%:%8624=3708%:%
%:%8625=3709%:%
%:%8626=3709%:%
%:%8627=3710%:%
%:%8628=3710%:%
%:%8629=3711%:%
%:%8630=3711%:%
%:%8631=3712%:%
%:%8632=3712%:%
%:%8633=3713%:%
%:%8634=3713%:%
%:%8635=3714%:%
%:%8636=3714%:%
%:%8637=3715%:%
%:%8638=3715%:%
%:%8639=3716%:%
%:%8640=3716%:%
%:%8641=3717%:%
%:%8642=3717%:%
%:%8643=3717%:%
%:%8644=3717%:%
%:%8645=3718%:%
%:%8646=3718%:%
%:%8647=3719%:%
%:%8648=3719%:%
%:%8649=3720%:%
%:%8650=3720%:%
%:%8651=3721%:%
%:%8652=3721%:%
%:%8653=3722%:%
%:%8654=3722%:%
%:%8655=3722%:%
%:%8656=3722%:%
%:%8657=3723%:%
%:%8658=3723%:%
%:%8659=3724%:%
%:%8660=3724%:%
%:%8661=3725%:%
%:%8662=3725%:%
%:%8663=3726%:%
%:%8664=3726%:%
%:%8665=3727%:%
%:%8666=3727%:%
%:%8670=3731%:%
%:%8671=3732%:%
%:%8672=3732%:%
%:%8673=3733%:%
%:%8674=3733%:%
%:%8675=3734%:%
%:%8676=3735%:%
%:%8677=3735%:%
%:%8678=3735%:%
%:%8679=3736%:%
%:%8680=3736%:%
%:%8681=3736%:%
%:%8682=3737%:%
%:%8683=3737%:%
%:%8684=3737%:%
%:%8685=3738%:%
%:%8686=3738%:%
%:%8687=3739%:%
%:%8688=3739%:%
%:%8689=3740%:%
%:%8690=3741%:%
%:%8691=3741%:%
%:%8692=3741%:%
%:%8693=3741%:%
%:%8694=3741%:%
%:%8695=3742%:%
%:%8696=3742%:%
%:%8697=3742%:%
%:%8698=3742%:%
%:%8699=3743%:%
%:%8700=3743%:%
%:%8701=3743%:%
%:%8702=3743%:%
%:%8703=3743%:%
%:%8704=3744%:%
%:%8705=3744%:%
%:%8706=3745%:%
%:%8707=3745%:%
%:%8708=3746%:%
%:%8709=3746%:%
%:%8710=3747%:%
%:%8711=3747%:%
%:%8712=3748%:%
%:%8713=3748%:%
%:%8714=3749%:%
%:%8715=3749%:%
%:%8716=3749%:%
%:%8717=3749%:%
%:%8718=3750%:%
%:%8719=3750%:%
%:%8720=3751%:%
%:%8721=3751%:%
%:%8722=3752%:%
%:%8723=3752%:%
%:%8724=3753%:%
%:%8725=3753%:%
%:%8726=3754%:%
%:%8727=3754%:%
%:%8731=3758%:%
%:%8732=3759%:%
%:%8733=3759%:%
%:%8734=3760%:%
%:%8735=3760%:%
%:%8736=3761%:%
%:%8737=3762%:%
%:%8738=3762%:%
%:%8739=3762%:%
%:%8740=3763%:%
%:%8741=3763%:%
%:%8742=3763%:%
%:%8743=3764%:%
%:%8744=3764%:%
%:%8745=3764%:%
%:%8746=3765%:%
%:%8747=3765%:%
%:%8748=3766%:%
%:%8749=3766%:%
%:%8750=3767%:%
%:%8751=3768%:%
%:%8752=3768%:%
%:%8753=3768%:%
%:%8754=3768%:%
%:%8755=3768%:%
%:%8756=3769%:%
%:%8757=3769%:%
%:%8758=3769%:%
%:%8759=3769%:%
%:%8760=3770%:%
%:%8761=3770%:%
%:%8762=3770%:%
%:%8763=3770%:%
%:%8764=3770%:%
%:%8765=3771%:%
%:%8766=3771%:%
%:%8767=3772%:%
%:%8768=3772%:%
%:%8769=3773%:%
%:%8770=3773%:%
%:%8771=3774%:%
%:%8772=3774%:%
%:%8773=3775%:%
%:%8774=3775%:%
%:%8775=3776%:%
%:%8776=3776%:%
%:%8777=3776%:%
%:%8778=3776%:%
%:%8779=3777%:%
%:%8780=3777%:%
%:%8781=3778%:%
%:%8782=3778%:%
%:%8783=3779%:%
%:%8784=3779%:%
%:%8785=3780%:%
%:%8786=3780%:%
%:%8787=3781%:%
%:%8788=3781%:%
%:%8792=3785%:%
%:%8793=3786%:%
%:%8794=3786%:%
%:%8795=3787%:%
%:%8796=3787%:%
%:%8797=3788%:%
%:%8798=3789%:%
%:%8799=3789%:%
%:%8800=3789%:%
%:%8801=3790%:%
%:%8802=3790%:%
%:%8803=3790%:%
%:%8804=3791%:%
%:%8805=3791%:%
%:%8806=3791%:%
%:%8807=3792%:%
%:%8808=3792%:%
%:%8809=3793%:%
%:%8810=3793%:%
%:%8811=3794%:%
%:%8812=3795%:%
%:%8813=3795%:%
%:%8814=3795%:%
%:%8815=3795%:%
%:%8816=3795%:%
%:%8817=3796%:%
%:%8818=3796%:%
%:%8819=3796%:%
%:%8820=3796%:%
%:%8821=3797%:%
%:%8822=3797%:%
%:%8823=3797%:%
%:%8824=3797%:%
%:%8825=3797%:%
%:%8826=3798%:%
%:%8827=3798%:%
%:%8828=3799%:%
%:%8829=3799%:%
%:%8830=3800%:%
%:%8831=3800%:%
%:%8832=3801%:%
%:%8833=3801%:%
%:%8834=3802%:%
%:%8835=3802%:%
%:%8836=3803%:%
%:%8837=3803%:%
%:%8841=3807%:%
%:%8842=3808%:%
%:%8843=3808%:%
%:%8844=3809%:%
%:%8845=3809%:%
%:%8846=3810%:%
%:%8847=3811%:%
%:%8848=3811%:%
%:%8849=3811%:%
%:%8850=3812%:%
%:%8851=3812%:%
%:%8852=3812%:%
%:%8853=3813%:%
%:%8854=3813%:%
%:%8855=3813%:%
%:%8856=3814%:%
%:%8857=3814%:%
%:%8858=3815%:%
%:%8859=3815%:%
%:%8860=3816%:%
%:%8861=3817%:%
%:%8862=3817%:%
%:%8863=3817%:%
%:%8864=3817%:%
%:%8865=3817%:%
%:%8866=3818%:%
%:%8867=3818%:%
%:%8868=3818%:%
%:%8869=3818%:%
%:%8870=3819%:%
%:%8871=3819%:%
%:%8872=3819%:%
%:%8873=3819%:%
%:%8874=3819%:%
%:%8875=3820%:%
%:%8876=3820%:%
%:%8877=3821%:%
%:%8878=3821%:%
%:%8879=3822%:%
%:%8880=3822%:%
%:%8881=3823%:%
%:%8882=3823%:%
%:%8883=3824%:%
%:%8884=3824%:%
%:%8885=3825%:%
%:%8886=3825%:%
%:%8887=3826%:%
%:%8888=3826%:%
%:%8889=3827%:%
%:%8890=3827%:%
%:%8891=3828%:%
%:%8892=3828%:%
%:%8893=3828%:%
%:%8894=3828%:%
%:%8895=3829%:%
%:%8896=3829%:%
%:%8897=3830%:%
%:%8898=3830%:%
%:%8899=3831%:%
%:%8900=3831%:%
%:%8901=3832%:%
%:%8902=3832%:%
%:%8903=3833%:%
%:%8904=3833%:%
%:%8908=3837%:%
%:%8909=3838%:%
%:%8910=3838%:%
%:%8911=3839%:%
%:%8912=3839%:%
%:%8913=3840%:%
%:%8914=3841%:%
%:%8915=3841%:%
%:%8916=3841%:%
%:%8917=3842%:%
%:%8918=3843%:%
%:%8919=3843%:%
%:%8920=3843%:%
%:%8921=3844%:%
%:%8922=3844%:%
%:%8923=3844%:%
%:%8924=3845%:%
%:%8925=3846%:%
%:%8926=3846%:%
%:%8927=3846%:%
%:%8928=3847%:%
%:%8929=3847%:%
%:%8930=3847%:%
%:%8931=3848%:%
%:%8932=3849%:%
%:%8933=3849%:%
%:%8934=3849%:%
%:%8935=3850%:%
%:%8936=3850%:%
%:%8937=3850%:%
%:%8938=3850%:%
%:%8939=3851%:%
%:%8940=3851%:%
%:%8941=3851%:%
%:%8942=3851%:%
%:%8943=3852%:%
%:%8944=3852%:%
%:%8945=3852%:%
%:%8946=3852%:%
%:%8947=3852%:%
%:%8948=3853%:%
%:%8949=3853%:%
%:%8950=3854%:%
%:%8951=3854%:%
%:%8952=3855%:%
%:%8953=3855%:%
%:%8954=3856%:%
%:%8955=3856%:%
%:%8956=3857%:%
%:%8957=3857%:%
%:%8958=3858%:%
%:%8959=3858%:%
%:%8960=3858%:%
%:%8961=3858%:%
%:%8962=3859%:%
%:%8963=3859%:%
%:%8964=3860%:%
%:%8965=3860%:%
%:%8966=3861%:%
%:%8967=3861%:%
%:%8968=3862%:%
%:%8969=3862%:%
%:%8970=3863%:%
%:%8971=3863%:%
%:%8975=3867%:%
%:%8976=3868%:%
%:%8977=3868%:%
%:%8978=3869%:%
%:%8979=3869%:%
%:%8980=3870%:%
%:%8981=3871%:%
%:%8982=3871%:%
%:%8983=3871%:%
%:%8984=3872%:%
%:%8985=3873%:%
%:%8986=3873%:%
%:%8987=3873%:%
%:%8988=3874%:%
%:%8989=3874%:%
%:%8990=3874%:%
%:%8991=3875%:%
%:%8992=3876%:%
%:%8993=3876%:%
%:%8994=3876%:%
%:%8995=3877%:%
%:%8996=3877%:%
%:%8997=3877%:%
%:%8998=3878%:%
%:%8999=3879%:%
%:%9000=3879%:%
%:%9001=3879%:%
%:%9002=3880%:%
%:%9003=3880%:%
%:%9004=3880%:%
%:%9005=3881%:%
%:%9006=3882%:%
%:%9007=3882%:%
%:%9008=3882%:%
%:%9009=3883%:%
%:%9010=3883%:%
%:%9011=3883%:%
%:%9012=3884%:%
%:%9013=3884%:%
%:%9014=3885%:%
%:%9015=3885%:%
%:%9016=3886%:%
%:%9017=3887%:%
%:%9018=3887%:%
%:%9019=3887%:%
%:%9020=3887%:%
%:%9021=3887%:%
%:%9022=3888%:%
%:%9023=3888%:%
%:%9024=3888%:%
%:%9025=3888%:%
%:%9026=3889%:%
%:%9027=3889%:%
%:%9028=3889%:%
%:%9029=3889%:%
%:%9030=3890%:%
%:%9031=3890%:%
%:%9032=3890%:%
%:%9033=3890%:%
%:%9034=3890%:%
%:%9035=3891%:%
%:%9036=3891%:%
%:%9037=3892%:%
%:%9038=3892%:%
%:%9039=3893%:%
%:%9040=3893%:%
%:%9041=3894%:%
%:%9042=3894%:%
%:%9043=3895%:%
%:%9044=3895%:%
%:%9045=3896%:%
%:%9046=3896%:%
%:%9047=3896%:%
%:%9048=3896%:%
%:%9049=3897%:%
%:%9050=3897%:%
%:%9051=3898%:%
%:%9052=3898%:%
%:%9053=3899%:%
%:%9054=3899%:%
%:%9055=3900%:%
%:%9056=3900%:%
%:%9057=3901%:%
%:%9058=3901%:%
%:%9062=3905%:%
%:%9063=3906%:%
%:%9064=3906%:%
%:%9065=3907%:%
%:%9066=3907%:%
%:%9067=3908%:%
%:%9068=3909%:%
%:%9069=3909%:%
%:%9070=3909%:%
%:%9071=3910%:%
%:%9072=3911%:%
%:%9073=3911%:%
%:%9074=3911%:%
%:%9075=3912%:%
%:%9076=3912%:%
%:%9077=3912%:%
%:%9078=3913%:%
%:%9079=3914%:%
%:%9080=3914%:%
%:%9081=3914%:%
%:%9082=3915%:%
%:%9083=3915%:%
%:%9084=3915%:%
%:%9085=3916%:%
%:%9086=3917%:%
%:%9087=3917%:%
%:%9088=3917%:%
%:%9089=3918%:%
%:%9090=3918%:%
%:%9091=3918%:%
%:%9092=3918%:%
%:%9093=3919%:%
%:%9094=3919%:%
%:%9095=3919%:%
%:%9096=3919%:%
%:%9097=3920%:%
%:%9098=3920%:%
%:%9099=3920%:%
%:%9100=3920%:%
%:%9101=3920%:%
%:%9102=3921%:%
%:%9103=3921%:%
%:%9104=3922%:%
%:%9105=3922%:%
%:%9106=3923%:%
%:%9107=3923%:%
%:%9108=3924%:%
%:%9109=3924%:%
%:%9110=3925%:%
%:%9111=3925%:%
%:%9112=3926%:%
%:%9113=3926%:%
%:%9117=3930%:%
%:%9118=3931%:%
%:%9119=3931%:%
%:%9120=3932%:%
%:%9121=3932%:%
%:%9122=3933%:%
%:%9123=3934%:%
%:%9124=3934%:%
%:%9125=3934%:%
%:%9126=3935%:%
%:%9127=3936%:%
%:%9128=3936%:%
%:%9129=3936%:%
%:%9130=3937%:%
%:%9131=3937%:%
%:%9132=3937%:%
%:%9133=3938%:%
%:%9134=3939%:%
%:%9135=3939%:%
%:%9136=3939%:%
%:%9137=3940%:%
%:%9138=3940%:%
%:%9139=3940%:%
%:%9140=3941%:%
%:%9141=3942%:%
%:%9142=3942%:%
%:%9143=3942%:%
%:%9144=3943%:%
%:%9145=3943%:%
%:%9146=3943%:%
%:%9147=3944%:%
%:%9148=3945%:%
%:%9149=3945%:%
%:%9150=3945%:%
%:%9151=3946%:%
%:%9152=3946%:%
%:%9153=3946%:%
%:%9154=3947%:%
%:%9155=3947%:%
%:%9156=3948%:%
%:%9157=3948%:%
%:%9158=3949%:%
%:%9159=3950%:%
%:%9160=3950%:%
%:%9161=3950%:%
%:%9162=3950%:%
%:%9163=3950%:%
%:%9164=3951%:%
%:%9165=3951%:%
%:%9166=3951%:%
%:%9167=3951%:%
%:%9168=3952%:%
%:%9169=3952%:%
%:%9170=3952%:%
%:%9171=3952%:%
%:%9172=3953%:%
%:%9173=3953%:%
%:%9174=3953%:%
%:%9175=3953%:%
%:%9176=3953%:%
%:%9177=3954%:%
%:%9178=3954%:%
%:%9179=3955%:%
%:%9180=3955%:%
%:%9181=3956%:%
%:%9182=3956%:%
%:%9183=3957%:%
%:%9184=3957%:%
%:%9185=3958%:%
%:%9186=3958%:%
%:%9187=3959%:%
%:%9188=3959%:%
%:%9189=3960%:%
%:%9190=3960%:%
%:%9191=3961%:%
%:%9192=3961%:%
%:%9193=3962%:%
%:%9194=3962%:%
%:%9195=3963%:%
%:%9196=3963%:%
%:%9197=3964%:%
%:%9198=3964%:%
%:%9199=3965%:%
%:%9200=3965%:%
%:%9201=3966%:%
%:%9202=3966%:%
%:%9203=3967%:%
%:%9204=3967%:%
%:%9205=3968%:%
%:%9206=3969%:%
%:%9212=3969%:%
%:%9215=3970%:%
%:%9216=3971%:%
%:%9217=3971%:%
%:%9218=3972%:%
%:%9219=3973%:%
%:%9226=3974%:%
%:%9227=3974%:%
%:%9228=3975%:%
%:%9229=3975%:%
%:%9230=3976%:%
%:%9231=3977%:%
%:%9232=3977%:%
%:%9233=3978%:%
%:%9234=3978%:%
%:%9235=3979%:%
%:%9236=3979%:%
%:%9237=3980%:%
%:%9238=3980%:%
%:%9239=3980%:%
%:%9240=3981%:%
%:%9241=3981%:%
%:%9242=3982%:%
%:%9243=3982%:%
%:%9244=3982%:%
%:%9245=3983%:%
%:%9246=3983%:%
%:%9247=3984%:%
%:%9248=3984%:%
%:%9249=3985%:%
%:%9250=3985%:%
%:%9251=3985%:%
%:%9252=3985%:%
%:%9253=3986%:%
%:%9254=3986%:%
%:%9255=3987%:%
%:%9256=3987%:%
%:%9257=3988%:%
%:%9258=3988%:%
%:%9259=3989%:%
%:%9260=3989%:%
%:%9261=3990%:%
%:%9262=3990%:%
%:%9263=3990%:%
%:%9264=3990%:%
%:%9265=3991%:%
%:%9266=3991%:%
%:%9267=3992%:%
%:%9268=3992%:%
%:%9269=3993%:%
%:%9270=3993%:%
%:%9271=3994%:%
%:%9272=3994%:%
%:%9273=3995%:%
%:%9274=3995%:%
%:%9278=3999%:%
%:%9279=4000%:%
%:%9280=4000%:%
%:%9281=4001%:%
%:%9282=4001%:%
%:%9283=4002%:%
%:%9284=4003%:%
%:%9285=4003%:%
%:%9286=4003%:%
%:%9287=4004%:%
%:%9288=4004%:%
%:%9289=4004%:%
%:%9290=4005%:%
%:%9291=4005%:%
%:%9292=4005%:%
%:%9293=4006%:%
%:%9294=4006%:%
%:%9295=4007%:%
%:%9296=4007%:%
%:%9297=4008%:%
%:%9298=4008%:%
%:%9299=4008%:%
%:%9300=4008%:%
%:%9301=4008%:%
%:%9302=4009%:%
%:%9303=4009%:%
%:%9304=4009%:%
%:%9305=4009%:%
%:%9306=4010%:%
%:%9307=4010%:%
%:%9308=4010%:%
%:%9309=4010%:%
%:%9310=4010%:%
%:%9311=4011%:%
%:%9312=4011%:%
%:%9313=4012%:%
%:%9314=4012%:%
%:%9315=4013%:%
%:%9316=4013%:%
%:%9317=4014%:%
%:%9318=4014%:%
%:%9319=4015%:%
%:%9320=4015%:%
%:%9321=4016%:%
%:%9322=4016%:%
%:%9323=4016%:%
%:%9324=4016%:%
%:%9325=4017%:%
%:%9326=4017%:%
%:%9327=4018%:%
%:%9328=4018%:%
%:%9329=4019%:%
%:%9330=4019%:%
%:%9331=4020%:%
%:%9332=4020%:%
%:%9333=4021%:%
%:%9334=4021%:%
%:%9338=4025%:%
%:%9339=4026%:%
%:%9340=4026%:%
%:%9341=4027%:%
%:%9342=4027%:%
%:%9343=4028%:%
%:%9344=4029%:%
%:%9345=4030%:%
%:%9346=4030%:%
%:%9347=4030%:%
%:%9348=4031%:%
%:%9349=4032%:%
%:%9350=4032%:%
%:%9351=4032%:%
%:%9352=4033%:%
%:%9353=4033%:%
%:%9354=4033%:%
%:%9355=4034%:%
%:%9356=4035%:%
%:%9357=4035%:%
%:%9358=4036%:%
%:%9359=4036%:%
%:%9360=4037%:%
%:%9361=4037%:%
%:%9362=4037%:%
%:%9363=4037%:%
%:%9364=4037%:%
%:%9365=4038%:%
%:%9366=4038%:%
%:%9367=4038%:%
%:%9368=4038%:%
%:%9369=4039%:%
%:%9370=4039%:%
%:%9371=4039%:%
%:%9372=4039%:%
%:%9373=4039%:%
%:%9374=4040%:%
%:%9375=4040%:%
%:%9376=4041%:%
%:%9377=4041%:%
%:%9378=4042%:%
%:%9379=4042%:%
%:%9380=4043%:%
%:%9381=4043%:%
%:%9382=4044%:%
%:%9383=4044%:%
%:%9384=4045%:%
%:%9385=4045%:%
%:%9386=4045%:%
%:%9387=4045%:%
%:%9388=4046%:%
%:%9389=4046%:%
%:%9390=4047%:%
%:%9391=4047%:%
%:%9392=4048%:%
%:%9393=4048%:%
%:%9394=4049%:%
%:%9395=4049%:%
%:%9396=4050%:%
%:%9397=4050%:%
%:%9401=4054%:%
%:%9402=4055%:%
%:%9403=4055%:%
%:%9404=4056%:%
%:%9405=4056%:%
%:%9406=4057%:%
%:%9407=4058%:%
%:%9408=4058%:%
%:%9409=4058%:%
%:%9410=4059%:%
%:%9411=4059%:%
%:%9412=4059%:%
%:%9413=4060%:%
%:%9414=4060%:%
%:%9415=4060%:%
%:%9416=4061%:%
%:%9417=4061%:%
%:%9418=4062%:%
%:%9419=4062%:%
%:%9420=4063%:%
%:%9421=4063%:%
%:%9422=4063%:%
%:%9423=4063%:%
%:%9424=4063%:%
%:%9425=4064%:%
%:%9426=4064%:%
%:%9427=4064%:%
%:%9428=4064%:%
%:%9429=4065%:%
%:%9430=4065%:%
%:%9431=4065%:%
%:%9432=4065%:%
%:%9433=4065%:%
%:%9434=4066%:%
%:%9435=4066%:%
%:%9436=4067%:%
%:%9437=4067%:%
%:%9438=4068%:%
%:%9439=4068%:%
%:%9440=4069%:%
%:%9441=4069%:%
%:%9442=4070%:%
%:%9443=4070%:%
%:%9444=4071%:%
%:%9445=4071%:%
%:%9449=4075%:%
%:%9450=4076%:%
%:%9451=4076%:%
%:%9452=4077%:%
%:%9453=4077%:%
%:%9454=4078%:%
%:%9455=4079%:%
%:%9456=4079%:%
%:%9457=4079%:%
%:%9458=4080%:%
%:%9459=4081%:%
%:%9460=4081%:%
%:%9461=4081%:%
%:%9462=4082%:%
%:%9463=4082%:%
%:%9464=4082%:%
%:%9465=4083%:%
%:%9466=4084%:%
%:%9467=4084%:%
%:%9468=4085%:%
%:%9469=4085%:%
%:%9470=4086%:%
%:%9471=4086%:%
%:%9472=4086%:%
%:%9473=4086%:%
%:%9474=4086%:%
%:%9475=4087%:%
%:%9476=4087%:%
%:%9477=4087%:%
%:%9478=4087%:%
%:%9479=4088%:%
%:%9480=4088%:%
%:%9481=4088%:%
%:%9482=4088%:%
%:%9483=4088%:%
%:%9484=4089%:%
%:%9485=4089%:%
%:%9486=4090%:%
%:%9487=4090%:%
%:%9488=4091%:%
%:%9489=4091%:%
%:%9490=4092%:%
%:%9491=4092%:%
%:%9492=4093%:%
%:%9493=4093%:%
%:%9494=4094%:%
%:%9495=4094%:%
%:%9496=4095%:%
%:%9497=4095%:%
%:%9498=4096%:%
%:%9499=4096%:%
%:%9500=4097%:%
%:%9501=4097%:%
%:%9502=4097%:%
%:%9503=4097%:%
%:%9504=4098%:%
%:%9505=4098%:%
%:%9506=4099%:%
%:%9507=4099%:%
%:%9508=4100%:%
%:%9509=4100%:%
%:%9510=4101%:%
%:%9511=4101%:%
%:%9512=4102%:%
%:%9513=4102%:%
%:%9514=4102%:%
%:%9515=4102%:%
%:%9516=4103%:%
%:%9517=4103%:%
%:%9518=4104%:%
%:%9519=4104%:%
%:%9520=4105%:%
%:%9521=4105%:%
%:%9522=4106%:%
%:%9523=4106%:%
%:%9524=4107%:%
%:%9525=4107%:%
%:%9529=4111%:%
%:%9530=4112%:%
%:%9531=4112%:%
%:%9532=4113%:%
%:%9533=4113%:%
%:%9534=4114%:%
%:%9535=4115%:%
%:%9536=4116%:%
%:%9537=4116%:%
%:%9538=4116%:%
%:%9539=4117%:%
%:%9540=4117%:%
%:%9541=4117%:%
%:%9542=4118%:%
%:%9543=4118%:%
%:%9544=4118%:%
%:%9545=4119%:%
%:%9546=4119%:%
%:%9547=4120%:%
%:%9548=4120%:%
%:%9549=4121%:%
%:%9550=4121%:%
%:%9551=4121%:%
%:%9552=4121%:%
%:%9553=4121%:%
%:%9554=4122%:%
%:%9555=4122%:%
%:%9556=4122%:%
%:%9557=4122%:%
%:%9558=4123%:%
%:%9559=4123%:%
%:%9560=4123%:%
%:%9561=4123%:%
%:%9562=4123%:%
%:%9563=4124%:%
%:%9564=4124%:%
%:%9565=4125%:%
%:%9566=4125%:%
%:%9567=4126%:%
%:%9568=4126%:%
%:%9569=4127%:%
%:%9570=4127%:%
%:%9571=4128%:%
%:%9572=4128%:%
%:%9573=4129%:%
%:%9574=4129%:%
%:%9575=4129%:%
%:%9576=4129%:%
%:%9577=4130%:%
%:%9578=4130%:%
%:%9579=4131%:%
%:%9580=4131%:%
%:%9581=4132%:%
%:%9582=4132%:%
%:%9583=4133%:%
%:%9584=4133%:%
%:%9585=4134%:%
%:%9586=4134%:%
%:%9590=4138%:%
%:%9591=4139%:%
%:%9592=4139%:%
%:%9593=4140%:%
%:%9594=4140%:%
%:%9595=4141%:%
%:%9596=4142%:%
%:%9597=4143%:%
%:%9598=4143%:%
%:%9599=4143%:%
%:%9600=4144%:%
%:%9601=4145%:%
%:%9602=4145%:%
%:%9603=4145%:%
%:%9604=4146%:%
%:%9605=4146%:%
%:%9606=4146%:%
%:%9607=4147%:%
%:%9608=4148%:%
%:%9609=4148%:%
%:%9610=4149%:%
%:%9611=4149%:%
%:%9612=4150%:%
%:%9613=4150%:%
%:%9614=4150%:%
%:%9615=4151%:%
%:%9616=4152%:%
%:%9617=4152%:%
%:%9618=4152%:%
%:%9619=4153%:%
%:%9620=4153%:%
%:%9621=4153%:%
%:%9622=4154%:%
%:%9623=4154%:%
%:%9624=4155%:%
%:%9625=4155%:%
%:%9626=4156%:%
%:%9627=4157%:%
%:%9628=4158%:%
%:%9629=4158%:%
%:%9630=4158%:%
%:%9631=4158%:%
%:%9632=4158%:%
%:%9633=4159%:%
%:%9634=4159%:%
%:%9635=4159%:%
%:%9636=4159%:%
%:%9637=4159%:%
%:%9638=4160%:%
%:%9639=4160%:%
%:%9640=4160%:%
%:%9641=4160%:%
%:%9642=4161%:%
%:%9643=4161%:%
%:%9644=4161%:%
%:%9645=4161%:%
%:%9646=4161%:%
%:%9647=4162%:%
%:%9648=4162%:%
%:%9649=4163%:%
%:%9650=4163%:%
%:%9651=4164%:%
%:%9652=4164%:%
%:%9653=4165%:%
%:%9654=4165%:%
%:%9655=4166%:%
%:%9656=4166%:%
%:%9657=4167%:%
%:%9658=4167%:%
%:%9659=4167%:%
%:%9660=4167%:%
%:%9661=4168%:%
%:%9662=4168%:%
%:%9663=4169%:%
%:%9664=4169%:%
%:%9665=4170%:%
%:%9666=4170%:%
%:%9667=4171%:%
%:%9668=4171%:%
%:%9669=4172%:%
%:%9670=4172%:%
%:%9674=4176%:%
%:%9675=4177%:%
%:%9676=4177%:%
%:%9677=4178%:%
%:%9678=4178%:%
%:%9679=4179%:%
%:%9680=4180%:%
%:%9681=4181%:%
%:%9682=4181%:%
%:%9683=4181%:%
%:%9684=4182%:%
%:%9685=4182%:%
%:%9686=4182%:%
%:%9687=4183%:%
%:%9688=4183%:%
%:%9689=4183%:%
%:%9690=4184%:%
%:%9691=4184%:%
%:%9692=4185%:%
%:%9693=4185%:%
%:%9694=4186%:%
%:%9695=4186%:%
%:%9696=4186%:%
%:%9697=4186%:%
%:%9698=4186%:%
%:%9699=4187%:%
%:%9700=4187%:%
%:%9701=4187%:%
%:%9702=4187%:%
%:%9703=4188%:%
%:%9704=4188%:%
%:%9705=4188%:%
%:%9706=4188%:%
%:%9707=4188%:%
%:%9708=4189%:%
%:%9709=4189%:%
%:%9710=4190%:%
%:%9711=4190%:%
%:%9712=4191%:%
%:%9713=4191%:%
%:%9714=4192%:%
%:%9715=4192%:%
%:%9716=4193%:%
%:%9717=4193%:%
%:%9718=4194%:%
%:%9719=4194%:%
%:%9723=4198%:%
%:%9724=4199%:%
%:%9725=4199%:%
%:%9726=4200%:%
%:%9727=4200%:%
%:%9728=4201%:%
%:%9729=4202%:%
%:%9730=4203%:%
%:%9731=4203%:%
%:%9732=4203%:%
%:%9733=4204%:%
%:%9734=4205%:%
%:%9735=4205%:%
%:%9736=4205%:%
%:%9737=4206%:%
%:%9738=4206%:%
%:%9739=4206%:%
%:%9740=4207%:%
%:%9741=4208%:%
%:%9742=4208%:%
%:%9743=4209%:%
%:%9744=4209%:%
%:%9745=4210%:%
%:%9746=4210%:%
%:%9747=4210%:%
%:%9748=4211%:%
%:%9749=4212%:%
%:%9750=4212%:%
%:%9751=4212%:%
%:%9752=4213%:%
%:%9753=4213%:%
%:%9754=4213%:%
%:%9755=4214%:%
%:%9756=4214%:%
%:%9757=4215%:%
%:%9758=4215%:%
%:%9759=4216%:%
%:%9760=4217%:%
%:%9761=4218%:%
%:%9762=4218%:%
%:%9763=4218%:%
%:%9764=4218%:%
%:%9765=4218%:%
%:%9766=4219%:%
%:%9767=4219%:%
%:%9768=4219%:%
%:%9769=4219%:%
%:%9770=4219%:%
%:%9771=4220%:%
%:%9772=4220%:%
%:%9773=4220%:%
%:%9774=4220%:%
%:%9775=4221%:%
%:%9776=4221%:%
%:%9777=4221%:%
%:%9778=4221%:%
%:%9779=4221%:%
%:%9780=4222%:%
%:%9781=4222%:%
%:%9782=4223%:%
%:%9783=4223%:%
%:%9784=4224%:%
%:%9785=4224%:%
%:%9786=4225%:%
%:%9787=4225%:%
%:%9788=4226%:%
%:%9789=4226%:%
%:%9790=4227%:%
%:%9791=4227%:%
%:%9792=4228%:%
%:%9793=4228%:%
%:%9794=4229%:%
%:%9795=4229%:%
%:%9796=4230%:%
%:%9797=4230%:%
%:%9798=4230%:%
%:%9799=4230%:%
%:%9800=4231%:%
%:%9801=4231%:%
%:%9802=4232%:%
%:%9803=4232%:%
%:%9804=4233%:%
%:%9805=4233%:%
%:%9806=4234%:%
%:%9807=4234%:%
%:%9808=4235%:%
%:%9809=4235%:%
%:%9810=4235%:%
%:%9811=4235%:%
%:%9812=4236%:%
%:%9813=4236%:%
%:%9814=4237%:%
%:%9815=4237%:%
%:%9816=4238%:%
%:%9817=4238%:%
%:%9818=4239%:%
%:%9819=4239%:%
%:%9820=4240%:%
%:%9821=4240%:%
%:%9825=4244%:%
%:%9826=4245%:%
%:%9827=4245%:%
%:%9828=4246%:%
%:%9829=4246%:%
%:%9830=4247%:%
%:%9831=4248%:%
%:%9832=4248%:%
%:%9833=4248%:%
%:%9834=4249%:%
%:%9835=4249%:%
%:%9836=4249%:%
%:%9837=4250%:%
%:%9838=4250%:%
%:%9839=4250%:%
%:%9840=4251%:%
%:%9841=4251%:%
%:%9842=4252%:%
%:%9843=4252%:%
%:%9844=4253%:%
%:%9845=4253%:%
%:%9846=4253%:%
%:%9847=4253%:%
%:%9848=4253%:%
%:%9849=4254%:%
%:%9850=4254%:%
%:%9851=4254%:%
%:%9852=4254%:%
%:%9853=4255%:%
%:%9854=4255%:%
%:%9855=4255%:%
%:%9856=4255%:%
%:%9857=4255%:%
%:%9858=4256%:%
%:%9859=4256%:%
%:%9860=4257%:%
%:%9861=4257%:%
%:%9862=4258%:%
%:%9863=4258%:%
%:%9864=4259%:%
%:%9865=4259%:%
%:%9866=4260%:%
%:%9867=4260%:%
%:%9868=4261%:%
%:%9869=4261%:%
%:%9870=4261%:%
%:%9871=4261%:%
%:%9872=4262%:%
%:%9873=4262%:%
%:%9874=4263%:%
%:%9875=4263%:%
%:%9876=4264%:%
%:%9877=4264%:%
%:%9878=4265%:%
%:%9879=4265%:%
%:%9880=4266%:%
%:%9881=4266%:%
%:%9885=4270%:%
%:%9886=4271%:%
%:%9887=4271%:%
%:%9888=4272%:%
%:%9889=4272%:%
%:%9890=4273%:%
%:%9891=4274%:%
%:%9892=4275%:%
%:%9893=4275%:%
%:%9894=4275%:%
%:%9895=4276%:%
%:%9896=4277%:%
%:%9897=4277%:%
%:%9898=4277%:%
%:%9899=4278%:%
%:%9900=4278%:%
%:%9901=4278%:%
%:%9902=4279%:%
%:%9903=4280%:%
%:%9904=4280%:%
%:%9905=4281%:%
%:%9906=4281%:%
%:%9907=4282%:%
%:%9908=4282%:%
%:%9909=4282%:%
%:%9910=4282%:%
%:%9911=4282%:%
%:%9912=4283%:%
%:%9913=4283%:%
%:%9914=4283%:%
%:%9915=4283%:%
%:%9916=4284%:%
%:%9917=4284%:%
%:%9918=4284%:%
%:%9919=4284%:%
%:%9920=4284%:%
%:%9921=4285%:%
%:%9922=4285%:%
%:%9923=4286%:%
%:%9924=4286%:%
%:%9925=4287%:%
%:%9926=4287%:%
%:%9927=4288%:%
%:%9928=4288%:%
%:%9929=4289%:%
%:%9930=4289%:%
%:%9931=4290%:%
%:%9932=4290%:%
%:%9933=4290%:%
%:%9934=4290%:%
%:%9935=4291%:%
%:%9936=4291%:%
%:%9937=4292%:%
%:%9938=4292%:%
%:%9939=4293%:%
%:%9940=4293%:%
%:%9941=4294%:%
%:%9942=4294%:%
%:%9943=4295%:%
%:%9944=4295%:%
%:%9948=4299%:%
%:%9949=4300%:%
%:%9950=4300%:%
%:%9951=4301%:%
%:%9952=4301%:%
%:%9953=4302%:%
%:%9954=4303%:%
%:%9955=4303%:%
%:%9956=4303%:%
%:%9957=4304%:%
%:%9958=4304%:%
%:%9959=4304%:%
%:%9960=4305%:%
%:%9961=4305%:%
%:%9962=4305%:%
%:%9963=4306%:%
%:%9964=4306%:%
%:%9965=4307%:%
%:%9966=4307%:%
%:%9967=4308%:%
%:%9968=4308%:%
%:%9969=4308%:%
%:%9970=4308%:%
%:%9971=4308%:%
%:%9972=4309%:%
%:%9973=4309%:%
%:%9974=4309%:%
%:%9975=4309%:%
%:%9976=4310%:%
%:%9977=4310%:%
%:%9978=4310%:%
%:%9979=4310%:%
%:%9980=4310%:%
%:%9981=4311%:%
%:%9982=4311%:%
%:%9983=4312%:%
%:%9984=4312%:%
%:%9985=4313%:%
%:%9986=4313%:%
%:%9987=4314%:%
%:%9988=4314%:%
%:%9989=4315%:%
%:%9990=4315%:%
%:%9991=4316%:%
%:%9992=4316%:%
%:%9996=4320%:%
%:%9997=4321%:%
%:%9998=4321%:%
%:%9999=4322%:%
%:%10000=4322%:%
%:%10001=4323%:%
%:%10002=4324%:%
%:%10003=4325%:%
%:%10004=4325%:%
%:%10005=4325%:%
%:%10006=4326%:%
%:%10007=4327%:%
%:%10008=4327%:%
%:%10009=4327%:%
%:%10010=4328%:%
%:%10011=4328%:%
%:%10012=4328%:%
%:%10013=4329%:%
%:%10014=4330%:%
%:%10015=4330%:%
%:%10016=4331%:%
%:%10017=4331%:%
%:%10018=4332%:%
%:%10019=4332%:%
%:%10020=4332%:%
%:%10021=4332%:%
%:%10022=4332%:%
%:%10023=4333%:%
%:%10024=4333%:%
%:%10025=4333%:%
%:%10026=4333%:%
%:%10027=4334%:%
%:%10028=4334%:%
%:%10029=4334%:%
%:%10030=4334%:%
%:%10031=4334%:%
%:%10032=4335%:%
%:%10033=4335%:%
%:%10034=4336%:%
%:%10035=4336%:%
%:%10036=4337%:%
%:%10037=4337%:%
%:%10038=4338%:%
%:%10039=4338%:%
%:%10040=4339%:%
%:%10041=4339%:%
%:%10042=4340%:%
%:%10043=4340%:%
%:%10044=4341%:%
%:%10045=4341%:%
%:%10046=4342%:%
%:%10047=4342%:%
%:%10048=4343%:%
%:%10049=4343%:%
%:%10050=4343%:%
%:%10051=4343%:%
%:%10052=4344%:%
%:%10053=4344%:%
%:%10054=4345%:%
%:%10055=4345%:%
%:%10056=4346%:%
%:%10057=4346%:%
%:%10058=4347%:%
%:%10059=4347%:%
%:%10060=4348%:%
%:%10061=4348%:%
%:%10065=4352%:%
%:%10066=4353%:%
%:%10067=4353%:%
%:%10068=4354%:%
%:%10069=4354%:%
%:%10070=4355%:%
%:%10071=4356%:%
%:%10072=4356%:%
%:%10073=4356%:%
%:%10074=4357%:%
%:%10075=4357%:%
%:%10076=4357%:%
%:%10077=4358%:%
%:%10078=4358%:%
%:%10079=4358%:%
%:%10080=4359%:%
%:%10081=4359%:%
%:%10082=4360%:%
%:%10083=4360%:%
%:%10084=4361%:%
%:%10085=4361%:%
%:%10086=4361%:%
%:%10087=4361%:%
%:%10088=4361%:%
%:%10089=4362%:%
%:%10090=4362%:%
%:%10091=4362%:%
%:%10092=4362%:%
%:%10093=4363%:%
%:%10094=4363%:%
%:%10095=4363%:%
%:%10096=4363%:%
%:%10097=4363%:%
%:%10098=4364%:%
%:%10099=4364%:%
%:%10100=4365%:%
%:%10101=4365%:%
%:%10102=4366%:%
%:%10103=4366%:%
%:%10104=4367%:%
%:%10105=4367%:%
%:%10106=4368%:%
%:%10107=4368%:%
%:%10108=4369%:%
%:%10109=4369%:%
%:%10110=4369%:%
%:%10111=4369%:%
%:%10112=4370%:%
%:%10113=4370%:%
%:%10114=4371%:%
%:%10115=4371%:%
%:%10116=4372%:%
%:%10117=4372%:%
%:%10118=4373%:%
%:%10119=4373%:%
%:%10120=4374%:%
%:%10121=4374%:%
%:%10125=4378%:%
%:%10126=4379%:%
%:%10127=4379%:%
%:%10128=4380%:%
%:%10129=4380%:%
%:%10130=4381%:%
%:%10131=4382%:%
%:%10132=4383%:%
%:%10133=4383%:%
%:%10134=4383%:%
%:%10135=4384%:%
%:%10136=4385%:%
%:%10137=4385%:%
%:%10138=4385%:%
%:%10139=4386%:%
%:%10140=4386%:%
%:%10141=4386%:%
%:%10142=4387%:%
%:%10143=4388%:%
%:%10144=4388%:%
%:%10145=4389%:%
%:%10146=4389%:%
%:%10147=4390%:%
%:%10148=4390%:%
%:%10149=4390%:%
%:%10150=4391%:%
%:%10151=4392%:%
%:%10152=4392%:%
%:%10153=4392%:%
%:%10154=4393%:%
%:%10155=4393%:%
%:%10156=4393%:%
%:%10157=4394%:%
%:%10158=4394%:%
%:%10159=4395%:%
%:%10160=4395%:%
%:%10161=4396%:%
%:%10162=4397%:%
%:%10163=4398%:%
%:%10164=4398%:%
%:%10165=4398%:%
%:%10166=4398%:%
%:%10167=4398%:%
%:%10168=4399%:%
%:%10169=4399%:%
%:%10170=4399%:%
%:%10171=4399%:%
%:%10172=4399%:%
%:%10173=4400%:%
%:%10174=4400%:%
%:%10175=4400%:%
%:%10176=4400%:%
%:%10177=4401%:%
%:%10178=4401%:%
%:%10179=4401%:%
%:%10180=4401%:%
%:%10181=4401%:%
%:%10182=4402%:%
%:%10183=4402%:%
%:%10184=4403%:%
%:%10185=4403%:%
%:%10186=4404%:%
%:%10187=4404%:%
%:%10188=4405%:%
%:%10189=4405%:%
%:%10190=4406%:%
%:%10191=4406%:%
%:%10192=4407%:%
%:%10193=4407%:%
%:%10194=4407%:%
%:%10195=4407%:%
%:%10196=4408%:%
%:%10197=4408%:%
%:%10198=4409%:%
%:%10199=4409%:%
%:%10200=4410%:%
%:%10201=4410%:%
%:%10202=4411%:%
%:%10203=4411%:%
%:%10204=4412%:%
%:%10205=4412%:%
%:%10209=4416%:%
%:%10210=4417%:%
%:%10211=4417%:%
%:%10212=4418%:%
%:%10213=4418%:%
%:%10214=4419%:%
%:%10215=4420%:%
%:%10216=4420%:%
%:%10217=4420%:%
%:%10218=4421%:%
%:%10219=4421%:%
%:%10220=4421%:%
%:%10221=4422%:%
%:%10222=4422%:%
%:%10223=4422%:%
%:%10224=4423%:%
%:%10225=4423%:%
%:%10226=4424%:%
%:%10227=4424%:%
%:%10228=4425%:%
%:%10229=4425%:%
%:%10230=4425%:%
%:%10231=4425%:%
%:%10232=4425%:%
%:%10233=4426%:%
%:%10234=4426%:%
%:%10235=4426%:%
%:%10236=4426%:%
%:%10237=4427%:%
%:%10238=4427%:%
%:%10239=4427%:%
%:%10240=4427%:%
%:%10241=4427%:%
%:%10242=4428%:%
%:%10243=4428%:%
%:%10244=4429%:%
%:%10245=4429%:%
%:%10246=4430%:%
%:%10247=4430%:%
%:%10248=4431%:%
%:%10249=4431%:%
%:%10250=4432%:%
%:%10251=4432%:%
%:%10252=4433%:%
%:%10253=4433%:%
%:%10257=4437%:%
%:%10258=4438%:%
%:%10259=4438%:%
%:%10260=4439%:%
%:%10261=4439%:%
%:%10262=4440%:%
%:%10263=4441%:%
%:%10264=4442%:%
%:%10265=4442%:%
%:%10266=4442%:%
%:%10267=4443%:%
%:%10268=4444%:%
%:%10269=4444%:%
%:%10270=4444%:%
%:%10271=4445%:%
%:%10272=4445%:%
%:%10273=4445%:%
%:%10274=4446%:%
%:%10275=4447%:%
%:%10276=4447%:%
%:%10277=4448%:%
%:%10278=4448%:%
%:%10279=4449%:%
%:%10280=4449%:%
%:%10281=4449%:%
%:%10282=4450%:%
%:%10283=4451%:%
%:%10284=4451%:%
%:%10285=4451%:%
%:%10286=4452%:%
%:%10287=4452%:%
%:%10288=4452%:%
%:%10289=4453%:%
%:%10290=4453%:%
%:%10291=4454%:%
%:%10292=4454%:%
%:%10293=4455%:%
%:%10294=4456%:%
%:%10295=4457%:%
%:%10296=4457%:%
%:%10297=4457%:%
%:%10298=4457%:%
%:%10299=4457%:%
%:%10300=4458%:%
%:%10301=4458%:%
%:%10302=4458%:%
%:%10303=4458%:%
%:%10304=4458%:%
%:%10305=4459%:%
%:%10306=4459%:%
%:%10307=4459%:%
%:%10308=4459%:%
%:%10309=4460%:%
%:%10310=4460%:%
%:%10311=4460%:%
%:%10312=4460%:%
%:%10313=4460%:%
%:%10314=4461%:%
%:%10315=4461%:%
%:%10316=4462%:%
%:%10317=4462%:%
%:%10318=4463%:%
%:%10319=4463%:%
%:%10320=4464%:%
%:%10321=4464%:%
%:%10322=4465%:%
%:%10323=4465%:%
%:%10324=4466%:%
%:%10325=4466%:%
%:%10326=4467%:%
%:%10327=4467%:%
%:%10328=4468%:%
%:%10329=4468%:%
%:%10330=4469%:%
%:%10331=4469%:%
%:%10332=4470%:%
%:%10333=4470%:%
%:%10334=4471%:%
%:%10335=4471%:%
%:%10336=4472%:%
%:%10337=4473%:%
%:%10338=4473%:%
%:%10339=4474%:%
%:%10340=4474%:%
%:%10341=4475%:%
%:%10342=4475%:%
%:%10343=4476%:%
%:%10344=4477%:%
%:%10350=4477%:%
%:%10353=4478%:%
%:%10354=4479%:%
%:%10355=4479%:%
%:%10356=4480%:%
%:%10357=4481%:%
%:%10364=4482%:%
%:%10365=4482%:%
%:%10366=4483%:%
%:%10367=4483%:%
%:%10368=4483%:%
%:%10369=4484%:%
%:%10370=4484%:%
%:%10371=4485%:%
%:%10372=4485%:%
%:%10373=4486%:%
%:%10374=4486%:%
%:%10375=4487%:%
%:%10376=4487%:%
%:%10377=4488%:%
%:%10378=4488%:%
%:%10379=4489%:%
%:%10380=4489%:%
%:%10381=4490%:%
%:%10382=4490%:%
%:%10383=4491%:%
%:%10384=4491%:%
%:%10385=4492%:%
%:%10391=4492%:%
%:%10394=4493%:%
%:%10395=4494%:%
%:%10396=4494%:%
%:%10397=4495%:%
%:%10404=4496%:%
%:%10405=4496%:%
%:%10406=4497%:%
%:%10407=4497%:%
%:%10408=4498%:%
%:%10409=4498%:%
%:%10410=4498%:%
%:%10411=4499%:%
%:%10412=4499%:%
%:%10413=4500%:%
%:%10414=4501%:%
%:%10415=4501%:%
%:%10416=4502%:%
%:%10417=4502%:%
%:%10418=4503%:%
%:%10419=4503%:%
%:%10420=4503%:%
%:%10421=4504%:%
%:%10422=4504%:%
%:%10423=4505%:%
%:%10424=4505%:%
%:%10425=4506%:%
%:%10426=4506%:%
%:%10427=4507%:%
%:%10428=4507%:%
%:%10429=4507%:%
%:%10430=4508%:%
%:%10431=4508%:%
%:%10432=4509%:%
%:%10433=4509%:%
%:%10434=4510%:%
%:%10435=4510%:%
%:%10436=4511%:%
%:%10437=4511%:%
%:%10438=4511%:%
%:%10439=4512%:%
%:%10440=4512%:%
%:%10441=4513%:%
%:%10442=4513%:%
%:%10443=4514%:%
%:%10444=4514%:%
%:%10445=4515%:%
%:%10446=4515%:%
%:%10447=4516%:%
%:%10448=4516%:%
%:%10449=4517%:%
%:%10450=4518%:%
%:%10451=4518%:%
%:%10452=4519%:%
%:%10453=4519%:%
%:%10454=4520%:%
%:%10455=4520%:%
%:%10456=4521%:%
%:%10457=4521%:%
%:%10458=4522%:%
%:%10459=4522%:%
%:%10460=4523%:%
%:%10461=4523%:%
%:%10462=4524%:%
%:%10463=4524%:%
%:%10464=4525%:%
%:%10465=4525%:%
%:%10466=4526%:%
%:%10467=4526%:%
%:%10468=4527%:%
%:%10469=4527%:%
%:%10470=4527%:%
%:%10471=4527%:%
%:%10472=4528%:%
%:%10473=4528%:%
%:%10474=4529%:%
%:%10475=4529%:%
%:%10476=4530%:%
%:%10482=4530%:%
%:%10485=4531%:%
%:%10486=4532%:%
%:%10487=4532%:%
%:%10488=4533%:%
%:%10495=4534%:%
%:%10496=4534%:%
%:%10497=4535%:%
%:%10498=4535%:%
%:%10499=4536%:%
%:%10500=4536%:%
%:%10501=4536%:%
%:%10502=4536%:%
%:%10503=4537%:%
%:%10504=4537%:%
%:%10505=4538%:%
%:%10506=4538%:%
%:%10507=4539%:%
%:%10508=4539%:%
%:%10509=4540%:%
%:%10510=4540%:%
%:%10511=4540%:%
%:%10512=4541%:%
%:%10513=4541%:%
%:%10514=4542%:%
%:%10515=4542%:%
%:%10516=4543%:%
%:%10517=4543%:%
%:%10518=4544%:%
%:%10519=4544%:%
%:%10520=4544%:%
%:%10521=4545%:%
%:%10522=4545%:%
%:%10523=4546%:%
%:%10524=4546%:%
%:%10525=4547%:%
%:%10526=4547%:%
%:%10527=4548%:%
%:%10528=4548%:%
%:%10529=4548%:%
%:%10530=4549%:%
%:%10531=4549%:%
%:%10532=4550%:%
%:%10533=4550%:%
%:%10534=4551%:%
%:%10535=4551%:%
%:%10536=4552%:%
%:%10537=4552%:%
%:%10538=4553%:%
%:%10539=4553%:%
%:%10540=4554%:%
%:%10541=4555%:%
%:%10542=4555%:%
%:%10543=4556%:%
%:%10544=4556%:%
%:%10545=4557%:%
%:%10546=4557%:%
%:%10547=4558%:%
%:%10548=4558%:%
%:%10549=4559%:%
%:%10550=4559%:%
%:%10551=4560%:%
%:%10552=4560%:%
%:%10553=4561%:%
%:%10554=4561%:%
%:%10555=4562%:%
%:%10556=4562%:%
%:%10557=4563%:%
%:%10558=4563%:%
%:%10559=4564%:%
%:%10560=4564%:%
%:%10561=4564%:%
%:%10562=4564%:%
%:%10563=4565%:%
%:%10564=4565%:%
%:%10565=4566%:%
%:%10566=4566%:%
%:%10567=4567%:%
%:%10573=4567%:%
%:%10576=4568%:%
%:%10577=4569%:%
%:%10578=4569%:%
%:%10579=4570%:%
%:%10580=4571%:%
%:%10587=4572%:%
%:%10588=4572%:%
%:%10589=4573%:%
%:%10590=4573%:%
%:%10591=4574%:%
%:%10592=4574%:%
%:%10593=4574%:%
%:%10594=4575%:%
%:%10595=4575%:%
%:%10596=4576%:%
%:%10597=4576%:%
%:%10598=4577%:%
%:%10599=4577%:%
%:%10600=4578%:%
%:%10601=4578%:%
%:%10602=4578%:%
%:%10603=4579%:%
%:%10604=4579%:%
%:%10605=4580%:%
%:%10606=4580%:%
%:%10607=4581%:%
%:%10608=4581%:%
%:%10609=4582%:%
%:%10610=4582%:%
%:%10611=4583%:%
%:%10612=4583%:%
%:%10613=4584%:%
%:%10614=4584%:%
%:%10615=4585%:%
%:%10616=4585%:%
%:%10617=4586%:%
%:%10618=4586%:%
%:%10619=4586%:%
%:%10620=4587%:%
%:%10621=4587%:%
%:%10622=4588%:%
%:%10623=4588%:%
%:%10624=4589%:%
%:%10625=4589%:%
%:%10626=4590%:%
%:%10627=4590%:%
%:%10628=4590%:%
%:%10629=4591%:%
%:%10630=4591%:%
%:%10631=4592%:%
%:%10632=4592%:%
%:%10633=4593%:%
%:%10634=4593%:%
%:%10635=4594%:%
%:%10636=4594%:%
%:%10637=4595%:%
%:%10638=4595%:%
%:%10639=4596%:%
%:%10640=4596%:%
%:%10641=4597%:%
%:%10642=4597%:%
%:%10643=4598%:%
%:%10644=4598%:%
%:%10645=4598%:%
%:%10646=4599%:%
%:%10647=4599%:%
%:%10648=4599%:%
%:%10649=4600%:%
%:%10650=4600%:%
%:%10651=4601%:%
%:%10652=4601%:%
%:%10653=4602%:%
%:%10654=4602%:%
%:%10655=4602%:%
%:%10656=4603%:%
%:%10657=4603%:%
%:%10658=4604%:%
%:%10659=4604%:%
%:%10660=4605%:%
%:%10661=4605%:%
%:%10662=4606%:%
%:%10663=4606%:%
%:%10664=4607%:%
%:%10665=4607%:%
%:%10666=4608%:%
%:%10667=4608%:%
%:%10668=4609%:%
%:%10669=4609%:%
%:%10670=4610%:%
%:%10671=4611%:%
%:%10672=4611%:%
%:%10673=4612%:%
%:%10674=4612%:%
%:%10675=4613%:%
%:%10676=4613%:%
%:%10677=4614%:%
%:%10678=4614%:%
%:%10679=4615%:%
%:%10680=4615%:%
%:%10681=4616%:%
%:%10682=4616%:%
%:%10683=4617%:%
%:%10684=4617%:%
%:%10685=4618%:%
%:%10686=4618%:%
%:%10687=4619%:%
%:%10688=4619%:%
%:%10689=4620%:%
%:%10690=4620%:%
%:%10691=4621%:%
%:%10692=4621%:%
%:%10693=4622%:%
%:%10694=4622%:%
%:%10695=4623%:%
%:%10696=4623%:%
%:%10697=4624%:%
%:%10698=4624%:%
%:%10699=4624%:%
%:%10700=4624%:%
%:%10701=4625%:%
%:%10702=4625%:%
%:%10703=4626%:%
%:%10704=4626%:%
%:%10705=4626%:%
%:%10706=4626%:%
%:%10707=4627%:%
%:%10708=4627%:%
%:%10709=4628%:%
%:%10710=4628%:%
%:%10711=4629%:%
%:%10712=4629%:%
%:%10713=4630%:%
%:%10714=4630%:%
%:%10715=4630%:%
%:%10716=4630%:%
%:%10717=4631%:%
%:%10718=4631%:%
%:%10719=4632%:%
%:%10720=4632%:%
%:%10721=4633%:%
%:%10722=4633%:%
%:%10723=4634%:%
%:%10724=4634%:%
%:%10725=4634%:%
%:%10726=4634%:%
%:%10727=4635%:%
%:%10728=4635%:%
%:%10729=4636%:%
%:%10730=4636%:%
%:%10731=4637%:%
%:%10737=4637%:%
%:%10740=4638%:%
%:%10741=4639%:%
%:%10742=4639%:%
%:%10743=4640%:%
%:%10750=4641%:%
%:%10751=4641%:%
%:%10752=4642%:%
%:%10753=4642%:%
%:%10754=4643%:%
%:%10755=4643%:%
%:%10756=4643%:%
%:%10757=4644%:%
%:%10758=4644%:%
%:%10759=4645%:%
%:%10760=4645%:%
%:%10761=4646%:%
%:%10762=4646%:%
%:%10763=4647%:%
%:%10764=4647%:%
%:%10765=4647%:%
%:%10766=4648%:%
%:%10767=4648%:%
%:%10768=4649%:%
%:%10769=4649%:%
%:%10770=4650%:%
%:%10771=4650%:%
%:%10772=4651%:%
%:%10773=4651%:%
%:%10774=4651%:%
%:%10775=4652%:%
%:%10776=4652%:%
%:%10777=4653%:%
%:%10778=4653%:%
%:%10779=4654%:%
%:%10780=4654%:%
%:%10781=4655%:%
%:%10782=4655%:%
%:%10783=4656%:%
%:%10784=4656%:%
%:%10785=4657%:%
%:%10786=4658%:%
%:%10787=4658%:%
%:%10788=4659%:%
%:%10789=4659%:%
%:%10790=4660%:%
%:%10791=4660%:%
%:%10792=4660%:%
%:%10793=4661%:%
%:%10794=4661%:%
%:%10795=4662%:%
%:%10796=4662%:%
%:%10797=4663%:%
%:%10798=4663%:%
%:%10799=4664%:%
%:%10800=4664%:%
%:%10801=4665%:%
%:%10802=4665%:%
%:%10803=4666%:%
%:%10804=4666%:%
%:%10805=4666%:%
%:%10806=4666%:%
%:%10807=4667%:%
%:%10808=4667%:%
%:%10809=4668%:%
%:%10810=4668%:%
%:%10811=4669%:%
%:%10817=4669%:%
%:%10820=4670%:%
%:%10821=4671%:%
%:%10822=4671%:%
%:%10823=4672%:%
%:%10830=4673%:%
%:%10831=4673%:%
%:%10832=4674%:%
%:%10833=4674%:%
%:%10834=4675%:%
%:%10835=4675%:%
%:%10836=4675%:%
%:%10837=4676%:%
%:%10838=4676%:%
%:%10839=4677%:%
%:%10840=4677%:%
%:%10841=4678%:%
%:%10842=4678%:%
%:%10843=4679%:%
%:%10844=4679%:%
%:%10845=4679%:%
%:%10846=4680%:%
%:%10847=4680%:%
%:%10848=4680%:%
%:%10849=4681%:%
%:%10850=4681%:%
%:%10851=4682%:%
%:%10852=4682%:%
%:%10853=4683%:%
%:%10854=4683%:%
%:%10855=4683%:%
%:%10856=4684%:%
%:%10857=4684%:%
%:%10858=4685%:%
%:%10859=4685%:%
%:%10860=4686%:%
%:%10861=4686%:%
%:%10862=4687%:%
%:%10863=4687%:%
%:%10864=4688%:%
%:%10865=4688%:%
%:%10866=4689%:%
%:%10867=4690%:%
%:%10868=4690%:%
%:%10869=4691%:%
%:%10870=4691%:%
%:%10871=4692%:%
%:%10872=4692%:%
%:%10873=4693%:%
%:%10874=4693%:%
%:%10875=4694%:%
%:%10876=4695%:%
%:%10877=4695%:%
%:%10878=4696%:%
%:%10879=4696%:%
%:%10880=4696%:%
%:%10881=4697%:%
%:%10882=4697%:%
%:%10883=4698%:%
%:%10884=4698%:%
%:%10885=4699%:%
%:%10886=4699%:%
%:%10887=4700%:%
%:%10888=4700%:%
%:%10889=4701%:%
%:%10890=4701%:%
%:%10891=4701%:%
%:%10892=4701%:%
%:%10893=4702%:%
%:%10894=4702%:%
%:%10895=4703%:%
%:%10896=4703%:%
%:%10897=4704%:%
%:%10903=4704%:%
%:%10906=4705%:%
%:%10907=4706%:%
%:%10908=4706%:%
%:%10909=4707%:%
%:%10912=4708%:%
%:%10916=4708%:%
%:%10917=4708%:%
%:%10918=4708%:%
%:%10923=4708%:%
%:%10926=4709%:%
%:%10927=4710%:%
%:%10928=4710%:%
%:%10929=4711%:%
%:%10930=4712%:%
%:%10933=4713%:%
%:%10937=4713%:%
%:%10938=4713%:%
%:%10939=4714%:%
%:%10940=4714%:%
%:%10941=4715%:%
%:%10946=4715%:%
%:%10949=4716%:%
%:%10950=4717%:%
%:%10951=4717%:%
%:%10952=4718%:%
%:%10959=4719%:%
%:%10960=4719%:%
%:%10961=4720%:%
%:%10962=4720%:%
%:%10963=4721%:%
%:%10964=4721%:%
%:%10965=4721%:%
%:%10966=4722%:%
%:%10967=4722%:%
%:%10968=4723%:%
%:%10969=4723%:%
%:%10970=4724%:%
%:%10971=4724%:%
%:%10972=4725%:%
%:%10973=4725%:%
%:%10974=4725%:%
%:%10975=4726%:%
%:%10976=4726%:%
%:%10977=4726%:%
%:%10978=4727%:%
%:%10979=4727%:%
%:%10980=4728%:%
%:%10981=4728%:%
%:%10982=4729%:%
%:%10983=4729%:%
%:%10984=4729%:%
%:%10985=4730%:%
%:%10986=4730%:%
%:%10987=4730%:%
%:%10988=4731%:%
%:%10989=4731%:%
%:%10990=4732%:%
%:%10991=4732%:%
%:%10992=4733%:%
%:%10993=4733%:%
%:%10994=4733%:%
%:%10995=4734%:%
%:%10996=4734%:%
%:%10997=4735%:%
%:%10998=4735%:%
%:%10999=4736%:%
%:%11000=4736%:%
%:%11001=4737%:%
%:%11002=4737%:%
%:%11003=4738%:%
%:%11004=4738%:%
%:%11005=4739%:%
%:%11006=4740%:%
%:%11007=4740%:%
%:%11008=4741%:%
%:%11009=4741%:%
%:%11010=4742%:%
%:%11011=4742%:%
%:%11012=4743%:%
%:%11013=4743%:%
%:%11014=4744%:%
%:%11015=4745%:%
%:%11016=4745%:%
%:%11017=4746%:%
%:%11018=4746%:%
%:%11019=4747%:%
%:%11020=4747%:%
%:%11021=4747%:%
%:%11022=4748%:%
%:%11023=4748%:%
%:%11024=4749%:%
%:%11025=4749%:%
%:%11026=4749%:%
%:%11027=4749%:%
%:%11028=4750%:%
%:%11029=4750%:%
%:%11030=4751%:%
%:%11031=4751%:%
%:%11032=4752%:%
%:%11038=4752%:%
%:%11041=4753%:%
%:%11042=4754%:%
%:%11043=4754%:%
%:%11044=4755%:%
%:%11047=4756%:%
%:%11051=4756%:%
%:%11052=4756%:%
%:%11053=4756%:%
%:%11058=4756%:%
%:%11061=4757%:%
%:%11062=4758%:%
%:%11063=4758%:%
%:%11064=4759%:%
%:%11067=4760%:%
%:%11071=4760%:%
%:%11072=4760%:%
%:%11073=4760%:%
%:%11078=4760%:%
%:%11081=4761%:%
%:%11082=4762%:%
%:%11083=4762%:%
%:%11084=4763%:%
%:%11085=4764%:%
%:%11088=4765%:%
%:%11092=4765%:%
%:%11093=4765%:%
%:%11094=4766%:%
%:%11095=4767%:%
%:%11096=4767%:%
%:%11097=4768%:%
%:%11102=4768%:%
%:%11107=4769%:%
%:%11112=4770%:%



\section{Acknowledgements}

This work was conducted as part of my MSc Thesis \cite{tfm} under the supervision of Prof. Francisco Jesús Martín Mateos, without whose advise and assistance its completion would not have been possible.

\nocite{*}
% optional bibliography
\bibliographystyle{abbrv}
\bibliography{root}

\end{document}

%%% Local Variables:
%%% mode: latex
%%% TeX-master: t
%%% End:
\endinput
%:%file=~/Documentos/Quantum_Fourier_Transform/document/root.tex%:%
